\addchap{Jeremias}

\lettrine{1} Palavras de Jeremias, filho de Hilquias, um dos
sacerdotes que estavam em Anatote, na terra de Benjamim; ao qual
veio a palavra do Senhor, nos dias de Josias, filho de Amom, rei de
Judá, no décimo terceiro ano do seu reinado. E lhe veio também
nos dias de Jeoiaquim, filho de Josias, rei de Judá, até ao fim do
ano undécimo de Zedequias, filho de Josias, rei de Judá, até que
Jerusalém foi levada em cativeiro no quinto mês.

Assim veio a mim a palavra do Senhor, dizendo: Antes que te
formasse no ventre te conheci, e antes que saísses da madre, te
santifiquei; às nações te dei por profeta. Então disse eu: Ah,
Senhor Deus! Eis que não sei falar; porque ainda sou um menino.
Mas o Senhor me disse: Não digas: Eu sou um menino; porque a
todos a quem eu te enviar, irás; e tudo quanto te mandar, falarás.
Não temas diante deles; porque estou contigo para te livrar, diz
o Senhor. E estendeu o Senhor a sua mão, e tocou-me na boca; e
disse-me o Senhor: Eis que ponho as minhas palavras na tua boca.
Olha, ponho-te neste dia sobre as nações, e sobre os reinos,
para arrancares, e para derrubares, e para destruíres, e para
arruinares; e também para edificares e para plantares.

Ainda veio a mim a palavra do Senhor, dizendo: Que é que vês,
Jeremias? E eu disse: Vejo uma vara de amendoeira. E disse-me
o Senhor: Viste bem; porque eu velo sobre a minha palavra para
cumpri-la. E veio a mim a palavra do Senhor segunda vez,
dizendo: Que é que vês? E eu disse: Vejo uma panela a ferver, cuja
face está para o lado do norte. E disse-me o Senhor: Do norte
se descobrirá o mal sobre todos os habitantes da terra.
Porque eis que eu convoco todas as famílias dos reinos do
norte, diz o Senhor; e virão, e cada um porá o seu trono à entrada
das portas de Jerusalém, e contra todos os seus muros em redor, e
contra todas as cidades de Judá. E eu pronunciarei contra
eles os meus juízos, por causa de toda a sua malícia; pois me
deixaram, e queimaram incenso a deuses estranhos, e se encurvaram
diante das obras das suas mãos. Tu, pois, cinge os teus
lombos, e levanta-te, e dize-lhes tudo quanto eu te mandar; não te
espantes diante deles, para que eu não te envergonhe diante deles.
Porque, eis que hoje te ponho por cidade forte, e por coluna
de ferro, e por muros de bronze, contra toda a terra, contra os reis
de Judá, contra os seus príncipes, contra os seus sacerdotes, e
contra o povo da terra. E pelejarão contra ti, mas não
prevalecerão contra ti; porque eu sou contigo, diz o Senhor, para te
livrar.

\medskip

\lettrine{2} E veio a mim a palavra do Senhor, dizendo:
Vai, e clama aos ouvidos de Jerusalém, dizendo: Assim diz o
Senhor: Lembro-me de ti, da piedade da tua mocidade, e do amor do
teu noivado, quando me seguias no deserto, numa terra que não se
semeava. Então Israel era santidade para o Senhor, e as
primícias da sua novidade; todos os que o devoravam eram tidos por
culpados; o mal vinha sobre eles, diz o Senhor. Ouvi a palavra
do Senhor, ó casa de Jacó, e todas as famílias da casa de Israel;
assim diz o Senhor: Que injustiça acharam vossos pais em mim,
para se afastarem de mim, indo após a vaidade, e tornando-se
levianos? E não disseram: Onde está o Senhor, que nos fez subir
da terra do Egito, que nos guiou através do deserto, por uma terra
árida, e de covas, por uma terra de sequidão e sombra de morte, por
uma terra pela qual ninguém transitava, e na qual não morava homem
algum? E eu vos introduzi numa terra fértil, para comerdes o seu
fruto e o seu bem; mas quando nela entrastes contaminastes a minha
terra, e da minha herança fizestes uma abominação. Os sacerdotes
não disseram: Onde está o Senhor? E os que tratavam da lei não me
conheciam, e os pastores prevaricavam contra mim, e os profetas
profetizavam por Baal, e andaram após o que é de nenhum proveito.

Portanto ainda contenderei convosco, diz o Senhor; e até com os
filhos de vossos filhos contenderei. Pois, passai às ilhas de
Quitim, e vede; e enviai a Quedar, e atentai bem, e vede se jamais
sucedeu coisa semelhante. Houve alguma nação que trocasse os
seus deuses, ainda que não fossem deuses? Todavia o meu povo trocou
a sua glória por aquilo que é de nenhum proveito.
Espantai-vos disto, ó céus, e horrorizai-vos! Ficai
verdadeiramente desolados, diz o Senhor. Porque o meu povo
fez duas maldades: a mim me deixaram, o manancial de águas vivas, e
cavaram cisternas, cisternas rotas, que não retêm águas.

Acaso é Israel um servo? É ele um escravo nascido em casa?
Porque, pois, veio a ser presa? Os filhos de leão rugiram
sobre ele, levantaram a sua voz; e fizeram da sua terra uma
desolação; as suas cidades se queimaram, e ninguém habita nelas.
Até os filhos de Nofe e de Tafnes te quebraram o alto da
cabeça. Porventura não fizeste isto a ti mesmo, deixando o
Senhor teu Deus, no tempo em que ele te guiava pelo caminho?
Agora, pois, que te importa a ti o caminho do Egito, para
beberes as águas de Sior? E que te importa a ti o caminho da
Assíria, para beberes as águas do rio? A tua malícia te
castigará, e as tuas apostasias te repreenderão; sabe, pois, e vê,
que mal e quão amargo é deixares ao Senhor teu Deus, e não teres em
ti o meu temor, diz o Senhor Deus dos Exércitos.

Quando eu já há muito quebrava o teu jugo, e rompia as tuas
ataduras, dizias tu: Nunca mais transgredirei; contudo em todo o
outeiro alto e debaixo de toda a árvore verde te andas encurvando e
prostituindo-te. Eu mesmo te plantei como vide excelente, uma
semente inteiramente fiel; como, pois, te tornaste para mim uma
planta degenerada como vide estranha? Por isso, ainda que te
laves com salitre\footnote{O nome moderno indica nitrato de sódio ou
de potássio, mas o nome bíblico se refere ao natro (carbonato de
soda), que provinha principalmente dos `lagos de soda' do Baixo
Egito. Nesta passagem, sentido purificador: misturado com azeite
formava uma espécie de sabão. Que ocorre fora do tempo;
extemporâneo, tardio.}, e amontoes sabão, a tua iniqüidade está
gravada diante de mim, diz o Senhor Deus. Como dizes logo:
Não estou contaminada nem andei após os baalins? Vê o teu caminho no
vale, conhece o que fizeste; dromedária ligeira és, que anda
torcendo os seus caminhos. Jumenta montês, acostumada ao
deserto, que, conforme o desejo da sua alma, sorve o vento, quem a
deteria no seu cio? Todos os que a buscarem não se cansarão; no mês
dela a acharão. Evita que o teu pé ande descalço, e a tua
garganta tenha sede. Mas tu dizes: Não há esperança; porque amo os
estranhos, após eles andarei. Como fica confundido o ladrão
quando o apanham, assim se confundem os da casa de Israel; eles, os
seus reis, os seus príncipes, e os seus sacerdotes, e os seus
profetas, que dizem ao pau: Tu és meu pai; e à pedra: Tu me
geraste; porque me viraram as costas, e não o rosto; mas no tempo da
sua angústia dirão: Levanta-te, e livra-nos. Onde, pois,
estão os teus deuses, que fizeste para ti? Que se levantem, se te
podem livrar no tempo da tua angústia; porque os teus deuses, ó
Judá, são tão numerosos como as tuas cidades.

Por que contendeis comigo? Todos vós transgredistes contra mim,
diz o Senhor. Em vão castiguei os vossos filhos; eles não
aceitaram a correção; a vossa espada devorou os vossos profetas como
um leão destruidor. Oh geração! Considerai vós a palavra do
Senhor: Porventura tenho eu sido para Israel um deserto? Ou uma
terra da mais espessa escuridão? Por que, pois, diz o meu povo:
Temos determinado; não viremos mais a ti? Porventura
esquece-se a virgem dos seus enfeites, ou a noiva dos seus adornos?
Todavia o meu povo se esqueceu de mim por inumeráveis dias.
Por que ornamentas o teu caminho, para buscares o amor? Pois
até às malignas ensinaste os teus caminhos. Até nas orlas dos
teus vestidos se achou o sangue das almas dos inocentes e
necessitados; não cavei para achar, pois se vê em todas estas
coisas. E ainda dizes: Eu estou inocente; certamente a sua
ira se desviou de mim. Eis que entrarei em juízo contigo, porquanto
dizes: Não pequei. Por que te desvias tanto, mudando o teu
caminho? Também do Egito serás envergonhada, como foste envergonhada
da Assíria. Também daquele sairás com as mãos sobre a tua
cabeça; porque o Senhor rejeitou a tua confiança, e não prosperarás
com eles.

\medskip

\lettrine{3} Eles dizem: Se um homem despedir sua mulher, e
ela o deixar, e se ajuntar a outro homem, porventura tornará ele
outra vez para ela? Não se poluirá de todo aquela terra? Ora, tu te
prostituíste com muitos amantes; mas ainda assim, torna para mim,
diz o Senhor. Levanta os teus olhos aos altos, e vê: onde não te
prostituíste? Nos caminhos te assentavas para eles, como o árabe no
deserto; assim poluíste a terra com as tuas fornicações e com a tua
malícia. Por isso foram retiradas as chuvas, e não houve chuva
serôdia\footnote{Que aparece fora da estação própria.}; mas tu tens
a fronte de uma prostituta, e não queres ter vergonha. Ao menos
desde agora não chamarás por mim, dizendo: Pai meu, tu és o guia da
minha mocidade? Conservará ele para sempre a sua ira? Ou a
guardará continuamente? Eis que tens falado e feito quantas maldades
pudeste.

Disse mais o Senhor nos dias do rei Josias: Viste o que fez a
rebelde Israel? Ela foi a todo o monte alto, e debaixo de toda a
árvore verde, e ali andou prostituindo-se. E eu disse: Depois
que fizer tudo isto, voltará para mim; mas não voltou; e viu isto a
sua aleivosa irmã Judá. E vi que, por causa de tudo isto, por
ter cometido adultério a rebelde Israel, a despedi, e lhe dei a sua
carta de divórcio, que a aleivosa Judá, sua irmã, não temeu; mas se
foi e também ela mesma se prostituiu. E sucedeu que pela fama da
sua prostituição, contaminou a terra; porque adulterou com a pedra e
com a madeira. E, contudo, apesar de tudo isso a sua aleivosa
irmã Judá não voltou para mim de todo o seu coração, mas falsamente,
diz o Senhor. E o Senhor me disse: Já a rebelde Israel
mostrou-se mais justa do que a aleivosa Judá.

Vai, pois, e apregoa estas palavras para o lado norte, e dize:
Volta, ó rebelde Israel, diz o Senhor, e não farei cair a minha ira
sobre ti; porque misericordioso sou, diz o Senhor, e não conservarei
para sempre a minha ira. Somente reconhece a tua iniqüidade,
que transgrediste contra o Senhor teu Deus; e estendeste os teus
caminhos aos estranhos, debaixo de toda a árvore verde, e não deste
ouvidos à minha voz, diz o Senhor. Convertei-vos, ó filhos
rebeldes, diz o Senhor; pois eu vos desposei; e vos tomarei, a um de
uma cidade, e a dois de uma família; e vos levarei a Sião. E
dar-vos-ei pastores segundo o meu coração, os quais vos apascentarão
com ciência e com inteligência. E sucederá que, quando vos
multiplicardes e frutificardes na terra, naqueles dias, diz o
Senhor, nunca mais se dirá: A arca da aliança do Senhor, nem lhes
virá ao coração; nem dela se lembrarão, nem a visitarão; nem se fará
outra. Naquele tempo chamarão a Jerusalém o trono do Senhor,
e todas as nações se ajuntarão a ela, em nome do Senhor, em
Jerusalém; e nunca mais andarão segundo o propósito do seu coração
maligno. Naqueles dias andará a casa de Judá com a casa de
Israel; e virão juntas da terra do norte, para a terra que dei em
herança a vossos pais. Mas eu dizia: Como te porei entre os
filhos, e te darei a terra desejável, a excelente herança dos
exércitos das nações? Mas eu disse: Tu me chamarás meu pai, e de mim
não te desviarás.

Deveras, como a mulher se aparta aleivosamente do seu marido,
assim aleivosamente te houveste comigo, ó casa de Israel, diz o
Senhor. Nos lugares altos se ouviu uma voz, pranto e súplicas
dos filhos de Israel; porquanto perverteram o seu caminho, e se
esqueceram do Senhor seu Deus. Voltai, ó filhos rebeldes, eu
curarei as vossas rebeliões. Eis-nos aqui, vimos a ti; porque tu és
o Senhor nosso Deus. Certamente em vão se confia nos outeiros
e na multidão das montanhas; deveras no Senhor nosso Deus está a
salvação de Israel. Porque a confusão devorou o trabalho de
nossos pais desde a nossa mocidade; as suas ovelhas e o seu gado, os
seus filhos e as suas filhas. Deitemo-nos em nossa vergonha;
e cubra-nos a nossa confusão, porque pecamos contra o Senhor nosso
Deus, nós e nossos pais, desde a nossa mocidade até o dia de hoje; e
não demos ouvidos à voz do Senhor nosso Deus.

\medskip

\lettrine{4} Se voltares, ó Israel, diz o Senhor, volta para
mim; e se tirares as tuas abominações de diante de mim, não andarás
mais vagueando, e jurarás: Vive o Senhor na verdade, no juízo e
na justiça; e nele se bendirão as nações, e nele se gloriarão.

Porque assim diz o Senhor aos homens de Judá e a Jerusalém:
Preparai para vós o campo de lavoura, e não semeeis entre espinhos.
Circuncidai-vos ao Senhor, e tirai os prepúcios do vosso
coração, ó homens de Judá e habitantes de Jerusalém, para que o meu
furor não venha a sair como fogo, e arda de modo que não haja quem o
apague, por causa da malícia das vossas obras.

Anunciai em Judá, e fazei ouvir em Jerusalém, e dizei: Tocai a
trombeta na terra, gritai em alta voz, dizendo: Ajuntai-vos, e
entremos nas cidades fortificadas. Arvorai a bandeira rumo a
Sião, fugi, não vos detenhais; porque eu trago do norte um mal, e
uma grande destruição. Já um leão subiu da sua
ramada\footnote{Molho de ramos que se põe nos rios para que neles se
acolha o peixe. Ornamento feito com ramos; enramada. Cobertura feita
de ramos, para abrigo ou sombreamento; enramada.}, e um destruidor
dos gentios; ele já partiu, e saiu do seu lugar para fazer da tua
terra uma desolação, a fim de que as tuas cidades sejam destruídas,
e ninguém habite nelas. Por isto cingi-vos de sacos, lamentai, e
uivai, porque o ardor da ira do Senhor não se desviou de nós. E
sucederá naquele tempo, diz o Senhor, que se desfará o coração do
rei e o coração dos príncipes; e os sacerdotes pasmarão, e os
profetas se maravilharão. Então disse eu: Ah, Senhor Deus!
Verdadeiramente enganaste grandemente a este povo e a Jerusalém,
dizendo: Tereis paz; pois a espada penetra-lhe até à alma.
Naquele tempo se dirá a este povo e a Jerusalém: Um vento
seco das alturas do deserto veio ao caminho da filha do meu povo;
não para padejar\footnote{Revolver com a pá; atirar (o pão) ao ar
com a pá, para limpá-lo na eira. Exercer a profissão de padeiro,
fazer pão.}, nem para limpar; mas um vento mais veemente virá
da minha parte; agora também eu pronunciarei juízos contra eles.
Eis que virá subindo como nuvens e os seus carros como a
tormenta; os seus cavalos serão mais ligeiros do que as águias; ai
de nós, que somos assolados! Lava o teu coração da malícia, ó
Jerusalém, para que sejas salva; até quando permanecerão no meio de
ti os pensamentos da tua iniqüidade? Porque uma voz anuncia
desde Dã, e faz ouvir a calamidade desde o monte de Efraim.
Lembrai isto às nações; fazei ouvir contra Jerusalém, que
vigias vêm de uma terra remota, e levantarão a sua voz contra as
cidades de Judá. Como os guardas de um campo, estão contra
ela ao redor; porquanto ela se rebelou contra mim, diz o Senhor.
O teu caminho e as tuas obras te fizeram estas coisas; esta é
a tua maldade, e amargosa é, que te chega até ao coração.

Ah, entranhas minhas, entranhas minhas! Estou com dores no meu
coração! O meu coração se agita em mim. Não posso me calar; porque
tu, ó minha alma, ouviste o som da trombeta e o alarido da guerra.
Destruição sobre destruição se apregoa; porque já toda a
terra está destruída; de repente foram destruídas as minhas tendas,
e as minhas cortinas num momento. Até quando verei a
bandeira, e ouvirei a voz da trombeta? Deveras o meu povo
está louco, já não me conhece; são filhos néscios, e não entendidos;
são sábios para fazer mal, mas não sabem fazer o bem.
Observei a terra, e eis que era sem forma e vazia; também os
céus, e não tinham a sua luz. Observei os montes, e eis que
estavam tremendo; e todos os outeiros estremeciam. Observei,
e eis que não havia homem algum; e todas as aves do céu tinham
fugido. Vi também que a terra fértil era um deserto; e todas
as suas cidades estavam derrubadas diante do Senhor, diante do furor
da sua ira. Porque assim diz o Senhor: Toda esta terra será
assolada; de todo, porém, não a consumirei. Por isto
lamentará a terra, e os céus em cima se enegrecerão; porquanto assim
o disse, assim o propus, e não me arrependi nem me desviarei disso.
Ao clamor dos cavaleiros e dos flecheiros fugiram todas as
cidades; entraram pelas matas e treparam pelos penhascos; todas as
cidades ficaram abandonadas, e já ninguém habita nelas.
Agora, pois, que farás, ó assolada? Ainda que te vistas de
carmesim, ainda que te adornes com enfeites de ouro, ainda que te
pintes em volta dos teus olhos, debalde te farias bela; os amantes
te desprezam, e procuram tirar-te a vida. Porquanto ouço uma
voz, como a de uma mulher que está de parto, uma angústia como a de
que está com dores de parto do primeiro filho; a voz da filha de
Sião, ofegante, que estende as suas mãos, dizendo: Oh! ai de mim
agora, porque já a minha alma desmaia por causa dos assassinos.

\medskip

\lettrine{5} Dai voltas às ruas de Jerusalém, e vede agora; e
informai-vos, e buscai pelas suas praças, a ver se achais alguém, ou
se há homem que pratique a justiça ou busque a verdade; e eu lhe
perdoarei. E ainda que digam: Vive o Senhor, de certo falsamente
juram. Ah Senhor, porventura não atentam os teus olhos para a
verdade? Feriste-os, e não lhes doeu; consumiste-os, e não quiseram
receber a correção; endureceram as suas faces mais do que uma rocha;
não quiseram voltar. Eu, porém, disse: Deveras estes são pobres;
são loucos, pois não sabem o caminho do Senhor, nem o juízo do seu
Deus. Irei aos grandes, e falarei com eles; porque eles sabem o
caminho do Senhor, o juízo do seu Deus; mas estes juntamente
quebraram o jugo, e romperam as ataduras. Por isso um leão do
bosque os feriu, um lobo dos desertos os assolará; um leopardo vigia
contra as suas cidades; qualquer que sair delas será despedaçado;
porque as suas transgressões se avolumam, multiplicaram-se as suas
apostasias. Como, vendo isto, te perdoaria? Teus filhos me
deixam a mim e juram pelos que não são deuses; quando os fartei,
então adulteraram, e em casa de meretrizes se ajuntaram em bandos.
Como cavalos bem fartos, levantam-se pela manhã, rinchando cada
um à mulher do seu próximo. Deixaria eu de castigar por estas
coisas, diz o Senhor, ou não se vingaria a minha alma de uma nação
como esta?

Subi aos seus muros, e destruí-os (porém não façais uma
destruição final); tirai os seus ramos, porque não são do Senhor.
Porque aleivosissimamente\footnote{Aleivosia: Traição ou
crime cometido com falsas demonstrações de amizade; perfídia,
deslealdade. Qualidade de quem engana, atraiçoa; dolo, fraude.
Acusação fundamentada numa mentira (ger. feita por acinte); injúria,
calúnia.} se houveram contra mim a casa de Israel e a casa de Judá,
diz o Senhor. Negaram ao Senhor, e disseram: Não é ele; nem
mal nos sobrevirá, nem veremos espada nem fome. E até os
profetas serão como vento, porque a palavra não está com eles; assim
se lhes sucederá. Portanto assim diz o Senhor Deus dos
Exércitos: Porquanto disseste tal palavra, eis que converterei as
minhas palavras na tua boca em fogo, e a este povo em lenha, eles
serão consumidos. Eis que trarei sobre vós uma nação de
longe, ó casa de Israel, diz o Senhor; é uma nação robusta, é uma
nação antiqüíssima, uma nação cuja língua ignorarás, e não
entenderás o que ela falar. A sua aljava\footnote{Coldre ou
estojo sem tampa em que se guardavam e transportavam as setas, e que
era carregado nas costas, pendente do ombro.} é como uma sepultura
aberta; todos eles são poderosos. E comerão a tua sega e o
teu pão, que teus filhos e tuas filhas haviam de comer; comerão as
tuas ovelhas e as tuas vacas; comerão a tua vide e a tua figueira;
as tuas cidades fortificadas, em que confiavas, abatê-las-ão à
espada. Contudo, ainda naqueles dias, diz o Senhor, não farei
de vós uma destruição final. E sucederá que, quando
disserdes: Por que nos fez o Senhor nosso Deus todas estas coisas?
Então lhes dirás: Como vós me deixastes, e servistes a deuses
estranhos na vossa terra, assim servireis a estrangeiros, em terra
que não é vossa.

Anunciai isto na casa de Jacó, e fazei-o ouvir em Judá, dizendo:
Ouvi agora isto, ó povo insensato, e sem coração, que tendes
olhos e não vedes, que tendes ouvidos e não ouvis. Porventura
não me temereis a mim? diz o Senhor; não temereis diante de mim, que
pus a areia por limite ao mar, por ordenança eterna, que ele não
traspassará? Ainda que se levantem as suas ondas, não prevalecerão;
ainda que bramem, não a traspassarão. Mas este povo é de
coração rebelde e pertinaz\footnote{Que demonstra muita tenacidade;
persistente.}: rebelaram-se e foram-se. E não dizem no seu
coração: Temamos agora ao Senhor nosso Deus, que dá chuva, a temporã
e a tardia, ao seu tempo; e nos conserva as semanas determinadas da
sega.

As vossas iniqüidades desviam estas coisas, e os vossos pecados
apartam de vós o bem. Porque ímpios se acham entre o meu
povo; andam espiando, como quem arma laços; põem armadilhas, com que
prendem os homens. Como uma gaiola está cheia de pássaros,
assim as suas casas estão cheias de engano; por isso se
engrandeceram, e enriqueceram; engordam-se, estão
nédios\footnote{Que, por ter lustre, reluz; brilhante, luzidio. De
aspecto lustroso, devido à gordura.}, e ultrapassam até os feitos
dos malignos; não julgam a causa do órfão; todavia prosperam; nem
julgam o direito dos necessitados. Porventura não castigaria
eu por causa destas coisas? diz o Senhor; não me vingaria eu de uma
nação como esta? Coisa espantosa e horrenda se anda fazendo
na terra. Os profetas profetizam falsamente, e os sacerdotes
dominam pelas mãos deles, e o meu povo assim o deseja; mas que
fareis ao fim disto?

\medskip

\lettrine{6} Fugi para salvação vossa, filhos de Benjamim, do
meio de Jerusalém; e tocai a buzina em Tecoa, e levantai um sinal de
fogo sobre Bete-Haquerém; porque do lado norte surge um mal e uma
grande destruição. À formosa e delicada assemelhei a filha de
Sião. Mas contra ela virão pastores com os seus rebanhos;
levantarão contra ela tendas em redor, e cada um apascentará no seu
lugar. Preparai a guerra contra ela, levantai-vos, e subamos ao
pino do meio dia. Ai de nós! Já declina o dia, já se vão estendendo
as sombras da tarde. Levantai-vos, e subamos de noite, e
destruamos os seus palácios. Porque assim diz o Senhor dos
Exércitos: Cortai árvores, e levantai trincheiras contra Jerusalém;
esta é a cidade que há de ser castigada, só opressão há no meio
dela. Como a fonte produz as suas águas, assim ela produz a sua
malícia; violência e estrago se ouvem nela; enfermidade e feridas há
diante de mim continuamente. Corrige-te, ó Jerusalém, para que a
minha alma não se aparte de ti, para que não te torne em assolação e
terra não habitada.

Assim diz o Senhor dos Exércitos: Diligentemente
respigarão\footnote{Apanhar no campo (as espigas que ali ficaram
após a colheita). Rebuscar: apanhar (frutos que ficaram na planta
após a colheita); respigar. Ex.: r. a videira.} os resíduos de
Israel como uma vinha; torna a tua mão, como o vindimador, aos
cestos. A quem falarei e testemunharei, para que ouça? Eis
que os seus ouvidos estão incircuncisos, e não podem ouvir; eis que
a palavra do Senhor é para eles coisa vergonhosa, e não gostam dela.
Por isso estou cheio do furor do Senhor; estou cansado de o
conter; derramá-lo-ei sobre os meninos pelas ruas e na reunião de
todos os jovens; porque até o marido com a mulher serão presos, e o
velho com o que está cheio de dias. E as suas casas passarão
a outros, como também as suas herdades\footnote{Propriedade rural de
dimensões consideráveis; fazenda, quinta. Herança.} e as suas
mulheres juntamente; porque estenderei a minha mão contra os
habitantes desta terra, diz o Senhor. Porque desde o menor
deles até ao maior, cada um se dá à avareza; e desde o profeta até
ao sacerdote, cada um usa de falsidade. E curam
superficialmente a ferida da filha do meu povo, dizendo: Paz, paz;
quando não há paz. Porventura envergonham-se de cometer
abominação? Pelo contrário, de maneira nenhuma se envergonham, nem
tampouco sabem que coisa é envergonhar-se; portanto cairão entre os
que caem; no tempo em que eu os visitar, tropeçarão, diz o Senhor.
Assim diz o Senhor: Ponde-vos nos caminhos, e vede, e
perguntai pelas veredas antigas, qual é o bom caminho, e andai por
ele; e achareis descanso para as vossas almas; mas eles dizem: Não
andaremos nele. Também pus atalaias sobre vós, dizendo: Estai
atentos ao som da trombeta; mas dizem: Não escutaremos.

Portanto ouvi, vós, nações; e informa-te tu, ó congregação, do
que se faz entre eles! Ouve tu, ó terra! Eis que eu trarei
mal sobre este povo, o próprio fruto dos seus pensamentos; porque
não estão atentos às minhas palavras, e rejeitam a minha lei.
Para que, pois, me vem o incenso de Sabá e a melhor cana
aromática de terras remotas? Vossos holocaustos não me agradam, nem
me são suaves os vossos sacrifícios. Portanto assim diz o
Senhor: Eis que armarei tropeços a este povo; e tropeçarão neles
pais e filhos juntamente; o vizinho e o seu companheiro perecerão.
Assim diz o Senhor: Eis que um povo vem da terra do norte, e
uma grande nação se levantará das extremidades da terra. Arco
e lança trarão; são cruéis, e não usarão de misericórdia; a sua voz
rugirá como o mar, e em cavalos virão montados, dispostos como
homens de guerra contra ti, ó filha de Sião. Ouvimos a sua
fama, afrouxaram-se as nossas mãos; angústia nos tomou, e dores como
as de parturiente. Não saiais ao campo, nem andeis pelo
caminho; porque espada do inimigo e espanto há ao redor. Ó
filha do meu povo, cinge-te de saco, e revolve-te na cinza; pranteia
como por um filho único, pranto de amargura; porque de repente virá
o destruidor sobre nós. Por torre de guarda te pus entre o
meu povo, por fortaleza, para que soubesses e examinasses o seu
caminho. Todos eles são os mais rebeldes, andam murmurando;
são duros como bronze e ferro; todos eles são corruptores. Já
o fole se queimou, o chumbo se consumiu com o fogo; em vão fundiu o
fundidor tão diligentemente, pois os maus não são arrancados.
Prata rejeitada lhes chamarão, porque o Senhor os rejeitou.

\medskip

\lettrine{7} A palavra que da parte do Senhor, veio a
Jeremias, dizendo: Põe-te à porta da casa do Senhor, e proclama
ali esta palavra, e dize: Ouvi a palavra do Senhor, todos de Judá,
os que entrais por estas portas, para adorardes ao Senhor. Assim
diz o Senhor dos Exércitos, o Deus de Israel: Melhorai os vossos
caminhos e as vossas obras, e vos farei habitar neste lugar. Não
vos fieis em palavras falsas, dizendo: Templo do Senhor, templo do
Senhor, templo do Senhor é este. Mas, se deveras melhorardes os
vossos caminhos e as vossas obras; se deveras praticardes o juízo
entre um homem e o seu próximo; se não oprimirdes o estrangeiro,
e o órfão, e a viúva, nem derramardes sangue inocente neste lugar,
nem andardes após outros deuses para vosso próprio mal, eu vos
farei habitar neste lugar, na terra que dei a vossos pais, desde os
tempos antigos e para sempre. Eis que vós confiais em palavras
falsas, que para nada vos aproveitam. Porventura furtareis, e
matareis, e adulterareis, e jurareis falsamente, e queimareis
incenso a Baal, e andareis após outros deuses que não conhecestes,
e então vireis, e vos poreis diante de mim nesta casa, que se
chama pelo meu nome, e direis: Fomos libertados para fazermos todas
estas abominações? É pois esta casa, que se chama pelo meu
nome, uma caverna de salteadores aos vossos olhos? Eis que eu, eu
mesmo, vi isto, diz o Senhor. Mas ide agora ao meu lugar, que
estava em Siló, onde, ao princípio, fiz habitar o meu nome, e vede o
que lhe fiz, por causa da maldade do meu povo Israel. Agora,
pois, porquanto fazeis todas estas obras, diz o Senhor, e eu vos
falei, madrugando, e falando, e não ouvistes, e chamei-vos, e não
respondestes, farei também a esta casa, que se chama pelo meu
nome, na qual confiais, e a este lugar, que vos dei a vós e a vossos
pais, como fiz a Siló. E lançar-vos-ei de diante de minha
face, como lancei a todos os vossos irmãos, a toda a geração de
Efraim.

Tu, pois, não ores por este povo, nem levantes por ele clamor ou
oração, nem me supliques, porque eu não te ouvirei.
Porventura não vês tu o que andam fazendo nas cidades de
Judá, e nas ruas de Jerusalém? Os filhos apanham a lenha, e
os pais acendem o fogo, e as mulheres preparam a massa, para fazerem
bolos à rainha dos céus, e oferecem libações a outros deuses, para
me provocarem à ira. Acaso é a mim que eles provocam à ira?
diz o Senhor, e não a si mesmos, para confusão dos seus rostos?
Portanto assim diz o Senhor Deus: Eis que a minha ira e o meu
furor se derramarão sobre este lugar, sobre os homens e sobre os
animais, e sobre as árvores do campo, e sobre os frutos da terra; e
acender-se-á, e não se apagará.

Assim diz o Senhor dos Exércitos, o Deus de Israel: Ajuntai os
vossos holocaustos aos vossos sacrifícios, e comei carne.
Porque nunca falei a vossos pais, no dia em que os tirei da
terra do Egito, nem lhes ordenei coisa alguma acerca de holocaustos
ou sacrifícios. Mas isto lhes ordenei, dizendo: Dai ouvidos à
minha voz, e eu serei o vosso Deus, e vós sereis o meu povo; e andai
em todo o caminho que eu vos mandar, para que vos vá bem. Mas
não ouviram, nem inclinaram os seus ouvidos, mas andaram nos seus
próprios conselhos, no propósito do seu coração malvado; e andaram
para trás, e não para diante. Desde o dia em que vossos pais
saíram da terra do Egito, até hoje, enviei-vos todos os meus servos,
os profetas, todos os dias madrugando e enviando-os. Mas não
me deram ouvidos, nem inclinaram os seus ouvidos, mas endureceram a
sua cerviz, e fizeram pior do que seus pais. Dir-lhes-ás,
pois, todas estas palavras, mas não te darão ouvidos; chamá-los-ás,
mas não te responderão. E lhes dirás: Esta é a nação que não
deu ouvidos à voz do Senhor seu Deus e não aceitou a correção; já
pereceu a verdade, e foi cortada da sua boca.

Corta o teu cabelo e lança-o de ti, e levanta um pranto sobre as
alturas; porque já o Senhor rejeitou e desamparou a geração do seu
furor. Porque os filhos de Judá fizeram o que era mau aos
meus olhos, diz o Senhor; puseram as suas abominações na casa que se
chama pelo meu nome, para contaminá-la. E edificaram os altos
de Tofete, que está no Vale do Filho de Hinom, para queimarem no
fogo a seus filhos e a suas filhas, o que nunca ordenei, nem me
subiu ao coração. Portanto, eis que vêm dias, diz o Senhor,
em que não se chamará mais Tofete, nem Vale do Filho de Hinom, mas o
Vale da Matança; e enterrarão em Tofete, por não haver outro lugar.
E os cadáveres deste povo servirão de pasto às aves dos céus
e aos animais da terra; e ninguém os espantará. E farei
cessar nas cidades de Judá, e nas ruas de Jerusalém, a voz de gozo,
e a voz de alegria, a voz de esposo e a voz de esposa; porque a
terra se tornará em desolação.

\medskip

\lettrine{8} Naquele tempo, diz o Senhor, tirarão para fora
das suas sepulturas os ossos dos reis de Judá, e os ossos dos seus
príncipes, e os ossos dos sacerdotes, e os ossos dos profetas, e os
ossos dos habitantes de Jerusalém; e expô-los-ão ao sol, e à
lua, e a todo o exército do céu, a quem tinham amado, e a quem
tinham servido, e após quem tinham ido, e a quem tinham buscado e
diante de quem se tinham prostrado; não serão recolhidos nem
sepultados; serão como esterco sobre a face da terra. E será
escolhida antes a morte do que a vida por todos os que restarem
desta raça maligna, que ficarem em todos os lugares onde os lancei,
diz o Senhor dos Exércitos.

Dize-lhes mais: Assim diz o Senhor: Porventura cairão e não se
tornarão a levantar? Desviar-se-ão, e não voltarão? Por que,
pois, se desvia este povo de Jerusalém com uma apostasia tão
contínua? Persiste no engano, não quer voltar. Eu escutei e
ouvi; não falam o que é reto, ninguém há que se arrependa da sua
maldade, dizendo: Que fiz eu? Cada um se desvia na sua carreira,
como um cavalo que arremete com ímpeto na batalha. Até a cegonha
no céu conhece os seus tempos determinados; e a rola, e o
grou\footnote{Design. comum às aves da fam. dos gruídeos,
encontradas em planícies e zonas pantanosas de todo o mundo, com
exceção da América do Sul e Antártica; de grande porte, pernas e
pescoço longos, cabeça parcialmente nua, bico reto e plumagem com
penas brancas, cinzas ou marrons (Algumas spp. estão ameaçadas de
extinção.)} e a andorinha observam o tempo da sua
arribação\footnote{Deslocamento de animais, ger. aves, de uma região
para outra em determinadas épocas ou estações do ano.}; mas o meu
povo não conhece o juízo do Senhor. Como, pois, dizeis: Nós
somos sábios, e a lei do Senhor está conosco? Eis que em vão tem
trabalhado a falsa pena dos escribas. Os sábios são
envergonhados, espantados e presos; eis que rejeitaram a palavra do
Senhor; que sabedoria, pois, têm eles? Portanto darei suas
mulheres a outros, e os seus campos a novos possuidores; porque
desde o menor até ao maior, cada um deles se dá à avareza; desde o
profeta até ao sacerdote, cada um deles usa de falsidade. E
curam a ferida da filha de meu povo levianamente, dizendo: Paz, paz;
quando não há paz. Porventura envergonham-se de cometerem
abominação? Não; de maneira nenhuma se envergonham, nem sabem que
coisa é envergonhar-se; portanto cairão entre os que caem e
tropeçarão no tempo em que eu os visitar, diz o Senhor.

Certamente os apanharei, diz o Senhor; já não há uvas na vide,
nem figos na figueira, e até a folha caiu; e o que lhes dei passará
deles. Por que nos assentamos ainda? Juntai-vos e entremos
nas cidades fortificadas, e ali pereçamos; pois já o Senhor nosso
Deus nos destinou a perecer e nos deu a beber água de
fel\footnote{Sabor amargo; amargor, amaridão.}; porquanto pecamos
contra o Senhor. Espera-se a paz, mas não há bem; o tempo da
cura, e eis o terror. Já desde Dã se ouve o resfolegar dos
seus cavalos, toda a terra treme ao som dos rinchos dos seus fortes;
e vêm, e devoram a terra, e sua abundância, a cidade e os que
habitam nela. Porque eis que envio entre vós serpentes e
basiliscos, contra os quais não há encantamento, e vos morderão, diz
o Senhor. Oh! se eu pudesse consolar-me na minha tristeza! O
meu coração desfalece em mim. Eis a voz do clamor da filha do
meu povo de terra mui remota; não está o Senhor em Sião? Não está
nela o seu rei? Por que me provocaram à ira com as suas imagens de
escultura, com vaidades estranhas? Passou a sega, findou o
verão, e nós não estamos salvos. Estou quebrantado pela
ferida da filha do meu povo; ando de luto; o espanto se apoderou de
mim. Porventura não há bálsamo em Gileade? Ou não há lá
médico? Por que, pois, não se realizou a cura da filha do meu povo?

\medskip

\lettrine{9} Oh! Se a minha cabeça se tornasse em águas, e os
meus olhos numa fonte de lágrimas! Então choraria de dia e de noite
os mortos da filha do meu povo. Oh! se tivesse no deserto uma
estalagem de caminhantes! Então deixaria o meu povo, e me apartaria
dele, porque todos eles são adúlteros, um bando de aleivosos. E
encurvam a língua como se fosse o seu arco, para a mentira;
fortalecem-se na terra, mas não para a verdade; porque avançam de
malícia em malícia, e a mim não me conhecem, diz o Senhor.
Guardai-vos cada um do seu próximo, e de irmão nenhum vos fieis;
porque todo o irmão não faz mais do que enganar, e todo o próximo
anda caluniando. E zombará cada um do seu próximo, e não falam a
verdade; ensinam a sua língua a falar a mentira, andam-se cansando
em proceder perversamente. A tua habitação está no meio do
engano; pelo engano recusam conhecer-me, diz o Senhor. Portanto
assim diz o Senhor dos Exércitos: Eis que eu os fundirei e os
provarei; pois, de que outra maneira procederia com a filha do meu
povo? Uma flecha mortífera é a língua deles; fala engano; com a
sua boca fala cada um de paz com o seu próximo mas no seu coração
arma-lhe ciladas. Porventura por estas coisas não os castigaria?
diz o Senhor; ou não se vingaria a minha alma de nação tal como
esta? Pelos montes levantarei choro e pranto, e pelas
pastagens do deserto lamentação; porque já estão queimadas, e
ninguém passa por elas; nem se ouve mugido de gado; desde as aves
dos céus, até os animais, andaram vagueando, e fugiram. E
farei de Jerusalém montões de pedras, morada de chacais, e das
cidades de Judá farei assolação, de sorte que não haja habitante.

Quem é o homem sábio, que entenda isto? e a quem falou a boca do
Senhor, para que o possa anunciar? Por que razão pereceu a terra, e
se queimou como deserto, sem que ninguém passa por ela? E
disse o Senhor: Porque deixaram a minha lei, que pus perante eles, e
não deram ouvidos à minha voz, nem andaram nela, antes
andaram após o propósito do seu próprio coração, e após os baalins,
como lhes ensinaram os seus pais. Portanto assim diz o Senhor
dos Exércitos, Deus de Israel: Eis que darei de comer
losna\footnote{Absinto: erva aromática (Artemisia absinthium) da
fam. das compostas, muito ramosa, nativa da Europa e cultivada em
todo o mundo, esp. pelas raízes e folhas, us. em infusão e de que se
extrai óleo volátil tóxico, us. no licor de absinto, com ação sobre
o sistema nervoso.} a este povo, e lhe darei a beber água de fel.
E os espalharei entre gentios, que não conheceram, nem eles
nem seus pais, e mandarei a espada após eles, até que venha a
consumi-los. Assim diz o Senhor dos Exércitos: Considerai, e
chamai carpideiras\footnote{Mulher mercenária que pranteava os
mortos durante os funerais. Derivação (por extensão de sentido):
mulher que se lamenta, que chora com freqüência.} que venham; e
mandai procurar mulheres hábeis, para que venham. E se
apressem, e levantem o seu lamento sobre nós; e desfaçam-se em
lágrimas os nossos olhos, e as nossas pálpebras destilem águas.
Porque uma voz de pranto se ouviu de Sião: Como estamos
arruinados! Estamos mui envergonhados, porque deixamos a terra, e
por terem eles lançado fora as nossas moradas. Ouvi, pois,
vós, mulheres, a palavra do Senhor, e os vossos ouvidos recebam a
palavra da sua boca; e ensinai o pranto a vossas filhas, e cada uma
à sua vizinha a lamentação; porque a morte subiu pelas nossas
janelas, e entrou em nossos palácios, para exterminar as crianças
das ruas e os jovens das praças. Fala: Assim diz o Senhor:
Até os cadáveres dos homens jazerão como esterco sobre a face do
campo, e como gavela\footnote{Feixe de espigas ceifadas. Molho de
palha ou feno.} atrás do segador, e não há quem a recolha.

Assim diz o Senhor: Não se glorie o sábio na sua sabedoria, nem
se glorie o forte na sua força; não se glorie o rico nas suas
riquezas, mas o que se gloriar, glorie-se nisto: em me
entender e me conhecer, que eu sou o Senhor, que faço beneficência,
juízo e justiça na terra; porque destas coisas me agrado, diz o
Senhor. Eis que vêm dias, diz o Senhor, em que castigarei a
todo o circuncidado com o incircunciso. Ao Egito, e a Judá, e
a Edom, e aos filhos de Amom, e a Moabe, e a todos os que cortam os
cantos do seu cabelo, que habitam no deserto; porque todas as nações
são incircuncisas, e toda a casa de Israel é incircuncisa de
coração.

\medskip

\lettrine{10} Ouvi a palavra que o Senhor vos fala a vós, ó
casa de Israel. Assim diz o Senhor: Não aprendais o caminho dos
gentios, nem vos espanteis dos sinais dos céus; porque com eles se
atemorizam as nações. Porque os costumes dos povos são vaidade;
pois corta-se do bosque um madeiro, obra das mãos do artífice, feita
com machado; com prata e com ouro o enfeitam, com pregos e com
martelos o firmam, para que não se mova. São como a palmeira,
obra torneada, porém não podem falar; certamente são levados,
porquanto não podem andar. Não tenhais receio deles, pois não podem
fazer mal, nem tampouco têm poder de fazer bem. Ninguém há
semelhante a ti, ó Senhor; tu és grande, e grande o teu nome em
poder. Quem não te temeria a ti, ó Rei das nações? Pois isto só
a ti pertence; porquanto entre todos os sábios das nações, e em todo
o seu reino, ninguém há semelhante a ti. Mas eles todos se
embruteceram e tornaram-se loucos; ensino de vaidade é o madeiro.
Trazem prata batida de Társis e ouro de Ufaz, trabalho do
artífice, e das mãos do fundidor; fazem suas roupas de azul e
púrpura; obra de peritos são todos eles. Mas o Senhor Deus é
a verdade; ele mesmo é o Deus vivo e o Rei eterno; ao seu furor
treme a terra, e as nações não podem suportar a sua indignação.
Assim lhes direis: Os deuses que não fizeram os céus e a
terra desaparecerão da terra e de debaixo deste céu. Ele fez
a terra com o seu poder; ele estabeleceu o mundo com a sua
sabedoria, e com a sua inteligência estendeu os céus. Fazendo
ele soar a sua voz, logo há rumor de águas no céu, e faz subir os
vapores da extremidade da terra; faz os relâmpagos para a chuva, e
dos seus tesouros faz sair o vento. Todo o homem é
embrutecido no seu conhecimento; envergonha-se todo o fundidor da
sua imagem de escultura; porque sua imagem fundida é mentira, e
nelas não há espírito. Vaidade são, obra de enganos: no tempo
da sua visitação virão a perecer. Não é semelhante a estes
aquele que é a porção de Jacó; porque ele é o que formou tudo, e
Israel é a vara da sua herança: Senhor dos Exércitos é o seu nome.

Ajunta da terra a tua mercadoria, ó tu que habitas em lugar
sitiado. Porque assim diz o Senhor: Eis que desta vez
arrojarei como se fora com uma funda aos moradores da terra, e os
angustiarei, para que venham a achá-lo, dizendo: Ai de mim
por causa do meu quebrantamento! A minha chaga me causa grande dor;
e eu havia dito: Certamente isto é enfermidade que eu poderei
suportar. A minha tenda está destruída, e todas as minhas
cordas se romperam; os meus filhos foram-se de mim, e não existem;
ninguém há mais que estenda a minha tenda, nem que levante as minhas
cortinas, porque os pastores se embruteceram, e não buscaram
ao Senhor; por isso não prosperaram, e todos os seus rebanhos se
espalharam. Eis que vem uma voz de rumor, grande tremor da
terra do norte, para fazer das cidades de Judá uma assolação, uma
morada de chacais. Eu sei, ó Senhor, que não é do homem o seu
caminho; nem do homem que caminha o dirigir os seus passos.
Castiga-me, ó Senhor, porém com juízo, não na tua ira, para
que não me reduzas a nada. Derrama a tua indignação sobre os
gentios que não te conhecem, e sobre as gerações que não invocam o
teu nome; porque devoraram a Jacó, e devoraram-no e consumiram-no, e
assolaram a sua morada.

\medskip

\lettrine{11} A palavra que veio a Jeremias, da parte do
Senhor, dizendo: Ouvi as palavras desta aliança, e falai aos
homens de Judá, e aos habitantes de Jerusalém. Dize-lhes pois:
Assim diz o Senhor Deus de Israel: Maldito o homem que não escutar
as palavras desta aliança, que ordenei a vossos pais no dia em
que os tirei da terra do Egito, da fornalha de ferro, dizendo: Dai
ouvidos à minha voz, e fazei conforme a tudo quanto vos mando; e vós
sereis o meu povo, e eu serei o vosso Deus. Para que confirme o
juramento que fiz a vossos pais de dar-lhes uma terra que manasse
leite e mel, como se vê neste dia. Então eu respondi, e disse: Amém,
ó Senhor. E disse-me o Senhor: Apregoa todas estas palavras nas
cidades de Judá, e nas ruas de Jerusalém, dizendo: Ouvi as palavras
desta aliança, e cumpri-as. Porque deveras adverti a vossos
pais, no dia em que os tirei da terra do Egito, até ao dia de hoje,
madrugando, e protestando, e dizendo: Dai ouvidos à minha voz.
Mas não ouviram, nem inclinaram os seus ouvidos, antes andaram
cada um conforme o propósito do seu coração malvado; por isso trouxe
sobre eles todas as palavras desta aliança que lhes mandei que
cumprissem, porém não cumpriram. Disse-me mais o Senhor: Uma
conspiração se achou entre os homens de Judá, entre os habitantes de
Jerusalém. Tornaram às maldades de seus primeiros pais, que
não quiseram ouvir as minhas palavras; e eles andaram após outros
deuses para os servir; a casa de Israel e a casa de Judá quebraram a
minha aliança, que tinha feito com seus pais.

Portanto assim diz o Senhor: Eis que trarei mal sobre eles, de
que não poderão escapar; e clamarão a mim, mas eu não os ouvirei.
Então irão as cidades de Judá e os habitantes de Jerusalém e
clamarão aos deuses a quem eles queimaram incenso; estes, porém, de
nenhum modo os livrarão no tempo do seu mal. Porque, segundo
o número das tuas cidades, são os teus deuses, ó Judá! E, segundo o
número das ruas de Jerusalém, levantastes altares à
impudência\footnote{Caráter ou qualidade de impudente. Falta de
pudor; descaramento, impudor. Falta de moral; cinismo, desfaçatez.},
altares para queimardes incenso a Baal. Tu, pois, não ores
por este povo, nem levantes por ele clamor nem oração; porque não os
ouvirei no tempo em que eles clamarem a mim, por causa do seu mal.
Que direito tem a minha amada na minha casa, visto que com
muitos tem cometido grande lascívia\footnote{Qualidade ou caráter de
lascivo. Propensão para a luxúria, sensualidade exagerada;
lubricidade. Caráter do que está marcado pela sensualidade ou do que
produz a propensão para a sensualidade; impudicícia.}? Crês que os
sacrifícios e as carnes santificadas poderão afastar de ti o mal?
Então saltarias de prazer. Denominou-te o Senhor oliveira
verde, formosa por seus deliciosos frutos, mas agora à voz de um
grande tumulto acendeu fogo ao redor dela e se quebraram os seus
ramos. Porque o Senhor dos Exércitos, que te plantou,
pronunciou contra ti o mal, pela maldade da casa de Israel e da casa
de Judá, que para si mesma fizeram, pois me provocaram à ira,
queimando incenso a Baal.

E o Senhor me fez saber, e assim o soube; então me fizeste ver as
suas ações. E eu era como um cordeiro, como um boi que levam
à matança; porque não sabia que maquinavam propósitos contra mim,
dizendo: Destruamos a árvore com o seu fruto, e cortemo-lo da terra
dos viventes, e não haja mais memória do seu nome. Mas, ó
Senhor dos Exércitos, justo Juiz, que provas os rins e o coração,
veja eu a tua vingança sobre eles; pois a ti descobri a minha causa.
Portanto, assim diz o Senhor acerca dos homens de Anatote,
que buscam a tua vida, dizendo: Não profetizes no nome do Senhor,
para que não morras às nossas mãos. Portanto, assim diz o
Senhor dos Exércitos: Eis que eu os castigarei; os jovens morrerão à
espada, os seus filhos e suas filhas morrerão de fome. E não
haverá deles um remanescente, porque farei vir o mal sobre os homens
de Anatote, no ano da sua visitação.

\medskip

\lettrine{12} Justo serias, ó Senhor, ainda que eu entrasse
contigo num pleito; contudo falarei contigo dos teus juízos. Por que
prospera o caminho dos ímpios, e vivem em paz todos os que procedem
aleivosamente? Plantaste-os, e eles se arraigaram; crescem, dão
também fruto; chegado estás à sua boca, porém longe dos seus rins.
Mas tu, ó Senhor, me conheces, tu me vês, e provas o meu coração
para contigo; arranca-os como as ovelhas para o matadouro, e
dedica-os para o dia da matança. Até quando lamentará a terra, e
se secará a erva de todo o campo? Pela maldade dos que habitam nela,
perecem os animais e as aves; porquanto dizem: Ele não verá o nosso
fim. Se te fatigas correndo com homens que vão a pé, como
poderás competir com os cavalos? Se tão-somente numa terra de paz
estás confiado, como farás na enchente do Jordão? Porque até os
teus irmãos, e a casa de teu pai, eles próprios procedem
deslealmente contigo; eles mesmos clamam após ti em altas vozes: Não
te fies neles, ainda que te digam coisas boas.

Desamparei a minha casa, abandonei a minha herança; entreguei a
amada da minha alma na mão de seus inimigos. Tornou-se a minha
herança para mim como leão numa floresta; levantou a sua voz contra
mim, por isso eu a odiei. A minha herança é para mim ave de
rapina de várias cores. Andam as aves de rapina contra ela em redor.
Vinde, pois, ajuntai todos os animais do campo, trazei-os para a
devorarem. Muitos pastores destruíram a minha vinha, pisaram
o meu campo; tornaram em desolado deserto o meu campo desejado.
Em desolação a puseram, e clama a mim na sua desolação; e
toda a terra está desolada, porquanto não há ninguém que tome isso a
sério. Sobre todos os lugares altos do deserto vieram
destruidores; porque a espada do Senhor devora desde um extremo da
terra até o outro; não há paz para nenhuma carne. Semearam
trigo, e segaram espinhos; cansaram-se, mas de nada se aproveitaram;
envergonhados sereis das vossas colheitas, e por causa do ardor da
ira do Senhor.

Assim diz o Senhor, acerca de todos os meus maus vizinhos, que
tocam a minha herança, que fiz herdar ao meu povo Israel: Eis que os
arrancarei da sua terra, e a casa de Judá arrancarei do meio deles.
E será que, depois de os haver arrancado, tornarei, e me
compadecerei deles, e os farei voltar cada um à sua herança, e cada
um à sua terra. E será que, se diligentemente aprenderem os
caminhos do meu povo, jurando pelo meu nome: Vive o Senhor, como
ensinaram o meu povo a jurar por Baal; então edificar-se-ão no meio
do meu povo. Mas se não quiserem ouvir, totalmente arrancarei
a tal nação, e a farei perecer, diz o Senhor.

\medskip

\lettrine{13} Assim me disse o Senhor: Vai, e compra um cinto
de linho e põe-no sobre os teus lombos, mas não o coloques na água.
E comprei o cinto, conforme a palavra do Senhor, e o pus sobre
os meus lombos. Então me veio a palavra do Senhor pela segunda
vez, dizendo: Toma o cinto que compraste, e que trazes sobre os
teus lombos, e levanta-te; vai ao Eufrates, e esconde-o ali na fenda
de uma rocha. E fui, e escondi-o junto ao Eufrates, como o
Senhor me havia ordenado. Sucedeu, ao final de muitos dias, que
me disse o Senhor: Levanta-te, vai ao Eufrates, e toma dali o cinto
que te ordenei que o escondesses ali. E fui ao Eufrates, e
cavei, e tomei o cinto do lugar onde o havia escondido; e eis que o
cinto tinha apodrecido, e para nada prestava. Então veio a mim a
palavra do Senhor, dizendo: Assim diz o Senhor: Do mesmo modo
farei apodrecer a soberba de Judá, e a muita soberba de Jerusalém.
Este povo maligno, que recusa ouvir as minhas palavras, que
caminha segundo a dureza do seu coração, e anda após deuses alheios,
para servi-los, e inclinar-se diante deles, será tal como este
cinto, que para nada presta. Porque, como o cinto está pegado
aos lombos do homem, assim eu liguei a mim toda a casa de Israel, e
toda a casa de Judá, diz o Senhor, para me serem por povo, e por
nome, e por louvor, e por glória; mas não deram ouvidos.

Portanto, dize-lhes esta palavra: Assim diz o Senhor Deus de
Israel: Todo o odre se encherá de vinho; e dir-te-ão: Porventura não
sabemos nós muito bem que todo o odre se encherá de vinho?
Mas tu dize-lhes: Assim diz o Senhor: Eis que eu encherei de
embriaguez a todos os habitantes desta terra, e aos reis da estirpe
de Davi, que estão assentados sobre o seu trono, e aos sacerdotes, e
aos profetas, e a todos os habitantes de Jerusalém. E
fá-los-ei em pedaços atirando uns contra os outros, e juntamente os
pais com os filhos, diz o Senhor; não perdoarei, nem pouparei, nem
terei deles compaixão, para que não os destrua. Escutai, e
inclinai os ouvidos; não vos ensoberbeçais; porque o Senhor falou:
Dai glória ao Senhor vosso Deus, antes que venha a escuridão
e antes que tropecem vossos pés nos montes tenebrosos; antes que,
esperando vós luz, ele a mude em sombra de morte, e a reduza à
escuridão. E, se isto não ouvirdes, a minha alma chorará em
lugares ocultos, por causa da vossa soberba; e amargamente chorarão
os meus olhos, e se desfarão em lágrimas, porquanto o rebanho do
Senhor foi levado cativo. Dize ao rei e à rainha:
Humilhai-vos, e assentai-vos no chão; porque já caiu todo o ornato
de vossas cabeças, a coroa da vossa glória. As cidades do sul
estão fechadas, e ninguém há que as abra; todo o Judá foi levado
cativo, sim, inteiramente cativo. Levantai os vossos olhos, e
vede os que vêm do norte; onde está o rebanho que se te deu, o
rebanho da tua glória? Que dirás, quando ele te castigar
porque os ensinaste a serem capitães, e chefe sobre ti? Porventura
não te tomarão as dores, como à mulher que está de parto?

Quando, pois, disseres no teu coração: Por que me sobrevieram
estas coisas? Pela multidão das tuas maldades se descobriram as tuas
fraldas\footnote{A parte de baixo da camisa. A parte inferior de
qualquer peça do vestuário feminino ou masculino; falda. Vestimenta,
vestes. Saia branca que se põe por baixo do vestido.}, e os teus
calcanhares sofrem violência. Porventura pode o etíope mudar
a sua pele, ou o leopardo as suas manchas? Então podereis vós fazer
o bem, sendo ensinados a fazer o mal. Assim os espalharei
como o restolho, que passa com o vento do deserto. Esta será
a tua sorte, a porção que te será medida por mim, diz o Senhor; pois
te esqueceste de mim, e confiaste em mentiras. Assim também
eu levantarei as tuas fraldas sobre o teu rosto; e aparecerá a tua
ignomínia\footnote{Grande desonra infligida por um julgamento
público; degradação social; opróbrio. Caráter daquilo que degrada,
humilha; ação, palavra que desonra, que envergonha.}. Já vi
as tuas abominações, e os teus adultérios, e os teus rinchos, e a
enormidade da tua prostituição sobre os outeiros no campo. Ai de ti,
Jerusalém! Até quando ainda não te purificarás?

\medskip

\lettrine{14} A palavra do Senhor, que veio a Jeremias, a
respeito da grande seca. Anda chorando Judá, e as suas portas
estão enfraquecidas; andam de luto até ao chão, e o clamor de
Jerusalém vai subindo. E os seus mais ilustres enviam os seus
pequenos a buscar água; vão às cisternas, e não acham água; voltam
com os seus cântaros vazios; envergonham-se e confundem-se, e cobrem
as suas cabeças. Por causa da terra que se fendeu, porque não há
chuva sobre a terra, os lavradores se envergonham e cobrem as suas
cabeças. Porque até as cervas no campo têm as suas crias, e
abandonam seus filhos, porquanto não há erva. E os jumentos
monteses se põem nos lugares altos, sorvem o vento como os chacais;
desfalecem os seus olhos, porquanto não há erva. Posto que as
nossas maldades testificam contra nós, ó Senhor, age por amor do teu
nome; porque as nossas rebeldias se multiplicaram; contra ti
pecamos. Ó esperança de Israel, e Redentor seu no tempo da
angústia, por que serias como um estrangeiro na terra e como o
viandante que se retira a passar a noite? Por que serias como
homem surpreendido, como poderoso que não pode livrar? Mas tu estás
no meio de nós, ó Senhor, e nós somos chamados pelo teu nome; não
nos desampares.

Assim diz o Senhor, acerca deste povo: Pois que tanto gostaram de
andar errantes, e não retiveram os seus pés, por isso o Senhor não
se agrada deles, mas agora se lembrará da iniqüidade deles, e
visitará os seus pecados. Disse-me mais o Senhor: Não rogues
por este povo para seu bem. Quando jejuarem, não ouvirei o
seu clamor, e quando oferecerem holocaustos e ofertas de alimentos,
não me agradarei deles; antes eu os consumirei pela espada, e pela
fome e pela peste. Então disse eu: Ah! Senhor Deus, eis que
os profetas lhes dizem: Não vereis espada, e não tereis fome; antes
vos darei paz verdadeira neste lugar. E disse-me o Senhor: Os
profetas profetizam falsamente no meu nome; nunca os enviei, nem
lhes dei ordem, nem lhes falei; visão falsa, e adivinhação, e
vaidade, e o engano do seu coração é o que eles vos profetizam.
Portanto assim diz o Senhor acerca dos profetas que
profetizam no meu nome, sem que eu os tenha mandado, e que dizem:
Nem espada, nem fome haverá nesta terra: À espada e à fome, serão
consumidos esses profetas. E o povo a quem eles profetizam
será lançado nas ruas de Jerusalém, por causa da fome e da espada; e
não haverá quem os sepultem, tanto a eles, como as suas mulheres, e
os seus filhos e as suas filhas; porque derramarei sobre eles a sua
maldade.

Portanto lhes dirás esta palavra: Os meus olhos derramem lágrimas
de noite e de dia, e não cessem; porque a virgem, filha do meu povo,
está gravemente ferida, de chaga mui dolorosa. Se eu saio ao
campo, eis ali os mortos à espada, e, se entro na cidade, estão ali
os debilitados pela fome; e até os profetas e os sacerdotes
percorrem uma terra, que não conhecem. Porventura já de todo
rejeitaste a Judá? Ou repugna a tua alma a Sião? Por que nos feriste
de tal modo que já não há cura para nós? Aguardamos a paz, e não
aparece o bem; e o tempo da cura, e eis aqui turbação\footnote{Ato
ou efeito de turbar(-se); turbamento. Perturbação, desordem,
tumulto.}. Ah! Senhor! conhecemos a nossa impiedade e a
maldade de nossos pais; porque pecamos contra ti. Não nos
rejeites por amor do teu nome; não abatas o trono da tua glória;
lembra-te, e não anules a tua aliança conosco. Porventura há,
entre as vaidades dos gentios, alguém que faça chover? Ou podem os
céus dar chuvas? Não és tu, ó Senhor nosso Deus? Portanto em ti
esperamos, pois tu fazes todas estas coisas.

\medskip

\lettrine{15} Disse-me, porém, o Senhor: Ainda que Moisés e
Samuel se pusessem diante de mim, não estaria a minha alma com este
povo; lança-os de diante da minha face, e saiam. E será que,
quando te disserem: Para onde iremos? Dir-lhes-ás: Assim diz o
Senhor: Os que para a morte, para a morte, e os que para a espada,
para a espada; e os que para a fome, para a fome; e os que para o
cativeiro, para o cativeiro. Porque visitá-los-ei com quatro
gêneros de males, diz o Senhor: com espada para matar, e com cães,
para os arrastarem, e com aves dos céus, e com animais da terra,
para os devorarem e destruírem. Entregá-los-ei ao desterro em
todos os reinos da terra; por causa de Manassés, filho de Ezequias,
rei de Judá, e por tudo quanto fez em Jerusalém. Porque quem se
compadeceria de ti, ó Jerusalém? Ou quem se entristeceria por ti? Ou
quem se desviaaria a perguntar pela tua paz Tu me deixaste, diz
o Senhor, e tornaste-te para trás; por isso estenderei a minha mão
contra ti, e te destruirei; já estou cansado de me arrepender. E
padejá-los-ei com a pá nas portas da terra; já desfilhei, e destruí
o meu povo; não voltaram dos seus caminhos. As suas viúvas mais
se multiplicaram do que a areia dos mares; trouxe ao meio dia um
destruidor sobre a mãe dos jovens; fiz que caísse de repente sobre
ela, e enchesse a cidade de terrores. A que dava à luz sete se
enfraqueceu; expirou a sua alma; pôs-se-lhe o sol sendo ainda de
dia, confundiu-se, e envergonhou-se; e os que ficarem dela
entregarei à espada, diante dos seus inimigos, diz o Senhor.

Ai de mim, minha mãe, por que me deste à luz homem de rixa e
homem de contendas para toda a terra? Nunca lhes emprestei com
usura, nem eles me emprestaram com usura, todavia cada um deles me
amaldiçoa. Disse o Senhor: De certo que o teu remanescente
será para o bem; de certo, no tempo da calamidade, e no tempo da
angústia, farei que o inimigo te dirija súplicas. Pode alguém
quebrar o ferro, o ferro do norte, ou o aço? As tuas riquezas
e os teus tesouros entregarei sem preço ao saque; e isso por todos
os teus pecados, mesmo em todos os teus limites. E te farei
passar aos teus inimigos numa terra que não conheces; porque o fogo
se acendeu em minha ira, e sobre vós arderá.

Tu, ó Senhor, o sabes; lembra-te de mim, e visita-me, e vinga-me
dos meus perseguidores; não me arrebates por tua longanimidade; sabe
que por amor de ti tenho sofrido afronta. Achando-se as tuas
palavras, logo as comi, e a tua palavra foi para mim o gozo e
alegria do meu coração; porque pelo teu nome sou chamado, ó Senhor
Deus dos Exércitos. Nunca me assentei na assembléia dos
zombadores, nem me regozijei; por causa da tua mão me assentei
solitário; pois me encheste de indignação. Por que dura a
minha dor continuamente, e a minha ferida me dói, e já não admite
cura? Serias tu para mim como coisa mentirosa e como águas
inconstantes? Portanto assim diz o Senhor: Se tu voltares,
então te trarei, e estarás diante de mim; e se apartares o precioso
do vil, serás como a minha boca; tornem-se eles para ti, mas não
voltes tu para eles. E eu te porei contra este povo como
forte muro de bronze; e pelejarão contra ti, mas não prevalecerão
contra ti; porque eu sou contigo para te guardar, para te livrar
deles, diz o Senhor. E arrebatar-te-ei da mão dos malignos, e
livrar-te-ei da mão dos fortes.

\medskip

\lettrine{16} E veio a mim a palavra do Senhor, dizendo:
Não tomarás para ti mulher, nem terás filhos nem filhas neste
lugar. Porque assim diz o Senhor, acerca dos filhos e das filhas
que nascerem neste lugar, acerca de suas mães, que os tiverem, e de
seus pais que os gerarem nesta terra: Morrerão de enfermidades
dolorosas, e não serão pranteados nem sepultados; servirão de
esterco sobre a face da terra; e pela espada e pela fome serão
consumidos, e os seus cadáveres servirão de mantimento para as aves
do céu e para os animais da terra. Porque assim diz o Senhor:
Não entres na casa do luto, nem vás a lamentar, nem te compadeças
deles; porque deste povo, diz o Senhor, retirei a minha paz,
benignidade e misericórdia. E morrerão grandes e pequenos nesta
terra, e não serão sepultados, e não os prantearão, nem se farão por
eles incisões, nem por eles se raparão os cabelos. E não se
partirá pão para consolá-los por causa de seus mortos; nem lhes
darão a beber do copo de consolação, pelo pai ou pela mãe de alguém.
Nem entres na casa do banquete, para te assentares com eles a
comer e a beber. Porque assim diz o Senhor dos Exércitos, o Deus
de Israel: Eis que farei cessar, neste lugar, perante os vossos
olhos, e em vossos dias, a voz de gozo e a voz de alegria, a voz do
esposo e a voz da esposa.

E será que, quando anunciares a este povo todas estas palavras, e
eles te disserem: Por que pronuncia o Senhor sobre nós todo este
grande mal? E qual é a nossa iniqüidade, e qual é o nosso pecado,
que cometemos contra o Senhor nosso Deus? Então lhes dirás:
Porquanto vossos pais me deixaram, diz o Senhor, e se foram após
outros deuses, e os serviram, e se inclinaram diante deles, e a mim
me deixaram, e a minha lei não a guardaram. E vós fizestes
pior do que vossos pais; porque, eis que cada um de vós anda segundo
o propósito do seu mau coração, para não me dar ouvidos a mim.
Portanto lançar-vos-ei fora desta terra, para uma terra que
não conhecestes, nem vós nem vossos pais; e ali servireis a deuses
alheios de dia e de noite, porque não usarei de misericórdia
convosco.

Portanto, eis que dias vêm, diz o Senhor, em que nunca mais se
dirá: Vive o Senhor, que fez subir os filhos de Israel da terra do
Egito. Mas: Vive o Senhor, que fez subir os filhos de Israel
da terra do norte, e de todas as terras para onde os tinha lançado;
porque eu os farei voltar à sua terra, a qual dei a seus pais.
Eis que mandarei muitos pescadores, diz o Senhor, os quais os
pescarão; e depois enviarei muitos caçadores, os quais os caçarão de
sobre todo o monte, e de sobre todo o outeiro, e até das fendas das
rochas. Porque os meus olhos estão sobre todos os seus
caminhos; não se escondem da minha face, nem a sua maldade se
encobre aos meus olhos. E primeiramente pagarei em dobro a
sua maldade e o seu pecado, porque profanaram a minha terra com os
cadáveres das suas coisas detestáveis, e das suas abominações
encheram a minha herança. Ó Senhor, fortaleza minha, e força
minha, e refúgio meu no dia da angústia; a ti virão os gentios desde
os fins da terra, e dirão: Nossos pais herdaram só mentiras, e
vaidade, em que não havia proveito. Porventura fará um homem
deuses para si, que contudo não são deuses? Portanto, eis que
lhes farei conhecer, desta vez lhes farei conhecer a minha mão e o
meu poder; e saberão que o meu nome é o Senhor.

\medskip

\lettrine{17} O pecado de Judá está escrito com um ponteiro de
ferro, com ponta de diamante, gravado na tábua do seu coração e nas
pontas dos vossos altares; como também seus filhos se lembram
dos seus altares, e dos seus bosques, junto às árvores frondosas,
sobre os altos outeiros. Ó meu monte no campo! a tua riqueza e
todos os teus tesouros darei por presa, como também os teus altos,
por causa do pecado, em todos os teus termos. Assim por ti mesmo
te privarás da tua herança que te dei, e far-te-ei servir os teus
inimigos, na terra que não conheces; porque o fogo que acendeste na
minha ira arderá para sempre.

Assim diz o Senhor: Maldito o homem que confia no homem, e faz da
carne o seu braço, e aparta o seu coração do Senhor! Porque será
como a tamargueira\footnote{Arbusto ou árvore pequena (Tamarix
africana), da fam. das tamaricáceas, de folhas ovadas, acuminadas e
decíduas, flores pequenas, em racemos curtos, e cápsulas com
diminutas e numerosas sementes; tamarga, tamarisco, tamariz,
tramaga, tramagueira (Nativo da região mediterrânea, a casca encerra
tanino e tem uso como adstringente).} no deserto, e não verá quando
vem o bem; antes morará nos lugares secos do deserto, na terra
salgada e inabitável. Bendito o homem que confia no Senhor, e
cuja confiança é o Senhor. Porque será como a árvore plantada
junto às águas, que estende as suas raízes para o ribeiro, e não
receia quando vem o calor, mas a sua folha fica verde; e no ano de
sequidão não se afadiga, nem deixa de dar fruto. Enganoso é o
coração, mais do que todas as coisas, e perverso; quem o conhecerá?
Eu, o Senhor, esquadrinho o coração e provo os rins; e isto
para dar a cada um segundo os seus caminhos e segundo o fruto das
suas ações. Como a perdiz, que choca ovos que não pôs, assim
é aquele que ajunta riquezas, mas não retamente; no meio de seus
dias as deixará, e no seu fim será um insensato.

Um trono de glória, posto bem alto desde o princípio, é o lugar
do nosso santuário. Ó Senhor, esperança de Israel, todos
aqueles que te deixam serão envergonhados; os que se apartam de mim
serão escritos sobre a terra; porque abandonam o Senhor, a fonte das
águas vivas. Cura-me, Senhor, e sararei; salva-me, e serei
salvo; porque tu és o meu louvor. Eis que eles me dizem: Onde
está a palavra do Senhor? Venha agora. Porém eu não me
apressei em ser o pastor seguindo-te; nem tampouco desejei o dia da
aflição, tu o sabes; o que saiu dos meus lábios está diante de tua
face. Não me sejas por espanto; meu refúgio és tu no dia do
mal. Envergonhem-se os que me perseguem, e não me envergonhe
eu; assombrem-se eles, e não me assombre eu; traze sobre eles o dia
do mal, e destrói-os com dobrada destruição.

Assim me disse o Senhor: Vai, e põe-te à porta dos filhos do
povo, pela qual entram os reis de Judá, e pela qual saem; como
também em todas as portas de Jerusalém. E dize-lhes: Ouvi a
palavra do Senhor, vós, reis de Judá e todo o Judá, e todos os
moradores de Jerusalém que entrais por estas portas. Assim
diz o Senhor: Guardai as vossas almas, e não tragais cargas no dia
de sábado, nem as introduzais pelas portas de Jerusalém; nem
tireis cargas de vossas casas no dia de sábado, nem façais obra
alguma; antes santificai o dia de sábado, como eu ordenei a vossos
pais. Mas não escutaram, nem inclinaram os seus ouvidos;
antes endureceram a sua cerviz, para não ouvirem, e para não
receberem correção. Mas se vós diligentemente me ouvirdes,
diz o Senhor, não introduzindo cargas pelas portas desta cidade no
dia de sábado, e santificardes o dia de sábado, não fazendo nele
obra alguma, então entrarão pelas portas desta cidade reis e
príncipes, que se assentem sobre o trono de Davi, andando em carros
e em cavalos; e eles e seus príncipes, os homens de Judá, e os
moradores de Jerusalém; e esta cidade será habitada para sempre.
E virão das cidades de Judá, e dos arredores de Jerusalém, e
da terra de Benjamim, e das planícies, e das montanhas, e do sul,
trazendo holocaustos, e sacrifícios, e ofertas de alimentos, e
incenso, trazendo também sacrifícios de louvores à casa do Senhor.
Mas, se não me ouvirdes, para santificardes o dia de sábado,
e para não trazerdes carga alguma, quando entrardes pelas portas de
Jerusalém no dia de sábado, então acenderei fogo nas suas portas, o
qual consumirá os palácios de Jerusalém, e não se apagará.

\medskip

\lettrine{18} A palavra do Senhor, que veio a Jeremias,
dizendo: Levanta-te, e desce à casa do oleiro, e lá te farei
ouvir as minhas palavras. E desci à casa do oleiro, e eis que
ele estava fazendo a sua obra sobre as rodas. Como o vaso, que
ele fazia de barro, quebrou-se na mão do oleiro, tornou a fazer dele
outro vaso, conforme o que pareceu bem aos olhos do oleiro fazer.
Então veio a mim a palavra do Senhor, dizendo: Não poderei
eu fazer de vós como fez este oleiro, ó casa de Israel? diz o
Senhor. Eis que, como o barro na mão do oleiro, assim sois vós na
minha mão, ó casa de Israel. No momento em que falar contra uma
nação, e contra um reino para arrancar, e para derrubar, e para
destruir, se a tal nação, porém, contra a qual falar se
converter da sua maldade, também eu me arrependerei do mal que
pensava fazer-lhe. No momento em que falar de uma nação e de um
reino, para edificar e para plantar, se fizer o mal diante
dos meus olhos, não dando ouvidos à minha voz, então me arrependerei
do bem que tinha falado que lhe faria.

Ora, pois, fala agora aos homens de Judá, e aos moradores de
Jerusalém, dizendo: Assim diz o Senhor: Eis que estou forjando mal
contra vós; e projeto um plano contra vós; convertei-vos, pois,
agora cada um do seu mau caminho, e melhorai os vossos caminhos e as
vossas ações. Mas eles dizem: Não há esperança, porque
andaremos segundo as nossas imaginações; e cada um fará segundo o
propósito do seu mau coração. Portanto, assim diz o Senhor:
Perguntai agora entre os gentios quem ouviu tal coisa? Coisa mui
horrenda fez a virgem de Israel. Porventura a neve do Líbano
deixará a rocha do campo ou esgotar-se-ão as águas frias que correm
de terras estranhas? Contudo o meu povo se tem esquecido de
mim, queimando incenso à vaidade, que os fez tropeçar nos seus
caminhos, e nas veredas antigas, para que andassem por veredas
afastadas, não aplainadas; para fazerem da sua terra objeto
de espanto e de perpétuos assobios; todo aquele que passar por ela
se espantará, e meneará a sua cabeça; com vento oriental os
espalharei diante do inimigo; mostrar-lhes-ei as costas e não o
rosto, no dia da sua perdição.

Então disseram: Vinde, e maquinemos projetos contra Jeremias;
porque não perecerá a lei do sacerdote, nem o conselho do sábio, nem
a palavra do profeta; vinde e firamo-lo com a língua, e não
atendamos a nenhuma das suas palavras. Olha para mim, Senhor,
e ouve a voz dos que contendem comigo. Porventura pagar-se-á
mal por bem? Pois cavaram uma cova para a minha alma. Lembra-te de
que eu compareci à tua presença, para falar a favor deles, e para
desviar deles a tua indignação; portanto entrega seus filhos
à fome, e entrega-os ao poder da espada, e sejam suas mulheres
roubadas dos filhos, e fiquem viúvas; e seus maridos sejam feridos
de morte, e os seus jovens sejam feridos à espada na peleja.
Ouça-se o clamor de suas casas, quando de repente trouxeres
uma tropa sobre eles. Porquanto cavaram uma cova para prender-me e
armaram laços aos meus pés. Mas tu, ó Senhor, sabes todo o
seu conselho contra mim para matar-me; não perdoes a sua maldade,
nem apagues o seu pecado de diante da tua face; mas tropecem diante
de ti; trata-os assim no tempo da tua ira.

\medskip

\lettrine{19} Assim disse o Senhor: Vai, e compra uma
botija\footnote{Vasilhame de barro ou de grés em forma de garrafa
cilíndrica ou bojuda, de gargalo fino e curto, ger. provido de asa.}
de oleiro, e leva contigo alguns dos anciãos do povo e alguns dos
anciãos dos sacerdotes; e sai ao Vale do Filho de Hinom, que
está à entrada da porta do sol, e apregoa ali as palavras que eu te
disser; e dirás: Ouvi a palavra do Senhor, ó reis de Judá, e
moradores de Jerusalém. Assim diz o Senhor dos Exércitos, o Deus de
Israel: Eis que trarei um mal sobre este lugar, e quem quer que dele
ouvir retinir-lhe-ão os ouvidos. Porquanto me deixaram e
alienaram este lugar, e nele queimaram incenso a outros deuses, que
nunca conheceram, nem eles nem seus pais, nem os reis de Judá; e
encheram este lugar de sangue de inocentes. Porque edificaram os
altos de Baal, para queimarem seus filhos no fogo em holocaustos a
Baal; o que nunca lhes ordenei, nem falei, nem me veio ao
pensamento. Por isso eis que dias vêm, diz o Senhor, em que este
lugar não se chamará mais Tofete, nem o Vale do Filho de Hinom, mas
o Vale da Matança. Porque dissiparei o conselho de Judá e de
Jerusalém neste lugar, e os farei cair à espada diante de seus
inimigos, e pela mão dos que buscam a vida deles; e darei os seus
cadáveres para pasto às aves dos céus e aos animais da terra. E
farei esta cidade objeto de espanto e de assobio; todo aquele que
passar por ela se espantará, e assobiará por causa de todas as suas
pragas. E lhes farei comer a carne de seus filhos e a carne de
suas filhas, e comerá cada um a carne do seu amigo, no cerco e no
aperto em que os apertarão os seus inimigos, e os que buscam a vida
deles.

Então quebrarás a botija à vista dos homens que forem contigo.
E dir-lhes-ás: Assim diz o Senhor dos Exércitos: Deste modo
quebrarei eu a este povo, e a esta cidade, como se quebra o vaso do
oleiro, que não pode mais refazer-se, e os enterrarão em Tofete,
porque não haverá mais lugar para os enterrar. Assim farei a
este lugar, diz o Senhor, e aos seus moradores; sim, para pôr a esta
cidade como a Tofete. E as casas de Jerusalém, e as casas dos
reis de Judá, serão imundas como o lugar de Tofete, como também
todas as casas, sobre cujos terraços queimaram incenso a todo o
exército dos céus, e ofereceram libações a deuses estranhos.
Vindo, pois, Jeremias de Tofete onde o tinha enviado o Senhor
a profetizar, se pôs em pé no átrio da casa do Senhor, e disse a
todo o povo: Assim diz o Senhor dos Exércitos, o Deus de
Israel: Eis que trarei sobre esta cidade, e sobre todas as suas
vilas, todo o mal que pronunciei contra ela, porquanto endureceram a
sua cerviz, para não ouvirem as minhas palavras.

\medskip

\lettrine{20} E Pasur, filho de Imer, o sacerdote, que havia
sido nomeado presidente na casa do Senhor, ouviu a Jeremias, que
profetizava estas palavras. E feriu Pasur ao profeta Jeremias, e
o colocou no cepo que está na porta superior de Benjamim, na casa do
Senhor. E sucedeu que no dia seguinte Pasur tirou a Jeremias do
cepo. Então disse-lhe Jeremias: O Senhor não chama o teu nome Pasur,
mas, Terror por todos os lados. Porque assim diz o Senhor: Eis
que farei de ti um terror para ti mesmo, e para todos os teus
amigos. Eles cairão à espada de seus inimigos, e teus olhos o verão.
Entregarei todo o Judá na mão do rei de Babilônia; ele os levará
presos a Babilônia, e feri-los-á à espada. Também entregarei
toda a riqueza desta cidade, e todo o seu trabalho, e todas as suas
coisas preciosas, sim, todos os tesouros dos reis de Judá entregarei
na mão de seus inimigos, e saqueá-los-ão, e tomá-los-ão e
levá-los-ão a Babilônia. E tu, Pasur, e todos os moradores da
tua casa ireis para o cativeiro; e virás a Babilônia, e ali
morrerás, e ali serás sepultado, tu, e todos os teus amigos, aos
quais profetizaste falsamente.

Persuadiste-me, ó Senhor, e persuadido fiquei; mais forte foste do
que eu, e prevaleceste; sirvo de escárnio todo o dia; cada um deles
zomba de mim. Porque desde que falo, grito, clamo: Violência e
destruição; porque se tornou a palavra do Senhor um opróbrio e
ludíbrio\footnote{Ato ou efeito de ludibriar, enganar. Ação de jogar
com a credulidade de (alguém ou algo); brincadeira maldosa; logro,
zombaria.} todo o dia. Então disse eu: Não me lembrarei dele, e
não falarei mais no seu nome; mas isso foi no meu coração como fogo
ardente, encerrado nos meus ossos; e estou fatigado de sofrer, e não
posso mais. Porque ouvi a murmuração de muitos, terror de
todos os lados: Denunciai, e o denunciaremos; todos os que têm paz
comigo aguardam o meu manquejar, dizendo: Bem pode ser que se deixe
persuadir; então prevaleceremos contra ele e nos vingaremos dele.
Mas o Senhor está comigo como um valente terrível; por isso
tropeçarão os meus perseguidores, e não prevalecerão; ficarão muito
confundidos; porque não se houveram prudentemente, terão uma
confusão perpétua que nunca será esquecida. Tu, pois, ó
Senhor dos Exércitos, que provas o justo, e vês os rins e o coração,
permite que eu veja a tua vingança contra eles; pois já te revelei a
minha causa. Cantai ao Senhor, louvai ao Senhor; pois livrou
a alma do necessitado da mão dos malfeitores.

Maldito o dia em que nasci; não seja bendito o dia em que minha
mãe me deu à luz. Maldito o homem que deu as novas a meu pai,
dizendo: Nasceu-te um filho; alegrando-o com isso grandemente.
E seja esse homem como as cidades que o Senhor destruiu e não
se arrependeu; e ouça clamor pela manhã, e ao tempo do meio-dia um
alarido. Por que não me matou na madre? Assim minha mãe teria
sido a minha sepultura, e teria ficado grávida perpetuamente!
Por que saí da madre, para ver trabalho e tristeza, e para
que os meus dias se consumam na vergonha?

\medskip

\lettrine{21} A palavra que veio a Jeremias da parte do
Senhor, quando o rei Zedequias lhe enviou a Pasur, filho de
Malquias, e a Sofonias, filho de Maaséia, o sacerdote, dizendo:
Pergunta agora por nós ao Senhor, por que Nabucodonosor, rei de
Babilônia, guerreia contra nós; bem pode ser que o Senhor trate
conosco segundo todas as suas maravilhas, e o faça retirar-se de
nós. Então Jeremias lhes disse: Assim direis a Zedequias:
Assim diz o Senhor Deus de Israel: Eis que virarei contra vós as
armas de guerra, que estão nas vossas mãos, com que vós pelejais
contra o rei de Babilônia, e contra os caldeus, que vos têm cercado
de fora dos muros, e ajuntá-los-ei no meio desta cidade. E eu
pelejarei contra vós com mão estendida e com braço forte, e com ira,
e com indignação e com grande furor. E ferirei os habitantes
desta cidade, assim os homens como os animais; de grande pestilência
morrerão. E depois disto, diz o Senhor, entregarei Zedequias,
rei de Judá, e seus servos, e o povo, e os que desta cidade restarem
da pestilência, e da espada, e da fome, na mão de Nabucodonosor, rei
de Babilônia, e na mão de seus inimigos, e na mão dos que buscam a
sua vida; e feri-los-á ao fio da espada; não os poupará, nem se
compadecerá, nem terá misericórdia.

E a este povo dirás: Assim diz o Senhor: Eis que ponho diante de
vós o caminho da vida e o caminho da morte. O que ficar nesta
cidade há de morrer à espada, ou de fome, ou de pestilência; mas o
que sair, e se render aos caldeus, que vos têm cercado, viverá, e
terá a sua vida por despojo. Porque pus o meu rosto contra
esta cidade para mal, e não para bem, diz o Senhor; na mão do rei de
Babilônia se entregará, e ele queimá-la-á a fogo. E à casa do
rei de Judá dirás: Ouvi a palavra do Senhor: Ó casa de Davi,
assim diz o Senhor: Julgai pela manhã justamente, e livrai o
espoliado da mão do opressor; para que não saia o meu furor como
fogo, e se acenda, sem que haja quem o apague, por causa da maldade
de vossas ações. Eis que eu sou contra ti, ó moradora do
vale, ó rocha da campina, diz o Senhor; contra vós que dizeis: Quem
descerá contra nós? Ou quem entrará nas nossas moradas? Eu
vos castigarei segundo o fruto das vossas ações, diz o Senhor; e
acenderei o fogo no seu bosque, que consumirá a tudo o que está em
redor dela.

\medskip

\lettrine{22} Assim diz o Senhor: Desce à casa do rei de Judá,
e anuncia ali esta palavra, e dize: Ouve a palavra do Senhor, ó
rei de Judá, que te assentas no trono de Davi, tu, e os teus servos,
o teu povo, que entrais por estas portas. Assim diz o Senhor:
Exercei o juízo e a justiça, e livrai o espoliado da mão do
opressor; e não oprimais ao estrangeiro, nem ao órfão, nem à viúva;
não façais violência, nem derrameis sangue inocente neste lugar.
Porque, se deveras cumprirdes esta palavra, entrarão pelas
portas desta casa os reis que se assentarão em lugar de Davi sobre o
seu trono, andando em carros e montados em cavalos, eles, e os seus
servos, e o seu povo. Mas, se não derdes ouvidos a estas
palavras, por mim mesmo tenho jurado, diz o Senhor, que esta casa se
tornará em assolação. Porque assim diz o Senhor acerca da casa
do rei de Judá: Tu és para mim Gileade, e a cabeça do Líbano; mas
por certo que farei de ti um deserto e cidades desabitadas.
Porque preparei contra ti destruidores, cada um com as suas
armas; e cortarão os teus cedros escolhidos, e lançá-los-ão no fogo.
E muitas nações passarão por esta cidade, e dirá cada um ao seu
próximo: Por que procedeu o Senhor assim com esta grande cidade?
E dirão: Porque deixaram a aliança do Senhor seu Deus, e se
inclinaram diante de outros deuses, e os serviram.

Não choreis o morto, nem o lastimeis; chorai abundantemente
aquele que sai, porque nunca mais tornará nem verá a terra onde
nasceu. Porque assim diz o Senhor acerca de Salum, filho de
Josias, rei de Judá, que reinou em lugar de Josias, seu pai, e que
saiu deste lugar: Nunca mais ali tornará. Mas no lugar para
onde o levaram cativo ali morrerá, e nunca mais verá esta terra.
Ai daquele que edifica a sua casa com injustiça, e os seus
aposentos sem direito, que se serve do serviço do seu próximo sem
remunerá-lo, e não lhe dá o salário do seu trabalho. Que diz:
Edificarei para mim uma casa espaçosa, e aposentos largos; e que lhe
abre janelas, forrando-a de cedro, e pintando-a de vermelhão.
Porventura reinarás tu, porque te encerras em cedro? Acaso
teu pai não comeu e bebeu, e não praticou o juízo e a justiça? Por
isso lhe sucedeu bem. Julgou a causa do aflito e necessitado;
então lhe sucedeu bem; porventura não é isto conhecer-me? diz o
Senhor. Mas os teus olhos e o teu coração não atentam senão
para a tua avareza, e para derramar sangue inocente, e para praticar
a opressão, e a violência. Portanto assim diz o Senhor acerca
de Jeoiaquim, filho de Josias, rei de Judá: Não o lamentarão,
dizendo: Ai, meu irmão, ou ai, minha irmã! Nem o lamentarão,
dizendo: Ai, senhor, ou, ai, sua glória! Em sepultura de
jumento será sepultado, sendo arrastado e lançado para bem longe,
fora das portas de Jerusalém.

Sobe ao Líbano, e clama, e levanta a tua voz em Basã, e clama
desde Abarim; porque estão destruídos todos os teus namorados.
Falei contigo na tua prosperidade, mas tu disseste: Não
ouvirei. Este tem sido o teu caminho, desde a tua mocidade, pois
nunca deste ouvidos à minha voz. O vento apascentará a todos
os teus pastores, e os teus namorados irão para o cativeiro;
certamente então te confundirás, e te envergonharás por causa de
toda a tua maldade. Ó tu, que habitas no Líbano e fazes o teu
ninho nos cedros, quão lastimada serás quando te vierem as dores e
os ais como da que está de parto. Vivo eu, diz o Senhor, que
ainda que Conias, filho de Jeoiaquim, rei de Judá, fosse o anel do
selo na minha mão direita, contudo dali te arrancaria. E
entregar-te-ei na mão dos que buscam a tua vida, e na mão daqueles
diante de quem tu temes, a saber, na mão de Nabucodonosor, rei de
Babilônia, e na mão dos caldeus. E lançar-te-ei, a ti e à tua
mãe que te deu à luz, para uma terra estranha, em que não nasceste,
e ali morrereis. Mas à terra, para a qual eles com toda a
alma desejam voltar, para lá não voltarão. É, pois, este
homem Conias um ídolo desprezado e quebrado, ou um vaso de que
ninguém se agrada? Por que razão foram arremessados fora, ele e a
sua geração, e arrojados para uma terra que não conhecem? Ó
terra, terra, terra! Ouve a palavra do Senhor. Assim diz o
Senhor: Escrevei que este homem está privado de filhos, homem que
não prosperará nos seus dias; porque nenhum da sua geração
prosperará, para se assentar no trono de Davi, e reinar ainda em
Judá.

\medskip

\lettrine{23} Ai dos pastores que destroem e dispersam as
ovelhas do meu pasto, diz o Senhor. Portanto assim diz o Senhor
Deus de Israel, contra os pastores que apascentam o meu povo: Vós
dispersastes as minhas ovelhas, e as afugentastes, e não as
visitastes; eis que visitarei sobre vós a maldade das vossas ações,
diz o Senhor. E eu mesmo recolherei o restante das minhas
ovelhas, de todas as terras para onde as tiver afugentado, e as
farei voltar aos seus apriscos; e frutificarão, e se multiplicarão.
E levantarei sobre elas pastores que as apascentem, e nunca mais
temerão, nem se assombrarão, e nem uma delas faltará, diz o Senhor.
Eis que vêm dias, diz o Senhor, em que levantarei a Davi um
Renovo justo; e, sendo rei, reinará e agirá sabiamente, e praticará
o juízo e a justiça na terra. Nos seus dias Judá será salvo, e
Israel habitará seguro; e este será o seu nome, com o qual Deus o
chamará: O Senhor JUSTIÇA NOSSA. Portanto, eis que vêm dias, diz
o Senhor, em que nunca mais dirão: Vive o Senhor, que fez subir os
filhos de Israel da terra do Egito; mas: Vive o Senhor, que fez
subir, e que trouxe a geração da casa de Israel da terra do norte, e
de todas as terras para onde os tinha arrojado; e habitarão na sua
terra.

Quanto aos profetas, já o meu coração está quebrantado dentro de
mim; todos os meus ossos estremecem; sou como um homem embriagado, e
como um homem vencido de vinho, por causa do Senhor, e por causa das
suas santas palavras. Porque a terra está cheia de adúlteros,
e a terra chora por causa da maldição; os pastos do deserto se
secam; porque a sua carreira é má, e a sua força não é reta.
Porque tanto o profeta, como o sacerdote, estão contaminados;
até na minha casa achei a sua maldade, diz o Senhor. Portanto
o seu caminho lhes será como lugares escorregadios na escuridão;
serão empurrados, e cairão nele; porque trarei sobre eles mal, no
ano da sua visitação, diz o Senhor. Nos profetas de Samaria
bem vi loucura; profetizavam da parte de Baal, e faziam errar o meu
povo Israel. Mas nos profetas de Jerusalém vejo uma coisa
horrenda: cometem adultérios, e andam com falsidade, e fortalecem as
mãos dos malfeitores, para que não se convertam da sua maldade; eles
têm-se tornado para mim como Sodoma, e os seus moradores como
Gomorra. Portanto assim diz o Senhor dos Exércitos acerca dos
profetas: Eis que lhes darei a comer losna, e lhes farei beber águas
de fel; porque dos profetas de Jerusalém saiu a contaminação sobre
toda a terra. Assim diz o Senhor dos Exércitos: Não deis
ouvidos às palavras dos profetas, que entre vós profetizam;
fazem-vos desvanecer; falam da visão do seu coração, não da boca do
Senhor. Dizem continuamente aos que me desprezam: O Senhor
disse: Paz tereis; e a qualquer que anda segundo a dureza do seu
coração, dizem: Não virá mal sobre vós. Porque, quem esteve
no conselho do Senhor, e viu, e ouviu a sua palavra? Quem esteve
atento à sua palavra, e ouviu? Eis que saiu com indignação a
tempestade do Senhor; e uma tempestade penosa cairá cruelmente sobre
a cabeça dos ímpios. Não se desviará a ira do Senhor, até que
execute e cumpra os desígnios do seu coração; nos últimos dias
entendereis isso claramente. Não mandei esses profetas,
contudo eles foram correndo; não lhes falei, contudo eles
profetizaram. Mas, se estivessem estado no meu conselho,
então teriam feito o meu povo ouvir as minhas palavras, e o teriam
feito voltar do seu mau caminho, e da maldade das suas ações.
Porventura sou eu Deus de perto, diz o Senhor, e não também
Deus de longe? Esconder-se-ia alguém em esconderijos, de modo
que eu não o veja? diz o Senhor. Porventura não encho eu os céus e a
terra? diz o Senhor. Tenho ouvido o que dizem aqueles
profetas, profetizando mentiras em meu nome, dizendo: Sonhei,
sonhei. Até quando sucederá isso no coração dos profetas que
profetizam mentiras, e que só profetizam do engano do seu coração?
Os quais cuidam fazer com que o meu povo se esqueça do meu
nome pelos seus sonhos que cada um conta ao seu próximo, assim como
seus pais se esqueceram do meu nome por causa de Baal. O
profeta que tem um sonho conte o sonho; e aquele que tem a minha
palavra, fale a minha palavra com verdade. Que tem a palha com o
trigo? diz o Senhor. Porventura a minha palavra não é como o
fogo, diz o Senhor, e como um martelo que esmiúça a pedra?
Portanto, eis que eu sou contra os profetas, diz o Senhor,
que furtam as minhas palavras, cada um ao seu próximo. Eis
que eu sou contra os profetas, diz o Senhor, que usam de sua própria
linguagem, e dizem: Ele disse. Eis que eu sou contra os que
profetizam sonhos mentirosos, diz o Senhor, e os contam, e fazem
errar o meu povo com as suas mentiras e com as suas leviandades;
pois eu não os enviei, nem lhes dei ordem; e não trouxeram proveito
algum a este povo, diz o Senhor.

Quando, pois, te perguntar este povo, ou qualquer profeta, ou
sacerdote, dizendo: Qual é o peso do Senhor? Então lhe dirás: Este é
o peso: Que vos deixarei, diz o Senhor. E, quanto ao profeta,
e ao sacerdote, e ao povo, que disser: Peso do Senhor, eu castigarei
o tal homem e a sua casa. Assim direis, cada um ao seu
próximo, e cada um ao seu irmão: Que respondeu o Senhor? e que falou
o Senhor? Mas nunca mais vos lembrareis do peso do Senhor;
porque a cada um lhe servirá de peso a sua própria palavra; pois
torceis as palavras do Deus vivo, do Senhor dos Exércitos, o nosso
Deus. Assim dirás ao profeta: Que te respondeu o Senhor, e
que falou o Senhor? Mas, porque dizeis: Peso do Senhor; assim
o diz o Senhor: Porque dizeis esta palavra: Peso do Senhor,
havendo-vos ordenado, dizendo: Não direis: Peso do Senhor;
por isso, eis que também eu me esquecerei totalmente de vós,
e tirarei da minha presença, a vós e a cidade que vos dei a vós e a
vossos pais; e porei sobre vós perpétuo opróbrio, e eterna
vergonha, que não será esquecida.

\medskip

\lettrine{24} Fez-me o Senhor ver, e eis dois cestos de figos,
postos diante do templo do Senhor, depois que Nabucodonosor, rei de
Babilônia, levou em cativeiro a Jeconias, filho de Jeoiaquim, rei de
Judá, e os príncipes de Judá, e os carpinteiros, e os ferreiros de
Jerusalém, e os trouxe a Babilônia. Um cesto tinha figos muito
bons, como os figos temporãos; mas o outro cesto tinha figos muito
ruins, que não se podiam comer, de ruins que eram. E disse-me o
Senhor: Que vês tu, Jeremias? E eu disse: Figos: os figos bons,
muito bons e os ruins, muito ruins, que não se podem comer, de ruins
que são. Então veio a mim a palavra do Senhor, dizendo:
Assim diz o Senhor, o Deus de Israel: Como a estes bons figos,
assim também conhecerei aos de Judá, levados em cativeiro; os quais
enviei deste lugar para a terra dos caldeus, para o seu bem.
Porei os meus olhos sobre eles, para o seu bem, e os farei
voltar a esta terra, e edificá-los-ei, e não os destruirei; e
plantá-los-ei, e não os arrancarei. E dar-lhes-ei coração para
que me conheçam, porque eu sou o Senhor; e ser-me-ão por povo, e eu
lhes serei por Deus; porque se converterão a mim de todo o seu
coração. E como os figos ruins, que se não podem comer, de ruins
que são (porque assim diz o Senhor), assim entregarei Zedequias, rei
de Judá, e os seus príncipes, e o restante de Jerusalém, que ficou
nesta terra, e os que habitam na terra do Egito. E
entregá-los-ei para que sejam um prejuízo, uma ofensa para todos os
reinos da terra, um opróbrio e um provérbio, e um escárnio, e uma
maldição em todos os lugares para onde eu os arrojar. E
enviarei entre eles a espada, a fome, e a peste, até que se consumam
de sobre a terra que lhes dei a eles e a seus pais.

\medskip

\lettrine{25} A palavra que veio a Jeremias acerca de todo o
povo de Judá no quarto ano de Jeoiaquim, filho de Josias, rei de
Judá (que é o primeiro ano de Nabucodonosor, rei de Babilônia),
a qual anunciou o profeta Jeremias a todo o povo de Judá, e a
todos os habitantes de Jerusalém, dizendo: Desde o ano treze de
Josias, filho de Amom, rei de Judá, até o dia de hoje, período de
vinte e três anos, tem vindo a mim a palavra do Senhor, e vo-la
tenho anunciado, madrugando e falando; mas vós não escutastes.
Também vos enviou o Senhor todos os seus servos, os profetas,
madrugando e enviando-os, mas vós não escutastes, nem inclinastes os
vossos ouvidos para ouvir, quando diziam: Convertei-vos agora
cada um do seu mau caminho, e da maldade das suas ações, e habitai
na terra que o Senhor vos deu, e a vossos pais, para sempre. E
não andeis após outros deuses para os servirdes, e para vos
inclinardes diante deles, nem me provoqueis à ira com a obra de
vossas mãos, para que não vos faça mal. Porém não me destes
ouvidos, diz o Senhor, mas me provocastes à ira com a obra de vossas
mãos, para vosso mal.

Portanto assim diz o Senhor dos Exércitos: Visto que não
escutastes as minhas palavras, eis que eu enviarei, e tomarei a
todas as famílias do norte, diz o Senhor, como também a
Nabucodonosor, rei de Babilônia, meu servo, e os trarei sobre esta
terra, e sobre os seus moradores, e sobre todas estas nações em
redor, e os destruirei totalmente, e farei que sejam objeto de
espanto, e de assobio, e de perpétuas desolações. E farei
desaparecer dentre eles a voz de gozo, e a voz de alegria, a voz do
esposo, e a voz da esposa, como também o som das mós, e a luz do
candeeiro. E toda esta terra virá a ser um deserto e um
espanto; e estas nações servirão ao rei de Babilônia setenta anos.
Acontecerá, porém, que, quando se cumprirem os setenta anos,
visitarei o rei de Babilônia, e esta nação, diz o Senhor, castigando
a sua iniqüidade, e a da terra dos caldeus; farei deles ruínas
perpétuas. E trarei sobre aquela terra todas as minhas
palavras, que disse contra ela, a saber, tudo quanto está escrito
neste livro, que profetizou Jeremias contra todas estas nações.
Porque também deles se servirão muitas nações e grandes reis;
assim lhes retribuirei segundo os seus feitos, e segundo as obras
das suas mãos.

Porque assim me disse o Senhor Deus de Israel: Toma da minha mão
este copo do vinho do furor, e darás a beber dele a todas as nações,
às quais eu te enviarei. Para que bebam e tremam, e
enlouqueçam, por causa da espada, que eu enviarei entre eles.
E tomei o copo da mão do Senhor, e dei a beber a todas as
nações, às quais o Senhor me enviou; a Jerusalém, e às
cidades de Judá, e aos seus reis, e aos seus príncipes, para fazer
deles uma desolação, um espanto, um assobio, e uma maldição, como
hoje se vê; a Faraó, rei do Egito, e a seus servos, e a seus
príncipes, e a todo o seu povo; e a toda a mistura de povo, e
a todos os reis da terra de Uz, e a todos os reis da terra dos
filisteus, e a Ascalom, e a Gaza, e a Ecrom, e ao remanescente de
Asdode, e a Edom, e a Moabe, e aos filhos de Amom; e a
todos os reis de Tiro, e a todos os reis de Sidom; e aos reis das
ilhas que estão além do mar; a Dedã, e a Tema, e a Buz e a
todos os que estão nos lugares mais distantes. E a todos os
reis da Arábia, e todos os reis do povo misto que habita no deserto;
e a todos os reis de Zinri, e a todos os reis de Elão, e a
todos os reis da Média; e a todos os reis do norte, os de
perto, e os de longe, tanto um como o outro, e a todos os reinos do
mundo, que estão sobre a face da terra, e o rei de Sesaque beberá
depois deles. Pois lhes dirás: Assim diz o Senhor dos
Exércitos, o Deus de Israel: Bebei, e embebedai-vos, e vomitai, e
caí, e não torneis a levantar-vos, por causa da espada que eu vos
enviarei. E será que, se não quiserem tomar o copo da tua mão
para beber, então lhes dirás: Assim diz o Senhor dos Exércitos:
Certamente bebereis. Porque, eis que na cidade que se chama
pelo meu nome começo a castigar; e ficareis vós totalmente impunes?
Não ficareis impunes, porque eu chamo a espada sobre todos os
moradores da terra, diz o Senhor dos Exércitos.

Tu, pois, lhes profetizarás todas estas palavras, e lhes dirás: O
Senhor desde o alto bramirá, e fará ouvir a sua voz desde a morada
da sua santidade; terrivelmente bramirá contra a sua habitação, com
grito de alegria, como dos que pisam as uvas, contra todos os
moradores da terra. Chegará o estrondo até à extremidade da
terra, porque o Senhor tem contenda com as nações, entrará em juízo
com toda a carne; os ímpios entregará à espada, diz o Senhor.
Assim diz o Senhor dos Exércitos: Eis que o mal passa de
nação para nação, e grande tormenta se levantará dos confins da
terra. E serão os mortos do Senhor, naquele dia, desde uma
extremidade da terra até à outra; não serão pranteados, nem
recolhidos, nem sepultados; mas serão por esterco sobre a face da
terra. Uivai, pastores, e clamai, e revolvei-vos na cinza,
principais do rebanho, porque já se cumpriram os vossos dias para
serdes mortos, e dispersos, e vós então caireis como um vaso
precioso. E não haverá refúgio para os pastores, nem
salvamento para os principais do rebanho. Voz de grito dos
pastores, e uivos dos principais do rebanho; porque o Senhor está
destruindo o pasto deles. Porque as suas malhadas pacíficas
serão desarraigadas, por causa do furor da ira do Senhor.
Deixou a sua tenda, como o filho de leão; porque a sua terra
foi posta em desolação, por causa do furor do opressor, e por causa
do furor da sua ira.

\medskip

\lettrine{26} No princípio do reinado de Jeoiaquim, filho de
Josias, rei de Judá, veio esta palavra do Senhor, dizendo: Assim
diz o Senhor: Põe-te no átrio da casa do Senhor e dize a todas as
cidades de Judá, que vêm adorar na casa do Senhor, todas as palavras
que te mandei que lhes dissesses; não omitas nenhuma palavra.
Bem pode ser que ouçam, e se convertam cada um do seu mau
caminho, e eu me arrependa do mal que intento fazer-lhes por causa
da maldade das suas ações. Dize-lhes pois: Assim diz o Senhor:
Se não me derdes ouvidos para andardes na minha lei, que pus diante
de vós, para que ouvísseis as palavras dos meus servos, os
profetas, que eu vos envio, madrugando e enviando, mas não ouvistes;
então farei que esta casa seja como Siló, e farei desta cidade
uma maldição para todas as nações da terra.

Os sacerdotes, e os profetas, e todo o povo, ouviram a Jeremias,
falando estas palavras na casa do Senhor. E sucedeu que,
acabando Jeremias de dizer tudo quanto o Senhor lhe havia ordenado
que dissesse a todo o povo, pegaram nele os sacerdotes, e os
profetas, e todo o povo, dizendo: Certamente morrerás, por que
profetizaste no nome do Senhor, dizendo: Como Siló será esta casa, e
esta cidade será assolada, de sorte que não fique nenhum morador
nela? E ajuntou-se todo o povo contra Jeremias, na casa do Senhor.
E, ouvindo os príncipes de Judá estas palavras, subiram da
casa do rei à casa do Senhor, e se assentaram à entrada da porta
nova do Senhor. Então falaram os sacerdotes e os profetas aos
príncipes e a todo o povo, dizendo: Este homem é réu de morte,
porque profetizou contra esta cidade, como ouvistes com os vossos
ouvidos. E falou Jeremias a todos os príncipes e a todo o
povo, dizendo: O Senhor me enviou a profetizar contra esta casa, e
contra esta cidade, todas as palavras que ouvistes. Agora,
pois, melhorai os vossos caminhos e as vossas ações, e ouvi a voz do
Senhor vosso Deus, e arrepender-se-á o Senhor do mal que falou
contra vós. Quanto a mim, eis que estou nas vossas mãos;
fazei de mim conforme o que for bom e reto aos vossos olhos.
Sabei, porém, com certeza que, se me matardes, trareis sangue
inocente sobre vós, e sobre esta cidade, e sobre os seus habitantes;
porque, na verdade, o Senhor me enviou a vós, para dizer aos vossos
ouvidos todas estas palavras.

Então disseram os príncipes, e todo o povo aos sacerdotes e aos
profetas: Este homem não é réu de morte, porque em nome do Senhor,
nosso Deus, nos falou. Também se levantaram alguns homens
dentre os anciãos da terra, e falaram a toda a congregação do povo,
dizendo: Miquéias, o morastita, profetizou nos dias de
Ezequias, rei de Judá, e falou a todo o povo de Judá, dizendo: Assim
disse o Senhor dos Exércitos: Sião será lavrada como um campo, e
Jerusalém se tornará em montões de pedras, e o monte desta casa como
os altos de um bosque. Mataram-no, porventura, Ezequias, rei
de Judá, e todo o Judá? Antes não temeu ao Senhor, e não implorou o
favor do Senhor? E o Senhor não se arrependeu do mal que falara
contra eles? Nós fazemos um grande mal contra as nossas almas.
Também houve outro homem que profetizava em nome do Senhor, a
saber: Urias, filho de Semaías de Quiriate-Jearim, o qual profetizou
contra esta cidade, e contra esta terra, conforme todas as palavras
de Jeremias. E, ouvindo o rei Jeoiaquim, e todos os seus
poderosos e todos os príncipes, as suas palavras, procurou o rei
matá-lo; mas ouvindo isto, Urias temeu e fugiu, e foi para o Egito;
mas o rei Jeoiaquim enviou alguns homens ao Egito, a saber:
Elnatã, filho de Acbor, e outros homens com ele, ao Egito. Os
quais tiraram a Urias do Egito, e o trouxeram ao rei Jeoiaquim, que
o feriu à espada, e lançou o seu cadáver nas sepulturas dos filhos
do povo. Porém a mão de Aicão, filho de Safã, foi com
Jeremias, para que o não entregassem na mão do povo, para ser morto.

\medskip

\lettrine{27} No princípio do reinado de Jeoiaquim, filho de
Josias, rei de Judá, veio esta palavra a Jeremias da parte do
Senhor, dizendo: Assim me disse o Senhor: Faze uns grilhões e
jugos, e põe-nos ao teu pescoço. E envia-os ao rei de Edom, e ao
rei de Moabe, e ao rei dos filhos de Amom, e ao rei de Tiro, e ao
rei de Sidom, pela mão dos mensageiros que vêm a Jerusalém a ter com
Zedequias, rei de Judá. E lhes ordenarás que digam aos seus
senhores: Assim diz o Senhor dos Exércitos, o Deus de Israel: Assim
direis a vossos senhores: Eu fiz a terra, o homem, e os animais
que estão sobre a face da terra, com o meu grande poder, e com o meu
braço estendido, e a dou a quem é reto aos meus olhos. E agora
eu entreguei todas estas terras na mão de Nabucodonosor, rei de
Babilônia, meu servo; e ainda até os animais do campo lhe dei, para
que o sirvam. E todas as nações servirão a ele, e a seu filho, e
ao filho de seu filho, até que também venha o tempo da sua própria
terra, quando muitas nações e grandes reis se servirão dele. E
acontecerá que, se alguma nação e reino não servirem o mesmo
Nabucodonosor, rei de Babilônia, e não puserem o seu pescoço debaixo
do jugo do rei de Babilônia, a essa nação castigarei com espada, e
com fome, e com peste, diz o Senhor, até que a consuma pela sua mão.
E vós não deis ouvidos aos vossos profetas, e aos vossos
adivinhos, e aos vossos sonhos, e aos vossos agoureiros, e aos
vossos encantadores, que vos falam, dizendo: Não servireis ao rei de
Babilônia. Porque mentiras vos profetizam, para vos mandarem
para longe da vossa terra, e para que eu vos expulse dela, e
pereçais. Mas a nação que colocar o seu pescoço sob o jugo do
rei de Babilônia, e o servir, eu a deixarei na sua terra, diz o
Senhor, e lavrá-la-á e habitará nela.

E falei com Zedequias, rei de Judá, conforme todas estas
palavras, dizendo: Colocai os vossos pescoços no jugo do rei de
Babilônia, e servi-o, a ele e ao seu povo, e vivereis. Por
que morrerias tu e o teu povo, à espada, e à fome, e de peste, como
o Senhor disse contra a nação que não servir ao rei de Babilônia?
E não deis ouvidos às palavras dos profetas, que vos falam,
dizendo: Não servireis ao rei de Babilônia; porque vos profetizam
mentiras. Porque não os enviei, diz o Senhor, e profetizam
falsamente em meu nome; para que eu vos lance fora, e pereçais, vós
e os profetas que vos profetizam. Também falei aos
sacerdotes, e a todo este povo, dizendo: Assim diz o Senhor: Não
deis ouvidos às palavras dos vossos profetas, que vos profetizam,
dizendo: Eis que os utensílios da casa do Senhor cedo voltarão de
Babilônia, porque vos profetizam mentiras. Não lhes deis
ouvidos, servi ao rei de Babilônia, e vivereis; por que se tornaria
esta cidade em desolação? Porém, se são profetas, e se há
palavras do Senhor com eles, orem agora ao Senhor dos Exércitos,
para que os utensílios que ficaram na casa do Senhor, e na casa do
rei de Judá, e em Jerusalém, não vão para a Babilônia. Porque
assim diz o Senhor dos Exércitos acerca das colunas, e do mar, e das
bases, e dos demais utensílios que ficaram na cidade, os
quais Nabucodonosor, rei de Babilônia, não levou, quando transportou
de Jerusalém para Babilônia a Jeconias, filho de Jeoiaquim, rei de
Judá, como também a todos os nobres de Judá e de Jerusalém;
assim, pois, diz o Senhor dos Exércitos, o Deus de Israel,
acerca dos utensílios que ficaram na casa do Senhor, e na casa do
rei de Judá, e em Jerusalém: À Babilônia serão levados, e ali
ficarão até o dia em que eu os visitarei, diz o Senhor; então os
farei subir, e os tornarei a trazer a este lugar.

\medskip

\lettrine{28} E sucedeu no mesmo ano, no princípio do reinado
de Zedequias, rei de Judá, no ano quarto, no mês quinto, que
Hananias, filho de Azur, o profeta que era de Gibeom, me falou na
casa do Senhor, na presença dos sacerdotes e de todo o povo,
dizendo: Assim fala o Senhor dos Exércitos, o Deus de Israel,
dizendo: Eu quebrei o jugo do rei de Babilônia. Depois de
passados dois anos completos, eu tornarei a trazer a este lugar
todos os utensílios da casa do Senhor, que deste lugar tomou
Nabucodonosor, rei de Babilônia, levando-os a Babilônia. Também
a Jeconias, filho de Jeoiaquim, rei de Judá, e a todos os do
cativeiro de Judá, que entraram em Babilônia, eu tornarei a trazer a
este lugar, diz o Senhor; porque quebrarei o jugo do rei de
Babilônia. Então falou o profeta Jeremias ao profeta Hananias,
na presença dos sacerdotes, e na presença de todo o povo que estava
na casa do Senhor. Disse, pois, Jeremias, o profeta: Amém! Assim
faça o Senhor; confirme o Senhor as tuas palavras, que profetizaste,
e torne ele a trazer os utensílios da casa do Senhor, e todos os do
cativeiro de Babilônia a este lugar. Mas ouve agora esta
palavra, que eu falo aos teus ouvidos e aos ouvidos de todo o povo:
Os profetas que houve antes de mim e antes de ti, desde a
antiguidade, profetizaram contra muitas terras, e contra grandes
reinos, acerca de guerra, e de mal, e de peste. O profeta que
profetizar de paz, quando se cumprir a palavra desse profeta, será
conhecido como aquele a quem o Senhor na verdade enviou.

Então Hananias, o profeta, tomou o jugo do pescoço do profeta
Jeremias, e o quebrou. E falou Hananias na presença de todo o
povo, dizendo: Assim diz o Senhor: Assim, passados dois anos
completos, quebrarei o jugo de Nabucodonosor, rei de Babilônia, de
sobre o pescoço de todas as nações. E Jeremias, o profeta, seguiu o
seu caminho. Mas veio a palavra do Senhor a Jeremias, depois
que Hananias, o profeta, quebrou o jugo de sobre o pescoço de
Jeremias, o profeta, dizendo: Vai, e fala a Hananias,
dizendo: Assim diz o Senhor: Jugos de madeira quebraste, mas em vez
deles farás jugos de ferro. Porque assim diz o Senhor dos
Exércitos, o Deus de Israel: Jugo de ferro pus sobre o pescoço de
todas estas nações, para servirem a Nabucodonosor, rei de Babilônia,
e servi-lo-ão, e até os animais do campo lhe dei. E disse o
profeta Jeremias ao profeta Hananias: Ouve agora, Hananias: Não te
enviou o Senhor, mas tu fizeste que este povo confiasse em mentiras.
Portanto, assim diz o Senhor: Eis que te lançarei de sobre a
face da terra; este ano morrerás, porque falaste em rebeldia contra
o Senhor. E morreu Hananias, o profeta, no mesmo ano, no
sétimo mês.

\medskip

\lettrine{29} E estas são as palavras da carta que Jeremias, o
profeta, enviou de Jerusalém, aos que restaram dos anciãos do
cativeiro, como também aos sacerdotes, e aos profetas, e a todo o
povo que Nabucodonosor havia deportado de Jerusalém para Babilônia
 (Depois que saíram de Jerusalém o rei Jeconias, e a rainha, e os
eunucos, e os príncipes de Judá e Jerusalém, e os carpinteiros e
ferreiros), pela mão de Elasa, filho de Safã, e de Gemarias,
filho de Hilquias (os quais Zedequias, rei de Judá, tinha enviado a
Babilônia, a Nabucodonosor, rei de Babilônia), dizendo: Assim
diz o Senhor dos Exércitos, o Deus de Israel, a todos os do
cativeiro, os quais fiz transportar de Jerusalém para Babilônia:
Edificai casas e habitai-as; e plantai jardins, e comei o seu
fruto. Tomai mulheres e gerai filhos e filhas, e tomai mulheres
para vossos filhos, e dai vossas filhas a maridos, para que tenham
filhos e filhas; e multiplicai-vos ali, e não vos diminuais. E
procurai a paz da cidade, para onde vos fiz transportar em
cativeiro, e orai por ela ao Senhor; porque na sua paz vós tereis
paz.

Porque assim diz o Senhor dos Exércitos, o Deus de Israel: Não vos
enganem os vossos profetas que estão no meio de vós, nem os vossos
adivinhos, nem deis ouvidos aos vossos sonhos, que sonhais;
porque eles vos profetizam falsamente em meu nome; não os
enviei, diz o Senhor. Porque assim diz o Senhor: Certamente
que passados setenta anos em Babilônia, vos visitarei, e cumprirei
sobre vós a minha boa palavra, tornando a trazer-vos a este lugar.
Porque eu bem sei os pensamentos que tenho a vosso respeito,
diz o Senhor; pensamentos de paz, e não de mal, para vos dar o fim
que esperais. Então me invocareis, e ireis, e orareis a mim,
e eu vos ouvirei. E buscar-me-eis, e me achareis, quando me
buscardes com todo o vosso coração. E serei achado de vós,
diz o Senhor, e farei voltar os vossos cativos e congregar-vos-ei de
todas as nações, e de todos os lugares para onde vos lancei, diz o
Senhor, e tornarei a trazer-vos ao lugar de onde vos transportei.

Porque dizeis: O Senhor nos levantou profetas em Babilônia.
Porque assim diz o Senhor acerca do rei que se assenta no
trono de Davi, e de todo o povo que habita nesta cidade, vossos
irmãos, que não saíram conosco para o cativeiro. Assim diz o
Senhor dos Exércitos: Eis que enviarei entre eles a espada, a fome e
a peste, e fá-los-ei como a figos podres que não se podem comer, de
ruins que são. E persegui-los-ei com a espada, com a fome, e
com a peste; e dá-los-ei para deslocarem-se por todos os reinos da
terra, para serem uma maldição, e um espanto, e um assobio, e um
opróbrio entre todas as nações para onde os tiver lançado.
Porquanto não deram ouvidos às minhas palavras, diz o Senhor,
mandando-lhes eu os meus servos, os profetas, madrugando e enviando;
mas vós não escutastes, diz o Senhor. Vós, pois, ouvi a
palavra do Senhor, todos os do cativeiro que enviei de Jerusalém a
Babilônia. Assim diz o Senhor dos Exércitos, o Deus de
Israel, acerca de Acabe, filho de Colaías, e de Zedequias, filho de
Maaséias, que vos profetizam falsamente em meu nome: Eis que os
entregarei na mão de Nabucodonosor, rei de Babilônia, e ele os
ferirá diante dos vossos olhos. E todos os transportados de
Judá, que estão em Babilônia, tomarão deles uma maldição, dizendo: O
Senhor te faça como Zedequias, e como Acabe, os quais o rei de
Babilônia assou no fogo; porquanto fizeram loucura em Israel,
e cometeram adultério com as mulheres dos seus vizinhos, e
anunciaram falsamente em meu nome uma palavra que não lhes mandei, e
eu o sei e sou testemunha disso, diz o Senhor.

E a Semaías, o neelamita, falarás, dizendo: Assim fala o
Senhor dos Exércitos, o Deus de Israel, dizendo: Porquanto tu
enviaste no teu nome cartas a todo o povo que está em Jerusalém,
como também a Sofonias, filho de Maaséias, o sacerdote, e a todos os
sacerdotes, dizendo: O Senhor te pôs por sacerdote em lugar
de Joiada, o sacerdote, para que sejas encarregado da casa do Senhor
sobre todo o homem fanático\footnote{KJ: The LORD hath made thee
priest in the stead of Jehoiada the priest, that ye should be
officers in the house of the LORD, for every man that is mad, and
maketh himself a prophet, that thou shouldest put him in prison, and
in the stocks.}, e que profetiza, para o lançares na prisão e no
tronco. Agora, pois, por que não repreendeste a Jeremias, o
anatotita, que vos profetiza? Porque até nos mandou dizer em
Babilônia: Ainda o cativeiro muito há de durar; edificai casas, e
habitai nelas; e plantai pomares, e comei o seu fruto. E leu
Sofonias, o sacerdote, esta carta aos ouvidos de Jeremias, o
profeta. E veio a palavra do Senhor a Jeremias, dizendo:
Manda a todos os do cativeiro, dizendo: Assim diz o Senhor
acerca de Semaías, o neelamita: Porquanto Semaías vos profetizou, e
eu não o enviei, e vos fez confiar em mentiras, portanto
assim diz o Senhor: Eis que castigarei a Semaías, o neelamita, e a
sua descendência; ele não terá ninguém que habite entre este povo, e
não verá o bem que hei de fazer ao meu povo, diz o Senhor, porque
falou em rebeldia contra o Senhor.

\medskip

\lettrine{30} A palavra que do Senhor veio a Jeremias,
dizendo: Assim diz o Senhor Deus de Israel: Escreve num livro
todas as palavras que te tenho falado. Porque eis que vêm dias,
diz o Senhor, em que farei voltar do cativeiro o meu povo Israel, e
de Judá, diz o Senhor; e tornarei a trazê-los à terra que dei a seus
pais, e a possuirão. E estas são as palavras que disse o Senhor,
acerca de Israel e de Judá. Porque assim diz o Senhor: Ouvimos
uma voz de tremor, de temor mas não de paz. Perguntai, pois, e
vede, se um homem pode dar à luz. Por que, pois, vejo a cada homem
com as mãos sobre os lombos como a que está dando à luz? e por que
se tornaram pálidos todos os rostos? Ah! porque aquele dia é tão
grande, que não houve outro semelhante; e é tempo de angústia para
Jacó; ele, porém, será salvo dela. Porque será naquele dia, diz
o Senhor dos Exércitos, que eu quebrarei o seu jugo de sobre o teu
pescoço, e quebrarei os teus grilhões; e nunca mais se servirão dele
os estrangeiros. Mas servirão ao Senhor, seu Deus, como também a
Davi, seu rei, que lhes levantarei.

Não temas, pois, tu, ó meu servo Jacó, diz o Senhor, nem te
espantes, ó Israel; porque eis que te livrarei de terras de longe, e
à tua descendência da terra do seu cativeiro; e Jacó voltará, e
descansará, e ficará em sossego, e não haverá quem o atemorize.
Porque eu sou contigo, diz o Senhor, para te salvar;
porquanto darei fim a todas as nações entre as quais te espalhei; a
ti, porém, não darei fim, mas castigar-te-ei com medida, e de todo
não te terei por inocente. Porque assim diz o Senhor: A tua
ferida é incurável; a tua chaga é dolorosa. Não há quem
defenda a tua causa para te aplicar curativo; não tens remédios que
possam curar. Todos os teus amantes se esqueceram de ti, e
não perguntam por ti; porque te feri com ferida de inimigo, e com
castigo de quem é cruel, pela grandeza da tua maldade e multidão de
teus pecados. Por que gritas por causa da tua ferida? Tua dor
é incurável. Pela grandeza de tua maldade, e multidão de teus
pecados, eu fiz estas coisas. Por isso todos os que te
devoram serão devorados; e todos os teus adversários irão, todos
eles, para o cativeiro; e os que te roubam serão roubados, e a todos
os que te despojam entregarei ao saque. Porque te restaurarei
a saúde, e te curarei as tuas chagas, diz o Senhor; porquanto te
chamaram a repudiada, dizendo: É Sião, já ninguém pergunta por ela.

Assim diz o Senhor: Eis que farei voltar do cativeiro as tendas
de Jacó, e apiedar-me-ei das suas moradas; e a cidade será
reedificada sobre o seu montão, e o palácio permanecerá como
habitualmente. E sairá deles o louvor e a voz de júbilo; e
multiplicá-los-ei, e não serão diminuídos, e glorificá-los-ei, e não
serão apoucados\footnote{Que se apoucou. Reduzido a pouco; limitado
em quantidade; escasso. Derivação (sentido figurado): pouco
valorizado; amesquinhado, humilhado. Derivação (sentido figurado):
sem iniciativa, de pouco préstimo; acanhado, tímido, pusilânime.}.
E seus filhos serão como na antiguidade, e a sua congregação
será confirmada diante de mim; e castigarei todos os seus
opressores. E os seus nobres serão deles; e o seu governador
sairá do meio deles, e o farei aproximar, e ele se chegará a mim;
pois, quem de si mesmo se empenharia para chegar-se a mim? diz o
Senhor. E ser-me-eis por povo, e eu vos serei por Deus.
Eis que a tempestade do Senhor, a sua indignação, já saiu;
uma tempestade varredora, cairá cruelmente sobre a cabeça dos
ímpios. Não voltará atrás o furor da ira do Senhor, até que
tenha executado e até que tenha cumprido os desígnios do seu
coração; no fim dos dias entendereis isto.

\medskip

\lettrine{31} Naquele tempo, diz o Senhor, serei o Deus de
todas as famílias de Israel, e elas serão o meu povo. Assim diz
o Senhor: O povo dos que escaparam da espada achou graça no deserto.
Israel mesmo, quando eu o fizer descansar. Há muito que o Senhor
me apareceu, dizendo: Porquanto com amor eterno te amei, por isso
com benignidade te atraí. Ainda te edificarei, e serás
edificada, ó virgem de Israel! Ainda serás adornada com os teus
tamboris, e sairás nas danças dos que se alegram. Ainda
plantarás vinhas nos montes de Samaria; os plantadores as plantarão
e comerão como coisas comuns. Porque haverá um dia em que
gritarão os vigias sobre o monte de Efraim: Levantai-vos, e subamos
a Sião, ao Senhor nosso Deus. Porque assim diz o Senhor: Cantai
sobre Jacó com alegria, e exultai por causa do chefe das nações;
proclamai, cantai louvores, e dizei: Salva, Senhor, ao teu povo, o
restante de Israel. Eis que os trarei da terra do norte, e os
congregarei das extremidades da terra; entre os quais haverá cegos e
aleijados, grávidas e as de parto juntamente; em grande congregação
voltarão para aqui. Virão com choro, e com súplicas os levarei;
guiá-los-ei aos ribeiros de águas, por caminho direito, no qual não
tropeçarão, porque sou um pai para Israel, e Efraim é o meu
primogênito.

Ouvi a palavra do Senhor, ó nações, e anunciai-a nas ilhas
longínquas, e dizei: Aquele que espalhou a Israel o congregará e o
guardará, como o pastor ao seu rebanho. Porque o Senhor
resgatou a Jacó, e o livrou da mão do que era mais forte do que ele.
Assim que virão, e exultarão no alto de Sião, e correrão aos
bens do Senhor, ao trigo, e ao mosto, e ao azeite, e aos cordeiros e
bezerros; e a sua alma será como um jardim regado, e nunca mais
andarão tristes. Então a virgem se alegrará na dança, como
também os jovens e os velhos juntamente; e tornarei o seu pranto em
alegria, e os consolarei, e lhes darei alegria em lugar de tristeza.
E saciarei a alma dos sacerdotes com gordura, e o meu povo se
fartará dos meus bens, diz o Senhor. Assim diz o Senhor: Uma
voz se ouviu em Ramá, lamentação, choro amargo; Raquel chora seus
filhos; não quer ser consolada quanto a seus filhos, porque já não
existem. Assim diz o Senhor: Reprime a tua voz de choro, e as
lágrimas de teus olhos; porque há galardão para o teu trabalho, diz
o Senhor, pois eles voltarão da terra do inimigo. E há
esperança quanto ao teu futuro, diz o Senhor, porque teus filhos
voltarão para os seus termos.

Bem ouvi eu que Efraim se queixava, dizendo: Castigaste-me e fui
castigado, como novilho ainda não domado; converte-me, e
converter-me-ei, porque tu és o Senhor meu Deus. Na verdade
que, depois que me converti, tive arrependimento; e depois que fui
instruído, bati na minha coxa; fiquei confuso, e também me
envergonhei; porque suportei o opróbrio da minha mocidade.
Não é Efraim para mim um filho precioso, criança das minhas
delícias? Porque depois que falo contra ele, ainda me lembro dele
solicitamente; por isso se comovem por ele as minhas entranhas;
deveras me compadecerei dele, diz o Senhor. Levanta para ti
sinais, faze para ti altos marcos, aplica o teu coração à vereda, ao
caminho por onde andaste; volta, pois, ó virgem de Israel, regressa
a estas tuas cidades. Até quando andarás errante, ó filha
rebelde? Porque o Senhor criou uma coisa nova sobre a terra; uma
mulher cercará a um homem. Assim diz o Senhor dos Exércitos,
o Deus de Israel: Ainda dirão esta palavra na terra de Judá, e nas
suas cidades, quando eu vos restaurar do seu cativeiro: O Senhor te
abençoe, ó morada de justiça, ó monte de santidade! E nela
habitarão Judá, e todas as suas cidades juntamente; como também os
lavradores e os que pastoreiam o rebanho. Porque satisfiz a
alma cansada, e toda a alma entristecida saciei. Nisto
despertei, e olhei, e o meu sono foi doce para mim.

Eis que dias vêm, diz o Senhor, em que semearei a casa de Israel,
e a casa de Judá, com a semente de homens, e com a semente de
animais. E será que, como velei\footnote{Permanecer de vigia,
de sentinela, de guarda; vigiar. Ficar acordado, em vigília. Passar
(um período de tempo) acordado. Permanecer acordado, ao pé de
(alguém que dorme, que está enfermo, ou que está morto). Dispensar
cuidados, proteção a; tratar de, interessar-se, dedicar-se, zelar,
proteger.} sobre eles, para arrancar, e para derrubar, e para
transtornar, e para destruir, e para afligir, assim velarei sobre
eles, para edificar e para plantar, diz o Senhor. Naqueles
dias nunca mais dirão: Os pais comeram uvas verdes, e os dentes dos
filhos se embotaram. Mas cada um morrerá pela sua iniqüidade;
de todo o homem que comer as uvas verdes os dentes se embotarão.
Eis que dias vêm, diz o Senhor, em que farei uma aliança nova
com a casa de Israel e com a casa de Judá. Não conforme a
aliança que fiz com seus pais, no dia em que os tomei pela mão, para
os tirar da terra do Egito; porque eles invalidaram a minha aliança
apesar de eu os haver desposado, diz o Senhor. Mas esta é a
aliança que farei com a casa de Israel depois daqueles dias, diz o
Senhor: Porei a minha lei no seu interior, e a escreverei no seu
coração; e eu serei o seu Deus e eles serão o meu povo. E não
ensinará mais cada um a seu próximo, nem cada um a seu irmão,
dizendo: Conhecei ao Senhor; porque todos me conhecerão, desde o
menor até ao maior deles, diz o Senhor; porque lhes perdoarei a sua
maldade, e nunca mais me lembrarei dos seus pecados.

Assim diz o Senhor, que dá o sol para luz do dia, e as ordenanças
da lua e das estrelas para luz da noite, que agita o mar, bramando
as suas ondas; o Senhor dos Exércitos é o seu nome. Se
falharem estas ordenanças de diante de mim, diz o Senhor, deixará
também a descendência de Israel de ser uma nação diante de mim para
sempre. Assim disse o Senhor: Se puderem ser medidos os céus
lá em cima, e sondados os fundamentos da terra cá em baixo, também
eu rejeitarei toda a descendência de Israel, por tudo quanto
fizeram, diz o Senhor. Eis que vêm dias, diz o Senhor, em que
esta cidade será reedificada para o Senhor, desde a torre de
Hanameel até à porta da esquina. E a linha de medir
estender-se-á para diante dela, até ao outeiro de Garebe, e
virar-se-á para Goa. E todo o vale dos cadáveres e da cinza,
e todos os campos até ao ribeiro de Cedrom, até à esquina da porta
dos cavalos para o oriente, serão consagrados ao Senhor; não se
arrancará nem se derrubará mais eternamente.

\medskip

\lettrine{32} A palavra que veio a Jeremias da parte do
Senhor, no ano décimo de Zedequias, rei de Judá, o qual foi o décimo
oitavo de Nabucodonosor. Ora, nesse tempo o exército do rei de
Babilônia cercava Jerusalém; e Jeremias, o profeta, estava encerrado
no pátio da guarda que estava na casa do rei de Judá; porque
Zedequias, rei de Judá, o tinha encerrado, dizendo: Por que
profetizas tu, dizendo: Assim diz o Senhor: Eis que entrego esta
cidade na mão do rei de Babilônia, e ele a tomará; e Zedequias,
rei de Judá, não escapará das mãos dos caldeus; mas certamente será
entregue na mão do rei de Babilônia, e com ele falará boca a boca, e
os seus olhos verão os dele; e ele levará Zedequias para
Babilônia, e ali estará, até que eu o visite, diz o Senhor e, ainda
que pelejeis contra os caldeus, não ganhareis? Disse, pois,
Jeremias: Veio a mim a palavra do Senhor, dizendo: Eis que
Hanameel, filho de Salum, teu tio, virá a ti dizendo: Compra para ti
a minha herdade que está em Anatote, pois tens o direito de resgate
para comprá-la. Veio, pois, a mim Hanameel, filho de meu tio,
segundo a palavra do Senhor, ao pátio da guarda, e me disse: Compra
agora a minha herdade que está em Anatote, na terra de Benjamim;
porque teu é o direito de herança, e tens o resgate; compra-a para
ti. Então entendi que isto era a palavra do Senhor. Comprei,
pois, a herdade de Hanameel, filho de meu tio, a qual está em
Anatote; e pesei-lhe o dinheiro, dezessete siclos de prata. E
assinei a escritura, e selei-a, e fiz confirmar por testemunhas; e
pesei-lhe o dinheiro numa balança. E tomei a escritura da
compra, selada segundo a lei e os estatutos, e a cópia aberta.
E dei a escritura da compra a Baruque, filho de Nerias, filho
de Maaséias, na presença de Hanameel, filho de meu tio e na presença
das testemunhas, que subscreveram a escritura da compra, e na
presença de todos os judeus que se assentavam no pátio da guarda.
E dei ordem a Baruque, na presença deles, dizendo:
Assim diz o Senhor dos Exércitos, o Deus de Israel: Toma
estas escrituras, este auto de compra, tanto a selada, como a
aberta, e coloca-as num vaso de barro, para que se possam conservar
muitos dias. Porque assim diz o Senhor dos Exércitos, o Deus
de Israel: Ainda se comprarão casas, e campos, e vinhas nesta terra.

E depois que dei a escritura da compra a Baruque, filho de
Nerias, orei ao Senhor, dizendo: Ah Senhor Deus! Eis que tu
fizeste os céus e a terra com o teu grande poder, e com o teu braço
estendido; nada há que te seja demasiado difícil; tu que usas
de benignidade com milhares, e retribuis a maldade dos pais ao seio
dos filhos depois deles; o grande, o poderoso Deus cujo nome é o
Senhor dos Exércitos; grande em conselho, e magnífico em
obras; porque os teus olhos estão abertos sobre todos os caminhos
dos filhos dos homens, para dar a cada um segundo os seus caminhos e
segundo o fruto das suas obras; tu puseste sinais e
maravilhas na terra do Egito até ao dia de hoje, tanto em Israel,
como entre os outros homens, e te fizeste um nome, o qual tu tens
neste dia. E tiraste o teu povo Israel da terra do Egito, com
sinais e com maravilhas, e com mão forte, e com braço estendido, e
com grande espanto, e lhes deste esta terra, que juraste a
seus pais que lhes havias de dar, terra que mana leite e mel.
E entraram nela, e a possuíram, mas não obedeceram à tua voz,
nem andaram na tua lei; tudo o que lhes mandaste que fizessem, eles
não o fizeram; por isso ordenaste lhes sucedesse todo este mal.
Eis aqui os valados\footnote{Vala rodeada de tapume ou sebe,
destinada a servir de barreira em fortificações ou a proteger
propriedades rurais; fosso. Propriedade rural cercada de valas ou de
valados. Elevação de terra que delimita uma propriedade.}; já vieram
contra a cidade para tomá-la, e a cidade está entregue na mão dos
caldeus, que pelejam contra ela, pela espada, pela fome e pela
pestilência; e o que disseste se cumpriu, e eis aqui o estás
presenciando. Contudo tu me disseste, ó Senhor Deus: Compra
para ti o campo por dinheiro, e faze que o confirmem testemunhas,
embora a cidade já esteja entregue na mão dos caldeus.

Então veio a palavra do Senhor a Jeremias, dizendo: Eis
que eu sou o Senhor, o Deus de toda a carne; acaso haveria alguma
coisa demasiado difícil para mim? Portanto assim diz o
Senhor: Eis que eu entrego esta cidade na mão dos caldeus, e na mão
de Nabucodonosor, rei de Babilônia, e ele a tomará. E os
caldeus, que pelejam contra esta cidade, entrarão nela, e pôr-lhe-ão
fogo, e queimarão, as casas sobre cujos terraços queimaram incenso a
Baal e ofereceram libações a outros deuses, para me provocarem à
ira. Porque os filhos de Israel e os filhos de Judá não
fizeram senão mal aos meus olhos, desde a sua mocidade; porque os
filhos de Israel nada fizeram senão provocar-me à ira com as obras
das suas mãos, diz o Senhor. Porque para a minha ira e para o
meu furor me tem sido esta cidade, desde o dia em que a edificaram,
e até ao dia de hoje, para que a tirasse da minha presença;
por causa de toda a maldade dos filhos de Israel, e dos
filhos de Judá, que fizeram, para me provocarem à ira, eles e os
seus reis, os seus príncipes, os seus sacerdotes, e os seus
profetas, como também os homens de Judá e os moradores de Jerusalém.
E viraram-me as costas, e não o rosto; ainda que eu os
ensinava, madrugando e ensinando-os, contudo eles não deram ouvidos,
para receberem o ensino. Antes puseram as suas abominações na
casa que se chama pelo meu nome, para a profanarem. E
edificaram os altos de Baal, que estão no Vale do Filho de Hinom,
para fazerem passar seus filhos e suas filhas pelo fogo a Moloque; o
que nunca lhes ordenei, nem veio ao meu coração, que fizessem tal
abominação, para fazerem pecar a Judá. E por isso agora assim
diz o Senhor, o Deus de Israel, acerca desta cidade, da qual vós
dizeis: Já está dada na mão do rei de Babilônia, pela espada, pela
fome, e pela pestilência: Eis que eu os congregarei de todas
as terras, para onde os tenho lançado na minha ira, e no meu furor,
e na minha grande indignação; e os tornarei a trazer a este lugar, e
farei que habitem nele seguramente. E eles serão o meu povo,
e eu lhes serei o seu Deus; e lhes darei um mesmo coração, e
um só caminho, para que me temam todos os dias, para seu bem, e o
bem de seus filhos, depois deles. E farei com eles uma
aliança eterna de não me desviar de fazer-lhes o bem; e porei o meu
temor nos seus corações, para que nunca se apartem de mim. E
alegrar-me-ei deles, fazendo-lhes bem; e plantá-los-ei nesta terra
firmemente, com todo o meu coração e com toda a minha alma.
Porque assim diz o Senhor: Como eu trouxe sobre este povo
todo este grande mal, assim eu trarei sobre ele todo o bem que lhes
tenho declarado. E comprar-se-ão campos nesta terra, da qual
vós dizeis: Está desolada, sem homens, sem animais; está entregue na
mão dos caldeus. Comprarão campos por dinheiro, e assinarão
as escrituras, e as selarão, e farão que confirmem testemunhas, na
terra de Benjamim, e nos contornos de Jerusalém, e nas cidades de
Judá, e nas cidades das montanhas, e nas cidades das planícies, e
nas cidades do sul; porque os farei voltar do seu cativeiro, diz o
Senhor.

\medskip

\lettrine{33} E veio a palavra do Senhor a Jeremias, segunda
vez, estando ele ainda encarcerado no pátio da guarda, dizendo:
Assim diz o Senhor que faz isto, o Senhor que forma isto, para o
estabelecer; o Senhor é o seu nome. Clama a mim, e
responder-te-ei, e anunciar-te-ei coisas grandes e firmes que não
sabes. Porque assim diz o Senhor, o Deus de Israel, acerca das
casas desta cidade, e das casas dos reis de Judá, que foram
derrubadas com os aríetes e à espada. Eles entraram a pelejar
contra os caldeus, mas isso é para os encher de cadáveres de homens,
que feri na minha ira e no meu furor; porquanto escondi o meu rosto
desta cidade, por causa de toda a sua maldade. Eis que eu trarei
a ela saúde e cura, e os sararei, e lhes manifestarei abundância de
paz e de verdade. E removerei o cativeiro de Judá e o cativeiro
de Israel, e os edificarei como ao princípio. E os purificarei
de toda a sua maldade com que pecaram contra mim; e perdoarei todas
as suas maldades, com que pecaram e transgrediram contra mim; e
este lugar me servirá de nome, de gozo, de louvor, e de glória,
entre todas as nações da terra, que ouvirem todo o bem que eu lhe
faço; e espantar-se-ão e perturbar-se-ão por causa de todo o bem, e
por causa de toda a paz que eu lhe dou.

Assim diz o Senhor: Neste lugar de que vós dizeis que está
desolado, e sem homem, sem animal nas cidades de Judá, e nas ruas de
Jerusalém, que estão assoladas, sem homem, sem morador, sem animal,
ainda se ouvirá: A voz de gozo, e a voz de alegria, a voz do
esposo e a voz da esposa, e a voz dos que dizem: Louvai ao Senhor
dos Exércitos, porque bom é o Senhor, porque a sua benignidade dura
para sempre; dos que trazem ofertas de ação de graças à casa do
Senhor; pois farei voltar os cativos da terra como ao princípio, diz
o Senhor. Assim diz o Senhor dos Exércitos: Ainda neste
lugar, que está deserto, sem homem nem animal, e em todas as suas
cidades, haverá uma morada de pastores, que façam repousar aos seus
rebanhos. Nas cidades das montanhas, nas cidades das
planícies, e nas cidades do sul, e na terra de Benjamim, e nos
contornos de Jerusalém, e nas cidades de Judá, ainda passarão os
rebanhos pelas mãos dos contadores, diz o Senhor. Eis que vêm
dias, diz o Senhor, em que cumprirei a boa palavra que falei à casa
de Israel e à casa de Judá; naqueles dias e naquele tempo
farei brotar a Davi um Renovo de justiça, e ele fará juízo e justiça
na terra. Naqueles dias Judá será salvo e Jerusalém habitará
seguramente; e este é o nome com o qual Deus a chamará: O Senhor é a
nossa justiça.

Porque assim diz o Senhor: Nunca faltará a Davi homem que se
assente sobre o trono da casa de Israel; nem aos sacerdotes
levíticos faltará homem diante de mim, que ofereça holocausto,
queime oferta de alimentos e faça sacrifício todos os dias. E
veio a palavra do Senhor a Jeremias, dizendo: Assim diz o
Senhor: Se puderdes invalidar a minha aliança com o dia, e a minha
aliança com a noite, de tal modo que não haja dia e noite a seu
tempo, também se poderá invalidar a minha aliança com Davi,
meu servo, para que não tenha filho que reine no seu trono; como
também com os levitas, sacerdotes, meus ministros. Como não
se pode contar o exército dos céus, nem medir-se a areia do mar,
assim multiplicarei a descendência de Davi, meu servo, e os levitas
que ministram diante de mim. E veio ainda a palavra do Senhor
a Jeremias, dizendo: Porventura não tens visto o que este
povo está dizendo: As duas gerações, que o Senhor escolheu, agora as
rejeitou? Assim desprezam o meu povo, como se não fora mais uma
nação diante deles. Assim diz o Senhor: Se a minha aliança
com o dia e com a noite não permanecer, e eu não puser as ordenanças
dos céus e da terra, também rejeitarei a descendência de
Jacó, e de Davi, meu servo, para que não tome da sua descendência os
que dominem sobre a descendência de Abraão, Isaque, e Jacó; porque
removerei o seu cativeiro, e apiedar-me-ei deles.

\medskip

\lettrine{34} A palavra que do Senhor veio a Jeremias, quando
Nabucodonosor, rei de Babilônia, e todo o seu exército, e todos os
reinos da terra, que estavam sob o domínio da sua mão, e todos os
povos, pelejavam contra Jerusalém, e contra todas as suas cidades,
dizendo: Assim diz o Senhor, o Deus de Israel: Vai, e fala a
Zedequias, rei de Judá, e dize-lhe: Assim diz o Senhor: Eis que eu
entrego esta cidade na mão do rei de Babilônia, o qual queimá-la-á a
fogo. E tu não escaparás da sua mão, antes certamente serás
preso e entregue na sua mão; e teus olhos verão os olhos do rei de
Babilônia, e ele te falará boca a boca, e entrarás em Babilônia.
Todavia ouve a palavra do Senhor, ó Zedequias, rei de Judá;
assim diz o Senhor acerca de ti: Não morrerás à espada. Em paz
morrerás, e conforme as queimas para teus pais, os reis precedentes,
que foram antes de ti, assim queimarão para ti, e prantear-te-ão,
dizendo: Ah, Senhor! Pois eu disse a palavra, diz o Senhor. E
falou Jeremias, o profeta, a Zedequias, rei de Judá, todas estas
palavras, em Jerusalém, quando o exército do rei de Babilônia
pelejava contra Jerusalém, e contra todas as cidades que restavam de
Judá, contra Laquis e contra Azeca; porque estas fortes cidades
foram as que ficaram dentre as cidades de Judá.

A palavra que do Senhor veio a Jeremias, depois que o rei
Zedequias fez aliança com todo o povo que havia em Jerusalém, para
lhes apregoar a liberdade; que cada um despedisse livre o seu
servo, e cada um a sua serva, hebreu ou hebréia; de maneira que
ninguém se fizesse servir deles, sendo judeus, seus irmãos. E
obedeceram todos os príncipes, e todo o povo que havia entrado na
aliança, que cada um despedisse livre o seu servo, e cada um a sua
serva, de maneira que não se fizessem mais servir deles; obedeceram,
pois, e os soltaram, mas depois se arrependeram, e fizeram
voltar os servos e as servas que haviam libertado, e os sujeitaram
por servos e por servas. Veio, pois, a palavra do Senhor a
Jeremias, da parte do Senhor, dizendo: Assim diz o Senhor,
Deus de Israel: Eu fiz aliança com vossos pais, no dia em que os
tirei da terra do Egito, da casa da servidão, dizendo: Ao fim
de sete anos libertareis cada um a seu irmão hebreu, que te for
vendido, e te houver servido seis anos, e despedi-lo-ás livre de ti;
mas vossos pais não me ouviram, nem inclinaram os seus ouvidos.
E vos havíeis hoje arrependido, e fizestes o que é reto aos
meus olhos, apregoando liberdade cada um ao seu próximo; e fizestes
diante de mim uma aliança, na casa que se chama pelo meu nome;
mudastes, porém, e profanastes o meu nome, e fizestes voltar
cada um ao seu servo, e cada um à sua serva, os quais já tínheis
despedido libertos conforme a vontade deles; e os sujeitastes, para
que se vos fizessem servos e servas. Portanto assim diz o
Senhor: Vós não me ouvistes a mim, para apregoardes a liberdade,
cada um ao seu irmão, e cada um ao seu próximo; pois eis que eu vos
apregôo a liberdade, diz o Senhor, para a espada, para a
pestilência, e para a fome; e farei que sejais espanto a todos os
reinos da terra. E entregarei os homens que transgrediram a
minha aliança, que não cumpriram as palavras da aliança que fizeram
diante de mim, com o bezerro, que dividiram em duas partes, e
passaram pelo meio das suas porções; a saber, os príncipes de
Judá, e os príncipes de Jerusalém, os eunucos, e os sacerdotes, e
todo o povo da terra que passou por meio das porções do bezerro;
entregá-los-ei, digo, na mão de seus inimigos, e na mão dos
que procuram a sua morte, e os cadáveres deles servirão de alimento
para as aves dos céus e para os animais da terra. E até o rei
Zedequias, rei de Judá, e seus príncipes entregarei na mão de seus
inimigos e na mão dos que procuram a sua morte, a saber, na mão do
exército do rei de Babilônia, que já se retirou de vós. Eis
que eu darei ordem, diz o Senhor, e os farei voltar a esta cidade, e
pelejarão contra ela, e a tomarão, e a queimarão a fogo; e as
cidades de Judá porei em assolação, de sorte que ninguém habite
nelas.

\medskip

\lettrine{35} A palavra que do Senhor veio a Jeremias, nos
dias de Jeoiaquim, filho de Josias, rei de Judá, dizendo: Vai à
casa dos recabitas, e fala com eles, e leva-os à casa do Senhor, a
uma das câmaras e dá-lhes vinho a beber. Então tomei a Jazanias,
filho de Jeremias, filho de Habazinias, e a seus irmãos, e a todos
os seus filhos, e a toda a casa dos recabitas; e os levei à casa
do Senhor, à câmara dos filhos de Hanã, filho de Jigdalias, homem de
Deus, que estava junto à câmara dos príncipes, que ficava sobre a
câmara de Maaséias, filho de Salum, guarda do vestíbulo; e pus
diante dos filhos da casa dos recabitas taças cheias de vinho, e
copos, e disse-lhes: Bebei vinho. Porém eles disseram: Não
beberemos vinho, porque Jonadabe, filho de Recabe, nosso pai, nos
ordenou, dizendo: Nunca jamais bebereis vinho, nem vós nem vossos
filhos; não edificareis casa, nem semeareis semente, nem
plantareis vinha, nem a possuireis; mas habitareis em tendas todos
os vossos dias, para que vivais muitos dias sobre a face da terra,
em que vós andais peregrinando. Obedecemos, pois, à voz de
Jonadabe, filho de Recabe, nosso pai, em tudo quanto nos ordenou; de
maneira que não bebemos vinho em todos os nossos dias, nem nós, nem
nossas mulheres, nem nossos filhos, nem nossas filhas; nem
edificamos casas para nossa habitação; nem temos vinha, nem campo,
nem semente. Mas habitamos em tendas, e assim obedecemos e
fazemos conforme tudo quanto nos ordenou Jonadabe, nosso pai.
Sucedeu, porém, que, subindo Nabucodonosor, rei de Babilônia,
a esta terra, dissemos: Vinde, e vamo-nos a Jerusalém, por causa do
exército dos caldeus, e por causa do exército dos sírios; e assim
ficamos em Jerusalém.

Então veio a palavra do Senhor a Jeremias, dizendo: Assim
diz o Senhor dos Exércitos, o Deus de Israel: Vai, e dize aos homens
de Judá e aos moradores de Jerusalém: Porventura nunca aceitareis
instrução, para ouvirdes as minhas palavras? diz o Senhor. As
palavras de Jonadabe, filho de Recabe, que ordenou a seus filhos que
não bebessem vinho, foram guardadas; pois não beberam até este dia,
antes obedeceram o mandamento de seu pai; a mim, porém, que vos
tenho falado, madrugando e falando, não me ouvistes. E vos
tenho enviado todos os meus servos, os profetas, madrugando, e
insistindo, e dizendo: Convertei-vos, agora, cada um do seu mau
caminho, e fazei boas as vossas ações, e não sigais a outros deuses
para servi-los; e assim ficareis na terra que vos dei a vós e a
vossos pais; porém não inclinastes o vosso ouvido, nem me
obedecestes a mim. Visto que os filhos de Jonadabe, filho de
Recabe, guardaram o mandamento de seu pai que ele lhes ordenou, mas
este povo não me obedeceu, por isso assim diz o Senhor Deus
dos Exércitos, o Deus de Israel: Eis que trarei sobre Judá, e sobre
todos os moradores de Jerusalém, todo o mal que falei contra eles;
pois lhes tenho falado, e não ouviram; e clamei a eles, e não
responderam. E à casa dos recabitas disse Jeremias: Assim diz
o Senhor dos Exércitos, o Deus de Israel: Pois que obedecestes ao
mandamento de Jonadabe, vosso pai, e guardastes todos os seus
mandamentos, e fizestes conforme tudo quanto vos ordenou,
portanto assim diz o Senhor dos Exércitos, Deus de Israel:
Nunca faltará homem a Jonadabe, filho de Recabe, que esteja na minha
presença todos os dias.

\medskip

\lettrine{36} Sucedeu, pois, no ano quarto de Jeoiaquim, filho
de Josias, rei de Judá, que veio esta palavra do Senhor a Jeremias,
dizendo: Toma o rolo de um livro, e escreve nele todas as
palavras que te tenho falado de Israel, e de Judá, e de todas as
nações, desde o dia em que eu te falei, desde os dias de Josias até
ao dia de hoje. Porventura ouvirão os da casa de Judá todo o mal
que eu intento fazer-lhes; para que cada qual se converta do seu mau
caminho, e eu perdoe a sua maldade e o seu pecado. Então
Jeremias chamou a Baruque, filho de Nerias; e escreveu Baruque da
boca de Jeremias no rolo de um livro todas as palavras do Senhor,
que ele lhe tinha falado. E Jeremias deu ordem a Baruque,
dizendo: Eu estou encarcerado; não posso entrar na casa do Senhor.
Entra, pois, tu, e pelo rolo que escreveste da minha boca, lê as
palavras do Senhor aos ouvidos do povo, na casa do Senhor, no dia de
jejum; e também, aos ouvidos de todos os de Judá, que vêm das suas
cidades, as lerás. Pode ser que caia a sua súplica diante do
Senhor, e se converta cada um do seu mau caminho; porque grande é a
ira e o furor que o Senhor tem expressado contra este povo. E
fez Baruque, filho de Nerias, conforme tudo quanto lhe havia
ordenado Jeremias, o profeta, lendo naquele livro as palavras do
Senhor, na casa do Senhor.

E aconteceu, no quinto ano de Jeoiaquim, filho de Josias, rei de
Judá, no mês nono, que apregoaram jejum diante do Senhor a todo o
povo em Jerusalém, como também a todo o povo que vinha das cidades
de Judá a Jerusalém. Leu, pois, Baruque naquele livro as
palavras de Jeremias, na casa do Senhor, na câmara de Gemarias,
filho de Safã, o escriba, no átrio superior, à entrada da porta nova
da casa do Senhor, aos ouvidos de todo o povo. E, ouvindo
Micaías, filho de Gemarias, filho de Safã, todas as palavras do
Senhor, daquele livro, desceu à casa do rei, à câmara do
escriba. E eis que todos os príncipes estavam ali assentados, a
saber: Elisama, o escriba, e Delaías, filho de Semaías, e Elnatã,
filho de Acbor, e Gemarias, filho de Safã, e Zedequias, filho de
Hananias, e todos os outros príncipes. E Micaías
anunciou-lhes todas as palavras que ouvira, quando Baruque leu o
livro, aos ouvidos do povo. Então todos os príncipes mandaram
Jeudi, filho de Netanias, filho de Selemias, filho de Cusi, a
Baruque, para lhe dizer: O rolo que leste aos ouvidos do povo,
toma-o na tua mão, e vem. E Baruque, filho de Nerias, tomou o rolo
na sua mão, e foi ter com eles. E disseram-lhe: Assenta-te
agora, e lê-o aos nossos ouvidos. E leu Baruque aos ouvidos deles.
E sucedeu que, ouvindo eles todas aquelas palavras,
voltaram-se temerosos uns para os outros, e disseram a Baruque: Sem
dúvida alguma anunciaremos ao rei todas estas palavras. E
perguntaram a Baruque, dizendo: Declara-nos agora como escreveste da
sua boca todas estas palavras. E disse-lhes Baruque: Da sua
boca ele me ditava todas estas palavras, e eu com tinta as escrevia
no livro. Então disseram os príncipes a Baruque: Vai,
esconde-te, tu e Jeremias, e ninguém saiba onde estais.

E foram ter com o rei ao átrio: mas depositaram o rolo na câmara
de Elisama, o escriba, e anunciaram aos ouvidos do rei todas aquelas
palavras. Então enviou o rei a Jeudi, para que tomasse o
rolo; e Jeudi tomou-o da câmara de Elisama, o escriba, e leu-o aos
ouvidos do rei e aos ouvidos de todos os príncipes que estavam em
torno do rei. Ora, o rei estava assentado na casa de inverno,
pelo nono mês; e diante dele estava um braseiro aceso. E
sucedeu que, tendo Jeudi lido três ou quatro folhas, cortou-as com
um canivete de escrivão, e lançou-as no fogo que havia no braseiro,
até que todo o rolo se consumiu no fogo que estava sobre o braseiro.
E não temeram, nem rasgaram as suas vestes, nem o rei, nem
nenhum dos seus servos que ouviram todas aquelas palavras. E,
posto que Elnatã, e Delaías, e Gemarias tivessem rogado ao rei que
não queimasse o rolo, ele não lhes deu ouvidos. Antes deu
ordem o rei a Jerameel, filho de Hamaleque, e a Seraías, filho de
Azriel, e a Selemias, filho de Abdeel, que prendessem a Baruque, o
escrivão, e a Jeremias, o profeta; mas o Senhor os escondera.
Então veio a Jeremias a palavra do Senhor, depois que o rei
queimara o rolo, com as palavras que Baruque escrevera da boca de
Jeremias, dizendo: Toma ainda outro rolo, e escreve nele
todas aquelas palavras que estavam no primeiro rolo, que queimou
Jeoiaquim, rei de Judá. E a Jeoiaquim, rei de Judá, dirás:
Assim diz o Senhor: Tu queimaste este rolo, dizendo: Por que
escreveste nele, dizendo: Certamente virá o rei de Babilônia, e
destruirá esta terra e fará cessar nela homens e animais?
Portanto assim diz o Senhor, acerca de Jeoiaquim, rei de
Judá: Não terá quem se assente sobre o trono de Davi, e será lançado
o seu cadáver ao calor do dia, e à geada da noite. E
castigarei a sua iniqüidade nele, e na sua descendência, e nos seus
servos; e trarei sobre ele e sobre os moradores de Jerusalém, e
sobre os homens de Judá, todo aquele mal que lhes tenho falado, e
não ouviram. Tomou, pois, Jeremias outro rolo, e deu-o a
Baruque, filho de Nerias, o escrivão, o qual escreveu nele, da boca
de Jeremias, todas as palavras do livro que Jeoiaquim, rei de Judá,
tinha queimado no fogo; e ainda se lhes acrescentaram muitas
palavras semelhantes.

\medskip

\lettrine{37} E Zedequias, filho de Josias, a quem
Nabucodonosor, rei de Babilônia, constituiu rei na terra de Judá,
reinou em lugar de Conias, filho de Jeoiaquim. Mas nem ele, nem
os seus servos, nem o povo da terra deram ouvidos às palavras do
Senhor que falou pelo ministério de Jeremias, o profeta. Contudo
mandou o rei Zedequias a Jucal, filho de Selemias, e a Sofonias,
filho de Maaséias, o sacerdote, ao profeta Jeremias, para lhe dizer:
Roga agora por nós ao Senhor nosso Deus. E entrava e saía
Jeremias entre o povo, porque não o tinham posto na prisão. E o
exército de Faraó saíra do Egito; e quando os caldeus, que tinham
sitiado Jerusalém, ouviram esta notícia, retiraram-se de Jerusalém.
Então veio a Jeremias, o profeta, a palavra do Senhor, dizendo:
Assim diz o Senhor, Deus de Israel: Assim direis ao rei de Judá,
que vos enviou a mim para me consultar: Eis que o exército de Faraó,
que saiu em vosso socorro, voltará para a sua terra no Egito. E
voltarão os caldeus, e pelejarão contra esta cidade, e a tomarão, e
a queimarão a fogo. Assim diz o Senhor: Não enganeis as vossas
almas, dizendo: Sem dúvida se retirarão os caldeus de nós, pois não
se retirarão. Porque ainda que ferísseis a todo o exército
dos caldeus, que peleja contra vós, e só ficassem deles homens
feridos, cada um levantar-se-ia na sua tenda, e queimaria a fogo
esta cidade.

E sucedeu que, subindo de Jerusalém o exército dos caldeus, por
causa do exército de Faraó, saiu Jeremias de Jerusalém, a fim
de ir à terra de Benjamim, para dali se separar no meio do povo.
Mas, estando ele à porta de Benjamim, achava-se ali um
capitão da guarda, cujo nome era Jerias, filho de Selemias, filho de
Hananias, o qual prendeu a Jeremias, o profeta, dizendo: Tu foges
para os caldeus. E Jeremias disse: Isso é falso, não fujo
para os caldeus. Mas ele não lhe deu ouvidos; e assim Jerias prendeu
a Jeremias, e o levou aos príncipes. E os príncipes se iraram
muito contra Jeremias, e o feriram; e puseram-no na prisão, na casa
de Jônatas, o escrivão; porque a tinham transformado em cárcere.
Entrando, pois, Jeremias nas celas do calabouço, ali ficou
muitos dias. E mandou o rei Zedequias soltá-lo; e o rei lhe
perguntou em sua casa, em segredo: Há porventura alguma palavra do
Senhor? E disse Jeremias: Há. E disse ainda: Na mão do rei de
Babilônia serás entregue. Disse mais Jeremias ao rei
Zedequias: Em que tenho pecado contra ti, e contra os teus servos, e
contra este povo, para que me pusésseis na prisão? Onde estão
agora os vossos profetas, que vos profetizavam, dizendo: O rei de
Babilônia não virá contra vós nem contra esta terra? Ora,
pois, ouve agora, ó rei meu senhor: Seja aceita agora a minha
súplica diante de ti, e não me deixes tornar à casa de Jônatas, o
escriba, para que eu não venha a morrer ali. Então ordenou o
rei Zedequias que pusessem a Jeremias no átrio da guarda; e
deram-lhe um pão cada dia, da rua dos padeiros, até que se acabou
todo o pão da cidade; assim ficou Jeremias no átrio da guarda.

\medskip

\lettrine{38} Ouviram, pois, Sefatias, filho de Matã, e
Gedalias, filho de Pasur, e Jucal, filho de Selemias, e Pasur, filho
de Malquias, as palavras que anunciava Jeremias a todo o povo,
dizendo: Assim diz o Senhor: O que ficar nesta cidade morrerá à
espada, de fome e de pestilência; mas o que sair aos caldeus viverá;
porque a sua alma lhe será por despojo, e viverá. Assim diz o
Senhor: Esta cidade infalivelmente será entregue na mão do exército
do rei de Babilônia, e ele a tomará. E disseram os príncipes ao
rei: Morra este homem, visto que ele assim enfraquece as mãos dos
homens de guerra que restam nesta cidade, e as mãos de todo o povo,
dizendo-lhes tais palavras; porque este homem não busca a paz para
este povo, porém o mal. E disse o rei Zedequias: Eis que ele
está na vossa mão; porque o rei nada pode fazer contra vós.
Então tomaram a Jeremias, e o lançaram na cisterna de Malquias,
filho do rei, que estava no átrio da guarda; e desceram a Jeremias
com cordas; mas na cisterna não havia água, senão lama; e atolou-se
Jeremias na lama. E, ouvindo Ebede-Meleque, o etíope, um eunuco
que então estava na casa do rei, que tinham posto a Jeremias na
cisterna (estava, porém, o rei assentado à porta de Benjamim),
logo Ebede-Meleque saiu da casa do rei, e falou ao rei, dizendo:
Ó rei, senhor meu, estes homens agiram mal em tudo quanto
fizeram a Jeremias, o profeta, lançando-o na cisterna; de certo
morrerá de fome no lugar onde se acha, pois não há mais pão na
cidade. Então deu ordem o rei a Ebede-Meleque, o etíope,
dizendo: Toma contigo daqui trinta homens, e tira a Jeremias, o
profeta, da cisterna, antes que morra. E tomou Ebede-Meleque
os homens consigo, e foi à casa do rei, por debaixo da tesouraria, e
tomou dali uns trapos velhos e rotos, e roupas velhas, e desceu-os a
Jeremias na cisterna por meio de cordas. E disse
Ebede-Meleque, o etíope, a Jeremias: Põe agora estes trapos velhos e
rotos, já apodrecidos, nas axilas, calçando as cordas. E Jeremias
assim o fez. E puxaram a Jeremias com as cordas, e o alçaram
da cisterna; e ficou Jeremias no átrio da guarda.

Então o rei Zedequias mandou trazer à sua presença Jeremias, o
profeta, à terceira entrada da casa do Senhor; e disse o rei a
Jeremias: Pergunto-te uma coisa, não me encubras nada. E
disse Jeremias a Zedequias: Se eu te declarar, porventura não me
matarás? E se eu te aconselhar, não me ouvirás? Então jurou o
rei Zedequias a Jeremias, em segredo, dizendo: Vive o Senhor, que
nos fez esta alma, que não te matarei nem te entregarei na mão
destes homens que procuram a tua morte. Então Jeremias disse
a Zedequias: Assim diz o Senhor, Deus dos Exércitos, Deus de Israel:
Se voluntariamente saíres aos príncipes do rei de Babilônia, então
viverá a tua alma, e esta cidade não se queimará a fogo, e viverás
tu e a tua casa. Mas, se não saíres aos príncipes do rei de
Babilônia, então será entregue esta cidade na mão dos caldeus, e
queimá-la-ão a fogo, e tu não escaparás da mão deles. E disse
o rei Zedequias a Jeremias: Receio-me dos judeus, que se passaram
para os caldeus; que estes me entreguem na mão deles, e escarneçam
de mim. E disse Jeremias: Não te entregarão; ouve, peço-te, a
voz do Senhor, conforme a qual eu te falo; e bem te irá, e viverá a
tua alma. Mas, se tu não quiseres sair, esta é a palavra que
me mostrou o Senhor: Eis que todas as mulheres que ficaram na
casa do rei de Judá serão levadas aos príncipes do rei de Babilônia,
e elas mesmas dirão: Teus pacificadores te incitaram e prevaleceram
contra ti, mas agora que se atolaram os teus pés na lama, voltaram
atrás. Assim que a todas as tuas mulheres e a teus filhos
levarão aos caldeus, e nem tu escaparás da sua mão, antes pela mão
do rei de Babilônia serás preso, e esta cidade será queimada a fogo.
Então disse Zedequias a Jeremias: Ninguém saiba estas
palavras, e não morrerás. E quando os príncipes, ouvindo que
falei contigo, vierem a ti, e te disserem: Declara-nos agora o que
disseste ao rei e o que ele te disse, não no-lo encubras, e não te
mataremos; então lhes dirás: Eu lancei a minha súplica diante
do rei, que não me fizesse tornar à casa de Jônatas, para morrer
ali. Vindo, pois, todos os príncipes a Jeremias, e
interrogando-o, declarou-lhes todas as palavras que o rei lhe havia
ordenado; e calados o deixaram, porque o assunto não foi revelado.
E ficou Jeremias no átrio da guarda, até o dia em que
Jerusalém foi tomada, e ainda ali estava quando Jerusalém foi
tomada.

\medskip

\lettrine{39} No ano nono de Zedequias, rei de Judá, no décimo
mês, veio Nabucodonosor, rei de Babilônia, e todo o seu exército,
contra Jerusalém, e a cercaram. No ano undécimo de Zedequias, no
quarto mês, aos nove do mês, fez-se uma brecha na cidade.
Entraram nela todos os príncipes do rei de Babilônia, e pararam
na porta do meio, a saber: Nergal-Sarezer, Sangar-Nebo, Sarsequim,
Rabe-Saris, Nergal-Sarezer, Rabe-Mague, e todos os outros príncipes
do rei de Babilônia. E sucedeu que, vendo-os Zedequias, rei de
Judá, e todos os homens de guerra, fugiram, saindo de noite da
cidade, pelo caminho do jardim do rei, pela porta que está entre os
dois muros; e seguiram pelo caminho da campina. Mas o exército
dos caldeus os perseguiu, e alcançou a Zedequias nas campinas de
Jericó; e eles o prenderam, e fizeram-no subir a Nabucodonosor, rei
de Babilônia, a Ribla, na terra de Hamate, e o rei o sentenciou.
E o rei de Babilônia matou em Ribla os filhos de Zedequias,
diante dos seus olhos; também matou o rei de Babilônia a todos os
nobres de Judá. E cegou os olhos de Zedequias, e o atou com duas
cadeias de bronze, para levá-lo a Babilônia. E os caldeus
incendiaram a casa do rei e as casas do povo, e derrubaram os muros
de Jerusalém. E o restante do povo, que ficou na cidade, e os
desertores que se tinham passado para ele, e o restante do povo que
ficou, Nebuzaradã, capitão da guarda, levou cativo para a Babilônia.
Porém os pobres dentre o povo, que não tinham nada,
Nebuzaradã, capitão da guarda, deixou na terra de Judá; e deu-lhes
vinhas e campos naquele dia.

Mas Nabucodonosor, rei de Babilônia, havia ordenado acerca de
Jeremias, a Nebuzaradã, capitão da guarda, dizendo: Toma-o, e
põe sobre ele os teus olhos, e não lhe faças nenhum mal; antes como
ele te disser, assim procederás com ele. Por isso mandou
Nebuzaradã, capitão da guarda, e Nebusazbã, Rabe-Saris,
Nergal-Sarezer, Rabe-Mague, e todos os príncipes do rei de
Babilônia, mandaram retirar a Jeremias do átrio da guarda, e
o entregaram a Gedalias, filho de Aicão, filho de Safã, para que o
levassem à casa; e ele habitou entre o povo. Ora, tinha vindo
a Jeremias a palavra do Senhor, estando ele ainda encarcerado no
átrio da guarda, dizendo: Vai, e fala a Ebede-Meleque, o
etíope, dizendo: Assim diz o Senhor dos Exércitos, Deus de Israel:
Eis que eu trarei as minhas palavras sobre esta cidade para mal e
não para bem; e cumprir-se-ão diante de ti naquele dia. A ti,
porém, eu livrarei naquele dia, diz o Senhor, e não serás entregue
na mão dos homens, a quem temes. Porque certamente te
livrarei, e não cairás à espada; mas a tua alma terás por despojo,
porquanto confiaste em mim, diz o Senhor.

\medskip

\lettrine{40} A palavra que veio a Jeremias da parte do
Senhor, depois que Nebuzaradã, capitão da guarda, o deixara ir de
Ramá, quando o tomou, estando ele atado com cadeias no meio de todos
os do cativeiro de Jerusalém e de Judá, que foram levados cativos
para Babilônia. Tomou o capitão da guarda a Jeremias, e
disse-lhe: O Senhor teu Deus pronunciou este mal, contra este lugar.
E o Senhor o trouxe, e fez como havia falado; porque pecastes
contra o Senhor, e não obedecestes à sua voz, portanto vos sucedeu
isto. Agora, pois, eis que te soltei hoje das cadeias que
estavam sobre as tuas mãos. Se te apraz vir comigo para Babilônia,
vem, e eu cuidarei de ti, mas se não te apraz vir comigo para
Babilônia, deixa de vir. Olha, toda a terra está diante de ti; para
onde parecer bom e reto aos teus olhos ir, para ali vai. Mas,
como ele ainda não tinha voltado, disse-lhe: Volta a Gedalias, filho
de Aicão, filho de Safã, a quem o rei de Babilônia pôs sobre as
cidades de Judá, e habita com ele no meio do povo; ou se para
qualquer outra parte te aprouver ir, vai. E deu-lhe o capitão da
guarda sustento para o caminho, e um presente, e o deixou ir.
Assim veio Jeremias a Gedalias, filho de Aicão, a Mizpá; e
habitou com ele no meio do povo que havia ficado na terra.

Ouvindo, pois, todos os capitães dos exércitos, que estavam no
campo, eles e os seus homens, que o rei de Babilônia tinha nomeado a
Gedalias, filho de Aicão, governador da terra, e que lhe havia
confiado os homens, e as mulheres, e os meninos, e os mais pobres da
terra, que não foram levados cativos a Babilônia, vieram ter com
Gedalias, a Mizpá; a saber: Ismael, filho de Netanias, e Joanã e
Jônatas, filhos de Careá, e Seraías, filho de Tanumete, e os filhos
de Efai, o netofatita, e Jezanias, filho de um maacatita, eles e os
seus homens. E jurou Gedalias, filho de Aicão, filho de Safã, a
eles e aos seus homens, dizendo: Não temais servir aos caldeus;
ficai na terra, e servi o rei de Babilônia, e bem vos irá.
Quanto a mim, eis que habito em Mizpá, para estar às ordens
dos caldeus que vierem a nós; e vós recolhei o vinho, e as frutas de
verão, e o azeite, e colocai-os nos vossos vasos, e habitai nas
vossas cidades, que tomastes. Do mesmo modo todos os judeus
que estavam em Moabe, e entre os filhos de Amom, e em Edom, e os que
havia em todas aquelas terras, ouviram que o rei de Babilônia havia
deixado alguns em Judá, e que havia posto sobre eles a Gedalias,
filho de Aicão, filho de Safã, então voltaram todos os judeus
de todos os lugares, para onde foram lançados, e vieram à terra de
Judá, a Gedalias, a Mizpá; e recolheram vinho e frutas do verão com
muita abundância. Joanã, filho de Careá, e todos os capitães
dos exércitos, que estavam no campo, vieram a Gedalias, a Mizpá.
E disseram-lhe: Bem sabes que Baalis, rei dos filhos de Amom,
enviou a Ismael, filho de Netanias, para tirar-te a vida. Mas,
Gedalias, filho de Aicão, não lhes deu crédito. Todavia
Joanã, filho de Careá, falou a Gedalias em segredo, em Mizpá,
dizendo: Irei agora, e ferirei a Ismael, filho de Netanias, sem que
ninguém o saiba; por que razão te tiraria ele a vida, de modo que
todos os judeus, que se têm congregado a ti, fossem dispersos, e
perecesse o restante de Judá? Mas disse Gedalias, filho de
Aicão, a Joanã, filho de Careá: Não faças tal coisa; porque falas
falsamente contra Ismael.

\medskip

\lettrine{41} Sucedeu, porém, no mês sétimo, que veio Ismael,
filho de Netanias, filho de Elisama, de sangue real, e com ele dez
homens, príncipes do rei, a Gedalias, filho de Aicão, a Mizpá; e
comeram pão juntos ali em Mizpá. E levantou-se Ismael, filho de
Netanias, com os dez homens que estavam com ele, e feriram à espada
a Gedalias, filho de Aicão, filho de Safã, matando assim aquele que
o rei de Babilônia havia posto por governador sobre a terra.
Também matou Ismael a todos os judeus que com ele, com Gedalias,
estavam em Mizpá, como também aos caldeus, homens de guerra, que se
achavam ali. Sucedeu, pois, no dia seguinte, depois que ele
matara a Gedalias, sem ninguém o saber, que vieram homens de
Siquém, de Siló, e de Samaria; oitenta homens, com a barba rapada, e
as vestes rasgadas, e retalhando-se; e trazendo nas suas mãos
ofertas e incenso, para levarem à casa do Senhor. E, saindo-lhes
ao encontro Ismael, filho de Netanias, desde Mizpá, ia chorando; e
sucedeu que, encontrando-os lhes disse: Vinde a Gedalias, filho de
Aicão. Sucedeu, porém, que, entrando eles até ao meio da cidade,
matou-os Ismael, filho de Netanias, e os lançou num poço, ele e os
homens que estavam com ele. Mas houve entre eles dez homens que
disseram a Ismael: Não nos mates, porque temos, no campo, tesouros,
trigo, cevada, azeite e mel. E ele por isso os deixou, e não os
matou entre seus irmãos. E o poço em que Ismael lançou todos os
cadáveres dos homens que matou por causa de Gedalias é o mesmo que
fez o rei Asa, por causa de Baasa, rei de Israel; foi esse mesmo que
Ismael, filho de Netanias, encheu de mortos. E Ismael levou
cativo a todo o restante do povo que estava em Mizpá, isto é, as
filhas do rei, e todo o povo que ficara em Mizpá, que Nebuzaradã,
capitão da guarda, havia confiado a Gedalias, filho de Aicão; e
levou-os cativos Ismael, filho de Netanias, e se foi para passar aos
filhos de Amom.

Ouvindo, pois, Joanã, filho de Careá, e todos os capitães dos
exércitos que estavam com ele, todo o mal que havia feito Ismael,
filho de Netanias, tomaram todos os seus homens, e foram
pelejar contra Ismael, filho de Netanias; e acharam-no ao pé das
grandes águas que há em Gibeom. E aconteceu que, vendo todo o
povo, que estava com Ismael, a Joanã, filho de Careá, e a todos os
capitães dos exércitos, que vinham com ele, se alegrou. E
todo o povo que Ismael levara cativo de Mizpá virou as costas, e
voltou, e foi para Joanã, filho de Careá. Mas Ismael, filho
de Netanias, escapou com oito homens de diante de Joanã, e se foi
para os filhos de Amom. Então tomou Joanã, filho de Careá, e
todos os capitães dos exércitos que estavam com ele, a todo o
restante do povo que ele havia recobrado de Ismael, filho de
Netanias, desde Mizpá, depois de haver matado a Gedalias, filho de
Aicão, isto é, aos homens poderosos de guerra, e às mulheres, e aos
meninos, e aos eunucos que havia recobrado de Gibeom. E
partiram, indo habitar em Gerute-Quimã, que está perto de Belém,
para dali irem e entrarem no Egito, por causa dos caldeus;
porque os temiam, por ter Ismael, filho de Netanias, matado a
Gedalias, filho de Aicão, a quem o rei de Babilônia tinha feito
governador sobre a terra.

\medskip

\lettrine{42} Então chegaram todos os capitães dos exércitos,
e Joanã, filho de Careá, e Jezanias, filho de Hosaías, e todo o
povo, desde o menor até ao maior, e disseram a Jeremias, o
profeta: Aceita agora a nossa súplica diante de ti, e roga ao Senhor
teu Deus, por nós e por todo este remanescente; porque de muitos
restamos uns poucos, como nos vêem os teus olhos; para que o
Senhor teu Deus nos ensine o caminho por onde havemos de andar e
aquilo que havemos de fazer. E disse-lhes Jeremias, o profeta:
Eu vos tenho ouvido; eis que orarei ao Senhor vosso Deus conforme as
vossas palavras; e seja o que for que o Senhor vos responder eu
vo-lo declararei; não vos ocultarei uma só palavra. Então eles
disseram a Jeremias: Seja o Senhor entre nós testemunha verdadeira e
fiel, se não fizermos conforme toda a palavra com que te enviar a
nós o Senhor teu Deus. Seja ela boa, ou seja má, à voz do Senhor
nosso Deus, a quem te enviamos, obedeceremos, para que nos suceda
bem, obedecendo à voz do Senhor nosso Deus.

E sucedeu que ao fim de dez dias veio a palavra do Senhor a
Jeremias. Então chamou a Joanã, filho de Careá, e a todos os
capitães dos exércitos, que havia com ele, e a todo o povo, desde o
menor até ao maior, e disse-lhes: Assim diz o Senhor, Deus de
Israel, a quem me enviastes, para apresentar a vossa súplica diante
dele: Se de boa mente ficardes nesta terra, então vos
edificarei, e não vos derrubarei; e vos plantarei, e não vos
arrancarei; porque estou arrependido do mal que vos tenho feito.
Não temais o rei de Babilônia, a quem vós temeis; não o
temais, diz o Senhor, porque eu sou convosco, para vos salvar e para
vos livrar da sua mão. E vos concederei misericórdia, para
que ele tenha misericórdia de vós, e vos faça voltar à vossa terra.
Mas se vós disserdes: Não ficaremos nesta terra, não
obedecendo à voz do Senhor vosso Deus, dizendo: Não, antes
iremos à terra do Egito, onde não veremos guerra, nem ouviremos som
de trombeta, nem teremos fome de pão, e ali ficaremos, nesse
caso ouvi a palavra do Senhor, ó remanescente de Judá: Assim diz o
Senhor dos Exércitos, Deus de Israel: Se vós absolutamente
propuserdes a entrar no Egito, e entrardes para lá habitar,
acontecerá que a espada que vós temeis vos alcançará ali na
terra do Egito, e a fome que vós receais vos seguirá de perto no
Egito, e ali morrereis. Assim será com todos os homens que
puseram os seus rostos para entrarem no Egito, a fim de lá
habitarem: morrerão à espada, e de fome, e de peste; e deles não
haverá quem reste e escape do mal que eu farei vir sobre eles.
Porque assim diz o Senhor dos Exércitos, Deus de Israel: Como
se derramou a minha ira e a minha indignação sobre os habitantes de
Jerusalém, assim se derramará a minha indignação sobre vós, quando
entrardes no Egito; e sereis objeto de maldição, e de espanto, e de
execração, e de opróbrio, e não vereis mais este lugar. Falou
o Senhor acerca de vós, ó remanescente de Judá! Não entreis no
Egito; tende por certo que hoje testifiquei contra vós.
Porque vos enganastes a vós mesmos, pois me enviastes ao
Senhor vosso Deus, dizendo: Ora por nós ao Senhor nosso Deus; e
conforme tudo o que disser o Senhor nosso Deus, declara-no-lo assim,
e o faremos. E vo-lo tenho declarado hoje; mas não destes
ouvidos à voz do Senhor vosso Deus, em coisa alguma pela qual ele me
enviou a vós. Agora, pois, sabei por certo que morrereis à
espada, de fome e de peste no mesmo lugar onde desejais ir, para lá
morardes.

\medskip

\lettrine{43} E sucedeu que, acabando Jeremias de falar a todo
o povo todas as palavras do Senhor seu Deus, com as quais o Senhor
seu Deus lho havia enviado, para que lhes dissesse todas estas
palavras, então falaram Azarias, filho de Hosaías, e Joanã,
filho de Careá, e todos os homens soberbos, dizendo a Jeremias: Tu
dizes mentiras; o Senhor nosso Deus não te enviou a dizer: Não
entreis no Egito, para ali habitar; mas Baruque, filho de
Nerias, te incita contra nós, para entregar-nos na mão dos caldeus,
para nos matarem, ou para nos levarem cativos para Babilônia.
Não obedeceu, pois, Joanã, filho de Careá, nem nenhum de todos
os capitães dos exércitos, nem o povo todo, à voz do Senhor, para
ficarem na terra de Judá. Antes tomou Joanã, filho de Careá, e
todos os capitães dos exércitos a todo o restante de Judá, que havia
voltado dentre todas as nações, para onde haviam sido lançados, para
morarem na terra de Judá; aos homens, e às mulheres, e aos
meninos, e às filhas do rei, e a toda a alma que Nebuzaradã, capitão
da guarda, deixara com Gedalias, filho de Aicão, filho de Safã; como
também a Jeremias, o profeta, e a Baruque, filho de Nerias; e
entraram na terra do Egito, porque não obedeceram à voz do Senhor; e
vieram até Tafnes.

Então veio a palavra do Senhor a Jeremias, em Tafnes, dizendo:
Toma na tua mão pedras grandes, e esconde-as no barro, no forno
que está à entrada da casa de Faraó, em Tafnes, perante os olhos dos
homens de Judá, e dize-lhes: Assim diz o Senhor dos
Exércitos, Deus de Israel: Eis que eu enviarei, e tomarei a
Nabucodonosor, rei de Babilônia, meu servo, e porei o seu trono
sobre estas pedras que escondi; e ele estenderá a sua tenda real
sobre elas. E virá, e ferirá a terra do Egito; entregando
para a morte, quem é para a morte; e quem é para o cativeiro, para o
cativeiro; e quem é para a espada, para a espada. E lançarei
fogo às casas dos deuses do Egito, e queimá-los-á, e levá-los-á
cativos; e vestir-se-á da terra do Egito, como veste o pastor a sua
roupa, e sairá dali em paz. E quebrará as estátuas de
Bete-Semes, que está na terra do Egito; e as casas dos deuses do
Egito queimará a fogo.

\medskip

\lettrine{44} A palavra que veio a Jeremias, acerca de todos
os judeus, habitantes da terra do Egito, que habitavam em Migdol, e
em Tafnes, e em Nofe, e na terra de Patros, dizendo: Assim diz o
Senhor dos Exércitos, Deus de Israel: Vós vistes todo o mal que fiz
vir sobre Jerusalém, e sobre todas as cidades de Judá; e eis que
elas são hoje uma desolação, e ninguém habita nelas; por causa
da maldade que fizeram, para me irarem, indo queimar incenso, e
servir a deuses estranhos, que nunca conheceram, nem eles, nem vós,
nem vossos pais. E eu vos enviei todos os meus servos, os
profetas, madrugando e enviando a dizer: Ora, não façais esta coisa
abominável que odeio. Mas eles não escutaram, nem inclinaram os
seus ouvidos, para se converterem da sua maldade, para não queimarem
incenso a outros deuses. Derramou-se, pois, a minha indignação e
a minha ira, e acendeu-se nas cidades de Judá, e nas ruas de
Jerusalém, e elas tornaram-se em deserto e em desolação, como hoje
se vê. Agora, pois, assim diz o Senhor, Deus dos Exércitos, Deus
de Israel: Por que fazeis vós tão grande mal contra as vossas almas,
para vos desarraigardes, ao homem e à mulher, à criança e ao que
mama, do meio de Judá, a fim de não deixardes remanescente algum;
irando-me com as obras de vossas mãos, queimando incenso a
deuses estranhos na terra do Egito, aonde vós entrastes para lá
habitar; para que a vós mesmos vos desarraigueis, e para que sirvais
de maldição, e de opróbrio entre todas as nações da terra?
Esquecestes já as maldades de vossos pais, e as maldades dos
reis de Judá, e as maldades de suas mulheres, e as vossas maldades,
e as maldades de vossas mulheres, que cometeram na terra de Judá, e
nas ruas de Jerusalém? Não se humilharam até ao dia de hoje,
nem temeram, nem andaram na minha lei, nem nos meus estatutos, que
pus diante de vós e diante de vossos pais. Portanto assim diz
o Senhor dos Exércitos, Deus de Israel: Eis que eu ponho o meu rosto
contra vós para mal, e para desarraigar a todo o Judá. E
tomarei os que restam de Judá, os quais puseram os seus rostos para
entrarem na terra do Egito, para lá habitar e todos eles serão
consumidos na terra do Egito; cairão à espada, e de fome morrerão;
consumir-se-ão, desde o menor até ao maior; à espada e de fome
morrerão; e servirão de execração, e de espanto, e de maldição, e de
opróbrio. Porque castigarei os que habitam na terra do Egito,
como castiguei Jerusalém, com a espada, com a fome e com a peste.
De maneira que da parte remanescente de Judá, que entrou na
terra do Egito, para lá habitar, não haverá quem escape e fique para
tornar à terra de Judá, à qual eles suspiram voltar para nela morar;
porém não tornarão senão uns fugitivos.

Então responderam a Jeremias todos os homens que sabiam que suas
mulheres queimavam incenso a deuses estranhos, e todas as mulheres
que estavam presentes em grande multidão, como também todo o povo
que habitava na terra do Egito, em Patros, dizendo: Quanto à
palavra que nos anunciaste em nome do Senhor, não obedeceremos a ti;
mas certamente cumpriremos toda a palavra que saiu da nossa
boca, queimando incenso à rainha dos céus, e oferecendo-lhe
libações, como nós e nossos pais, nossos reis e nossos príncipes,
temos feito, nas cidades de Judá, e nas ruas de Jerusalém; e então
tínhamos fartura de pão, e andávamos alegres, e não víamos mal
algum. Mas desde que cessamos de queimar incenso à rainha dos
céus, e de lhe oferecer libações, tivemos falta de tudo, e fomos
consumidos pela espada e pela fome. E quando nós queimávamos
incenso à rainha dos céus, e lhe oferecíamos libações, acaso lhe
fizemos bolos, para a adorar, e oferecemos-lhe libações sem nossos
maridos?

Então disse Jeremias a todo o povo, aos homens e às mulheres, e a
todo o povo que lhe havia dado esta resposta, dizendo:
Porventura não se lembrou o Senhor, e não lhe veio ao coração
o incenso que queimastes nas cidades de Judá e nas ruas de
Jerusalém, vós e vossos pais, vossos reis e vossos príncipes, como
também o povo da terra? De maneira que o Senhor não podia por
mais tempo sofrer a maldade das vossas ações, as abominações que
cometestes; por isso se tornou a vossa terra em desolação, e em
espanto, e em maldição, sem habitantes, como hoje se vê.
Porque queimastes incenso, e porque pecastes contra o Senhor,
e não obedecestes à voz do Senhor, e na sua lei, e nos seus
testemunhos não andastes, por isso vos sucedeu este mal, como se vê
neste dia. Disse mais Jeremias a todo o povo e a todas as
mulheres: Ouvi a palavra do Senhor, vós, todo o Judá, que estais na
terra do Egito. Assim fala o Senhor dos Exércitos, Deus de
Israel, dizendo: Vós e vossas mulheres não somente falastes por
vossa boca, senão também o cumpristes por vossas mãos, dizendo:
Certamente cumpriremos os nossos votos que fizemos de queimar
incenso à rainha dos céus e de lhe oferecer libações; confirmai,
pois, os vossos votos, e perfeitamente cumpri-os. Portanto
ouvi a palavra do Senhor, todo o Judá, que habitais na terra do
Egito: Eis que eu juro pelo meu grande nome, diz o Senhor, que nunca
mais será pronunciado o meu nome pela boca de nenhum homem de Judá
em toda a terra do Egito dizendo: Vive o Senhor Deus! Eis que
velarei sobre eles para mal, e não para bem; e serão consumidos
todos os homens de Judá, que estão na terra do Egito, pela espada e
pela fome, até que de todo se acabem. E os que escaparem da
espada voltarão da terra do Egito à terra de Judá, poucos em número;
e todo o restante de Judá, que entrou na terra do Egito, para
habitar ali, saberá se subsistirá a minha palavra ou a sua. E
isto vos servirá de sinal, diz o Senhor, que eu vos castigarei neste
lugar, para que saibais que certamente subsistirão as minhas
palavras contra vós para mal. Assim diz o Senhor: Eis que eu
darei Faraó-Hofra, rei do Egito, na mão de seus inimigos, e na mão
dos que procuram a sua morte; como entreguei Zedequias, rei de Judá,
na mão de Nabucodonosor, rei de Babilônia, seu inimigo, e que
procurava a sua morte.

\medskip

\lettrine{45} A palavra que Jeremias, o profeta, falou a
Baruque, filho de Nerias, quando este escrevia, num livro, estas
palavras, da boca de Jeremias, no ano quarto de Jeoiaquim, filho de
Josias, rei de Judá, dizendo: Assim diz o Senhor, Deus de
Israel, acerca de ti, ó Baruque: Disseste: Ai de mim agora,
porque me acrescentou o Senhor tristeza sobre minha dor! Estou
cansado do meu gemido, e não acho descanso. Assim lhe dirás:
Isto diz o Senhor: Eis que o que edifiquei eu derrubo, e o que
plantei eu arranco, e isso em toda esta terra. E procuras tu
grandezas para ti mesmo? Não as procures; porque eis que trarei mal
sobre toda a carne, diz o Senhor; porém te darei a tua alma por
despojo, em todos os lugares para onde fores.

\medskip

\lettrine{46} A palavra do Senhor, que veio a Jeremias, o
profeta, contra os gentios, acerca do Egito, contra o exército
de Faraó-Neco, rei do Egito, que estava junto ao rio Eufrates em
Carquemis, ao qual feriu Nabucodonosor, rei de Babilônia, no ano
quarto de Jeoiaquim, filho de Josias, rei de Judá. Preparai o
escudo e o pavês, e chegai-vos para a peleja. Selai os cavalos e
montai, cavaleiros, e apresentai-vos com elmos; limpai as lanças,
vesti-vos de couraças. Por que razão vejo os medrosos voltando
as costas? Os seus valentes estão abatidos, e vão fugindo, sem
olharem para trás; terror há ao redor, diz o Senhor. Não fuja o
ligeiro, e não escape o valente; para o lado norte, junto à borda do
rio Eufrates tropeçaram e caíram. Quem é este que vem subindo
como o Nilo, cujas águas se movem como os rios? O Egito vem
subindo como o Nilo, e como rios cujas águas se movem; e disse:
Subirei, cobrirei a terra, destruirei a cidade, e os que nela
habitam. Subi, ó cavalos, e estrondeai, ó carros, e saiam os
valentes; os etíopes, e os do Líbano, que manejam o escudo, e os
lídios, que manejam e entesam o arco. Porque este dia é o dia
do Senhor Deus dos Exércitos, dia de vingança para ele se vingar dos
seus adversários; e a espada devorará, e fartar-se-á, e
embriagar-se-á com o sangue deles; porque o Senhor Deus dos
Exércitos tem um sacrifício na terra do norte, junto ao rio
Eufrates. Sobe a Gileade, e toma bálsamo, ó virgem filha do
Egito; debalde multiplicas remédios, pois já não há cura para ti.

As nações ouviram a tua vergonha, e a terra está cheia do teu
clamor; porque o valente tropeçou com o valente e ambos caíram
juntos. A palavra que falou o Senhor a Jeremias, o profeta,
acerca da vinda de Nabucodonosor, rei de Babilônia, para ferir a
terra do Egito. Anunciai no Egito, e fazei ouvir isto em
Migdol; fazei também ouvi-lo em Nofe, e em Tafnes, dizei:
Apresenta-te, e prepara-te; porque a espada já devorou o que está ao
redor de ti. Por que foram derrubados os teus valentes? Não
puderam manter-se firmes, porque o Senhor os abateu.
Multiplicou os que tropeçavam; também caíram uns sobre os
outros, e disseram: Levanta-te, e voltemos ao nosso povo, e à terra
do nosso nascimento, por causa da espada que oprime. Clamaram
ali: Faraó rei do Egito é apenas um barulho; deixou passar o tempo
assinalado. Vivo eu, diz o rei, cujo nome é o Senhor dos
Exércitos, que certamente como o Tabor entre os montes, e como o
Carmelo junto ao mar, certamente assim ele virá. Prepara os
utensílios para ires ao cativeiro, ó moradora, filha do Egito;
porque Nofe será tornada em desolação, e será incendiada, até que
ninguém mais aí more. Bezerra mui formosa é o Egito; mas já
vem a destruição, vem do norte. Até os seus mercenários no
meio dela são como bezerros cevados; mas também eles viraram as
costas, fugiram juntos; não ficaram firmes; porque veio sobre eles o
dia da sua ruína e o tempo do seu castigo. A sua voz irá como
a da serpente; porque marcharão com um exército, e virão contra ela
com machados, como cortadores de lenha. Cortarão o seu
bosque, diz o Senhor, embora seja impenetrável; porque se
multiplicaram mais do que os gafanhotos; são inumeráveis. A
filha do Egito será envergonhada; será entregue na mão do povo do
norte. Diz o Senhor dos Exércitos, o Deus de Israel: Eis que
eu castigarei a Amom de Nô, e a Faraó, e ao Egito, e aos seus
deuses, e aos seus reis; ao próprio Faraó, e aos que nele confiam.
E os entregarei na mão dos que procuram a sua morte, na mão
de Nabucodonosor, rei de Babilônia, e na mão dos seus servos; mas
depois será habitada, como nos dias antigos, diz o Senhor.
Mas não temas tu, servo meu, Jacó, nem te espantes, ó Israel;
porque eis que te livrarei mesmo de longe, como também a tua
descendência da terra do seu cativeiro; e Jacó voltará, e
descansará, e sossegará, e não haverá quem o atemorize. Tu
não temas, servo meu, Jacó, diz o Senhor, porque estou contigo;
porque porei termo a todas as nações entre as quais te lancei; mas a
ti não darei fim, mas castigar-te-ei com justiça, e não te darei de
todo por inocente.

\medskip

\lettrine{47} A palavra do Senhor, que veio a Jeremias, o
profeta, contra os filisteus, antes que Faraó ferisse a Gaza.
Assim diz o Senhor: Eis que se levantam as águas do norte, e
tornar-se-ão em torrente transbordante, e alagarão a terra e sua
plenitude, a cidade, e os que nela habitam; e os homens clamarão, e
todos os moradores da terra se lamentarão; ao ruído estrepitoso
dos cascos dos seus fortes cavalos, ao barulho de seus carros, ao
estrondo das suas rodas; os pais não atendem aos filhos, por causa
da fraqueza das mãos; por causa do dia que vem, para destruir a
todos os filisteus, para cortar de Tiro e de Sidom todo o restante
que os socorra; porque o Senhor destruirá os filisteus, o
remanescente da ilha de Caftor. A calvície veio sobre Gaza, foi
desarraigada Ascalom, com o restante do seu vale; até quando te
retalharás? Ah! Espada do Senhor! Até quando deixarás de
repousar? Volta para a tua bainha, descansa, e aquieta-te. Mas
como te aquietarás? Pois o Senhor deu ordem à espada contra Ascalom,
e contra a praia do mar, para onde ele a enviou.

\medskip

\lettrine{48} Contra Moabe, assim diz o Senhor dos Exércitos,
Deus de Israel: Ai de Nebo, porque foi destruída; envergonhada está
Quiriataim, já está tomada; Misgabe está envergonhada e desanimada.
A glória de Moabe já não existe mais; em Hesbom tramaram mal
contra ela, dizendo: Vinde, e exterminemo-la, para que não seja mais
nação; também tu, ó Madmém, serás silenciada; a espada te
perseguirá. Voz de clamor de Horonaim: ruína e grande
destruição! Está destruída Moabe; seus filhinhos fizeram ouvir
um clamor. Porque pela subida de Luíte eles irão com choro
contínuo; porque na descida de Horonaim os adversários de Moabe
ouviram as angústias do grito da destruição. Fugi, salvai a
vossa vida; sede como a tamargueira no deserto; porque, por
causa da tua confiança nas tuas obras, e nos teus tesouros, também
tu serás tomada; e Quemós sairá para o cativeiro, os seus sacerdotes
e os seus príncipes juntamente. Porque virá o destruidor sobre
cada uma das cidades, e nenhuma cidade escapará, e perecerá o vale,
e destruir-se-á a campina; porque o Senhor o disse. Dai asas a
Moabe; porque voando sairá, e as suas cidades se tornarão em
desolação, e ninguém morará nelas. Maldito aquele que fizer a
obra do Senhor fraudulosamente; e maldito aquele que retém a sua
espada do sangue. Moabe esteve descansado desde a sua
mocidade, e repousou nas suas fezes, e não foi mudado de vasilha
para vasilha, nem foi para o cativeiro; por isso conservou o seu
sabor, e o seu cheiro não se alterou. Portanto, eis que dias
vêm, diz o Senhor, em que lhe enviarei derramadores que o
derramarão; e despejarão as suas vasilhas, e romperão os seus odres.
E Moabe terá vergonha de Quemós como a casa de Israel se
envergonhou de Betel, sua confiança.

Como direis: Somos valentes e homens fortes para a guerra?
Moabe está destruído, e subiu das suas cidades, e os seus
jovens escolhidos desceram à matança, diz o Rei, cujo nome é o
Senhor dos Exércitos. Está prestes a vir a calamidade de
Moabe; e apressa-se muito a sua aflição. Condoei-vos dele
todos os que estais ao seu redor, e todos os que sabeis o seu nome;
dizei: Como se quebrou a vara forte, o cajado formoso? Desce
da tua glória, e assenta-te em terra seca, ó moradora, filha de
Dibom; porque o destruidor de Moabe subiu contra ti, e desfez as
tuas fortalezas. Põe-te no caminho, e espia, ó moradora de
Aroer; pergunta ao que vai fugindo; e à que escapou dize: Que
sucedeu? Moabe está envergonhado, porque foi quebrantado;
lamentai e gritai; anunciai em Arnom que Moabe está destruído.
Também o julgamento veio sobre a terra da campina; sobre
Holom, sobre Jaza, sobre Mefaate, sobre Dibom, sobre Nebo,
sobre Bete-Diblataim, sobre Quiriataim, sobre Bete-Gamul,
sobre Bete-Meom, sobre Queriote, e sobre Bozra; e até sobre
todas as cidades da terra de Moabe, as de longe e as de perto.
Já é cortado o poder de Moabe, e é quebrantado o seu braço,
diz o Senhor. Embriagai-o, porque contra o Senhor se
engrandeceu; e Moabe se revolverá no seu vômito, e ele também se
tornará objeto de escárnio. Pois não foi também Israel objeto
de escárnio? Porventura foi achado entre ladrões, para que sempre
que fales dele, saltes de alegria? Deixai as cidades, e
habitai no rochedo, ó moradores de Moabe; e sede como a pomba que se
aninha nos lados da boca da caverna. Ouvimos da soberba de
Moabe, que é soberbíssimo, como também da sua arrogância, e da sua
vaidade, e da sua altivez e do seu orgulhoso coração. Eu
conheço, diz o Senhor, a sua indignação, mas isso nada é; as suas
mentiras nada farão. Por isso gemerei por Moabe, sim,
gritarei por todo o Moabe; pelos homens de Quir-Heres lamentarei;
com o choro de Jazer chorar-te-ei, ó vide de Sibma; os teus
ramos passaram o mar, chegaram até ao mar de Jazer; porém o
destruidor caiu sobre os teus frutos do verão, e sobre a tua
vindima. Tirou-se, pois, o folguedo e a alegria do campo
fértil e da terra de Moabe; porque fiz cessar o vinho nos lagares;
já não pisarão uvas com júbilo; o júbilo não será júbilo. Por
causa do grito de Hesbom até Eleale e até Jaaz, se ouviu a sua voz
desde Zoar até Horonaim, como bezerra de três anos; porque até as
águas do Ninrim se tornarão em assolação. E farei cessar em
Moabe, diz o Senhor, quem sacrifique nos altos, e queime incenso aos
seus deuses. Por isso ressoará como flauta o meu coração por
Moabe, também ressoará como flauta o meu coração pelos homens de
Quir-Heres; porquanto a abundância que ajuntou se perdeu.
Porque toda a cabeça será tosquiada, e toda a barba será
diminuída; sobre todas as mãos haverá sarjaduras\footnote{Ou
sarjação: ato ou efeito de sarjar (= abrir sarja em; fazer incisão
em). Sarja: tecido entrançado de lã, algodão ou seda, us. para
confecção de roupas.}, e sobre os lombos, sacos. Sobre todos
os telhados de Moabe e nas suas ruas haverá um pranto geral; porque
quebrei a Moabe, como a um vaso que não agrada, diz o Senhor.
Como está quebrantado! Como gritam! Como virou Moabe a cerviz
envergonhado! Assim será Moabe objeto de escárnio e de desmaio, para
todos que estão em redor dele. Porque assim diz o Senhor: Eis
que voará como a águia, e estenderá as suas asas sobre Moabe.
São tomadas as cidades, e ocupadas as fortalezas; e naquele
dia será o coração dos valentes de Moabe como o coração da mulher
que está com dores de parto. E Moabe será destruído, para que
não seja povo; porque se engrandeceu contra o Senhor. Temor,
e cova, e laço, vêm sobre ti, ó morador de Moabe, diz o Senhor.
O que fugir do temor cairá na cova, e o que subir da cova
ficará preso no laço; porque trarei sobre ele, sobre Moabe, o ano do
seu castigo, diz o Senhor. Os que fugiam sem força pararam à
sombra de Hesbom; pois saiu fogo de Hesbom, e a labareda do meio de
Siom, e devorou o canto de Moabe e o alto da cabeça dos turbulentos.
Ai de ti, Moabe! Pereceu o povo de Quemós; porque teus filhos
ficaram cativos, e tuas filhas em cativeiro. Mas nos últimos
dias farei voltar os cativos de Moabe, diz o Senhor. Até aqui o
juízo de Moabe.

\medskip

\lettrine{49} Contra os filhos de Amom. Assim diz o Senhor:
Acaso Israel não tem filhos, nem tem herdeiro? Por que, pois, herdou
Malcã a Gade e o seu povo habitou nas suas cidades? Portanto,
eis que vêm dias, diz o Senhor, em que farei ouvir em Rabá dos
filhos de Amom o alarido de guerra, e tornar-se-á num montão de
ruínas, e os lugares da sua jurisdição serão queimados a fogo; e
Israel herdará aos que o herdaram, diz o Senhor. Lamenta, ó
Hesbom, porque é destruída Ai; clamai, ó filhas de Rabá, cingi-vos
de sacos, lamentai, e dai voltas pelos valados; porque Malcã irá em
cativeiro, juntamente com seus sacerdotes e os seus príncipes.
Por que te glorias nos vales, teus luxuriantes vales, ó filha
rebelde, que confias nos teus tesouros, dizendo: Quem virá contra
mim? Eis que eu trarei temor sobre ti, diz o Senhor Deus dos
Exércitos, de todos os que estão ao redor de ti; e sereis lançados
fora cada um diante de si, e ninguém recolherá o desgarrado. Mas
depois disto farei voltar os cativos dos filhos de Amom, diz o
Senhor.

Acerca de Edom. Assim diz o Senhor dos Exércitos: Acaso não há
mais sabedoria em Temã? Pereceu o conselho dos entendidos?
Corrompeu-se a sua sabedoria? Fugi, voltai-vos, buscai
profundezas para habitar, ó moradores de Dedã, porque eu trarei
sobre ele a ruína de Esaú, no tempo em que o castiguei. Se
vindimadores viessem a ti, não deixariam rabiscos? Se ladrões de
noite viessem, não te danificariam quanto lhes bastasse? Mas
eu despi a Esaú, descobri os seus esconderijos, e não se poderá
esconder; foi destruída a sua descendência, como também seus irmãos
e seus vizinhos, e ele já não existe. Deixa os teus órfãos,
eu os guardarei em vida; e as tuas viúvas confiem em mim.
Porque assim diz o Senhor: Eis que os que não estavam
condenados a beber do copo, totalmente o beberão; e tu ficarias
inteiramente impune? Não ficarás impune, mas certamente o beberás.
Porque por mim mesmo jurei, diz o Senhor, que Bozra servirá
de espanto, de opróbrio, de assolação, e de maldição; e todas as
suas cidades se tornarão em desolações perpétuas. Ouvi novas
vindas do Senhor, que um embaixador é enviado aos gentios, para lhes
dizer: Ajuntai-vos, e vinde contra ela, e levantai-vos para a
guerra. Porque eis que te fiz pequeno entre os gentios,
desprezado entre os homens. Quanto à tua terribilidade,
enganou-te a arrogância do teu coração, tu que habitas nas cavernas
das rochas, que ocupas as alturas dos outeiros; ainda que eleves o
teu ninho como a águia, de lá te derrubarei, diz o Senhor.
Assim servirá Edom de desolação; todo aquele que passar por
ela se espantará, e assobiará por causa de todas as suas pragas.
Será como a destruição de Sodoma e Gomorra, e dos seus
vizinhos, diz o Senhor; não habitará ninguém ali, nem morará nela
filho de homem. Eis que ele como leão subirá da enchente do
Jordão contra a morada do forte; porque num momento o farei correr
dali; e quem é o escolhido que porei sobre ela? Pois quem é
semelhante a mim? e quem me fixará o tempo? e quem é o pastor que
subsistirá perante mim? Portanto ouvi o conselho do Senhor,
que ele decretou contra Edom, e os seus desígnios que ele intentou
entre os moradores de Temã: Certamente os menores do rebanho serão
arrastados; certamente ele assolará as suas moradas com eles.
A terra estremeceu com o estrondo da sua queda; e do seu
grito, até ao Mar Vermelho se ouviu o som. Eis que ele como
águia subirá, e voará, e estenderá as suas asas contra Bozra; e o
coração dos valentes de Edom naquele dia será como o coração da
mulher que está com dores de parto.

Acerca de Damasco. Envergonhou-se Hamate e Arpade, porquanto
ouviram más novas, desmaiaram; no mar há angústia, não se pode
sossegar. Enfraquecida está Damasco; virou as costas para
fugir, e o tremor a tomou; angústia e dores a tomaram como da que
está de parto. Como está abandonada a cidade do louvor, a
cidade da minha alegria! Portanto cairão os seus jovens nas
suas ruas; e todos os homens de guerra serão consumidos naquele dia,
diz o Senhor dos Exércitos. E acenderei fogo no muro de
Damasco, e consumirá os palácios de Bene-Hadade.

Acerca de Quedar, e dos reinos de Hazor, que Nabucodonosor, rei
de Babilônia, feriu. Assim diz o Senhor: Levantai-vos, subi contra
Quedar, e destruí os filhos do oriente. Tomarão as suas
tendas, os seus gados, as suas cortinas e todos os seus utensílios,
e os seus camelos levarão para si; e lhes clamarão: Há medo por
todos os lados. Fugi, desviai-vos para muito longe, buscai
profundezas para habitar, ó moradores de Hazor, diz o Senhor; porque
Nabucodonosor, rei de Babilônia, tomou conselho contra vós, e formou
um desígnio contra vós. Levantai-vos, subi contra uma nação
tranqüila, que habita confiadamente, diz o Senhor, que não tem
portas, nem ferrolhos; habitam sós. E os seus camelos serão
para presa, e a multidão dos seus gados para despojo; e os
espalharei a todo o vento, àqueles que estão nos lugares mais
distantes, e de todos os seus lados lhes trarei a sua ruína, diz o
Senhor. E Hazor se tornará em morada de chacais, em assolação
para sempre; ninguém habitará ali, nem morará nela filho de homem.

A palavra do Senhor, que veio a Jeremias, o profeta, contra Elão,
no princípio do reinado de Zedequias, rei de Judá, dizendo:
Assim diz o Senhor dos Exércitos: Eis que eu quebrarei o arco
de Elão, o principal do seu poder. E trarei sobre Elão os
quatro ventos dos quatro cantos dos céus, e os espalharei na direção
de todos estes ventos; e não haverá nação aonde não cheguem os
fugitivos de Elão. E farei que Elão tema diante de seus
inimigos e diante dos que procuram a sua morte; e farei vir sobre
eles o mal, o furor da minha ira, diz o Senhor; e enviarei após eles
a espada, até que venha a consumi-los. E porei o meu trono em
Elão; e destruirei dali o rei e os príncipes, diz o Senhor.
Acontecerá, porém, nos últimos dias, que farei voltar os
cativos de Elão, diz o Senhor.

\medskip

\lettrine{50} A palavra que falou o Senhor contra a Babilônia,
contra a terra dos caldeus, por intermédio de Jeremias, o profeta.
Anunciai entre as nações; e fazei ouvir, e arvorai um
estandarte, fazei ouvir, não encubrais; dizei: Tomada está
Babilônia, confundido está Bel, espatifado está Merodaque,
confundidos estão os seus ídolos, e quebradas estão as suas imagens.
Porque subiu contra ela uma nação do norte, que fará da sua
terra uma solidão, e não haverá quem nela habite; tanto os homens
como os animais fugiram, e se foram. Naqueles dias, e naquele
tempo, diz o Senhor, os filhos de Israel virão, eles e os filhos de
Judá juntamente; andando e chorando virão, e buscarão ao Senhor seu
Deus. Pelo caminho de Sião perguntarão, para ali voltarão os
seus rostos, dizendo: Vinde, e unamo-nos ao Senhor, numa aliança
eterna que nunca será esquecida. Ovelhas perdidas têm sido o meu
povo, os seus pastores as fizeram errar, para os montes as
desviaram; de monte para outeiro andaram, esqueceram-se do lugar do
seu repouso. Todos os que as achavam as devoravam, e os seus
adversários diziam: Culpa nenhuma teremos; porque pecaram contra o
Senhor, a morada da justiça, sim, o Senhor, a esperança de seus
pais. Fugi do meio de Babilônia, e saí da terra dos caldeus, e
sede como os bodes diante do rebanho.

Porque eis que eu suscitarei e farei subir contra a Babilônia uma
congregação de grandes nações da terra do norte, e se prepararão
contra ela; dali será tomada; as suas flechas serão como as de
valente herói, nenhuma tornará sem efeito. A Caldéia servirá
de presa; todos os que a saquearam serão fartos, diz o Senhor.
Porquanto vos alegrastes, e vos regozijastes, ó saqueadores
da minha herança, porquanto vos engordastes como novilha no pasto, e
mugistes como touros. Será mui confundida vossa mãe, ficará
envergonhada a que vos deu à luz; eis que ela será a última das
nações, um deserto, uma terra seca e uma solidão. Por causa
do furor do Senhor não será habitada, antes se tornará em total
assolação; qualquer que passar por Babilônia se espantará, assobiará
por todas as suas pragas. Ordenai-vos contra Babilônia ao
redor, todos os que armais arcos; atirai-lhe, não poupeis as
flechas, porque pecou contra o Senhor. Gritai contra ela ao
redor, ela já se submeteu; caíram seus fundamentos, estão derrubados
os seus muros; porque esta é a vingança do Senhor; vingai-vos dela;
como ela fez, assim lhe fazei. Arrancai de Babilônia o que
semeia, e o que leva a foice no tempo da sega; por causa da espada
aflitiva virar-se-á cada um para o seu povo, e fugirá cada um para a
sua terra. Cordeiro desgarrado é Israel; os leões o
afugentaram; o primeiro a devorá-lo foi o rei da Assíria; e, por
último Nabucodonosor, rei de Babilônia, lhe quebrou os ossos.
Portanto, assim diz o Senhor dos Exércitos, Deus de Israel:
Eis que castigarei o rei de Babilônia, e a sua terra, como castiguei
o rei da Assíria. E farei tornar Israel para a sua morada, e
ele pastará no Carmelo e em Basã; e fartar-se-á a sua alma no monte
de Efraim e em Gileade. Naqueles dias, e naquele tempo, diz o
Senhor, buscar-se-á a maldade de Israel, e não será achada; e os
pecados de Judá, mas não se acharão; porque perdoarei os
remanescentes que eu deixar.

Sobe contra a terra de Merataim, sim, contra ela, e contra os
moradores de Pecode; assola e inteiramente destrói tudo após eles,
diz o Senhor, e faze conforme tudo o que te mandei. Estrondo
de batalha há na terra, e de grande destruição. Como foi
cortado e quebrado o martelo de toda a terra! Como se tornou
Babilônia objeto de espanto entre as nações! Laços te armei,
e também foste presa, ó Babilônia, e tu não o soubeste; foste
achada, e também apanhada; porque contra o Senhor te entremeteste.
O Senhor abriu o seu depósito, e tirou os instrumentos da sua
indignação; porque o Senhor Deus dos Exércitos tem uma obra a
realizar na terra dos caldeus. Vinde contra ela dos confins
da terra, abri os seus celeiros; fazei dela montões de ruínas, e
destruí-a de todo; nada lhe fique de sobra. Matai a todos os
seus novilhos, desçam à matança. Ai deles, porque veio o seu dia, o
tempo do seu castigo\footnote{KJ: `visitação' em vez de `castigo'.
RC: Matai à espada a todos os seus novilhos, que eles desçam ao
degoladouro; ai deles! Porque veio o seu dia, o tempo da sua
visitação. KJ: Slay all her bullocks; let them go down to the
slaughter: woe unto them! for their day is come, the time of their
visitation.}! Eis a voz dos que fugiram e escaparam da terra
de Babilônia, para anunciarem em Sião a vingança do Senhor nosso
Deus, a vingança do seu templo. Convocai contra Babilônia os
flecheiros, a todos os que armam arcos; acampai-vos contra ela em
redor, ninguém escape dela; pagai-lhe conforme a sua obra, conforme
tudo o que fez, fazei-lhe; porque se houve arrogantemente contra o
Senhor, contra o Santo de Israel. Portanto, cairão os seus
jovens nas suas ruas; e todos os seus homens de guerra serão
desarraigados naquele dia, diz o Senhor. Eis que eu sou
contra ti, ó soberbo, diz o Senhor Deus dos Exércitos; porque veio o
teu dia, o tempo em que te hei de castigar. Então tropeçará o
soberbo, e cairá, e ninguém haverá que o levante; e porei fogo nas
suas cidades, o qual consumirá todos os seus arredores.

Assim diz o Senhor dos Exércitos: Os filhos de Israel e os filhos
de Judá foram oprimidos juntamente; e todos os que os levaram
cativos os retiveram, não os quiseram soltar. Mas o seu
Redentor é forte, o Senhor dos Exércitos é o seu nome; certamente
pleiteará a causa deles, para dar descanso à terra, e inquietar os
moradores de Babilônia. A espada virá sobre os caldeus, diz o
Senhor, e sobre os moradores de Babilônia, e sobre os seus
príncipes, e sobre os seus sábios. A espada virá sobre os
mentirosos, e ficarão insensatos; a espada virá sobre os seus
poderosos, e desfalecerão. A espada virá sobre os seus
cavalos, e sobre os seus carros, e sobre toda a mistura de povos,
que está no meio dela; e tornar-se-ão como mulheres; a espada virá
sobre os seus tesouros, e serão saqueados. Cairá a seca sobre
as suas águas, e secarão; porque é uma terra de imagens esculpidas,
e pelos seus ídolos andam enfurecidos. Por isso habitarão
nela as feras do deserto, com os animais selvagens das ilhas; também
habitarão nela as avestruzes; e nunca mais será povoada, nem será
habitada de geração em geração. Como quando Deus subverteu a
Sodoma e a Gomorra, e as suas cidades vizinhas, diz o Senhor, assim
ninguém habitará ali, nem morará nela filho de homem. Eis que
um povo vem do norte; uma grande nação e muitos reis se levantarão
dos extremos da terra. Armam-se de arco e lança; eles são
cruéis, e não têm piedade; a sua voz bramará como o mar, e sobre
cavalos cavalgarão, todos postos em ordem como um homem para a
batalha, contra ti, ó filha de Babilônia. O rei de Babilônia
ouviu a sua fama, e desfaleceram as suas mãos; a angústia se
apoderou dele, como da que está de parto. Eis que ele como
leão subirá da enchente do Jordão, contra a morada forte, porque num
momento o farei correr dali; e quem é o escolhido que porei sobre
ela? porque quem é semelhante a mim, e quem me fixará o tempo? E
quem é o pastor que poderá permanecer perante mim? Portanto
ouvi o conselho do Senhor, que ele decretou contra Babilônia, e os
seus desígnios que intentou contra a terra dos caldeus: certamente
os pequenos do rebanho serão arrastados; certamente ele assolará as
suas moradas sobre eles. Ao estrondo da tomada de Babilônia
estremeceu a terra; e o grito se ouviu entre as nações.

\medskip

\lettrine{51} Assim diz o Senhor: Eis que levantarei um vento
destruidor contra Babilônia, e contra os que habitam no meio dos que
se levantam contra mim. E enviarei
padejadores\footnote{Padejador: que ou o que padeja ('revolve').}
contra Babilônia, que a padejarão, e despejarão a sua terra; porque
virão contra ela em redor no dia da calamidade. O flecheiro arme
o seu arco contra o que arma o seu arco, e contra o que se exalta na
sua couraça; e não perdoeis aos seus jovens; destruí a todo o seu
exército. E os mortos cairão na terra dos caldeus, e
atravessados nas suas ruas. Porque Israel e Judá não foram
abandonados do seu Deus, do Senhor dos Exércitos, ainda que a sua
terra esteja cheia de culpas contra o Santo de Israel. Fugi do
meio de Babilônia, e livrai cada um a sua alma, e não vos destruais
na sua maldade; porque este é o tempo da vingança do Senhor; que lhe
dará a sua recompensa. Babilônia era um copo de ouro na mão do
Senhor, o qual embriagava a toda a terra; do seu vinho beberam as
nações; por isso as nações enlouqueceram. Num momento caiu
Babilônia, e ficou arruinada; lamentai por ela, tomai bálsamo para a
sua dor, porventura sarará. Queríamos curar Babilônia, porém ela
não sarou; deixai-a, e vamo-nos cada um para a sua terra; porque o
seu juízo chegou até ao céu, e se elevou até às mais altas nuvens.
O Senhor trouxe a nossa justiça à luz; vinde e contemos em
Sião a obra do Senhor, nosso Deus. Aguçai as flechas,
preparai os escudos; o Senhor despertou o espírito dos reis da
Média; porque o seu intento é contra Babilônia para a destruir;
porque esta é a vingança do Senhor, a vingança do seu templo.
Arvorai um estandarte sobre os muros de Babilônia, reforçai a
guarda, colocai sentinelas, preparai as ciladas; porque como o
Senhor intentou, assim fez o que tinha falado contra os moradores de
Babilônia. Ó tu, que habitas sobre muitas águas, rica de
tesouros, é chegado o teu fim, a medida da tua avareza. Jurou
o Senhor dos Exércitos por si mesmo, dizendo: Ainda que te enchi de
homens, como de lagarta, contudo levantarão gritaria contra ti.
Ele fez a terra com o seu poder, e ordenou o mundo com a sua
sabedoria, e estendeu os céus com o seu entendimento. Fazendo
ele ouvir a sua voz, grande estrondo de águas há nos céus, e faz
subir os vapores desde o fim da terra; faz os relâmpagos com a
chuva, e tira o vento dos seus tesouros. Embrutecido é todo o
homem, no seu conhecimento; envergonha-se todo o artífice da imagem
de escultura; porque a sua imagem de fundição é mentira, e nelas não
há espírito. Vaidade são, obra de enganos; no tempo da sua
visitação perecerão. Não é semelhante a estes a porção de
Jacó; porque ele é o que formou tudo; e Israel é a tribo da sua
herança; o Senhor dos Exércitos é o seu nome. Tu és meu
machado de batalha e minhas armas de guerra, e por meio de ti
despedaçarei as nações e por ti destruirei os reis; e por
meio de ti despedaçarei o cavalo e o seu cavaleiro; e por meio de ti
despedaçarei o carro e o que nele vai; e por meio de ti
despedaçarei o homem e a mulher, e por meio de ti despedaçarei o
velho e o moço, e por meio de ti despedaçarei o jovem e a virgem;
e por meio de ti despedaçarei o pastor e o seu rebanho, e por
meio de ti despedaçarei o lavrador e a sua junta de bois, e por meio
de ti despedaçarei os capitães e os magistrados. E pagarei a
Babilônia, e a todos os moradores da Caldéia, toda a maldade que
fizeram em Sião, aos vossos olhos, diz o Senhor. Eis-me aqui
contra ti, ó monte destruidor, diz o Senhor, que destróis toda a
terra; e estenderei a minha mão contra ti, e te revolverei das
rochas, e farei de ti um monte de queima. E não tomarão de ti
pedra para esquina, nem pedra para fundamentos, porque te tornarás
em assolação perpétua, diz o Senhor. Arvorai um estandarte na
terra, tocai a buzina entre as nações, preparai as nações contra
ela, convocai contra ela os reinos de Ararate, Mini, e Asquenaz;
ordenai contra ela um capitão, fazei subir cavalos, como lagartas
eriçadas. Preparai contra ela as nações, os reis da Média, os
seus capitães, e todos os seus magistrados, e toda a terra do seu
domínio. Então tremerá a terra, e doer-se-á, porque cada um
dos desígnios do Senhor está firme contra Babilônia, para fazer da
terra de Babilônia uma desolação, sem habitantes. Os
poderosos de Babilônia cessaram de pelejar, ficaram nas fortalezas,
desfaleceu a sua força, tornaram-se como mulheres; incendiaram as
suas moradas, quebrados foram os seus ferrolhos. Um correio
correrá ao encontro de outro correio, e um mensageiro ao encontro de
outro mensageiro, para anunciar ao rei de Babilônia que a sua cidade
está tomada de todos os lados. E os vaus\footnote{Vau: local
raso de um rio, mar, lagoa, por onde se pode passar a pé ou a
cavalo. Banco de areia ou elevação do fundo do mar que às vezes
chega à superfície; baixio. Derivação (sentido figurado): ocasião
favorável, ensejo, oportunidade.} estão ocupados, e os canaviais
queimados a fogo; e os homens de guerra ficaram assombrados.
Porque assim diz o Senhor dos Exércitos, o Deus de Israel: A
filha de Babilônia é como uma eira\footnote{Local de terra batida,
cimentado ou lajeado, próprio para debulhar, trilhar, secar e limpar
cereais e legumes.}, no tempo da debulha; ainda um pouco, e o tempo
da sega lhe virá. Nabucodonosor, rei de Babilônia,
devorou-me, colocou-me de lado, fez de mim um vaso vazio, como
chacal me tragou, encheu o seu ventre das minhas delicadezas;
lançou-me fora. A violência que se fez a mim e à minha carne
venha sobre Babilônia, dirá a moradora de Sião; e o meu sangue caia
sobre os moradores da Caldéia, dirá Jerusalém. Portanto,
assim diz o Senhor: Eis que pleitearei a tua causa, e tomarei
vingança por ti; e secarei o seu mar, e farei que se esgote o seu
manancial. E Babilônia se tornará em montões, morada de
chacais, espanto e assobio, sem que haja quem nela habite.
Juntamente rugirão como filhos dos leões; bramarão como
filhotes de leões. Estando eles excitados, lhes darei a sua
bebida, e os embriagarei, para que andem saltando; porém dormirão um
perpétuo sono, e não acordarão, diz o Senhor. Fá-los-ei
descer como cordeiros à matança, como carneiros e bodes. Como
foi tomada Sesaque, e apanhada de surpresa a glória de toda a terra!
Como se tornou Babilônia objeto de espanto entre as nações! O
mar subiu sobre Babilônia; com a multidão das suas ondas se cobriu.
Tornaram-se as suas cidades em desolação, terra seca e
deserta, terra em que ninguém habita, nem passa por ela filho de
homem. E castigarei a Bel em Babilônia, e tirarei da sua boca
o que tragou, e nunca mais concorrerão a ele as nações; também o
muro de Babilônia caiu. Saí do meio dela, ó povo meu, e
livrai cada um a sua alma do ardor da ira do Senhor. E para
que porventura não se enterneça o vosso coração, e não temais pelo
rumor que se ouvir na terra; porque virá num ano um rumor, e depois
noutro ano outro rumor; e haverá violência na terra, dominador
contra dominador. Portanto, eis que vêm dias, em que farei
juízo sobre as imagens de escultura de Babilônia, e toda a sua terra
será envergonhada, e todos os seus mortos cairão no meio dela.
E os céus e a terra, com tudo quanto neles há, jubilarão
sobre Babilônia; porque do norte lhe virão os destruidores, diz o
Senhor. Como Babilônia fez cair mortos os de Israel, assim em
Babilônia cairão os mortos de toda a terra. Vós, que
escapastes da espada, ide-vos, não pareis; de longe lembrai-vos do
Senhor, e suba Jerusalém a vossa mente. Direis: Envergonhados
estamos, porque ouvimos opróbrio; vergonha cobriu o nosso rosto,
porquanto vieram estrangeiros contra os santuários da casa do
Senhor. Portanto, eis que vêm dias, diz o Senhor, em que
farei juízo sobre as suas imagens de escultura; e gemerão os feridos
em toda a sua terra. Ainda que Babilônia subisse aos céus, e
ainda que fortificasse a altura da sua fortaleza, todavia de mim
virão destruidores sobre ela, diz o Senhor. De Babilônia se
ouve clamor de grande destruição da terra dos caldeus; porque
o Senhor tem destruído Babilônia, e tem feito perecer nela a sua
grande voz; quando as suas ondas bramam como muitas águas, é emitido
o ruído da sua voz. Porque o destruidor vem sobre ela, sobre
Babilônia, e os seus poderosos serão presos, já estão quebrados os
seus arcos; porque o Senhor, Deus das recompensas, certamente lhe
retribuirá. E embriagarei os seus príncipes, e os seus sábios
e os seus capitães, e os seus magistrados, e os seus poderosos; e
dormirão um sono eterno, e não acordarão, diz o Rei, cujo nome é o
Senhor dos Exércitos. Assim diz o Senhor dos Exércitos: Os
largos muros de Babilônia serão totalmente derrubados, e as suas
altas portas serão abrasadas pelo fogo; e trabalharão os povos em
vão, e as nações no fogo, e eles se cansarão.

A palavra que Jeremias, o profeta, mandou a Seraías, filho de
Nerias, filho de Maaséias, indo ele com Zedequias, rei de Judá, a
Babilônia, no quarto ano do seu reinado. E Seraías era o
camareiro-mor. Escreveu, pois, Jeremias num livro todo o mal
que havia de vir sobre Babilônia, a saber, todas estas palavras que
estavam escritas contra Babilônia. E disse Jeremias a
Seraías: Quando chegares a Babilônia, verás e lerás todas estas
palavras. E dirás: Senhor, tu falaste contra este lugar, que
o havias de desarraigar, até não ficar nele morador algum, nem homem
nem animal, e que se tornaria em perpétua desolação. E será
que, acabando tu de ler este livro, atar-lhe-ás uma pedra e
lançá-lo-ás no meio do Eufrates. E dirás: Assim será afundada
Babilônia, e não se levantará, por causa do mal que eu hei de trazer
sobre ela; e eles se cansarão. Até aqui são as palavras de Jeremias.

\medskip

\lettrine{52} Era Zedequias da idade de vinte e um anos quando
começou a reinar, e reinou onze anos em Jerusalém; e o nome de sua
mãe era Hamutal, filha de Jeremias, de Libna. E fez o que era
mau aos olhos do Senhor, conforme tudo o que fizera Jeoiaquim.
Assim, por causa da ira do Senhor, contra Jerusalém e Judá, ele
os lançou de diante dele, e Zedequias se rebelou contra o rei de
Babilônia. E aconteceu que, no ano nono do seu reinado, no
décimo mês, no décimo dia do mês, veio Nabucodonosor, rei de
Babilônia, contra Jerusalém, ele e todo o seu exército, e se
acamparam contra ela, e levantaram contra ela trincheiras ao redor.
Assim esteve cercada a cidade, até ao undécimo ano do rei
Zedequias. No quarto mês, aos nove dias do mês, quando já a fome
prevalecia na cidade, e o povo da terra não tinha pão, então foi
aberta uma brecha na cidade, e todos os homens de guerra fugiram, e
saíram da cidade de noite, pelo caminho da porta entre os dois
muros, a qual estava perto do jardim do rei (porque os caldeus
cercavam a cidade ao redor), e foram pelo caminho da campina.
Mas o exército dos caldeus perseguiu o rei, e alcançou a
Zedequias nas campinas de Jericó, e todo o seu exército se espalhou,
abandonando-o. E prenderam o rei, e o fizeram subir ao rei de
Babilônia, a Ribla, na terra de Hamate, o qual lhe pronunciou a
sentença. E o rei de Babilônia degolou os filhos de Zedequias
à sua vista, e também degolou a todos os príncipes de Judá em Ribla.
E cegou os olhos a Zedequias, e o atou com cadeias; e o rei
de Babilônia o levou para Babilônia, e o conservou na prisão até o
dia da sua morte.

E no quinto mês, no décimo dia do mês, que era o décimo nono ano
do rei Nabucodonosor, rei de Babilônia, Nebuzaradã, capitão da
guarda, que assistia na presença do rei de Babilônia, veio a
Jerusalém. E queimou a casa do Senhor, e a casa do rei; e
também a todas as casas de Jerusalém, e a todas as casas dos grandes
ele as incendiou. E todo o exército dos caldeus, que estava
com o capitão da guarda, derrubou a todos os muros em redor de
Jerusalém. E dos mais pobres do povo, e a parte do povo, que
tinha ficado na cidade, e os rebeldes que se haviam passado para o
rei de Babilônia, e o mais da multidão, Nebuzaradã, capitão da
guarda, levou presos. Mas dos mais pobres da terra
Nebuzaradã, capitão da guarda, deixou ficar alguns, para serem
vinhateiros e lavradores. Quebraram mais os caldeus as
colunas de bronze, que estavam na casa do Senhor, e as bases, e o
mar de bronze, que estavam na casa do Senhor, e levaram todo o
bronze para Babilônia. Também tomaram os caldeirões, e as
pás, e as espevitadeiras, e as bacias, e as colheres, e todos os
utensílios de bronze, com que se ministrava. E tomou o
capitão da guarda as bacias, e os braseiros, e as tigelas, e os
caldeirões, e os castiçais, e as colheres, e os copos; tanto o que
era de puro ouro, como o que era de prata maciça. Quanto às
duas colunas, ao único mar, e aos doze bois de bronze, que estavam
debaixo das bases, que fizera o rei Salomão para a casa do Senhor, o
peso do bronze de todos estes utensílios era incalculável.
Quanto às colunas, a altura de cada uma era de dezoito
côvados, e um fio de doze côvados a cercava; e era a sua espessura
de quatro dedos, e era oca. E havia sobre ela um capitel de
bronze; e a altura do capitel era de cinco côvados; a rede e as
romãs ao redor do capitel eram de bronze; e semelhante a esta era a
segunda coluna, com as romãs. E havia noventa e seis romãs em
cada lado; as romãs todas, em redor da rede, eram cem.

Levou também o capitão da guarda a Seraías, o sacerdote chefe, e
a Sofonias, o segundo sacerdote, e aos três guardas da porta.
E da cidade tomou a um eunuco que tinha a seu cargo os homens
de guerra, e a sete homens que estavam próximos à pessoa do rei, que
se achavam na cidade, como também o escrivão-mor do exército, que
alistava o povo da terra para a guerra, e a sessenta homens do povo
da terra, que se achavam no meio da cidade. Tomando-os, pois,
Nebuzaradã, capitão da guarda, levou-os ao rei de Babilônia, a
Ribla. E o rei de Babilônia os feriu e os matou em Ribla, na
terra de Hamate; assim Judá foi levado cativo para fora da sua
terra. Este é o povo que Nabucodonosor levou cativo, no
sétimo ano: três mil e vinte e três judeus. No ano décimo
oitavo de Nabucodonosor, ele levou cativas de Jerusalém oitocentas e
trinta e duas pessoas. No ano vinte e três de Nabucodonosor,
Nebuzaradã, capitão da guarda, levou cativas, dos judeus, setecentas
e quarenta e cinco pessoas; todas as pessoas foram quatro mil e
seiscentas.

Sucedeu, pois, no ano trigésimo sétimo do cativeiro de Jeoiaquim,
rei de Judá, no duodécimo mês, aos vinte e cinco dias do mês, que
Evil-Merodaque, rei de Babilônia, no primeiro ano do seu reinado,
levantou a cabeça de Jeoiaquim, rei de Judá, e tirou-o do cárcere;
e falou com ele benignamente, e pôs o seu trono acima dos
tronos dos reis que estavam com ele em Babilônia; e lhe fez
mudar as vestes da sua prisão; e passou a comer pão sempre na
presença do rei, todos os dias da sua vida. E, quanto à sua
alimentação, foi-lhe dada refeição contínua do rei de Babilônia,
porção cotidiana, no seu dia, até o dia da sua morte, todos os dias
da sua vida.

