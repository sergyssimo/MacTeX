\addchap{Ester}

\lettrine{1} E sucedeu nos dias de Assuero, o Assuero que
reinou desde a Índia até Etiópia, sobre cento e vinte e sete
províncias, que, assentando-se o rei Assuero no trono do seu
reino, que estava na fortaleza de Susã, no terceiro ano do seu
reinado, fez um banquete a todos os seus príncipes e seus servos,
estando assim perante ele o poder da Pérsia e Média e os nobres e
príncipes das províncias, para mostrar as riquezas da glória do
seu reino, e o esplendor da sua excelente grandeza, por muitos dias,
a saber: cento e oitenta dias. E, acabados aqueles dias, fez o
rei um banquete a todo o povo que se achava na fortaleza de Susã,
desde o maior até ao menor, por sete dias, no pátio do jardim do
palácio real. As tapeçarias eram de pano branco, verde, e azul
celeste, pendentes de cordões de linho fino e púrpura, e argolas de
prata, e colunas de mármore; os leitos de ouro e de prata, sobre um
pavimento de mármore vermelho, e azul, e branco e preto. E
dava-se de beber em copos de ouro, e os copos eram diferentes uns
dos outros; e havia muito vinho real, segundo a generosidade do rei.
E o beber era por lei, sem constrangimento; porque assim tinha
ordenado o rei expressamente a todos os oficiais da sua casa, que
fizessem conforme a vontade de cada um. Também a rainha Vasti
deu um banquete às mulheres, na casa real, do rei Assuero.

E ao sétimo dia, estando já o coração do rei alegre do vinho,
mandou a Meumã, Bizta, Harbona, Bigtá, Abagta, Zetar e Carcas, os
sete camareiros que serviam na presença do rei Assuero, que
introduzissem na presença do rei a rainha Vasti, com a coroa real,
para mostrar aos povos e aos príncipes a sua beleza, porque era
formosa à vista. Porém a rainha Vasti recusou vir conforme a
palavra do rei, por meio dos camareiros; assim o rei muito se
enfureceu, e acendeu nele a sua ira. Então perguntou o rei
aos sábios que entendiam dos tempos (porque assim se tratavam os
negócios do rei na presença de todos os que sabiam a lei e o
direito; e os mais chegados a ele eram: Carsena, Setar,
Admata, Társis, Meres, Marsena, e Memucã, os sete príncipes dos
persas e dos medos, que viam a face do rei, e se assentavam como
principais no reino), o que, segundo a lei, se devia fazer à
rainha Vasti, por não ter obedecido ao mandado do rei Assuero, por
meio dos camareiros. Então disse Memucã na presença do rei e
dos príncipes: Não somente contra o rei pecou a rainha Vasti, porém
também contra todos os príncipes, e contra todos os povos que há em
todas as províncias do rei Assuero. Porque a notícia do que
fez a rainha chegará a todas as mulheres, de modo que aos seus olhos
desprezarão a seus maridos quando ouvirem dizer: Mandou o rei
Assuero que introduzissem à sua presença a rainha Vasti, porém ela
não veio. E neste mesmo dia as senhoras da Pérsia e da Média
ouvindo o que fez a rainha, dirão o mesmo a todos os príncipes do
rei; e assim haverá muito desprezo e indignação. Se bem
parecer ao rei, saia da sua parte um edito real, e escreva-se nas
leis dos persas e dos medos, e não se revogue, a saber: que Vasti
não entre mais na presença do rei Assuero, e o rei dê o reino dela a
outra que seja melhor do que ela. E, ouvindo-se o mandado,
que o rei decretara em todo o seu reino (porque é grande), todas as
mulheres darão honra a seus maridos, desde a maior até à menor.
E pareceram bem estas palavras aos olhos do rei e dos
príncipes; e fez o rei conforme a palavra de Memucã. Então
enviou cartas a todas as províncias do rei, a cada província segundo
a sua escrita, e a cada povo segundo a sua língua; que cada homem
fosse senhor em sua casa, e que se falasse conforme a língua do seu
povo.

\medskip

\lettrine{2} Passadas estas coisas, e apaziguado já o furor do
rei Assuero, lembrou-se de Vasti, e do que fizera, e do que se tinha
decretado a seu respeito. Então disseram os servos do rei, que
lhe serviam: Busquem-se para o rei moças virgens e formosas. E
ponha o rei oficiais em todas as províncias do seu reino, que
ajuntem a todas as moças virgens e formosas, na fortaleza de Susã,
na casa das mulheres, aos cuidados de Hegai, camareiro do rei,
guarda das mulheres, e dêem-se-lhes os seus enfeites. E a moça
que parecer bem aos olhos do rei, reine em lugar de Vasti. E isto
pareceu bem aos olhos do rei, e ele assim fez. Havia então um
homem judeu na fortaleza de Susã, cujo nome era Mardoqueu, filho de
Jair, filho de Simei, filho de Quis, homem benjamita, que fora
transportado de Jerusalém, com os cativos que foram levados com
Jeconias, rei de Judá, o qual transportara Nabucodonosor, rei de
Babilônia. Este criara a Hadassa (que é Ester, filha de seu
tio), porque não tinha pai nem mãe; e era jovem bela de presença e
formosa; e, morrendo seu pai e sua mãe, Mardoqueu a tomara por sua
filha. Sucedeu que, divulgando-se o mandado do rei e a sua lei,
e ajuntando-se muitas moças na fortaleza de Susã, aos cuidados de
Hegai, também levaram Ester à casa do rei, sob a custódia de Hegai,
guarda das mulheres. E a moça pareceu formosa aos seus olhos, e
alcançou graça perante ele; por isso se apressou a dar-lhe os seus
enfeites, e os seus quinhões, como também em lhe dar sete moças de
respeito da casa do rei; e a fez passar com as suas moças ao melhor
lugar da casa das mulheres. Ester, porém, não declarou o seu
povo e a sua parentela, porque Mardoqueu lhe tinha ordenado que o
não declarasse. E passeava Mardoqueu cada dia diante do pátio
da casa das mulheres, para se informar de como Ester passava, e do
que lhe sucederia. E, chegando a vez de cada moça, para vir
ao rei Assuero, depois que fora feito a ela segundo a lei das
mulheres, por doze meses (porque assim se cumpriam os dias das suas
purificações, seis meses com óleo de mirra, e seis meses com
especiarias, e com as coisas para a purificação das mulheres),
desta maneira, pois, vinha a moça ao rei; dava-se-lhe tudo
quanto ela desejava, para levar consigo da casa das mulheres à casa
do rei; à tarde entrava, e pela manhã tornava à segunda casa
das mulheres, sob os cuidados de Saasgaz, camareiro do rei, guarda
das concubinas; não tornava mais ao rei, salvo se o rei a desejasse,
e fosse chamada pelo nome. Chegando, pois, a vez de Ester,
filha de Abiail, tio de Mardoqueu (que a tomara por sua filha), para
ir ao rei, coisa nenhuma pediu, senão o que disse Hegai, camareiro
do rei, guarda das mulheres; e alcançava Ester graça aos olhos de
todos quantos a viam. Assim foi levada Ester ao rei Assuero,
à sua casa real, no décimo mês, que é o mês de tebete, no sétimo ano
do seu reinado. E o rei amou a Ester mais do que a todas as
mulheres, e alcançou perante ele graça e benevolência mais do que
todas as virgens; e pôs a coroa real na sua cabeça, e a fez rainha
em lugar de Vasti. Então o rei deu um grande banquete a todos
os seus príncipes e aos seus servos; era o banquete de Ester; e deu
alívio às províncias, e fez presentes segundo a generosidade do rei.
E reunindo-se segunda vez as virgens, Mardoqueu estava
assentado à porta do rei. Ester, porém, não declarava a sua
parentela e o seu povo, como Mardoqueu lhe ordenara; porque Ester
cumpria o mandado de Mardoqueu, como quando a criara.

Naqueles dias, assentando-se Mardoqueu à porta do rei, dois
camareiros do rei, dos guardas da porta, Bigtã e Teres, grandemente
se indignaram, e procuraram atentar contra o rei Assuero. E
veio isto ao conhecimento de Mardoqueu, e ele fez saber à rainha
Ester; e Ester o disse ao rei, em nome de Mardoqueu. E
inquiriu-se o negócio, e se descobriu, e ambos foram pendurados numa
forca; e foi escrito nas crônicas perante o rei.

\medskip

\lettrine{3} Depois destas coisas o rei Assuero engrandeceu a
Hamã, filho de Hamedata, agagita, e o exaltou, e pôs o seu assento
acima de todos os príncipes que estavam com ele. E todos os
servos do rei, que estavam à porta do rei, se inclinavam e se
prostravam perante Hamã; porque assim tinha ordenado o rei acerca
dele; porém Mardoqueu não se inclinava nem se prostrava. Então
os servos do rei, que estavam à porta do rei, disseram a Mardoqueu:
Por que transgride o mandado do rei? Sucedeu, pois, que,
dizendo-lhe eles isto, dia após dia, e não lhes dando ele ouvidos, o
fizeram saber a Hamã, para verem se as palavras de Mardoqueu se
sustentariam, porque ele lhes tinha declarado que era judeu.
Vendo, pois, Hamã que Mardoqueu não se inclinava nem se
prostrava diante dele, Hamã se encheu de furor. Porém teve como
pouco, nos seus propósitos, o pôr as mãos só em Mardoqueu (porque
lhe haviam declarado de que povo era Mardoqueu); Hamã, pois,
procurou destruir a todos os judeus, o povo de Mardoqueu, que havia
em todo o reino de Assuero.

No primeiro mês (que é o mês de Nisã), no ano duodécimo do rei
Assuero, se lançou Pur, isto é, a sorte, perante Hamã, para cada
dia, e para cada mês, até ao duodécimo mês, que é o mês de Adar.
E Hamã disse ao rei Assuero: Existe espalhado e dividido entre
os povos em todas as províncias do teu reino um povo, cujas leis são
diferentes das leis de todos os povos, e que não cumpre as leis do
rei; por isso não convém ao rei deixá-lo ficar. Se bem parecer
ao rei, decrete-se que os matem; e eu porei nas mãos dos que fizerem
a obra dez mil talentos de prata, para que entrem nos tesouros do
rei. Então tirou o rei o anel da sua mão, e o deu a Hamã,
filho de Hamedata, agagita, adversário dos judeus. E disse o
rei a Hamã: Essa prata te é dada como também esse povo, para fazeres
dele o que bem parecer aos teus olhos. Então chamaram os
escrivães do rei no primeiro mês, no dia treze do mesmo e, conforme
a tudo quanto Hamã mandou, se escreveu aos príncipes do rei, e aos
governadores que havia sobre cada província, e aos líderes, de cada
povo; a cada província segundo a sua escrita, e a cada povo segundo
a sua língua; em nome do rei Assuero se escreveu, e com o anel do
rei se selou. E enviaram-se as cartas por intermédio dos
correios a todas as províncias do rei, para que destruíssem,
matassem, e fizessem perecer a todos os judeus, desde o jovem até ao
velho, crianças e mulheres, em um mesmo dia, a treze do duodécimo
mês (que é o mês de Adar), e que saqueassem os seus bens. Uma
cópia do despacho que determinou a divulgação da lei em cada
província, foi enviada a todos os povos, para que estivessem
preparados para aquele dia. Os correios, pois, impelidos pela
palavra do rei, saíram, e a lei se proclamou na fortaleza de Susã. E
o rei e Hamã se assentaram a beber, porém a cidade de Susã estava
confusa.

\medskip

\lettrine{4} Quando Mardoqueu soube tudo quanto se havia
passado, rasgou as suas vestes, e vestiu-se de saco e de cinza, e
saiu pelo meio da cidade, e clamou com grande e amargo clamor; e
chegou até diante da porta do rei, porque ninguém vestido de saco
podia entrar pelas portas do rei. E em todas as províncias aonde
a palavra do rei e a sua lei chegava, havia entre os judeus grande
luto, com jejum, e choro, e lamentação; e muitos estavam deitados em
saco e em cinza. Então vieram as servas de Ester, e os seus
camareiros, e fizeram-na saber, do que a rainha muito se doeu; e
mandou roupas para vestir a Mardoqueu, e tirar-lhe o pano de saco;
porém ele não as aceitou.

Então Ester chamou a Hatá (um dos camareiros do rei, que este
tinha posto para servi-la), e deu-lhe ordem para ir a Mardoqueu,
para saber que era aquilo, e porquê. E, saindo Hatá a Mardoqueu,
à praça da cidade, que estava diante da porta do rei, Mardoqueu
lhe fez saber tudo quanto lhe tinha sucedido; como também a soma
exata do dinheiro, que Hamã dissera que daria para os tesouros do
rei, pelos judeus, para destruí-los. Também lhe deu a cópia da
lei escrita, que se publicara em Susã, para os destruir, para que a
mostrasse a Ester, e a fizesse saber; e para lhe ordenar que fosse
ter com o rei, e lhe pedisse e suplicasse na sua presença pelo seu
povo. Veio, pois, Hatá, e fez saber a Ester as palavras de
Mardoqueu. Então falou Ester a Hatá, mandando-o dizer a
Mardoqueu: Todos os servos do rei, e o povo das províncias do
rei, bem sabem que todo o homem ou mulher que chegar ao rei no pátio
interior, sem ser chamado, não há senão uma sentença, a de morte,
salvo se o rei estender para ele o cetro de ouro, para que viva; e
eu nestes trinta dias não tenho sido chamada para ir ao rei.
E fizeram saber a Mardoqueu as palavras de Ester.
Então Mardoqueu mandou que respondessem a Ester: Não imagines
no teu íntimo que por estares na casa do rei, escaparás só tu entre
todos os judeus. Porque, se de todo te calares neste tempo,
socorro e livramento de outra parte sairá para os judeus, mas tu e a
casa de teu pai perecereis; e quem sabe se para tal tempo como este
chegaste a este reino? Então disse Ester que tornassem a
dizer a Mardoqueu: Vai, ajunta a todos os judeus que se
acharem em Susã, e jejuai por mim, e não comais nem bebais por três
dias, nem de dia nem de noite, e eu e as minhas servas também assim
jejuaremos. E assim irei ter com o rei, ainda que não seja segundo a
lei; e se perecer, pereci. Então Mardoqueu foi, e fez
conforme a tudo quanto Ester lhe ordenou.

\medskip

\lettrine{5} Sucedeu, pois, que ao terceiro dia Ester se
vestiu com trajes reais, e se pôs no pátio interior da casa do rei,
defronte do aposento do rei; e o rei estava assentado sobre o seu
trono real, na casa real, defronte da porta do aposento. E
sucedeu que, vendo o rei à rainha Ester, que estava no pátio,
alcançou graça aos seus olhos; e o rei estendeu para Ester o cetro
de ouro, que tinha na sua mão, e Ester chegou, e tocou a ponta do
cetro. Então o rei lhe disse: Que é que queres, rainha Ester, ou
qual é a tua petição? Até metade do reino se te dará. E disse
Ester: Se parecer bem ao rei, venha hoje com Hamã ao banquete que
lhe tenho preparado. Então disse o rei: Fazei apressar a Hamã,
para que se atenda ao desejo de Ester. Vindo, pois, o rei e Hamã ao
banquete, que Ester tinha preparado, disse o rei a Ester, no
banquete do vinho: Qual é a tua petição? E ser-te-á concedida, e
qual é o teu desejo? E se fará ainda até metade do reino. Então
respondeu Ester, e disse: Minha petição e desejo é: Se achei
graça aos olhos do rei, e se bem parecer ao rei conceder-me a minha
petição, e cumprir o meu desejo, venha o rei com Hamã ao banquete
que lhes hei de preparar, e amanhã farei conforme a palavra do rei.

Então saiu Hamã naquele dia alegre e de bom ânimo; porém, vendo
Mardoqueu à porta do rei, e que ele não se levantara nem se movera
diante dele, então Hamã se encheu de furor contra Mardoqueu.
Hamã, porém, se refreou, e foi para sua casa; e enviou, e
mandou vir os seus amigos, e Zeres, sua mulher. E contou-lhes
Hamã a glória das suas riquezas, a multidão de seus filhos, e tudo
em que o rei o tinha engrandecido, e como o tinha exaltado sobre os
príncipes e servos do rei. Disse mais Hamã: Tampouco a rainha
Ester a ninguém fez vir com o rei ao banquete que tinha preparado,
senão a mim; e também para amanhã estou convidado por ela juntamente
com o rei. Porém tudo isto não me satisfaz, enquanto eu vir o
judeu Mardoqueu assentado à porta do rei. Então lhe disseram
Zeres, sua mulher, e todos os seus amigos: Faça-se uma forca de
cinqüenta côvados de altura, e amanhã dize ao rei que nela seja
enforcado Mardoqueu; e então entra alegre com o rei ao banquete. E
este conselho bem pareceu a Hamã, que mandou fazer a forca.

\medskip

\lettrine{6} Naquela mesma noite fugiu o sono do rei; então
mandou trazer o livro de registro das crônicas, as quais se leram
diante do rei. E achou-se escrito que Mardoqueu tinha denunciado
Bigtã e Teres, dois dos camareiros do rei, da guarda da porta, que
tinham procurado lançar mão do rei Assuero. Então disse o rei:
Que honra e distinção se deu por isso a Mardoqueu? E os servos do
rei, que ministravam junto a ele, disseram: Coisa nenhuma se lhe
fez.

Então disse o rei: Quem está no pátio? E Hamã tinha entrado no
pátio exterior da casa do rei, para dizer ao rei que enforcassem a
Mardoqueu na forca que lhe tinha preparado. E os servos do rei
lhe disseram: Eis que Hamã está no pátio. E disse o rei que
entrasse. E, entrando Hamã, o rei lhe disse: Que se fará ao
homem de cuja honra o rei se agrada? Então Hamã disse no seu
coração: De quem se agradaria o rei para lhe fazer honra mais do que
a mim? Assim disse Hamã ao rei: Para o homem, de cuja honra o
rei se agrada, tragam a veste real que o rei costuma vestir,
como também o cavalo em que o rei costuma andar montado, e
ponha-se-lhe a coroa real na sua cabeça. E entregue-se a veste e
o cavalo à mão de um dos príncipes mais nobres do rei, e vistam
delas aquele homem a quem o rei deseja honrar; e levem-no a cavalo
pelas ruas da cidade, e apregoe-se diante dele: Assim se fará ao
homem a quem o rei deseja honrar! Então disse o rei a Hamã:
Apressa-te, toma a veste e o cavalo, como disseste, e faze assim
para com o judeu Mardoqueu, que está assentado à porta do rei; e
coisa nenhuma omitas de tudo quanto disseste. E Hamã tomou a
veste e o cavalo, e vestiu a Mardoqueu, e o levou a cavalo pelas
ruas da cidade, e apregoou diante dele: Assim se fará ao homem a
quem o rei deseja honrar!

Depois disto Mardoqueu voltou para a porta do rei; porém Hamã se
retirou correndo à sua casa, triste, e de cabeça coberta. E
contou Hamã a Zeres, sua mulher, e a todos os seus amigos, tudo
quanto lhe tinha sucedido. Então os seus sábios e Zeres, sua mulher,
lhe disseram: Se Mardoqueu, diante de quem já começaste a cair, é da
descendência dos judeus, não prevalecerás contra ele, antes
certamente cairás diante dele. E estando eles ainda falando
com ele, chegaram os camareiros do rei, e se apressaram a levar Hamã
ao banquete que Ester preparara.

\medskip

\lettrine{7} Vindo, pois, o rei com Hamã, para beber com a
rainha Ester, disse outra vez o rei a Ester, no segundo dia, no
banquete do vinho: Qual é a tua petição, rainha Ester? E se te dará.
E qual é o teu desejo? Até metade do reino, se te dará. Então
respondeu a rainha Ester, e disse: Se, ó rei, achei graça aos teus
olhos, e se bem parecer ao rei, dê-se-me a minha vida como minha
petição, e o meu povo como meu desejo. Porque fomos vendidos, eu
e o meu povo, para nos destruírem, matarem, e aniquilarem de vez; se
ainda por servos e por servas nos vendessem, calar-me-ia; ainda que
o opressor não poderia ter compensado a perda do rei. Então
falou o rei Assuero, e disse à rainha Ester: Quem é esse e onde está
esse, cujo coração o instigou a assim fazer? E disse Ester: O
homem, o opressor, e o inimigo, é este mau Hamã. Então Hamã se
perturbou perante o rei e a rainha.

E o rei no seu furor se levantou do banquete do vinho e passou
para o jardim do palácio; e Hamã se pôs em pé, para rogar à rainha
Ester pela sua vida; porque viu que já o mal lhe estava determinado
pelo rei. Tornando, pois, o rei do jardim do palácio à casa do
banquete do vinho, Hamã tinha caído prostrado sobre o leito em que
estava Ester. Então disse o rei: Porventura quereria ele também
forçar a rainha perante mim nesta casa? Saindo esta palavra da boca
do rei, cobriram o rosto de Hamã. Então disse Harbona, um dos
camareiros que serviam diante do rei: Eis que também a forca de
cinqüenta côvados de altura que Hamã fizera para Mardoqueu, que
falara em defesa do rei, está junto à casa de Hamã. Então disse o
rei: Enforcai-o nela. Enforcaram, pois, a Hamã na forca, que
ele tinha preparado para Mardoqueu. Então o furor do rei se aplacou.

\medskip

\lettrine{8} Naquele mesmo dia deu o rei Assuero à rainha
Ester a casa de Hamã, inimigo dos judeus; e Mardoqueu veio perante o
rei, porque Ester tinha declarado quem ele era. E tirou o rei o
seu anel, que tinha tomado de Hamã e o deu a Mardoqueu. E Ester
encarregou Mardoqueu da casa de Hamã.

Falou mais Ester perante o rei, e se lhe lançou aos seus pés; e
chorou, e lhe suplicou que revogasse a maldade de Hamã, o agagita, e
o intento que tinha projetado contra os judeus. E estendeu o rei
para Ester o cetro de ouro. Então Ester se levantou, e pôs-se em pé
perante o rei, e disse: Se bem parecer ao rei, e se eu achei
graça perante ele, e se este negócio é reto diante do rei, e se eu
lhe agrado aos seus olhos, escreva-se que se revoguem as cartas
concebidas por Hamã filho de Hamedata, o agagita, as quais ele
escreveu para aniquilar os judeus, que estão em todas as províncias
do rei. Pois como poderei ver o mal que sobrevirá ao meu povo? E
como poderei ver a destruição da minha parentela? Então disse o
rei Assuero à rainha Ester e ao judeu Mardoqueu: Eis que dei a Ester
a casa de Hamã, e a ele penduraram numa forca, porquanto estendera
as mãos contra os judeus. Escrevei, pois, aos judeus, como
parecer bem aos vossos olhos, em nome do rei, e selai-o com o anel
do rei; porque o documento que se escreve em nome do rei, e que se
sela com o anel do rei, não se pode revogar. Então foram
chamados os escrivães do rei, naquele mesmo tempo, no terceiro mês
(que é o mês de Sivã), aos vinte e três dias; e se escreveu conforme
a tudo quanto ordenou Mardoqueu aos judeus, como também aos
sátrapas\footnote{Na antiga Pérsia, governador de uma satrapia.}, e
aos governadores, e aos líderes das províncias, que se estendem da
Índia até Etiópia, cento e vinte e sete províncias, a cada província
segundo o seu modo de escrever, e a cada povo conforme a sua língua;
como também aos judeus segundo o seu modo de escrever, e conforme a
sua língua. E escreveu-se em nome do rei Assuero e,
selando-as com o anel do rei, enviaram as cartas pela mão de
correios a cavalo, que cavalgavam sobre ginetes\footnote{Cavaleiro
armado de lança e adaga; cavalo bem proporcionado, adestrado e de
boa raça.}, que eram das cavalariças do rei.\footnote{King James:
And he wrote in the king Ahasuerus' name, and sealed it with the
king's ring, and sent letters by posts \emph{on horseback, and
riders on mules, camels, and young dromedaries}.} Nelas o rei
concedia aos judeus, que havia em cada cidade, que se reunissem, e
se dispusessem para defenderem as suas vidas, e para destruírem,
matarem e aniquilarem todas as forças do povo e da província que
viessem contra eles, crianças e mulheres, e que se saqueassem os
seus bens, num mesmo dia, em todas as províncias do rei
Assuero, no dia treze do duodécimo mês, que é o mês de Adar;
e uma cópia da carta seria divulgada como decreto em todas as
províncias, e publicada entre todos os povos, para que os judeus
estivessem preparados para aquele dia, para se vingarem dos seus
inimigos. Os correios, sobre ginetes velozes, saíram
apressuradamente, impelidos pela palavra do rei; e esta ordem foi
publicada na fortaleza de Susã.

Então Mardoqueu saiu da presença do rei com veste real
azul-celeste e branco, como também com uma grande coroa de ouro, e
com uma capa de linho fino e púrpura, e a cidade de Susã exultou e
se alegrou. E para os judeus houve luz, e alegria, e gozo, e
honra. Também em toda a província, e em toda a cidade, aonde
chegava a palavra do rei e a sua ordem, havia entre os judeus
alegria e gozo, banquetes e dias de folguedo; e muitos, dos povos da
terra, se fizeram judeus, porque o temor dos judeus tinha caído
sobre eles.

\medskip

\lettrine{9} E, NO duodécimo mês, que é o mês de Adar, no dia
treze do mesmo mês em que chegou a palavra do rei e a sua ordem para
se executar, no dia em que os inimigos dos judeus esperavam
assenhorear-se deles, sucedeu o contrário, porque os judeus foram os
que se assenhorearam dos que os odiavam. Porque os judeus nas
suas cidades, em todas as províncias do rei Assuero, se ajuntaram
para pôr as mãos naqueles que procuravam o seu mal; e ninguém podia
resistir-lhes, porque o medo deles caíra sobre todos aqueles povos.
E todos os líderes das províncias, e os sátrapas, e os
governadores, e os que faziam a obra do rei, auxiliavam os judeus
porque tinha caído sobre eles o temor de Mardoqueu. Porque
Mardoqueu era grande na casa do rei, e a sua fama crescia por todas
as províncias, porque o homem Mardoqueu ia sendo engrandecido.
Feriram, pois, os judeus a todos os seus inimigos, a golpes de
espada, com matança e com destruição; e fizeram dos seus inimigos o
que quiseram. E na fortaleza de Susã os judeus mataram e
destruíram quinhentos homens; como também a Parsandata, e a
Dalfom e a Aspata, e a Porata, e a Adalia, e a Aridata, e a
Farmasta, e a Arisai, e a Aridai, e a Vaisata; os dez filhos
de Hamã, filho de Hamedata, o inimigo dos judeus, mataram, porém ao
despojo não estenderam a sua mão. No mesmo dia foi comunicado
ao rei o número dos mortos na fortaleza de Susã. E disse o
rei à rainha Ester: Na fortaleza de Susã os judeus mataram e
destruíram quinhentos homens, e os dez filhos de Hamã; nas mais
províncias do rei que teriam feito? Qual é, pois, a tua petição? E
dar-se-te-á. Ou qual é ainda o teu requerimento? E far-se-á.
Então disse Ester: Se bem parecer ao rei, conceda-se aos
judeus que se acham em Susã que também façam amanhã conforme ao
mandado de hoje; e pendurem numa forca os dez filhos de Hamã.
 Então disse o rei que assim se fizesse; e publicou-se um edito em
Susã, e enforcaram os dez filhos de Hamã. E reuniram-se os
judeus que se achavam em Susã também no dia catorze do mês de Adar,
e mataram em Susã trezentos homens; porém ao despojo não estenderam
a sua mão. Também os demais judeus que se achavam nas
províncias do rei se reuniram e se dispuseram em defesa das suas
vidas, e tiveram descanso dos seus inimigos; e mataram dos seus
inimigos setenta e cinco mil; porém ao despojo não estenderam a sua
mão. Sucedeu isto no dia treze do mês de Adar; e descansaram
no dia catorze, e fizeram, daquele dia, dia de banquetes e de
alegria. Também os judeus, que se achavam em Susã se
ajuntaram nos dias treze e catorze do mesmo; e descansaram no dia
quinze, e fizeram, daquele dia, dia de banquetes e de alegria.
Os judeus, porém, das aldeias, que habitavam nas vilas,
fizeram do dia catorze do mês de Adar dia de alegria e de banquetes,
e dia de folguedo, e de mandarem presentes uns aos outros.

E Mardoqueu escreveu estas coisas, e enviou cartas a todos os
judeus que se achavam em todas as províncias do rei Assuero, aos de
perto, e aos de longe, ordenando-lhes que guardassem o dia
catorze do mês de Adar, e o dia quinze do mesmo, todos os anos,
como os dias em que os judeus tiveram repouso dos seus
inimigos, e o mês que se lhes mudou de tristeza em alegria, e de
luto em dia de festa, para que os fizessem dias de banquetes e de
alegria, e de mandarem presentes uns aos outros, e dádivas aos
pobres. E os judeus encarregaram-se de fazer o que já tinham
começado, como também o que Mardoqueu lhes tinha escrito.
Porque Hamã, filho de Hamedata, o agagita, inimigo de todos
os judeus, tinha intentado destruir os judeus, e tinha lançado Pur,
isto é, a sorte, para os assolar e destruir. Mas, vindo isto
perante o rei, mandou ele por cartas que o mau intento que Hamã
formara contra os judeus, se tornasse sobre a sua cabeça; pelo que
penduraram a ele e a seus filhos numa forca. Por isso aqueles
dias chamam Purim, do nome Pur; assim também por causa de todas as
palavras daquela carta, e do que viram sobre isso, e do que lhes
tinha sucedido, confirmaram os judeus, e tomaram sobre si, e
sobre a sua descendência, e sobre todos os que se achegassem a eles,
que não se deixaria de guardar estes dois dias conforme ao que se
escrevera deles, e segundo o seu tempo determinado, todos os anos.
E que estes dias seriam lembrados e guardados em cada
geração, família, província e cidade, e que esses dias de Purim não
fossem revogados entre os judeus, e que a memória deles nunca teria
fim entre os de sua descendência. Então a rainha Ester, filha
de Abiail, e Mardoqueu, o judeu, escreveram com toda autoridade uma
segunda vez, para confirmar a carta a respeito de Purim. E
mandaram cartas a todos os judeus, às cento e vinte e sete
províncias do reino de Assuero, com palavras de paz e verdade.
Para confirmarem estes dias de Purim nos seus tempos
determinados, como Mardoqueu, o judeu, e a rainha Ester lhes tinham
estabelecido, e como eles mesmos já o tinham estabelecido sobre si e
sobre a sua descendência, acerca do jejum e do seu clamor. E
o mandado de Ester estabeleceu os sucessos daquele Purim; e
escreveu-se no livro.

\medskip

\lettrine{10} Depois disto impôs o rei Assuero tributo sobre a
terra, e sobre as ilhas do mar. E todos os atos do seu poder e
do seu valor, e o relato da grandeza de Mardoqueu, a quem o rei
exaltou, porventura não estão escritos no livro das crônicas dos
reis da Média e da Pérsia? Porque o judeu Mardoqueu foi o
segundo depois do rei Assuero, e grande entre os judeus, e estimado
pela multidão de seus irmãos, procurando o bem do seu povo, e
proclamando a prosperidade de toda a sua descendência.

