\addchap{Juízes}

\lettrine{1} E sucedeu, depois da morte de Josué, que os
filhos de Israel perguntaram ao Senhor, dizendo: Quem dentre nós
primeiro subirá aos cananeus, para pelejar contra eles? E disse
o Senhor: Judá subirá; eis que entreguei esta terra na sua mão.
Então disse Judá a Simeão, seu irmão: Sobe comigo à minha
herança. E pelejemos contra os cananeus, e também eu contigo subirei
à tua herança. E Simeão partiu com ele. E subiu Judá, e o Senhor
lhe entregou na sua mão os cananeus e os perizeus; e feriram deles,
em Bezeque, a dez mil homens. E acharam Adoni-Bezeque em
Bezeque, e pelejaram contra ele; e feriram aos cananeus e aos
perizeus. Porém Adoni-Bezeque fugiu, mas o seguiram, e
prenderam-no e cortaram-lhe os dedos polegares das mãos e dos pés.
Então disse Adoni-Bezeque: Setenta reis, com os dedos polegares
das mãos e dos pés cortados, apanhavam as migalhas debaixo da minha
mesa; assim como eu fiz, assim Deus me pagou. E levaram-no a
Jerusalém, e morreu ali. E os filhos de Judá pelejaram contra
Jerusalém, e tomando-a, feriram-na ao fio da espada; e puseram fogo
na cidade.

E depois os filhos de Judá desceram a pelejar contra os cananeus,
que habitavam nas montanhas, e no sul, e nas planícies. E
partiu Judá contra os cananeus que habitavam em Hebrom (era porém
outrora o nome de Hebrom, Quiriate-Arba), e feriram a Sesai, e a
Aimã e Talmai. E dali partiu contra os moradores de Debir; e
era outrora o nome de Debir, Quiriate-Sefer. E disse Calebe:
Quem ferir a Quiriate-Sefer, e a tomar, lhe darei a minha filha Acsa
por mulher. E tomou-a Otniel, filho de Quenaz, o irmão de
Calebe, mais novo do que ele; e Calebe lhe deu a sua filha Acsa por
mulher. E sucedeu que, indo ela a ele, a persuadiu que
pedisse um campo a seu pai; e ela desceu do jumento, e Calebe lhe
disse: Que é que tens? E ela lhe disse: Dá-me uma bênção;
pois me deste uma terra seca, dá-me também fontes de águas. E Calebe
lhe deu as fontes superiores e as fontes inferiores. Também
os filhos do queneu, sogro de Moisés, subiram da cidade das
palmeiras com os filhos de Judá ao deserto de Judá, que está ao sul
de Arade, e foram, e habitaram com o povo. E foi Judá com
Simeão, seu irmão, e feriram aos cananeus que habitavam em Zefate; e
totalmente a destruíram, e chamou-se o nome desta cidade Hormá.
Tomou mais Judá a Gaza com o seu termo, e a Ascalom com o seu
termo, e a Ecrom com o seu termo. E estava o Senhor com Judá,
e despovoou as montanhas; porém não expulsou aos moradores do vale,
porquanto tinham carros de ferro. E deram Hebrom a Calebe,
como Moisés o dissera; e dali expulsou os três filhos de Anaque.

Porém os filhos de Benjamim não expulsaram os jebuseus que
habitavam em Jerusalém; antes os jebuseus ficaram habitando com os
filhos de Benjamim em Jerusalém, até ao dia de hoje. E subiu
também a casa de José contra Betel, e foi o Senhor com eles.
E a casa de José mandou espias a Betel, e foi antes o nome
desta cidade Luz. E viram os espias a um homem, que saía da
cidade, e lhe disseram: Ora, mostra-nos a entrada da cidade, e
usaremos contigo de misericórdia. E, mostrando-lhes ele a
entrada da cidade, feriram-na ao fio da espada; porém àquele homem e
a toda a sua família deixaram ir. Então aquele homem se foi à
terra dos heteus, e edificou uma cidade, e chamou o seu nome Luz;
este é o seu nome até ao dia de hoje. Manassés não expulsou
os habitantes de Bete-Seã, nem mesmo dos lugares da sua jurisdição;
nem a Taanaque, com os lugares da sua jurisdição; nem os moradores
de Dor, com os lugares da sua jurisdição; nem os moradores de
Ibleão, com os lugares da sua jurisdição; nem os moradores de
Megido, com os lugares da sua jurisdição; e resolveram os cananeus
habitar na mesma terra. E sucedeu que, quando Israel cobrou
mais forças, fez dos cananeus tributários; porém não os expulsou de
todo. Tampouco expulsou Efraim os cananeus que habitavam em
Gezer; antes os cananeus ficaram habitando com ele, em Gezer.
Tampouco expulsou Zebulom os moradores de Quitrom, nem os
moradores de Naalol; porém os cananeus ficaram habitando com ele, e
foram tributários. Tampouco Aser expulsou os moradores de
Aco, nem os moradores de Sidom; como nem de Alabe, nem de Aczibe,
nem de Helba, nem de Afeque, nem de Reobe; porém os aseritas
habitaram no meio dos cananeus que habitavam na terra; porquanto não
os expulsaram. Tampouco Naftali expulsou os moradores de
Bete-Semes, nem os moradores de Bete-Anate; mas habitou no meio dos
cananeus que habitavam na terra; porém lhes foram tributários os
moradores de Bete-Semes e Bete-Anate. E os amorreus impeliram
os filhos de Dã até às montanhas; porque nem os deixavam descer ao
vale. Também os amorreus quiseram habitar nas montanhas de
Heres, em Aijalom e em Saalbim; porém prevaleceu a mão da casa de
José, e ficaram tributários. E foi o termo dos amorreus desde
a subida de Acrabim, desde a penha\footnote{Grande massa de rocha
isolada e saliente, penhasco, penedo.}, e dali para cima.

\medskip

\lettrine{2} E subiu o anjo do Senhor de Gilgal a Boquim, e
disse: Do Egito vos fiz subir, e vos trouxe à terra que a vossos
pais tinha jurado e disse: Nunca invalidarei a minha aliança
convosco. E, quanto a vós, não fareis acordo com os moradores
desta terra, antes derrubareis os seus altares; mas vós não
obedecestes à minha voz. Por que fizestes isso? Assim também eu
disse: Não os expulsarei de diante de vós; antes estarão como
espinhos nas vossas ilhargas, e os seus deuses vos serão por laço.
E sucedeu que, falando o anjo do Senhor estas palavras a todos
os filhos de Israel, o povo levantou a sua voz e chorou. Por
isso chamaram àquele lugar, Boquim; e sacrificaram ali ao Senhor.

E havendo Josué despedido o povo foram-se os filhos de Israel,
cada um à sua herança, para possuírem a terra. E serviu o povo
ao Senhor todos os dias de Josué, e todos os dias dos anciãos que
ainda sobreviveram depois de Josué, e viram toda aquela grande obra
do Senhor, que fizera a Israel. Faleceu, porém, Josué, filho de
Num, servo do Senhor, com a idade de cento e dez anos; e
sepultaram-no no termo da sua herança, em Timnate-Heres, no monte de
Efraim, para o norte do monte de Gaás. E foi também
congregada toda aquela geração a seus pais, e outra geração após ela
se levantou, que não conhecia ao Senhor, nem tampouco a obra que ele
fizera a Israel. Então fizeram os filhos de Israel o que era
mau aos olhos do Senhor; e serviram aos baalins. E deixaram
ao Senhor Deus de seus pais, que os tirara da terra do Egito, e
foram-se após outros deuses, dentre os deuses dos povos, que havia
ao redor deles, e adoraram a eles; e provocaram o Senhor à ira.
Porquanto deixaram ao Senhor, e serviram a Baal e a Astarote.
Por isso a ira do Senhor se acendeu contra Israel, e os
entregou na mão dos espoliadores que os despojaram; e os entregou na
mão dos seus inimigos ao redor; e não puderam mais resistir diante
dos seus inimigos. Por onde quer que saíam, a mão do Senhor
era contra eles para mal, como o Senhor tinha falado, e como o
Senhor lhes tinha jurado; e estavam em grande aflição. E
levantou o Senhor juízes, que os livraram da mão dos que os
despojaram. Porém tampouco ouviram aos juízes, antes
prostituíram-se após outros deuses, e adoraram a eles; depressa se
desviaram do caminho, por onde andaram seus pais, obedecendo os
mandamentos do Senhor; mas eles assim não fizeram. E, quando
o Senhor lhes levantava juízes, o Senhor era com o juiz, e os
livrava da mão dos seus inimigos, todos os dias daquele juiz;
porquanto o Senhor se compadecia deles pelo seu gemido, por causa
dos que os oprimiam e afligiam. Porém sucedia que, falecendo
o juiz, reincidiam e se corrompiam mais do que seus pais, andando
após outros deuses, servindo-os, e adorando-os; nada deixavam das
suas obras, nem do seu obstinado caminho. Por isso a ira do
Senhor se acendeu contra Israel, e disse: Porquanto este povo
transgrediu a minha aliança, que tinha ordenado a seus pais, e não
deram ouvidos à minha voz, tampouco desapossarei mais de
diante deles a nenhuma das nações, que Josué deixou, quando morreu;
para por elas provar a Israel, se há de guardar, ou não, o
caminho do Senhor, como seus pais o guardaram, para nele andar.
Assim o Senhor deixou ficar aquelas nações, e não as
desterrou logo, nem as entregou na mão de Josué.

\medskip

\lettrine{3} Estas, pois, são as nações que o Senhor deixou
ficar, para por elas provar a Israel, a saber, a todos os que não
sabiam de todas as guerras de Canaã. Tão-somente para que as
gerações dos filhos de Israel delas soubessem (para lhes ensinar a
guerra), pelo menos os que dantes não sabiam delas. Cinco
príncipes dos filisteus, e todos os cananeus, e sidônios, e heveus
que habitavam nas montanhas do Líbano desde o monte de Baal-Hermom,
até à entrada de Hamate. Estes, pois, ficaram, para por eles
provar a Israel, para saber se dariam ouvido aos mandamentos do
Senhor, que ele tinha ordenado a seus pais, pelo ministério de
Moisés. Habitando, pois, os filhos de Israel no meio dos
cananeus, dos heteus, e amorreus, e perizeus, e heveus, e jebuseus,
tomaram de suas filhas para si por mulheres, e deram as suas
filhas aos filhos deles; e serviram aos seus deuses. E os filhos
de Israel fizeram o que era mau aos olhos do Senhor, e se esqueceram
do Senhor seu Deus; e serviram aos baalins e a Astarote.

Então a ira do Senhor se acendeu contra Israel, e ele os vendeu na
mão de Cusã-Risataim, rei da Mesopotâmia; e os filhos de Israel
serviram a Cusã-Risataim oito anos. E os filhos de Israel
clamaram ao Senhor, e o Senhor levantou-lhes um libertador, que os
libertou: Otniel, filho de Quenaz, irmão de Calebe, mais novo do que
ele. E veio sobre ele o Espírito do Senhor, e julgou a
Israel, e saiu à peleja; e o Senhor entregou na sua mão a
Cusã-Risataim, rei da Síria; contra o qual prevaleceu a sua mão.
Então a terra sossegou quarenta anos; e Otniel, filho de
Quenaz, faleceu.

Porém os filhos de Israel tornaram a fazer o que era mau aos
olhos do Senhor; então o Senhor fortaleceu a Eglom, rei dos
moabitas, contra Israel; porquanto fizeram o que era mau aos olhos
do Senhor. E reuniu consigo os filhos de Amom e os
amalequitas, e foi, e feriu a Israel, e tomaram a cidade das
palmeiras. E os filhos de Israel serviram a Eglom, rei dos
moabitas, dezoito anos. Então os filhos de Israel clamaram ao
Senhor, e o Senhor lhes levantou um libertador, a Eúde, filho de
Gera, filho de Jemim, homem canhoto. E os filhos de Israel enviaram
pela sua mão um presente a Eglom, rei dos moabitas. E Eúde
fez para si uma espada de dois fios, do comprimento de um côvado; e
cingiu-a por baixo das suas vestes, à sua coxa direita. E
levou aquele presente a Eglom, rei dos moabitas; e era Eglom homem
muito gordo. E sucedeu que, acabando de entregar o presente,
despediu a gente que o trouxera. Porém ele mesmo voltou das
imagens de escultura que estavam ao pé de Gilgal, e disse: Tenho uma
palavra secreta para ti, ó rei. O qual disse: Cala-te. E todos os
que lhe assistiam saíram de diante dele. E Eúde entrou numa
sala de verão, que o rei tinha só para si, onde estava sentado, e
disse: Tenho, para dizer-te, uma palavra de Deus. E levantou-se da
cadeira. Então Eúde estendeu a sua mão esquerda, e tirou a
espada de sobre sua coxa direita, e lha cravou no ventre, de
tal maneira que entrou até o cabo após a lâmina, e a gordura
encerrou a lâmina (porque não tirou a espada do ventre); e saiu-lhe
o excremento. Então Eúde saiu ao pátio, e fechou as portas da
sala e as trancou. E, saindo ele, vieram os servos do rei, e
viram, e eis que as portas da sala estavam fechadas; e disseram: Sem
dúvida está cobrindo seus pés na recâmara da sala de verão.
E, esperando até se alarmarem, eis que ele não abria as
portas da sala; então tomaram a chave, e abriram, e eis ali seu
senhor estendido morto em terra. E Eúde escapou, enquanto
eles se demoravam; porque ele passou pelas imagens de escultura, e
escapou para Seirá. E sucedeu que, chegando ele, tocou a
buzina nas montanhas de Efraim, e os filhos de Israel desceram com
ele das montanhas, e ele adiante deles. E disse-lhes:
Segui-me, porque o Senhor vos tem entregue vossos inimigos, os
moabitas, nas vossas mãos; e desceram após ele, e tomaram os
vaus\footnote{Trecho raso do rio ou do mar, onde se pode transitar a
pé ou a cavalo.} do Jordão contra Moabe, e a ninguém deixaram
passar. E naquele tempo feriram dos moabitas uns dez mil
homens, todos corpulentos, e todos homens valorosos; e não escapou
nenhum. Assim foi subjugado Moabe naquele dia debaixo da mão
de Israel; e a terra sossegou oitenta anos.

Depois dele foi Sangar, filho de Anate, que feriu a seiscentos
homens dos filisteus com uma aguilhada\footnote{Vara comprida com
ferrão na ponta, usada para tanger os bois.} de bois; e também ele
libertou a Israel.

\medskip

\lettrine{4} Porém os filhos de Israel tornaram a fazer o que
era mau aos olhos do Senhor, depois de falecer Eúde. E vendeu-os
o Senhor na mão de Jabim, rei de Canaã, que reinava em Hazor; e
Sísera era o capitão do seu exército, o qual então habitava em
Harosete dos gentios. Então os filhos de Israel clamaram ao
Senhor, porquanto ele tinha novecentos carros de ferro, e por vinte
anos oprimia violentamente os filhos de Israel.

E Débora, mulher profetisa, mulher de Lapidote, julgava a Israel
naquele tempo. Ela assentava-se debaixo das palmeiras de Débora,
entre Ramá e Betel, nas montanhas de Efraim; e os filhos de Israel
subiam a ela a juízo. E mandou chamar a Baraque, filho de
Abinoão de Quedes de Naftali, e disse-lhe: Porventura o Senhor Deus
de Israel não deu ordem, dizendo: Vai, e atrai gente ao monte Tabor,
e toma contigo dez mil homens dos filhos de Naftali e dos filhos de
Zebulom? E atrairei a ti para o ribeiro de Quisom, a Sísera,
capitão do exército de Jabim, com os seus carros, e com a sua
multidão; e o darei na tua mão. Então lhe disse Baraque: Se
fores comigo, irei; porém, se não fores comigo, não irei. E
disse ela: Certamente irei contigo, porém não será tua a honra da
jornada que empreenderes; pois à mão de uma mulher o Senhor venderá
a Sísera. E Débora se levantou, e partiu com Baraque para Quedes.

Então Baraque convocou a Zebulom e a Naftali em Quedes, e subiu
com dez mil homens após ele; e Débora subiu com ele. E Héber,
queneu, se tinha apartado dos queneus, dos filhos de Hobabe, sogro
de Moisés; e tinha estendido as suas tendas até ao carvalho de
Zaanaim, que está junto a Quedes, e anunciaram a Sísera que
Baraque, filho de Abinoão, tinha subido ao monte Tabor. E
Sísera convocou todos os seus carros, novecentos carros de ferro, e
todo o povo que estava com ele, desde Harosete dos gentios até ao
ribeiro de Quisom. Então disse Débora a Baraque: Levanta-te,
porque este é o dia em que o Senhor tem dado a Sísera na tua mão;
porventura o Senhor não saiu adiante de ti? Baraque, pois, desceu do
monte Tabor, e dez mil homens após ele. E o Senhor derrotou a
Sísera, e a todos os seus carros, e a todo o seu exército ao fio da
espada, diante de Baraque; e Sísera desceu do carro, e fugiu a pé.
E Baraque perseguiu os carros, e o exército, até Harosete dos
gentios; e todo o exército de Sísera caiu ao fio da espada, até não
ficar um só.

Porém Sísera fugiu a pé à tenda de Jael, mulher de Héber, queneu;
porquanto havia paz entre Jabim, rei de Hazor, e a casa de Héber,
queneu. E Jael saiu ao encontro de Sísera, e disse-lhe:
Entra, senhor meu, entra aqui, não temas. Ele entrou na sua tenda, e
ela o cobriu com uma coberta. Então ele lhe disse: Dá-me,
peço-te, de beber um pouco de água, porque tenho sede. Então ela
abriu um odre de leite, e deu-lhe de beber, e o cobriu. E ele
lhe disse: Põe-te à porta da tenda; e há de ser que se alguém vier e
te perguntar: Há aqui alguém? Responderás então: Não. Então
Jael, mulher de Héber, tomou uma estaca da tenda, e lançou mão de um
martelo, e chegou-se mansamente a ele, e lhe cravou a estaca na
fonte, de sorte que penetrou na terra, estando ele, porém, num
profundo sono, e já muito cansado; e assim morreu. E eis que,
seguindo Baraque a Sísera, Jael lhe saiu ao encontro, e disse-lhe:
Vem, e mostrar-te-ei o homem que buscas. E foi a ela, e eis que
Sísera jazia morto, com a estaca na fonte. Assim Deus naquele
dia sujeitou a Jabim, rei de Canaã, diante dos filhos de Israel.
E continuou a mão dos filhos de Israel a pesar e a
endurecer-se sobre Jabim, rei de Canaã; até que exterminaram a
Jabim, rei de Canaã.

\medskip

\lettrine{5} E cantou Débora e Baraque, filho de Abinoão,
naquele mesmo dia, dizendo: Louvai ao Senhor pela vingança de
Israel, quando o povo se ofereceu voluntariamente. Ouvi, reis;
dai ouvidos, príncipes; eu, eu cantarei ao Senhor; salmodiarei ao
Senhor Deus de Israel. Ó Senhor, saindo tu de Seir, caminhando
tu desde o campo de Edom, a terra estremeceu; até os céus gotejaram;
até as nuvens gotejaram águas. Os montes se derreteram diante do
Senhor, e até Sinai diante do Senhor Deus de Israel.

Nos dias de Sangar, filho de Anate, nos dias de Jael cessaram os
caminhos; e os que andavam por veredas iam por caminhos torcidos.
Cessaram as aldeias em Israel, cessaram; até que eu, Débora, me
levantei, por mãe em Israel me levantei. E se escolhia deuses
novos, logo a guerra estava às portas; via-se por isso escudo ou
lança entre quarenta mil em Israel? Meu coração é para os
legisladores de Israel, que voluntariamente se ofereceram entre o
povo; bendizei ao Senhor. Vós os que cavalgais sobre jumentas
brancas, que vos assentais em juízo, que andais pelo caminho, falai
disto. Donde se ouve o estrondo dos flecheiros, entre os
lugares onde se tiram águas, ali falai das justiças do Senhor, das
justiças que fez às suas aldeias em Israel; então o povo do Senhor
descia às portas.

Desperta, desperta, Débora, desperta, desperta, entoa um cântico;
levanta-te, Baraque, e leva presos os teus cativos, tu, filho de
Abinoão. Então fez dominar sobre os nobres entre o povo, aos
que restaram; fez-me o Senhor dominar sobre os poderosos. De
Efraim saiu a sua raiz contra Amaleque; e depois de ti vinha
Benjamim dentre os teus povos; de Maquir desceram os legisladores, e
de Zebulom os que levaram a cana do escriba. Também os
principais de Issacar foram com Débora; e como Issacar, assim também
Baraque, foi enviado a pé para o vale; nas divisões de Rúben foram
grandes as resoluções do coração. Por que ficaste tu entre os
currais para ouvires os balidos\footnote{Grito de ovelha ou de
cordeiro.} dos rebanhos? Nas divisões de Rúben tiveram grandes
esquadrinhações do coração. Gileade ficou além do Jordão, e
Dã por que se deteve nos navios? Aser se assentou na beira dos
mares, e ficou junto às suas baías. Zebulom é um povo que
expôs a sua vida à morte, como também Naftali, nas alturas do campo.
Vieram reis, pelejaram; então pelejaram os reis de Canaã em
Taanaque, junto às águas de Megido; não tomaram despojo de prata.
Desde os céus pelejaram; até as estrelas desde os lugares dos
seus cursos pelejaram contra Sísera. O ribeiro de Quisom os
arrastou, aquele antigo ribeiro, o ribeiro de Quisom. Pisaste, ó
minha alma, à força. Então os cascos dos cavalos se
despedaçaram; pelo galopar, o galopar dos seus valentes.
Amaldiçoai a Meroz, diz o anjo do Senhor, acremente
amaldiçoai aos seus moradores; porquanto não vieram ao socorro do
Senhor, ao socorro do Senhor com os valorosos.

Bendita seja entre as mulheres, Jael, mulher de Héber, o queneu;
bendita seja entre as mulheres nas tendas. Água pediu ele,
leite lhe deu ela; em prato de nobres lhe ofereceu manteiga.
À estaca estendeu a sua mão esquerda, e ao martelo dos
trabalhadores a sua direita; e matou a Sísera, e rachou-lhe a
cabeça, quando lhe pregou e atravessou as fontes. Entre os
seus pés se encurvou, caiu, ficou estirado; entre os seus pés se
encurvou, caiu; onde se encurvou, ali ficou abatido. A mãe de
Sísera olhava pela janela, e exclamava pela grade: Por que tarda em
vir o seu carro? Por que se demoram os ruídos dos seus carros?
As mais sábias das suas damas responderam; e até ela
respondia a si mesma: Porventura não achariam e repartiriam
despojos? Uma ou duas moças a cada homem? Para Sísera despojos de
estofos coloridos, despojos de estofos coloridos bordados; de
estofos coloridos bordados de ambos os lados como despojo para os
pescoços. Assim, ó Senhor, pereçam todos os teus inimigos!
Porém os que te amam sejam como o sol quando sai na sua força.
E sossegou a terra quarenta anos.

\medskip

\lettrine{6} Porém os filhos de Israel fizeram o que era mau
aos olhos do Senhor; e o Senhor os deu nas mãos dos midianitas por
sete anos. E, prevalecendo a mão dos midianitas sobre Israel,
fizeram os filhos de Israel para si, por causa dos midianitas, as
covas que estão nos montes, as cavernas e as fortificações.
Porque sucedia que, semeando Israel, os midianitas e os
amalequitas, e também os do oriente, contra ele subiam. E
punham-se contra ele em campo, e destruíam os frutos da terra, até
chegarem a Gaza; e não deixavam mantimento em Israel, nem ovelhas,
nem bois, nem jumentos. Porque subiam com os seus gados e
tendas; vinham como gafanhotos, em grande multidão que não se podia
contar, nem a eles nem aos seus camelos; e entravam na terra, para a
destruir. Assim Israel empobreceu muito pela presença dos
midianitas; então os filhos de Israel clamaram ao Senhor.

E sucedeu que, clamando os filhos de Israel ao Senhor por causa
dos midianitas, enviou o Senhor um profeta aos filhos de Israel,
que lhes disse: Assim diz o Senhor Deus de Israel: Do Egito eu vos
fiz subir, e vos tirei da casa da servidão; e vos livrei da mão
dos egípcios, e da mão de todos quantos vos oprimiam; e os expulsei
de diante de vós, e a vós dei a sua terra. E vos disse: Eu
sou o Senhor vosso Deus; não temais aos deuses dos amorreus, em cuja
terra habitais; mas não destes ouvidos à minha voz.

Então o anjo do Senhor veio, e assentou-se debaixo do carvalho
que está em Ofra, que pertencia a Joás, abiezrita; e Gideão, seu
filho, estava malhando o trigo no lagar, para o salvar dos
midianitas. Então o anjo do Senhor lhe apareceu, e lhe disse:
O Senhor é contigo, homem valoroso. Mas Gideão lhe respondeu:
Ai, Senhor meu, se o Senhor é conosco, por que tudo isto nos
sobreveio? E que é feito de todas as suas maravilhas que nossos pais
nos contaram, dizendo: Não nos fez o Senhor subir do Egito? Porém
agora o Senhor nos desamparou, e nos deu nas mãos dos midianitas.
Então o Senhor olhou para ele, e disse: Vai nesta tua força,
e livrarás a Israel das mãos dos midianitas; porventura não te
enviei eu? E ele lhe disse: Ai, Senhor meu, com que livrarei
a Israel? Eis que a minha família é a mais pobre em Manassés, e eu o
menor na casa de meu pai. E o Senhor lhe disse: Porquanto eu
hei de ser contigo, tu ferirás aos midianitas como se fossem um só
homem. E ele disse: Se agora tenho achado graça aos teus
olhos, dá-me um sinal de que és tu que falas comigo. Rogo-te
que daqui não te apartes, até que eu volte e traga o meu presente, e
o ponha perante ti. E disse: Eu esperarei até que voltes. E
entrou Gideão e preparou um cabrito e pães ázimos de um efa de
farinha; a carne pôs num cesto e o caldo pôs numa panela; e
trouxe-lho até debaixo do carvalho, e lho ofereceu. Porém o
anjo de Deus lhe disse: Toma a carne e os pães ázimos, e põe-nos
sobre esta penha e derrama-lhe o caldo. E assim fez. E o anjo
do Senhor estendeu a ponta do cajado, que estava na sua mão, e tocou
a carne e os pães ázimos; então subiu o fogo da penha, e consumiu a
carne e os pães ázimos; e o anjo do Senhor desapareceu de seus
olhos. Então viu Gideão que era o anjo do Senhor e disse: Ah,
Senhor DEUS, pois vi o anjo do Senhor face a face. Porém o
Senhor lhe disse: Paz seja contigo; não temas; não morrerás.
Então Gideão edificou ali um altar ao Senhor, e chamou-lhe: O
SENHOR É PAZ; e ainda até o dia de hoje está em Ofra dos abiezritas.

E aconteceu naquela mesma noite, que o Senhor lhe disse: Toma o
boi que pertence a teu pai, a saber, o segundo boi de sete anos, e
derruba o altar de Baal, que é de teu pai; e corta o bosque que está
ao pé dele. E edifica ao Senhor teu Deus um altar no cume
deste lugar forte, num lugar conveniente; e toma o segundo boi, e o
oferecerás em holocausto com a lenha que cortares do bosque.
Então Gideão tomou dez homens dentre os seus servos, e fez
como o Senhor lhe dissera; e sucedeu que, temendo ele a casa de seu
pai, e os homens daquela cidade, não o fez de dia, mas fê-lo de
noite. Levantando-se, pois, os homens daquela cidade, de
madrugada, eis que estava o altar de Baal derrubado, e o bosque
estava ao pé dele, cortado; e o segundo boi oferecido no altar que
fora edificado. E uns aos outros disseram: Quem fez esta
coisa? E, esquadrinhando, e inquirindo, disseram: Gideão, o filho de
Joás, fez esta coisa. Então os homens daquela cidade disseram
a Joás: Tira para fora a teu filho para que morra, pois derribou o
altar de Baal e cortou o bosque que estava ao pé dele. Porém
Joás disse a todos os que se puseram contra ele: Contendereis vós
por Baal? Livrá-lo-eis vós? Qualquer que por ele contender ainda
esta manhã será morto; se é deus, por si mesmo contenda; pois
derrubaram o seu altar. Por isso naquele dia lhe chamaram
Jerubaal, dizendo: Baal contenda contra ele, pois derrubou o seu
altar.

E todos os midianitas e amalequitas, e os filhos do oriente se
ajuntaram, e passaram, e acamparam no vale de Jizreel. Então
o Espírito do Senhor revestiu a Gideão, o qual tocou a buzina, e os
abiezritas se ajuntaram após ele. E enviou mensageiros por
toda a tribo de Manassés, que também se ajuntou após ele; também
enviou mensageiros a Aser, e a Zebulom, e a Naftali, que saíram-lhe
ao encontro. E disse Gideão a Deus: Se hás de livrar a Israel
por minha mão, como disseste, eis que eu porei um velo de lã
na eira; se o orvalho estiver somente no velo, e toda a terra ficar
seca, então conhecerei que hás de livrar a Israel por minha mão,
como disseste. E assim sucedeu; porque no outro dia se
levantou de madrugada, e apertou o velo; e do orvalho que espremeu
do velo, encheu uma taça de água. E disse Gideão a Deus: Não
se acenda contra mim a tua ira, se ainda falar só esta vez; rogo-te
que só esta vez faça a prova com o velo; rogo-te que só o velo fique
seco, e em toda a terra haja o orvalho. E Deus assim fez
naquela noite; pois só o velo ficou seco, e sobre toda a terra havia
orvalho.

\medskip

\lettrine{7} Então Jerubaal (que é Gideão) se levantou de
madrugada, e todo o povo que com ele havia, e se acamparam junto à
fonte de Harode, de maneira que tinha o arraial dos midianitas para
o norte, no vale, perto do outeiro de Moré. E disse o Senhor a
Gideão: Muito é o povo que está contigo, para eu dar aos midianitas
em sua mão; a fim de que Israel não se glorie contra mim, dizendo: A
minha mão me livrou. Agora, pois, apregoa aos ouvidos do povo,
dizendo: Quem for medroso e tímido, volte, e retire-se
apressadamente das montanhas de Gileade. Então voltaram do povo
vinte e dois mil, e dez mil ficaram. E disse o Senhor a Gideão:
Ainda há muito povo; faze-os descer às águas, e ali os provarei; e
será que, daquele de que eu te disser: Este irá contigo, esse
contigo irá; porém de todo aquele, de que eu te disser: Este não irá
contigo, esse não irá. E fez descer o povo às águas. Então o
Senhor disse a Gideão: Qualquer que lamber as águas com a sua
língua, como as lambe o cão, esse porás à parte; como também a todo
aquele que se abaixar de joelhos a beber. E foi o número dos que
lamberam, levando a mão à boca, trezentos homens; e todo o restante
do povo se abaixou de joelhos a beber as águas. E disse o Senhor
a Gideão: Com estes trezentos homens que lamberam as águas vos
livrarei, e darei os midianitas na tua mão; portanto, todos os
demais se retirem, cada um ao seu lugar. E o povo tomou na sua
mão a provisão e as suas buzinas, e enviou a todos os outros homens
de Israel cada um à sua tenda, porém os trezentos homens reteve; e
estava o arraial dos midianitas embaixo, no vale.

E sucedeu que, naquela mesma noite, o Senhor lhe disse:
Levanta-te, e desce ao arraial, porque o tenho dado na tua mão.
E, se ainda temes descer, desce tu e teu moço Purá, ao
arraial; e ouvirás o que dizem, e então, fortalecidas as tuas
mãos descerás ao arraial. Então desceu ele com o seu moço Purá até
ao extremo das sentinelas que estavam no arraial. E os
midianitas, os amalequitas, e todos os filhos do oriente jaziam no
vale como gafanhotos em multidão; e eram inumeráveis os seus
camelos, como a areia que há na praia do mar. Chegando, pois,
Gideão, eis que estava contando um homem ao seu companheiro um
sonho, e dizia: Eis que tive um sonho, eis que um pão de cevada
torrado rodava pelo arraial dos midianitas, e chegava até à tenda, e
a feriu, e caiu, e a transtornou de cima para baixo; e ficou caída.
E respondeu o seu companheiro, e disse: Não é isto outra
coisa, senão a espada de Gideão, filho de Joás, varão israelita.
Deus tem dado na sua mão aos midianitas, e todo este arraial.
E sucedeu que, ouvindo Gideão a narração deste sonho, e a sua
explicação, adorou; e voltou ao arraial de Israel, e disse:
Levantai-vos, porque o Senhor tem dado o arraial dos midianitas nas
nossas mãos.

Então dividiu os trezentos homens em três companhias; e deu-lhes
a cada um, nas suas mãos, buzinas, e cântaros vazios, com tochas
neles acesas. E disse-lhes: Olhai para mim, e fazei como eu
fizer; e eis que, chegando eu à extremidade do arraial, será que,
como eu fizer, assim fareis vós. Tocando eu a buzina, eu e
todos os que comigo estiverem, então também vós tocareis a buzina ao
redor de todo o arraial, e direis: Espada do Senhor, e de Gideão.
Chegou, pois, Gideão, e os cem homens que com ele iam, ao
extremo do arraial, ao princípio da vigília da meia noite, havendo
sido de pouco trocadas as guardas; então tocaram as buzinas, e
quebraram os cântaros, que tinham nas mãos. Assim tocaram as
três companhias as buzinas, e quebraram os cântaros; e tinham nas
suas mãos esquerdas as tochas acesas, e nas suas mãos direitas as
buzinas, para tocarem, e clamaram: Espada do Senhor, e de Gideão.
E conservou-se cada um no seu lugar ao redor do arraial;
então todo o exército pôs-se a correr e, gritando, fugiu.
Tocando, pois, os trezentos as buzinas, o Senhor tornou a
espada de um contra o outro, e isto em todo o arraial, que fugiu
para Zererá, até Bete-Sita, até aos limites de Abel-Meolá, acima de
Tabate.

Então os homens de Israel, de Naftali, de Aser e de todo o
Manassés foram convocados, e perseguiram aos midianitas.
Também Gideão enviou mensageiros a todas as montanhas de
Efraim, dizendo: Descei ao encontro dos midianitas, e tomai-lhes as
águas até Bete-Bara, e também o Jordão. Convocados, pois, todos os
homens de Efraim, tomaram-lhes as águas até Bete-Bara e o Jordão.
E prenderam a dois príncipes dos midianitas, a Orebe e a
Zeebe; e mataram a Orebe na penha de Orebe, e a Zeebe mataram no
lagar de Zeebe, e perseguiram aos midianitas; e trouxeram as cabeças
de Orebe e de Zeebe a Gideão, além do Jordão.

\medskip

\lettrine{8} Então os homens de Efraim lhes disseram: Que é
isto que nos fizeste, que não nos chamaste, quando foste pelejar
contra os midianitas? E contenderam com ele fortemente. Porém
ele lhes disse: Que mais fiz eu agora do que vós? Não são porventura
os rabiscos de Efraim melhores do que a vindima de Abiezer? Deus
vos deu na vossa mão os príncipes dos midianitas, Orebe e Zeebe; que
mais pude eu fazer do que vós? Então a sua ira se abrandou para com
ele, quando falou esta palavra.

E, como Gideão veio ao Jordão, passou com os trezentos homens que
com ele estavam, já cansados, mas ainda perseguindo. E disse aos
homens de Sucote: Dai, peço-vos, alguns pedaços de pão ao povo, que
segue as minhas pisadas; porque estão cansados, e eu vou ao encalço
de Zeba e Salmuna, reis dos midianitas. Porém os príncipes de
Sucote disseram: Estão já, Zeba e Salmuna, em tua mão, para que
demos pão ao teu exército? Então disse Gideão: Pois quando o
Senhor der na minha mão a Zeba e a Salmuna, trilharei a vossa carne
com os espinhos do deserto, e com os abrolhos. E dali subiu a
Penuel, e falou-lhes da mesma maneira; e os homens de Penuel lhe
responderam como os homens de Sucote lhe haviam respondido. Por
isso também falou aos homens de Penuel, dizendo: Quando eu voltar em
paz, derribarei esta torre. Estavam, pois, Zeba e Salmuna em
Carcor, e os seus exércitos com eles, uns quinze mil homens, todos
os que restaram do exército dos filhos do oriente; e os que caíram
foram cento e vinte mil homens, que puxavam da espada. E
subiu Gideão pelo caminho dos que habitavam em tendas, para o
oriente de Nobá e Jogbeá; e feriu aquele exército, porquanto o
exército estava descuidado. E fugiram Zeba e Salmuna; porém
ele os perseguiu, e tomou presos a ambos os reis dos midianitas, a
Zeba e a Salmuna, e afugentou a todo o exército. Voltando,
pois, Gideão, filho de Joás, da peleja, antes do nascer do sol,
tomou preso a um moço dos homens de Sucote, e lhe fez
perguntas; o qual lhe deu por escrito os nomes dos príncipes de
Sucote, e dos seus anciãos, setenta e sete homens. Então veio
aos homens de Sucote, e disse: Vede aqui a Zeba e a Salmuna, a
respeito dos quais desprezivelmente me escarnecestes, dizendo: Estão
já, Zeba e Salmuna, na tua mão, para que demos pão aos teus homens,
já cansados? E tomou os anciãos daquela cidade, e os espinhos
do deserto, e os abrolhos; e com eles ensinou aos homens de Sucote.
E derrubou a torre de Penuel, e matou os homens da cidade.

Depois perguntou a Zeba e a Salmuna: Que homens eram os que
matastes em Tabor? E disseram: Como és tu, assim eram eles; cada um
parecia filho de rei. Então disse ele: Meus irmãos eram,
filhos de minha mãe; vive o Senhor, que, se os tivésseis deixado com
vida, eu não vos mataria. E disse a Jeter, seu primogênito:
Levanta-te, mata-os. Porém o moço não puxou da sua espada, porque
temia; porquanto ainda era jovem. Então disseram Zeba e
Salmuna: Levanta-te, e acomete-nos; porque, qual o homem, tal a sua
valentia. Levantou-se, pois, Gideão, e matou a Zeba e a Salmuna, e
tomou os ornamentos que estavam nos pescoços dos seus camelos.

Então os homens de Israel disseram a Gideão: Domina sobre nós,
tanto tu, como teu filho e o filho de teu filho; porquanto nos
livraste da mão dos midianitas. Porém Gideão lhes disse:
Sobre vós eu não dominarei, nem tampouco meu filho sobre vós
dominará; o Senhor sobre vós dominará. E disse-lhes mais
Gideão: Uma petição vos farei: Dá-me, cada um de vós, os pendentes
do seu despojo (porque tinham pendentes de ouro, porquanto eram
ismaelitas). E disseram eles: De boa vontade os daremos. E
estenderam uma capa, e cada um deles deitou ali um pendente do seu
despojo. E foi o peso dos pendentes de ouro, que pediu, mil e
setecentos siclos de ouro, afora os ornamentos, e as cadeias, e as
vestes de púrpura que traziam os reis dos midianitas, e afora as
coleiras que os camelos traziam ao pescoço. E fez Gideão dele
um éfode, e colocou-o na sua cidade, em Ofra; e todo o Israel
prostituiu-se ali após ele; e foi por tropeço a Gideão e à sua casa.
Assim foram abatidos os midianitas diante dos filhos de
Israel, e nunca mais levantaram a sua cabeça; e sossegou a terra
quarenta anos nos dias de Gideão.

E foi Jerubaal, filho de Joás, e habitou em sua casa. E
teve Gideão setenta filhos, que procederam dele, porque tinha muitas
mulheres. E sua concubina, que estava em Siquém, lhe deu à
luz também um filho; e pôs-lhe por nome Abimeleque. E faleceu
Gideão, filho de Joás, numa boa velhice; e foi sepultado no sepulcro
de seu pai Joás, em Ofra dos abiezritas. E sucedeu que, como
Gideão faleceu, os filhos de Israel tornaram a se prostituir após os
baalins; e puseram a Baal-Berite por deus. E assim os filhos
de Israel não se lembraram do Senhor seu Deus, que os livrara da mão
de todos os seus inimigos ao redor. Nem usaram de
beneficência com a casa de Jerubaal, a saber, de Gideão, conforme a
todo o bem que ele havia feito a Israel.

\medskip

\lettrine{9} E Abimeleque, filho de Jerubaal, foi a Siquém,
aos irmãos de sua mãe, e falou-lhes e a toda a geração da casa do
pai de sua mãe, dizendo: Falai, peço-vos, aos ouvidos de todos
os cidadãos de Siquém: Qual é melhor para vós, que setenta homens,
todos os filhos de Jerubaal, dominem sobre vós, ou que um homem
sobre vós domine? Lembrai-vos também de que sou osso vosso e carne
vossa. Então os irmãos de sua mãe falaram acerca dele perante os
ouvidos de todos os cidadãos de Siquém todas aquelas palavras; e o
coração deles se inclinou a seguir Abimeleque, porque disseram: É
nosso irmão. E deram-lhe setenta peças de prata, da casa de
Baal-Berite; e com elas alugou Abimeleque uns homens ociosos e
levianos, que o seguiram. E veio à casa de seu pai, a Ofra e
matou a seus irmãos, os filhos de Jerubaal, setenta homens, sobre
uma pedra. Porém Jotão, filho menor de Jerubaal, ficou, porque se
tinha escondido. Então se ajuntaram todos os cidadãos de Siquém,
e toda a casa de Milo; e foram, e constituíram a Abimeleque rei,
junto ao carvalho alto que está perto de Siquém.

E, dizendo-o a Jotão, foi e pôs-se no cume do monte de Gerizim, e
levantou a sua voz, e clamou e disse-lhes: Ouvi-me, cidadãos de
Siquém, e Deus vos ouvirá a vós. Foram uma vez as árvores a
ungir para si um rei, e disseram à oliveira: Reina tu sobre nós.
Porém a oliveira lhes disse: Deixaria eu a minha gordura, que
Deus e os homens em mim prezam, e iria pairar sobre as árvores?
Então disseram as árvores à figueira: Vem tu, e reina sobre
nós. Porém a figueira lhes disse: Deixaria eu a minha doçura,
o meu bom fruto, e iria pairar sobre as árvores? Então
disseram as árvores à videira: Vem tu, e reina sobre nós.
Porém a videira lhes disse: Deixaria eu o meu mosto, que
alegra a Deus e aos homens, e iria pairar sobre as árvores?
Então todas as árvores disseram ao espinheiro: Vem tu, e
reina sobre nós. E disse o espinheiro às árvores: Se, na
verdade, me ungis por rei sobre vós, vinde, e confiai-vos debaixo da
minha sombra; mas, se não, saia fogo do espinheiro que consuma os
cedros do Líbano. Agora, pois, se é que em verdade e
sinceridade agistes, fazendo rei a Abimeleque, e se bem fizestes
para com Jerubaal e para com a sua casa, e se com ele usastes
conforme ao merecimento das suas mãos

meu pai pelejou por vós, e desprezou a sua vida, e vos
livrou da mão dos midianitas; porém vós hoje vos levantastes
contra a casa de meu pai, e matastes a seus filhos, setenta homens,
sobre uma pedra; e a Abimeleque, filho da sua serva, fizestes reinar
sobre os cidadãos de Siquém, porque é vosso irmão); pois, se
em verdade e sinceridade usastes com Jerubaal e com a sua casa hoje,
alegrai-vos com Abimeleque, e também ele se alegre convosco.
Mas, se não, saia fogo de Abimeleque, e consuma aos cidadãos
de Siquém, e a casa de Milo; e saia fogo dos cidadãos de Siquém, e
da casa de Milo, que consuma a Abimeleque. Então partiu
Jotão, e fugiu e foi para Beer; e ali habitou por medo de
Abimeleque, seu irmão.

Havendo, pois, Abimeleque dominado três anos sobre Israel,
enviou Deus um mau espírito entre Abimeleque e os cidadãos de
Siquém; e estes se houveram aleivosamente contra Abimeleque;
para que a violência feita aos setenta filhos de Jerubaal
viesse, e o seu sangue caísse sobre Abimeleque, seu irmão, que os
matara, e sobre os cidadãos de Siquém, que fortaleceram as mãos dele
para matar a seus irmãos; e os cidadãos de Siquém puseram
contra ele quem lhe armasse emboscadas sobre os cumes dos montes; e
a todo aquele que passava pelo caminho junto a eles o assaltavam; e
contou-se isso a Abimeleque. Veio também Gaal, filho de
Ebede, com seus irmãos, e passaram a Siquém; e os cidadãos de Siquém
confiaram nele. E saíram ao campo, e vindimaram as suas
vinhas, e pisaram as uvas, e fizeram festas; e foram à casa de seu
deus, e comeram, e beberam, e amaldiçoaram a Abimeleque. E
disse Gaal, filho de Ebede: Quem é Abimeleque, e quem é Siquém, para
que o sirvamos? Não é porventura filho de Jerubaal? E não é Zebul o
seu mordomo? Servi antes aos homens de Hamor, pai de Siquém; pois,
por que razão serviríamos nós a ele? Ah! se este povo
estivera na minha mão, eu expulsaria a Abimeleque. E diria a
Abimeleque: Multiplica o teu exército, e sai. E, ouvindo
Zebul, o maioral da cidade, as palavras de Gaal, filho de Ebede, se
acendeu a sua ira; e enviou astutamente mensageiros a
Abimeleque, dizendo: Eis que Gaal, filho de Ebede, e seus irmãos
vieram a Siquém, e eis que eles estão sublevando esta cidade contra
ti. Levanta-te, pois, de noite, tu e o povo que tiveres
contigo, e põe emboscadas no campo. E levanta-te pela manhã
ao sair o sol, e dá de golpe sobre a cidade; e eis que, saindo
contra ti, ele e o povo que tiver com ele, faze-lhe como puderes.
Levantou-se, pois, Abimeleque, e todo o povo que com ele
havia, de noite, e puseram emboscadas a Siquém, com quatro tropas.
E Gaal, filho de Ebede, saiu, e pôs-se à entrada da porta da
cidade; e Abimeleque, e todo o povo que com ele havia, se levantou
das emboscadas. E, vendo Gaal aquele povo, disse a Zebul: Eis
que desce gente dos cumes dos montes. Zebul, ao contrário, lhe
disse: As sombras dos montes vês como se fossem homens. Porém
Gaal ainda tornou a falar, e disse: Eis ali desce gente do meio da
terra, e uma tropa vem do caminho do carvalho de Meonenim.
Então lhe disse Zebul: Onde está agora a tua boca, com a qual
dizias: Quem é Abimeleque, para que o sirvamos? Não é este
porventura o povo que desprezaste? Sai pois, peço-te, e peleja
contra ele. E saiu Gaal à vista dos cidadãos de Siquém, e
pelejou contra Abimeleque. E Abimeleque o perseguiu porquanto
fugiu de diante dele; e muitos feridos caíram até à entrada da porta
da cidade. E Abimeleque ficou em Aruma. E Zebul expulsou a
Gaal e a seus irmãos, para que não pudessem habitar em Siquém.
E sucedeu no dia seguinte que o povo saiu ao campo; disto foi
avisado Abimeleque. Então tomou o povo, e o repartiu em três
tropas, e pôs emboscadas no campo; e olhou, e eis que o povo saía da
cidade, e levantou-se contra ele, e o feriu. Porque
Abimeleque, e as tropas que com ele havia, romperam de improviso, e
pararam à entrada da porta da cidade; e as outras duas tropas deram
de improviso sobre todos quantos estavam no campo, e os feriram.
E Abimeleque pelejou contra a cidade todo aquele dia, e tomou
a cidade, e matou o povo que nela havia; e assolou a cidade, e a
semeou de sal. O que ouvindo todos os cidadãos da torre de
Siquém, entraram na fortaleza, na casa do deus Berite. E
contou-se a Abimeleque que todos os cidadãos da torre de Siquém se
haviam congregado. Subiu, pois, Abimeleque ao monte Salmom,
ele e todo o povo que com ele havia; e Abimeleque tomou na sua mão
um machado, e cortou um ramo de árvore, e o levantou, e pô-lo ao seu
ombro, e disse ao povo, que com ele havia: O que me vistes fazer
apressai-vos a fazê-lo assim como eu. Assim, pois, cada um de
todo o povo, também cortou o seu ramo e seguiu a Abimeleque; e pondo
os ramos junto da fortaleza, queimaram-na a fogo com os que nela
estavam, de modo que todos os da torre de Siquém morreram, uns mil
homens e mulheres.

Então Abimeleque foi a Tebes e a sitiou, e a tomou. Havia,
porém, no meio da cidade uma torre forte; e todos os homens e
mulheres, e todos os cidadãos da cidade se refugiaram nela, e
fecharam após si as portas, e subiram ao eirado da torre. E
Abimeleque veio até à torre, e a combateu; e chegou-se até à porta
da torre, para a incendiar. Porém uma mulher lançou um pedaço
de uma mó sobre a cabeça de Abimeleque; e quebrou-lhe o crânio.
Então chamou logo ao moço, que levava as suas armas, e
disse-lhe: Desembainha a tua espada, e mata-me; para que não se diga
de mim: Uma mulher o matou. E o moço o atravessou e ele morreu.
Vendo, pois, os homens de Israel que Abimeleque já era morto,
foram-se cada um para o seu lugar. Assim Deus fez tornar
sobre Abimeleque o mal que tinha feito a seu pai, matando a seus
setenta irmãos. Como também todo o mal dos homens de Siquém
fez tornar sobre a cabeça deles; e a maldição de Jotão, filho de
Jerubaal, veio sobre eles.

\medskip

\lettrine{10} E depois de Abimeleque, se levantou, para livrar
a Israel, Tola, filho de Puá, filho de Dodo, homem de Issacar; e
habitava em Samir, na montanha de Efraim. E julgou a Israel
vinte e três anos; e morreu, e foi sepultado em Samir. E depois
dele se levantou Jair, gileadita, e julgou a Israel vinte e dois
anos. E tinha este trinta filhos, que cavalgavam sobre trinta
jumentos; e tinham trinta cidades, a que chamaram Havote-Jair, até
ao dia de hoje; as quais estão na terra de Gileade. E morreu
Jair, e foi sepultado em Camom.

Então tornaram os filhos de Israel a fazer o que era mau aos olhos
do Senhor, e serviram aos baalins, e a Astarote, e aos deuses da
Síria, e aos deuses de Sidom, e aos deuses de Moabe, e aos deuses
dos filhos de Amom, e aos deuses dos filisteus; e deixaram ao
Senhor, e não o serviram. E a ira do Senhor se acendeu contra
Israel; e vendeu-os nas mãos dos filisteus, e nas mãos dos filhos de
Amom. E naquele mesmo ano oprimiram e vexaram aos filhos de
Israel; dezoito anos oprimiram a todos os filhos de Israel que
estavam além do Jordão, na terra dos amorreus, que está em Gileade.
Até os filhos de Amom passaram o Jordão, para pelejar também
contra Judá, e contra Benjamim, e contra a casa de Efraim; de modo
que Israel ficou muito angustiado.

Então os filhos de Israel clamaram ao Senhor, dizendo: Contra ti
havemos pecado, visto que deixamos a nosso Deus, e servimos aos
baalins. Porém o Senhor disse aos filhos de Israel:
Porventura dos egípcios, e dos amorreus, e dos filhos de Amom, e dos
filisteus, e dos sidônios, e dos amalequitas, e dos maonitas,
que vos oprimiam, quando a mim clamastes, não vos livrei das suas
mãos? Contudo vós me deixastes a mim, e servistes a outros
deuses; pelo que não vos livrarei mais. Ide, e clamai aos
deuses que escolhestes; que eles vos livrem no tempo do vosso
aperto. Mas os filhos de Israel disseram ao Senhor: Pecamos;
faze-nos conforme a tudo quanto te parecer bem aos teus olhos;
tão-somente te rogamos que nos livres nesta vez. E tiraram os
deuses alheios do meio de si, e serviram ao Senhor; então se
angustiou a sua alma por causa da desgraça de Israel. E os
filhos de Amom se reuniram e se acamparam em Gileade; e também os de
Israel se congregaram, e se acamparam em Mizpá. Então o povo
e os príncipes de Gileade disseram uns aos outros: Quem será o homem
que começará a pelejar contra os filhos de Amom? Ele será por cabeça
de todos os moradores de Gileade.

\medskip

\lettrine{11} Era então Jefté, o gileadita, homem valoroso,
porém filho de uma prostituta; mas Gileade gerara a Jefté.
Também a mulher de Gileade lhe deu filhos, e, sendo os filhos
desta mulher já grandes, expulsaram a Jefté, e lhe disseram: Não
herdarás na casa de nosso pai, porque és filho de outra mulher.
Então Jefté fugiu de diante de seus irmãos, e habitou na terra
de Tobe; e homens levianos se ajuntaram a Jefté, e saíam com ele.

E aconteceu que, depois de algum tempo, os filhos de Amom
pelejaram contra Israel. E sucedeu que, como os filhos de Amom
pelejassem contra Israel, foram os anciãos de Gileade buscar a Jefté
na terra de Tobe. E disseram a Jefté: Vem, e sê o nosso chefe;
para que combatamos contra os filhos de Amom. Porém Jefté disse
aos anciãos de Gileade: Porventura não me odiastes a mim, e não me
expulsastes da casa de meu pai? Por que, pois, agora viestes a mim,
quando estais em aperto? E disseram os anciãos de Gileade a
Jefté: Por isso tornamos a ti, para que venhas conosco, e combatas
contra os filhos de Amom; e nos sejas por chefe sobre todos os
moradores de Gileade. Então Jefté disse aos anciãos de Gileade:
Se me levardes de volta para combater contra os filhos de Amom, e o
Senhor mos der diante de mim, então eu vos serei por chefe? E
disseram os anciãos de Gileade a Jefté: O Senhor será testemunha
entre nós, e assim o faremos conforme a tua palavra. Assim
Jefté foi com os anciãos de Gileade, e o povo o pôs por chefe e
príncipe sobre si; e Jefté falou todas as suas palavras perante o
Senhor em Mizpá.

E enviou Jefté mensageiros ao rei dos filhos de Amom, dizendo:
Que há entre mim e ti, que vieste a mim a pelejar contra a minha
terra? E disse o rei dos filhos de Amom aos mensageiros de
Jefté: É porque, saindo Israel do Egito, tomou a minha terra, desde
Arnom até Jaboque, e ainda até ao Jordão: Restitui-ma agora, em paz.
Porém Jefté prosseguiu ainda em enviar mensageiros ao rei dos
filhos de Amom, dizendo-lhe: Assim diz Jefté: Israel não
tomou, nem a terra dos moabitas, nem a terra dos filhos de Amom.
Porque, subindo Israel do Egito, andou pelo deserto até ao
Mar Vermelho, e chegou até Cades. E Israel enviou mensageiros
ao rei dos edomitas, dizendo: Rogo-te que me deixes passar pela tua
terra. Porém o rei dos edomitas não lhe deu ouvidos; enviou também
ao rei dos moabitas, o qual igualmente não consentiu; e assim Israel
ficou em Cades. Depois andou pelo deserto e rodeou a terra
dos edomitas e a terra dos moabitas, e veio do nascente do sol à
terra dos moabitas, e alojou-se além de Arnom; porém não entrou nos
limites dos moabitas, porque Arnom é limite dos moabitas. Mas
Israel enviou mensageiros a Siom, rei dos amorreus, rei de Hesbom; e
disse-lhe Israel: Deixa-nos, peço-te, passar pela tua terra até ao
meu lugar. Porém Siom não confiou em Israel para este passar
nos seus limites; antes Siom ajuntou todo o seu povo, e se acamparam
em Jasa, e combateu contra Israel. E o Senhor Deus de Israel
deu a Siom, com todo o seu povo, na mão de Israel, que os feriu; e
Israel tomou por herança toda a terra dos amorreus que habitavam
naquela região. E por herança tomaram todos os limites dos
amorreus, desde Arnom até Jaboque, e desde o deserto até ao Jordão.
Assim o Senhor Deus de Israel desapossou os amorreus de
diante do seu povo de Israel; e os possuirias tu? Não
possuirias tu aquilo que Quemós, teu deus, desapossasse de diante de
ti? Assim possuiremos nós todos quantos o Senhor nosso Deus
desapossar de diante de nós. Agora, pois, és tu ainda melhor
do que Balaque, filho de Zipor, rei dos moabitas? Porventura
contendeu ele em algum tempo com Israel, ou pelejou alguma vez
contra ele? Enquanto Israel habitou trezentos anos em Hesbom
e nas suas vilas, e em Aroer e nas suas vilas, em todas as cidades
que estão ao longo de Arnom, por que o não recuperastes naquele
tempo? Tampouco pequei eu contra ti! Porém tu usas mal comigo
em pelejar contra mim; o Senhor, que é juiz, julgue hoje entre os
filhos de Israel e entre os filhos de Amom. Porém o rei dos
filhos de Amom não deu ouvidos às palavras que Jefté lhe enviou.

Então o Espírito do Senhor veio sobre Jefté, e atravessou ele por
Gileade e Manassés, passando por Mizpá de Gileade, e de Mizpá de
Gileade passou até aos filhos de Amom. E Jefté fez um voto ao
Senhor, e disse: Se totalmente deres os filhos de Amom na minha mão,
aquilo que, saindo da porta de minha casa, me sair ao
encontro, voltando eu dos filhos de Amom em paz, isso será do
Senhor, e o oferecerei em holocausto. Assim Jefté passou aos
filhos de Amom, a combater contra eles; e o Senhor os deu na sua
mão. E os feriu com grande mortandade, desde Aroer até chegar
a Minite, vinte cidades, e até Abel-Queramim; assim foram subjugados
os filhos de Amom diante dos filhos de Israel. Vindo, pois,
Jefté a Mizpá, à sua casa, eis que a sua filha lhe saiu ao encontro
com adufes e com danças; e era ela a única filha; não tinha ele
outro filho nem filha. E aconteceu que, quando a viu, rasgou
as suas vestes, e disse: Ah! filha minha, muito me abateste, e estás
entre os que me turbam! Porque eu abri a minha boca ao Senhor, e não
tornarei atrás. E ela lhe disse: Meu pai, tu deste a palavra
ao Senhor, faze de mim conforme o que prometeste; pois o Senhor te
vingou dos teus inimigos, os filhos de Amom. Disse mais a seu
pai: Conceda-me isto: Deixa-me por dois meses que vá, e desça pelos
montes, e chore a minha virgindade, eu e as minhas companheiras.
E disse ele: Vai. E deixou-a ir por dois meses; então foi ela
com as suas companheiras, e chorou a sua virgindade pelos montes.
E sucedeu que, ao fim de dois meses, tornou ela para seu pai,
o qual cumpriu nela o seu voto que tinha feito; e ela não conheceu
homem; e daí veio o costume de Israel, que as filhas de
Israel iam de ano em ano lamentar, por quatro dias, a filha de
Jefté, o gileadita.

\medskip

\lettrine{12} Então se convocaram os homens de Efraim, e
passaram para o norte, e disseram a Jefté: Por que passaste a
combater contra os filhos de Amom, e não nos chamaste para ir
contigo? Queimaremos a fogo a tua casa contigo. E Jefté lhes
disse: Eu e o meu povo tivemos grande contenda com os filhos de
Amom; e chamei-vos, e não me livrastes da sua mão; e, vendo eu
que não me livráveis, arrisquei a minha vida, e passei contra os
filhos de Amom, e o Senhor mos entregou nas mãos; por que, pois,
subistes vós hoje, para combater contra mim? E ajuntou Jefté a
todos os homens de Gileade, e combateu contra Efraim; e os homens de
Gileade feriram a Efraim; porque este dissera-lhe: Fugitivos sois de
Efraim, vós gileaditas que habitais entre Efraim e Manassés,
porque tomaram os gileaditas aos efraimitas os vaus do Jordão; e
sucedeu que, quando algum dos fugitivos de Efraim dizia: Deixai-me
passar; então os gileaditas perguntavam: És tu efraimita? E dizendo
ele: Não, então lhe diziam: Dize, pois, Chibolete; porém ele
dizia: Sibolete; porque não o podia pronunciar bem; então pegavam
dele, e o degolavam nos vaus do Jordão; e caíram de Efraim naquele
tempo quarenta e dois mil. E Jefté julgou a Israel seis anos; e
Jefté, o gileadita, faleceu, e foi sepultado numa das cidades de
Gileade.

E depois dele julgou a Israel Ibzã de Belém. E tinha este
trinta filhos, e trinta filhas que casou fora; e trinta filhas
trouxe de fora para seus filhos; e julgou a Israel sete anos.
Então faleceu Ibzã, e foi sepultado em Belém. E depois
dele julgou a Israel Elom, o zebulonita; e julgou a Israel dez anos.
E faleceu Elom, o zebulonita, e foi sepultado em Aijalom, na
terra de Zebulom. E depois dele julgou a Israel Abdom, filho
de Hilel, o piratonita. E tinha este quarenta filhos, e
trinta netos, que cavalgavam sobre setenta jumentos; e julgou a
Israel oito anos. Então faleceu Abdom, filho de Hilel, o
piratonita; e foi sepultado em Piratom, na terra de Efraim, no monte
dos amalequitas.

\medskip

\lettrine{13} E os filhos de Israel tornaram a fazer o que era
mau aos olhos do Senhor, e o Senhor os entregou na mão dos filisteus
por quarenta anos. E havia um homem de Zorá, da tribo de Dã,
cujo nome era Manoá; e sua mulher, sendo estéril, não tinha filhos.
E o anjo do Senhor apareceu a esta mulher, e disse-lhe: Eis que
agora és estéril, e nunca tens concebido; porém conceberás, e terás
um filho. Agora, pois, guarda-te de beber vinho, ou bebida
forte, ou comer coisa imunda. Porque eis que tu conceberás e
terás um filho sobre cuja cabeça não passará navalha; porquanto o
menino será nazireu de Deus desde o ventre; e ele começará a livrar
a Israel da mão dos filisteus. Então a mulher entrou, e falou a
seu marido, dizendo: Um homem de Deus veio a mim, cuja aparência era
semelhante de um anjo de Deus, terribilíssima; e não lhe perguntei
donde era, nem ele me disse o seu nome. Porém disse-me: Eis que
tu conceberás e terás um filho; agora pois, não bebas vinho, nem
bebida forte, e não comas coisa imunda; porque o menino será nazireu
de Deus, desde o ventre até ao dia da sua morte.

Então Manoá orou ao Senhor, e disse: Ah! Senhor meu, rogo-te que o
homem de Deus, que enviaste, ainda venha para nós outra vez e nos
ensine o que devemos fazer ao menino que há de nascer. E Deus
ouviu a voz de Manoá; e o anjo de Deus veio outra vez à mulher, e
ela estava no campo, porém não estava com ela seu marido Manoá.
Apressou-se, pois, a mulher, e correu, e noticiou-o a seu
marido, e disse-lhe: Eis que aquele homem que veio a mim o outro dia
me apareceu. Então Manoá levantou-se, e seguiu a sua mulher,
e foi àquele homem, e disse-lhe: És tu aquele homem que falou a esta
mulher? E disse: Eu sou. Então disse Manoá: Cumpram-se as
tuas palavras; mas qual será o modo de viver e o serviço do menino?
E disse o anjo do Senhor a Manoá: De tudo quanto eu disse à
mulher guardará ela. De tudo quanto procede da videira não
comerá, nem vinho nem bebida forte beberá, nem coisa imunda comerá;
tudo quanto lhe tenho ordenado guardará.

Então Manoá disse ao anjo do Senhor: Ora deixa que te detenhamos,
e te preparemos um cabrito. Porém o anjo do Senhor disse a
Manoá: Ainda que me detenhas, não comerei de teu pão; e se fizeres
holocausto o oferecerás ao Senhor. Porque não sabia Manoá que era o
anjo do Senhor. E disse Manoá ao anjo do Senhor: Qual é o teu
nome, para que, quando se cumprir a tua palavra, te honremos?
E o anjo do Senhor lhe disse: Por que perguntas assim pelo
meu nome, visto que é maravilhoso? Então Manoá tomou um
cabrito e uma oferta de alimentos, e os ofereceu sobre uma penha ao
Senhor: e houve-se o anjo maravilhosamente, observando-o Manoá e sua
mulher. E sucedeu que, subindo a chama do altar para o céu, o
anjo do Senhor subiu na chama do altar; o que vendo Manoá e sua
mulher, caíram em terra sobre seus rostos. E nunca mais
apareceu o anjo do Senhor a Manoá, nem a sua mulher; então
compreendeu Manoá que era o anjo do Senhor. E disse Manoá à
sua mulher: Certamente morreremos, porquanto temos visto a Deus.
Porém sua mulher lhe disse: Se o Senhor nos quisesse matar,
não aceitaria da nossa mão o holocausto e a oferta de alimentos, nem
nos mostraria tudo isto, nem nos deixaria ouvir tais coisas neste
tempo.

Depois teve esta mulher um filho, a quem pôs o nome de Sansão; e
o menino cresceu, e o Senhor o abençoou. E o Espírito do
Senhor começou a incitá-lo de quando em quando para o campo de
Maané-Dã, entre Zorá e Estaol.

\medskip

\lettrine{14} E desceu Sansão a Timnate; e, vendo em Timnate
uma mulher das filhas dos filisteus, subiu, e declarou-o a seu
pai e a sua mãe, e disse: Vi uma mulher em Timnate, das filhas dos
filisteus; agora, pois, tomai-ma por mulher. Porém seu pai e sua
mãe lhe disseram: Não há, porventura, mulher entre as filhas de teus
irmãos, nem entre todo o meu povo, para que tu vás tomar mulher dos
filisteus, daqueles incircuncisos? E disse Sansão a seu pai: Toma-me
esta, porque ela agrada aos meus olhos. Mas seu pai e sua mãe
não sabiam que isto vinha do Senhor; pois buscava ocasião contra os
filisteus; porquanto naquele tempo os filisteus dominavam sobre
Israel. Desceu, pois, Sansão com seu pai e com sua mãe a
Timnate; e, chegando às vinhas de Timnate eis que um filho de leão,
rugindo, lhe saiu ao encontro. Então o Espírito do Senhor se
apossou dele tão poderosamente que despedaçou o leão, como quem
despedaça um cabrito, sem ter nada na sua mão; porém nem a seu pai
nem a sua mãe deu a saber o que tinha feito. E desceu, e falou
àquela mulher, e ela agradou aos olhos de Sansão. E depois de
alguns dias voltou ele para tomá-la; e, apartando-se do caminho para
ver o corpo do leão morto, eis que nele havia um enxame de abelhas
com mel. E tomou-o nas suas mãos, e foi andando e comendo dele;
e foi a seu pai e a sua mãe, e deu-lhes do mel, e comeram; porém não
lhes deu a saber que tomara o mel do corpo do leão.

Descendo, pois, seu pai àquela mulher, fez Sansão ali um
banquete; porque assim os moços costumavam fazer. E sucedeu
que, como o vissem, trouxeram trinta companheiros para estarem com
ele. Disse-lhes, pois, Sansão: Eu vos darei um enigma para
decifrar; e, se nos sete dias das bodas o decifrardes e
descobrirdes, eu vos darei trinta lençóis e trinta mudas de roupas.
E, se não puderdes decifrar, vós me dareis a mim trinta
lençóis e as trinta mudas de roupas. E eles lhe disseram: Dá-nos o
teu enigma a decifrar, para que o ouçamos. Então lhes disse:
Do comedor saiu comida, e do forte saiu doçura. E em três dias não
puderam decifrar o enigma. E sucedeu que, ao sétimo dia,
disseram à mulher de Sansão: Persuade a teu marido que nos declare o
enigma, para que porventura não queimemos a fogo a ti e à casa de
teu pai; chamastes-nos aqui para vos apossardes do que é nosso, não
é assim? E a mulher de Sansão chorou diante dele, e disse:
Tão-somente me desprezas, e não me amas; pois deste aos filhos do
meu povo um enigma para decifrar, e ainda não o declaraste a mim. E
ele lhe disse: Eis que nem a meu pai nem a minha mãe o declarei, e
to declararia a ti? E chorou diante dele os sete dias em que
celebravam as bodas; sucedeu, pois, que ao sétimo dia lho declarou,
porquanto o importunava; então ela declarou o enigma aos filhos do
seu povo. Disseram, pois, a Sansão os homens daquela cidade,
ao sétimo dia, antes de se pôr o sol: Que coisa há mais doce do que
o mel? E que coisa há mais forte do que o leão? E ele lhes disse: Se
vós não lavrásseis com a minha novilha, nunca teríeis descoberto o
meu enigma. Então o Espírito do Senhor tão poderosamente se
apossou dele, que desceu aos ascalonitas, e matou deles trinta
homens, e tomou as suas roupas, e deu as mudas de roupas aos que
declararam o enigma; porém acendeu-se a sua ira, e subiu à casa de
seu pai. E a mulher de Sansão foi dada ao seu companheiro que
antes o acompanhava.

\medskip

\lettrine{15} E aconteceu, depois de alguns dias, que, na sega
do trigo, Sansão visitou a sua mulher, com um cabrito, e disse:
Entrarei na câmara de minha mulher. Porém o pai dela não o deixou
entrar. E disse-lhe seu pai: Por certo pensava eu que de todo a
desprezavas; de sorte que a dei ao teu companheiro; porém não é sua
irmã mais nova, mais formosa do que ela? Toma-a, pois, em seu lugar.
Então Sansão disse acerca deles: Inocente sou esta vez para com
os filisteus, quando lhes fizer algum mal. E foi Sansão, e pegou
trezentas raposas; e, tomando tochas, as virou cauda a cauda, e lhes
pôs uma tocha no meio de cada duas caudas. E chegou fogo às
tochas, e largou-as na seara dos filisteus; e assim abrasou os
molhos com a sega do trigo, e as vinhas e os olivais. Então
perguntaram os filisteus: Quem fez isto? E responderam: Sansão, o
genro do timnita, porque lhe tomou a sua mulher, e a deu a seu
companheiro. Então subiram os filisteus, e queimaram a fogo a ela e
a seu pai. Então lhes disse Sansão: É assim que fazeis? Pois,
havendo-me vingado eu de vós, então cessarei. E feriu-os com
grande ferimento, pernas juntamente com coxa; e desceu, e habitou na
fenda da rocha de Etã.

Então os filisteus subiram, e acamparam-se contra Judá, e
estenderam-se por Leí. E per\-gun\-ta\-ram-lhes os homens de Judá:
Por que subistes contra nós? E eles responderam: Subimos para
amarrar a Sansão, para lhe fazer a ele como ele nos fez a nós.
Então três mil homens de Judá desceram até a fenda da rocha
de Etã, e disseram a Sansão: Não sabias tu que os filisteus dominam
sobre nós? Por que, pois, nos fizeste isto? E ele lhes disse: Assim
como eles me fizeram a mim, eu lhes fiz a eles. E
disseram-lhe: Descemos para te amarrar e te entregar nas mãos dos
filisteus. Então Sansão lhes disse: Jurai-me que vós mesmos não me
acometereis. E eles lhe falaram, dizendo: Não, mas fortemente
te amarraremos, e te entregaremos nas mãos deles; porém de maneira
nenhuma te mataremos. E amarraram-no com duas cordas novas e
fizeram-no subir da rocha. E, vindo ele a Leí, os filisteus
lhe saíram ao encontro, jubilando; porém o Espírito do Senhor
poderosamente se apossou dele, e as cordas que ele tinha nos braços
se tornaram como fios de linho que se queimaram no fogo, e as suas
amarraduras se desfizeram das suas mãos. E achou uma queixada
fresca de um jumento, e estendeu a sua mão, e tomou-a, e feriu com
ela mil homens. Então disse Sansão: Com uma queixada de
jumento, montões sobre montões; com uma queixada de jumento feri a
mil homens. E aconteceu que, acabando ele de falar, lançou a
queixada da sua mão; e chamou aquele lugar Ramate-Leí.

E como tivesse grande sede, clamou ao Senhor, e disse: Pela mão
do teu servo tu deste esta grande salvação; morrerei eu pois agora
de sede, e cairei na mão destes incircuncisos? Então Deus
fendeu uma cavidade que estava na queixada; e saiu dela água, e
bebeu; e recobrou o seu espírito e reanimou-se; por isso chamou
aquele lugar: A fonte do que clama, que está em Leí até ao dia de
hoje. E julgou a Israel, nos dias dos filisteus, vinte anos.

\medskip

\lettrine{16} E foi Sansão a Gaza, e viu ali uma mulher
prostituta, e entrou a ela. E foi dito aos gazitas: Sansão
entrou aqui. Cercaram-no, e toda a noite lhe puseram espias à porta
da cidade; porém toda a noite estiveram quietos, dizendo: Até à luz
da manhã esperaremos; então o mataremos. Porém Sansão deitou-se
até à meia noite, e à meia noite se levantou, e arrancou as portas
da entrada da cidade com ambas as umbreiras, e juntamente com a
tranca as tomou, pondo-as sobre os ombros; e levou-as para cima até
ao cume do monte que está defronte de Hebrom.

E depois disto aconteceu que se afeiçoou a uma mulher do vale de
Soreque, cujo nome era Dalila. Então os príncipes dos filisteus
subiram a ela, e lhe disseram: Persuade-o, e vê em que consiste a
sua grande força, e como poderíamos assenhorear-nos dele e
amarrá-lo, para assim o afligirmos; e te daremos, cada um de nós,
mil e cem moedas de prata. Disse, pois, Dalila a Sansão:
Declara-me, peço-te, em que consiste a tua grande força, e com que
poderias ser amarrado para te poderem afligir. Disse-lhe Sansão:
Se me amarrassem com sete vergas de vimes frescos, que ainda não
estivessem secos, então me enfraqueceria, e seria como qualquer
outro homem. Então os príncipes dos filisteus lhe trouxeram sete
vergas de vimes frescos, que ainda não estavam secos; e amarraram-no
com elas. E o espia estava com ela na câmara interior. Então ela
lhe disse: Os filisteus vêm sobre ti, Sansão. Então quebrou as
vergas de vimes, como se quebra o fio da estopa ao cheiro do fogo;
assim não se soube em que consistia a sua força. Então disse
Dalila a Sansão: Eis que zombaste de mim, e me disseste mentiras;
ora declara-me agora com que poderias ser amarrado. E ele
disse: Se me amarrassem fortemente com cordas novas, que ainda não
houvessem sido usadas, então me enfraqueceria, e seria como qualquer
outro homem. Então Dalila tomou cordas novas, e o amarrou com
elas, e disse-lhe: Os filisteus vêm sobre ti, Sansão. E o espia
estava na recâmara interior. Então as quebrou de seus braços como a
um fio. E disse Dalila a Sansão: Até agora zombaste de mim, e
me disseste mentiras; declara-me pois, agora, com que poderias ser
amarrado? E ele lhe disse: Se teceres sete tranças dos cabelos da
minha cabeça com o liço\footnote{Cada um dos fios, entre dois
liçaróis (liçarol: cada uma das travessas que seguram os liços) do
tear, que sobem e descem para serem atravessados pelos fios da
tecelagem.} da teia. E ela as fixou com uma estaca, e
disse-lhe: Os filisteus vêm sobre ti, Sansão: Então ele despertou do
seu sono, e arrancou a estaca das tranças tecidas, juntamente com o
liço da teia. Então ela lhe disse: Como dirás: Tenho-te amor,
não estando comigo o teu coração? Já três vezes zombaste de mim, e
ainda não me declaraste em que consiste a tua força. E
sucedeu que, importunando-o ela todos os dias com as suas palavras,
e molestando-o, a sua alma se angustiou até a morte. E
descobriu-lhe todo o seu coração, e disse-lhe: Nunca passou navalha
pela minha cabeça, porque sou nazireu de Deus desde o ventre de
minha mãe; se viesse a ser rapado, ir-se-ia de mim a minha força, e
me enfraqueceria, e seria como qualquer outro homem.

Vendo, pois, Dalila que já lhe descobrira todo o seu coração,
mandou chamar os príncipes dos filisteus, dizendo: Subi esta vez,
porque agora me descobriu ele todo o seu coração. E os príncipes dos
filisteus subiram a ter com ela, trazendo com eles o dinheiro.
Então ela o fez dormir sobre os seus joelhos, e chamou a um
homem, e rapou-lhe as sete tranças do cabelo de sua cabeça; e
começou a afligi-lo, e retirou-se dele a sua força. E disse
ela: Os filisteus vêm sobre ti, Sansão. E despertou ele do seu sono,
e disse: Sairei ainda esta vez como dantes, e me sacudirei. Porque
ele não sabia que já o Senhor se tinha retirado dele. Então
os filisteus pegaram nele, e arrancaram-lhe os olhos, e fizeram-no
descer a Gaza, e amarraram-no com duas cadeias de bronze, e girava
ele um moinho no cárcere.

E o cabelo da sua cabeça começou a crescer, como quando foi
rapado. Então os príncipes dos filisteus se ajuntaram para
oferecer um grande sacrifício ao seu deus Dagom, e para se
alegrarem, e diziam: Nosso deus nos entregou nas mãos a Sansão,
nosso inimigo. Semelhantemente, vendo-o o povo, louvava ao
seu deus; porque dizia: Nosso deus nos entregou nas mãos o nosso
inimigo, e ao que destruía a nossa terra, e ao que multiplicava os
nossos mortos. E sucedeu que, alegrando-se-lhes o coração,
disseram: Chamai a Sansão, para que brinque diante de nós. E
chamaram a Sansão do cárcere, que brincava diante deles, e
fizeram-no estar em pé entre as colunas. Então disse Sansão
ao moço que o tinha pela mão: Guia-me para que apalpe as colunas em
que se sustém a casa, para que me encoste a elas. Ora estava
a casa cheia de homens e mulheres; e também ali estavam todos os
príncipes dos filisteus; e sobre o telhado havia uns três mil homens
e mulheres, que estavam vendo Sansão brincar. Então Sansão
clamou ao Senhor, e disse: Senhor Deus, peço-te que te lembres de
mim, e fortalece-me agora só esta vez, ó Deus, para que de uma vez
me vingue dos filisteus, pelos meus dois olhos. Abraçou-se,
pois, Sansão com as duas colunas do meio, em que se sustinha a casa,
e arrimou-se sobre elas, com a sua mão direita numa, e com a sua
esquerda na outra. E disse Sansão: Morra eu com os filisteus.
E inclinou-se com força, e a casa caiu sobre os príncipes e sobre
todo o povo que nela havia; e foram mais os mortos que matou na sua
morte do que os que matara em sua vida. Então seus irmãos
desceram, e toda a casa de seu pai, e tomaram-no, e subiram com ele,
e sepultaram-no entre Zorá e Estaol, no sepulcro de Manoá, seu pai.
Ele julgou a Israel vinte anos.

\medskip

\lettrine{17} E havia um homem da montanha de Efraim, cujo
nome era Mica. O qual disse à sua mãe: As mil e cem moedas de
prata que te foram tiradas, por cuja causa lançaste maldições, e de
que também me falaste, eis que esse dinheiro está comigo; eu o
tomei. Então lhe disse sua mãe: Bendito do Senhor seja meu filho.
Assim restituiu as mil e cem moedas de prata à sua mãe; porém
sua mãe disse: Inteiramente tenho dedicado este dinheiro da minha
mão ao Senhor, para meu filho fazer uma imagem de escultura e uma de
fundição; de sorte que agora to tornarei a dar. Porém ele
restituiu aquele dinheiro à sua mãe; e sua mãe tomou duzentas moedas
de prata, e as deu ao ourives, o qual fez delas uma imagem de
escultura e uma de fundição, que ficaram em casa de Mica. E teve
este homem, Mica, uma casa de deuses; e fez um éfode e terafins, e
consagrou um de seus filhos, para que lhe fosse por sacerdote.
Naqueles dias não havia rei em Israel; cada um fazia o que
parecia bem aos seus olhos.

E havia um moço de Belém de Judá, da tribo de Judá, que era
levita, e peregrinava ali. este homem partiu da cidade de
Belém de Judá para peregrinar onde quer que achasse conveniente.
Chegando ele, pois, à montanha de Efraim, até à casa de Mica,
seguindo o seu caminho, disse-lhe Mica: Donde vens? E ele lhe
disse: Sou levita de Belém de Judá, e vou peregrinar onde quer que
achar conveniente. Então lhe disse Mica: Fica comigo, e sê-me
por pai e sacerdote; e cada ano te darei dez moedas de prata, e
vestuário, e o sustento. E o levita entrou. E consentiu o
levita em ficar com aquele homem; e o moço lhe foi como um de seus
filhos. E Mica consagrou o levita, e aquele moço lhe foi por
sacerdote; e esteve em casa de Mica. Então disse Mica: Agora
sei que o Senhor me fará bem; porquanto tenho um levita por
sacerdote.

\medskip

\lettrine{18} Naqueles dias não havia rei em Israel; e nos
mesmos dias a tribo dos danitas buscava para si herança para
habitar; porquanto até àquele dia entre as tribos de Israel não lhe
havia caído por sorte sua herança. E enviaram os filhos de Dã,
da sua tribo, cinco homens dentre eles, homens valorosos, de Zorá e
de Estaol, a espiar e reconhecer a terra, e lhes disseram: Ide,
reconhecei a terra. E chegaram à montanha de Efraim, até à casa de
Mica, e passaram ali a noite. E quando eles estavam junto da
casa de Mica, reconheceram a voz do moço, do levita; e dirigindo-se
para lá, lhe disseram: Quem te trouxe aqui? Que fazes aqui? E que é
que tens aqui? E ele lhes disse: Assim e assim me tem feito
Mica; pois me tem contratado, e eu lhe sirvo de sacerdote. Então
lhe disseram: Consulta a Deus, para que possamos saber se prosperará
o caminho que seguimos. E disse-lhes o sacerdote: Ide em paz; o
caminho que seguis está perante o Senhor.

Então foram-se aqueles cinco homens, e chegaram a Laís; e viram
que o povo que havia no meio dela estava seguro, conforme ao costume
dos sidônios, quieto e confiado; nem havia autoridade alguma do
reino que por qualquer coisa envergonhasse a alguém naquela terra;
também estavam longe dos sidônios, e não tinham relação com ninguém.
Então voltaram a seus irmãos, a Zorá e a Estaol, os quais lhes
disseram: Que dizeis vós? E eles disseram: Levantai-vos, e
subamos contra eles; porque examinamos a terra, e eis que é
muitíssimo boa. E vós estareis aqui tranqüilos? Não sejais
preguiçosos em irdes para entrar a possuir esta terra. Quando
lá chegardes, vereis um povo confiado, e a terra é larga de
extensão; porque Deus vo-la entregou nas mãos; lugar em que não há
falta de coisa alguma que há na terra. Então partiram dali,
da tribo dos danitas, de Zorá e de Estaol, seiscentos homens munidos
de armas de guerra. E subiram, e acamparam-se em
Quiriate-Jearim, em Judá; então chamaram a este lugar Maané-Dã, até
ao dia de hoje; eis que está por detrás de Quiriate-Jearim. E
dali passaram à montanha de Efraim; e chegaram até a casa de Mica.

Então responderam os cinco homens, que foram espiar a terra de
Laís, e disseram a seus irmãos: Sabeis vós também que naquelas casas
há um éfode, e terafins, e uma imagem de escultura e uma de
fundição? Vede, pois, agora o que haveis de fazer. Então se
dirigiram para lá, e chegaram à casa do moço, o levita, em casa de
Mica, e o saudaram. E os seiscentos homens, que eram dos
filhos de Dã, munidos com suas armas de guerra, ficaram à entrada da
porta. Porém subindo os cinco homens, que foram espiar a
terra, entraram ali, e tomaram a imagem de escultura, o éfode, e os
terafins, e a imagem de fundição, ficando o sacerdote em pé à
entrada da porta, com os seiscentos homens que estavam munidos com
as armas de guerra. Entrando eles, pois, em casa de Mica, e
tomando a imagem de escultura, e o éfode, e os terafins, e a imagem
de fundição, disse-lhes o sacerdote: Que estais fazendo? E
eles lhe disseram: Cala-te, põe a mão na boca, e vem conosco, e
sê-nos por pai e sacerdote. É melhor ser sacerdote da casa de um só
homem, do que ser sacerdote de uma tribo e de uma família em Israel?
Então alegrou-se o coração do sacerdote, e tomou o éfode, e
os terafins, e a imagem de escultura; e entrou no meio do povo.
Assim viraram, e partiram; e os meninos, e o gado, e a
bagagem puseram diante de si. E, estando já longe da casa de
Mica, os homens que estavam nas casas junto à casa de Mica,
reuniram-se, e alcançaram os filhos de Dã. E clamaram após os
filhos de Dã, os quais viraram os seus rostos, e disseram a Mica:
Que tens, que tanta gente convocaste? Então ele disse: Os
meus deuses, que eu fiz, me tomastes, juntamente com o sacerdote, e
partistes; que mais me resta agora? Como, pois, me dizeis: Que é que
tens? Porém os filhos de Dã lhe disseram: Não nos faças ouvir
a tua voz, para que porventura homens de ânimo mau não se lancem
sobre vós, e tu percas a tua vida, e a vida dos da tua casa.
Assim seguiram o seu caminho os filhos de Dã; e Mica, vendo
que eram mais fortes do que ele, virou-se, e voltou à sua casa.

Eles, pois, tomaram o que Mica tinha feito, e o sacerdote que
tivera, e chegaram a Laís, a um povo quieto e confiado, e os feriram
ao fio da espada, e queimaram a cidade a fogo. E ninguém
houve que os livrasse, porquanto estavam longe de Sidom, e não
tinham relações com ninguém, e a cidade estava no vale que está
junto de Bete-Reobe; depois reedificaram a cidade e habitaram nela.
E chamaram-lhe Dã, conforme ao nome de Dã, seu pai, que
nascera a Israel; era, porém, antes o nome desta cidade Laís.
E os filhos de Dã levantaram para si aquela imagem de
escultura; e Jônatas, filho de Gérson, o filho de Manassés, ele e
seus filhos foram sacerdotes da tribo dos danitas, até ao dia do
cativeiro da terra. Assim, pois, estabeleceram para si a
imagem de escultura, que fizera Mica, por todos os dias em que a
casa de Deus esteve em Siló.

\medskip

\lettrine{19} Aconteceu também naqueles dias, em que não havia
rei em Israel, que houve um homem levita, que, peregrinando aos
lados da montanha de Efraim, tomou para si uma concubina, de Belém
de Judá. Porém a sua concubina adulterou contra ele, e
deixando-o, foi para a casa de seu pai, em Belém de Judá, e esteve
ali alguns dias, a saber, quatro meses. E seu marido se
levantou, e foi atrás dela, para lhe falar conforme ao seu coração,
e para tornar a trazê-la; e o seu moço e um par de jumentos iam com
ele; e ela o levou à casa de seu pai, e, vendo-o o pai da moça,
alegrou-se ao encontrar-se com ele. E seu sogro, o pai da moça,
o deteve, e ficou com ele três dias; e comeram e beberam, e passaram
ali a noite. E sucedeu que ao quarto dia pela manhã, de
madrugada, ele levantou-se para partir; então o pai da moça disse a
seu genro: Fortalece o teu coração com um bocado de pão, e depois
partireis. Assentaram-se, pois, e comeram ambos juntos, e
beberam; e disse o pai da moça ao homem: Peço-te que ainda esta
noite queiras passá-la aqui, e alegre-se o teu coração. Porém o
homem levantou-se para partir; mas seu sogro o constrangeu a tornar
a passar ali a noite. E, madrugando ao quinto dia pela manhã
para partir, disse o pai da moça: Ora, conforta o teu coração. E
detiveram-se até já declinar o dia; e ambos juntos comeram.
Então o homem levantou-se para partir, ele, e a sua concubina, e
o seu moço; e disse-lhe seu sogro, o pai da moça: Eis que já o dia
declina e a tarde já vem chegando; peço-te que aqui passes a noite;
eis que o dia já vai acabando, passa aqui a noite, e que o teu
coração se alegre; e amanhã de madrugada levanta-te a caminhar, e
irás para a tua tenda. Porém o homem não quis ali passar a
noite, mas levantou-se, e partiu, e chegou até defronte de Jebus
(que é Jerusalém), e com ele o par de jumentos albardados, como
também a sua concubina. Estando, pois, já perto de Jebus, e
tendo-se já declinado muito o dia, disse o moço a seu senhor: Vamos
agora, e retiremo-nos a esta cidade dos jebuseus, e passemos ali a
noite. Porém disse-lhe seu senhor: Não nos retiraremos a
nenhuma cidade estranha, que não seja dos filhos de Israel; mas
iremos até Gibeá. Disse mais a seu moço: Vamos, e cheguemos a
um daqueles lugares, e passemos a noite em Gibeá ou em Ramá.
Passaram, pois, adiante, e caminharam, e o sol se lhes pôs
junto a Gibeá, que é cidade de Benjamim. E retiraram-se para
lá, para passarem a noite em Gibeá; e, entrando ele, assentou-se na
praça da cidade, porque não houve quem os recolhesse em casa para
ali passarem a noite.

E eis que um velho homem vinha à tarde do seu trabalho do campo;
e era este homem da montanha de Efraim, mas peregrinava em Gibeá;
eram porém os homens deste lugar filhos de Benjamim.
Levantando ele, pois, os olhos, viu a este viajante na praça
da cidade, e disse o ancião: Para onde vais, e donde vens? E
ele lhe disse: Viajamos de Belém de Judá até aos lados da montanha
de Efraim, de onde sou; porquanto fui a Belém de Judá, porém agora
vou à casa do Senhor; e ninguém há que me recolha em casa,
todavia temos palha e pasto para os nossos jumentos, e também
pão e vinho há para mim, e para a tua serva, e para o moço que vem
com os teus servos; de coisa nenhuma há falta. Então disse o
ancião: Paz seja contigo; tudo quanto te faltar fique ao meu cargo;
tão-somente não passes a noite na praça. E levou-o à sua
casa, e deu pasto aos jumentos; e, lavando-se os pés, comeram e
beberam.

Estando eles alegrando o seu coração, eis que os homens daquela
cidade (homens que eram filhos de Belial) cercaram a casa, batendo à
porta; e falaram ao ancião, senhor da casa, dizendo: Tira para fora
o homem que entrou em tua casa, para que o conheçamos. E o
homem, dono da casa, saiu a eles e disse-lhes: Não, irmãos meus, ora
não façais semelhante mal; já que este homem entrou em minha casa,
não façais tal loucura. Eis que a minha filha virgem e a
concubina dele vo-las tirarei fora; humilhai-as a elas, e fazei
delas o que parecer bem aos vossos olhos; porém a este homem não
façais essa loucura. Porém aqueles homens não o quiseram
ouvir; então aquele homem pegou da sua concubina, e lha tirou para
fora; e eles a conheceram e abusaram dela toda a noite até pela
manhã, e, subindo a alva, a deixaram. E ao romper da manhã
veio a mulher, e caiu à porta da casa daquele homem, onde estava seu
senhor, e ficou ali até que se fez claro. E, levantando-se
seu senhor pela manhã, e abrindo as portas da casa, e saindo a
seguir o seu caminho, eis que a mulher, sua concubina, jazia à porta
da casa, com as mãos sobre o limiar. E ele lhe disse:
Levanta-te, e vamo-nos, porém ela não respondeu; então,
levantando-se o homem a pôs sobre o jumento, e foi para o seu lugar.
Chegando, pois, à sua casa, tomou um cutelo, e pegou na sua
concubina, e a despedaçou com os seus ossos em doze partes; e
enviou-as por todos os termos de Israel. E sucedeu que cada
um que via aquilo dizia: Nunca tal se fez, nem se viu desde o dia em
que os filhos de Israel subiram da terra do Egito, até ao dia de
hoje; ponderai isto, considerai, e falai.

\medskip

\lettrine{20} Então todos os filhos de Israel saíram, e a
congregação se ajuntou, perante o Senhor em Mizpá, como se fora um
só homem, desde Dã até Berseba, como também a terra de Gileade.
E os principais de todo o povo, de todas as tribos de Israel, se
apresentaram na congregação do povo de Deus; quatrocentos mil homens
de pé que tiravam a espada ouviram, pois, os filhos de Benjamim
que os filhos de Israel haviam subido a Mizpá). E disseram os filhos
de Israel: Falai, como sucedeu esta maldade? Então respondeu o
homem levita, marido da mulher que fora morta, e disse: Cheguei com
a minha concubina a Gibeá, cidade de Benjamim, para passar a noite.
E os cidadãos de Gibeá se levantaram contra mim, e cercaram a
casa de noite; intentaram matar-me, e violaram a minha concubina, de
maneira que morreu. Então peguei na minha concubina, e fi-la em
pedaços, e a enviei por toda a terra da herança de Israel; porquanto
fizeram tal malefício e loucura em Israel. Eis que todos sois
filhos de Israel; dai aqui a vossa palavra e conselho. Então
todo o povo se levantou como um só homem, dizendo: Nenhum de nós irá
à sua tenda nem nenhum de nós voltará à sua casa. Porém isto é o
que faremos a Gibeá: procederemos contra ela por sorte. E de
todas as tribos de Israel, tomaremos dez homens de cada cem, e cem
de cada mil, e mil de cada dez mil, para providenciarem mantimento
para o povo; para que, vindo ele a Gibeá de Benjamim, lhe façam
conforme a toda a loucura que tem feito em Israel. Assim
ajuntaram-se contra esta cidade todos os homens de Israel, unidos
como um só homem.

E as tribos de Israel enviaram homens por toda a tribo de
Benjamim, dizendo: Que maldade é esta que se fez entre vós?
Dai-nos, pois, agora aqueles homens, filhos de Belial, que
estão em Gibeá, para que os matemos, e tiremos de Israel o mal.
Porém os filhos de Benjamim não quiseram ouvir a voz de seus irmãos,
os filhos de Israel. Antes os filhos de Benjamim se ajuntaram
das cidades em Gibeá, para saírem a pelejar contra os filhos de
Israel. E contaram-se naquele dia os filhos de Benjamim, das
cidades, vinte e seis mil homens que tiravam a espada, afora os
moradores de Gibeá, de que se contaram setecentos homens escolhidos.
Entre todo este povo havia setecentos homens escolhidos,
canhotos, os quais atiravam com a funda uma pedra em um cabelo, e
não erravam. E contaram-se dos homens de Israel, afora os de
Benjamim, quatrocentos mil homens que tiravam da espada, e todos
eles homens de guerra.

E levantaram-se os filhos de Israel, e subiram a Betel; e
consultaram a Deus, dizendo: Quem dentre nós subirá primeiro a
pelejar contra Benjamim? E disse o Senhor: Judá subirá primeiro.
Levantaram-se, pois, os filhos de Israel pela manhã, e
acamparam-se contra Gibeá. E os homens de Israel saíram à
peleja contra Benjamim; e os homens de Israel ordenaram a batalha
contra eles, ao pé de Gibeá. Então os filhos de Benjamim
saíram de Gibeá, e derrubaram por terra, naquele dia, vinte e dois
mil homens de Israel. Porém esforçou-se o povo, isto é, os
homens de Israel, e tornaram a ordenar a peleja no lugar onde no
primeiro dia a tinham ordenado. E subiram os filhos de
Israel, e choraram perante o Senhor até à tarde, e perguntaram ao
Senhor, dizendo: Tornar-me-ei a chegar à peleja contra os filhos de
Benjamim, meu irmão? E disse o Senhor: Subi contra ele.
Chegaram-se, pois, os filhos de Israel aos filhos de
Benjamim, no dia seguinte. Também os de Benjamim no dia
seguinte lhes saíram ao encontro fora de Gibeá, e derrubaram ainda
por terra mais dezoito mil homens, todos dos que tiravam a espada.

Então todos os filhos de Israel, e todo o povo, subiram, e vieram
a Betel e choraram, e estiveram ali perante o Senhor, e jejuaram
aquele dia até à tarde; e ofereceram holocaustos e ofertas pacíficas
perante o Senhor. E os filhos de Israel perguntaram ao Senhor
(porquanto a arca da aliança de Deus estava ali naqueles dias;
e Finéias, filho de Eleazar, filho de Arão, estava perante
ele naqueles dias), dizendo: Tornarei ainda a pelejar contra os
filhos de Benjamim, meu irmão, ou pararei? E disse o Senhor: Subi,
que amanhã eu to entregarei na mão. Então Israel pôs
emboscadas em redor de Gibeá. E subiram os filhos de Israel
ao terceiro dia contra os filhos de Benjamim, e ordenaram a peleja
junto a Gibeá, como das outras vezes. Então os filhos de
Benjamim saíram ao encontro do povo, e desviaram-se da cidade; e
começaram a ferir alguns do povo, atravessando-os, como das outras
vezes, pelos caminhos (um dos quais sobe para Betel, e o outro para
Gibeá pelo campo), uns trinta dos homens de Israel. Então os
filhos de Benjamim disseram: Estão derrotados diante de nós como
dantes. Porém os filhos de Israel disseram: Fujamos, e desviemo-los
da cidade para os caminhos. Então todos os homens de Israel
se levantaram do seu lugar, e ordenaram a peleja em Baal-Tamar; e a
emboscada de Israel saiu do seu lugar, da caverna de Gibeá. E
dez mil homens escolhidos de todo o Israel vieram contra Gibeá, e a
peleja se agravou; porém eles não sabiam o mal que lhes tocaria.
Então feriu o Senhor a Benjamim diante de Israel; e
destruíram os filhos de Israel, naquele dia, vinte e cinco mil e cem
homens de Benjamim, todos dos que tiravam a espada. E viram
os filhos de Benjamim que estavam feridos; porque os homens de
Israel deram lugar aos benjamitas, porquanto estavam confiados na
emboscada que haviam posto contra Gibeá. E a emboscada se
apressou, e acometeu a Gibeá; e a emboscada arremeteu contra ela, e
feriu ao fio da espada toda a cidade. E os homens de Israel
tinham um sinal determinado com a emboscada, que era fazer levantar
da cidade uma grande nuvem de fumaça. Viraram-se, pois, os
homens de Israel na peleja; e já Benjamim começava a ferir, dos
homens de Israel, quase trinta homens, pois diziam: Já
infalivelmente estão derrotados diante de nós, como na peleja
passada. Então a nuvem de fumaça começou a se levantar da
cidade, como uma coluna; e, virando-se Benjamim a olhar para trás de
si, eis que a fumaça da cidade subia ao céu. E os homens de
Israel viraram os rostos, e os homens de Benjamim pasmaram; porque
viram que o mal lhes tocaria. E viraram as costas diante dos
homens de Israel, para o caminho do deserto; porém a peleja os
apertou; e os que saíam das cidades os destruíram no meio deles.
E cercaram aos de Benjamim, e os perseguiram, e à vontade os
pisaram, até diante de Gibeá, para o nascente do sol. E
caíram de Benjamim dezoito mil homens, todos estes sendo homens
valentes. Então viraram as costas, e fugiram para o deserto,
à penha de Rimom; colheram ainda deles pelos caminhos uns cinco mil
homens; e de perto os seguiram até Gidom, e feriram deles dois mil
homens. E, todos os que caíram de Benjamim, naquele dia,
foram vinte e cinco mil homens que tiravam a espada, todos eles
homens valentes. Porém seiscentos homens viraram as costas, e
fugiram para o deserto, à penha de Rimom; e ficaram na penha de
Rimom quatro meses. E os homens de Israel voltaram para os
filhos de Benjamim, e os feriram ao fio da espada, desde os homens
da cidade até aos animais, até a tudo quanto se achava, como também
a todas as cidades, quantas acharam, puseram fogo.

\medskip

\lettrine{21} Ora, tinham jurado os homens de Israel em Mizpá,
dizendo: Nenhum de nós dará sua filha por mulher aos benjamitas.
Veio, pois, o povo a Betel, e ali ficou até à tarde diante de
Deus; e todos levantaram a sua voz, e prantearam com grande pranto,
e disseram: Ah! Senhor Deus de Israel, por que sucedeu isto, que
hoje falte uma tribo em Israel? E sucedeu que, no dia seguinte,
o povo, pela manhã se levantou, e edificou ali um altar; e ofereceu
holocaustos e ofertas pacíficas. E disseram os filhos de Israel:
Quem de todas as tribos de Israel não subiu à assembléia do Senhor?
Porque se tinha feito um grande juramento acerca dos que não fossem
ao Senhor em Mizpá, dizendo: Morrerá certamente. E
arrependeram-se os filhos de Israel acerca de Benjamim, seu irmão, e
disseram: Cortada é hoje de Israel uma tribo. Como havemos de
conseguir mulheres para os que restaram deles, pois nós temos jurado
pelo Senhor que nenhuma de nossas filhas lhes daríamos por mulher?
E disseram: Há algumas das tribos de Israel que não subiram ao
Senhor a Mizpá? E eis que ninguém de Jabes-Gileade viera ao arraial,
à assembléia. Porquanto, quando se contou o povo, eis que nenhum
dos moradores de Jabes-Gileade se achou ali. Então a
assembléia enviou para lá doze mil homens dos mais valentes, e lhes
ordenou, dizendo: Ide, e ao fio da espada feri aos moradores de
Jabes-Gileade, e às mulheres e aos meninos. Porém isto é o
que haveis de fazer: A todo o homem e a toda a mulher que se houver
deitado com um homem totalmente destruireis. E acharam entre
os moradores de Jabes-Gileade quatrocentas moças virgens, que não
tinham conhecido homem; e as trouxeram ao arraial, a Siló, que está
na terra de Canaã. Então toda a assembléia
enviou\footnote{Enviou o quê? Quem? Edição Contemporânea: ``Então
toda a congregação enviou mensageiros aos filhos de Benjamin,
\ldots''. King James: ``And the whole congregation sent \emph{some}
to speak to the children of Benjamin that \emph{were} in the rock
Rimmon, and to call peaceably unto them.'' RA: Toda a congregação,
pois, enviou mensageiros aos filhos de Benjamim que estavam na penha
Rimom, e lhes proclamaram a paz.}, e falou aos filhos de Benjamim,
que estavam na penha de Rimom, e lhes proclamou a paz. E ao
mesmo tempo voltaram os benjamitas; e deram-lhes as mulheres que
haviam guardado com vida, das mulheres de Jabes-Gileade; porém estas
ainda não lhes bastaram. Então o povo se arrependeu por causa
de Benjamim; porquanto o Senhor tinha feito brecha nas tribos de
Israel.

E disseram os anciãos da assembléia: Que faremos acerca de
mulheres para os que restaram, pois foram destruídas as mulheres de
Benjamim? Disseram mais: Tenha Benjamim uma herança nos que
restaram, e não seja destruída nenhuma tribo de Israel. Porém
nós não lhes poderemos dar mulheres de nossas filhas, porque os
filhos de Israel juraram, dizendo: Maldito aquele que der mulher aos
benjamitas. Então disseram: Eis que de ano em ano há
solenidade do Senhor em Siló, que se celebra para o norte de Betel
do lado do nascente do sol, pelo caminho alto que sobe de Betel a
Siquém, e para o sul de Lebona. E mandaram aos filhos de
Benjamim, dizendo: Ide, e emboscai-vos nas vinhas. E olhai, e
eis aí as filhas de Siló a dançar em rodas, saí vós das vinhas, e
arrebatai cada um sua mulher das filhas de Siló, e ide-vos à terra
de Benjamim. E será que, quando seus pais ou seus irmãos
vierem a litigar conosco, nós lhes diremos: Por amor de nós, tende
compaixão deles, pois nesta guerra não tomamos mulheres para cada um
deles; por que não lhas destes vós, para que agora ficásseis
culpados. E os filhos de Benjamim o fizeram assim, e levaram
mulheres conforme ao número deles, das que arrebataram das rodas que
dançavam; e foram-se, e voltaram à sua herança, e reedificaram as
cidades, e habitaram nelas. Também os filhos de Israel
partiram dali, cada um para a sua tribo e para a sua família; e
saíram dali, cada um para a sua herança. Naqueles dias não
havia rei em Israel; porém cada um fazia o que parecia reto aos seus
olhos.

