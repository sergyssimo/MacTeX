\addchap{Isaías}

\lettrine{1}{}Visão de Isaías, filho de Amós, que ele teve a
respeito de Judá e Jerusalém, nos dias de Uzias, Jotão, Acaz, e
Ezequias, reis de Judá.

Ouvi, ó céus, e dá ouvidos, tu, ó terra; porque o Senhor tem
falado: Criei filhos, e engrandeci-os; mas eles se rebelaram contra
mim. O boi conhece o seu possuidor, e o jumento a manjedoura do
seu dono; mas Israel não tem conhecimento, o meu povo não entende.
Ai, nação pecadora, povo carregado de iniqüidade, descendência
de malfeitores, filhos corruptores; deixaram ao Senhor, blasfemaram
o Santo de Israel, voltaram para trás. Por que seríeis ainda
castigados, se mais vos rebelaríeis? Toda a cabeça está enferma e
todo o coração fraco. Desde a planta do pé até a cabeça não há
nele coisa sã, senão feridas, e inchaços, e chagas podres não
espremidas, nem ligadas, nem amolecidas com óleo. A vossa terra
está assolada, as vossas cidades estão abrasadas pelo fogo; a vossa
terra os estranhos a devoram em vossa presença; e está como
devastada, numa subversão de estranhos. E a filha de Sião é
deixada como a cabana na vinha, como a choupana no pepinal, como uma
cidade sitiada. Se o Senhor dos Exércitos não nos tivesse
deixado algum remanescente, já como Sodoma seríamos, e semelhantes a
Gomorra.

Ouvi a palavra do Senhor, vós poderosos de Sodoma; dai ouvidos à
lei do nosso Deus, ó povo de Gomorra. De que me serve a mim a
multidão de vossos sacrifícios, diz o Senhor? Já estou farto dos
holocaustos de carneiros, e da gordura de animais cevados; nem me
agrado de sangue de bezerros, nem de cordeiros, nem de bodes.
Quando vindes para comparecer perante mim, quem requereu isto
de vossas mãos, que viésseis a pisar os meus átrios? Não
continueis a trazer ofertas vãs; o incenso é para mim abominação, e
as luas novas, e os sábados, e a convocação das assembléias; não
posso suportar iniqüidade, nem mesmo a reunião solene. As
vossas luas novas, e as vossas solenidades, a minha alma as odeia;
já me são pesadas; já estou cansado de as sofrer. Por isso,
quando estendeis as vossas mãos, escondo de vós os meus olhos; e
ainda que multipliqueis as vossas orações, não as ouvirei, porque as
vossas mãos estão cheias de sangue.

Lavai-vos, purificai-vos, tirai a maldade de vossos atos de
diante dos meus olhos; cessai de fazer mal. Aprendei a fazer
bem; procurai o que é justo; ajudai o oprimido; fazei justiça ao
órfão; tratai da causa das viúvas. Vinde então, e argüi-me,
diz o Senhor: ainda que os vossos pecados sejam como a escarlata,
eles se tornarão brancos como a neve; ainda que sejam vermelhos como
o carmesim, se tornarão como a branca lã. Se quiserdes, e
obedecerdes, comereis o bem desta terra. Mas se recusardes, e
fordes rebeldes, sereis devorados à espada; porque a boca do Senhor
o disse.

Como se fez prostituta a cidade fiel! Ela que estava cheia de
retidão! A justiça habitava nela, mas agora homicidas. A tua
prata tornou-se em escórias, o teu vinho se misturou com água.
Os teus príncipes são rebeldes, e companheiros de ladrões;
cada um deles ama as peitas, e anda atrás das recompensas; não fazem
justiça ao órfão, e não chega perante eles a causa da viúva.
Portanto diz o Senhor, o Senhor dos Exércitos, o Forte de
Israel: Ah! tomarei satisfações dos meus adversários, e vingar-me-ei
dos meus inimigos. E voltarei contra ti a minha mão, e
purificarei inteiramente as tuas escórias; e tirar-te-ei toda a
impureza. E te restituirei os teus juízes, como foram dantes;
e os teus conselheiros, como antigamente; e então te chamarão cidade
de justiça, cidade fiel. Sião será remida com juízo, e os que
voltam para ela com justiça. Mas os transgressores e os
pecadores serão juntamente destruídos; e os que deixarem o Senhor
serão consumidos. Porque vos envergonhareis pelos carvalhos
que cobiçastes, e sereis confundidos pelos jardins que escolhestes.
Porque sereis como o carvalho, ao qual caem as folhas, e como
o jardim que não tem água. E o forte se tornará em estopa, e
a sua obra em faísca; e ambos arderão juntamente, e não haverá quem
os apague.

\medskip

\lettrine{2}{}Palavra que viu Isaías, filho de Amós, a respeito
de Judá e de Jerusalém. E acontecerá nos últimos dias que se
firmará o monte da casa do Senhor no cume dos montes, e se elevará
por cima dos outeiros; e concorrerão a ele todas as nações. E
irão muitos povos, e dirão: Vinde, subamos ao monte do Senhor, à
casa do Deus de Jacó, para que nos ensine os seus caminhos, e
andemos nas suas veredas; porque de Sião sairá a lei, e de Jerusalém
a palavra do Senhor. E ele julgará entre as nações, e
repreenderá a muitos povos; e estes converterão as suas espadas em
enxadões e as suas lanças em foices; uma nação não levantará espada
contra outra nação, nem aprenderão mais a guerrear. Vinde, ó
casa de Jacó, e andemos na luz do Senhor.

Mas tu desamparaste o teu povo, a casa de Jacó, porque se encheram
dos costumes do oriente e são agoureiros como os filisteus; e
associam-se com os filhos dos estrangeiros, e a sua terra está
cheia de prata e ouro, e não têm fim os seus tesouros; também a sua
terra está cheia de cavalos, e os seus carros não têm fim.
Também a sua terra está cheia de ídolos; inclinam-se perante a
obra das suas mãos, diante daquilo que fabricaram os seus dedos.
E o povo se abate, e os nobres se humilham; portanto não lhes
perdoarás.

Entra nas rochas, e esconde-te no pó, do terror do Senhor e da
glória da sua majestade. Os olhos altivos dos homens serão
abatidos, e a sua altivez será humilhada; e só o Senhor será
exaltado naquele dia. Porque o dia do Senhor dos Exércitos
será contra todo o soberbo e altivo, e contra todo o que se exalta,
para que seja abatido; e contra todos os cedros do Líbano,
altos e sublimes; e contra todos os carvalhos de Basã; e
contra todos os montes altos, e contra todos os outeiros elevados;
e contra toda a torre alta, e contra todo o muro fortificado;
e contra todos os navios de Társis, e contra todas as
pinturas desejáveis. E a arrogância do homem será humilhada,
e a sua altivez se abaterá, e só o Senhor será exaltado naquele dia.
E todos os ídolos desaparecerão totalmente. Então os
homens entrarão nas cavernas das rochas, e nas covas da terra, do
terror do Senhor, e da glória da sua majestade, quando ele se
levantar para assombrar a terra. Naquele dia o homem lançará
às toupeiras e aos morcegos os seus ídolos de prata, e os seus
ídolos de ouro, que fizeram para diante deles se prostrarem.
E entrarão nas fendas das rochas, e nas cavernas das penhas,
por causa do terror do Senhor, e da glória da sua majestade, quando
ele se levantar para abalar terrivelmente a terra. Deixai-vos
do homem cujo fôlego está nas suas narinas; pois em que se deve ele
estimar?

\medskip

\lettrine{3}{}Porque, eis que o Senhor, o Senhor dos Exércitos,
tirará de Jerusalém e de Judá o sustento e o apoio; a todo o
sustento de pão e a todo o sustento de água; o poderoso, e o
homem de guerra, o juiz, e o profeta, e o adivinho, e o ancião,
o capitão de cinqüenta, e o homem respeitável, e o conselheiro,
e o sábio entre os artífices, e o eloqüente orador. E
dar-lhes-ei meninos por príncipes, e crianças governarão sobre eles.
E o povo será oprimido; um será contra o outro, e cada um contra
o seu próximo; o menino se atreverá contra o ancião, e o vil contra
o nobre. Quando alguém pegar de seu irmão na casa de seu pai,
dizendo: Tu tens roupa, sê nosso governador, e toma sob a tua mão
esta ruína; naquele dia levantará este a sua voz, dizendo: Não
posso ser médico, nem tampouco há em minha casa pão, ou roupa
alguma; não me haveis de constituir governador sobre o povo.
Porque Jerusalém está arruinada, e Judá caída; porque a sua
língua e as suas obras são contra o Senhor, para provocarem os olhos
da sua glória.

O aspecto do seu rosto testifica contra eles; e publicam os seus
pecados, como Sodoma; não os dissimulam. Ai da sua alma! Porque
fazem mal a si mesmos. Dizei ao justo que bem lhe irá; porque
comerão do fruto das suas obras. Ai do ímpio! Mal lhe irá;
porque se lhe fará o que as suas mãos fizeram. Os opressores
do meu povo são crianças, e mulheres dominam sobre ele; ah, povo
meu! Os que te guiam te enganam, e destroem o caminho das tuas
veredas. O Senhor se levanta para pleitear, e põe-se de pé
para julgar os povos. O Senhor entrará em juízo contra os
anciãos do seu povo, e contra os seus príncipes; é que fostes vós
que consumistes esta vinha; o espólio do pobre está em vossas casas.
Que tendes vós, que esmagais o meu povo e moeis as faces dos
pobres? Diz o Senhor Deus dos Exércitos.

Diz ainda mais o Senhor: Porquanto as filhas de Sião se exaltam,
e andam com o pescoço erguido, lançando olhares
impudentes\footnote{Que ou o que não tem pudor; despudorado,
impudico; que revela impudência, falta de vergonha.}; e quando
andam, caminham afetadamente, fazendo um tilintar com os seus pés;
portanto o Senhor fará tinhoso o alto da cabeça das filhas de
Sião, e o Senhor porá a descoberto a sua nudez, naquele dia
tirará o Senhor os ornamentos dos pés, e as toucas, e adornos em
forma de lua, os pendentes, e os braceletes, as estolas,
os gorros, e os ornamentos das pernas, e os cintos e as
caixinhas de perfumes, e os brincos, os anéis, e as jóias do
nariz, os vestidos de festa, e os mantos, e os xales, e as
bolsas. Os espelhos, e o linho finíssimo, e os turbantes, e
os véus. E será que em lugar de perfume haverá mau cheiro; e
por cinto uma corda; e em lugar de
encrespadura\footnote{Encrespamento: ato ou efeito de
encrespar(-se), tornar(-se) crespo.} de cabelos, calvície; e em
lugar de veste luxuosa, pano de saco; e queimadura em lugar de
formosura. Teus homens cairão à espada e teus poderosos na
peleja. E as suas portas gemerão e prantearão; e ela,
desolada, se assentará no chão.

\medskip

\lettrine{4}{}E sete mulheres naquele dia lançarão mão de um
homem, dizendo: Nós comeremos do nosso pão, e nos vestiremos do que
é nosso; tão-somente queremos ser chamadas pelo teu nome; tira o
nosso opróbrio.

Naquele dia o renovo do Senhor será cheio de beleza e de glória; e
o fruto da terra excelente e formoso para os que escaparem de
Israel. E será que aquele que for deixado em Sião, e ficar em
Jerusalém, será chamado santo; todo aquele que estiver inscrito
entre os viventes em Jerusalém; quando o Senhor lavar a
imundícia das filhas de Sião, e limpar o sangue de Jerusalém, do
meio dela, com o espírito de justiça, e com o espírito de ardor.
E criará o Senhor sobre todo o lugar do monte de Sião, e sobre
as suas assembléias, uma nuvem de dia e uma fumaça, e um resplendor
de fogo flamejante de noite; porque sobre toda a glória haverá
proteção. E haverá um tabernáculo para sombra contra o calor do
dia; e para refúgio e esconderijo contra a tempestade e a chuva.

\medskip

\lettrine{5}{}Agora cantarei ao meu amado o cântico do meu
querido a respeito da sua vinha. O meu amado tem uma vinha num
outeiro fértil. E cercou-a, e limpando-a das pedras, plantou-a
de excelentes vides; e edificou no meio dela uma torre, e também
construiu nela um lagar; e esperava que desse uvas boas, porém deu
uvas bravas. Agora, pois, ó moradores de Jerusalém, e homens de
Judá, julgai, vos peço, entre mim e a minha vinha. Que mais se
podia fazer à minha vinha, que eu lhe não tenha feito? Por que,
esperando eu que desse uvas boas, veio a dar uvas bravas? Agora,
pois, vos farei saber o que eu hei de fazer à minha vinha: tirarei a
sua sebe, para que sirva de pasto; derrubarei a sua parede, para que
seja pisada; e a tornarei em deserto; não será podada nem
cavada; porém crescerão nela sarças e espinheiros; e às nuvens darei
ordem que não derramem chuva sobre ela. Porque a vinha do Senhor
dos Exércitos é a casa de Israel, e os homens de Judá são a planta
das suas delícias; e esperou que exercesse juízo, e eis aqui
opressão; justiça, e eis aqui clamor.

Ai dos que ajuntam casa a casa, reúnem campo a campo, até que não
haja mais lugar, e fiquem como únicos moradores no meio da terra!
A meus ouvidos disse o Senhor dos Exércitos: Em verdade que
muitas casas ficarão desertas, e até as grandes e excelentes sem
moradores. E dez jeiras de vinha não darão mais do que um
bato; e um ômer de semente não dará mais do que um efa. Ai
dos que se levantam pela manhã, e seguem a bebedice; e continuam até
à noite, até que o vinho os esquente! E harpas e alaúdes,
tamboris e gaitas, e vinho há nos seus banquetes; e não olham para a
obra do Senhor, nem consideram as obras das suas mãos.
Portanto o meu povo será levado cativo, por falta de
entendimento; e os seus nobres terão fome, e a sua multidão se
secará de sede. Portanto o inferno grandemente se alargou, e
se abriu a sua boca desmesuradamente; e para lá descerão o seu
esplendor, e a sua multidão, e a sua pompa, e os que entre eles se
alegram. Então o plebeu se abaterá, e o nobre se humilhará; e
os olhos dos altivos se humilharão. Porém o Senhor dos
Exércitos será exaltado em juízo; e Deus, o Santo, será santificado
em justiça. Então os cordeiros pastarão como de costume, e os
estranhos comerão dos lugares devastados pelos gordos.

Ai dos que puxam a iniqüidade com cordas de vaidade, e o pecado
com tirantes\footnote{Designação comum às correias ou cordas que
prendem a(s) cavalgadura(s) ao veículo.} de carro! E dizem:
Avie-se\footnote{Intransitivo e pronominal: mover-se ou agir com
maior pressa; apressar(-se). Ex.: Avie-se, que o tempo é curto.}, e
acabe a sua obra, para que a vejamos; e aproxime-se e venha o
conselho do Santo de Israel, para que o conheçamos. Ai dos
que ao mal chamam bem, e ao bem mal; que fazem das trevas luz, e da
luz trevas; e fazem do amargo doce, e do doce amargo! Ai dos
que são sábios a seus próprios olhos, e prudentes diante de si
mesmos! Ai dos que são poderosos para beber vinho, e homens
de poder para misturar bebida forte; dos que justificam ao
ímpio por suborno, e aos justos negam a justiça! Por isso,
como a língua de fogo consome a palha, e o restolho se desfaz pela
chama, assim será a sua raiz como podridão, e a sua flor se
esvaecerá como pó; porquanto rejeitaram a lei do Senhor dos
Exércitos, e desprezaram a palavra do Santo de Israel. Por
isso se acendeu a ira do Senhor contra o seu povo, e estendeu a sua
mão contra ele, e o feriu, de modo que as montanhas tremeram, e os
seus cadáveres se fizeram como lixo no meio das ruas; com tudo isto
não tornou atrás a sua ira, mas a sua mão ainda está estendida.
E ele arvorará o estandarte para as nações de longe, e lhes
assobiará para que venham desde a extremidade da terra; e eis que
virão apressurada e ligeiramente. Não haverá entre eles
cansado, nem quem tropece; ninguém tosquenejará nem dormirá; não se
lhe desatará o cinto dos seus lombos, nem se lhe quebrará a correia
dos seus sapatos. As suas flechas serão agudas, e todos os
seus arcos retesados; os cascos dos seus cavalos são reputados como
pederneiras, e as rodas dos seus carros como redemoinho. O
seu rugido será como o do leão; rugirão como filhos de leão; sim,
rugirão e arrebatarão a presa, e a levarão, e não haverá quem a
livre. E bramarão contra eles naquele dia, como o bramido do
mar; então olharão para a terra, e eis que só verão trevas e ânsia,
e a luz se escurecerá nos céus.

\medskip

\lettrine{6}{}No ano em que morreu o rei Uzias, eu vi também ao
Senhor assentado sobre um alto e sublime trono; e o seu
séquito\footnote{Ou séqüito: conjunto das pessoas que acompanham
outra(s); cortejo que acompanha uma pessoa, ger. distinta, para
servi-la ou honrá-la; comitiva.} enchia o templo. Serafins
estavam por cima dele; cada um tinha seis asas; com duas cobriam os
seus rostos, e com duas cobriam os seus pés, e com duas voavam.
E clamavam uns aos outros, dizendo: Santo, Santo, Santo é o
Senhor dos Exércitos; toda a terra está cheia da sua glória. E
os umbrais das portas se moveram à voz do que clamava, e a casa se
encheu de fumaça.

Então disse eu: Ai de mim! Pois estou perdido; porque sou um homem
de lábios impuros, e habito no meio de um povo de impuros lábios; os
meus olhos viram o Rei, o Senhor dos Exércitos. Porém um dos
serafins voou para mim, trazendo na sua mão uma brasa viva, que
tirara do altar com uma tenaz; e com a brasa tocou a minha boca,
e disse: Eis que isto tocou os teus lábios; e a tua iniqüidade foi
tirada, e expiado o teu pecado. Depois disto ouvi a voz do
Senhor, que dizia: A quem enviarei, e quem há de ir por nós? Então
disse eu: Eis-me aqui, envia-me a mim.

Então disse ele: Vai, e dize a este povo: Ouvis, de fato, e não
entendeis, e vedes, em verdade, mas não percebeis. Engorda o
coração deste povo, e faze-lhe pesados os ouvidos, e fecha-lhe os
olhos; para que ele não veja com os seus olhos, e não ouça com os
seus ouvidos, nem entenda com o seu coração, nem se converta e seja
sarado. Então disse eu: Até quando Senhor? E respondeu: Até
que sejam desoladas as cidades e fiquem sem habitantes, e as casas
sem moradores, e a terra seja de todo assolada. E o Senhor
afaste dela os homens, e no meio da terra seja grande o desamparo.
Porém ainda a décima parte ficará nela, e tornará a ser
pastada; e como o carvalho, e como a azinheira\footnote{Árvore de
até 10 m (Quercus ilex), da fam. das fagáceas, de folhas discolores,
tb. denteadas e espinhosas nos espécimes adultos, flores masculinas
em amentos, as femininas em panículas, e frutos ovóides, revestidos,
em parte, por escamas (Nativa da Europa e Norte da África, a madeira
é us. em obras internas, em pequenos objetos, no fabrico de carvão,
e as folhas servem de alimento ao bicho-da-seda.)}, que depois de se
desfolharem, ainda ficam firmes, assim a santa semente será a
firmeza dela.

\medskip

\lettrine{7}{}Sucedeu, pois, nos dias de Acaz, filho de Jotão,
filho de Uzias, rei de Judá, que Rezim, rei da Síria, e Peca, filho
de Remalias, rei de Israel, subiram a Jerusalém, para pelejarem
contra ela, mas nada puderam contra ela. E deram aviso à casa de
Davi, dizendo: A Síria fez aliança com Efraim. Então se moveu o seu
coração, e o coração do seu povo, como se movem as árvores do bosque
com o vento. Então disse o Senhor a Isaías: Agora, tu e teu
filho Sear-Jasube, saí ao encontro de Acaz, ao fim do canal do
tanque superior, no caminho do campo do lavandeiro. E dize-lhe:
Acautela-te, e aquieta-te; não temas, nem se desanime o teu coração
por causa destes dois pedaços de tições fumegantes; por causa do
ardor da ira de Rezim, e da Síria, e do filho de Remalias.
Porquanto a Síria teve contra ti maligno conselho, com Efraim, e
com o filho de Remalias, dizendo: Vamos subir contra Judá, e
molestemo-lo e repartamo-lo entre nós, e façamos reinar no meio dele
o filho de Tabeal. Assim diz o Senhor Deus: Isto não subsistirá,
nem tampouco acontecerá. Porém a cabeça da Síria será Damasco, e
a cabeça de Damasco Rezim; e dentro de sessenta e cinco anos Efraim
será destruído, e deixará de ser povo. Entretanto a cabeça de
Efraim será Samaria, e a cabeça de Samaria o filho de Remalias; se
não o crerdes, certamente não haveis de permanecer.

E continuou o Senhor a falar com Acaz, dizendo: Pede para
ti ao Senhor teu Deus um sinal; pede-o, ou em baixo nas profundezas,
ou em cima nas alturas. Acaz, porém, disse: Não pedirei, nem
tentarei ao Senhor. Então ele disse: Ouvi agora, ó casa de
Davi: Pouco vos é afadigardes os homens, senão que também
afadigareis ao meu Deus? Portanto o mesmo Senhor vos dará um
sinal: Eis que a virgem conceberá, e dará à luz um filho, e chamará
o seu nome Emanuel. Manteiga e mel comerá, quando ele souber
rejeitar o mal e escolher o bem. Na verdade, antes que este
menino saiba rejeitar o mal e escolher o bem, a terra, de que te
enfadas, será desamparada dos seus dois reis.

Porém o Senhor fará vir sobre ti, e sobre o teu povo, e sobre a
casa de teu pai, pelo rei da Assíria, dias tais, quais nunca vieram,
desde o dia em que Efraim se separou de Judá. Porque há de
acontecer que naquele dia assobiará o Senhor às moscas, que há no
extremo dos rios do Egito, e às abelhas que estão na terra da
Assíria; e todas elas virão, e pousarão nos vales desertos e
nas fendas das rochas, e em todos os espinheiros e em todos os
arbustos. Naquele mesmo dia rapará o Senhor com uma navalha
alugada, que está além do rio, isto é, com o rei da Assíria, a
cabeça e os cabelos dos pés; e até a barba totalmente tirará.
E sucederá naquele dia que um homem criará uma novilha e duas
ovelhas. E acontecerá que por causa da abundância do leite
que elas hão de dar, comerá manteiga; e manteiga e mel comerá todo
aquele que restar no meio da terra. Sucederá também naquele
dia que todo o lugar em que houver mil vides, do valor de mil siclos
de prata, será para as sarças e para os espinheiros. Com arco
e flecha se entrará ali, porque toda a terra será sarças e
espinheiros. E quanto a todos os montes, que costumavam cavar
com enxadas, para ali não irás por causa do temor das sarças e dos
espinheiros; porém servirão para se mandarem para lá os bois e para
serem pisados pelas ovelhas.

\medskip

\lettrine{8}{}Disse-me também o Senhor: Toma um grande rolo, e
escreve nele com caneta de homem: Apressando-se ao despojo,
apressurou-se à presa. Então tomei comigo fiéis testemunhas, a
Urias sacerdote, e a Zacarias, filho de Jeberequias, e fui ter
com a profetiza, e ela concebeu, e deu à luz um filho; e o Senhor me
disse: Põe-lhe o nome de Maer-Salal-Has-Baz. Porque antes que o
menino saiba dizer meu pai, ou minha mãe, se levarão as riquezas de
Damasco, e os despojos de Samaria, diante do rei da Assíria. E
continuou o Senhor a falar ainda comigo, dizendo: Porquanto este
povo desprezou as águas de Siloé que correm brandamente, e
alegrou-se com Rezim e com o filho de Remalias, portanto eis que
o Senhor fará subir sobre eles as águas do rio, fortes e impetuosas,
isto é, o rei da Assíria, com toda a sua glória; e subirá sobre
todos os seus leitos, e transbordará por todas as suas ribanceiras.
E passará a Judá, inundando-o, e irá passando por ele e chegará
até ao pescoço; e a extensão de suas asas encherá a largura da tua
terra, ó Emanuel.

Ajuntai-vos, ó povos, e sereis quebrantados; dai ouvidos, todos os
que sois de terras longínquas; cingi-vos e sereis feitos em pedaços,
cingi-vos e sereis feitos em pedaços. Tomai juntamente
conselho, e ele será frustrado; dizei uma palavra, e ela não
subsistirá, porque Deus é conosco. Porque assim o Senhor me
disse com mão forte, e me ensinou que não andasse pelo caminho deste
povo, dizendo: Não chameis conjuração\footnote{Associação de
indivíduos, às vezes por juramento, para um fim comum. Essa
associação, secreta ou clandestina, ger. contra um governo;
conspiração, trama, inconfidência. União harmônica; conjunção,
entendimento, aliança.}, a tudo quanto este povo chama conjuração; e
não temais o que ele teme, nem tampouco vos assombreis. Ao
Senhor dos Exércitos, a ele santificai; e seja ele o vosso temor e
seja ele o vosso assombro. Então ele vos será por santuário;
mas servirá de pedra de tropeço, e rocha de escândalo, às duas casas
de Israel; por armadilha e laço aos moradores de Jerusalém. E
muitos entre eles tropeçarão, e cairão, e serão quebrantados, e
enlaçados, e presos.

Liga o testemunho, sela a lei entre os meus discípulos. E
esperarei ao Senhor, que esconde o seu rosto da casa de Jacó, e a
ele aguardarei. Eis-me aqui, com os filhos que me deu o
Senhor, por sinais e por maravilhas em Israel, da parte do Senhor
dos Exércitos, que habita no monte de Sião. Quando, pois, vos
disserem: Consultai os que têm espíritos familiares e os adivinhos,
que chilreiam\footnote{Chilrear: emitir chilros (os pássaros);
chalrear, chilrar. Derivação: sentido figurado: cantar ou falar
livre e animadamente, produzindo sons indistintos.} e murmuram:
Porventura não consultará o povo a seu Deus? A favor dos vivos
consultar-se-á aos mortos? À lei e ao testemunho! Se eles não
falarem segundo esta palavra, é porque não há luz neles. E
passarão pela terra duramente oprimidos e famintos; e será que,
tendo fome, e enfurecendo-se, então amaldiçoarão ao seu rei e ao seu
Deus, olhando para cima. E, olhando para a terra, eis que
haverá angústia e escuridão, e sombras de ansiedade, e serão
empurrados para as trevas.

\medskip

\lettrine{9}{}Mas a terra, que foi angustiada, não será
entenebrecida\footnote{Entenebrecer: recobrir(-se) de trevas;
turbar(-se), toldar(-se), entenebrar(-se), entrevar(-se). Sentido
figurado: causar tristeza a; entenebrar, entristecer.};
envileceu\footnote{Envilecer: tornar(-se) vil, desprezível;
aviltar(-se), deslustrar(-se), desonrar(-se), humilhar(-se).
Reduzir(-se) o valor de; baixar de preço; depreciar(-se).} nos
primeiros tempos, a terra de Zebulom, e a terra de Naftali; mas nos
últimos tempos a enobreceu junto ao caminho do mar, além do Jordão,
na Galiléia das nações. O povo que andava em trevas, viu uma
grande luz, e sobre os que habitavam na região da sombra da morte
resplandeceu a luz. Tu multiplicaste a nação, a alegria lhe
aumentaste; todos se alegrarão perante ti, como se alegram na ceifa,
e como exultam quando se repartem os despojos. Porque tu
quebraste o jugo da sua carga, e o bordão do seu ombro, e a vara do
seu opressor, como no dia dos midianitas. Porque todo calçado
que levava o guerreiro no tumulto da batalha, e todo o manto
revolvido em sangue, serão queimados, servindo de combustível ao
fogo. Porque um menino nos nasceu, um filho se nos deu, e o
principado está sobre os seus ombros, e se chamará o seu nome:
Maravilhoso, Conselheiro, Deus Forte, Pai da Eternidade, Príncipe da
Paz. Do aumento deste principado e da paz não haverá fim, sobre
o trono de Davi e no seu reino, para o firmar e o fortificar com
juízo e com justiça, desde agora e para sempre; o zelo do Senhor dos
Exércitos fará isto.

O Senhor enviou uma palavra a Jacó, e ela caiu em Israel. E
todo este povo o saberá, Efraim e os moradores de Samaria, que em
soberba e altivez de coração, dizem: Os tijolos caíram, mas
com cantaria tornaremos a edificar; cortaram-se os sicômoros, mas em
cedros as mudaremos. Portanto o Senhor suscitará, contra ele,
os adversários de Rezim, e juntará os seus inimigos. Pela
frente virão os sírios, e por detrás os filisteus, e devorarão a
Israel à boca escancarada; e nem com tudo isto cessou a sua ira, mas
ainda está estendida a sua mão. Todavia este povo não se
voltou para quem o feria, nem buscou ao Senhor dos Exércitos.
Assim o Senhor cortará de Israel a cabeça e a cauda, o ramo e
o junco, num mesmo dia15(O ancião e o homem de respeito é a cabeça;
e o profeta que ensina a falsidade é a cauda). Porque os
guias deste povo são enganadores, e os que por eles são guiados são
destruídos. Por isso o Senhor não se regozija nos seus
jovens, e não se compadecerá dos seus órfãos e das suas viúvas,
porque todos eles são hipócritas e malfazejos\footnote{Que ou aquele
que se compraz fazendo o mal; perverso, malvado. Que traz prejuízo;
nocivo, daninho, maléfico.}, e toda a boca profere
doidices\footnote{Perturbação das faculdades mentais, do juízo;
loucura. Derivação: por extensão de sentido: ato ou dito impensado,
extravagante, exagerado que traduza algum descontrole; doidaria,
doudaria.}; e nem com tudo isto cessou a sua ira, mas ainda está
estendida a sua mão. Porque a impiedade lavra como um fogo,
ela devora as sarças e os espinheiros; e ela se ateará no emaranhado
da floresta; e subirão em espessas nuvens de fumaça. Por
causa da ira do Senhor dos Exércitos a terra se escurecerá, e será o
povo como combustível para o fogo; ninguém poupará ao seu irmão.
Se colher à direita, ainda terá fome, e se comer à esquerda,
ainda não se fartará; cada um comerá a carne de seu braço.
Manassés a Efraim, e Efraim a Manassés, e ambos serão contra
Judá. Com tudo isto não cessou a sua ira, mas ainda está estendida a
sua mão.

\medskip

\lettrine{10}{}Ai dos que decretam leis injustas, e dos
escrivães que prescrevem opressão. Para desviarem os pobres do
seu direito, e para arrebatarem o direito dos aflitos do meu povo;
para despojarem as viúvas e roubarem os órfãos! Mas que fareis
vós no dia da visitação, e na desolação, que há de vir de longe? A
quem recorrereis para obter socorro, e onde deixareis a vossa
glória, sem que cada um se abata entre os presos, e caia entre
mortos? Com tudo isto a sua ira não cessou, mas ainda está estendida
a sua mão.

Ai da Assíria, a vara da minha ira, porque a minha indignação é
como bordão nas suas mãos. Enviá-la-ei contra uma nação
hipócrita, e contra o povo do meu furor lhe darei ordem, para que
lhe roube a presa, e lhe tome o despojo, e o ponha para ser pisado
aos pés, como a lama das ruas. Ainda que ele não cuide assim,
nem o seu coração assim o imagine; antes no seu coração intenta
destruir e desarraigar não poucas nações. Porque diz: Não são
meus príncipes todos eles reis? Não é Calno como Carquemis? Não
é Hamate como Arpade? E Samaria como Damasco? Como a minha
mão alcançou os reinos dos ídolos, cujas imagens esculpidas eram
melhores do que as de Jerusalém e do que as de Samaria,
porventura como fiz a Samaria e aos seus ídolos, não o faria
igualmente a Jerusalém e aos seus ídolos? Por isso acontecerá
que, havendo o Senhor acabado toda a sua obra no monte Sião e em
Jerusalém, então castigarei o fruto da arrogante grandeza do coração
do rei da Assíria e a pompa da altivez dos seus olhos.
Porquanto disse: Com a força da minha mão o fiz, e com a
minha sabedoria, porque sou prudente; e removi os limites dos povos,
e roubei os seus tesouros, e como valente abati aos habitantes.
E achou a minha mão as riquezas dos povos como a um ninho, e
como se ajuntam os ovos abandonados, assim eu ajuntei a toda a
terra, e não houve quem movesse a asa, ou abrisse a boca, ou
murmurasse. Porventura gloriar-se-á o machado contra o que
corta com ele, ou presumirá a serra contra o que puxa por ela, como
se o bordão movesse aos que o levantam, ou a vara levantasse como
não sendo pau? Por isso o Senhor, o Senhor dos Exércitos,
fará definhar os que entre eles são gordos, e debaixo da sua glória
ateará um incêndio, como incêndio de fogo. Porque a Luz de
Israel virá a ser como fogo e o seu Santo por labareda, que abrase e
consuma os seus espinheiros e as suas sarças num só dia.
Também consumirá a glória da sua floresta, e do seu campo
fértil, desde a alma até à carne, e será como quando desmaia o
porta-bandeira. E o resto das árvores da sua floresta será
tão pouco em número, que um menino poderá contá-las.

E acontecerá naquele dia que os restantes de Israel, e os que
tiverem escapado da casa de Jacó, nunca mais se estribarão sobre
aquele que os feriu; antes estribar-se-ão verdadeiramente sobre o
Senhor, o Santo de Israel. Os restantes se converterão ao
Deus forte, sim, os restantes de Jacó. Porque ainda que o teu
povo, ó Israel, seja como a areia do mar, só um remanescente dele se
converterá; uma destruição está determinada, transbordando em
justiça. Porque determinada já a destruição, o Senhor Deus
dos Exércitos a executará no meio de toda esta terra.

Por isso assim diz o Senhor Deus dos Exércitos: Povo meu, que
habitas em Sião, não temas à Assíria, quando te ferir com a vara, e
contra ti levantar o seu bordão à maneira dos egípcios.
Porque daqui a bem pouco se cumprirá a minha indignação e a
minha ira, para a consumir. Porque o Senhor dos Exércitos
suscitará contra ela um flagelo, como na matança de Midiã junto à
rocha de Orebe; e a sua vara estará sobre o mar, e ele a levantará
como sucedeu aos egípcios. E acontecerá, naquele dia, que a
sua carga será tirada do teu ombro, e o seu jugo do teu pescoço; e o
jugo será despedaçado por causa da unção. Já vem chegando a
Aiate, já vai passando por Migrom, e em Micmás deixa a sua bagagem.
Já passaram o desfiladeiro, já se alojam em Geba; já Ramá
treme, e Gibeá de Saul vai fugindo. Clama alto com a tua voz,
ó filha de Galim! Ouve, ó Laís! Ó tu pobre Anatote! Madmena
já se foi; os moradores de Gebim vão fugindo em bandos. Ainda
um dia parará em Nobe; acenará com a sua mão contra o monte da filha
de Sião, o outeiro de Jerusalém. Mas eis que o Senhor, o
Senhor dos Exércitos, cortará os ramos com violência, e os de alta
estatura serão cortados, e os altivos serão abatidos. E
cortará com ferro a espessura da floresta, e o Líbano cairá à mão de
um poderoso.

\medskip

\lettrine{11}{}Porque brotará um rebento do tronco de Jessé, e
das suas raízes um renovo frutificará. E repousará sobre ele o
Espírito do Senhor, o espírito de sabedoria e de entendimento, o
espírito de conselho e de fortaleza, o espírito de conhecimento e de
temor do Senhor. E deleitar-se-á no temor do Senhor; e não
julgará segundo a vista dos seus olhos, nem repreenderá segundo o
ouvir dos seus ouvidos. Mas julgará com justiça aos pobres, e
repreenderá com eqüidade aos mansos da terra; e ferirá a terra com a
vara de sua boca, e com o sopro dos seus lábios matará ao ímpio,
e a justiça será o cinto dos seus lombos, e a fidelidade o cinto
dos seus rins. E morará o lobo com o cordeiro, e o leopardo com
o cabrito se deitará, e o bezerro, e o filho de leão e o animal
cevado andarão juntos, e um menino pequeno os guiará. A vaca e a
ursa pastarão juntas, seus filhos se deitarão juntos, e o leão
comerá palha como o boi. E brincará a criança de peito sobre a
toca da áspide, e a desmamada colocará a sua mão na cova do
basilisco\footnote{Lagarto ou serpente fabulosa, cujo olhar e cujo
bafo, dizia-se, tinham o poder de matar. Lagartos do gên.
Basiliscus, da fam. dos iguanídeos, que ocorrem do México à
Colômbia; de longas patas traseiras e dedos com expansões laterais,
o que lhes permite correr sobre a água.}. Não se fará mal nem
dano algum em todo o meu santo monte, porque a terra se encherá do
conhecimento do Senhor, como as águas cobrem o mar.

E acontecerá naquele dia que a raiz de Jessé, a qual estará posta
por estandarte dos povos, será buscada pelos gentios; e o lugar do
seu repouso será glorioso. E há de ser que naquele dia o
Senhor tornará a pôr a sua mão para adquirir outra vez o
remanescente do seu povo, que for deixado, da Assíria, e do Egito, e
de Patros, e da Etiópia, e de Elã, e de Sinar, e de Hamate, e das
ilhas do mar. E levantará um estandarte entre as nações, e
ajuntará os desterrados de Israel, e os dispersos de Judá congregará
desde os quatro confins da terra. E afastar-se-á a inveja de
Efraim, e os adversários de Judá serão desarraigados; Efraim não
invejará a Judá, e Judá não oprimirá a Efraim. Antes voarão
sobre os ombros dos filisteus ao ocidente; juntos despojarão aos do
oriente; em Edom e Moabe porão as suas mãos, e os filhos de Amom
lhes obedecerão. E o Senhor destruirá totalmente a língua do
mar do Egito, e moverá a sua mão contra o rio com a força do seu
vento e, ferindo-o, dividi-lo-á em sete correntes e fará que por ele
passem com sapatos secos. E haverá caminho plano para o
remanescente do seu povo, que for deixado da Assíria, como sucedeu a
Israel no dia em que subiu da terra do Egito.

\medskip

\lettrine{12}{}E dirás naquele dia: Graças te dou, ó Senhor,
porque, ainda que te iraste contra mim, a tua ira se retirou, e tu
me consolas. Eis que Deus é a minha salvação; nele confiarei, e
não temerei, porque o Senhor Deus é a minha força e o meu cântico, e
se tornou a minha salvação. E vós com alegria tirareis águas das
fontes da salvação.

E direis naquele dia: Dai graças ao Senhor, invocai o seu nome,
fazei notório os seus feitos entre os povos, contai quão excelso é o
seu nome. Cantai ao Senhor, porque fez coisas grandiosas;
saiba-se isto em toda a terra. Exulta e jubila, ó habitante de
Sião, porque grande é o Santo de Israel no meio de ti.

\medskip

\lettrine{13}{}Peso de Babilônia, que viu Isaías, filho de
Amós. Alçai uma bandeira sobre o monte elevado, levantai a voz
para eles; acenai-lhes com a mão, para que entrem pelas portas dos
nobres. Eu dei ordens aos meus santificados; sim, já chamei os
meus poderosos para executarem a minha ira, os que exultam com a
minha majestade. Já se ouve a gritaria da multidão sobre os
montes, como a de muito povo; o som do rebuliço de reinos e de
nações congregados. O Senhor dos Exércitos passa em revista o
exército de guerra. Já vem de uma terra remota, desde a
extremidade do céu, o Senhor, e os instrumentos da sua indignação,
para destruir toda aquela terra.

Clamai, pois, o dia do Senhor está perto; vem do Todo-Poderoso
como assolação. Portanto, todas as mãos se debilitarão, e o
coração de todos os homens se desanimará. E assombrar-se-ão, e
apoderar-se-ão deles dores e ais, e se angustiarão, como a mulher
com dores de parto; cada um se espantará do seu próximo; os seus
rostos serão rostos flamejantes. Eis que vem o dia do Senhor,
horrendo, com furor e ira ardente, para pôr a terra em assolação, e
dela destruir os pecadores. Porque as estrelas dos céus e as
suas constelações não darão a sua luz; o sol se escurecerá ao
nascer, e a lua não resplandecerá com a sua luz. E visitarei
sobre o mundo a maldade, e sobre os ímpios a sua iniqüidade; e farei
cessar a arrogância dos atrevidos, e abaterei a soberba dos tiranos.
Farei que o homem seja mais precioso do que o ouro puro, e
mais raro do que o ouro fino de Ofir. Por isso farei
estremecer os céus; e a terra se moverá do seu lugar, por causa do
furor do Senhor dos Exércitos, e por causa do dia da sua ardente
ira. E cada um será como a corça que foge, e como a ovelha
que ninguém recolhe; cada um voltará para o seu povo, e cada um
fugirá para a sua terra. Todo o que for achado será
transpassado; e todo o que se unir a ele cairá à espada. E
suas crianças serão despedaçadas perante os seus olhos; as suas
casas serão saqueadas, e as suas mulheres violadas. Eis que
eu despertarei contra eles os medos, que não farão caso da prata,
nem tampouco desejarão ouro. E os seus arcos despedaçarão os
jovens, e não se compadecerão do fruto do ventre; os seus olhos não
pouparão aos filhos.

E Babilônia, o ornamento dos reinos, a glória e a soberba dos
caldeus, será como Sodoma e Gomorra, quando Deus as transtornou.
Nunca mais será habitada, nem nela morará alguém de geração
em geração; nem o árabe armará ali a sua tenda, nem tampouco os
pastores ali farão deitar os seus rebanhos. Mas as feras do
deserto repousarão ali, e as suas casas se encherão de horríveis
animais; e ali habitarão os avestruzes, e os sátiros\footnote{Homem
devasso, luxurioso. Relativo aos sátiros, povo fabuloso da África,
ou indivíduo desse povo.} pularão ali. E os animais selvagens
das ilhas uivarão em suas casas vazias, como também os chacais nos
seus palácios de prazer; pois bem perto já vem chegando o seu tempo,
e os seus dias não se prolongarão.

\medskip

\lettrine{14}{}Porque o Senhor se compadecerá de Jacó, e ainda
escolherá a Israel e os porá na sua própria terra; e ajuntar-se-ão
com eles os estrangeiros, e se achegarão à casa de Jacó. E os
povos os receberão, e os levarão aos seus lugares, e a casa de
Israel os possuirá por servos, e por servas, na terra do Senhor; e
cativarão aqueles que os cativaram, e dominarão sobre os seus
opressores. E acontecerá que no dia em que o Senhor vier a
dar-te descanso do teu sofrimento, e do teu pavor, e da dura
servidão com que te fizeram servir.

Então proferirás este provérbio contra o rei de Babilônia, e
dirás: Como já cessou o opressor, como já cessou a cidade dourada!
Já quebrantou o Senhor o bastão dos ímpios e o cetro dos
dominadores. Aquele que feria aos povos com furor, com golpes
incessantes, e que com ira dominava sobre as nações agora é
perseguido, sem que alguém o possa impedir. Já descansa, já está
sossegada toda a terra; rompem cantando. Até as
faias\footnote{Design. comum às árvores do gên. Fagus e Nothofagus,
da fam. das fagáceas. Árvore (Fagus sylvatica) com madeira de
qualidade, folhas ovadas ou elípticas, flores monóicas e cápsulas
deiscentes e espinhosas, nativa da Europa, cultivada como ornamental
e pelas sementes de que se extrai óleo transparente e adocicado;
faia-ordinária.} se alegram sobre ti, e os cedros do Líbano,
dizendo: Desde que tu caíste ninguém sobe contra nós para nos
cortar. O inferno desde o profundo se turbou por ti, para te
sair ao encontro na tua vinda; despertou por ti os mortos, e todos
os chefes da terra, e fez levantar dos seus tronos a todos os reis
das nações. Estes todos responderão, e te dirão: Tu também
adoeceste como nós, e foste semelhante a nós. Já foi
derrubada na sepultura a tua soberba com o som das tuas violas; os
vermes debaixo de ti se estenderão, e os bichos te cobrirão.
Como caíste desde o céu, ó estrela da manhã\footnote{AV: How
art thou fallen from heaven, \textbf{O Lucifer, son of the morning}!
how art thou cut down to the ground, which didst weaken the
nations!}, filha da alva! Como foste cortado por terra, tu que
debilitavas as nações! E tu dizias no teu coração: Eu subirei
ao céu, acima das estrelas de Deus exaltarei o meu trono, e no monte
da congregação me assentarei, aos lados do norte. Subirei
sobre as alturas das nuvens, e serei semelhante ao Altíssimo.
E contudo levado serás ao inferno, ao mais profundo do
abismo. Os que te virem te contemplarão, considerar-te-ão, e
dirão: É este o homem que fazia estremecer a terra e que fazia
tremer os reinos? Que punha o mundo como o deserto, e
assolava as suas cidades? Que não abria a casa de seus cativos?
Todos os reis das nações, todos eles, jazem com honra, cada
um na sua morada. Porém tu és lançado da tua sepultura, como
um renovo abominável, como as vestes dos que foram mortos
atravessados à espada, como os que descem ao covil de pedras, como
um cadáver pisado. Com eles não te reunirás na sepultura;
porque destruíste a tua terra e mataste o teu povo; a descendência
dos malignos não será jamais nomeada. Preparai a matança para
os seus filhos por causa da maldade de seus pais, para que não se
levantem, e nem possuam a terra, e encham a face do mundo de
cidades. Porque me levantarei contra eles, diz o Senhor dos
Exércitos, e extirparei de Babilônia o nome, e os sobreviventes, o
filho e o neto, diz o Senhor. E farei dela uma possessão de
ouriços e a lagoas de águas; e varrê-la-ei com vassoura de perdição,
diz o Senhor dos Exércitos.

O Senhor dos Exércitos jurou, dizendo: Como pensei, assim
sucederá, e como determinei, assim se efetuará. Quebrantarei
a Assíria na minha terra, e nas minhas montanhas a pisarei, para que
o seu jugo se aparte deles e a sua carga se desvie dos seus ombros.
Este é o propósito que foi determinado sobre toda a terra; e
esta é a mão que está estendida sobre todas as nações. Porque
o Senhor dos Exércitos o determinou; quem o invalidará? E a sua mão
está estendida; quem pois a fará voltar atrás? No ano em que
morreu o rei Acaz, foi dada esta sentença. Não te alegres,
tu, toda a Filístia, por estar quebrada a vara que te feria; porque
da raiz da cobra sairá um basilisco, e o seu fruto será uma serpente
ardente, voadora. E os primogênitos dos pobres serão
apascentados, e os necessitados se deitarão seguros; porém farei
morrer de fome a tua raiz, e ele matará os teus sobreviventes.
Dá uivos, ó porta, grita, ó cidade; tu, ó Filístia, estás
toda derretida; porque do norte vem uma fumaça, e não haverá quem
fique sozinho nas suas convocações. Que se responderá, pois,
aos mensageiros da nação? Que o Senhor fundou a Sião, para que os
opressos do seu povo nela encontrem refúgio.

\medskip

\lettrine{15}{}Peso de Moabe. Certamente numa noite foi
destruída Ar de Moabe, e foi desfeita; certamente numa noite foi
destruída Quir de Moabe e foi desfeita. Vai subindo a Bajite, e
a Dibom, aos lugares altos, para chorar; por Nebo e por Medeba
clamará Moabe; todas as cabeças ficarão calvas, e toda a barba será
rapada. Cingiram-se de sacos nas suas ruas; nos seus terraços e
nas suas praças todos andam gritando, e choram abundantemente.
Assim Hesbom como Eleale, andam gritando; até Jaaz se ouve a sua
voz; por isso os armados de Moabe clamam; a sua alma lhes será
penosa. O meu coração clama por causa de Moabe; os seus
fugitivos foram até Zoar, como uma novilha de três anos; porque vão
chorando pela subida de Luíte, porque no caminho de Horonaim
levantam um lastimoso pranto.

Porque as águas de Ninrim serão pura assolação; porque já secou o
feno, acabou a erva, e não há verdura alguma. Por isso a
abundância que ajuntaram, e o que guardaram, ao ribeiro dos
salgueiros o levarão. Porque o pranto rodeará aos limites de
Moabe; até Eglaim chegará o seu clamor, e ainda até Beer-Elim
chegará o seu lamento. Porquanto as águas de Dimom estão cheias
de sangue, porque ainda acrescentarei mais a Dimom; leões contra
aqueles que escaparem de Moabe e contra o restante da terra.

\medskip

\lettrine{16}{}Enviai o cordeiro ao governador da terra, desde
Sela, no deserto, até ao monte da filha de Sião. De outro modo
sucederá que serão as filhas de Moabe junto aos vaus de Arnom como o
pássaro vagueante, lançado fora do ninho. Toma conselho, executa
juízo, põe a tua sombra no pino do meio-dia como a noite; esconde os
desterrados, e não descubras os fugitivos. Habitem contigo os
meus desterrados, ó Moabe; serve-lhes de refúgio perante a face do
destruidor; porque o homem violento terá fim; a destruição é
desfeita, e os opressores são consumidos sobre a terra. Porque o
trono se firmará em benignidade, e sobre ele no tabernáculo de Davi
se assentará em verdade um que julgue, e busque o juízo, e se
apresse a fazer justiça.

Ouvimos da soberba de Moabe, que é soberbíssimo; da sua altivez,
da sua soberba, e do seu furor; porém, as suas mentiras não serão
firmes. Portanto Moabe clamará por Moabe; todos clamarão;
gemereis pelos fundamentos de Quir-Haresete, pois certamente já
estão abatidos. Porque os campos de Hesbom enfraqueceram, e a
vinha de Sibma; os senhores dos gentios quebraram as suas melhores
plantas que haviam chegado a Jazer e vagueiam no deserto; os seus
rebentos se estenderam e passaram além do mar. Por isso
prantearei, com o pranto de Jazer, a vinha de Sibma; regar-te-ei com
as minhas lágrimas, ó Hesbom e Eleale; porque o júbilo dos teus
frutos de verão e da tua sega desapareceu. E fugiu a alegria
e o regozijo do campo fértil, e nas vinhas não se canta, nem há
júbilo algum; já não se pisarão as uvas nos lagares. Eu fiz cessar o
júbilo. Por isso o meu íntimo vibra por Moabe como harpa, e o
meu interior por Quir-Heres. E será que, quando virem Moabe
cansado nos altos, então entrará no seu santuário a orar, porém não
prevalecerá. Esta é a palavra que o Senhor falou contra Moabe
desde aquele tempo. Porém agora falou o Senhor, dizendo:
Dentro de três anos (tais como os anos de
jornaleiros\footnote{Diz-se de ou trabalhador a quem se paga jornal
(remuneração salarial feita por dia de trabalho)}), será envilecida
a gloria de Moabe, com toda a sua grande multidão; e o restante será
pouco, pequeno e impotente.

\medskip

\lettrine{17}{}Peso de Damasco. Eis que Damasco será tirada, e
já não será cidade, antes será um montão de ruínas. As cidades
de Aroer serão abandonadas; hão de ser para os rebanhos que se
deitarão sem que alguém os espante. E a fortaleza de Efraim
cessará, como também o reino de Damasco e o restante da Síria; serão
como a glória dos filhos de Israel, diz o Senhor dos Exércitos.
E naquele dia será diminuída a glória de Jacó, e a gordura da
sua carne ficará emagrecida. Porque será como o segador que
colhe a cana do trigo e com o seu braço sega as espigas; e será
também como o que colhe espigas no vale de Refaim.

Porém ainda ficarão nele alguns rabiscos, como no sacudir da
oliveira: duas ou três azeitonas na mais alta ponta dos ramos, e
quatro ou cinco nos seus ramos mais frutíferos, diz o Senhor Deus de
Israel. Naquele dia atentará o homem para o seu Criador, e os
seus olhos olharão para o Santo de Israel. E não atentará para
os altares, obra das suas mãos, nem olhará para o que fizeram seus
dedos, nem para os bosques, nem para as imagens.

Naquele dia as suas cidades fortificadas serão como lugares
abandonados, no bosque ou sobre o cume das montanhas, os quais foram
abandonados ante os filhos de Israel; e haverá assolação.
Porque te esqueceste do Deus da tua salvação, e não te
lembraste da rocha da tua fortaleza, portanto farás plantações
formosas, e assentarás nelas sarmentos\footnote{Ramo de videira.}
estranhos. E no dia em que as plantares as farás crescer, e
pela manhã farás que a tua semente brote; mas a colheita voará no
dia da angústia e das dores insofríveis.

Ai do bramido dos grandes povos que bramam como bramam os mares,
e do rugido das nações que rugem como rugem as impetuosas águas.
Rugirão as nações, como rugem as muitas águas, mas Deus as
repreenderá e elas fugirão para longe; e serão afugentadas como a
pragana\footnote{Sobra dos grãos, depois de joeirados; rabeira,
moinha.} dos montes diante do vento, e como o que rola levado pelo
tufão. Ao anoitecer eis que há pavor, mas antes que amanheça
já não existe; esta é a parte daqueles que nos despojam, e a sorte
daqueles que nos saqueiam.

\medskip

\lettrine{18}{}Ai da terra que ensombreia com as suas asas, que
está além dos rios da Etiópia. Que envia embaixadores por mar em
navios de junco sobre as águas, dizendo: Ide, mensageiros velozes, a
um povo de elevada estatura e de pele lisa; a um povo terrível desde
o seu princípio; a uma nação forte e esmagadora, cuja terra os rios
dividem. Vós, todos os habitantes do mundo, e vós os moradores
da terra, quando se arvorar a bandeira nos montes, o vereis; e
quando se tocar a trombeta, o ouvireis. Porque assim me disse o
Senhor: Estarei quieto, olhando desde a minha morada, como o ardor
do sol resplandecente depois da chuva, como a nuvem do orvalho no
calor da sega. Porque antes da sega, quando já o fruto está
perfeito e, passada a flor, as uvas verdes amadurecerem, então, com
foice podará os sarmentos e tirará os ramos e os lançará fora.
Serão deixados juntos às aves dos montes e aos animais da terra;
e sobre eles veranearão\footnote{Veranear: passar o verão de folga,
em local aprazível.} as aves de rapina, e todos os animais da terra
invernarão sobre eles. Naquele tempo trará um presente ao Senhor
dos Exércitos um povo de elevada estatura e de pele lisa, e um povo
terrível desde o seu princípio; uma nação forte e esmagadora, cuja
terra os rios dividem; ao lugar do nome do Senhor dos Exércitos, ao
monte Sião.

\medskip

\lettrine{19}{}Peso do Egito. Eis que o Senhor vem cavalgando
numa nuvem ligeira, e entrará no Egito; e os ídolos do Egito
estremecerão diante dele, e o coração dos egípcios se derreterá no
meio deles. Porque farei com que os egípcios, se levantem contra
os egípcios, e cada um pelejará contra o seu irmão, e cada um contra
o seu próximo, cidade contra cidade, reino contra reino. E o
espírito do Egito se esvaecerá no seu interior, e destruirei o seu
conselho; e eles consultarão aos seus ídolos, e encantadores, e
aqueles que têm espíritos familiares e feiticeiros. E entregarei
os egípcios nas mãos de um senhor cruel, e um rei rigoroso os
dominará, diz o Senhor, o Senhor dos Exércitos. E secarão as
águas do mar, e o rio se esgotará e ressequirá. Também os rios
exalarão mau cheiro e se esgotarão e secarão os canais do Egito; as
canas e os juncos murcharão. A relva junto ao rio, junto às
ribanceiras dos rios, e tudo o que foi semeado junto ao rio, secará,
será arrancado e não subsistirá. E os pescadores gemerão, e
suspirarão todos os que lançam anzol ao rio, e os que estendem rede
sobre as águas desfalecerão. E envergonhar-se-ão os que
trabalham em linho fino, e os que tecem pano branco. E os
seus fundamentos serão despedaçados, e todos os que trabalham por
salário ficarão com tristeza de alma. Na verdade são loucos
os príncipes de Zoã; o conselho dos sábios conselheiros de Faraó se
embruteceu; como, pois, a Faraó direis: Sou filho de sábios, filho
de antigos reis? Onde estão agora os teus sábios?
Notifiquem-te agora, ou informem-te sobre o que o Senhor dos
Exércitos determinou contra o Egito. Loucos tornaram-se os
príncipes de Zoã, enganados estão os príncipes de Nofe; eles fizeram
errar o Egito, aqueles que são a pedra de esquina das suas tribos.
O Senhor derramou no meio dele um perverso espírito; e eles
fizeram errar o Egito em toda a sua obra, como o bêbado quando se
revolve no seu vômito. E não aproveitará ao Egito obra alguma
que possa fazer a cabeça, a cauda, o ramo, ou o junco.
Naquele tempo os egípcios serão como mulheres, e tremerão e
temerão por causa do movimento da mão do Senhor dos Exércitos, que
há de levantar-se contra eles. E a terra de Judá será um
espanto para o Egito; todo aquele a quem isso se anunciar se
assombrará, por causa do propósito do Senhor dos Exércitos, que
determinou contra eles.

Naquele tempo haverá cinco cidades na terra do Egito que falarão
a língua de Canaã e farão juramento ao Senhor dos Exércitos; e uma
se chamará: Cidade de destruição. Naquele tempo o Senhor terá
um altar no meio da terra do Egito, e uma coluna se erigirá ao
Senhor, junto da sua fronteira. E servirá de sinal e de
testemunho ao Senhor dos Exércitos na terra do Egito, porque ao
Senhor clamarão por causa dos opressores, e ele lhes enviará um
salvador e um protetor, que os livrará. E o Senhor se dará a
conhecer ao Egito, e os egípcios conhecerão ao Senhor naquele dia, e
o adorarão com sacrifícios e ofertas, e farão votos ao Senhor, e os
cumprirão. E ferirá o Senhor ao Egito, ferirá e o curará; e
converter-se-ão ao Senhor, e mover-se-á às suas orações, e os
curará. Naquele dia haverá estrada do Egito até à Assíria, e
os assírios virão ao Egito, e os egípcios irão à Assíria; e os
egípcios servirão com os assírios. Naquele dia Israel será o
terceiro com os egípcios e os assírios, uma bênção no meio da terra.
Porque o Senhor dos Exércitos os abençoará, dizendo: Bendito
seja o Egito, meu povo, e a Assíria, obra de minhas mãos, e Israel,
minha herança.

\medskip

\lettrine{20}{}No ano em que Tartã, enviado por Sargom, rei da
Assíria, veio a Asdode, e guerreou contra ela, e a tomou, nesse
mesmo tempo falou o Senhor por intermédio de Isaías, filho de Amós,
dizendo: Vai, solta o cilício\footnote{Antiga veste ou faixa de
crina ou de pano grosseiro e áspero us. sobre a pele por penitência.
Cinto ou cordão eriçado de cerdas ou correntes de ferro, cheio de
pontas, com que os penitentes cingem o corpo diretamente sobre a
pele. Derivação - sentido figurado -sacrifício ou mortificação a que
alguém se sujeita voluntariamente.} de teus lombos, e descalça os
sapatos dos teus pés. E ele assim o fez, indo nu e descalço.
Então disse o Senhor: Assim como o meu servo Isaías andou três
anos nu e descalço, por sinal e prodígio sobre o Egito e sobre a
Etiópia, assim o rei da Assíria levará em cativeiro os presos do
Egito, e os exilados da Etiópia, tanto moços como velhos, nus e
descalços, e com as nádegas descobertas, para vergonha do Egito.
E assombrar-se-ão, e envergonhar-se-ão, por causa dos etíopes,
sua esperança, como também dos egípcios, sua glória. Então os
moradores desta ilha dirão naquele dia: Vede que tal é a nossa
esperança, à qual fugimos por socorro, para nos livrarmos da face do
rei da Assíria! Como pois escaparemos nós?

\medskip

\lettrine{21}{}Peso do deserto do mar. Como os tufões de vento
do sul, que tudo assolam, ele virá do deserto, de uma terra
horrível. Dura visão me foi anunciada: o pérfido trata
perfidamente, e o destruidor anda destruindo. Sobe, ó Elão, sitia, ó
Média, que já fiz cessar todo o seu gemido. Por isso os meus
lombos estão cheios de angústia; dores se apoderam de mim como as
dores daquela que dá à luz; fiquei abatido quando ouvi, e desanimado
vendo isso. O meu coração se agita, o horror apavora-me; a noite
que desejava, se me tornou em temor. Põem-se a mesa, estão de
atalaia, comem, bebem; levantai-vos, príncipes, e untai o escudo.
Porque assim me disse o Senhor: Vai, põe uma sentinela, e ela
que diga o que vir. E quando vir um carro com um par de
cavaleiros, um carro com jumentos, e um carro com camelos, ela que
observe atentamente com grande cuidado. E clamou: Um leão, meu
Senhor!\footnote{RA: Então, o atalaia gritou como um leão: Senhor,
sobre a torre de vigia estou em pé continuamente durante o dia e de
guarda me ponho noites inteiras. RC: E clamou como um leão: Senhor,
sobre a torre de vigia estou em pé continuamente de dia e de guarda
me ponho noites inteiras. KJ: And he cried, A lion: My lord, I stand
continually upon the watchtower in the daytime, and I am set in my
ward whole nights:} Sobre a torre de vigia estou em pé continuamente
de dia, e de guarda me ponho noites inteiras. E eis agora vem um
carro com homens, e um par de cavaleiros. Então respondeu e disse:
Caída é Babilônia, caída é! E todas as imagens de escultura dos seus
deuses quebraram-se no chão. Ah, malhada minha, e trigo da
minha eira! O que ouvi do Senhor dos Exércitos, Deus de Israel, isso
vos anunciei.

Peso de Dumá. Gritam-me de Seir: Guarda, que houve de noite?
Guarda, que houve de noite? E disse o guarda: Vem a manhã, e
também a noite; se quereis perguntar, perguntai; voltai, vinde.

Peso contra Arábia. Nos bosques da Arábia passareis a noite, ó
viandantes de Dedanim. Saí com água ao encontro dos sedentos;
moradores da terra de Tema, saí com pão ao encontro dos fugitivos.
Porque fogem de diante das espadas, de diante da espada
desembainhada, e de diante do arco armado, e de diante do peso da
guerra. Porque assim me disse o Senhor: Dentro de um ano,
como os anos de jornaleiro, desaparecerá toda a glória de Quedar.
E os restantes do número dos flecheiros, os poderosos dos
filhos de Quedar, serão diminuídos, porque assim disse o Senhor Deus
de Israel.

\medskip

\lettrine{22}{}Peso do vale da visão. Que tens agora, pois que
com todos os teus subiste aos telhados? Tu, cheia de clamores,
cidade turbulenta, cidade alegre, os teus mortos não foram mortos à
espada, nem morreram na guerra. Todos os teus governadores
juntamente fugiram, foram atados pelos arqueiros; todos os que em ti
se acharam, foram amarrados juntamente, e fugiram para longe.
Portanto digo: Desviai de mim a vista, e chorarei amargamente;
não vos canseis mais em consolar-me pela destruição da filha do meu
povo. Porque dia de alvoroço, e de atropelamento, e de confusão
é este da parte do Senhor Deus dos Exércitos, no vale da visão; dia
de derrubar o muro e de clamar até aos montes. Porque Elão tomou
a aljava, juntamente com carros de homens e cavaleiros; e Quir
descobriu os escudos. E os teus mais formosos vales se encherão
de carros, e os cavaleiros se colocarão em ordem às portas.

E ele tirou a coberta de Judá, e naquele dia olhaste para as armas
da casa do bosque. E vistes as brechas da cidade de Davi,
porquanto já eram muitas, e ajuntastes as águas do tanque de baixo.
Também contastes as casas de Jerusalém, e derrubastes as
casas, para fortalecer os muros. Fizestes também um
reservatório entre os dois muros para as águas do tanque velho,
porém não olhastes acima, para aquele que isto tinha feito, nem
considerastes o que o formou desde a antiguidade. E o Senhor
Deus dos Exércitos, chamou naquele dia para chorar e para prantear,
e para raspar a cabeça, e cingir com o cilício. Porém eis
aqui gozo e alegria, matam-se bois e degolam-se ovelhas, come-se
carne, e bebe-se vinho, e diz-se: Comamos e bebamos, porque amanhã
morreremos. Mas o Senhor dos Exércitos revelou-se aos meus
ouvidos, dizendo: Certamente esta maldade não vos será expiada até
que morrais, diz o Senhor Deus dos Exércitos.

Assim diz o Senhor Deus dos Exércitos: Anda e vai ter com este
tesoureiro, com Sebna, o mordomo, e dize-lhe: Que é que tens
aqui, ou a quem tens tu aqui, para que cavasses aqui uma sepultura?
Cavando em lugar alto a sua sepultura, e cinzelando na rocha uma
morada para ti mesmo? Eis que o Senhor te arrojará
violentamente como um homem forte, e de todo te envolverá.
Certamente com violência te fará rolar, como se faz rolar uma
bola num país espaçoso; ali morrerás, e ali acabarão os carros da
tua glória, ó opróbrio da casa do teu Senhor. E demitir-te-ei
do teu posto, e te arrancarei do teu assento. E será naquele
dia que chamarei a meu servo Eliaquim, filho de Hilquias; e
vesti-lo-ei da tua túnica, e cingi-lo-ei com o teu cinto, e
entregarei nas suas mãos o teu domínio, e será como pai para os
moradores de Jerusalém, e para a casa de Judá. E porei a
chave da casa de Davi sobre o seu ombro, e abrirá, e ninguém
fechará; e fechará, e ninguém abrirá.
 E fixá-lo-ei como a um prego num lugar firme, e será como um
trono de honra para a casa de seu pai. E nele pendurarão toda
a honra da casa de seu pai, a prole e os descendentes, como também
todos os vasos menores, desde as taças até os frascos.
Naquele dia, diz o Senhor dos Exércitos, o prego fincado em
lugar firme será tirado; e será cortado, e cairá, e a carga que nele
estava se desprenderá, porque o Senhor o disse.

\medskip

\lettrine{23}{}Peso de Tiro. Uivai, navios de Társis, porque
está assolada, a ponto de não haver nela casa nenhuma, e de ninguém
mais entrar nela; desde a terra de Quitim lhes foi isto revelado.
Calai-vos, moradores da ilha, vós a quem encheram os mercadores
de Sidom, navegando pelo mar. E a sua provisão era a semente de
Sior, que vinha com as muitas águas, a ceifa do Nilo, e ela era a
feira das nações. Envergonha-te, ó Sidom, porque o mar, a
fortaleza do mar, fala, dizendo: Eu não tive dores de parto, nem dei
à luz, nem ainda criei jovens, nem eduquei virgens. Como quando
se ouviram as novas do Egito, assim haverá dores quando se ouvirem
as de Tiro. Passai a Társis; clamai, moradores da ilha. É
esta, porventura, a vossa cidade exultante, cuja origem é dos dias
antigos, cujos pés a levaram para longe a peregrinar? Quem
formou este desígnio contra Tiro, distribuidora de coroas, cujos
mercadores são príncipes e cujos negociantes são os mais nobres da
terra? O Senhor dos Exércitos formou este desígnio para denegrir
a soberba de toda a glória, e envilecer os mais nobres da terra.
Passa como o Nilo pela tua terra, ó filha de Társis; já não
há quem te restrinja. Ele estendeu a sua mão sobre o mar, e
turbou os reinos; o Senhor deu ordens contra Canaã, para que se
destruíssem as suas fortalezas. E disse: Nunca mais exultarás
de alegria, ó oprimida virgem, filha de Sidom; levanta-te, passa a
Quitim, e ainda ali não terás descanso. Vede a terra dos
caldeus, ainda este povo não era povo; a Assíria a fundou para os
que moravam no deserto; levantaram as suas fortalezas, e edificaram
os seus palácios; porém converteu-a em ruína. Uivai, navios
de Társis, porque está destruída a vossa fortaleza.

Naquele dia Tiro será posta em esquecimento por setenta anos,
conforme os dias de um rei; porém no fim de setenta anos Tiro
cantará como uma prostituta. Toma a harpa, rodeia a cidade, ó
prostituta entregue ao esquecimento; faça doces melodias, canta
muitas canções, para que haja memória de ti. Porque será no
fim de setenta anos que o Senhor visitará a Tiro, e ela tornará à
sua ganância de prostituta, e prostituir-se-á com todos os reinos
que há sobre a face da terra. E o seu comércio e a sua
ganância de prostituta serão consagrados ao Senhor; não se
entesourará, nem se fechará; mas o seu comércio será para os que
habitam perante o Senhor, para que comam até se saciarem, e tenham
vestimenta durável.

\medskip

\lettrine{24}{}Eis que o Senhor esvazia a terra, e a desola, e
transtorna a sua superfície, e dispersa os seus moradores. E o
que suceder ao povo, assim sucederá ao sacerdote; ao servo, como ao
seu senhor; à serva, como à sua senhora; ao comprador, como ao
vendedor; ao que empresta, como ao que toma emprestado; ao que dá
usura, como ao que paga usura. De todo se esvaziará a terra, e
de todo será saqueada, porque o Senhor pronunciou esta palavra.
A terra pranteia e se murcha; o mundo enfraquece e se murcha;
enfraquecem os mais altos do povo da terra. Na verdade a terra
está contaminada por causa dos seus moradores; porquanto têm
transgredido as leis, mudado os estatutos, e quebrado a aliança
eterna. Por isso a maldição tem consumido a terra; e os que
habitam nela são desolados; por isso são queimados os moradores da
terra, e poucos homens restam. Pranteia o mosto, enfraquece a
vide; e suspiram todos os alegres de coração. Cessa o folguedo
dos tamboris, acaba o ruído dos que exultam, e cessa a alegria da
harpa. Com canções não beberão vinho; a bebida forte será amarga
para os que a beberem. Demolida está a cidade vazia, todas as
casas fecharam, ninguém pode entrar. Há lastimoso clamor nas
ruas por falta do vinho; toda a alegria se escureceu, desterrou-se o
gozo da terra. Na cidade só ficou a desolação, a porta ficou
reduzida a ruínas.

Porque assim será no interior da terra, e no meio destes povos,
como a sacudidura da oliveira, e como os rabiscos\footnote{KJ: When
thus it shall be in the midst of the land among the people, there
shall be as the shaking of an olive tree, and as the gleaning grapes
when the vintage is done. RA: Porque será na terra, no meio destes
povos, como o varejar da oliveira e como o rebuscar, quando está
acabada a vindima.}, quando está acabada a vindima. Estes
alçarão a sua voz, e cantarão com alegria; e por causa da glória do
Senhor exultarão desde o mar. Por isso glorificai ao Senhor
no oriente, e nas ilhas do mar, ao nome do Senhor Deus de Israel.

Dos confins da terra ouvimos cantar: Glória ao justo. Mas eu
disse: Emagreço, emagreço, ai de mim! Os pérfidos têm tratado
perfidamente; sim, os pérfidos têm tratado perfidamente. O
temor, e a cova, e o laço vêm sobre ti, ó morador da terra. E
será que aquele que fugir da voz de temor cairá na cova, e o que
subir da cova o laço o prenderá; porque as janelas do alto estão
abertas, e os fundamentos da terra tremem. De todo está
quebrantada a terra, de todo está rompida a terra, e de todo é
movida a terra. De todo cambaleará a terra como o ébrio, e
será movida e removida como a choça\footnote{RA: A terra cambaleará
como um bêbado e balanceará como rede de dormir; a sua transgressão
pesa sobre ela, ela cairá e jamais se levantará. KJ: The earth shall
reel to and fro like a drunkard, and shall be removed like a
cottage; and the transgression thereof shall be heavy upon it; and
it shall fall, and not rise again.} de noite; e a sua transgressão
se agravará sobre ela, e cairá, e nunca mais se levantará. E
será que naquele dia o Senhor castigará os exércitos do alto nas
alturas, e os reis da terra sobre a terra. E serão ajuntados
como presos numa masmorra, e serão encerrados num cárcere; e outra
vez serão castigados depois de muitos dias. E a lua se
envergonhará, e o sol se confundirá quando o Senhor dos Exércitos
reinar no monte Sião e em Jerusalém, e perante os seus anciãos
gloriosamente.

\medskip

\lettrine{25}{}Ó Senhor, tu és o meu Deus; exaltar-te-ei, e
louvarei o teu nome, porque fizeste maravilhas; os teus conselhos
antigos são verdade e firmeza. Porque da cidade fizeste um
montão de pedras, e da cidade forte uma ruína, e do
paço\footnote{Habitação suntuosa para a realeza ou o episcopado;
palácio. Derivação: por metonímia: conjunto de pessoas que habitam
esse palácio. Edifício onde se reúne o conselho ou a câmara
municipal.} dos estranhos, que não seja mais cidade, e jamais se
torne a edificar. Por isso te glorificará um povo poderoso, e a
cidade das nações formidáveis te temerá. Porque foste a
fortaleza do pobre, e a fortaleza do necessitado, na sua angústia;
refúgio contra a tempestade, e sombra contra o calor; porque o sopro
dos opressores é como a tempestade contra o muro. Como o calor
em lugar seco, assim abaterás o ímpeto dos estranhos; como se
abranda o calor pela sombra da espessa nuvem, assim o cântico dos
tiranos será humilhado.

E o Senhor dos Exércitos dará neste monte a todos os povos uma
festa com animais gordos, uma festa de vinhos velhos, com tutanos
gordos, e com vinhos velhos, bem purificados. E destruirá neste
monte a face da cobertura, com que todos os povos andam cobertos, e
o véu com que todas as nações se cobrem. Aniquilará a morte para
sempre, e assim enxugará o Senhor Deus as lágrimas de todos os
rostos, e tirará o opróbrio do seu povo de toda a terra; porque o
Senhor o disse.

E naquele dia se dirá: Eis que este é o nosso Deus, a quem
aguardávamos, e ele nos salvará; este é o Senhor, a quem
aguardávamos; na sua salvação gozaremos e nos alegraremos.
Porque a mão do Senhor descansará neste monte; mas Moabe será
trilhado debaixo dele, como se trilha a palha no
monturo\footnote{Monte de lixo, aglomeração de coisas velhas e
descartadas; montureira. Lugar onde se deposita o lixo. Amontoado de
coisas repugnantes, repulsivas, asquerosas.}. E estenderá as
suas mãos por entre eles, como as estende o nadador para nadar; e
abaterá a sua altivez com as ciladas das suas mãos. E
abaixará as altas fortalezas dos teus muros, abatê-las-á e
derrubá-las-á por terra até ao pó.

\medskip

\lettrine{26}{}Naquele dia se entoará este cântico na terra de
Judá: Temos uma cidade forte, a que Deus pôs a salvação por muros e
antemuros. Abri as portas, para que entre nelas a nação justa,
que observa a verdade. Tu conservarás em paz aquele cuja mente
está firme em ti; porque ele confia em ti. Confiai no Senhor
perpetuamente; porque o Senhor Deus é uma rocha eterna.

Porque ele abate os que habitam no alto, na cidade elevada;
humilha-a, humilha-a até ao chão, e derruba-a até ao pó. O pé
pisá-la-á; os pés dos aflitos, e os passos dos pobres. O caminho
do justo é todo plano; tu retamente pesas o andar do justo.
Também no caminho dos teus juízos, Senhor, te esperamos; no teu
nome e na tua memória está o desejo da nossa alma. Com minha
alma te desejei de noite, e com o meu espírito, que está dentro de
mim, madrugarei a buscar-te; porque, havendo os teus juízos na
terra, os moradores do mundo aprendem justiça. Ainda que se
mostre favor ao ímpio, nem por isso aprende a justiça; até na terra
da retidão ele pratica a iniqüidade, e não atenta para a majestade
do Senhor. Senhor, a tua mão está exaltada, mas nem por isso
a vêem; vê-la-ão, porém, e confundir-se-ão por causa do zelo que
tens do teu povo; e o fogo consumirá os teus adversários.

Senhor, tu nos darás a paz, porque tu és o que fizeste em nós
todas as nossas obras. Ó Senhor Deus nosso, já outros
senhores têm tido domínio sobre nós; porém, por ti só, nos lembramos
de teu nome. Morrendo eles, não tornarão a viver; falecendo,
não ressuscitarão; por isso os visitaste e destruíste, e apagaste
toda a sua memória. Tu, Senhor, aumentaste a esta nação, tu
aumentaste a esta nação, fizeste-te glorioso; alargaste todos os
confins da terra. Ó Senhor, na angústia te buscaram; vindo
sobre eles a tua correção, derramaram a sua oração secreta.
Como a mulher grávida, quando está próxima a sua hora, tem
dores de parto, e dá gritos nas suas dores, assim fomos nós diante
de ti, ó Senhor! Bem concebemos nós e tivemos dores de parto,
porém demos à luz o vento; livramento não trouxemos à terra, nem
caíram os moradores do mundo. Os teus mortos e também o meu
cadáver viverão e ressuscitarão; despertai e exultai, os que
habitais no pó, porque o teu orvalho será como o orvalho das ervas,
e a terra lançará de si os mortos.

Vai, pois, povo meu, entra nos teus quartos, e fecha as tuas
portas sobre ti; esconde-te só por um momento, até que passe a ira.
Porque eis que o Senhor sairá do seu lugar, para castigar os
moradores da terra, por causa da sua iniqüidade, e a terra
descobrirá o seu sangue, e não encobrirá mais os seu mortos.

\medskip

\lettrine{27}{}Naquele dia o Senhor castigará com a sua dura
espada, grande e forte, o leviatã, serpente veloz, e o leviatã, a
serpente tortuosa, e matará o dragão, que está no mar. Naquele
dia haverá uma vinha de vinho tinto; cantai-lhe. Eu, o Senhor, a
guardo, e cada momento a regarei; para que ninguém lhe faça dano, de
noite e de dia a guardarei. Não há indignação em mim. Quem me
poria sarças e espinheiros diante de mim na guerra? Eu iria contra
eles e juntamente os queimaria. Ou que se apodere da minha
força, e faça paz comigo; sim, que faça paz comigo. Dias virão
em que Jacó lançará raízes, e florescerá e brotará Israel, e
encherão de fruto a face do mundo.

Feriu-o como feriu aos que o feriram? Ou matou-o, assim como matou
aos que foram mortos por ele? Com medida contendeste com ela,
quando a rejeitaste, quando a tirou com o seu vento forte, no tempo
do vento leste. Por isso se expiará a iniqüidade de Jacó, e este
será todo o fruto de se haver tirado seu pecado; quando ele fizer a
todas as pedras do altar como pedras de cal feitas em pedaços, então
os bosques e as imagens não poderão ficar em pé. Porque a
cidade fortificada ficará solitária, será uma habitação rejeitada e
abandonada como um deserto; ali pastarão os bezerros, e ali se
deitarão, e devorarão os seus ramos. Quando os seus ramos se
secarem, serão quebrados, e vindo as mulheres, os acenderão, porque
este povo não é povo de entendimento, assim aquele que o fez não se
compadecerá dele, e aquele que o formou não lhe mostrará nenhum
favor. E será naquele dia que o Senhor debulhará seus cereais
desde as correntes do rio, até ao rio do Egito; e vós, ó filhos de
Israel, sereis colhidos um a um. E será naquele dia que se
tocará uma grande trombeta, e os que andavam perdidos pela terra da
Assíria, e os que foram desterrados para a terra do Egito, tornarão
a vir, e adorarão ao Senhor no monte santo em Jerusalém.

\medskip

\lettrine{28}{}Ai da coroa de soberba dos bêbados de Efraim,
cujo glorioso ornamento é como a flor que cai, que está sobre a
cabeça do fértil vale dos vencidos do vinho. Eis que o Senhor
tem um forte e poderoso; como tempestade de saraiva, tormenta
destruidora, e como tempestade de impetuosas águas que transbordam,
ele, com a mão, derrubará por terra. A coroa de soberba dos
bêbados de Efraim será pisada aos pés. E a flor caída do seu
glorioso ornamento, que está sobre a cabeça do fértil vale, será
como o fruto temporão antes do verão, que, vendo-o alguém, e tendo-o
ainda na mão, o engole. Naquele dia o Senhor dos Exércitos será
por coroa gloriosa, e por diadema formosa, para os restantes de seu
povo. E por espírito de juízo, para o que se assenta a julgar, e
por fortaleza para os que fazem recuar a peleja até à porta. Mas
também estes erram por causa do vinho, e com a bebida forte se
desencaminham; até o sacerdote e o profeta erram por causa da bebida
forte; são absorvidos pelo vinho; desencaminham-se por causa da
bebida forte; andam errados na visão e tropeçam no juízo. Porque
todas as suas mesas estão cheias de vômitos e imundícia, e não há
lugar limpo.

A quem, pois, se ensinaria o conhecimento? E a quem se daria a
entender doutrina? Ao desmamado do leite, e ao arrancado dos seios?
Porque é mandamento sobre mandamento, mandamento sobre
mandamento, regra sobre regra, regra sobre regra, um pouco aqui, um
pouco ali. Assim por lábios gaguejantes, e por outra língua,
falará a este povo. Ao qual disse: Este é o descanso, dai
descanso ao cansado; e este é o refrigério; porém não quiseram
ouvir. Assim, pois, a palavra do Senhor lhes será mandamento
sobre mandamento, mandamento sobre mandamento, regra sobre regra,
regra sobre regra, um pouco aqui, um pouco ali; para que vão, e
caiam para trás, e se quebrantem e se enlacem, e sejam presos.

Ouvi, pois, a palavra do Senhor, homens escarnecedores, que
dominais este povo que está em Jerusalém. Porquanto dizeis:
Fizemos aliança com a morte, e com o inferno fizemos acordo; quando
passar o dilúvio do açoite, não chegará a nós, porque pusemos a
mentira por nosso refúgio, e debaixo da falsidade nos escondemos.
Portanto assim diz o Senhor Deus: Eis que eu assentei em Sião
uma pedra, uma pedra já provada, pedra preciosa de esquina, que está
bem firme e fundada; aquele que crer não se apresse. E
regrarei o juízo pela linha, e a justiça pelo prumo, e a saraiva
varrerá o refúgio da mentira, e as águas cobrirão o esconderijo.
E a vossa aliança com a morte se anulará; e o vosso acordo
com o inferno não subsistirá; e, quando o dilúvio do açoite passar,
então sereis por ele pisados. Desde que comece a passar, vos
arrebatará, porque manhã após manhã passará, de dia e de noite; e
será que somente o ouvir tal notícia causará grande turbação.
Porque a cama será tão curta que ninguém se poderá estender
nela; e o cobertor tão estreito que ninguém se poderá cobrir com
ele. Porque o Senhor se levantará como no monte Perazim, e se
irará, como no vale de Gibeão, para fazer a sua obra, a sua estranha
obra, e para executar o seu ato, o seu estranho ato. Agora,
pois, não mais escarneçais, para que vossos grilhões não se façam
mais fortes; porque já ao Senhor Deus dos Exércitos ouvi falar de
uma destruição, e essa já está determinada sobre toda a terra.

Inclinai os ouvidos, e ouvi a minha voz; atendei bem e ouvi o meu
discurso. Porventura lavra todo o dia o lavrador, para
semear? Ou abre e desterroa\footnote{Desterroar: retirar terra de.}
todo o dia a sua terra? Não é antes assim: quando já tem
nivelado a sua superfície, então espalha nela
ervilhaca\footnote{Design. comum a várias ervas do gên. Vicia, da
fam. das leguminosas, subfam. papilionoídea. Erva anual ou bienal
(Vicia sativa), de flores purpúreas, azuladas ou raramente brancas e
vagens eretas, achatadas, nativa da Europa e us. como forragem e
adubo verde.}, e semeia cominho\footnote{Design. comum a plantas de
diferentes fam., esp. às do gên. Cuminum e Carum, das umbelíferas,
us. como condimentares. Erva anual de até 50 cm (Cuminum cyminum),
nativa do Mediterrâneo, de folhas multífidas, com segmentos
filiformes, flores freq. brancas, tb. róseas ou avermelhadas, em
umbelas, e frutos pubescentes, biaquênicos, com aquênios amarelados
ou ruivos, oleaginosos, esp. us. como condimento, desde a
Antigüidade, mas ger. substituídos pelos da alcaravia.}; ou lança
nela do melhor trigo, ou cevada escolhida, ou centeio, cada qual no
seu lugar? O seu Deus o ensina, e o instrui acerca do que há
de fazer. Porque a ervilhaca não se trilha com trilho, nem
sobre o cominho passa roda de carro; mas com uma vara se sacode a
ervilhaca, e o cominho com um pau. O trigo é esmiuçado, mas
não se trilha continuamente, nem se esmiúça com as rodas do seu
carro, nem se quebra com os seus cavaleiros. Até isto procede
do Senhor dos Exércitos; porque é maravilhoso em conselho e grande
em obra.

\medskip

\lettrine{29}{}Ai de Ariel, Ariel, a cidade onde Davi acampou!
Acrescentai ano a ano, e sucedam-se as festas. Contudo porei a
Ariel em aperto, e haverá pranto e tristeza; e ela será para mim
como Ariel. Porque te cercarei com o meu arraial, e te sitiarei
com baluartes, e levantarei trincheiras contra ti. Então serás
abatida, falarás de debaixo da terra, e a tua fala desde o pó sairá
fraca, e será a tua voz debaixo da terra, como a de um que tem
espírito familiar, e a tua fala assobiará desde o pó. E a
multidão dos teus inimigos será como o pó miúdo, e a multidão dos
tiranos como a pragana que passa, e num momento repentino isso
acontecerá. Do Senhor dos Exércitos serás visitada com trovões,
e com terremotos, e grande ruído com tufão de vento, e tempestade, e
labareda de fogo consumidor. E como o sonho e uma visão de noite
será a multidão de todas as nações que hão de pelejar contra Ariel,
como também todos os que pelejarem contra ela e contra a sua
fortaleza, e a puserem em aperto. Será também como o faminto que
sonha, que está a comer, porém, acordando, sente-se vazio; ou como o
sedento que sonha que está a beber, porém, acordando, eis que ainda
desfalecido se acha, e a sua alma com sede; assim será toda a
multidão das nações, que pelejarem contra o monte Sião.

Tardai, e maravilhai-vos, folgai, e clamai; bêbados estão, mas não
de vinho, andam titubeando, mas não de bebida forte. Porque o
Senhor derramou sobre vós um espírito de profundo sono, e fechou os
vossos olhos, vendou os profetas, e os vossos principais videntes.
Por isso toda a visão vos é como as palavras de um livro
selado que se dá ao que sabe ler, dizendo: Lê isto, peço-te; e ele
dirá: Não posso, porque está selado. Ou dá-se o livro ao que
não sabe ler, dizendo: Lê isto, peço-te; e ele dirá: Não sei ler.
Porque o Senhor disse: Pois que este povo se aproxima de mim,
e com a sua boca, e com os seus lábios me honra, mas o seu coração
se afasta para longe de mim e o seu temor para comigo consiste só em
mandamentos de homens, em que foi instruído; portanto eis que
continuarei a fazer uma obra maravilhosa no meio deste povo, uma
obra maravilhosa e um assombro; porque a sabedoria dos seus sábios
perecerá, e o entendimento dos seus prudentes se esconderá.
Ai dos que querem esconder profundamente o seu propósito do
Senhor, e fazem as suas obras às escuras, e dizem: Quem nos vê? E
quem nos conhece? Vós tudo perverteis, como se o oleiro fosse
igual ao barro, e a obra dissesse do seu artífice: Não me fez; e o
vaso formado dissesse do seu oleiro: Nada sabe.

Porventura não se converterá o Líbano, num breve momento, em
campo fértil? E o campo fértil não se reputará por um bosque?
E naquele dia os surdos ouvirão as palavras do livro, e
dentre a escuridão e dentre as trevas os olhos dos cegos as verão.
E os mansos terão gozo sobre gozo no Senhor; e os
necessitados entre os homens se alegrarão no Santo de Israel.
Porque o tirano é reduzido a nada, e se consome o
escarnecedor, e todos os que se dão à iniqüidade são desarraigados;
os que fazem culpado ao homem por uma palavra, e armam laços
ao que repreende na porta, e os que sem motivo põem de parte o
justo. Portanto assim diz o Senhor, que remiu a Abraão,
acerca da casa de Jacó: Jacó não será agora envergonhado, nem agora
se descorará a sua face. Mas quando ele vir seus filhos, obra
das minhas mãos no meio dele, santificarão o meu nome; sim,
santificarão ao Santo de Jacó, e temerão ao Deus de Israel. E
os errados de espírito virão a ter entendimento, e os murmuradores
aprenderão doutrina.

\medskip

\lettrine{30}{}Ai dos filhos rebeldes, diz o Senhor, que tomam
conselho, mas não de mim; e que se cobrem, com uma cobertura, mas
não do meu espírito, para acrescentarem pecado sobre pecado; que
descem ao Egito, sem pedirem o meu conselho; para se fortificarem
com a força de Faraó, e para confiarem na sombra do Egito.
Porque a força de Faraó se vos tornará em vergonha, e a
confiança na sombra do Egito em confusão. Porque os seus
príncipes já estão em Zoã, e os seus embaixadores já chegaram a
Hanes. Todos se envergonharão de um povo que de nada lhes
servirá nem de ajuda, nem de proveito, porém de vergonha, e de
opróbrio. Peso dos animais do sul. Para a terra de aflição e de
angústia (de onde vêm a leoa e o leão, a víbora, e a serpente
ardente, voadora) levarão às costas de jumentinhos as suas riquezas,
e sobre as corcovas de camelos os seus tesouros, a um povo que de
nada lhes aproveitará. Porque o Egito os ajudará em vão, e para
nenhum fim; por isso clamei acerca disto: No estarem quietos será a
sua força.

Vai, pois, agora, escreve isto numa tábua perante eles e
registra-o num livro; para que fique até ao último dia, para sempre
e perpetuamente. Porque este é um povo rebelde, filhos
mentirosos, filhos que não querem ouvir a lei do Senhor. Que
dizem aos videntes: Não vejais; e aos profetas: Não profetizeis para
nós o que é reto; dizei-nos coisas aprazíveis, e vede para nós
enganos. Desviai-vos do caminho, apartai-vos da vereda; fazei
que o Santo de Israel cesse de estar perante nós. Por isso,
assim diz o Santo de Israel: porquanto rejeitais esta palavra, e
confiais na opressão e perversidade, e sobre isso vos estribais,
por isso esta maldade vos será como a brecha de um alto muro
que, formando uma barriga, está prestes a cair e cuja quebra virá
subitamente. E ele o quebrará como se quebra o vaso do oleiro
e, quebrando-o, não se compadecerá; de modo que não se achará entre
os seus pedaços um caco para tomar fogo do lar, ou tirar água da
poça. Porque assim diz o Senhor Deus, o Santo de Israel:
Voltando e descansando sereis salvos; no sossego e na confiança
estaria a vossa força, mas não quisestes. Mas dizeis: Não;
antes sobre cavalos fugiremos; portanto fugireis; e, sobre cavalos
ligeiros cavalgaremos; por isso os vossos perseguidores também serão
ligeiros. Mil homens fugirão ao grito de um, e ao grito de
cinco todos vós fugireis, até que sejais deixados como o mastro no
cume do monte, e como a bandeira no outeiro.

Por isso, o Senhor esperará, para ter misericórdia de vós; e por
isso se levantará, para se compadecer de vós, porque o Senhor é um
Deus de eqüidade; bem-aventurados todos os que nele esperam.
Porque o povo habitará em Sião, em Jerusalém; não chorarás
mais; certamente se compadecerá de ti, à voz do teu clamor e,
ouvindo-a, te responderá. Bem vos dará o Senhor pão de
angústia e água de aperto, mas os teus mestres nunca mais fugirão de
ti, como voando com asas; antes os teus olhos verão a todos os teus
mestres. E os teus ouvidos ouvirão a palavra do que está por
detrás de ti, dizendo: Este é o caminho, andai nele, sem vos
desviardes nem para a direita nem para a esquerda. E terás
por contaminadas as coberturas de tuas esculturas de prata, e o
revestimento das tuas esculturas fundidas de ouro; e as lançarás
fora como um pano imundo, e dirás a cada uma delas: Fora daqui.
Então te dará chuva sobre a tua semente, com que semeares a
terra, como também pão da novidade da terra; e esta será fértil e
cheia; naquele dia o teu gado pastará em largos pastos. E os
bois e os jumentinhos, que lavram a terra, comerão grão puro, que
for padejado com a pá, e cirandado com a ciranda\footnote{Peneira
grossa de palha para joeirar grãos.}. E em todo o monte alto,
e em todo o outeiro levantado, haverá ribeiros e correntes de águas,
no dia da grande matança, quando caírem as torres. E a luz da
lua será como a luz do sol, e a luz do sol sete vezes maior, como a
luz de sete dias, no dia em que o Senhor ligar a quebradura do seu
povo, e curar a chaga da sua ferida.

Eis que o nome do Senhor vem de longe, ardendo a sua ira, sendo
pesada a sua carga; os seus lábios estão cheios de indignação, e a
sua língua é como um fogo consumidor. E a sua respiração como
o ribeiro transbordante, que chega até ao pescoço, para peneirar as
nações com peneira de destruição, e um freio de fazer errar nas
queixadas dos povos. Um cântico haverá entre vós, como na
noite em que se celebra uma festa santa; e alegria de coração, como
a daquele que vai com flauta, para entrar no monte do Senhor, à
Rocha de Israel. E o Senhor fará ouvir a sua voz majestosa e
fará ver o abaixamento do seu braço, com indignação de ira, e
labareda de fogo consumidor, raios e dilúvio e pedras de saraiva.
Porque com a voz do Senhor será desfeita em pedaços a
Assíria, que feriu com a vara. E a cada pancada do bordão do
juízo que o Senhor lhe der, haverá tamboris e harpas; e com combates
de agitação combaterá contra eles. Porque Tofete\footnote{Era
um `lugar alto' no vale de Hinom, nas cercanias imediatas de
Jerusalém, onde em tempos idos eram sacrificadas crianças em honra a
certa divindade chamada Moloque. Josias profanou esse santuário
idólatra, e Jeremias profetizou que o lugar seria usado como um
cemitério (7.32). O significado e a etimologia do vocábulo são
incertos. A derivação geralmente aceita assevera que o termo se
deriva da raiz aramaica tpt; nesse caso significaria `fornalha'.
Isso concorda bem com o fato que os sacrifícios a Moloque eram
realizados com fogo (cf. 2 Rs 23.10).} já há muito está preparada;
sim, está preparada para o rei; ele a fez profunda e larga; a sua
pira é de fogo, e tem muita lenha; o assopro do Senhor como torrente
de enxofre a acenderá.

\medskip

\lettrine{31}{}Ai dos que descem ao Egito a buscar socorro, e
se estribam em cavalos; e têm confiança em carros, porque são
muitos; e nos cavaleiros, porque são poderosíssimos; e não atentam
para o Santo de Israel, e não buscam ao Senhor. Todavia também
ele é sábio, e fará vir o mal, e não retirará as suas palavras; e
levantar-se-á contra a casa dos malfeitores, e contra a ajuda dos
que praticam a iniqüidade. Porque os egípcios são homens, e não
Deus; e os seus cavalos, carne, e não espírito; e quando o Senhor
estender a sua mão, tanto tropeçará o auxiliador, como cairá o
ajudado, e todos juntamente serão consumidos. Porque assim me
disse o Senhor: Como o leão e o leãozinho rugem sobre a sua presa,
ainda que se convoque contra ele uma multidão de pastores, não se
espantam das suas vozes, nem se abatem pela sua multidão, assim o
Senhor dos Exércitos descerá, para pelejar sobre o monte Sião, e
sobre o seu outeiro. Como as aves voam, assim o Senhor dos
Exércitos amparará a Jerusalém; ele a amparará, a livrará e,
passando, a salvará.

Convertei-vos, pois, àquele contra quem os filhos de Israel se
rebelaram tão profundamente. Porque naquele dia cada um lançará
fora os seus ídolos de prata, e os seus ídolos de ouro, que vos
fabricaram as vossas mãos para pecardes, e a Assíria cairá pela
espada, não de poderoso homem; e a espada, não de homem desprezível,
a consumirá; e fugirá perante a espada e os seus jovens serão
tributários. E de medo passará a sua rocha, e os seus príncipes
terão pavor da bandeira, diz o Senhor, cujo fogo está em Sião e a
sua fornalha em Jerusalém.

\medskip

\lettrine{32}{}Eis que reinará um rei com justiça, e dominarão
os príncipes segundo o juízo. E será aquele homem como um
esconderijo contra o vento, e um refúgio contra a tempestade, como
ribeiros de águas em lugares secos, e como a sombra de uma grande
rocha em terra sedenta. E os olhos dos que vêem não olharão para
trás; e os ouvidos dos que ouvem estarão atentos. E o coração
dos imprudentes entenderá o conhecimento; e a língua dos gagos
estará pronta para falar distintamente. Ao vil nunca mais se
chamará liberal; e do avarento nunca mais se dirá que é generoso.
Porque o vil fala obscenidade, e o seu coração pratica a
iniqüidade, para usar hipocrisia, e para proferir mentiras contra o
Senhor, para deixar vazia a alma do faminto, e fazer com que o
sedento venha a ter falta de bebida. Também todas as armas do
avarento são más; ele maquina invenções malignas, para destruir os
mansos com palavras falsas, mesmo quando o pobre chega a falar
retamente. Mas o liberal projeta coisas liberais, e pela
liberalidade está em pé.

Levantai-vos, mulheres, que estais sossegadas, e ouvi a minha voz;
e vós, filhas, que estais tão seguras, inclinai os ouvidos às minhas
palavras. Porque num ano e dias vireis a ser turbadas, ó
mulheres que estais tão seguras; porque a vindima se acabará, e a
colheita não virá. Tremei, mulheres que estais sossegadas, e
turbai-vos vós, que estais tão seguras; despi-vos, e ponde-vos nuas,
e cingi com saco os vossos lombos. Baterão nos peitos, pelos
campos desejáveis, e pelas vinhas frutíferas. Sobre a terra
do meu povo virão espinheiros e sarças, como também sobre todas as
casas onde há alegria, na cidade jubilosa. Porque os palácios
serão abandonados, a multidão da cidade cessará; e as fortificações
e as torres servirão de cavernas para sempre, para alegria dos
jumentos monteses, e para pasto dos rebanhos; até que se
derrame sobre nós o espírito lá do alto; então o deserto se tornará
em campo fértil, e o campo fértil será reputado por um bosque.
E o juízo habitará no deserto, e a justiça morará no campo
fértil. E o efeito da justiça será paz, e a operação da
justiça, repouso e segurança para sempre. E o meu povo
habitará em morada de paz, e em moradas bem seguras, e em lugares
quietos de descanso. Mas, descendo ao bosque, cairá saraiva e
a cidade será inteiramente abatida. Bem-aventurados vós os
que semeais junto a todas as águas; e deixais livres os pés do boi e
do jumento.

\medskip

\lettrine{33}{}Ai de ti, despojador, que não foste despojado, e
que procedes perfidamente contra os que não procederam perfidamente
contra ti! Acabando tu de despojar, serás despojado; e, acabando tu
de tratar perfidamente, perfidamente te tratarão. Senhor, tem
misericórdia de nós, por ti temos esperado; sê tu o nosso braço cada
manhã, como também a nossa salvação no tempo da tribulação. Ao
ruído do tumulto fugirão os povos; à tua exaltação as nações serão
dispersas. Então ajuntar-se-á o vosso despojo como se ajunta a
lagarta; como os gafanhotos saltam, assim ele saltará sobre eles.
O Senhor está exaltado, pois habita nas alturas; encheu a Sião
de juízo e justiça. E haverá estabilidade nos teus tempos,
abundância de salvação, sabedoria e conhecimento; e o temor do
Senhor será o seu tesouro. Eis que os seus embaixadores estão
clamando de fora; e os mensageiros de paz estão chorando
amargamente. As estradas estão desoladas, cessou o que passava
pela vereda, ele rompeu a aliança, desprezou as cidades, e já não
faz caso dos homens. A terra geme e pranteia, o Líbano se
envergonha e se murcha; Sarom se tornou como um deserto; e Basã e
Carmelo foram sacudidos. Agora, pois, me levantarei, diz o
Senhor; agora me erguerei. Agora serei exaltado. Concebestes
palha, dareis à luz restolho; e o vosso espírito vos devorará como o
fogo. E os povos serão como as queimas de cal; como espinhos
cortados arderão no fogo.

Ouvi, vós os que estais longe, o que tenho feito; e vós que
estais vizinhos, conhecei o meu poder. Os pecadores de Sião
se assombraram, o tremor surpreendeu os hipócritas. Quem dentre nós
habitará com o fogo consumidor? Quem dentre nós habitará com as
labaredas eternas? O que anda em justiça, e o que fala com
retidão; o que rejeita o ganho da opressão, o que sacode das suas
mãos todo o presente; o que tapa os seus ouvidos para não ouvir
falar de derramamento de sangue e fecha os seus olhos para não ver o
mal. Este habitará nas alturas; as fortalezas das rochas
serão o seu alto refúgio, o seu pão lhe será dado, as suas águas
serão certas. Os teus olhos verão o rei na sua formosura, e
verão a terra que está longe. O teu coração considerará o
assombro dizendo: Onde está o escrivão? Onde está o que pesou o
tributo? Onde está o que conta as torres? Não verás mais
aquele povo atrevido, povo de fala obscura, que não se pode
compreender e de língua tão estranha que não se pode entender.
Olha para Sião, a cidade das nossas solenidades; os teus
olhos verão a Jerusalém, habitação quieta, tenda que não será
removida, cujas estacas nunca serão arrancadas e das suas cordas
nenhuma se quebrará. Mas ali o glorioso Senhor será para nós
um lugar de rios e correntes largas; barco nenhum de remo passará
por ele, nem navio grande navegará por ele. Porque o Senhor é
o nosso Juiz; o Senhor é o nosso legislador; o Senhor é o nosso rei,
ele nos salvará. As tuas cordas se afrouxaram; não puderam
ter firme o seu mastro, e nem desfraldar a vela; então a presa de
abundantes despojos se repartirá; e até os coxos dividirão a presa.
E morador nenhum dirá: Enfermo estou; porque o povo que
habitar nela será absolvido da iniqüidade.

\medskip

\lettrine{34}{}Chegai-vos, nações, para ouvir, e vós povos,
escutai; ouça a terra, e a sua plenitude, o mundo, e tudo quanto
produz. Porque a indignação do Senhor está sobre todas as
nações, e o seu furor sobre todo o exército delas; ele as destruiu
totalmente, entregou-as à matança. E os seus mortos serão
arremessados e dos seus cadáveres subirá o seu mau cheiro; e os
montes se derreterão com o seu sangue. E todo o exército dos
céus se dissolverá, e os céus se enrolarão como um livro; e todo o
seu exército cairá, como cai a folha da vide e como cai o figo da
figueira. Porque a minha espada se embriagou nos céus; eis que
sobre Edom descerá, e sobre o povo do meu anátema para exercer
juízo. A espada do Senhor está cheia de sangue, está engordurada
da gordura do sangue de cordeiros e de bodes, da gordura dos rins de
carneiros; porque o Senhor tem sacrifício em Bozra, e grande matança
na terra de Edom. E os bois selvagens cairão com eles, e os
bezerros com os touros; e a sua terra embriagar-se-á de sangue até
se fartar, e o seu pó se engrossará com a gordura. Porque será o
dia da vingança do Senhor, ano de retribuições pela contenda de
Sião.

E os seus ribeiros se tornarão em pez\footnote{Resina que exsuda
dos pinheiros. Substância densa obtida pela destilação de alcatrão;
piche. Nome dado freqüentemente aos betumes naturais.}, e o seu pó
em enxofre, e a sua terra em pez ardente. Nem de noite nem de
dia se apagará; para sempre a sua fumaça subirá; de geração em
geração será assolada; pelos séculos dos séculos ninguém passará por
ela. Mas o pelicano e a coruja a possuirão, e o
bufo\footnote{Espécie de coruja.} e o corvo habitarão nela; e ele
estenderá sobre ela o cordel de confusão e nível de vaidade.
Eles chamarão ao reino os seus nobres, mas nenhum haverá; e
todos os seus príncipes não serão coisa alguma. E nos seus
palácios crescerão espinhos, urtigas e cardos nas suas fortalezas; e
será uma habitação de chacais, e sítio para avestruzes. As
feras do deserto se encontrarão com as feras da ilha, e o
sátiro\footnote{Dic. Bíblico: ``sã`ïr'', `cabeludo', `bode'.
Houaiss: Homem devasso, luxurioso. Relativo aos sátiros, povo
fabuloso da África, ou indivíduo desse povo.} clamará ao seu
companheiro; e os animais noturnos ali pousarão, e acharão lugar de
repouso para si. Ali se aninhará a coruja e porá os seus
ovos, e tirará os seus filhotes, e os recolherá debaixo da sua
sombra; também ali os abutres se ajuntarão uns com os outros.
Buscai no livro do Senhor, e lede; nenhuma destas coisas
faltará, ninguém faltará com a sua companheira; porque a minha boca
tem ordenado, e o seu espírito mesmo as tem ajuntado. Porque
ele mesmo lançou as sortes por elas, e sua mão lhas tens repartido
com o cordel; para sempre a possuirão, de geração em geração
habitarão nela.

\medskip

\lettrine{35}{}O deserto e o lugar solitário se alegrarão
disto; e o ermo exultará e florescerá como a rosa.
Abundantemente florescerá, e também jubilará de alegria e
cantará; a glória do Líbano se lhe deu, a excelência do Carmelo e
Sarom; eles verão a glória do Senhor, o esplendor do nosso Deus.
Fortalecei as mãos fracas, e firmai os joelhos trementes.
Dizei aos turbados de coração: Sede fortes, não temais; eis que
o vosso Deus virá com vingança, com recompensa de Deus; ele virá, e
vos salvará.

Então os olhos dos cegos serão abertos, e os ouvidos dos surdos se
abrirão. Então os coxos saltarão como cervos, e a língua dos
mudos cantará; porque águas arrebentarão no deserto e ribeiros no
ermo. E a terra seca se tornará em lagos, e a terra sedenta em
mananciais de águas; e nas habitações em que jaziam os chacais
haverá erva com canas e juncos. E ali haverá uma estrada, um
caminho, que se chamará o caminho santo; o imundo não passará por
ele, mas será para aqueles; os caminhantes, até mesmo os loucos, não
errarão.\footnote{KJ: And an highway shall be there, and a way, and
it shall be called The way of holiness; the unclean shall not pass
over it; but it shall be for those: the wayfaring men, though fools,
shall not err therein. RA: E ali haverá bom caminho, caminho que se
chamará o Caminho Santo; o imundo não passará por ele, pois será
somente para o seu povo; quem quer que por ele caminhe não errará,
nem mesmo o louco. RC: E ali haverá um alto caminho, um caminho que
se chamará O Caminho Santo; o imundo não passará por ele, mas será
para o povo de Deus; os caminhantes, até mesmo os loucos, não
errarão. RC-1969: E ali haverá um alto caminho, um caminho que se
chamará o caminho santo; o imundo não passará por ele, mas será para
aqueles: os caminhantes, até mesmo os loucos, não errarão.} Ali
não haverá leão, nem animal feroz subirá a ele, nem se achará nele;
porém só os remidos andarão por ele. E os resgatados do
Senhor voltarão; e virão a Sião com júbilo, e alegria eterna haverá
sobre as suas cabeças; gozo e alegria alcançarão, e deles fugirá a
tristeza e o gemido.

\medskip

\lettrine{36}{}E aconteceu no ano décimo quarto do rei
Ezequias, que Senaqueribe, rei da Assíria, subiu contra todas as
cidades fortificadas de Judá, e as tomou. Então o rei da Assíria
enviou a Rabsaqué, de Laquis a Jerusalém, ao rei Ezequias com um
grande exército, e ele parou junto ao aqueduto do açude superior,
junto ao caminho do campo do lavandeiro. Então saíram a ter com
ele Eliaquim, filho de Hilquias, o mordomo, e Sebna, o escrivão, e
Joá, filho de Asafe, o cronista. E Rabsaqué lhes disse: Ora
dizei a Ezequias: Assim diz o grande rei, o rei da Assíria: Que
confiança é esta, em que esperas? Bem posso eu dizer: Teu
conselho e poder para a guerra são apenas vãs palavras; em quem,
pois, agora confias, que contra mim te rebelas? Eis que confias
no Egito, aquele bordão de cana quebrada, o qual, se alguém se
apoiar nele lhe entrará pela mão, e a furará; assim é Faraó, rei do
Egito, para com todos os que nele confiam. Porém se me disseres:
No Senhor, nosso Deus, confiamos; porventura não é este aquele cujos
altos e altares Ezequias tirou, e disse a Judá e a Jerusalém:
Perante este altar adorareis? Ora, pois, empenha-te com meu
senhor, o rei da Assíria, e dar-te-ei dois mil cavalos, se tu
puderes dar cavaleiros para eles. Como, pois, poderás repelir a
um só capitão dos menores servos do meu senhor, quando confias no
Egito, por causa dos carros e cavaleiros? Agora, pois, subi
eu sem o Senhor contra esta terra, para destruí-la? O Senhor mesmo
me disse: Sobe contra esta terra, e destrói-a.

Então disseram Eliaquim, Sebna e Joá a Rabsaqué: Pedimos-te que
fales aos teus servos em siríaco, porque bem o entendemos, e não nos
fales em judaico, aos ouvidos do povo que está sobre o muro.
Rabsaqué, porém, disse: Porventura mandou-me o meu senhor ao
teu senhor e a ti, para dizer estas palavras e não antes aos homens
que estão assentados sobre o muro, para que comam convosco o seu
esterco, e bebam a sua urina? Rabsaqué, pois, se pôs em pé, e
clamou em alta voz em judaico, e disse: Ouvi as palavras do grande
rei, do rei da Assíria. Assim diz o rei: Não vos engane
Ezequias; porque não vos poderá livrar. Nem tampouco Ezequias
vos faça confiar no Senhor, dizendo: Infalivelmente nos livrará o
Senhor, e esta cidade não será entregue nas mãos do rei da Assíria.
Não deis ouvidos a Ezequias; porque assim diz o rei da
Assíria: Aliai-vos comigo, e saí a mim, e coma cada um da sua vide,
e da sua figueira, e beba cada um da água da sua cisterna;
até que eu venha, e vos leve para uma terra como a vossa;
terra de trigo e de mosto, terra de pão e de vinhas. Não vos
engane Ezequias, dizendo: O Senhor nos livrará. Porventura os deuses
das nações livraram cada um a sua terra das mãos do rei da Assíria?
Onde estão os deuses de Hamate e de Arpade? Onde estão os
deuses de Sefarvaim? Porventura livraram a Samaria da minha mão?
Quais dentre todos os deuses destes países livraram a sua
terra das minhas mãos, para que o Senhor livrasse a Jerusalém das
minhas mãos? Eles, porém, se calaram, e não lhe responderam
palavra alguma; porque havia mandado do rei, dizendo: Não lhe
respondereis. Então Eliaquim, filho de Hilquias, o mordomo, e
Sebna, o escrivão, e Joá, filho de Asafe, o cronista, vieram a
Ezequias, com as vestes rasgadas, e lhe fizeram saber as palavras de
Rabsaqué.

\medskip

\lettrine{37}{}E aconteceu que, tendo ouvido isso, o rei
Ezequias rasgou as suas vestes, e se cobriu de saco, e entrou na
casa do Senhor. Então enviou Eliaquim, o mordomo, e Sebna, o
escrivão, e os anciãos dos sacerdotes, cobertos de sacos, ao profeta
Isaías, filho de Amós. E disseram-lhe: Assim diz Ezequias: Este
dia é dia de angústia, e de vitupério\footnote{Vituperar: dirigir
vitupérios a; afrontar, insultar. Manifestar desaprovação ou censura
a; repreender. Tratar com desprezo a; menoscabar, menosprezar.}, e
de blasfêmias; porque chegados são os filhos ao parto, e força não
há para dá-los à luz.
 Porventura o Senhor teu Deus terá ouvido as palavras de Rabsaqué,
a quem o rei da Assíria, seu senhor, enviou para afrontar o Deus
vivo, e para vituperá-lo com as palavras que o Senhor teu Deus tem
ouvido; faze oração pelo remanescente que ficou. E os servos do
rei Ezequias foram ter com Isaías. E Isaías lhes disse: Assim
direis a vosso senhor: Assim diz o Senhor: Não temas à vista das
palavras que ouviste, com as quais os servos do rei da Assíria me
blasfemaram. Eis que porei nele um espírito, e ele ouvirá um
rumor, e voltará para a sua terra; e fá-lo-ei cair morto à espada na
sua terra.

Voltou, pois, Rabsaqué, e achou ao rei da Assíria pelejando contra
Libna; porque ouvira que já se havia retirado de Laquis. E ouviu
ele dizer que Tiraca, rei da Etiópia, tinha saído para lhe fazer
guerra. Assim que ouviu isto, enviou mensageiros a Ezequias,
dizendo: Assim falareis a Ezequias, rei de Judá, dizendo: Não
te engane o teu Deus, em quem confias, dizendo: Jerusalém não será
entregue na mão do rei da Assíria. Eis que já tens ouvido o
que fizeram os reis da Assíria a todas as terras, destruindo-as
totalmente; e escaparias tu? Porventura as livraram os deuses
das nações que meus pais destruíram: Gozã, e Harã, e Rezefe, e os
filhos de Éden, que estavam em Telassar? Onde está o rei de
Hamate, e o rei de Arpade, e o rei da cidade de Sefarvaim, Hena e
Iva? Recebendo, pois, Ezequias as cartas das mãos dos
mensageiros, e lendo-as, subiu à casa do Senhor; e Ezequias as
estendeu perante o Senhor. E orou Ezequias ao Senhor,
dizendo: Ó Senhor dos Exércitos, Deus de Israel, que habitas
entre os querubins; tu mesmo, só tu és Deus de todos os reinos da
terra; tu fizeste os céus e a terra. Inclina, ó Senhor, o teu
ouvido, e ouve; abre, Senhor, os teus olhos, e vê; e ouve todas as
palavras de Senaqueribe, as quais ele enviou para afrontar o Deus
vivo. Verdade é, Senhor, que os reis da Assíria assolaram
todas as nações e suas terras. E lançaram no fogo os seus
deuses; porque deuses não eram, senão obra de mãos de homens,
madeira e pedra; por isso os destruíram. Agora, pois, ó
Senhor nosso Deus, livra-nos da sua mão; e assim saberão todos os
reinos da terra, que só tu és o Senhor.

Então Isaías, filho de Amós, mandou dizer a Ezequias: Assim diz o
Senhor Deus de Israel: Quanto ao que pediste acerca de Senaqueribe,
rei da Assíria, esta é a palavra que o Senhor falou a
respeito dele: A virgem, a filha de Sião, te despreza, de ti zomba;
a filha de Jerusalém meneia a cabeça por detrás de ti. A quem
afrontaste e blasfemaste? E contra quem alçaste a voz, e ergueste os
teus olhos ao alto? Contra o Santo de Israel. Por meio de
teus servos afrontaste o Senhor, e disseste: Com a multidão dos meus
carros subi eu aos cumes dos montes, aos últimos recessos do Líbano;
e cortarei os seus altos cedros e as suas faias escolhidas, e
entrarei na altura do seu cume, ao bosque do seu campo fértil.
Eu cavei, e bebi as águas; e com as plantas de meus pés
sequei todos os rios dos lugares sitiados. Porventura não
ouviste que já há muito tempo eu fiz isto, e já desde os dias
antigos o tinha formado? Agora porém o fiz vir, para que tu fosses o
que destruísse as cidades fortificadas, e as reduzisse a montões de
ruínas. Por isso os seus moradores, dispondo de pouca força,
andaram atemorizados e envergonhados; tornaram-se como a erva do
campo, e a relva verde, e o feno dos telhados, e o trigo queimado
antes da seara. Porém eu conheço o teu assentar, e o teu
sair, e o teu entrar, e o teu furor contra mim. Por causa do
teu furor contra mim, e porque a tua arrogância subiu até aos meus
ouvidos, portanto porei o meu anzol no teu nariz e o meu freio nos
teus lábios, e te farei voltar pelo caminho por onde vieste.
E isto te será por sinal: Este ano se comerá o que
espontaneamente nascer, e no segundo ano o que daí proceder; porém
no terceiro ano semeai e segai, e plantai vinhas, e comei os frutos
delas. Porque o que escapou da casa de Judá, e restou,
tornará a lançar raízes para baixo, e dará fruto para cima.
Porque de Jerusalém sairá o restante, e do monte de Sião os
que escaparem; o zelo do Senhor dos Exércitos fará isto.
Portanto, assim diz o Senhor acerca do rei da Assíria: Não
entrará nesta cidade, nem lançará nela flecha alguma; tampouco virá
perante ela com escudo, ou levantará trincheira contra ela.
Pelo caminho por onde vier, por esse voltará; porém nesta
cidade não entrará, diz o Senhor. Porque eu ampararei esta
cidade, para livrá-la, por amor de mim e por amor do meu servo Davi.
Então saiu o anjo do Senhor, e feriu no arraial dos assírios
a cento e oitenta e cinco mil deles; e, quando se levantaram pela
manhã cedo, eis que todos estes eram corpos mortos. Assim
Senaqueribe, rei da Assíria, se retirou, e se foi, e voltou, e
habitou em Nínive. E sucedeu que, estando ele prostrado na
casa de Nisroque, seu deus, Adrameleque e Sarezer, seus filhos, o
feriram à espada; escaparam para a terra de Ararate; e Esar-Hadom,
seu filho, reinou em seu lugar.

\medskip

\lettrine{38}{}Naqueles dias Ezequias adoeceu de uma
enfermidade mortal; e veio a ele o profeta Isaías, filho de Amós, e
lhe disse: Assim diz o Senhor: Põe em ordem a tua casa, porque
morrerás, e não viverás. Então virou Ezequias o seu rosto para a
parede, e orou ao Senhor. E disse: Ah! Senhor, peço-te,
lembra-te agora\footnote{RC-1969: E disse: Ah! Senhor, lembra-te,
peço-te, de que andei diante de ti em verdade, e com coração
perfeito, e fiz o que era reto aos teus olhos. E chorou Ezequias
muitíssimo.} de que andei diante de ti em verdade, e com coração
perfeito, e fiz o que era reto aos teus olhos. E chorou Ezequias
muitíssimo. Então veio a palavra do Senhor a Isaías, dizendo:
Vai, e dize a Ezequias: Assim diz o Senhor, o Deus de Davi teu
pai: Ouvi a tua oração, e vi as tuas lágrimas; eis que acrescentarei
aos teus dias quinze anos. E livrar-te-ei das mãos do rei da
Assíria, a ti, e a esta cidade, e defenderei esta cidade. E isto
te será da parte do Senhor como sinal de que o Senhor cumprirá esta
palavra que falou. Eis que farei retroceder dez graus a sombra
lançada pelo sol declinante no relógio de Acaz. Assim retrocedeu o
sol os dez graus que já tinha declinado.

O escrito de Ezequias, rei de Judá, de quando adoeceu e sarou de
sua enfermidade: Eu disse: No cessar de meus dias ir-me-ei às
portas da sepultura; já estou privado do restante de meus anos.
Disse: Não verei ao Senhor, o Senhor na terra dos viventes;
jamais verei o homem com os moradores do mundo. Já o tempo da
minha vida se foi, e foi arrebatada de mim, como tenda de pastor;
cortei a minha vida como tecelão; ele me cortará do tear; desde a
manhã até à noite me acabarás. Esperei com paciência até à
madrugada; como um leão quebrou todos os meus ossos; desde a manhã
até à noite me acabarás. Como o grou\footnote{Design. comum
às aves da fam. dos gruídeos, encontradas em planícies e zonas
pantanosas de todo o mundo, com exceção da América do Sul e
Antártica; de grande porte, pernas e pescoço longos, cabeça
parcialmente nua, bico reto e plumagem com penas brancas, cinzas ou
marrons (Algumas spp. estão ameaçadas de extinção).}, ou a
andorinha, assim eu chilreava\footnote{Emitir chilros (os pássaros);
chalrear, chilrar. Derivação: sentido figurado: cantar ou falar
livre e animadamente, produzindo sons indistintos.}, e gemia como a
pomba; alçava os meus olhos ao alto; ó Senhor, ando oprimido, fica
por meu fiador. Que direi? Como me prometeu, assim o fez;
assim passarei mansamente por todos os meus anos, por causa da
amargura da minha alma. Senhor, por estas coisas se vive, e
em todas elas está a vida do meu espírito, portanto cura-me e
faze-me viver. Eis que foi para a minha paz que tive grande
amargura, mas a ti agradou livrar a minha alma da cova da corrupção;
porque lançaste para trás das tuas costas todos os meus pecados.
Porque não te louvará a sepultura, nem a morte te
glorificará; nem esperarão em tua verdade os que descem à cova.
O vivente, o vivente, esse te louvará, como eu hoje o faço; o
pai aos filhos fará notória a tua verdade. O Senhor veio
salvar-me; por isso, tangendo em meus instrumentos, nós o louvaremos
todos os dias de nossa vida na casa do Senhor. E dissera
Isaías: Tomem uma pasta de figos, e a ponham como emplastro sobre a
chaga; e sarará. Também dissera Ezequias: Qual será o sinal
de que hei de subir à casa do Senhor?

\medskip

\lettrine{39}{}Naquele tempo enviou Merodaque-Baladã, filho de
Baladã, rei de Babilônia, cartas e um presente a Ezequias, porque
tinha ouvido dizer que havia estado doente e que já tinha
convalescido. E Ezequias se alegrou com eles, e lhes mostrou a
casa do seu tesouro, a prata, e o ouro, e as especiarias, e os
melhores ungüentos, e toda a sua casa de armas, e tudo quanto se
achava nos seus tesouros; coisa nenhuma houve, nem em sua casa, nem
em todo o seu domínio, que Ezequias não lhes mostrasse. Então o
profeta Isaías veio ao rei Ezequias, e lhe disse: Que foi que
aqueles homens disseram, e de onde vieram a ti? E disse Ezequias: De
uma terra remota vieram a mim, de Babilônia. E disse ele: Que
foi que viram em tua casa? E disse Ezequias: Viram tudo quanto há em
minha casa; coisa nenhuma há nos meus tesouros que eu deixasse de
lhes mostrar.

Então disse Isaías a Ezequias: Ouve a palavra do Senhor dos
Exércitos: Eis que virão dias em que tudo quanto houver em tua
casa, e o que entesouraram teus pais até ao dia de hoje, será levado
para Babilônia; não ficará coisa alguma, disse o Senhor. E até
de teus filhos, que procederem de ti, e tu gerares, tomarão, para
que sejam eunucos no palácio do rei de Babilônia. Então disse
Ezequias a Isaías: Boa é a palavra do Senhor que disseste. Disse
mais: Pois haverá paz e verdade em meus dias.

\medskip

\lettrine{40}{}Consolai, consolai o meu povo, diz o vosso Deus.
Falai benignamente a Jerusalém, e bradai-lhe que já a sua
milícia é acabada, que a sua iniqüidade está expiada e que já
recebeu em dobro da mão do Senhor, por todos os seus pecados.

Voz do que clama no deserto: Preparai o caminho do Senhor;
endireitai no ermo vereda a nosso Deus. Todo o vale será
exaltado, e todo o monte e todo o outeiro será abatido; e o que é
torcido se endireitará, e o que é áspero se aplainará. E a
glória do Senhor se manifestará, e toda a carne juntamente a verá,
pois a boca do Senhor o disse. Uma voz diz: Clama; e alguém
disse: Que hei de clamar? Toda a carne é erva e toda a sua beleza
como a flor do campo. Seca-se a erva, e cai a flor, soprando
nela o Espírito do Senhor. Na verdade o povo é erva. Seca-se a
erva, e cai a flor, porém a palavra de nosso Deus subsiste
eternamente.

Tu, ó Sião, que anuncias boas novas, sobe a um monte alto. Tu, ó
Jerusalém, que anuncias boas novas, levanta a tua voz fortemente;
levanta-a, não temas, e dize às cidades de Judá: Eis aqui está o
vosso Deus. Eis que o Senhor Deus virá com poder e seu braço
dominará por ele; eis que o seu galardão está com ele, e o seu
salário diante da sua face. Como pastor apascentará o seu
rebanho; entre os seus braços recolherá os cordeirinhos, e os levará
no seu regaço; as que amamentam guiará suavemente.

Quem mediu na concha da sua mão as águas, e tomou a medida dos
céus aos palmos, e recolheu numa medida o pó da terra e pesou os
montes com peso e os outeiros em balanças? Quem guiou o
Espírito do Senhor, ou como seu conselheiro o ensinou? Com
quem tomou ele conselho, que lhe desse entendimento, e lhe ensinasse
o caminho do juízo, e lhe ensinasse conhecimento, e lhe mostrasse o
caminho do entendimento? Eis que as nações são consideradas
por ele como a gota de um balde, e como o pó miúdo das balanças; eis
que ele levanta as ilhas como a uma coisa pequeníssima. Nem
todo o Líbano basta para o fogo, nem os seus animais bastam para
holocaustos. Todas as nações são como nada perante ele; ele
as considera menos do que nada e como uma coisa vã.

A quem, pois, fareis semelhante a Deus, ou com que o comparareis?
O artífice funde a imagem, e o ourives a cobre de ouro, e
forja para ela cadeias de prata. O empobrecido, que não pode
oferecer tanto, escolhe madeira que não se apodrece; artífice sábio
busca, para gravar uma imagem que não se pode mover.\footnote{KJ: He
that is so impoverished that he hath no oblation chooseth a tree
that will not rot; he seeketh unto him a cunning workman to prepare
a graven image, that shall not be moved.} Porventura não
sabeis? Porventura não ouvis, ou desde o princípio não se vos
notificou, ou não atentastes para os fundamentos da terra?
Ele é o que está assentado sobre o círculo da terra, cujos
moradores são para ele como gafanhotos; é ele o que estende os céus
como cortina, e os desenrola como tenda, para neles habitar;
o que reduz a nada os príncipes, e torna em coisa vã os
juízes da terra. E mal se tem plantado, mal se tem semeado, e
mal se tem arraigado na terra o seu tronco, já se secam, quando ele
sopra sobre eles, e um tufão os leva como a pragana.
 A quem, pois, me fareis semelhante, para que eu lhe seja igual?
diz o Santo. Levantai ao alto os vossos olhos, e vede quem
criou estas coisas; foi aquele que faz sair o exército delas segundo
o seu número; ele as chama a todas pelos seus nomes; por causa da
grandeza das suas forças, e porquanto é forte em poder, nenhuma
delas faltará.

Por que dizes, ó Jacó, e tu falas, ó Israel: O meu caminho está
encoberto ao Senhor, e o meu juízo passa despercebido ao meu Deus?
Não sabes, não ouviste que o eterno Deus, o Senhor, o Criador
dos fins da terra, nem se cansa nem se fatiga? É inescrutável o seu
entendimento. Dá força ao cansado, e multiplica as forças ao
que não tem nenhum vigor. Os jovens se cansarão e se
fatigarão, e os moços certamente cairão; mas os que esperam
no Senhor renovarão as forças, subirão com asas como águias;
correrão, e não se cansarão; caminharão, e não se fatigarão.

\medskip

\lettrine{41}{}Calai-vos perante mim, ó ilhas, e os povos
renovem as forças; cheguem-se, e então falem; cheguemo-nos juntos a
juízo. Quem suscitou do oriente o justo e o chamou para o seu
pé? Quem deu as nações à sua face e o fez dominar sobre reis? Ele os
entregou à sua espada como o pó e como pragana arrebatada pelo vento
ao seu arco. Ele os persegue e passa em paz, por uma vereda por
onde os seus pés nunca tinham caminhado. Quem operou e fez isto,
chamando as gerações desde o princípio? Eu o Senhor, o primeiro, e
com os últimos eu mesmo. As ilhas o viram, e temeram; os fins da
terra tremeram; aproximaram-se, e vieram. Um ao outro ajudou, e
ao seu irmão disse: Esforça-te. E o artífice animou ao ourives,
e o que alisa com o martelo ao que bate na bigorna, dizendo da coisa
soldada: Boa é. Então com pregos a firma, para que não venha a
mover-se. Porém tu, ó Israel, servo meu, tu Jacó, a quem elegi
descendência de Abraão, meu amigo; tu a quem tomei desde os fins
da terra, e te chamei dentre os seus mais excelentes, e te disse: Tu
és o meu servo, a ti escolhi e nunca te rejeitei.

Não temas, porque eu sou contigo; não te assombres, porque eu sou
teu Deus; eu te fortaleço, e te ajudo, e te sustento com a destra da
minha justiça. Eis que, envergonhados e confundidos serão
todos os que se indignaram contra ti; tornar-se-ão em nada, e os que
contenderem contigo, perecerão. Buscá-los-ás, porém não os
acharás; os que pelejarem contigo, tornar-se-ão em nada, e como
coisa que não é nada, os que guerrearem contigo. Porque eu, o
Senhor teu Deus, te tomo pela tua mão direita; e te digo: Não temas,
eu te ajudo. Não temas, tu verme de Jacó, povozinho de
Israel; eu te ajudo, diz o Senhor, e o teu redentor é o Santo de
Israel. Eis que farei de ti um trilho novo, que tem dentes
agudos; os montes trilharás e moerás; e os outeiros tornarás como a
pragana. Tu os padejarás e o vento os levará, e o redemoinho
os espalhará; mas tu te alegrarás no Senhor e te gloriarás no Santo
de Israel. Os aflitos e necessitados buscam águas, e não há,
e a sua língua se seca de sede; eu o Senhor os ouvirei, eu, o Deus
de Israel não os desampararei. Abrirei rios em lugares altos,
e fontes no meio dos vales; tornarei o deserto em lagos de águas, e
a terra seca em mananciais de água. Plantarei no deserto o
cedro, a acácia, e a murta\footnote{Arbusto ou árvore pequena
(Myrtus communis), com raízes e casca us. para extração de tanino,
madeira de qualidade, folhas ricas em óleo us. em perfumes, assim
como as flores brancas, aromáticas, e as bagas carnosas. [De origem
incerta, prov. mediterrânea, cultivada como ornamental e por usos
medicinais, há milênios está associada a rituais e cerimônias, e as
flores são muito us. em grinaldas de noivas.]}, e a oliveira; porei
no ermo juntamente a faia, o pinheiro e o álamo. Para que
todos vejam, e saibam, e considerem, e juntamente entendam que a mão
do Senhor fez isto, e o Santo de Israel o criou.

Apresentai a vossa demanda, diz o Senhor; trazei as vossas firmes
razões, diz o Rei de Jacó. Tragam e anunciem-nos as coisas
que hão de acontecer; anunciai-nos as coisas passadas, para que
atentemos para elas, e saibamos o fim delas; ou fazei-nos ouvir as
coisas futuras. Anunciai-nos as coisas que ainda hão de vir,
para que saibamos que sois deuses; ou fazei bem, ou fazei mal, para
que nos assombremos, e juntamente o vejamos. Eis que sois
menos do que nada e a vossa obra é menos do que nada; abominação é
quem vos escolhe. Suscitei a um do norte, e ele há de vir;
desde o nascimento do sol invocará o meu nome; e virá sobre os
príncipes, como sobre o lodo e, como o oleiro pisa o barro, os
pisará. Quem anunciou isto desde o princípio, para que o
possamos saber, ou desde antes, para que digamos: Justo é? Porém não
há quem anuncie, nem tampouco quem manifeste, nem tampouco quem ouça
as vossas palavras. Eu sou o que primeiro direi a Sião: Eis
que ali estão; e a Jerusalém darei um anunciador de boas novas.
E quando olhei, não havia ninguém; nem mesmo entre estes,
conselheiro algum havia a quem perguntasse ou que me respondesse
palavra. Eis que todos são vaidade; as suas obras não são
coisa alguma; as suas imagens de fundição são vento e confusão.

\medskip

\lettrine{42}{}Eis aqui o meu servo, a quem sustenho, o meu
eleito, em quem se apraz a minha alma; pus o meu espírito sobre ele;
ele trará justiça aos gentios. Não clamará, não se exaltará, nem
fará ouvir a sua voz na praça. A cana trilhada não quebrará, nem
apagará o pavio que fumega; com verdade trará justiça. Não
faltará, nem será quebrantado, até que ponha na terra a justiça; e
as ilhas aguardarão a sua lei.

Assim diz Deus, o Senhor, que criou os céus, e os estendeu, e
espraiou a terra, e a tudo quanto produz; que dá a respiração ao
povo que nela está, e o espírito aos que andam nela. Eu, o
Senhor, te chamei em justiça, e te tomarei pela mão, e te guardarei,
e te darei por aliança do povo, e para luz dos gentios. Para
abrir os olhos dos cegos, para tirar da prisão os presos, e do
cárcere os que jazem em trevas. Eu sou o Senhor; este é o meu
nome; a minha glória, pois, a outrem não darei, nem o meu louvor às
imagens de escultura. Eis que as primeiras coisas já se
cumpriram, e as novas eu vos anuncio, e, antes que venham à luz,
vo-las faço ouvir. Cantai ao Senhor um cântico novo, e o seu
louvor desde a extremidade da terra; vós os que navegais pelo mar, e
tudo quanto há nele; vós, ilhas, e seus habitantes. Alcem a
voz o deserto e as suas cidades, com as aldeias que Quedar habita;
exultem os que habitam nas rochas, e clamem do cume dos montes.
Dêem a glória ao Senhor, e anunciem o seu louvor nas ilhas.

O Senhor sairá como poderoso, como homem de guerra despertará o
zelo; clamará, e fará grande ruído, e prevalecerá contra seus
inimigos. Por muito tempo me calei; estive em silêncio, e me
contive; mas agora darei gritos como a que está de parto, e a todos
os assolarei e juntamente devorarei. Os montes e outeiros
tornarei em deserto, e toda a sua erva farei secar, e tornarei os
rios em ilhas, e as lagoas secarei. E guiarei os cegos pelo
caminho que nunca conheceram, fá-los-ei caminhar pelas veredas que
não conheceram; tornarei as trevas em luz perante eles, e as coisas
tortas farei direitas. Estas coisas lhes farei, e nunca os
desampararei. Tornarão atrás e confundir-se-ão de vergonha os
que confiam em imagens de escultura, e dizem às imagens de fundição:
Vós sois nossos deuses.

Surdos, ouvi, e vós, cegos, olhai, para que possais ver.
Quem é cego, senão o meu servo, ou surdo como o meu
mensageiro, a quem envio? E quem é cego como o que é perfeito, e
cego como o servo do Senhor? Tu vês muitas coisas, mas não as
guardas; ainda que tenhas os ouvidos abertos, nada ouves. O
Senhor se agradava dele por amor da sua justiça; engrandeceu-o pela
lei, e o fez glorioso. Mas este é um povo roubado e saqueado;
todos estão enlaçados em cavernas, e escondidos em cárceres; são
postos por presa, e ninguém há que os livre; por despojo, e ninguém
diz: Restitui. Quem há entre vós que ouça isto, que atenda e
ouça o que há de ser depois? Quem entregou a Jacó por
despojo, e a Israel aos roubadores? Porventura não foi o Senhor,
aquele contra quem pecamos, e nos caminhos do qual não queriam
andar, não dando ouvidos à sua lei? Por isso derramou sobre
eles a indignação da sua ira, e a força da guerra, e lhes pôs
labaredas em redor; porém nisso não atentaram; e os queimou, mas não
puseram nisso o coração.

\medskip

\lettrine{43}{}Mas agora, assim diz o Senhor que te criou, ó
Jacó, e que te formou, ó Israel: Não temas, porque eu te remi;
chamei-te pelo teu nome, tu és meu. Quando passares pelas águas
estarei contigo, e quando pelos rios, eles não te submergirão;
quando passares pelo fogo, não te queimarás, nem a chama arderá em
ti. Porque eu sou o Senhor teu Deus, o Santo de Israel, o teu
Salvador; dei o Egito por teu resgate, a Etiópia e a Seba em teu
lugar. Visto que foste precioso aos meus olhos, também foste
honrado, e eu te amei, assim dei os homens por ti, e os povos pela
tua vida. Não temas, pois, porque estou contigo; trarei a tua
descendência desde o oriente, e te ajuntarei desde o ocidente.
Direi ao norte: Dá; e ao sul: Não retenhas; trazei meus filhos
de longe e minhas filhas das extremidades da terra. A todos os
que são chamados pelo meu nome e os que criei para a minha glória,
os formei, e também os fiz.

Trazei o povo cego, que tem olhos; e os surdos, que têm ouvidos.
Todas as nações se congreguem, e os povos se reúnam; quem dentre
eles pode anunciar isto, e fazer-nos ouvir as coisas antigas?
Apresentem as suas testemunhas, para que se justifiquem, e se ouça,
e se diga: Verdade é. Vós sois as minhas testemunhas, diz o
Senhor, e meu servo, a quem escolhi; para que o saibais, e me
creiais, e entendais que eu sou o mesmo, e que antes de mim deus
nenhum se formou, e depois de mim nenhum haverá. Eu, eu sou o
Senhor, e fora de mim não há Salvador. Eu anunciei, e eu
salvei, e eu o fiz ouvir, e deus estranho não houve entre vós, pois
vós sois as minhas testemunhas, diz o Senhor; eu sou Deus.
Ainda antes que houvesse dia, eu sou; e ninguém há que possa
fazer escapar das minhas mãos; agindo eu, quem o impedirá?

Assim diz o Senhor, vosso Redentor, o Santo de Israel: Por amor
de vós enviei a Babilônia, e a todos fiz descer como fugitivos, os
caldeus, nos navios com que se vangloriavam. Eu sou o Senhor,
vosso Santo, o Criador de Israel, vosso Rei. Assim diz o
Senhor, o que preparou no mar um caminho, e nas águas impetuosas uma
vereda; o que fez sair o carro e o cavalo, o exército e a
força; eles juntamente se deitaram, e nunca se levantarão; estão
extintos; como um pavio se apagaram. Não vos lembreis das
coisas passadas, nem considereis as antigas. Eis que faço uma
coisa nova, agora sairá à luz; porventura não a percebeis? Eis que
porei um caminho no deserto, e rios no ermo. Os animais do
campo me honrarão, os chacais, e os avestruzes; porque porei águas
no deserto, e rios no ermo, para dar de beber ao meu povo, ao meu
eleito.
 A esse povo que formei para mim; o meu louvor relatarão.

Contudo tu não me invocaste a mim, ó Jacó, mas te cansaste de
mim, ó Israel. Não me trouxeste o gado miúdo dos teus
holocaustos, nem me honraste com os teus sacrifícios; não te fiz
servir com ofertas, nem te fatiguei com incenso. Não me
compraste por dinheiro cana aromática, nem com a gordura dos teus
sacrifícios me satisfizeste, mas me deste trabalho com os teus
pecados, e me cansaste com as tuas iniqüidades. Eu, eu mesmo,
sou o que apago as tuas transgressões por amor de mim, e dos teus
pecados não me lembro. Faze-me lembrar; entremos juntos em
juízo; conta tu as tuas razões, para que te possas justificar.
Teu primeiro pai pecou, e os teus intérpretes prevaricaram
contra mim. Por isso profanei os príncipes do santuário; e
entreguei Jacó ao anátema, e Israel ao opróbrio.

\medskip

\lettrine{44}{}Agora, pois, ouve, ó Jacó, servo meu, e tu, ó
Israel, a quem escolhi. Assim diz o Senhor que te criou e te
formou desde o ventre, e que te ajudará: Não temas, ó Jacó, servo
meu, e tu, Jesurum, a quem escolhi. Porque derramarei água sobre
o sedento, e rios sobre a terra seca; derramarei o meu Espírito
sobre a tua posteridade, e a minha bênção sobre os teus
descendentes. E brotarão como a erva, como salgueiros junto aos
ribeiros das águas. Este dirá: Eu sou do Senhor; e aquele se
chamará do nome de Jacó; e aquele outro escreverá com a sua mão ao
Senhor, e por sobrenome tomará o nome de Israel. Assim diz o
Senhor, Rei de Israel, e seu Redentor, o Senhor dos Exércitos: Eu
sou o primeiro, e eu sou o último, e fora de mim não há Deus. E
quem proclamará como eu, e anunciará isto, e o porá em ordem perante
mim, desde que ordenei um povo eterno? E anuncie-lhes as coisas
vindouras, e as que ainda hão de vir. Não vos assombreis, nem
temais; porventura desde então não vo-lo fiz ouvir, e não vo-lo
anunciei? Porque vós sois as minhas testemunhas. Porventura há outro
Deus fora de mim? Não, não há outra Rocha que eu conheça.

Todos os artífices de imagens de escultura são vaidade, e as suas
coisas mais desejáveis são de nenhum préstimo; e suas próprias
testemunhas, nada vêem nem entendem para que sejam envergonhados.
Quem forma um deus, e funde uma imagem de escultura, que é de
nenhum préstimo? Eis que todos os seus companheiros ficarão
confundidos, pois os mesmos artífices não passam de homens;
ajuntem-se todos, e levantem-se; assombrar-se-ão, e serão juntamente
confundidos. O ferreiro, com a tenaz, trabalha nas brasas, e
o forma com martelos, e o lavra com a força do seu braço; ele tem
fome e a sua força enfraquece, e não bebe água, e desfalece.
O carpinteiro estende a régua, desenha-o com uma linha,
aplaina-o com a plaina, e traça-o com o compasso; e o faz à
semelhança de um homem, segundo a forma de um homem, para ficar em
casa. Quando corta para si cedros, toma, também, o cipreste e
o carvalho; assim escolhe dentre as árvores do bosque; planta um
olmeiro, e a chuva o faz crescer. Então serve ao homem para
queimar; e toma deles, e se aquenta, e os acende, e coze o pão;
também faz um deus, e se prostra diante dele; também fabrica uma
imagem de escultura, e ajoelha-se diante dela. Metade dele
queima no fogo, com a outra metade prepara a carne para comer,
assa-a e farta-se dela; também se aquenta, e diz: Ora já me
aquentei, já vi o fogo. Então do resto faz um deus, uma
imagem de escultura; ajoelha-se diante dela, e se inclina, e
roga-lhe, e diz: Livra-me, porquanto tu és o meu deus. Nada
sabem, nem entendem; porque tapou os olhos para que não vejam, e os
seus corações para que não entendam. E nenhum deles cai em
si, e já não têm conhecimento nem entendimento para dizer: Metade
queimei no fogo, e cozi pão sobre as suas brasas, assei sobre elas
carne, e a comi; e faria eu do resto uma abominação? Ajoelhar-me-ei
ao que saiu de uma árvore? Apascenta-se de cinza; o seu
coração enganado o desviou, de maneira que já não pode livrar a sua
alma, nem dizer: Porventura não há uma mentira na minha mão direita?

Lembra-te destas coisas, ó Jacó, e Israel, porquanto és meu
servo; eu te formei, meu servo és, ó Israel, não me esquecerei de
ti. Apaguei as tuas transgressões como a névoa, e os teus
pecados como a nuvem; torna-te para mim, porque eu te remi.
Cantai alegres, vós, ó céus, porque o Senhor o fez; exultai
vós, as partes mais baixas da terra; vós, montes, retumbai com
júbilo; também vós, bosques, e todas as suas árvores; porque o
Senhor remiu a Jacó, e glorificou-se em Israel. Assim diz o
Senhor, teu redentor, e que te formou desde o ventre: Eu sou o
Senhor que faço tudo, que sozinho estendo os céus, e espraio a terra
por mim mesmo; que desfaço os sinais dos inventores de
mentiras, e enlouqueço os adivinhos; que faço tornar atrás os
sábios, e converto em loucura o conhecimento deles; que
confirmo a palavra do seu servo, e cumpro o conselho dos seus
mensageiros; que digo a Jerusalém: Tu serás habitada, e às cidades
de Judá: Sereis edificadas, e eu levantarei as suas ruínas;
que digo à profundeza: Seca-te, e eu secarei os teus rios.
Que digo de Ciro: É meu pastor, e cumprirá tudo o que me
apraz, dizendo também a Jerusalém: Tu serás edificada; e ao templo:
Tu serás fundado.

\medskip

\lettrine{45}{}Assim diz o Senhor ao seu ungido, a Ciro, a quem
tomo pela mão direita, para abater as nações diante de sua face, e
descingir os lombos dos reis, para abrir diante dele as portas, e as
portas não se fecharão. Eu irei adiante de ti, e endireitarei os
caminhos tortuosos; quebrarei as portas de bronze, e despedaçarei os
ferrolhos de ferro. Dar-te-ei os tesouros escondidos, e as
riquezas encobertas, para que saibas que eu sou o Senhor, o Deus de
Israel, que te chama pelo teu nome. Por amor de meu servo Jacó,
e de Israel, meu eleito, eu te chamei pelo teu nome, pus o teu
sobrenome, ainda que não me conhecesses.

Eu sou o Senhor, e não há outro; fora de mim não há Deus; eu te
cingirei, ainda que tu não me conheças; para que se saiba desde
o nascente do sol, e desde o poente, que fora de mim não há outro;
eu sou o Senhor, e não há outro. Eu formo a luz, e crio as
trevas; eu faço a paz, e crio o mal; eu, o Senhor, faço todas estas
coisas. Destilai, ó céus, dessas alturas, e as nuvens chovam
justiça; abra-se a terra, e produza a salvação, e ao mesmo tempo
frutifique a justiça; eu, o Senhor, as criei. Ai daquele que
contende com o seu Criador! o caco entre outros cacos de barro!
Porventura dirá o barro ao que o formou: Que fazes? ou a tua obra:
Não tens mãos? Ai daquele que diz ao pai: Que é o que geras?
E à mulher: Que dás tu à luz?

Assim diz o Senhor, o Santo de Israel, aquele que o formou:
Perguntai-me as coisas futuras; demandai-me acerca de meus filhos, e
acerca da obra das minhas mãos. Eu fiz a terra, e criei nela
o homem; eu o fiz; as minhas mãos estenderam os céus, e a todos os
seus exércitos dei as minhas ordens. Eu o despertei em
justiça, e todos os seus caminhos endireitarei; ele edificará a
minha cidade, e soltará os meus cativos, não por preço nem por
presente, diz o Senhor dos Exércitos. Assim diz o Senhor: O
trabalho do Egito, e o comércio dos etíopes e dos sabeus, homens de
alta estatura, passarão para ti, e serão teus; irão atrás de ti,
virão em grilhões, e diante de ti se prostrarão; far-te-ão as suas
súplicas, dizendo: Deveras Deus está em ti, e não há nenhum outro
deus. Verdadeiramente tu és o Deus que te ocultas, o Deus de
Israel, o Salvador. Envergonhar-se-ão, e também se
confundirão todos; cairão juntamente na afronta os que fabricam
imagens. Porém Israel é salvo pelo Senhor, com uma eterna
salvação; por isso não sereis envergonhados nem confundidos em toda
a eternidade. Porque assim diz o Senhor que tem criado os
céus, o Deus que formou a terra, e a fez; ele a confirmou, não a
criou vazia, mas a formou para que fosse habitada: Eu sou o Senhor e
não há outro. Não falei em segredo, nem em lugar algum escuro
da terra; não disse à descendência de Jacó: Buscai-me em vão; eu sou
o Senhor, que falo a justiça, e anuncio coisas retas.

Congregai-vos, e vinde; chegai-vos juntos, os que escapastes das
nações; nada sabem os que conduzem em procissão as suas imagens de
escultura, feitas de madeira, e rogam a um deus que não pode salvar.
Anunciai, e chegai-vos, e tomai conselho todos juntos; quem
fez ouvir isto desde a antiguidade? Quem desde então o anunciou?
Porventura não sou eu, o Senhor? Pois não há outro Deus senão eu;
Deus justo e Salvador não há além de mim. Olhai para mim, e
sereis salvos, vós, todos os termos da terra; porque eu sou Deus, e
não há outro. Por mim mesmo tenho jurado, já saiu da minha
boca a palavra de justiça, e não tornará atrás; que diante de mim se
dobrará todo o joelho, e por mim jurará toda a língua. De mim
se dirá: Deveras no Senhor há justiça e força; até ele virão, mas
serão envergonhados todos os que se indignarem contra ele.
Mas no Senhor será justificada, e se gloriará toda a
descendência de Israel.

\medskip

\lettrine{46}{}Bel está abatido, Nebo se encurvou, os seus
ídolos são postos sobre os animais e sobre as feras; as cargas dos
vossos fardos são canseiras para as feras já cansadas.
Juntamente se encurvaram e se abateram; não puderam livrar-se da
carga, mas a sua alma entrou em cativeiro. Ouvi-me, ó casa de
Jacó, e todo o restante da casa de Israel; vós a quem trouxe nos
braços desde o ventre, e sois levados desde a madre. E até à
velhice eu serei o mesmo, e ainda até às cãs eu vos carregarei; eu
vos fiz, e eu vos levarei, e eu vos trarei, e vos livrarei.

A quem me assemelhareis, e com quem me igualareis, e me
comparareis, para que sejamos semelhantes? Gastam o ouro da
bolsa, e pesam a prata nas balanças; assalariam o ourives, e ele faz
um deus, e diante dele se prostram e se inclinam. Sobre os
ombros o tomam, o levam, e o põem no seu lugar; ali fica em pé, do
seu lugar não se move; e, se alguém clama a ele, resposta nenhuma
dá, nem livra alguém da sua tribulação. Lembrai-vos disto, e
considerai; trazei-o à memória, ó prevaricadores. Lembrai-vos
das coisas passadas desde a antiguidade; que eu sou Deus, e não há
outro Deus, não há outro semelhante a mim. Que anuncio o fim
desde o princípio, e desde a antiguidade as coisas que ainda não
sucederam; que digo: O meu conselho será firme, e farei toda a minha
vontade. Que chamo a ave de rapina desde o oriente, e de uma
terra remota o homem do meu conselho; porque assim o disse, e assim
o farei vir; eu o formei, e também o farei. Ouvi-me, ó duros
de coração, os que estais longe da justiça. Faço chegar a
minha justiça, e não estará ao longe, e a minha salvação não
tardará; mas estabelecerei em Sião a salvação, e em Israel a minha
glória.

\medskip

\lettrine{47}{}Desce, e assenta-te no pó, ó virgem filha de
Babilônia; assenta-te no chão; já não há trono, ó filha dos caldeus,
porque nunca mais serás chamada a tenra nem a delicada. Toma a
mó, e mói a farinha; remove o teu véu, descalça os pés, descobre as
pernas e passa os rios. A tua vergonha se descobrirá, e ver-se-á
o teu opróbrio; tomarei vingança, e não pouparei a homem algum.
O nosso redentor cujo nome é o Senhor dos Exércitos, é o Santo
de Israel. Assenta-te calada, e entra nas trevas, ó filha dos
caldeus, porque nunca mais serás chamada senhora de reinos.
Muito me agastei contra o meu povo, profanei a minha herança, e
os entreguei na tua mão; porém não usaste com eles de misericórdia,
e até sobre os velhos fizeste muito pesado o teu jugo.

E disseste: Eu serei senhora para sempre; até agora não te
importaste com estas coisas, nem te lembraste do fim delas.
Agora, pois, ouve isto, tu que és dada a prazeres, que habitas
tão segura, que dizes no teu coração: Eu o sou, e fora de mim não há
outra; não ficarei viúva, nem conhecerei a perda de filhos.
Porém ambas estas coisas virão sobre ti num momento, no mesmo
dia, perda de filhos e viuvez; em toda a sua plenitude virão sobre
ti, por causa da multidão das tuas feitiçarias, e da grande
abundância dos teus muitos encantamentos. Porque confiaste na
tua maldade e disseste: Ninguém me pode ver; a tua sabedoria e o teu
conhecimento, isso te fez desviar, e disseste no teu coração: Eu
sou, e fora de mim não há outra. Portanto sobre ti virá o
mal, sem que saibas a sua origem, e tal destruição cairá sobre ti,
sem que a possas evitar; e virá sobre ti de repente desolação que
não poderás conhecer. Deixa-te estar com os teus
encantamentos, e com a multidão das tuas feitiçarias, em que
trabalhaste desde a tua mocidade, a ver se podes tirar proveito, ou
se porventura te podes fortalecer. Cansaste-te na multidão
dos teus conselhos; levantem-se pois agora os agoureiros dos céus,
os que contemplavam os astros, os prognosticadores das luas novas, e
salvem-te do que há de vir sobre ti. Eis que serão como a
pragana, o fogo os queimará; não poderão salvar a sua vida do poder
das chamas; não haverá brasas, para se aquentar, nem fogo para se
assentar junto dele. Assim serão para contigo aqueles com
quem trabalhaste, os teus negociantes desde a tua mocidade; cada
qual irá vagueando pelo seu caminho; ninguém te salvará.

\medskip

\lettrine{48}{}Ouvi isto, casa de Jacó, que vos chamais do nome
de Israel, e saístes das águas de Judá, que jurais pelo nome do
Senhor, e fazeis menção do Deus de Israel, mas não em verdade nem em
justiça. E até da santa cidade tomam o nome e se firmam sobre o
Deus de Israel; o Senhor dos Exércitos é o seu nome. As
primeiras coisas desde a antiguidade as anunciei; da minha boca
saíram, e eu as fiz ouvir; apressuradamente as fiz, e aconteceram.
Porque eu sabia que eras duro, e a tua cerviz um nervo de ferro,
e a tua testa de bronze. Por isso te anunciei desde então, e te
fiz ouvir antes que acontecesse, para que não dissesses: O meu ídolo
fez estas coisas, e a minha imagem de escultura, e a minha imagem de
fundição as mandou. Já o tens ouvido; olha bem para tudo isto;
porventura não o anunciareis? Desde agora te faço ouvir coisas novas
e ocultas, e que nunca conheceste. Agora são criadas, e não de
há muito, e antes deste dia não as ouviste, para que porventura não
digas: Eis que eu já as sabia. Nem tu as ouviste, nem tu as
conheceste, nem tampouco há muito foi aberto o teu ouvido, porque eu
sabia que procederias muito perfidamente, e que eras chamado
transgressor desde o ventre.

Por amor do meu nome retardarei a minha ira, e por amor do meu
louvor me refrearei para contigo, para que te não venha a cortar.
Eis que já te purifiquei, mas não como a prata; escolhi-te na
fornalha da aflição. Por amor de mim, por amor de mim o
farei, porque, como seria profanado o meu nome? E a minha glória não
a darei a outrem. Dá-me ouvidos, ó Jacó, e tu, ó Israel, a
quem chamei; eu sou o mesmo, eu o primeiro, eu também o último.
Também a minha mão fundou a terra, e a minha destra mediu os
céus a palmos; eu os chamarei, e aparecerão juntos.
Ajuntai-vos todos vós, e ouvi: Quem, dentre eles, tem
anunciado estas coisas? O Senhor o amou, e executará a sua vontade
contra Babilônia, e o seu braço será contra os caldeus. Eu,
eu o tenho falado; também já o chamei, e o trarei, e farei próspero
o seu caminho.

Chegai-vos a mim, ouvi isto: Não falei em segredo desde o
princípio; desde o tempo em que aquilo se fez eu estava ali, e agora
o Senhor Deus me enviou a mim, e o seu Espírito. Assim diz o
Senhor, o teu Redentor, o Santo de Israel: Eu sou o Senhor teu Deus,
que te ensina o que é útil, e te guia pelo caminho em que deves
andar. Ah! se tivesses dado ouvidos aos meus mandamentos,
então seria a tua paz como o rio, e a tua justiça como as ondas do
mar! Também a tua descendência seria como a areia, e os que
procedem das tuas entranhas como os seus grãos; o seu nome nunca
seria cortado nem destruído de diante de mim. Saí de
Babilônia, fugi de entre os caldeus. E anunciai com voz de júbilo,
fazei ouvir isso, e levai-o até ao fim da terra; dizei: O Senhor
remiu a seu servo Jacó. E não tinham sede, quando os levava
pelos desertos; fez-lhes correr água da rocha; fendeu a rocha, e as
águas correram. Mas os ímpios não têm paz, diz o Senhor.

\medskip

\lettrine{49}{}Ouvi-me, ilhas, e escutai vós, povos de longe: O
Senhor me chamou desde o ventre, desde as entranhas de minha mãe fez
menção do meu nome. E fez a minha boca como uma espada aguda,
com a sombra da sua mão me cobriu; e me pôs como uma flecha limpa, e
me escondeu na sua aljava; e me disse: Tu és meu servo; és
Israel, aquele por quem hei de ser glorificado. Porém eu disse:
Debalde tenho trabalhado, inútil e vãmente gastei as minhas forças;
todavia o meu direito está perante o Senhor, e o meu galardão
perante o meu Deus. E agora diz o Senhor, que me formou desde o
ventre para ser seu servo, para que torne a trazer Jacó; porém
Israel não se deixará ajuntar; contudo aos olhos do Senhor serei
glorificado, e o meu Deus será a minha força. Disse mais: Pouco
é que sejas o meu servo, para restaurares as tribos de Jacó, e
tornares a trazer os preservados de Israel; também te dei para luz
dos gentios, para seres a minha salvação até à extremidade da terra.

Assim diz o Senhor, o Redentor de Israel, o seu Santo, à alma
desprezada, ao que a nação abomina, ao servo dos que dominam: Os
reis o verão, e se levantarão, como também os príncipes, e eles
diante de ti se inclinarão, por amor do Senhor, que é fiel, e do
Santo de Israel, que te escolheu. Assim diz o Senhor: No tempo
aceitável te ouvi e no dia da salvação te ajudei, e te guardarei, e
te darei por aliança do povo, para restaurares a terra, e dar-lhes
em herança as herdades assoladas; para dizeres aos presos: Saí;
e aos que estão em trevas: Aparecei. Eles pastarão nos caminhos, e
em todos os lugares altos haverá o seu pasto. Nunca terão
fome, nem sede, nem o calor, nem o sol os afligirá; porque o que se
compadece deles os guiará e os levará mansamente aos mananciais das
águas. E farei de todos os meus montes um caminho; e as
minhas estradas serão levantadas. Eis que estes virão de
longe, e eis que aqueles do norte, e do ocidente, e aqueles outros
da terra de Sinim.

Exultai, ó céus, e alegra-te, ó terra, e vós, montes, estalai com
júbilo, porque o Senhor consolou o seu povo, e dos seus aflitos se
compadecerá. Porém Sião diz: Já me desamparou o Senhor, e o
meu Senhor se esqueceu de mim. Porventura pode uma mulher
esquecer-se tanto de seu filho que cria, que não se compadeça dele,
do filho do seu ventre? Mas ainda que esta se esquecesse dele,
contudo eu não me esquecerei de ti. Eis que nas palmas das
minhas mãos eu te gravei; os teus muros estão continuamente diante
de mim. Os teus filhos pressurosamente virão, mas os teus
destruidores e os teus assoladores sairão do meio de ti.

Levanta os teus olhos ao redor, e olha; todos estes que se
ajuntam vêm a ti; vivo eu, diz o Senhor, que de todos estes te
vestirás, como de um ornamento, e te cingirás deles como noiva.
Porque nos teus desertos, e nos teus lugares solitários, e na
tua terra destruída, agora te verás apertada de moradores, e os que
te devoravam se afastarão para longe de ti. E até mesmo os
filhos da tua orfandade dirão aos teus ouvidos: Muito estreito é
para mim este lugar; aparta-te de mim, para que possa habitar nele.
E dirás no teu coração: Quem me gerou estes? Pois eu estava
desfilhada e solitária; entrara em cativeiro, e me retirara; quem,
pois, me criou estes? Eis que eu fui deixada sozinha; e estes onde
estavam? Assim diz o Senhor Deus: Eis que levantarei a minha
mão para os gentios, e ante os povos arvorarei a minha bandeira;
então trarão os teus filhos nos braços, e as tuas filhas serão
levadas sobre os ombros. E os reis serão os teus aios, e as
suas rainhas as tuas amas; diante de ti se inclinarão com o rosto em
terra, e lamberão o pó dos teus pés; e saberás que eu sou o Senhor,
que os que confiam em mim não serão confundidos.

Porventura tirar-se-ia a presa ao poderoso, ou escapariam os
legalmente presos? Mas assim diz o Senhor: Por certo que os
presos se tirarão ao poderoso, e a presa do tirano escapará; porque
eu contenderei com os que contendem contigo, e os teus filhos eu
remirei. E sustentarei os teus opressores com a sua própria
carne, e com o seu próprio sangue se embriagarão, como com mosto; e
toda a carne saberá que eu sou o Senhor, o teu Salvador, e o teu
Redentor, o Forte de Jacó.

\medskip

\lettrine{50}{}Assim diz o Senhor: Onde está a carta de
divórcio de vossa mãe, pela qual eu a repudiei? Ou quem é o meu
credor a quem eu vos tenha vendido? Eis que por vossas maldades
fostes vendidos, e por vossas transgressões vossa mãe foi repudiada.
Por que razão vim eu, e ninguém apareceu? Chamei, e ninguém
respondeu? Porventura tanto se encolheu a minha mão, que já não
possa remir? Ou não há mais força em mim para livrar? Eis que com a
minha repreensão faço secar o mar, torno os rios em deserto, até que
cheirem mal os seus peixes, porquanto não têm água e morrem de sede.
Eu visto os céus de negridão, pôr-lhes-ei um saco para a sua
cobertura.

O Senhor Deus me deu uma língua erudita, para que eu saiba dizer a
seu tempo uma boa palavra ao que está cansado. Ele desperta-me todas
as manhãs, desperta-me o ouvido para que ouça, como aqueles que
aprendem. O Senhor Deus me abriu os ouvidos, e eu não fui
rebelde; não me retirei para trás. As minhas costas ofereci aos
que me feriam, e a minha face aos que me arrancavam os cabelos; não
escondi a minha face dos que me afrontavam e me cuspiam. Porque
o Senhor Deus me ajuda, assim não me confundo; por isso pus o meu
rosto como um seixo, porque sei que não serei envergonhado.
Perto está o que me justifica; quem contenderá comigo?
Compareçamos juntamente; quem é meu adversário? Chegue-se para mim.
Eis que o Senhor Deus me ajuda; quem há que me condene? Eis que
todos eles como roupas se envelhecerão, e a traça os comerá.

Quem há entre vós que tema ao Senhor e ouça a voz do seu servo?
Quando andar em trevas, e não tiver luz nenhuma, confie no nome do
Senhor, e firme-se sobre o seu Deus. Eis que todos vós, que
acendeis fogo, e vos cingis com faíscas, andai entre as labaredas do
vosso fogo, e entre as faíscas, que acendestes. Isto vos sobrevirá
da minha mão, e em tormentos jazereis.

\medskip

\lettrine{51}{}Ouvi-me, vós os que seguis a justiça, os que
buscais ao Senhor. Olhai para a rocha de onde fostes cortados, e
para a caverna do poço de onde fostes cavados. Olhai para
Abraão, vosso pai, e para Sara, que vos deu à luz; porque, sendo ele
só, o chamei, e o abençoei e o multipliquei. Porque o Senhor
consolará a Sião; consolará a todos os seus lugares assolados, e
fará o seu deserto como o Éden, e a sua solidão como o jardim do
Senhor; gozo e alegria se achará nela, ação de graças, e voz de
melodia.

Atendei-me, povo meu, e nação minha, inclinai os ouvidos para mim;
porque de mim sairá a lei, e o meu juízo farei repousar para a luz
dos povos. Perto está a minha justiça, vem saindo a minha
salvação, e os meus braços julgarão os povos; as ilhas me
aguardarão, e no meu braço esperarão. Levantai os vossos olhos
para os céus, e olhai para a terra em baixo, porque os céus
desaparecerão como a fumaça, e a terra se envelhecerá como roupa, e
os seus moradores morrerão semelhantemente; porém a minha salvação
durará para sempre, e a minha justiça não será abolida. Ouvi-me,
vós que conheceis a justiça, povo em cujo coração está a minha lei;
não temais o opróbrio dos homens, nem vos turbeis pelas suas
injúrias. Porque a traça os roerá como a roupa, e o bicho os
comerá como a lã; mas a minha justiça durará para sempre, e a minha
salvação de geração em geração.

Desperta, desperta, veste-te de força, ó braço do Senhor; desperta
como nos dias passados, como nas gerações antigas. Não és tu aquele
que cortou em pedaços a Raabe, o que feriu ao chacal? Não és
tu aquele que secou o mar, as águas do grande abismo? O que fez o
caminho no fundo do mar, para que passassem os remidos? Assim
voltarão os resgatados do Senhor, e virão a Sião com júbilo, e
perpétua alegria haverá sobre as suas cabeças; gozo e alegria
alcançarão, a tristeza e o gemido fugirão. Eu, eu sou aquele
que vos consola; quem, pois, és tu para que temas o homem que é
mortal, ou o filho do homem, que se tornará em erva? E te
esqueces do Senhor que te criou, que estendeu os céus, e fundou a
terra, e temes continuamente todo o dia o furor do angustiador,
quando se prepara para destruir; pois onde está o furor do que te
atribulava? O exilado cativo depressa será solto, e não
morrerá na caverna, e o seu pão não lhe faltará. Porque eu
sou o Senhor teu Deus, que agito o mar, de modo que bramem as suas
ondas. O Senhor dos Exércitos é o seu nome. E ponho as minhas
palavras na tua boca, e te cubro com a sombra da minha mão; para
plantar os céus, e para fundar a terra, e para dizer a Sião: Tu és o
meu povo.

Desperta, desperta, levanta-te, ó Jerusalém, que bebeste da mão
do Senhor o cálice do seu furor; bebeste e sorveste os sedimentos do
cálice do atordoamento. De todos os filhos que ela teve,
nenhum há que a guie mansamente; e de todos os filhos que criou,
nenhum há que a tome pela mão. Estas duas coisas te
aconteceram; quem terá compaixão de ti? A assolação, e o
quebrantamento, e a fome, e a espada! Por quem te consolarei?
Os teus filhos já desmaiaram, jazem nas entradas de todos os
caminhos, como o antílope na rede; cheios estão do furor do Senhor e
da repreensão do teu Deus. Portanto agora ouve isto, ó
aflita, e embriagada, mas não de vinho. Assim diz o teu
Senhor o Senhor, e o teu Deus, que pleiteará a causa do seu povo:
Eis que eu tomo da tua mão o cálice do atordoamento, os sedimentos
do cálice do meu furor, nunca mais dele beberás. Porém,
pô-lo-ei nas mãos dos que te entristeceram, que disseram à tua alma:
Abaixa-te, e passaremos sobre ti; e tu puseste as tuas costas como
chão, e como caminho, aos viandantes.

\medskip

\lettrine{52}{}Desperta, desperta, veste-te da tua fortaleza, ó
Sião; veste-te das tuas roupas formosas, ó Jerusalém, cidade santa,
porque nunca mais entrará em ti nem incircunciso nem imundo.
Sacode-te do pó, levanta-te, e assenta-te, ó Jerusalém: solta-te
das cadeias de teu pescoço, ó cativa filha de Sião. Porque assim
diz o Senhor: Por nada fostes vendidos; também sem dinheiro sereis
resgatados. Porque assim diz o Senhor Deus: O meu povo em tempos
passados desceu ao Egito, para peregrinar lá, e a Assíria sem razão
o oprimiu. E agora, que tenho eu que fazer aqui, diz o Senhor,
pois o meu povo foi tomado sem nenhuma razão? Os que dominam sobre
ele dão uivos, diz o Senhor; e o meu nome é blasfemado
incessantemente o dia todo. Portanto o meu povo saberá o meu
nome; pois, naquele dia, saberá que sou eu mesmo o que falo: Eis-me
aqui.

Quão formosos são, sobre os montes, os pés do que anuncia as boas
novas, que faz ouvir a paz, do que anuncia o bem, que faz ouvir a
salvação, do que diz a Sião: O teu Deus reina! Eis a voz dos
teus atalaias! Eles alçam a voz, juntamente exultam; porque olho a
olho verão, quando o Senhor fizer Sião voltar. Clamai cantando,
exultai juntamente, desertos de Jerusalém; porque o Senhor consolou
o seu povo, remiu a Jerusalém. O Senhor desnudou o seu santo
braço perante os olhos de todas as nações; e todos os confins da
terra verão a salvação do nosso Deus. Retirai-vos,
retirai-vos, saí daí, não toqueis coisa imunda; saí do meio dela,
purificai-vos, os que levais os vasos do Senhor. Porque vós
não saireis apressadamente, nem ireis fugindo; porque o Senhor irá
diante de vós, e o Deus de Israel será a vossa retaguarda.

Eis que o meu servo procederá com prudência; será exaltado, e
elevado, e mui sublime. Como pasmaram muitos à vista dele,
pois o seu parecer estava tão desfigurado, mais do que o de outro
qualquer, e a sua figura mais do que a dos outros filhos dos homens.
Assim borrifará muitas nações, e os reis fecharão as suas
bocas por causa dele; porque aquilo que não lhes foi anunciado
verão, e aquilo que eles não ouviram entenderão.

\medskip

\lettrine{53}{}Quem deu crédito à nossa pregação? E a quem se
manifestou o braço do Senhor? Porque foi subindo como renovo
perante ele, e como raiz de uma terra seca; não tinha beleza nem
formosura e, olhando nós para ele, não havia boa aparência nele,
para que o desejássemos. Era desprezado, e o mais rejeitado
entre os homens, homem de dores, e experimentado nos trabalhos; e,
como um de quem os homens escondiam o rosto, era desprezado, e não
fizemos dele caso algum.

Verdadeiramente ele tomou sobre si as nossas enfermidades, e as
nossas dores levou sobre si; e nós o reputávamos por aflito, ferido
de Deus, e oprimido. Mas ele foi ferido por causa das nossas
transgressões, e moído por causa das nossas iniqüidades; o castigo
que nos traz a paz estava sobre ele, e pelas suas pisaduras fomos
sarados. Todos nós andávamos desgarrados como ovelhas; cada um
se desviava pelo seu caminho; mas o Senhor fez cair sobre ele a
iniqüidade de nós todos. Ele foi oprimido e afligido, mas não
abriu a sua boca; como um cordeiro foi levado ao matadouro, e como a
ovelha muda perante os seus tosquiadores, assim ele não abriu a sua
boca. Da opressão e do juízo foi tirado; e quem contará o tempo
da sua vida? Porquanto foi cortado da terra dos viventes; pela
transgressão do meu povo ele foi atingido. E puseram a sua
sepultura com os ímpios, e com o rico na sua morte; ainda que nunca
cometeu injustiça, nem houve engano na sua boca.

Todavia, ao Senhor agradou moê-lo, fazendo-o enfermar; quando a
sua alma se puser por expiação do pecado, verá a sua posteridade,
prolongará os seus dias; e o bom prazer do Senhor prosperará na sua
mão. Ele verá o fruto do trabalho da sua alma, e ficará
satisfeito; com o seu conhecimento o meu servo, o justo, justificará
a muitos; porque as iniqüidades deles levará sobre si. Por
isso lhe darei a parte de muitos, e com os poderosos repartirá ele o
despojo; porquanto derramou a sua alma na morte, e foi contado com
os transgressores; mas ele levou sobre si o pecado de muitos, e
intercedeu pelos transgressores.

\medskip

\lettrine{54}{}Canta alegremente, ó estéril, que não deste à
luz; rompe em cântico, e exclama com alegria, tu que não tiveste
dores de parto; porque mais são os filhos da mulher solitária, do
que os filhos da casada, diz o Senhor. Amplia o lugar da tua
tenda, e estendam-se as cortinas das tuas habitações; não o impeças;
alonga as tuas cordas, e fixa bem as tuas estacas. Porque
transbordarás para a direita e para a esquerda; e a tua descendência
possuirá os gentios e fará que sejam habitadas as cidades assoladas.
Não temas, porque não serás envergonhada; e não te envergonhes,
porque não serás humilhada; antes te esquecerás da vergonha da tua
mocidade, e não te lembrarás mais do opróbrio da tua viuvez.
Porque o teu Criador é o teu marido; o Senhor dos Exércitos é o
seu nome; e o Santo de Israel é o teu Redentor; que é chamado o Deus
de toda a terra.

Porque o Senhor te chamou como a mulher desamparada e triste de
espírito; como a mulher da mocidade, que fora desprezada, diz o teu
Deus. Por um breve momento te deixei, mas com grandes
misericórdias te recolherei; com um pouco de ira escondi a minha
face de ti por um momento; mas com benignidade eterna me
compadecerei de ti, diz o Senhor, o teu Redentor. Porque isto
será para mim como as águas de Noé; pois jurei que as águas de Noé
não passariam mais sobre a terra; assim jurei que não me irarei mais
contra ti, nem te repreenderei. Porque os montes se
retirarão, e os outeiros serão abalados; porém a minha benignidade
não se apartará de ti, e a aliança da minha paz não mudará, diz o
Senhor que se compadece de ti.

Tu, oprimida, arrojada com a tormenta e desconsolada, eis que eu
assentarei as tuas pedras com todo o ornamento, e te fundarei sobre
as safiras. E farei os teus vitrais de rubis, e as tuas
portas de carbúnculos, e todos os teus termos de pedras aprazíveis.
E todos os teus filhos serão ensinados do Senhor; e a paz de
teus filhos será abundante. Com justiça serás estabelecida;
estarás longe da opressão, porque já não temerás; e também do
terror, porque não chegará a ti. Eis que seguramente poderão
vir a juntar-se contra ti, mas não será por mim; quem se ajuntar
contra ti cairá por causa de ti. Eis que eu criei o ferreiro,
que assopra as brasas no fogo, e que produz a ferramenta para a sua
obra; também criei o assolador, para destruir. Toda a
ferramenta preparada contra ti não prosperará, e toda a língua que
se levantar contra ti em juízo tu a condenarás; esta é a herança dos
servos do Senhor, e a sua justiça que de mim procede, diz o Senhor.

\medskip

\lettrine{55}{}Ó vós, todos os que tendes sede, vinde às águas,
e os que não tendes dinheiro, vinde, comprai, e comei; sim, vinde,
comprai, sem dinheiro e sem preço, vinho e leite. Por que
gastais o dinheiro naquilo que não é pão? E o produto do vosso
trabalho naquilo que não pode satisfazer? Ouvi-me atentamente, e
comei o que é bom, e a vossa alma se deleite com a gordura.
Inclinai os vossos ouvidos, e vinde a mim; ouvi, e a vossa alma
viverá; porque convosco farei uma aliança perpétua, dando-vos as
firmes beneficências de Davi. Eis que eu o dei por testemunha
aos povos, como líder e governador dos povos. Eis que chamarás a
uma nação que não conheces, e uma nação que nunca te conheceu
correrá para ti, por amor do Senhor teu Deus, e do Santo de Israel;
porque ele te glorificou.

Buscai ao Senhor enquanto se pode achar, invocai-o enquanto está
perto. Deixe o ímpio o seu caminho, e o homem maligno os seus
pensamentos, e se converta ao Senhor, que se compadecerá dele; torne
para o nosso Deus, porque grandioso é em perdoar. Porque os meus
pensamentos não são os vossos pensamentos, nem os vossos caminhos os
meus caminhos, diz o Senhor. Porque assim como os céus são mais
altos do que a terra, assim são os meus caminhos mais altos do que
os vossos caminhos, e os meus pensamentos mais altos do que os
vossos pensamentos. Porque, assim como desce a chuva e a neve
dos céus, e para lá não tornam, mas regam a terra, e a fazem
produzir, e brotar, e dar semente ao semeador, e pão ao que come,
assim será a minha palavra, que sair da minha boca; ela não
voltará para mim vazia, antes fará o que me apraz, e prosperará
naquilo para que a enviei. Porque com alegria saireis, e em
paz sereis guiados; os montes e os outeiros romperão em cântico
diante de vós, e todas as árvores do campo baterão palmas. Em
lugar do espinheiro crescerá a faia, e em lugar da sarça crescerá a
murta; o que será para o Senhor por nome, e por sinal eterno, que
nunca se apagará.

\medskip

\lettrine{56}{}Assim diz o Senhor: Guardai o juízo, e fazei
justiça, porque a minha salvação está prestes a vir, e a minha
justiça, para se manifestar. Bem-aventurado o homem que fizer
isto, e o filho do homem que lançar mão disto; que se guarda de
profanar o sábado, e guarda a sua mão de fazer algum mal.

E não fale o filho do estrangeiro, que se houver unido ao Senhor,
dizendo: Certamente o Senhor me separará do seu povo; nem tampouco
diga o eunuco: Eis que sou uma árvore seca.
 Porque assim diz o Senhor a respeito dos eunucos, que guardam os
meus sábados, e escolhem aquilo em que eu me agrado, e abraçam a
minha aliança: Também lhes darei na minha casa e dentro dos meus
muros um lugar e um nome, melhor do que o de filhos e filhas; um
nome eterno darei a cada um deles, que nunca se apagará. E aos
filhos dos estrangeiros, que se unirem ao Senhor, para o servirem, e
para amarem o nome do Senhor, e para serem seus servos, todos os que
guardarem o sábado, não o profanando, e os que abraçarem a minha
aliança, também os levarei ao meu santo monte, e os alegrarei na
minha casa de oração; os seus holocaustos e os seus sacrifícios
serão aceitos no meu altar; porque a minha casa será chamada casa de
oração para todos os povos. Assim diz o Senhor Deus, que
congrega os dispersos de Israel: Ainda ajuntarei outros aos que já
se lhe ajuntaram.

Vós, todos os animais do campo, todos os animais dos bosques,
vinde comer. Todos os seus atalaias são cegos, nada sabem;
todos são cães mudos, não podem ladrar; andam adormecidos, estão
deitados, e gostam do sono. E estes cães são gulosos, não se
podem fartar; e eles são pastores que nada compreendem; todos eles
se tornam para o seu caminho, cada um para a sua ganância, cada um
por sua parte. Vinde, dizem, trarei vinho, e beberemos bebida
forte; e o dia de amanhã será como este, e ainda muito mais
abundante.

\medskip

\lettrine{57}{}Perece o justo, e não há quem considere isso em
seu coração, e os homens compassivos são recolhidos, sem que alguém
considere que o justo é levado antes do mal. Entrará em paz;
descansarão nas suas camas, os que houverem andado na sua retidão.

Mas chegai-vos aqui, vós os filhos da agoureira, descendência
adulterina, e de prostituição. De quem fazeis o vosso
passatempo? Contra quem escancarais a boca, e deitais para fora a
língua? Porventura não sois filhos da transgressão, descendência da
falsidade, que vos inflamais com os deuses debaixo de toda a
árvore verde, e sacrificais os filhos nos ribeiros, nas fendas dos
penhascos? Nas pedras lisas dos ribeiros está a tua parte;
estas, estas são a tua sorte; sobre elas também derramaste a tua
libação, e lhes ofereceste ofertas; contentar-me-ia eu com estas
coisas? Sobre o monte alto e levantado pões a tua cama; e lá
subiste para oferecer sacrifícios. E detrás das portas, e dos
umbrais puseste o teu memorial; pois te descobriste a outros que não
a mim, e subiste, alargaste a tua cama, e fizeste aliança com alguns
deles; amaste a sua cama, onde quer que a viste. E foste ao rei
com óleo, e multiplicaste os teus perfumes e enviaste os teus
embaixadores para longe, e te abateste até ao inferno. Na tua
comprida viagem te cansaste; porém não disseste: Não há esperança;
achaste novo vigor na tua mão; por isso não adoeceste. Mas de
quem tiveste receio, ou temor, para que mentisses, e não te
lembrasses de mim, nem no teu coração me pusesses? Não é porventura
porque eu me calei, e isso há muito tempo, e não me temes? Eu
publicarei a tua justiça, e as tuas obras, que não te aproveitarão.

Quando clamares, livrem-te os ídolos que ajuntaste; mas o vento a
todos levará, e um sopro os arrebatará; mas o que confia em mim
possuirá a terra, e herdará o meu santo monte. E dir-se-á:
Aplanai, aplanai a estrada, preparai o caminho; tirai os tropeços do
caminho do meu povo. Porque assim diz o Alto e o Sublime, que
habita na eternidade, e cujo nome é Santo: Num alto e santo lugar
habito; como também com o contrito e abatido de espírito, para
vivificar o espírito dos abatidos, e para vivificar o coração dos
contritos. Porque não contenderei para sempre, nem
continuamente me indignarei; porque o espírito perante a minha face
se desfaleceria, e as almas que eu fiz.

Pela iniqüidade da sua avareza me indignei, e o feri; escondi-me,
e indignei-me; contudo, rebelde, seguiu o caminho do seu coração.
Eu vejo os seus caminhos, e o sararei, e o guiarei, e lhe
tornarei a dar consolação, a saber, aos seus pranteadores. Eu
crio os frutos dos lábios: paz, paz, para o que está longe; e para o
que está perto, diz o Senhor, e eu o sararei. Mas os ímpios
são como o mar bravo, porque não se pode aquietar, e as suas águas
lançam de si lama e lodo. Não há paz para os ímpios, diz o
meu Deus.

\medskip

\lettrine{58}{}Clama em alta voz, não te detenhas, levanta a
tua voz como a trombeta e anuncia ao meu povo a sua transgressão, e
à casa de Jacó os seus pecados. Todavia me procuram cada dia,
tomam prazer em saber os meus caminhos, como um povo que pratica
justiça, e não deixa o direito do seu Deus; perguntam-me pelos
direitos da justiça, e têm prazer em se chegarem a Deus,
dizendo: Por que jejuamos nós, e tu não atentas para isso? Por
que afligimos as nossas almas, e tu não o sabes? Eis que no dia em
que jejuais achais o vosso próprio contentamento, e requereis todo o
vosso trabalho. Eis que para contendas e debates jejuais, e para
ferirdes com punho iníquo; não jejueis como hoje, para fazer ouvir a
vossa voz no alto. Seria este o jejum que eu escolheria, que o
homem um dia aflija a sua alma, que incline a sua cabeça como o
junco, e estenda debaixo de si saco e cinza? Chamarias tu a isto
jejum e dia aprazível ao Senhor? Porventura não é este o jejum
que escolhi, que soltes as ligaduras da impiedade, que desfaças as
ataduras do jugo e que deixes livres os oprimidos, e despedaces todo
o jugo? Porventura não é também que repartas o teu pão com o
faminto, e recolhas em casa os pobres abandonados; e, quando vires o
nu, o cubras, e não te escondas da tua carne?

Então romperá a tua luz como a alva, e a tua cura apressadamente
brotará, e a tua justiça irá adiante de ti, e a glória do Senhor
será a tua retaguarda. Então clamarás, e o Senhor te responderá;
gritarás, e ele dirá: Eis-me aqui. Se tirares do meio de ti o jugo,
o estender do dedo, e o falar iniquamente; e se abrires a tua
alma ao faminto, e fartares a alma aflita; então a tua luz nascerá
nas trevas, e a tua escuridão será como o meio-dia. E o
Senhor te guiará continuamente, e fartará a tua alma em lugares
áridos, e fortificará os teus ossos; e serás como um jardim regado,
e como um manancial, cujas águas nunca faltam. E os que de ti
procederem edificarão as antigas ruínas; e levantarás os fundamentos
de geração em geração; e chamar-te-ão reparador das roturas, e
restaurador de veredas para morar.

Se desviares o teu pé do sábado, de fazeres a tua vontade no meu
santo dia, e chamares ao sábado deleitoso, e o santo dia do Senhor,
digno de honra, e o honrares não seguindo os teus caminhos, nem
pretendendo fazer a tua própria vontade, nem falares as tuas
próprias palavras, então te deleitarás no Senhor, e te farei
cavalgar sobre as alturas da terra, e te sustentarei com a herança
de teu pai Jacó; porque a boca do Senhor o disse.

\medskip

\lettrine{59}{}Eis que a mão do Senhor não está encolhida, para
que não possa salvar; nem agravado o seu ouvido, para não poder
ouvir. Mas as vossas iniqüidades fazem separação entre vós e o
vosso Deus; e os vossos pecados encobrem o seu rosto de vós, para
que não vos ouça. Porque as vossas mãos estão contaminadas de
sangue, e os vossos dedos de iniqüidade; os vossos lábios falam
falsidade, a vossa língua pronuncia perversidade. Ninguém há que
clame pela justiça, nem ninguém que compareça em juízo pela verdade;
confiam na vaidade, e falam mentiras; concebem o mal, e dão à luz a
iniqüidade. Chocam ovos de basilisco, e tecem teias de aranha; o
que comer dos ovos deles, morrerá; e, quebrando-os, sairá uma
víbora. As suas teias não prestam para vestes nem se poderão
cobrir com as suas obras; as suas obras são obras de iniqüidade, e
obra de violência há nas suas mãos. Os seus pés correm para o
mal, e se apressam para derramarem o sangue inocente; os seus
pensamentos são pensamentos de iniqüidade; destruição e
quebrantamento há nas suas estradas. Não conhecem o caminho da
paz, nem há justiça nos seus passos; fizeram para si veredas
tortuosas; todo aquele que anda por elas não tem conhecimento da
paz.

Por isso o juízo está longe de nós, e a justiça não nos alcança;
esperamos pela luz, e eis que só há trevas; pelo resplendor, mas
andamos em escuridão. Apalpamos as paredes como cegos, e como
os que não têm olhos andamos apalpando; tropeçamos ao meio-dia como
nas trevas, e nos lugares escuros como mortos. Todos nós
bramamos como ursos, e continuamente gememos como pombas; esperamos
pelo juízo, e não o há; pela salvação, e está longe de nós.
Porque as nossas transgressões se multiplicaram perante ti, e
os nossos pecados testificam contra nós; porque as nossas
transgressões estão conosco, e conhecemos as nossas iniqüidades;
como o prevaricar, e mentir contra o Senhor, e o
desviarmo-nos do nosso Deus, o falar de opressão e rebelião, o
conceber e proferir do coração palavras de falsidade. Por
isso o direito se tornou atrás, e a justiça se pôs de longe; porque
a verdade anda tropeçando pelas ruas, e a eqüidade não pode entrar.
Sim, a verdade desfalece, e quem se desvia do mal arrisca-se
a ser despojado; e o Senhor viu, e pareceu mal aos seus olhos que
não houvesse justiça.

E vendo que ninguém havia, maravilhou-se de que não houvesse um
intercessor; por isso o seu próprio braço lhe trouxe a salvação, e a
sua própria justiça o susteve. Pois vestiu-se de justiça,
como de uma couraça, e pôs o capacete da salvação na sua cabeça, e
por vestidura pôs sobre si vestes de vingança, e cobriu-se de zelo,
como de um manto. Conforme forem as obras deles, assim será a
sua retribuição, furor aos seus adversários, e recompensa aos seus
inimigos; às ilhas dará ele a sua recompensa. Então temerão o
nome do Senhor desde o poente, e a sua glória desde o nascente do
sol; vindo o inimigo como uma corrente de águas, o Espírito do
Senhor arvorará contra ele a sua bandeira. E virá um Redentor
a Sião e aos que em Jacó se converterem da transgressão, diz o
Senhor. Quanto a mim, esta é a minha aliança com eles, diz o
Senhor: o meu espírito, que está sobre ti, e as minhas palavras, que
pus na tua boca, não se desviarão da tua boca nem da boca da tua
descendência, nem da boca da descendência da tua descendência, diz o
Senhor, desde agora e para todo o sempre.

\medskip

\lettrine{60}{}Levanta-te, resplandece, porque vem a tua luz, e
a glória do Senhor vai nascendo sobre ti; porque eis que as
trevas cobriram a terra, e a escuridão os povos; mas sobre ti o
Senhor virá surgindo, e a sua glória se verá sobre ti. E os
gentios caminharão à tua luz, e os reis ao resplendor que te nasceu.
 Levanta em redor os teus olhos, e vê; todos estes já se ajuntaram,
e vêm a ti; teus filhos virão de longe, e tuas filhas serão criadas
ao teu lado. Então o verás, e serás iluminado, e o teu coração
estremecerá e se alargará; porque a abundância do mar se tornará a
ti, e as riquezas dos gentios virão a ti. A multidão de camelos
te cobrirá, os dromedários de Midiã e Efá; todos virão de Sabá; ouro
e incenso trarão, e publicarão os louvores do Senhor. Todas as
ovelhas de Quedar se congregarão a ti; os carneiros de Nebaiote te
servirão; com agrado subirão ao meu altar, e eu glorificarei a casa
da minha glória. Quem são estes que vêm voando como nuvens, e
como pombas às suas janelas?

Certamente as ilhas me aguardarão, e primeiro os navios de Társis,
para trazer teus filhos de longe, e com eles a sua prata e o seu
ouro, para o nome do Senhor teu Deus, e para o Santo de Israel,
porquanto ele te glorificou. E os filhos dos estrangeiros
edificarão os teus muros, e os seus reis te servirão; porque no meu
furor te feri, mas na minha benignidade tive misericórdia de ti.
E as tuas portas estarão abertas de contínuo, nem de dia nem
de noite se fecharão; para que tragam a ti as riquezas dos gentios,
e, conduzidos com elas, os seus reis. Porque a nação e o
reino que não te servirem perecerão; sim, essas nações serão de todo
assoladas. A glória do Líbano virá a ti; a faia, o pinheiro,
e o álamo conjuntamente, para ornarem o lugar do meu santuário, e
glorificarei o lugar dos meus pés. Também virão a ti,
inclinando-se, os filhos dos que te oprimiram; e prostrar-se-ão às
plantas dos teus pés todos os que te desprezaram; e chamar-te-ão a
cidade do Senhor, a Sião do Santo de Israel.

Em lugar de seres deixada, e odiada, de modo que ninguém passava
por ti, far-te-ei uma excelência perpétua, um gozo de geração em
geração. E mamarás o leite dos gentios, e alimentar-te-ás ao
peito dos reis; e saberás que eu sou o Senhor, o teu Salvador, e o
teu Redentor, o Poderoso de Jacó. Por cobre trarei ouro, e
por ferro trarei prata, e por madeira, bronze, e por pedras, ferro;
e farei pacíficos os teus oficiais e justos os teus exatores.
Nunca mais se ouvirá de violência na tua terra, desolação nem
destruição nos teus termos; mas aos teus muros chamarás Salvação, e
às tuas portas Louvor. Nunca mais te servirá o sol para luz
do dia nem com o seu resplendor a lua te iluminará; mas o Senhor
será a tua luz perpétua, e o teu Deus a tua glória. Nunca
mais se porá o teu sol, nem a tua lua minguará; porque o Senhor será
a tua luz perpétua, e os dias do teu luto findarão. E todos
os do teu povo serão justos, para sempre herdarão a terra; serão
renovos por mim plantados, obra das minhas mãos, para que eu seja
glorificado. O menor virá a ser mil, e o mínimo uma nação
forte; eu, o Senhor, ao seu tempo o farei prontamente.

\medskip

\lettrine{61}{}O Espírito do Senhor Deus está sobre mim; porque
o Senhor me ungiu, para pregar boas novas aos mansos; enviou-me a
restaurar os contritos de coração, a proclamar liberdade aos
cativos, e a abertura de prisão aos presos; a apregoar o ano
aceitável do Senhor e o dia da vingança do nosso Deus; a consolar
todos os tristes; a ordenar acerca dos tristes de Sião que se
lhes dê glória em vez de cinza, óleo de gozo em vez de tristeza,
vestes de louvor em vez de espírito angustiado; a fim de que se
chamem árvores de justiça, plantações do Senhor, para que ele seja
glorificado.

E edificarão os lugares antigamente assolados, e restaurarão os
anteriormente destruídos, e renovarão as cidades assoladas,
destruídas de geração em geração. E haverá estrangeiros, que
apascentarão os vossos rebanhos; e estranhos serão os vossos
lavradores e os vossos vinhateiros. Porém vós sereis chamados
sacerdotes do Senhor, e vos chamarão ministros de nosso Deus;
comereis a riqueza dos gentios, e na sua glória vos gloriareis.
Em lugar da vossa vergonha tereis dupla honra; e em lugar da
afronta exultareis na vossa parte; por isso na sua terra possuirão o
dobro, e terão perpétua alegria. Porque eu, o Senhor, amo o
juízo, odeio o que foi roubado oferecido em holocausto; portanto,
firmarei em verdade a sua obra; e farei uma aliança eterna com eles.
E a sua posteridade será conhecida entre os gentios, e os seus
descendentes no meio dos povos; todos quantos os virem os
conhecerão, como descendência bendita do Senhor.

Regozijar-me-ei muito no Senhor, a minha alma se alegrará no meu
Deus; porque me vestiu de roupas de salvação, cobriu-me com o manto
de justiça, como um noivo se adorna com turbante sacerdotal, e como
a noiva que se enfeita com as suas jóias. Porque, como a
terra produz os seus renovos, e como o jardim faz brotar o que nele
se semeia, assim o Senhor Deus fará brotar a justiça e o louvor para
todas as nações.

\medskip

\lettrine{62}{}Por amor de Sião não me calarei, e por amor de
Jerusalém não me aquietarei, até que saia a sua justiça como um
resplendor, e a sua salvação como uma tocha acesa. E os gentios
verão a tua justiça, e todos os reis a tua glória; e chamar-te-ão
por um nome novo, que a boca do Senhor designará. E serás uma
coroa de glória na mão do Senhor, e um diadema real na mão do teu
Deus. Nunca mais te chamarão: Desamparada, nem a tua terra se
denominará jamais: Assolada; mas chamar-te-ão: O meu prazer está
nela, e à tua terra: A casada; porque o Senhor se agrada de ti, e a
tua terra se casará. Porque, como o jovem se casa com a virgem,
assim teus filhos se casarão contigo; e como o noivo se alegra da
noiva, assim se alegrará de ti o teu Deus.

Ó Jerusalém, sobre os teus muros pus guardas, que todo o dia e
toda a noite jamais se calarão; ó vós, os que fazeis lembrar ao
Senhor, não haja descanso em vós, nem deis a ele descanso, até
que confirme, e até que ponha a Jerusalém por louvor na terra.
Jurou o Senhor pela sua mão direita, e pelo braço da sua força:
Nunca mais darei o teu trigo por comida aos teus inimigos, nem os
estrangeiros beberão o teu mosto, em que trabalhaste. Mas os que
o ajuntarem o comerão, e louvarão ao Senhor; e os que o colherem
beberão nos átrios do meu santuário.

Passai, passai pelas portas; preparai o caminho ao povo;
aplainai, aplainai a estrada, limpai-a das pedras; arvorai a
bandeira aos povos. Eis que o Senhor fez ouvir até às
extremidades da terra: Dizei à filha de Sião: Eis que vem a tua
salvação; eis que com ele vem o seu galardão, e a sua obra diante
dele. E chamar-lhes-ão: Povo santo, remidos do Senhor; e tu
serás chamada: Procurada, a cidade não desamparada.

\medskip

\lettrine{63}{}Quem é este, que vem de Edom, de Bozra, com
vestes tintas; este que é glorioso em sua vestidura, que marcha com
a sua grande força? Eu, que falo em justiça, poderoso para salvar.
Por que está vermelha a tua vestidura, e as tuas roupas como as
daquele que pisa no lagar? Eu sozinho pisei no lagar, e dos
povos ninguém houve comigo; e os pisei na minha ira, e os esmaguei
no meu furor; e o seu sangue salpicou as minhas vestes, e manchei
toda a minha vestidura. Porque o dia da vingança estava no meu
coração; e o ano dos meus remidos é chegado. E olhei, e não
havia quem me ajudasse; e admirei-me de não haver quem me
sustivesse, por isso o meu braço me trouxe a salvação, e o meu furor
me susteve. E atropelei os povos na minha ira, e os embriaguei
no meu furor; e a sua força derrubei por terra.

As benignidades do Senhor mencionarei, e os muitos louvores do
Senhor, conforme tudo quanto o Senhor nos concedeu; e grande bondade
para com a casa de Israel, que usou com eles segundo as suas
misericórdias, e segundo a multidão das suas benignidades.
Porque dizia: Certamente eles são meu povo, filhos que não
mentirão; assim ele se fez o seu Salvador. Em toda a angústia
deles ele foi angustiado, e o anjo da sua presença os salvou; pelo
seu amor, e pela sua compaixão ele os remiu; e os tomou, e os
conduziu todos os dias da antiguidade. Mas eles foram
rebeldes, e contristaram o seu Espírito Santo; por isso se lhes
tornou em inimigo, e ele mesmo pelejou contra eles. Todavia
se lembrou dos dias da antiguidade, de Moisés, e do seu povo,
dizendo: Onde está agora o que os fez subir do mar com os pastores
do seu rebanho? Onde está o que pôs no meio deles o seu Espírito
Santo? Aquele cujo braço glorioso ele fez andar à mão direita
de Moisés, que fendeu as águas diante deles, para fazer para si um
nome eterno? Aquele que os guiou pelos abismos, como o cavalo
no deserto, de modo que nunca tropeçaram? Como o animal que
desce ao vale, o Espírito do Senhor lhes deu descanso; assim guiaste
ao teu povo, para te fazeres um nome glorioso.

Atenta desde os céus, e olha desde a tua santa e gloriosa
habitação. Onde estão o teu zelo e as tuas obras poderosas? A
comoção das tuas entranhas, e das tuas misericórdias, detém-se para
comigo? Mas tu és nosso Pai, ainda que Abraão não nos
conhece, e Israel não nos reconhece; tu, ó Senhor, és nosso Pai;
nosso Redentor desde a antiguidade é o teu nome. Por que, ó
Senhor, nos fazes errar dos teus caminhos? Por que endureces o nosso
coração, para que não te temamos? Volta, por amor dos teus servos,
às tribos da tua herança. Só por um pouco de tempo o teu
santo povo a possuiu; nossos adversários pisaram o teu santuário.
Somos feitos como aqueles sobre quem tu nunca dominaste, e
como os que nunca se chamaram pelo teu nome.

\medskip

\lettrine{64}{}Oh! se fendesses os céus, e descesses, e os
montes se escoassem de diante da tua face, como o fogo abrasador
de fundição, fogo que faz ferver as águas, para fazeres notório o
teu nome aos teus adversários, e assim as nações tremessem da tua
presença! Quando fazias coisas terríveis, que nunca esperávamos,
descias, e os montes se escoavam diante da tua face. Porque
desde a antiguidade não se ouviu, nem com ouvidos se percebeu, nem
com os olhos se viu um Deus além de ti que trabalha para aquele que
nele espera. Saíste ao encontro daquele que se alegrava e
praticava justiça e dos que se lembram de ti nos teus caminhos; eis
que te iraste, porque pecamos; neles há eternidade, para que sejamos
salvos?

Mas todos nós somos como o imundo, e todas as nossas justiças como
trapo da imundícia; e todos nós murchamos como a folha, e as nossas
iniqüidades como um vento nos arrebatam. E já ninguém há que
invoque o teu nome, que se desperte, e te detenhas; porque escondes
de nós o teu rosto, e nos fazes derreter, por causa das nossas
iniqüidades. Mas agora, ó Senhor, tu és nosso Pai; nós o barro e
tu o nosso oleiro; e todos nós a obra das tuas mãos. Não te
enfureças tanto, ó Senhor, nem perpetuamente te lembres da
iniqüidade; olha, pois, nós te pedimos, todos nós somos o teu povo.
As tuas santas cidades tornaram-se um deserto; Sião está
feita um deserto, Jerusalém está assolada. A nossa santa e
gloriosa casa, em que te louvavam nossos pais, foi queimada a fogo;
e todas as nossas coisas preciosas se tornaram em assolação.
Conter-te-ias tu ainda sobre estas coisas, ó Senhor? Ficarias
calado, e nos afligirias tanto?

\medskip

\lettrine{65}{}Fui buscado dos que não perguntavam por mim, fui
achado daqueles que não me buscavam; a uma nação que não se chamava
do meu nome eu disse: Eis-me aqui. Eis-me aqui. Estendi as
minhas mãos o dia todo a um povo rebelde, que anda por caminho, que
não é bom, após os seus pensamentos; povo que de contínuo me
irrita diante da minha face, sacrificando em jardins e queimando
incenso sobre altares de tijolos; que habita entre as
sepulturas, e passa as noites junto aos lugares secretos; come carne
de porco e tem caldo de coisas abomináveis nos seus vasos; que
dizem: Fica onde estás, e não te chegues a mim, porque sou mais
santo do que tu. Estes são fumaça no meu nariz, um fogo que arde
todo o dia. Eis que está escrito diante de mim: não me calarei;
mas eu pagarei, sim, pagarei no seu seio, as vossas iniqüidades,
e juntamente as iniqüidades de vossos pais, diz o Senhor, que
queimaram incenso nos montes, e me afrontaram nos outeiros; assim
lhes tornarei a medir as suas obras antigas no seu seio.

Assim diz o Senhor: Como quando se acha mosto num cacho de uvas,
dizem: Não o desperdices, pois há bênção nele, assim farei por amor
de meus servos, que não os destrua a todos, e produzirei
descendência a Jacó, e a Judá um herdeiro que possua os meus montes;
e os meus eleitos herdarão a terra e os meus servos habitarão ali.
E Sarom servirá de curral de rebanhos, e o vale de Acor lugar
de repouso de gados, para o meu povo, que me buscou.

Mas a vós, os que vos apartais do Senhor, os que vos esqueceis do
meu santo monte, os que preparais uma mesa para a Fortuna, e que
misturais a bebida para o Destino. Também vos destinareis à
espada, e todos vos encurvareis à matança; porquanto chamei, e não
respondestes; falei, e não ouvistes; mas fizestes o que era mau aos
meus olhos, e escolhestes aquilo em que não tinha prazer.
Portanto assim diz o Senhor Deus: Eis que os meus servos
comerão, mas vós padecereis fome; eis que os meus servos beberão,
porém vós tereis sede; eis que os meus servos se alegrarão, mas vós
vos envergonhareis; eis que os meus servos exultarão pela
alegria de coração, mas vós gritareis pela tristeza de coração; e
uivareis pelo quebrantamento de espírito. E deixareis o vosso
nome aos meus eleitos por maldição; e o Senhor Deus vos matará; e a
seus servos chamará por outro nome. Assim que aquele que se
bendisser na terra, se bendirá no Deus da verdade; e aquele que
jurar na terra, jurará pelo Deus da verdade; porque já estão
esquecidas as angústias passadas, e estão escondidas dos meus olhos.

Porque, eis que eu crio novos céus e nova terra; e não haverá
mais lembrança das coisas passadas, nem mais se recordarão.
Mas vós folgareis e exultareis perpetuamente no que eu crio;
porque eis que crio para Jerusalém uma alegria, e para o seu povo
gozo. E exultarei em Jerusalém, e me alegrarei no meu povo; e
nunca mais se ouvirá nela voz de choro nem voz de clamor. Não
haverá mais nela criança de poucos dias, nem velho que não cumpra os
seus dias; porque o menino morrerá de cem anos; porém o pecador de
cem anos será amaldiçoado. E edificarão casas, e as
habitarão; e plantarão vinhas, e comerão o seu fruto. Não
edificarão para que outros habitem; não plantarão para que outros
comam; porque os dias do meu povo serão como os dias da árvore, e os
meus eleitos gozarão das obras das suas mãos. Não trabalharão
debalde, nem terão filhos para a perturbação; porque são a
posteridade bendita do Senhor, e os seus descendentes estarão com
eles. E será que antes que clamem eu responderei; estando
eles ainda falando, eu os ouvirei. O lobo e o cordeiro se
apascentarão juntos, e o leão comerá palha como o boi; e pó será a
comida da serpente. Não farão mal nem dano algum em todo o meu santo
monte, diz o Senhor.

\medskip

\lettrine{66}{}Assim diz o Senhor: O céu é o meu trono, e a
terra o escabelo dos meus pés; que casa me edificaríeis vós? E qual
seria o lugar do meu descanso? Porque a minha mão fez todas
estas coisas, e assim todas elas foram feitas, diz o Senhor; mas
para esse olharei, para o pobre e abatido de espírito, e que treme
da minha palavra. Quem mata um boi é como o que tira a vida a um
homem; quem sacrifica um cordeiro é como o que degola um cão; quem
oferece uma oblação\footnote{Oferenda, sacrificio a Deus, oblata.} é
como o que oferece sangue de porco; quem queima incenso em memorial
é como o que bendiz a um ídolo; também estes escolhem os seus
próprios caminhos, e a sua alma se deleita nas suas abominações.
Também eu escolherei as suas calamidades, farei vir sobre eles
os seus temores; porquanto clamei e ninguém respondeu, falei e não
escutaram; mas fizeram o que era mau aos meus olhos, e escolheram
aquilo em que eu não tinha prazer.

Ouvi a palavra do Senhor, os que tremeis da sua palavra. Vossos
irmãos, que vos odeiam e que para longe vos lançam por amor do meu
nome, dizem: Seja glorificado o Senhor, para que vejamos a vossa
alegria; mas eles serão confundidos. Uma voz de grande rumor
virá da cidade, uma voz do templo, a voz do Senhor, que dá o pago
aos seus inimigos. Antes que estivesse de parto, deu à luz;
antes que lhe viessem as dores, deu à luz um menino. Quem jamais
ouviu tal coisa? Quem viu coisas semelhantes? Poder-se-ia fazer
nascer uma terra num só dia? Nasceria uma nação de uma só vez? Mas
Sião esteve de parto e já deu à luz seus filhos. Abriria eu a
madre, e não geraria? diz o Senhor; geraria eu, e fecharia a madre?
diz o teu Deus. Regozijai-vos com Jerusalém, e alegrai-vos
por ela, vós todos os que a amais; enchei-vos por ela de alegria,
todos os que por ela pranteastes; para que mameis, e vos
farteis dos peitos das suas consolações; para que sugueis, e vos
deleiteis com a abundância da sua glória. Porque assim diz o
Senhor: Eis que estenderei sobre ela a paz como um rio, e a glória
dos gentios como um ribeiro que transborda; então mamareis, ao colo
vos trarão, e sobre os joelhos vos afagarão. Como alguém a
quem consola sua mãe, assim eu vos consolarei; e em Jerusalém vós
sereis consolados. E vós vereis e alegrar-se-á o vosso
coração, e os vossos ossos reverdecerão como a erva tenra; então a
mão do Senhor será notória aos seus servos, e ele se indignará
contra os seus inimigos.

Porque, eis que o Senhor virá com fogo; e os seus carros como um
torvelinho; para tornar a sua ira em furor, e a sua repreensão em
chamas de fogo. Porque com fogo e com a sua espada entrará o
Senhor em juízo com toda a carne; e os mortos do Senhor serão
multiplicados. Os que se santificam, e se purificam, nos
jardins uns após outros; os que comem carne de porco, e a
abominação, e o rato, juntamente serão consumidos, diz o Senhor.
Porque conheço as suas obras e os seus pensamentos; vem o dia
em que ajuntarei todas as nações e línguas; e virão e verão a minha
glória. E porei entre eles um sinal, e os que deles escaparem
enviarei às nações, a Társis, Pul, e Lude, flecheiros, a Tubal e
Javã, até às ilhas de mais longe, que não ouviram a minha fama, nem
viram a minha glória; e anunciarão a minha glória entre os gentios.
E trarão a todos os vossos irmãos, dentre todas as nações,
por oferta ao Senhor, sobre cavalos, e em carros, e em liteiras, e
sobre mulas, e sobre dromedários, trarão ao meu santo monte, a
Jerusalém, diz o Senhor; como quando os filhos de Israel trazem as
suas ofertas em vasos limpos à casa do Senhor. E também deles
tomarei a alguns para sacerdotes e para levitas, diz o Senhor.
Porque, como os novos céus, e a nova terra, que hei de fazer,
estarão diante da minha face, diz o Senhor, assim também há de estar
a vossa posteridade e o vosso nome. E será que desde uma lua
nova até à outra, e desde um sábado até ao outro, virá toda a carne
a adorar perante mim, diz o Senhor. E sairão, e verão os
cadáveres dos homens que prevaricaram contra mim; porque o seu verme
nunca morrerá, nem o seu fogo se apagará; e serão um horror a toda a
carne.

