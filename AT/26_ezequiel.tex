\addchap{Ezequiel}

\lettrine{1} E aconteceu no trigésimo ano, no quarto mês, no
quinto dia do mês, que estando eu no meio dos cativos, junto ao rio
Quebar, se abriram os céus, e eu tive visões de Deus. No quinto
dia do mês, no quinto ano do cativeiro do rei Jeoiaquim, veio
expressamente a palavra do Senhor a Ezequiel, filho de Buzi, o
sacerdote, na terra dos caldeus, junto ao rio Quebar, e ali esteve
sobre ele a mão do Senhor.

Olhei, e eis que um vento tempestuoso vinha do norte, uma grande
nuvem, com um fogo revolvendo-se nela, e um resplendor ao redor, e
no meio dela havia uma coisa, como de cor de âmbar\footnote{Que tem
um tom entre o acastanhado e o amarelado. Diz-se dessa cor. A cor do
âmbar-amarelo. Por analogia: cheiro suave; aroma.}, que saía do meio
do fogo. E do meio dela saía a semelhança de quatro seres
viventes. E esta era a sua aparência: tinham a semelhança de homem.
E cada um tinha quatro rostos, como também cada um deles quatro
asas. E os seus pés eram pés direitos; e as plantas dos seus pés
como a planta do pé de uma bezerra, e luziam como a cor de cobre
polido. E tinham mãos de homem debaixo das suas asas, aos quatro
lados; e assim todos quatro tinham seus rostos e suas asas.
Uniam-se as suas asas uma à outra; não se viravam quando
andavam, e cada qual andava continuamente em frente. E a
semelhança dos seus rostos era como o rosto de homem; e do lado
direito todos os quatro tinham rosto de leão, e do lado esquerdo
todos os quatro tinham rosto de boi; e também tinham rosto de águia
todos os quatro. Assim eram os seus rostos. As suas asas
estavam estendidas por cima; cada qual tinha duas asas juntas uma a
outra, e duas cobriam os corpos deles. E cada qual andava
para adiante de si; para onde o espírito havia de ir, iam; não se
viravam quando andavam. E, quanto à semelhança dos seres
viventes, o seu aspecto era como ardentes brasas de fogo, com uma
aparência de lâmpadas; o fogo subia e descia por entre os seres
viventes, e o fogo resplandecia, e do fogo saíam relâmpagos;
e os seres viventes corriam, e voltavam, à semelhança de um
clarão de relâmpago.

E vi os seres viventes; e eis que havia uma roda sobre a terra
junto aos seres viventes, uma para cada um dos quatro rostos.
O aspecto das rodas, e a obra delas, era como a cor de
berilo\footnote{Silicato de berílio e alumínio hexagonal, prismático
ou tabular, que forma cristais com até vários metros de comprimento
(Apresenta diversas variedades de interesse gemológico, como
esmeralda, água-marinha etc., além de ser a principal fonte do
berílio.)}; e as quatro tinham uma mesma semelhança; e o seu
aspecto, e a sua obra, era como se estivera uma roda no meio de
outra roda. Andando elas, andavam pelos seus quatro lados ;
não se viravam quando andavam. E os seus aros eram tão altos,
que faziam medo; e estas quatro tinham as suas
cambotas\footnote{Armação arqueada, de madeira, que serve de molde e
de suporte para a construção de arcos e abóbadas; címbrio, gambota.
Peça, ger. de madeira, em arco, que assenta horizontalmente no alto
de nichos, altares etc. para servir de apoio ao dossel. Camba (`peça
curva de roda de veículo').} cheias de olhos ao redor. E,
andando os seres viventes, andavam as rodas ao lado deles; e,
elevando-se os seres viventes da terra, elevavam-se também as rodas.
Para onde o espírito queria ir, eles iam; para onde o
espírito tinha de ir; e as rodas se elevavam defronte deles, porque
o espírito do ser vivente estava nas rodas. Andando eles,
andavam elas e, parando eles, paravam elas e, elevando-se eles da
terra, elevavam-se também as rodas defronte deles; porque o espírito
do ser vivente estava nas rodas. E sobre as cabeças dos seres
viventes havia uma semelhança de firmamento, com a aparência de
cristal terrível, estendido por cima, sobre as suas cabeças.
E debaixo do firmamento estavam as suas asas direitas uma em
direção à outra; cada um tinha duas, que lhe cobriam o corpo de um
lado; e cada um tinha outras duas asas, que os cobriam do outro
lado. E, andando eles, ouvi o ruído das suas asas, como o
ruído de muitas águas, como a voz do Onipotente, um tumulto como o
estrépito de um exército; parando eles, abaixavam as suas asas.
E ouviu-se uma voz vinda do firmamento, que estava por cima
das suas cabeças; parando eles, abaixavam as suas asas.

E por cima do firmamento, que estava por cima das suas cabeças,
havia algo semelhante a um trono que parecia de pedra de safira; e
sobre esta espécie de trono havia uma figura semelhante à de um
homem, na parte de cima, sobre ele. E vi-a como a cor de
âmbar, como a aparência do fogo pelo interior dele ao redor, desde o
aspecto dos seus lombos, e daí para cima; e, desde o aspecto dos
seus lombos e daí para baixo, vi como a semelhança de fogo, e um
resplendor ao redor dele. Como o aspecto do arco que aparece
na nuvem no dia da chuva, assim era o aspecto do resplendor em
redor. Este era o aspecto da semelhança da glória do Senhor; e,
vendo isto, caí sobre o meu rosto, e ouvi a voz de quem falava.

\medskip

\lettrine{2} E disse-me: Filho do homem, põe-te em pé, e
falarei contigo. Então entrou em mim o Espírito, quando ele
falava comigo, e me pôs em pé, e ouvi o que me falava. E
disse-me: Filho do homem, eu te envio aos filhos de Israel, às
nações rebeldes que se rebelaram contra mim; eles e seus pais
transgrediram contra mim até este mesmo dia. E os filhos são de
semblante duro, e obstinados de coração; eu te envio a eles, e lhes
dirás: Assim diz o Senhor Deus. E eles, quer ouçam quer deixem
de ouvir (porque eles são casa rebelde), hão de saber, contudo, que
esteve no meio deles um profeta.

E tu, ó filho do homem, não os temas, nem temas as suas palavras;
ainda que estejam contigo sarças e espinhos, e tu habites entre
escorpiões, não temas as suas palavras, nem te assustes com os seus
semblantes, porque são casa rebelde. Mas tu lhes dirás as minhas
palavras, quer ouçam quer deixem de ouvir, pois são rebeldes.
Mas tu, ó filho do homem, ouve o que eu te falo, não sejas
rebelde como a casa rebelde; abre a tua boca, e come o que eu te
dou. Então vi, e eis que uma mão se estendia para mim, e eis que
nela havia um rolo de livro. E estendeu-o diante de mim, e
ele estava escrito por dentro e por fora; e nele estavam escritas
lamentações, e suspiros e ais.

\medskip

\lettrine{3} Depois me disse: Filho do homem, come o que
achares; come este rolo, e vai, fala à casa de Israel. Então
abri a minha boca, e me deu a comer o rolo. E disse-me: Filho do
homem, dá de comer ao teu ventre, e enche as tuas entranhas deste
rolo que eu te dou. Então o comi, e era na minha boca doce como o
mel. E disse-me ainda: Filho do homem, vai, entra na casa de
Israel, e dize-lhe as minhas palavras. Porque tu não és enviado
a um povo de estranha fala, nem de língua difícil, mas à casa de
Israel; nem a muitos povos de estranha fala, e de língua
difícil, cujas palavras não possas entender; se eu aos tais te
enviara, certamente te dariam ouvidos. Mas a casa de Israel não
te quererá dar ouvidos, porque não me querem dar ouvidos a mim; pois
toda a casa de Israel é de fronte obstinada e dura de coração.
Eis que fiz duro o teu rosto contra os seus rostos, e forte a
tua fronte contra a sua fronte. Fiz como diamante a tua fronte,
mais forte do que a pederneira\footnote{Sílex (rocha dura, de grão
muito fino e cor variável, freq. amarelo-clara, pardacenta ou negra,
composta de sílica mais ou menos cristalizada sob a forma de
calcedônia ou de quartzo, hidratada em maior ou menor teor,
encontrada sob a forma de concreções em outras rochas, ger.
calcárias) pirômaco (que ou corpo que, percutido pelo ferro, produz
centelhas), capaz de produzir centelhas quando percutido ou atritado
por peças de metal, esp. ferro (Us. em peças antigas de artilharia,
espingardas, isqueiros etc.).}; não os temas, pois, nem te assombres
com os seus rostos, porque são casa rebelde. Disse-me mais:
Filho do homem, recebe no teu coração todas as minhas palavras que
te hei de dizer, e ouve-as com os teus ouvidos. Eia, pois,
vai aos do cativeiro, aos filhos do teu povo, e lhes falarás e lhes
dirás: Assim diz o Senhor Deus, quer ouçam quer deixem de ouvir.
E levantou-me o Espírito, e ouvi por detrás de mim uma voz de
grande estrondo, que dizia: Bendita seja a glória do Senhor, desde o
seu lugar. E ouvi o ruído das asas dos seres viventes, que
tocavam umas nas outras, e o ruído das rodas defronte deles, e o
sonido\footnote{Qualquer som; ruído, rumor. Som muito forte;
estrépito, estrondo.} de um grande estrondo. Então o Espírito
me levantou, e me levou; e eu me fui amargurado, na indignação do
meu espírito; porém a mão do Senhor era forte sobre mim. E
fui a Tel-Abibe, aos do cativeiro, que moravam junto ao rio Quebar,
e eu morava onde eles moravam; e fiquei ali sete dias, pasmado no
meio deles.

E sucedeu que, ao fim de sete dias, veio a palavra do Senhor a
mim, dizendo: Filho do homem: Eu te dei por atalaia sobre a
casa de Israel; e tu da minha boca ouvirás a palavra e avisá-los-ás
da minha parte. Quando eu disser ao ímpio: Certamente
morrerás; e tu não o avisares, nem falares para avisar o ímpio
acerca do seu mau caminho, para salvar a sua vida, aquele ímpio
morrerá na sua iniqüidade, mas o seu sangue, da tua mão o
requererei. Mas, se avisares ao ímpio, e ele não se converter
da sua impiedade e do seu mau caminho, ele morrerá na sua
iniqüidade, mas tu livraste a tua alma. Semelhantemente,
quando o justo se desviar da sua justiça, e cometer a iniqüidade, e
eu puser diante dele um tropeço, ele morrerá: porque tu não o
avisaste, no seu pecado morrerá; e suas justiças, que tiver
praticado, não serão lembradas, mas o seu sangue, da tua mão o
requererei. Mas, avisando tu o justo, para que não peque, e
ele não pecar, certamente viverá; porque foi avisado; e tu livraste
a tua alma.

E a mão do Senhor estava sobre mim ali, e ele me disse:
Levanta-te, e sai ao vale, e ali falarei contigo. E
levantei-me, e saí ao vale, e eis que a glória do Senhor estava ali,
como a glória que vi junto ao rio Quebar; e caí sobre o meu rosto.
Então entrou em mim o Espírito, e me pôs em pé, e falou
comigo, e me disse: Entra, encerra-te dentro da tua casa. E
quanto a ti, ó filho do homem, eis que porão cordas sobre ti, e te
ligarão com elas; não sairás, pois, ao meio deles. E eu farei
que a tua língua se pegue ao teu paladar, e ficarás mudo, e não lhes
servirás de repreendedor; porque eles são casa rebelde. Mas,
quando eu falar contigo, abrirei a tua boca, e lhes dirás: Assim diz
o Senhor Deus: Quem ouvir ouça, e quem deixar de ouvir, deixe;
porque eles são casa rebelde.

\medskip

\lettrine{4} Tu, pois, ó filho do homem, toma um tijolo, e
pô-lo-ás diante de ti, e grava nele a cidade de Jerusalém. E põe
contra ela um cerco, e edifica contra ela uma fortificação, e
levanta contra ela uma trincheira, e põe contra ela
arraiais\footnote{Lugarejo de caráter provisório, temporário. No
caso: acampamento militar.}, e põe-lhe aríetes\footnote{Máquina de
guerra com que se derrubavam as muralhas ou as portas das cidades
sitiadas.} em redor. E tu toma uma sertã\footnote{Frigideira
rasa e ger. larga, de barro ou de ferro.} de ferro, e põe-na por
muro de ferro entre ti e a cidade; e dirige para ela o teu rosto, e
assim será cercada, e a cercarás; isto servirá de sinal à casa de
Israel. Tu também deita-te sobre o teu lado esquerdo, e põe a
iniqüidade da casa de Israel sobre ele; conforme o número dos dias
que te deitares sobre ele, levarás as suas iniqüidades. Porque
eu já te tenho fixado os anos da sua iniqüidade, conforme o número
dos dias, trezentos e noventa dias; e levarás a iniqüidade da casa
de Israel. E, quando tiveres cumprido estes dias, tornar-te-ás a
deitar sobre o teu lado direito, e levarás a iniqüidade da casa de
Judá quarenta dias; um dia te dei para cada ano. Dirigirás,
pois, o teu rosto para o cerco de Jerusalém, com o teu braço
descoberto, e profetizarás contra ela. E eis que porei sobre ti
cordas; assim tu não te voltarás de um lado para o outro, até que
cumpras os dias do teu cerco.

E tu, toma trigo, e cevada, e favas, e lentilhas, e milho e aveia,
e coloca-os numa vasilha, e faze deles pão; conforme o número dos
dias que tu te deitares sobre o teu lado, trezentos e noventa dias,
comerás disso. E a tua comida, que hás de comer, será do peso
de vinte siclos por dia; de tempo em tempo a comerás. Também
beberás a água por medida, a saber, a sexta parte de um him; de
tempo em tempo beberás. E o que comeres será como bolos de
cevada, e cozê-los-ás sobre o esterco que sai do homem, diante dos
olhos deles. E disse o Senhor: Assim comerão os filhos de
Israel o seu pão imundo, entre os gentios para onde os lançarei.
Então disse eu: Ah! Senhor Deus! Eis que a minha alma não foi
contaminada, pois desde a minha mocidade até agora, nunca comi
daquilo que morrer de si mesmo, ou que é despedaçado por feras; nem
carne abominável entrou na minha boca. E disse-me: Vê, dei-te
esterco de vacas, em lugar de esterco de homem; e sobre ele
prepararás o teu pão. Disse-me ainda: Filho do homem, eis que
eu quebrarei o sustento de pão em Jerusalém, e comerão o pão por
peso, e com ansiedade; e a água beberão por medida, e com espanto;
para que lhes falte o pão e a água, e se espantem uns com os
outros, e se consumam nas suas iniqüidades.

\medskip

\lettrine{5} E tu, ó filho do homem, toma uma faca afiada,
como navalha de barbeiro, e a farás passar pela tua cabeça e pela
tua barba; então tomarás uma balança de peso, e repartirás os
cabelos. Uma terça parte queimarás no fogo, no meio da cidade,
quando se cumprirem os dias do cerco; então tomarás outra terça
parte, e feri-la-ás com uma faca ao redor dela; e a outra terça
parte espalharás ao vento; porque desembainharei a espada atrás
deles. Também tomarás dali um pequeno número, e atá-los-ás nas
bordas do teu manto. E ainda destes tomarás alguns, e os
lançarás no meio do fogo e os queimarás a fogo; e dali sairá um fogo
contra toda a casa de Israel.

Assim diz o Senhor Deus: Esta é Jerusalém; coloquei-a no meio das
nações e das terras que estão ao redor dela. Ela, porém, mudou
em impiedade os meus juízos, mais do que as nações, e os meus
estatutos mais do que as terras que estão ao redor dela; porque
rejeitaram os meus juízos e os meus estatutos, e não andaram neles.
Portanto assim diz o Senhor Deus: Porque multiplicastes mais do
que as nações, que estão ao redor de vós, e não andastes nos meus
estatutos, nem guardastes os meus juízos, nem ainda procedestes
segundo os juízos das nações que estão ao redor de vós; por isso
assim diz o Senhor Deus: Eis que eu, sim eu, estou contra ti; e
executarei juízos no meio de ti aos olhos das nações. E farei em
ti o que nunca fiz, e o que jamais farei, por causa de todas as tuas
abominações. Portanto os pais comerão a seus filhos no meio
de ti, e os filhos comerão a seus pais; e executarei em ti juízos, e
tudo o que restar de ti, espalharei a todos os ventos.
Portanto, como eu vivo, diz o Senhor Deus, certamente,
porquanto profanaste o meu santuário com todas as tuas coisas
detestáveis, e com todas as tuas abominações, também eu te
diminuirei, e o meu olho não te perdoará, nem também terei piedade.
Uma terça parte de ti morrerá de peste, e se consumirá de
fome no meio de ti; e outra terça parte cairá à espada em redor de
ti; e a outra terça parte espalharei a todos os ventos, e
desembainharei a espada atrás deles. Assim se cumprirá a
minha ira, e satisfarei neles o meu furor, e me consolarei; e
saberão que eu, o Senhor, tenho falado no meu zelo, quando eu
cumprir neles o meu furor. E pôr-te-ei em desolação, e por
objeto de opróbrio entre as nações que estão em redor de ti, aos
olhos de todos os que passarem. E será objeto de opróbrio e
blasfêmia, instrução e espanto às nações que estão em redor de ti,
quando eu executar em ti juízos com ira, e com furor, e com
terríveis castigos. Eu, o Senhor, falei. Quando eu enviar as
malignas flechas da fome contra eles, que servirão para destruição,
as quais eu mandarei para vos destruir, então aumentarei a fome
sobre vós, e vos quebrarei o sustento do pão. E enviarei
sobre vós a fome, e as feras que te desfilharão; e a peste e o
sangue passarão por ti; e trarei a espada sobre ti. Eu, o Senhor,
falei.

\medskip

\lettrine{6} E veio a mim a palavra do Senhor, dizendo:
Filho do homem, dirige o teu rosto para os montes de Israel, e
profetiza contra eles. E dirás: Montes de Israel, ouvi a palavra
do Senhor Deus: Assim diz o Senhor Deus aos montes, aos outeiros,
aos ribeiros e aos vales: Eis que eu, sim eu, trarei a espada sobre
vós, e destruirei os vossos lugares altos. E serão assolados os
vossos altares, e quebradas as vossas imagens do sol e derrubarei os
vossos mortos, diante dos vossos ídolos. E porei os cadáveres
dos filhos de Israel diante dos seus ídolos; e espalharei os vossos
ossos em redor dos vossos altares. Em todos os vossos lugares
habitáveis, as cidades serão destruídas, e os lugares altos
assolados; para que os vossos altares sejam destruídos e assolados,
e os vossos ídolos se quebrem e se acabem, e as vossas imagens sejam
cortadas, e desfeitas as vossas obras. E os mortos cairão no
meio de vós, para que saibais que eu sou o Senhor.

Porém deixarei um remanescente, para que tenhais entre as nações
alguns que escaparem da espada, quando fordes espalhados pelas
terras. Então os que dentre vós escaparem se lembrarão de mim
entre as nações para onde foram levados em cativeiro; porquanto me
quebrantei por causa do seu coração corrompido, que se desviou de
mim, e por causa dos seus olhos, que andaram se corrompendo após os
seus ídolos; e terão nojo de si mesmos, por causa das maldades que
fizeram em todas as suas abominações. E saberão que eu sou o
Senhor, e que não disse debalde que lhes faria este mal.

Assim diz o Senhor Deus: Bate com a mão, e bate com o teu pé, e
dize: Ah! Por todas as grandes abominações da casa de Israel! Porque
cairão à espada, e de fome, e de peste. O que estiver longe
morrerá de peste, e o que está perto cairá à espada; e o que restar
e ficar cercado morrerá de fome; assim cumprirei o meu furor sobre
eles. Então sabereis que eu sou o Senhor, quando os seus
mortos estiverem no meio dos seus ídolos, em redor dos seus altares,
em todo o outeiro alto, em todos os cumes dos montes, e debaixo de
toda a árvore verde, e debaixo de todo o carvalho frondoso, no lugar
onde ofereciam cheiro suave a todos os seus ídolos. E
estenderei a minha mão sobre eles, e farei a terra desolada, e mais
devastada do que o deserto do lado de Dibla, em todas as suas
habitações; e saberão que eu sou o Senhor.

\medskip

\lettrine{7} Depois veio a mim a palavra do Senhor, dizendo:
E tu, ó filho do homem, assim diz o Senhor Deus acerca da terra
de Israel: Vem o fim, o fim vem sobre os quatro cantos da terra.
Agora vem o fim sobre ti, e enviarei sobre ti a minha ira, e te
julgarei conforme os teus caminhos, e trarei sobre ti todas as tuas
abominações. E não te poupará o meu olho, nem terei piedade de
ti, mas porei sobre ti os teus caminhos, e as tuas abominações
estarão no meio de ti; e sabereis que eu sou o Senhor. Assim diz
o Senhor Deus: Um mal, eis que um só mal vem. Vem o fim, o fim
vem, despertou-se contra ti; eis que vem. A manhã vem para ti, ó
habitante da terra. Vem o tempo; chegado é o dia da
turbação\footnote{Perturbação, desordem, tumulto.}, e não mais o
sonido de alegria dos montes. Agora depressa derramarei o meu
furor sobre ti, e cumprirei a minha ira contra ti, e te julgarei
conforme os teus caminhos, e porei sobre ti todas as tuas
abominações. E não te poupará o meu olho, nem terei piedade de
ti; conforme os teus caminhos, assim te punirei, e as tuas
abominações estarão no meio de ti; e sabereis que eu, o Senhor, é
que firo. Eis aqui o dia, eis que vem; veio a manhã, já
floresceu a vara, já reverdeceu\footnote{Reverdecer (além de: tornar
verde (a vegetação); cobrir-se de verde): dar nova força ou vigor
(a); rejuvenescer(-se), revitalizar(-se). Tomar ou ganhar novo
impulso, nova força; renascer. Trazer à lembrança; relembrar,
rememorar.} a soberba. A violência se levantou em vara de
impiedade; nada restará deles, nem da sua multidão, nem do seu
rumor, nem haverá lamentação por eles. Vem o tempo, é chegado
o dia; o que compra não se alegre, e o que vende não se entristeça;
porque a ira ardente está sobre toda a multidão deles. Porque
o que vende não tornará a possuir o que vendeu, ainda que esteja
entre os viventes; porque a visão, sobre toda a sua multidão, não
tornará para trás, nem ninguém fortalecerá a sua vida com a sua
iniqüidade. Já tocaram a trombeta, e tudo prepararam, mas não
há quem vá à peleja, porque a minha ardente ira está sobre toda a
sua multidão. Fora está a espada, e dentro a peste e a fome;
o que estiver no campo morrerá à espada, e o que estiver na cidade a
fome e a peste o consumirão.

E escaparão os que fugirem deles, mas estarão pelos montes, como
pombas dos vales, todos gemendo, cada um por causa da sua
iniqüidade. Todas as mãos se enfraquecerão, e todos os
joelhos serão débeis como água. E cingir-se-ão de sacos, e o
terror os cobrirá; e sobre todos os rostos haverá vergonha, e sobre
todas as suas cabeças, calva. A sua prata lançarão pelas
ruas, e o seu ouro será removido; nem a sua prata nem o seu ouro os
poderá livrar no dia do furor do Senhor; eles não fartarão a sua
alma, nem lhes encherão o estômago, porque isto foi o tropeço da sua
iniqüidade. E a glória do seu ornamento ele a pôs em
magnificência, mas eles fizeram nela imagens das suas abominações e
coisas detestáveis; por isso eu lha tenho feito coisa imunda.
E entregá-la-ei por presa, na mão dos estrangeiros, e aos
ímpios da terra por despojo; e a profanarão. E desviarei
deles o meu rosto, e profanarão o meu lugar oculto; porque entrarão
nele saqueadores, e o profanarão.

Faze uma cadeia, porque a terra está cheia de crimes de sangue, e
a cidade está cheia de violência. E farei vir os piores
dentre os gentios e possuirão as suas casas; e farei cessar a
arrogância dos fortes, e os seus lugares santos serão profanados.
Vem a destruição; eles buscarão a paz, mas não há nenhuma.
Miséria sobre miséria virá, e se levantará rumor sobre rumor;
então buscarão do profeta uma visão, mas do sacerdote perecerá a lei
e dos anciãos o conselho. O rei lamentará, e o príncipe se
vestirá de desolação, e as mãos do povo da terra se conturbarão;
conforme o seu caminho lhes farei, e conforme os seus merecimentos
os julgarei; e saberão que eu sou o Senhor.

\medskip

\lettrine{8} Sucedeu, pois, no sexto ano, no sexto mês, no
quinto dia do mês, estando eu assentado na minha casa, e os anciãos
de Judá assentados diante de mim, que ali a mão do Senhor Deus caiu
sobre mim. E olhei, e eis uma semelhança como o aspecto de fogo;
desde o aspecto dos seus lombos, e daí para baixo, era fogo; e dos
seus lombos e daí para cima como o aspecto de um resplendor como a
cor de âmbar. E estendeu a forma de uma mão, e tomou-me pelos
cabelos da minha cabeça; e o Espírito me levantou entre a terra e o
céu, e levou-me a Jerusalém em visões de Deus, até à entrada da
porta do pátio de dentro, que olha para o norte, onde estava o
assento da imagem do ciúmes, que provoca ciúmes. E eis que a
glória do Deus de Israel estava ali, conforme o aspecto que eu tinha
visto no vale. E disse-me: Filho do homem, levanta agora os teus
olhos para o caminho do norte. E levantei os meus olhos para o
caminho do norte, e eis que ao norte da porta do altar estava esta
imagem de ciúmes na entrada. E disse-me: Filho do homem, vês tu
o que eles estão fazendo? As grandes abominações que a casa de
Israel faz aqui, para que me afaste do meu santuário? Mas ainda
tornarás a ver maiores abominações.

E levou-me à porta do átrio; então olhei, e eis que havia um
buraco na parede. E disse-me: Filho do homem, cava agora naquela
parede. E cavei na parede, e eis que havia uma porta. Então me
disse: Entra, e vê as malignas abominações que eles fazem aqui.
E entrei, e olhei, e eis que toda a forma de répteis, e
animais abomináveis, e de todos os ídolos da casa de Israel, estavam
pintados na parede em todo o redor. E estavam em pé diante
deles setenta homens dos anciãos da casa de Israel, e Jaazanias,
filho de Safã, em pé, no meio deles, e cada um tinha na mão o seu
incensário; e subia uma espessa nuvem de incenso. Então me
disse: Viste, filho do homem, o que os anciãos da casa de Israel
fazem nas trevas, cada um nas suas câmaras pintadas de imagens? Pois
dizem: O Senhor não nos vê; o Senhor abandonou a terra.

E disse-me: Ainda tornarás a ver maiores abominações, que estes
fazem. E levou-me à entrada da porta da casa do Senhor, que
está do lado norte, e eis que estavam ali mulheres assentadas
chorando a Tamuz. E disse-me: Vês isto, filho do homem? Ainda
tornarás a ver abominações maiores do que estas. E levou-me
para o átrio interior da casa do Senhor, e eis que estavam à entrada
do templo do Senhor, entre o pórtico e o altar, cerca de vinte e
cinco homens, de costas para o templo do Senhor, e com os rostos
para o oriente; e eles, virados para o oriente adoravam o sol.
Então me disse: Vês isto, filho do homem? Há porventura coisa
mais leviana para a casa de Judá do que tais abominações que fazem
aqui? Havendo enchido a terra de violência, tornam a irritar-me; e
ei-los a chegar o ramo ao seu nariz. Por isso também eu os
tratarei com furor; o meu olho não poupará, nem terei piedade; ainda
que me gritem aos ouvidos com grande voz, contudo não os ouvirei.

\medskip

\lettrine{9} Então me gritou aos ouvidos com grande voz,
dizendo: Fazei chegar os intendentes\footnote{Antigo magistrado
superior da polícia. Que ou aquele que dirige ou administra alguma
coisa. Diz-se de ou funcionário ou oficial das forças armadas
encarregado de questões administrativas e materiais, bem como da
contabilidade. Que ou aquele que é contratado para dirigir a
propriedade de outra pessoa e administrar seus bens; administrador.}
da cidade, cada um com as suas armas destruidoras na mão. E eis
que vinham seis homens a caminho da porta superior, que olha para o
norte, e cada um com a sua arma destruidora na mão, e entre eles um
homem vestido de linho, com um tinteiro de escrivão à sua cintura; e
entraram, e se puseram junto ao altar de bronze. E a glória do
Deus de Israel se levantou de sobre o querubim, sobre o qual estava,
indo até a entrada da casa; e clamou ao homem vestido de linho, que
tinha o tinteiro de escrivão à sua cintura. E disse-lhe o
Senhor: Passa pelo meio da cidade, pelo meio de Jerusalém, e marca
com um sinal as testas dos homens que suspiram e que gemem por causa
de todas as abominações que se cometem no meio dela.

E aos outros disse ele, ouvindo eu: Passai pela cidade após ele, e
feri; não poupe o vosso olho, nem vos compadeçais. Matai velhos,
jovens, virgens, meninos e mulheres, até exterminá-los; mas a todo o
homem que tiver o sinal não vos chegueis; e começai pelo meu
santuário. E começaram pelos homens mais velhos que estavam diante
da casa. E disse-lhes: Contaminai a casa e enchei os átrios de
mortos; saí. E saíram, e feriram na cidade. Sucedeu, pois, que,
havendo-os ferido, e ficando eu sozinho, caí sobre a minha face, e
clamei, e disse: Ah! Senhor Deus! dar-se-á caso que destruas todo o
restante de Israel, derramando a tua indignação sobre Jerusalém?
Então me disse: A maldade da casa de Israel e de Judá é
grandíssima, e a terra se encheu de sangue e a cidade se encheu de
perversidade; porque dizem: O Senhor abandonou a terra, e o Senhor
não vê. Pois, também, quanto a mim, não poupará o meu olho,
nem me compadecerei; sobre a cabeça deles farei recair o seu
caminho. Eis que o homem que estava vestido de linho, a cuja
cintura estava o tinteiro, tornou com a resposta, dizendo: Fiz como
me mandaste.

\medskip

\lettrine{10} Depois olhei, e eis que no firmamento, que
estava por cima da cabeça dos querubins, apareceu sobre eles uma
como pedra de safira, semelhante a forma de um trono. E falou ao
homem vestido de linho, dizendo: Vai por entre as rodas, até debaixo
do querubim, e enche as tuas mãos de brasas acesas dentre os
querubins e espalha-as sobre a cidade. E ele entrou à minha vista.
E os querubins estavam ao lado direito da casa, quando entrou
aquele homem; e uma nuvem encheu o átrio interior. Então se
levantou a glória do Senhor de sobre o querubim indo para a entrada
da casa; e encheu-se a casa de uma nuvem, e o átrio se encheu do
resplendor da glória do Senhor. E o ruído das asas dos querubins
se ouviu até ao átrio exterior, como a voz do Deus Todo-Poderoso,
quando fala. Sucedeu, pois, que, dando ele ordem ao homem
vestido de linho, dizendo: Toma fogo dentre as rodas, dentre os
querubins, entrou ele, e parou junto às rodas. Então estendeu um
querubim a sua mão dentre os querubins para o fogo que estava entre
os querubins; e tomou dele, e o pôs nas mãos do que estava vestido
de linho; o qual o tomou, e saiu.

E apareceu nos querubins uma semelhança de mão de homem debaixo
das suas asas. Então olhei, e eis quatro rodas junto aos
querubins, uma roda junto a um querubim, e outra roda junto a outro
querubim; e o aspecto das rodas era como a cor da pedra de berilo.
E, quanto ao seu aspecto, as quatro tinham uma mesma
semelhança; como se estivesse uma roda no meio de outra roda.
Andando estes, andavam para os quatro lados deles; não se
viravam quando andavam, mas para o lugar para onde olhava a cabeça,
para esse seguiam; não se viravam quando andavam. E todo o
seu corpo, as suas costas, as suas mãos, as suas asas e as rodas, as
rodas que os quatro tinham, estavam cheias de olhos ao redor.
E, quanto às rodas, ouvindo eu, se lhes gritava: Roda!
E cada um tinha quatro rostos; o rosto do primeiro era rosto
de querubim, e o rosto do segundo, rosto de homem, e do terceiro era
rosto de leão, e do quarto, rosto de águia. E os querubins se
elevaram ao alto; estes são os mesmos seres viventes que vi junto ao
rio Quebar. E, andando os querubins, andavam as rodas
juntamente com eles; e, levantando os querubins as suas asas, para
se elevarem de sobre a terra, também as rodas não se separavam
deles. Parando eles, paravam elas; e, elevando-se eles
elevavam-se elas, porque o espírito do ser vivente estava nelas.
Então saiu a glória do Senhor de sobre a entrada da casa, e
parou sobre os querubins. E os querubins alçaram as suas
asas, e se elevaram da terra aos meus olhos, quando saíram; e as
rodas os acompanhavam; e cada um parou à entrada da porta oriental
da casa do Senhor; e a glória do Deus de Israel estava em cima,
sobre eles. Estes são os seres viventes que vi debaixo do
Deus de Israel, junto ao rio Quebar, e conheci que eram querubins.
Cada um tinha quatro rostos e cada um quatro asas, e a
semelhança de mãos de homem debaixo das suas asas. E a
semelhança dos seus rostos era a dos rostos que eu tinha visto junto
ao rio Quebar, o aspecto deles, e eles mesmos; cada um andava para
diante do seu rosto.

\medskip

\lettrine{11} Então me levantou o Espírito, e me levou à porta
oriental da casa do Senhor, a qual olha para o oriente; e eis que
estavam à entrada da porta vinte e cinco homens; e no meio deles vi
a Jaazanias, filho de Azur, e a Pelatias, filho de Benaia, príncipes
do povo. E disse-me: Filho do homem, estes são os homens que
maquinam perversidade, e dão mau conselho nesta cidade. Os quais
dizem: Não está próximo o tempo de edificar casas; esta cidade é o
caldeirão, e nós a carne. Portanto, profetiza contra eles;
profetiza, ó filho do homem. Caiu, pois, sobre mim o Espírito do
Senhor, e disse-me: Fala: Assim diz o Senhor: Assim haveis falado, ó
casa de Israel, porque, quanto às coisas que vos sobem ao espírito,
eu as conheço. Multiplicastes os vossos mortos nesta cidade, e
enchestes as suas ruas de mortos. Portanto, assim diz o Senhor
Deus: Vossos mortos, que deitastes no meio dela, esses são a carne e
ela é o caldeirão; a vós, porém, vos tirarei do meio dela.
Temestes a espada, e a espada trarei sobre vós, diz o Senhor
Deus. E vos farei sair do meio dela, e vos entregarei na mão de
estrangeiros, e exercerei os meus juízos entre vós. Caireis à
espada, e nos confins de Israel vos julgarei; e sabereis que eu sou
o Senhor. Esta cidade não vos servirá de caldeirão, nem vós
servireis de carne no meio dela; nos confins de Israel vos julgarei.
E sabereis que eu sou o Senhor, porque não andastes nos meus
estatutos, nem cumpristes os meus juízos; antes fizestes conforme os
juízos dos gentios que estão ao redor de vós. E aconteceu
que, profetizando eu, morreu Pelatias, filho de Benaia; então caí
sobre o meu rosto, e clamei com grande voz, e disse: Ah! Senhor
Deus! Porventura darás tu fim ao remanescente de Israel?

Então veio a mim a palavra do Senhor, dizendo: Filho do
homem, teus irmãos, sim, teus irmãos, os homens de teu parentesco, e
toda a casa de Israel, todos eles são aqueles a quem os habitantes
de Jerusalém disseram: Apartai-vos para longe do Senhor; esta terra
nos foi dada em possessão. Portanto, dize: Assim diz o Senhor
Deus: Ainda que os lancei para longe entre os gentios, e ainda que
os espalhei pelas terras, todavia lhes serei como um pequeno
santuário, nas terras para onde forem. Portanto, dize: Assim
diz o Senhor Deus: Hei de ajuntar-vos do meio dos povos, e vos
recolherei das terras para onde fostes lançados, e vos darei a terra
de Israel. E virão ali, e tirarão dela todas as suas coisas
detestáveis e todas as suas abominações. E lhes darei um só
coração, e um espírito novo porei dentro deles; e tirarei da sua
carne o coração de pedra, e lhes darei um coração de carne;
para que andem nos meus estatutos, e guardem os meus juízos,
e os cumpram; e eles me serão por povo, e eu lhes serei por Deus.
Mas, quanto àqueles cujo coração andar conforme o coração das
suas coisas detestáveis, e as suas abominações, farei recair nas
suas cabeças o seu caminho, diz o Senhor Deus.

Então os querubins elevaram as suas asas, e as rodas os
acompanhavam; e a glória do Deus de Israel estava em cima sobre
eles. E a glória do Senhor se alçou desde o meio da cidade; e
se pôs sobre o monte que está ao oriente da cidade. Depois o
Espírito me levantou, e me levou à Caldéia, para os do cativeiro, em
visão, pelo Espírito de Deus; e subiu de sobre mim a visão que eu
tinha tido. E falei aos do cativeiro todas as coisas que o
Senhor me havia mostrado.

\medskip

\lettrine{12} E veio a mim a palavra do Senhor, dizendo:
Filho do homem, tu habitas no meio da casa rebelde, que tem
olhos para ver e não vê, e tem ouvidos para ouvir e não ouve; porque
eles são casa rebelde. Tu, pois, ó filho do homem, prepara
mobílias para mudares, e de dia muda aos olhos deles; e do teu lugar
mudarás para outro lugar aos olhos deles; bem pode ser que reparem
nisso, ainda que eles são casa rebelde. Aos olhos deles, pois,
tirarás para fora, de dia, as tuas mobílias, como quem vai mudar;
então tu sairás de tarde aos olhos deles, como quem sai mudando para
o cativeiro. Faze para ti, à vista deles, uma abertura na
parede, e tira-as para fora, por ali. Aos olhos deles, nos seus
ombros, às escuras as tirarás, e cobrirás o teu rosto, para que não
vejas a terra; porque te dei por sinal à casa de Israel. E fiz
assim, como se me deu ordem; as minhas mobílias tirei para fora de
dia, como mobílias do cativeiro; então à tarde fiz, com a mão, uma
abertura na parede; às escuras as tirei para fora, e nos meus ombros
as levei, aos olhos deles. E, pela manhã, veio a mim a palavra
do Senhor, dizendo: Filho do homem, porventura não te disse a
casa de Israel, aquela casa rebelde: Que fazes tu? Dize-lhes:
Assim diz o Senhor Deus: Esta carga refere-se ao príncipe em
Jerusalém, e a toda a casa de Israel, que está no meio dela.
Dize: Eu sou o vosso sinal. Assim como eu fiz, assim se lhes
fará a eles; irão para o exílio em cativeiro. E o príncipe
que está no meio deles levará aos ombros as mobílias, e às escuras
sairá; farão uma abertura na parede para as tirarem por ela; o seu
rosto cobrirá, para que com os seus olhos não veja a terra.
Também estenderei a minha rede sobre ele, e será apanhado no
meu laço; e o levarei à Babilônia, à terra dos caldeus, e contudo
não a verá, ainda que ali morrerá. E a todos os ventos
espalharei os que estiverem ao redor dele para seu socorro, e a
todas as suas tropas; e desembainharei a espada atrás deles.
Assim saberão que eu sou o Senhor, quando eu os dispersar
entre as nações e os espalhar pelas terras. Mas deles
deixarei ficar alguns poucos, escapos da espada, da fome, e da
peste, para que contem todas as suas abominações entre as nações
para onde forem; e saberão que eu sou o Senhor.

Então veio a mim a palavra do Senhor, dizendo: Filho do
homem, o teu pão comerás com tremor, e a tua água beberás com
estremecimento e com receio. E dirás ao povo da terra: Assim
diz o Senhor Deus acerca dos habitantes de Jerusalém, na terra de
Israel: O seu pão comerão com receio, e a sua água beberão com
susto, pois a sua terra será despojada de sua abundância, por causa
da violência de todos os que nela habitam. E as cidades
habitadas serão devastadas, e a terra se tornará em desolação; e
sabereis que eu sou o Senhor.

E veio ainda a mim a palavra do Senhor, dizendo: Filho do
homem, que provérbio é este que vós tendes na terra de Israel,
dizendo: Prolongar-se-ão os dias, e perecerá toda a visão?
Portanto, dize-lhes: Assim diz o Senhor Deus: Farei cessar
este provérbio, e já não se servirão mais dele em Israel; mas
dize-lhes: Os dias estão próximos e o cumprimento de toda a visão.
Porque não haverá mais alguma visão vã, nem adivinhação
lisonjeira, no meio da casa de Israel. Porque eu, o Senhor,
falarei, e a palavra que eu falar se cumprirá; não será mais adiada;
porque em vossos dias, ó casa rebelde, falarei uma palavra e a
cumprirei, diz o Senhor Deus. Veio mais a mim a palavra do
Senhor, dizendo: Filho do homem, eis que os da casa de Israel
dizem: A visão que este tem é para muitos dias, e ele profetiza de
tempos que estão longe. Portanto dize-lhes: Assim diz o
Senhor Deus: Não será mais adiada nenhuma das minhas palavras; e a
palavra que falei se cumprirá, diz o Senhor Deus.

\medskip

\lettrine{13} E veio a mim a palavra do Senhor, dizendo:
Filho do homem, profetiza contra os profetas de Israel que
profetizam, e dize aos que só profetizam de seu coração: Ouvi a
palavra do Senhor; assim diz o Senhor Deus: Ai dos profetas
loucos, que seguem o seu próprio espírito e que nada viram! Os
teus profetas, ó Israel, são como raposas nos desertos. Não
subistes às brechas, nem reparastes o muro para a casa de Israel,
para estardes firmes na peleja no dia do Senhor. Viram vaidade e
adivinhação mentirosa os que dizem: O Senhor disse; quando o Senhor
não os enviou; e fazem que se espere o cumprimento da palavra.
Porventura não tivestes visão de vaidade, e não falastes
adivinhação mentirosa, quando dissestes: O Senhor diz, sendo que eu
tal não falei? Portanto assim diz o Senhor Deus: Como tendes
falado vaidade, e visto a mentira, portanto eis que eu sou contra
vós, diz o Senhor Deus. E a minha mão será contra os profetas
que vêem vaidade e que adivinham mentira; não estarão na congregação
do meu povo, nem nos registros da casa de Israel se escreverão, nem
entrarão na terra de Israel; e sabereis que eu sou o Senhor Deus.

Porquanto, sim, porquanto andam enganando o meu povo, dizendo:
Paz, não havendo paz; e quando um edifica uma parede, eis que outros
a cobrem com argamassa não temperada; dize aos que a cobrem
com argamassa não temperada que ela cairá. Haverá uma grande pancada
de chuva, e vós, ó pedras grandes de saraiva, caireis, e um vento
tempestuoso a fenderá. Ora, eis que, caindo a parede, não vos
dirão: Onde está a argamassa com que a cobristes? Portanto
assim diz o Senhor Deus: Fendê-la-ei no meu furor com vento
tempestuoso, e chuva de inundar haverá na minha ira, e grandes
pedras de saraiva na minha indignação, para a consumir. E
derrubarei a parede que cobristes com argamassa não temperada, e
darei com ela por terra, e o seu fundamento se descobrirá; assim
cairá, e perecereis no meio dela, e sabereis que eu sou o Senhor.
Assim cumprirei o meu furor contra a parede, e contra os que
a cobriram com argamassa não temperada; e vos direi: Já não há
parede, nem existem os que a cobriram; os profetas de Israel,
que profetizam acerca de Jerusalém, e vêem para ela visão de paz,
não havendo paz, diz o Senhor Deus.

E tu, ó filho do homem, dirige o teu rosto contra as filhas do
teu povo, que profetizam de seu coração, e profetiza contra elas,
e dize: Assim diz o Senhor Deus: Ai das que cosem almofadas
para todas as axilas, e que fazem véus para as cabeças de pessoas de
toda a estatura, para caçarem as almas! Porventura caçareis as almas
do meu povo, e as almas guardareis em vida para vós? E vós me
profanastes entre o meu povo, por punhados de cevada, e por pedaços
de pão, para matardes as almas que não haviam de morrer, e para
guardardes em vida as almas que não haviam de viver, mentindo assim
ao meu povo que escuta a mentira? Portanto assim diz o Senhor
Deus: Eis aí vou eu contra as vossas almofadas, com que vós ali
caçais as almas fazendo-as voar, e as arrancarei de vossos braços, e
soltarei as almas, sim, as almas que vós caçais fazendo-as voar.
E rasgarei os vossos véus, e livrarei o meu povo das vossas
mãos, e nunca mais estará em vossas mãos para ser caçado; e sabereis
que eu sou o Senhor. Visto que entristecestes o coração do
justo com falsidade, não o havendo eu entristecido; e fortalecestes
as mãos do ímpio, para que não se desviasse do seu mau caminho, para
conservá-lo em vida. Portanto não vereis mais vaidade, nem
mais fareis adivinhações; mas livrarei o meu povo da vossa mão, e
sabereis que eu sou o Senhor.

\medskip

\lettrine{14} E vieram a mim alguns homens dos anciãos de
Israel, e se assentaram diante de mim. Então veio a mim a
palavra do Senhor, dizendo: Filho do homem, estes homens
levantaram os seus ídolos nos seus corações, e o tropeço da sua
maldade puseram diante da sua face; devo eu de alguma maneira ser
interrogado por eles? Portanto fala com eles, e dize-lhes: Assim
diz o Senhor Deus: Qualquer homem da casa de Israel, que levantar os
seus ídolos no seu coração, e puser o tropeço da sua maldade diante
da sua face, e vier ao profeta, eu, o Senhor, vindo ele, lhe
responderei conforme a multidão dos seus ídolos; para que eu
possa apanhar a casa de Israel no seu coração, porquanto todos se
apartaram de mim para seguirem os seus ídolos. Portanto dize à
casa de Israel: Assim diz o Senhor Deus: Convertei-vos, e tornai-vos
dos vossos ídolos; e desviai os vossos rostos de todas as vossas
abominações; porque qualquer homem da casa de Israel, e dos
estrangeiros que peregrinam em Israel, que se alienar de mim, e
levantar os seus ídolos no seu coração, e puser o tropeço da sua
maldade diante do seu rosto, e vier ao profeta, para me consultar
por meio dele, eu, o Senhor, lhe responderei por mim mesmo. E
porei o meu rosto contra o tal homem, e o assolarei para que sirva
de sinal e provérbio, e arrancá-lo-ei do meio do meu povo; e
sabereis que eu sou o Senhor. E se o profeta for enganado, e
falar alguma coisa, eu, o Senhor, terei enganado esse profeta; e
estenderei a minha mão contra ele, e destruí-lo-ei do meio do meu
povo Israel. E levarão sobre si o castigo da sua iniqüidade;
o castigo do profeta será como o castigo de quem o consultar.
Para que a casa de Israel não se desvie mais de mim, nem mais
se contamine com todas as suas transgressões; então eles serão o meu
povo, e eu lhes serei o seu Deus, diz o Senhor Deus.

Veio ainda a mim a palavra do Senhor, dizendo:

Filho do homem, quando uma terra pecar contra mim, se rebelando
gravemente, então estenderei a minha mão contra ela, e lhe quebrarei
o sustento do pão, e enviarei contra ela fome, e cortarei dela
homens e animais. Ainda que estivessem no meio dela estes
três homens, Noé, Daniel e Jó, eles pela sua justiça livrariam
apenas as suas almas, diz o Senhor Deus. Se eu fizer passar
pela terra as feras selvagens, e elas a desfilharem de modo que
fique desolada, e ninguém possa passar por ela por causa das feras;
e estes três homens estivessem no meio dela, vivo eu, diz o
Senhor Deus, que nem a filhos nem a filhas livrariam; eles só
ficariam livres, e a terra seria assolada. Ou, se eu trouxer
a espada sobre aquela terra, e disser: Espada, passa pela terra; e
eu cortar dela homens e animais; ainda que aqueles três
homens estivessem nela, vivo eu, diz o Senhor Deus, que nem filhos
nem filhas livrariam, mas somente eles ficariam livres. Ou,
se eu enviar a peste sobre aquela terra, e derramar o meu furor
sobre ela com sangue, para cortar dela homens e animais,
ainda que Noé, Daniel e Jó estivessem no meio dela, vivo eu,
diz o Senhor Deus, que nem um filho nem uma filha eles livrariam,
mas somente eles livrariam as suas próprias almas pela sua justiça.
Porque assim diz o Senhor Deus: Quanto mais, se eu enviar os
meus quatro maus juízos, a espada, a fome, as feras, e a peste,
contra Jerusalém, para cortar dela homens e feras? Mas eis
que alguns fugitivos restarão nela, que serão levados para fora,
assim filhos e filhas; eis que eles virão a vós, e vereis o seu
caminho e os seus feitos; e ficareis consolados do mal que eu trouxe
sobre Jerusalém, e de tudo o que trouxe sobre ela. E sereis
consolados, quando virdes o seu caminho e os seus feitos; e sabereis
que não fiz sem razão tudo quanto nela tenho feito, diz o Senhor
Deus.

\medskip

\lettrine{15} E veio a mim a palavra do Senhor, dizendo:
Filho do homem, que mais é a árvore da videira do que qualquer
outra árvore, ou do que o sarmento\footnote{Ramo de videira.
Derivação: por extensão de sentido: qualquer ramo delgado, lenhoso e
flexível, com os nós ger. bem demarcados.} que está entre as árvores
do bosque? Toma-se dela madeira para fazer alguma obra? Ou
toma-se dela alguma estaca, para que se lhe pendure um vaso? Eis
que é lançado no fogo, para ser consumido; ambas as suas
extremidades consome o fogo, e o meio dela fica também queimado;
serviria porventura para alguma obra? Ora, se estando inteiro,
não servia para obra alguma, quanto menos sendo consumido pelo fogo,
e, sendo queimado, se faria ainda obra dele? Portanto, assim diz
o Senhor Deus: Como a árvore da videira entre as árvores do bosque,
que tenho entregue ao fogo para que seja consumido, assim entregarei
os habitantes de Jerusalém. E porei a minha face contra eles; do
fogo sairão, mas o fogo os consumirá; e sabereis que eu sou o
Senhor, quando tiver posto a minha face contra eles. E tornarei
a terra em desolação, porquanto grandemente transgrediram, diz o
Senhor Deus.

\medskip

\lettrine{16} E veio a mim outra vez a palavra do Senhor,
dizendo: Filho do homem, faze conhecer a Jerusalém as suas
abominações. E dize: Assim diz o Senhor Deus a Jerusalém: A tua
origem e o teu nascimento procedem da terra dos cananeus. Teu pai
era amorreu, e tua mãe hetéia. E, quanto ao teu nascimento, no
dia em que nasceste não te foi cortado o umbigo, nem foste lavada
com água para te limpar; nem tampouco foste esfregada com sal, nem
envolta em faixas. Não se apiedou de ti olho algum, para te
fazer alguma coisa disto, compadecendo-se de ti; antes foste lançada
em pleno campo, pelo nojo da tua pessoa, no dia em que nasceste.

E, passando eu junto de ti, vi-te a revolver-te no teu sangue, e
disse-te: Ainda que estejas no teu sangue, vive; sim, disse-te:
Ainda que estejas no teu sangue, vive. Eu te fiz multiplicar
como o renovo do campo, e cresceste, e te engrandeceste, e chegaste
à grande formosura; avultaram os seios, e cresceu o teu cabelo; mas
estavas nua e descoberta. E, passando eu junto de ti, vi-te, e
eis que o teu tempo era tempo de amores; e estendi sobre ti a aba do
meu manto, e cobri a tua nudez; e dei-te juramento, e entrei em
aliança contigo, diz o Senhor Deus, e tu ficaste sendo minha.
Então te lavei com água, e te enxuguei do teu sangue, e te ungi
com óleo. E te vesti com roupas bordadas, e te calcei com
pele de texugo, e te cingi com linho fino, e te cobri de seda.
E te enfeitei com adornos, e te pus braceletes nas mãos e um
colar ao redor do teu pescoço. E te pus um pendente na testa,
e brincos nas orelhas, e uma coroa de glória na cabeça. E
assim foste ornada de ouro e prata, e o teu vestido foi de linho
fino, e de seda e de bordados; nutriste-te de flor de farinha, e mel
e azeite; e foste formosa em extremo, e foste próspera, até chegares
a realeza. E correu de ti a tua fama entre os gentios, por
causa da tua formosura, pois era perfeita, por causa da minha glória
que eu pusera em ti, diz o Senhor Deus.

Mas confiaste na tua formosura, e te corrompeste por causa da tua
fama, e prostituías-te a todo o que passava, para seres dele.
E tomaste dos teus vestidos, e fizeste lugares altos pintados
de diversas cores, e te prostituíste sobre eles, como nunca
sucedera, nem sucederá. E tomaste as tuas jóias de enfeite,
que eu te dei do meu ouro e da minha prata, e fizeste imagens de
homens, e te prostituíste com elas. E tomaste os teus
vestidos bordados, e as cobriste; e o meu azeite e o meu perfume
puseste diante delas. E o meu pão que te dei, a flor de
farinha, e o azeite e o mel com que eu te sustentava, também puseste
diante delas em cheiro suave; e assim foi, diz o Senhor Deus.
Além disto, tomaste a teus filhos e tuas filhas, que me
tinhas gerado, e os sacrificaste a elas, para serem consumidos;
acaso é pequena a tua prostituição? E mataste a meus filhos,
e os entregaste a elas para os fazerem passar pelo fogo. E em
todas as tuas abominações, e nas tuas prostituições, não te
lembraste dos dias da tua mocidade, quando tu estavas nua e
descoberta, e revolvida no teu sangue. E sucedeu, depois de
toda a tua maldade (ai, ai de ti! diz o Senhor Deus), que
edificaste uma abóbada, e fizeste lugares altos em cada rua.
A cada canto do caminho edificaste o teu lugar alto, e
fizeste abominável a tua formosura, e alargaste os teus pés a todo o
que passava, e multiplicaste as tuas prostituições. Também te
prostituíste com os filhos do Egito, teus vizinhos grandes de carne,
e multiplicaste a tua prostituição para me provocares à ira.
Por isso estendi a minha mão sobre ti, e diminuí a tua
porção; e te entreguei à vontade das que te odeiam, das filhas dos
filisteus, as quais se envergonhavam do teu caminho depravado.
Também te prostituíste com os filhos da Assíria, porquanto
eras insaciável; e prostituindo-te com eles, nem ainda assim ficaste
farta. Antes multiplicaste as tuas prostituições na terra de
Canaã até Caldéia, e nem ainda com isso te fartaste. Quão
fraco é o teu coração, diz o Senhor Deus, fazendo tu todas estas
coisas, obras de uma meretriz imperiosa! Edificando tu a tua
abóbada ao canto de cada caminho, e fazendo o teu lugar alto em cada
rua! Nem foste como a meretriz, pois desprezaste a paga;
foste como a mulher adúltera que, em lugar de seu marido,
recebe os estranhos. A todas as meretrizes dão paga, mas tu
dás os teus presentes a todos os teus amantes; e lhes dás presentes,
para que venham a ti de todas as partes, pelas tuas prostituições.
Assim que contigo sucede o contrário das outras mulheres nas
tuas prostituições, pois ninguém te procura para prostituição;
porque, dando tu a paga, e a ti não sendo dada a paga, fazes o
contrário.

Portanto, ó meretriz, ouve a palavra do Senhor. Assim diz
o Senhor Deus: Porquanto se derramou o teu dinheiro, e se descobriu
a tua nudez nas tuas prostituições com os teus amantes, como também
com todos os ídolos das tuas abominações, e do sangue de teus filhos
que lhes deste; portanto, eis que ajuntarei a todos os teus
amantes, com os quais te deleitaste, como também a todos os que
amaste, com todos os que odiaste, e ajuntá-los-ei contra ti em
redor, e descobrirei a tua nudez diante deles, para que vejam toda a
tua nudez. E julgar-te-ei como são julgadas as adúlteras e as
que derramam sangue; e entregar-te-ei ao sangue de furor e de ciúme.
E entregar-te-ei nas mãos deles; e eles derrubarão a tua
abóbada, e transtornarão os teus altos lugares, e te despirão os
teus vestidos, e tomarão as tuas jóias de enfeite, e te deixarão nua
e descoberta. Então farão subir contra ti uma multidão, e te
apedrejarão, e te traspassarão com as suas espadas. E
queimarão as tuas casas a fogo, e executarão juízos contra ti aos
olhos de muitas mulheres; e te farei cessar de ser meretriz, e paga
não darás mais. Assim satisfarei em ti o meu furor, e os meus
ciúmes se desviarão de ti, e me aquietarei, e nunca mais me
indignarei. Porquanto não te lembraste dos dias da tua
mocidade, e me provocaste à ira com tudo isto, eis que também eu
farei recair o teu caminho sobre a tua cabeça, diz o Senhor Deus, e
não mais farás tal perversidade sobre todas as tuas abominações.

Eis que todo o que usa de provérbios usará contra ti este
provérbio, dizendo: Tal mãe, tal filha. Tu és filha de tua
mãe, que tinha nojo de seu marido e de seus filhos; e tu és irmã de
tuas irmãs, que tinham nojo de seus maridos e de seus filhos; vossa
mãe foi hetéia, e vosso pai amorreu. E tua irmã, a maior, é
Samaria, ela e suas filhas, a qual habita à tua esquerda; e a tua
irmã menor, que habita à tua mão direita, é Sodoma e suas filhas.
Todavia não andaste nos seus caminhos, nem fizeste conforme
as suas abominações; mas como se isto fora mui pouco, ainda te
corrompeste mais do que elas, em todos os teus caminhos. Vivo
eu, diz o Senhor Deus, que não fez Sodoma, tua irmã, nem ela, nem
suas filhas, como fizeste tu e tuas filhas. Eis que esta foi
a iniqüidade de Sodoma, tua irmã: Soberba, fartura de pão, e
abundância de ociosidade teve ela e suas filhas; mas nunca
fortaleceu a mão do pobre e do necessitado. E se
ensoberbeceram, e fizeram abominações diante de mim; portanto, vendo
eu isto as tirei dali. Também Samaria não cometeu a metade de
teus pecados; e multiplicaste as tuas abominações mais do que elas,
e justificaste a tuas irmãs, com todas as tuas abominações que
fizeste. Tu, também, que julgaste a tuas irmãs, leva a tua
vergonha pelos pecados, que cometeste, mais abomináveis do que elas;
mais justas são do que tu; envergonha-te logo também, e leva a tua
vergonha, pois justificaste a tuas irmãs. Eu, pois, farei
voltar os cativos delas; os cativos de Sodoma e suas filhas, e os
cativos de Samaria e suas filhas, e os cativos do teu cativeiro
dentre elas; para que leves a tua vergonha, e sejas
envergonhada por tudo o que fizeste, dando-lhes tu consolação.
Quando tuas irmãs, Sodoma e suas filhas, tornarem ao seu
primeiro estado, e também Samaria e suas filhas tornarem ao seu
primeiro estado, também tu e tuas filhas tornareis ao vosso primeiro
estado. Nem mesmo Sodoma, tua irmã, foi mencionada pela tua
boca, no dia da tua soberba, antes que se descobrisse a tua
maldade, como no tempo do desprezo das filhas da Síria, e de todos
os que estavam ao redor dela, as filhas dos filisteus, que te
desprezavam em redor. A tua perversidade e as tuas
abominações tu levarás, diz o Senhor. Porque assim diz o
Senhor Deus: Eu te farei como fizeste, que desprezaste o juramento,
quebrando a aliança.

Contudo eu me lembrarei da minha aliança, que fiz contigo nos
dias da tua mocidade; e estabelecerei contigo uma aliança eterna.
Então te lembrarás dos teus caminhos, e te confundirás,
quando receberes tuas irmãs maiores do que tu, com as menores do que
tu, porque tas darei por filhas, mas não pela tua aliança.
Porque eu estabelecerei a minha aliança contigo, e saberás
que eu sou o Senhor; para que te lembres disso, e te
envergonhes, e nunca mais abras a tua boca, por causa da tua
vergonha, quando eu te expiar de tudo quanto fizeste, diz o Senhor
Deus.

\medskip

\lettrine{17} E veio a mim a palavra do Senhor, dizendo:
Filho do homem, propõe um enigma, e profere uma parábola para
com a casa de Israel. E disse: Assim diz o Senhor Deus: Uma
grande águia, de grandes asas, de plumagem comprida, e cheia de
penas de várias cores, veio ao Líbano e levou o mais alto ramo de um
cedro. E arrancou a ponta mais alta dos seus renovos, e a levou
a uma terra de mercância\footnote{Ato, processo ou efeito de
mercanciar, de mercadejar; mercadejo. Mercadoria (`produto',
`negócio', `ocupação'). Aquilo que se mercou, que se comprou;
compra.}; numa cidade de mercadores a pôs. Tomou da semente da
terra, e a lançou num solo frutífero; tomando-a, colocou-a junto às
muitas águas, plantando-a como salgueiro. E brotou, e tornou-se
numa videira muito larga, de pouca altura, virando-se para ela os
seus ramos, porque as suas raízes estavam debaixo dela; e tornou-se
numa videira, e produzia sarmentos, e brotava renovos. E houve
mais uma grande águia, de grandes asas, e cheia de penas; e eis que
esta videira lançou para ela as suas raízes, e estendeu para ela os
seus ramos, desde as covas do seu plantio, para que a regasse.
Num bom campo, junto a muitas águas, estava ela plantada, para
produzir ramos, e para dar fruto, a fim de que fosse videira
excelente. Dize: Assim diz o Senhor Deus: Porventura há de
prosperar? Não lhe arrancará as suas raízes, e não cortará o seu
fruto, para que se seque? Para que sequem todas as folhas de seus
renovos, e isto não com grande força, nem muita gente, para
arrancá-la pelas suas raízes. Mas, estando plantada,
prosperará? Porventura, tocando-lhe vento oriental, de todo não se
secará? Nas covas do seu plantio se secará. Então veio a mim
a palavra do Senhor, dizendo: Dize agora à casa rebelde: Não
sabeis o que significam estas coisas? Dize: Eis que veio o rei de
Babilônia a Jerusalém, e tomou o seu rei e os seus príncipes, e os
levou consigo para Babilônia. E tomou um da descendência
real, e fez aliança com ele, e o fez prestar juramento; e tomou
consigo os poderosos da terra, para que o reino ficasse
humilhado, e não se levantasse, embora, guardando a sua aliança,
pudesse subsistir. Mas rebelou-se contra ele, enviando os
seus mensageiros ao Egito, para que se lhe mandassem cavalos e muita
gente. Porventura prosperará ou escapará aquele que faz tais coisas,
ou quebrará a aliança, e ainda escapará? Vivo eu, diz o
Senhor Deus, que no lugar em que habita o rei que o fez reinar, cujo
juramento desprezou, e cuja aliança quebrou, sim, com ele no meio de
Babilônia certamente morrerá. E Faraó, nem com grande
exército, nem com uma companhia numerosa, fará coisa alguma com ele
em guerra, levantando trincheiras e edificando baluartes, para
destruir muitas vidas. Porque desprezou o juramento,
quebrando a aliança; eis que ele tinha dado a sua mão; contudo fez
todas estas coisas; não escapará. Portanto, assim diz o
Senhor Deus: Vivo eu, que o meu juramento, que desprezou, e a minha
aliança, que quebrou, isto farei recair sobre a sua cabeça. E
estenderei sobre ele a minha rede, e ficará preso no meu laço; e o
levarei a Babilônia, e ali entrarei em juízo com ele por causa da
rebeldia que praticou contra mim. E todos os seus fugitivos,
com todas as suas tropas, cairão à espada, e os que restarem serão
espalhados a todo o vento; e sabereis que eu, o Senhor, o disse.

Assim diz o Senhor Deus: Também eu tomarei um broto do topo do
cedro, e o plantarei; do principal dos seus renovos cortarei o mais
tenro, e o plantarei sobre um monte alto e sublime. No monte
alto de Israel o plantarei, e produzirá ramos, e dará fruto, e se
fará um cedro excelente; e habitarão debaixo dele aves de toda
plumagem, à sombra dos seus ramos habitarão. Assim saberão
todas as árvores do campo que eu, o Senhor, abati a árvore alta,
elevei a árvore baixa, sequei a árvore verde, e fiz reverdecer a
árvore seca; eu, o Senhor, o disse, e o fiz.

\medskip

\lettrine{18} E veio a mim a palavra do Senhor, dizendo:
Que pensais, vós, os que usais esta parábola sobre a terra de
Israel, dizendo: Os pais comeram uvas verdes, e os dentes dos filhos
se embotaram? Vivo eu, diz o Senhor Deus, que nunca mais direis
esta parábola em Israel. Eis que todas as almas são minhas; como
o é a alma do pai, assim também a alma do filho é minha: a alma que
pecar, essa morrerá. Sendo, pois, o homem justo, e praticando
juízo e justiça, não comendo sobre os montes, nem levantando os
seus olhos para os ídolos da casa de Israel, nem contaminando a
mulher do seu próximo, nem se chegando à mulher na sua separação,
não oprimindo a ninguém, tornando ao devedor o seu penhor, não
roubando, dando o seu pão ao faminto, e cobrindo ao nu com roupa,
não dando o seu dinheiro à usura, e não recebendo demais,
desviando a sua mão da injustiça, e fazendo verdadeiro juízo entre
homem e homem; andando nos meus estatutos, e guardando os meus
juízos, e procedendo segundo a verdade, o tal justo certamente
viverá, diz o Senhor Deus.

E se ele gerar um filho ladrão, derramador de sangue, que fizer a
seu irmão qualquer destas coisas; e não cumprir todos aqueles
deveres, mas antes comer sobre os montes, e contaminar a mulher de
seu próximo, oprimir ao pobre e necessitado, praticar roubos,
não tornar o penhor, e levantar os seus olhos para os ídolos, e
cometer abominação, e emprestar com usura, e receber demais,
porventura viverá? Não viverá. Todas estas abominações ele fez,
certamente morrerá; o seu sangue será sobre ele. E eis que
também, se ele gerar um filho que veja todos os pecados que seu pai
fez e, vendo-os, não cometer coisas semelhantes, não comer
sobre os montes, e não levantar os seus olhos para os ídolos da casa
de Israel, e não contaminar a mulher de seu próximo, e não
oprimir a ninguém, e não retiver o penhor, e não roubar, der o seu
pão ao faminto, e cobrir ao nu com roupa, desviar do pobre a
sua mão, não receber usura e juros, cumprir os meus juízos, e andar
nos meus estatutos, o tal não morrerá pela iniqüidade de seu pai;
certamente viverá. Seu pai, porque praticou a extorsão,
roubou os bens do irmão, e fez o que não era bom no meio de seu
povo, eis que ele morrerá pela sua iniqüidade. Mas dizeis:
Por que não levará o filho a iniqüidade do pai? Porque o filho
procedeu com retidão e justiça, e guardou todos os meus estatutos, e
os praticou, por isso certamente viverá. A alma que pecar,
essa morrerá; o filho não levará a iniqüidade do pai, nem o pai
levará a iniqüidade do filho. A justiça do justo ficará sobre ele e
a impiedade do ímpio cairá sobre ele.

Mas se o ímpio se converter de todos os pecados que cometeu, e
guardar todos os meus estatutos, e proceder com retidão e justiça,
certamente viverá; não morrerá. De todas as transgressões que
cometeu não haverá lembrança contra ele; pela justiça que praticou
viverá. Desejaria eu, de qualquer maneira, a morte do ímpio?
diz o Senhor Deus; Não desejo antes que se converta dos seus
caminhos, e viva? Mas, desviando-se o justo da sua justiça, e
cometendo a iniqüidade, fazendo conforme todas as abominações que
faz o ímpio, porventura viverá? De todas as justiças que tiver feito
não se fará memória; na sua transgressão com que transgrediu, e no
seu pecado com que pecou, neles morrerá. Dizeis, porém: O
caminho do Senhor não é direito. Ouvi agora, ó casa de Israel:
Porventura não é o meu caminho direito? Não são os vossos caminhos
tortuosos? Desviando-se o justo da sua justiça, e cometendo
iniqüidade, morrerá por ela; na iniqüidade, que cometeu, morrerá.
Mas, convertendo-se o ímpio da impiedade que cometeu, e
procedendo com retidão e justiça, conservará este a sua alma em
vida. Pois que reconsidera, e se converte de todas as suas
transgressões que cometeu; certamente viverá, não morrerá.
Contudo, diz a casa de Israel: O caminho do Senhor não é
direito. Porventura não são direitos os meus caminhos, ó casa de
Israel? E não são tortuosos os vossos caminhos?

Portanto, eu vos julgarei, cada um conforme os seus caminhos, ó
casa de Israel, diz o Senhor Deus. Tornai-vos, e convertei-vos de
todas as vossas transgressões, e a iniqüidade não vos servirá de
tropeço. Lançai de vós todas as vossas transgressões com que
transgredistes, e fazei-vos um coração novo e um espírito novo;
pois, por que razão morreríeis, ó casa de Israel? Porque não
tenho prazer na morte do que morre, diz o Senhor Deus;
convertei-vos, pois, e vivei.

\medskip

\lettrine{19} E tu levanta uma lamentação sobre os príncipes
de Israel, e dize: Quem foi tua mãe? Uma leoa entre os leões a
qual, deitada no meio dos leõezinhos, criou os seus filhotes. E
educou um dos seus filhotes, o qual veio a ser leãozinho e aprendeu
a apanhar a presa, e devorou homens, e, ouvindo falar dele as
nações, foi apanhado na cova delas, e o trouxeram com cadeias à
terra do Egito. Vendo, pois, ela que havia esperado muito, e que
a sua expectação era perdida, tomou outro dos seus filhotes, e fez
dele um leãozinho. Este, pois, andando continuamente no meio dos
leões, veio a ser leãozinho, e aprendeu a apanhar a presa, e devorou
homens. E conheceu os seus palácios, e destruiu as suas cidades;
e assolou-se a terra, e a sua plenitude, ao som do seu rugido.
Então se ajuntaram contra ele os povos das províncias ao redor,
e estenderam sobre ele a rede, e foi apanhado na cova deles. E
com cadeias colocaram-no em uma jaula, e o levaram ao rei de
Babilônia; fizeram-no entrar nos lugares fortes, para que não se
ouvisse mais a sua voz nos montes de Israel.

Tua mãe era como uma videira no teu sangue, plantada junto às
águas; ela frutificou, e encheu-se de ramos, por causa das muitas
águas. E tinha varas fortes para cetros de dominadores, e
elevou-se a sua estatura entre os espessos ramos, e foi vista na sua
altura com a multidão dos seus ramos. Mas foi arrancada com
furor, foi lançada por terra, e o vento oriental secou o seu fruto;
quebraram-se e secaram-se as suas fortes varas, o fogo as consumiu,
e agora está plantada no deserto, numa terra seca e sedenta.
E de uma vara dos seus ramos saiu fogo que consumiu o seu
fruto de maneira que nela não há mais vara forte, cetro para
dominar. Esta é a lamentação, e servirá de lamentação.

\medskip

\lettrine{20} E aconteceu, no sétimo ano, no quinto mês, aos
dez do mês, que vieram alguns dos anciãos de Israel, para
consultarem o Senhor; e assentaram-se diante de mim. Então veio
a mim a palavra do Senhor, dizendo: Filho do homem, fala aos
anciãos de Israel, e dize-lhes: Assim diz o Senhor Deus: Viestes
consultar-me? Vivo eu, que não me deixarei ser consultado por vós,
diz o Senhor Deus. Porventura tu os julgarias, julgarias tu, ó
filho do homem? Notifica-lhes as abominações de seus pais.

E dize-lhes: Assim diz o Senhor Deus: No dia em que escolhi a
Israel, levantei a minha mão para a descendência da casa de Jacó, e
me dei a conhecer a eles na terra do Egito, e levantei a minha mão
para eles, dizendo: Eu sou o Senhor vosso Deus; naquele dia
levantei a minha mão para eles, para os tirar da terra do Egito,
para uma terra que já tinha previsto para eles, a qual mana leite e
mel, e é a glória de todas as terras. Então lhes disse: Cada um
lance de si as abominações dos seus olhos, e não vos contamineis com
os ídolos do Egito; eu sou o Senhor vosso Deus. Mas rebelaram-se
contra mim, e não me quiseram ouvir; ninguém lançava de si as
abominações dos seus olhos, nem deixava os ídolos do Egito; então eu
disse que derramaria sobre eles o meu furor, para cumprir a minha
ira contra eles no meio da terra do Egito. O que fiz, porém, foi
por amor do meu nome, para que não fosse profanado diante dos olhos
dos gentios, no meio dos quais estavam, a cujos olhos eu me dei a
conhecer a eles, para os tirar da terra do Egito.

E os tirei da terra do Egito, e os levei ao deserto. E
dei-lhes os meus estatutos e lhes mostrei os meus juízos, os quais,
cumprindo-os o homem, viverá por eles. E também lhes dei os
meus sábados, para que servissem de sinal entre mim e eles; para que
soubessem que eu sou o Senhor que os santifica. Mas a casa de
Israel se rebelou contra mim no deserto, não andando nos meus
estatutos, e rejeitando os meus juízos, os quais, cumprindo-os, o
homem viverá por eles; e profanaram grandemente os meus sábados; e
eu disse que derramaria sobre eles o meu furor no deserto, para os
consumir. O que fiz, porém, foi por amor do meu nome, para
que não fosse profanado diante dos olhos dos gentios perante a vista
dos quais os fiz sair. E, contudo, eu levantei a minha mão
para eles no deserto, para não os deixar entrar na terra que lhes
tinha dado, a qual mana leite e mel, e é a glória de todas as
terras; porque rejeitaram os meus juízos, e não andaram nos
meus estatutos, e profanaram os meus sábados; porque o seu coração
andava após os seus ídolos. Não obstante o meu olho lhes
perdoou, e eu não os destruí nem os consumi no deserto. Mas
disse eu a seus filhos no deserto: Não andeis nos estatutos de
vossos pais, nem guardeis os seus juízos, nem vos contamineis com os
seus ídolos. Eu sou o Senhor vosso Deus; andai nos meus
estatutos, e guardai os meus juízos, e executai-os. E
santificai os meus sábados, e servirão de sinal entre mim e vós,
para que saibais que eu sou o Senhor vosso Deus. Mas também
os filhos se rebelaram contra mim, e não andaram nos meus estatutos,
nem guardaram os meus juízos para os fazer, os quais, cumprindo-os,
o homem viverá por eles; eles profanaram os meus sábados; por isso
eu disse que derramaria sobre eles o meu furor, para cumprir contra
eles a minha ira no deserto. Mas contive a minha mão, e o fiz
por amor do meu nome, para que não fosse profanado perante os olhos
dos gentios, à vista dos quais os fiz sair. Também levantei a
minha mão para eles no deserto, para os espalhar entre os gentios, e
os derramar pelas terras, porque não executaram os meus
juízos, e rejeitaram os meus estatutos, e profanaram os meus
sábados, e os seus olhos iam após os ídolos de seus pais. Por
isso também lhes dei estatutos que não eram bons, juízos pelos quais
não haviam de viver; e os contaminei em seus próprios dons,
nos quais faziam passar pelo fogo tudo o que abre a madre; para
assolá-los para que soubessem que eu sou o Senhor.

Portanto fala à casa de Israel, ó filho do homem, e dize-lhe:
Assim diz o Senhor Deus: Ainda até nisto me blasfemaram vossos pais,
e que procederam traiçoeiramente contra mim. Porque,
havendo-os eu introduzido na terra sobre a qual eu levantara a minha
mão, para lha dar, então olharam para todo o outeiro alto, e para
toda a árvore frondosa, e ofereceram ali os seus sacrifícios e
apresentaram ali a provocação das suas ofertas; puseram ali os seus
cheiros suaves, e ali derramaram as suas libações. E eu lhes
disse: Que alto é este, aonde vós ides? E seu nome tem sido Bamá até
o dia de hoje. Portanto dize à casa de Israel: Assim diz o
Senhor Deus: Contaminai-vos a vós mesmos a maneira de vossos pais? E
vos prostituístes com as suas abominações? E, quando
ofereceis os vossos dons, e fazeis passar os vossos filhos pelo
fogo, não é certo que estais contaminados com todos os vossos
ídolos, até este dia? E vós me consultaríeis, ó casa de Israel? Vivo
eu, diz o Senhor Deus, que vós não me consultareis. E o que
veio à vossa mente de modo algum sucederá, quando dizeis: Seremos
como os gentios, como as outras famílias da terra, servindo ao
madeiro\footnote{Qualquer peça de madeira robusta; madeira, lenho.
Tronco forte de madeira que sustenta as vigas dos sobrados e tetos.
m.q. cruz (hist.rel).} e à pedra.

Vivo eu, diz o Senhor Deus, que com mão forte, e com braço
estendido, e com indignação derramada, hei de reinar sobre vós.
E vos tirarei dentre os povos, e vos congregarei das terras
nas quais andais espalhados, com mão forte, e com braço estendido, e
com indignação derramada. E vos levarei ao deserto dos povos;
e ali face a face entrarei em juízo convosco; como entrei em
juízo com vossos pais, no deserto da terra do Egito, assim entrarei
em juízo convosco, diz o Senhor Deus. Também vos farei passar
debaixo da vara, e vos farei entrar no vínculo da aliança. E
separarei dentre vós os rebeldes, e os que transgrediram contra mim;
da terra das suas peregrinações os tirarei, mas à terra de Israel
não voltarão; e sabereis que eu sou o Senhor. Quanto a vós, ó
casa de Israel, assim diz o Senhor Deus; Ide, sirva cada um os seus
ídolos, pois que a mim não me quereis ouvir; mas não profaneis mais
o meu santo nome com as vossas dádivas e com os vossos ídolos.
Porque no meu santo monte, no monte alto de Israel, diz o
Senhor Deus, ali me servirá toda a casa de Israel, toda ela naquela
terra; ali me deleitarei neles, e ali requererei as vossas ofertas
alçadas, e as primícias das vossas oblações, com todas as vossas
coisas santas; com cheiro suave me deleitarei em vós, quando
eu vos tirar dentre os povos e vos congregar das terras em que
andais espalhados; e serei santificado em vós perante os olhos dos
gentios. E sabereis que eu sou o Senhor, quando eu vos
introduzir na terra de Israel, terra pela qual levantei a minha mão
para dá-la a vossos pais. E ali vos lembrareis de vossos
caminhos, e de todos os vossos atos com que vos contaminastes, e
tereis nojo de vós mesmos, por causa de todas as vossas maldades que
tendes cometido. E sabereis que eu sou o Senhor, quando eu
proceder para convosco por amor do meu nome; não conforme os vossos
maus caminhos, nem conforme os vossos atos corruptos, ó casa de
Israel, disse o Senhor Deus.

E veio a mim a palavra do Senhor, dizendo: Filho do homem,
dirige o teu rosto para o caminho do sul, e derrama as tuas palavras
contra o sul, e profetiza contra o bosque do campo do sul. E
dize ao bosque do sul: Ouve a palavra do Senhor: Assim diz o Senhor
Deus: Eis que acenderei em ti um fogo que em ti consumirá toda a
árvore verde e toda a árvore seca; não se apagará a chama
flamejante, antes com ela se queimarão todos os rostos, desde o sul
até ao norte. E verá toda a carne que eu, o Senhor, o acendi;
não se apagará. Então disse eu: Ah! Senhor Deus! Eles dizem
de mim: Não é este um proferidor de parábolas?

\medskip

\lettrine{21} E veio a mim a palavra do Senhor, dizendo:
Filho do homem, dirige o teu rosto contra Jerusalém, e derrama
as tuas palavras sobre os santuários, e profetiza sobre a terra de
Israel. E dize à terra de Israel: Assim diz o Senhor: Eis que
sou contra ti, e tirarei a minha espada da bainha, e exterminarei do
meio de ti o justo e o ímpio. E, por isso que hei de exterminar
do meio de ti o justo e o ímpio, a minha espada sairá da sua bainha
contra toda a carne, desde o sul até o norte. E saberá toda a
carne que eu, o Senhor, tirei a minha espada da bainha; nunca mais
voltará a ela. Tu, porém, ó filho do homem, suspira; suspira aos
olhos deles, com quebrantamento dos teus lombos e com amargura.
E será que, quando eles te disserem: Por que suspiras tu? Dirás:
Por causa das novas, porque vêm; e todo o coração desmaiará, e todas
as mãos se enfraquecerão, e todo o espírito se angustiará, e todos
os joelhos se desfarão em águas; eis que vêm, e se cumprirão, diz o
Senhor Deus.

E veio a mim a palavra do Senhor, dizendo: Filho do homem,
profetiza, e dize: Assim diz o Senhor: dize: A espada, a espada está
afiada e polida. Para grande matança está afiada, para
reluzir está polida. Alegrar-nos-emos pois? A vara de meu filho é
que despreza todo o madeiro. E foi dada a polir, para ser
manejada; esta espada está afiada, e está polida, para ser posta na
mão do matador. Grita e geme, ó filho do homem, porque ela
será contra o meu povo, contra todos os príncipes de Israel. Estes,
juntamente com o meu povo, estão espantados com a espada; bate,
pois, na tua coxa. Pois se faz uma prova; e que seria se a
espada desprezasse mesmo a vara? Ela não seria mais, diz o Senhor
Deus. Tu, pois, ó filho do homem, profetiza e bate com as
mãos uma na outra; e dobre-se a espada até a terceira vez, a espada
dos mortos; ela é a espada para a grande matança, que os traspassará
até o seu interior. Para que desmaie o coração, e se
multipliquem as destruições, contra todas as suas portas, pus a
ponta da espada, a que foi feita para reluzir, e está preparada para
a matança! Ó espada, une-te, vira-te para a direita;
prepara-te, vira-te para a esquerda, para onde quer que o teu rosto
se dirigir. E também eu baterei com as minhas mãos uma na
outra, e farei descansar a minha indignação; eu, o Senhor, o disse.

E veio a mim a palavra do Senhor, dizendo: Tu, pois, ó
filho do homem, propõe dois caminhos, por onde venha a espada do rei
de Babilônia. Ambos procederão de uma mesma terra, e escolhe um
lugar; escolhe-o no cimo do caminho da cidade. Um caminho
proporás, por onde virá a espada contra Rabá dos filhos de Amom, e
contra Judá, em Jerusalém, a fortificada. Porque o rei de
Babilônia parará na encruzilhada, no cimo dos dois caminhos, para
fazer adivinhações; aguçará as suas flechas, consultará as imagens,
atentará para o fígado. À sua direita estará a adivinhação
sobre Jerusalém, para ordenar aos capitães, para abrirem a boca,
ordenando a matança, para levantarem a voz com júbilo, para porem os
aríetes contra as portas, para levantarem trincheiras, para
edificarem baluartes. Isto será como adivinhação vã, aos
olhos daqueles que lhes fizeram juramentos; mas ele se lembrará da
iniqüidade, para que sejam apanhados. Portanto assim diz o
Senhor Deus: Visto que me fazeis lembrar da vossa iniqüidade,
descobrindo-se as vossas transgressões, aparecendo os vossos pecados
em todos os vossos atos; visto que viestes em memória, sereis
apanhados com a mão. E tu, ó profano e ímpio príncipe de
Israel, cujo dia virá no tempo da extrema iniqüidade, assim
diz o Senhor Deus: Tira o diadema, e remove a coroa; esta não será a
mesma; exalta ao humilde, e humilha ao soberbo. Ao revés, ao
revés, ao revés porei aquela coroa, e ela não mais será, até que
venha aquele a quem pertence de direito; a ele a darei.

E tu, ó filho do homem, profetiza, e dize: Assim diz o Senhor
Deus acerca dos filhos de Amom, e acerca do seu opróbrio; dize pois:
A espada, a espada está desembainhada, polida para a matança, para
consumir, por estar reluzente; entretanto te profetizam
vaidade, te adivinham mentira, para te porem no pescoço dos ímpios,
daqueles que estão mortos, cujo dia veio no tempo da iniqüidade
final. Torne a tua espada à sua bainha. No lugar em que foste
criado, na terra do teu nascimento, eu te julgarei. E
derramarei sobre ti a minha indignação, assoprarei contra ti o fogo
do meu furor, entregar-te-ei nas mãos dos homens brutais, inventores
de destruição. Ao fogo servirás para ser consumido; o teu
sangue estará no meio da terra; já não serás mais lembrado, porque
eu, o Senhor, o disse.

\medskip

\lettrine{22} E veio a mim a palavra do Senhor, dizendo:
Tu, pois, ó filho do homem, porventura julgarás, julgarás a
cidade sanguinária? Faze-lhe conhecer, pois, todas as suas
abominações. E dize: Assim diz o Senhor Deus: Ai da cidade que
derrama o sangue no meio de si para que venha o seu tempo! Que faz
ídolos contra si mesma, para se contaminar! Pelo teu sangue que
derramaste te fizeste culpada, e pelos teus ídolos que fabricaste te
contaminaste, e fizeste aproximarem-se os teus dias, e tem chegado o
fim dos teus anos; por isso eu te fiz o opróbrio das nações e o
escárnio de todas as terras. As que estão perto de ti e as que
estão longe escarnecerão de ti, infamada, cheia de inquietação.
Eis que os príncipes de Israel, cada um conforme o seu poder,
estavam em ti para derramarem sangue. Ao pai e à mãe desprezaram
em ti; para com o estrangeiro usaram de opressão no meio de ti; ao
órfão e à viúva oprimiram em ti. As minhas coisas santas
desprezaste, e os meus sábados profanaste. Homens caluniadores
se acharam em ti, para derramarem sangue; e em ti sobre os montes
comeram; perversidade cometeram no meio de ti. A vergonha do
pai descobriram em ti; a que estava imunda, na sua separação,
humilharam no meio de ti. Um cometeu abominação com a mulher
do seu próximo, outro contaminou abominavelmente a sua nora, e outro
humilhou no meio de ti a sua irmã, filha de seu pai.
Presentes receberam no meio de ti para derramarem sangue;
usura e juros ilícitos tomaste, e usaste de avareza com o teu
próximo, oprimindo-o; mas de mim te esqueceste, diz o Senhor Deus.
E eis que bati as mãos contra a avareza que cometeste, e por
causa do sangue que houve no meio de ti. Porventura estará
firme o teu coração? Porventura estarão fortes as tuas mãos, nos
dias em que eu tratarei contigo? Eu, o Senhor, o disse, e o farei.
E espalhar-te-ei entre as nações, e dispersar-te-ei pelas
terras, e porei termo à tua imundícia. E tu serás profanada
em ti mesma aos olhos dos gentios, e saberás que eu sou o Senhor.

E veio a mim a palavra do Senhor, dizendo: Filho do homem,
a casa de Israel se tornou para mim em escórias; todos eles são
bronze, e estanho, e ferro, e chumbo no meio do forno; em escórias
de prata se tornaram. Portanto assim diz o Senhor Deus: Pois
que todos vós vos tornastes em escórias, por isso eis que eu vos
ajuntarei no meio de Jerusalém. Como se ajuntam a prata, e o
bronze, e o ferro, e o chumbo, e o estanho, no meio do forno, para
assoprar o fogo sobre eles, a fim de se fundirem, assim vos
ajuntarei na minha ira e no meu furor, e ali vos deixarei e
fundirei. E congregar-vos-ei, e assoprarei sobre vós o fogo
do meu furor; e sereis fundidos no meio dela. Como se funde a
prata no meio do forno, assim sereis fundidos no meio dela; e
sabereis que eu, o Senhor, derramei o meu furor sobre vós.

E veio a mim a palavra do Senhor, dizendo: Filho do homem,
dize-lhe: Tu és uma terra que não está purificada; e que não tem
chuva no dia da indignação. Conspiração dos seus profetas há
no meio dela, como um leão que ruge, que arrebata a presa; eles
devoram as almas; tomam tesouros e coisas preciosas, multiplicam as
suas viúvas no meio dela. Os seus sacerdotes violentam a
minha lei, e profanam as minhas coisas santas; não fazem diferença
entre o santo e o profano, nem discernem o impuro do puro; e de meus
sábados escondem os seus olhos, e assim sou profanado no meio deles.
Os seus príncipes no meio dela são como lobos que arrebatam a
presa, para derramarem sangue, para destruírem as almas, para
seguirem a avareza. E os seus profetas têm feito para eles
cobertura com argamassa não temperada, profetizando vaidade,
adivinhando-lhes mentira, dizendo: Assim diz o Senhor Deus; sem que
o Senhor tivesse falado. Ao povo da terra oprimem gravemente,
e andam roubando, e fazendo violência ao pobre e necessitado, e ao
estrangeiro oprimem sem razão. E busquei dentre eles um homem
que estivesse tapando o muro, e estivesse na brecha perante mim por
esta terra, para que eu não a destruísse; porém a ninguém achei.
Por isso eu derramei sobre eles a minha indignação; com o
fogo do meu furor os consumi; fiz que o seu caminho recaísse sobre a
sua cabeça, diz o Senhor Deus.

\medskip

\lettrine{23} Veio mais a mim a palavra do Senhor, dizendo:
Filho do homem, houve duas mulheres, filhas de uma mesma mãe.
Estas se prostituíram no Egito; prostituíram-se na sua mocidade;
ali foram apertados os seus seios, e ali foram apalpados os seios da
sua virgindade. E os seus nomes eram: Aolá, a mais velha, e
Aolibá, sua irmã; e foram minhas, e tiveram filhos e filhas; e,
quanto aos seus nomes, Samaria é Aolá, e Jerusalém é Aolibá. E
prostituiu-se Aolá, sendo minha; e enamorou-se dos seus amantes, dos
assírios, seus vizinhos, vestidos de azul, capitães e
magistrados, todos jovens cobiçáveis, cavaleiros montados a cavalo.
Assim cometeu ela as suas devassidões com eles, que eram todos a
flor dos filhos da Assíria, e com todos os de quem se enamorava; com
todos os seus ídolos se contaminou. E as suas prostituições, que
trouxe do Egito, não as deixou; porque com ela se deitaram na sua
mocidade, e eles apalparam os seios da sua virgindade, e derramaram
sobre ela a sua impudicícia. Portanto a entreguei na mão dos
seus amantes, na mão dos filhos da Assíria, de quem se enamorara.
Estes descobriram a sua vergonha, levaram seus filhos e suas
filhas, mas a ela mataram à espada; e tornou-se falada entre as
mulheres, e sobre ela executaram os juízos.

Vendo isto sua irmã Aolibá, corrompeu o seu imoderado amor mais
do que ela, e as suas devassidões foram mais do que as de sua irmã.
Enamorou-se dos filhos da Assíria, dos capitães e dos
magistrados seus vizinhos, vestidos com primor, cavaleiros que andam
montados em cavalos, todos jovens cobiçáveis. E vi que se
tinha contaminado; o caminho de ambas era o mesmo. E aumentou
as suas impudicícias, porque viu homens pintados na parede, imagens
dos caldeus, pintadas de vermelho; cingidos de cinto nos seus
lombos, e tiaras largas e tingidas nas suas cabeças, todos com
parecer de príncipes, semelhantes aos filhos de Babilônia em
Caldéia, terra do seu nascimento. E enamorou-se deles, ao
lançar sobre eles os seus olhos; e lhes mandou mensageiros à
Caldéia. Então vieram a ela os filhos de Babilônia para o
leito dos amores, e a contaminaram com as suas impudicícias; e ela
se contaminou com eles; então a sua alma apartou-se deles.
Assim pôs a descoberto as suas devassidões, e descobriu a sua
vergonha; então a minha alma se apartou dela, como já tinha se
apartado a minha alma de sua irmã. Todavia ela multiplicou as
suas prostituições, lembrando-se dos dias da sua mocidade, em que se
prostituíra na terra do Egito. E enamorou-se dos seus
amantes, cuja carne é como a de jumentos, e cujo fluxo é como o de
cavalos. Assim trouxeste à memória a perversidade da tua
mocidade, quando os do Egito apalpavam os teus seios, por causa dos
peitos da tua mocidade.

Por isso, ó Aolibá, assim diz o Senhor Deus: Eis que eu
suscitarei contra ti os teus amantes, dos quais se tinha apartado a
tua alma, e os trarei contra ti de toda a parte em derredor.
Os filhos de Babilônia, e todos os caldeus de Pecode, e de
Soa, e de Coa, e todos os filhos da Assíria com eles, jovens
cobiçáveis, capitães e magistrados todos eles, grandes e afamados
senhores, todos eles montados a cavalo. E virão contra ti com
carros, carretas e rodas, e com multidão de povos; e se colocarão
contra ti em redor com paveses, e escudos e capacetes; e porei
diante deles o juízo, e julgar-te-ão segundo os seus juízos.
E porei contra ti o meu zelo, e usarão de indignação contigo.
Tirar-te-ão o nariz e as orelhas, e o que restar cairá à espada.
Eles tomarão teus filhos e tuas filhas, e o que ficar por último em
ti será consumido pelo fogo. Também te despirão as tuas
vestes, e te tomarão as tuas belas jóias. Assim farei cessar
em ti a tua perversidade e a tua prostituição trazida da terra do
Egito; e não levantarás os teus olhos para eles, nem te lembrarás
nunca mais do Egito. Porque assim diz o Senhor Deus: Eis que
eu te entregarei na mão dos que odeias, na mão daqueles de quem tem
se apartado a tua alma. E eles te tratarão com ódio, e
levarão todo o fruto do teu trabalho, e te deixarão nua e despida; e
descobrir-se-á a vergonha da tua prostituição, e a tua perversidade,
e as tuas devassidões. Estas coisas se te farão, porque te
prostituíste após os gentios, e te contaminaste com os seus ídolos.
No caminho de tua irmã andaste; por isso entregarei o seu
cálice na tua mão. Assim diz o Senhor Deus: Beberás o cálice
de tua irmã, fundo e largo; servirás de riso e escárnio; pois nele
cabe muito. De embriaguez e de dor te encherás; o cálice de
tua irmã Samaria é cálice de espanto e de assolação.
Bebê-lo-ás, pois, e esgotá-lo-ás, e os seus cacos roerás, e
os teus seios arrancarás; porque eu o falei, diz o Senhor Deus.
Portanto, assim diz o Senhor Deus: Como te esqueceste de mim,
e me lançaste para trás das tuas costas, também carregarás com a tua
perversidade e as tuas devassidões.

Disse-me ainda o Senhor: Filho do homem, porventura julgarás tu a
Aolá e a Aolibá? Mostra-lhes, pois, as suas abominações.
Porque adulteraram, e sangue se acha nas suas mãos, e com os
seus ídolos adulteraram, e até os seus filhos, que de mim geraram,
fizeram passar pelo fogo, para os consumir. E ainda isto me
fizeram: contaminaram o meu santuário no mesmo dia, e profanaram os
meus sábados. Porquanto, havendo sacrificado seus filhos aos
seus ídolos, vinham ao meu santuário no mesmo dia para o profanarem;
e eis que assim fizeram no meio da minha casa. E, mais ainda,
mandaram vir alguns homens, de longe, aos quais fora enviado um
mensageiro, e eis que vieram. Por amor deles te lavaste, coloriste
os teus olhos, e te ornaste de enfeites. E te assentaste
sobre um leito de honra, diante do qual estava uma mesa preparada; e
puseste sobre ela o meu incenso e o meu azeite. Com ela se
ouvia a voz de uma multidão satisfeita; com homens de classe baixa
foram trazidos beberrões do deserto; e puseram braceletes nas mãos
das mulheres e coroas de esplendor nas suas cabeças. Então
disse à envelhecida em adultérios: Agora deveras se prostituirão com
ela, e ela com eles? E entraram a ela, como quem entra a uma
prostituta; assim entraram a Aolá e a Aolibá, mulheres infames.
De maneira que homens justos as julgarão como se julgam as
adúlteras, e como se julgam as que derramam sangue; porque são
adúlteras, e sangue há nas suas mãos. Porque assim diz o
Senhor Deus: Farei subir contra elas uma multidão, e as entregarei
ao desterro e ao saque. E a multidão as apedrejará, e as
golpeará com as suas espadas; eles a seus filhos e a suas filhas
matarão, e as suas casas queimarão a fogo. Assim farei cessar
a perversidade da terra, para que se escarmentem\footnote{Infligir
castigo ou punição. Repreender ou censurar de maneira enérgica.
Tornar experiente, avisado ou advertido por meio de escarmento.
Ficar cientificado ou advertido por punição ou prejuízo recebido.}
todas as mulheres, e não façam conforme a vossa perversidade;
o castigo da vossa perversidade eles farão recair sobre vós,
e levareis os pecados dos vossos ídolos; e sabereis que eu sou o
Senhor Deus.

\medskip

\lettrine{24} E veio a mim a palavra do Senhor, no nono ano,
no décimo mês, aos dez do mês, dizendo: Filho do homem, escreve
o nome deste dia, deste mesmo dia; porque o rei de Babilônia se pôs
contra Jerusalém neste mesmo dia. E fala por parábola à casa
rebelde, e dize-lhes: Assim diz o Senhor Deus: Põe a panela ao lume,
põe-na, e deita-lhe também água dentro. Ajunta nela pedaços,
todos os bons pedaços, as coxas e as espáduas; enche-a de ossos
escolhidos. Escolhe o melhor do rebanho, e queima também os
ossos debaixo dela; faze-a ferver bem, e cozam-se dentro dela os
seus ossos. Portanto, assim diz o Senhor Deus: Ai da cidade
sanguinária, da panela que escuma\footnote{Espuma. Gente de baixa
extração social ou moral; ralé.} por dentro, e cuja escuma não saiu
dela! Tira dela pedaço por pedaço; não caia sorte sobre ela;
porque o seu sangue está no meio dela, sobre uma penha
descalvada\footnote{Ou escalvar: tornar calvo, sem cabelos;
descalvar. Extirpar a vegetação a; tornar árido, estéril.} o pôs;
não o derramou sobre a terra, para o cobrir com pó. Para fazer
subir a indignação, para tomar vingança, eu pus o seu sangue numa
penha descalvada, para que não fosse coberto. Portanto, assim
diz o Senhor Deus: Ai da cidade sanguinária! Também eu farei uma
grande fogueira. Amontoa muita lenha, acende o fogo, ferve
bem a carne, e tempera o caldo, e ardam os ossos. Então a
porás vazia sobre as suas brasas, para que ela aqueça, e se queime o
seu cobre, e se funda a sua imundícia no meio dela, e se consuma a
sua escuma. Ela com mentiras se cansou; e não saiu dela a sua
muita escuma; ao fogo irá a sua escuma. Na imundícia está a
infâmia, porquanto te purifiquei, e não permaneceste pura; nunca
mais serás purificada da tua imundícia, enquanto eu não fizer
descansar sobre ti a minha indignação. Eu, o Senhor, o disse:
viva isso, e o farei, não me tornarei atrás, e não pouparei, nem me
arrependerei; conforme os teus caminhos, e conforme os teus feitos,
te julgarão, diz o Senhor Deus.

E veio a mim a palavra do Senhor, dizendo: Filho do homem,
eis que, de um golpe tirarei de ti o desejo dos teus olhos, mas não
lamentarás, nem chorarás, nem te correrão as lágrimas. Geme
em silêncio, não faças luto por mortos; ata o teu turbante, e põe
nos pés os teus sapatos, e não cubras os teus lábios, e não comas o
pão dos homens. E falei ao povo pela manhã, e à tarde morreu
minha mulher; e fiz pela manhã como me foi mandado. E o povo
me disse: Porventura não nos farás saber o que significam para nós
estas coisas que estás fazendo? E eu lhes disse: Veio a mim a
palavra do Senhor, dizendo: Dize à casa de Israel: Assim diz
o Senhor Deus: Eis que eu profanarei o meu santuário, a glória da
vossa força, o desejo dos vossos olhos, e o anelo\footnote{Desejo
intenso; anelação, anélito, aspiração.} das vossas almas; e vossos
filhos e vossas filhas, que deixastes, cairão à espada. E
fareis como eu fiz; não vos cobrireis os lábios, e não comereis o
pão dos homens. E tereis nas cabeças os vossos turbantes, e
os vossos sapatos nos pés; não lamentareis, nem chorareis, mas
definhar-vos-eis nas vossas maldades, e gemereis uns com os outros.
Assim vos servirá Ezequiel de sinal; conforme tudo quanto ele
fez, fareis; quando isso suceder, sabereis que eu sou o Senhor Deus.
E quanto a ti, filho do homem, não sucederá que no dia que eu
lhes tirar a sua força, a alegria da sua glória, o desejo dos seus
olhos, e o anelo de suas almas, com seus filhos e suas filhas,
nesse dia virá ter contigo aquele que escapar, para te dar
notícias pessoalmente? Naquele dia abrir-se-á a tua boca para
com aquele que escapar, e falarás, e não mais ficarás mudo; assim
virás a ser para eles um sinal, e saberão que eu sou o Senhor.

\medskip

\lettrine{25} E veio a mim a palavra do Senhor, dizendo:
Filho do homem, dirige o teu rosto contra os filhos de Amom, e
profetiza contra eles. E dize aos filhos de Amom: Ouvi a palavra
do Senhor Deus: Assim diz o Senhor Deus: Porquanto tu disseste: Ah!
contra o meu santuário, quando foi profanado; e contra a terra de
Israel, quando foi assolada; e contra a casa de Judá, quando foi ao
cativeiro; portanto, eis que te entregarei em possessão aos do
oriente, e em ti estabelecerão os seus acampamentos, e porão em ti
as suas moradas; eles comerão os teus frutos, e eles beberão o teu
leite. E farei de Rabá uma estrebaria de camelos, e dos filhos
de Amom um curral de ovelhas; e sabereis que eu sou o Senhor.
Porque assim diz o Senhor Deus: Porquanto bateste com as mãos, e
pateaste\footnote{Patear: bater com os pés ou patas. Calcar com os
pés ou patas. Bater com os pés no chão, em sinal de protesto ou
desagrado.} com os pés, e com todo o desprezo do teu coração te
alegraste contra a terra de Israel, portanto, eis que eu tenho
estendido a minha mão sobre ti, e te darei por despojo aos gentios,
e te arrancarei dentre os povos, e te destruirei dentre as terras, e
acabarei de todo contigo; e saberás que eu sou o Senhor.

Assim diz o Senhor Deus: Porquanto dizem Moabe e Seir: Eis que a
casa de Judá é como todos os gentios; portanto, eis que eu
abrirei o lado de Moabe desde as cidades, desde as suas cidades da
fronteira, a glória da terra, Bete-Jesimote, Baal-Meom, e
Quiriataim. E aos do oriente, contra os filhos de Amom, o
entregarei em possessão, para que não haja memória dos filhos de
Amom entre as nações. Também executarei juízos sobre Moabe, e
saberão que eu sou o Senhor. Assim diz o Senhor Deus:
Porquanto Edom se houve vingativamente para com a casa de Judá, e se
fez culpadíssimo, quando se vingou deles; portanto assim diz
o Senhor Deus: Também estenderei a minha mão sobre Edom, e
arrancarei dela homens e animais; e a tornarei em deserto, e desde
Temã até Dedã cairão à espada. E exercerei a minha vingança
sobre Edom, pela mão do meu povo de Israel; e farão em Edom segundo
a minha ira e segundo o meu furor; e conhecerão a minha vingança,
diz o Senhor Deus. Assim diz o Senhor Deus: Porquanto os
filisteus se houveram vingativamente, e executaram vingança com
desprezo de coração, para destruírem com perpétua inimizade,
portanto assim diz o Senhor Deus: Eis que eu estendo a minha
mão sobre os filisteus, e arrancarei os quereteus, e destruirei o
restante da costa do mar. E executarei sobre eles grandes
vinganças, com furiosos castigos, e saberão que eu sou o Senhor,
quando eu tiver exercido a minha vingança sobre eles.

\medskip

\lettrine{26} E sucedeu no undécimo ano, ao primeiro do mês,
que veio a mim a palavra do Senhor, dizendo: Filho do homem,
visto que Tiro disse contra Jerusalém: Ah! está quebrada a porta dos
povos; virou-se para mim; eu me encherei, agora que ela está
assolada; portanto assim diz o Senhor Deus: Eis que eu estou
contra ti, ó Tiro, e farei subir contra ti muitas nações, como o mar
faz subir as suas ondas, elas destruirão os muros de Tiro, e
derrubarão as suas torres; e eu lhe varrerei o seu pó, e dela farei
uma penha descalvada. No meio do mar virá a ser um
enxugadouro\footnote{Local onde se estendem roupas ou se colocam
objetos para enxugar. Lugar onde os tijolos são enxambrados antes de
ir ao forno.} das redes; porque eu o falei, diz o Senhor Deus; e
servirá de despojo para as nações. E suas filhas, que estão no
campo, serão mortas à espada; e saberão que eu sou o Senhor.
Porque assim diz o Senhor Deus: Eis que eu, desde o norte,
trarei contra Tiro a Nabucodonosor, rei de Babilônia, o rei dos
reis, com cavalos, e com carros, e com cavaleiros, e companhias, e
muito povo. As tuas filhas que estão no campo, ele as matará à
espada, e levantará um baluarte contra ti, e fundará uma trincheira
contra ti, e levantará paveses contra ti. E disporá os seus
aríetes contra os teus muros, e derrubará as tuas torres com os seus
machados. Por causa da multidão de seus cavalos te cobrirá o
seu pó; os teus muros tremerão com o estrondo dos cavaleiros, e das
rodas, e dos carros, quando ele entrar pelas tuas portas, como os
homens entram numa cidade em que se fez brecha. Com os cascos
dos seus cavalos pisará todas as tuas ruas; ao teu povo matará à
espada, e as tuas fortes colunas cairão por terra. E roubarão
as tuas riquezas, e saquearão as tuas mercadorias, e derrubarão os
teus muros, e arrasarão as tuas casas agradáveis; e lançarão no meio
das águas as tuas pedras, e as tuas madeiras, e o teu pó. E
farei cessar o ruído das tuas cantigas, e o som das tuas harpas não
se ouvirá mais. E farei de ti uma penha descalvada; virás a
ser um enxugadouro das redes, nunca mais serás edificada; porque eu
o Senhor o falei, diz o Senhor Deus.

Assim diz o Senhor Deus a Tiro: Porventura não tremerão as ilhas
com o estrondo da tua queda, quando gemerem os feridos, quando se
fizer uma espantosa matança no meio de ti? E todos os
príncipes do mar descerão dos seus tronos, e tirarão de si os seus
mantos, e despirão as suas vestes bordadas; se vestirão de tremores,
sobre a terra se assentarão, e estremecerão a cada momento; e por
tua causa pasmarão. E levantarão uma lamentação sobre ti, e
te dirão: Como pereceste, ó bem povoada e afamada cidade, que foste
forte no mar; ela e os seus moradores, que atemorizaram a todos os
seus habitantes! Agora, estremecerão as ilhas no dia da tua
queda; sim, as ilhas, que estão no mar, turbar-se-ão com tua saída.
Porque assim diz o Senhor Deus: Quando eu te fizer uma cidade
assolada, como as cidades que não se habitam, quando eu fizer subir
sobre ti o abismo, e as muitas águas te cobrirem, então te
farei descer com os que descem à cova, ao povo antigo, e te farei
habitar nas mais baixas partes da terra, em lugares desertos
antigos, com os que descem à cova, para que não sejas habitada; e
estabelecerei a glória na terra dos viventes. Farei de ti um
grande espanto, e não mais existirás; e quando te buscarem então
nunca mais serás achada para sempre, diz o Senhor Deus.

\medskip

\lettrine{27} E veio a mim a palavra do Senhor, dizendo:
Tu pois, ó filho do homem, levanta uma lamentação sobre Tiro.
E dize a Tiro, que habita nas entradas do mar, e negocia com os
povos em muitas ilhas: Assim diz o Senhor Deus: Ó Tiro, tu dizes: Eu
sou perfeita em formosura. No coração dos mares estão os teus
termos; os que te edificaram aperfeiçoaram a tua formosura.
Fabricaram todos os teus conveses\footnote{Convés: qualquer dos
pisos ou pavimentos de um navio, esp. aqueles a céu aberto, ou
protegidos por toldo. Nos veleiros do séc. XIX, piso da coberta da
bateria de bocas de fogo (no caso de navio de uma só bateria), ou da
primeira coberta dentre aquelas em que se alojava a artilharia. Nos
veleiros, até o séc. XVIII, parte do piso descoberto entre o mastro
grande e a proa ou o castelo de proa.} de faias de Senir; trouxeram
cedros do Líbano para te fazerem mastros. Fizeram os teus remos
de carvalhos de Basã; os teus bancos fizeram-nos de marfim engastado
em buxo\footnote{Arbusto ou árvore pequena (Buxus sempervirens), que
ocorre na Europa e nas costas do Mediterrâneo, de folhas coriáceas,
flores apétalas, brancas, fétidas, e cápsulas com sementes trígonas;
buxeira, buxeiro, buxo-arborescente, buxo-arbustivo [Tem muitas
variedades, cultivadas pela madeira nobre, dura, pesada, amarelada e
marrom, ou como ornamentais.] . A madeira dessas árvores ou
arbustos.} das ilhas dos quiteus. Linho fino bordado do Egito
era a tua cortina, para te servir de vela; azul e púrpura das ilhas
de Elisá era a tua cobertura. Os moradores de Sidom e de Arvade
foram os teus remadores; os teus sábios, ó Tiro, que se achavam em
ti, esses foram os teus pilotos. Os anciãos de Gebal e seus
sábios foram em ti os que consertavam as tuas fendas; todos os
navios do mar e os marinheiros se acharam em ti, para tratarem dos
teus negócios. Os persas, e os lídios, e os de Pute eram no
teu exército os teus soldados; escudos e capacetes penduraram em ti;
eles manifestaram a tua beleza. Os filhos de Arvade e o teu
exército estavam sobre os teus muros em redor, e os gamaditas nas
tuas torres; penduravam os seus escudos nos teus muros em redor;
eles aperfeiçoavam a tua formosura. Társis negociava contigo,
por causa da abundância de toda a casta de riquezas; com prata,
ferro, estanho e chumbo, negociavam em tuas feiras. Javã,
Tubal e Meseque eram teus mercadores; em troca das tuas mercadorias
davam pessoas de homens e objetos de bronze. Os da casa de
Togarma trocavam pelas tuas mercadorias, cavalos, e cavaleiros e
mulos. Os filhos de Dedã eram os teus mercadores; muitas
ilhas eram o comércio da tua mão; dentes de marfim e pau de ébano
tornavam a dar-te em presente. A Síria negociava contigo por
causa da multidão das tuas manufaturas; pelas tuas mercadorias davam
esmeralda, púrpura, obra bordada, linho fino, corais e
ágata.\footnote{Variedade de calcedônia que apresenta anéis
concêntricos, ger. de várias cores, e que é us. como gema na
confecção de jóias e objetos ornamentais. Ferro esmaltado.}
Judá e a terra de Israel, eram os teus mercadores; pelas tuas
mercadorias trocavam trigo de Minite, e Panague, e mel, azeite e
bálsamo. Damasco negociava contigo, por causa da multidão das
tuas obras, por causa da abundância de toda a sorte de riqueza,
dando em troca vinho de Helbom e lã branca. Também Dã e Javã,
de Uzal, pelas tuas mercadorias, davam em troca ferro trabalhado,
cássia\footnote{Design. comum às plantas do gên. Cassia, da fam. das
leguminosas, subfam. cesalpinioídea, que reúne cerca de 30 spp.
arbóreas, de ocorrência pantropical, ger. cultivadas como
ornamentais e tb. como medicinais.} e cálamo\footnote{Erva (Acorus
calamus) que ocorre desde as regiões temperadas do hemisfério norte
até a Índia e a Nova Guiné, de que se extrai óleo essencial us. em
medicina, o óleo sagrado a que se refere a Bíblia; ácoro-aromático,
cálamo, cálamo-aromático, cana-cheirosa, dringo.} aromático, que
assim entravam no teu comércio. Dedã negociava contigo com
panos preciosos para carros. A Arábia, e todos os príncipes
de Quedar, eram mercadores ao teu serviço, com cordeiros, carneiros
e bodes; nestas coisas negociavam contigo. Os mercadores de
Sabá e Raamá eram os teus mercadores; em todos os seus mais finos
aromas, em toda a pedra preciosa e ouro, negociaram nas tuas feiras.
Harã, e Cane e Éden, os mercadores de Sabá, Assur e Quilmade
negociavam contigo. Estes eram teus mercadores em roupas
escolhidas, em pano de azul, e bordados, e em cofres de roupas
preciosas, amarrados com cordas e feitos de cedros, entre tua
mercadoria. Os navios de Társis eram as tuas caravanas que
traziam tuas mercadorias; e te encheste, e te glorificaste muito no
meio dos mares.

Os teus remadores te conduziram sobre grandes águas; o vento
oriental te quebrou no meio dos mares. As tuas riquezas, as
tuas feiras, e tuas mercadorias, os teus marinheiros, os teus
pilotos, os que consertavam as tuas fendas, os que faziam os teus
negócios, e todos os teus soldados, que estão em ti, juntamente com
toda a tua companhia, que está no meio de ti, cairão no meio dos
mares no dia da tua queda, ao estrondo da gritaria dos teus
pilotos tremerão os arrabaldes. E todos os que pegam no remo,
os marinheiros, e todos os pilotos do mar descerão de seus navios, e
pararão em terra. E farão ouvir a sua voz sobre ti, e
gritarão amargamente; e lançarão pó sobre as cabeças, e na cinza se
revolverão. E far-se-ão calvos por tua causa, e cingir-se-ão
de sacos, e chorarão sobre ti com amargura de alma, e com amarga
lamentação. E no seu pranto levantarão uma lamentação sobre
ti, e lamentarão sobre ti, dizendo: Quem foi como Tiro, como a que
foi destruída no meio do mar? Quando as tuas mercadorias
saiam pelos mares, fartaste a muitos povos; com a multidão das tuas
riquezas e do teu negócio, enriqueceste os reis da terra. No
tempo em que foste quebrantada pelos mares, nas profundezas das
águas, caíram, no meio de ti, os teus negócios e toda a tua
companhia. Todos os moradores das ilhas estão a teu respeito
cheios de espanto; e os seus reis tremeram sobremaneira, e ficaram
perturbados nos seus rostos; os mercadores dentre os povos
assobiaram contra ti; tu te tornaste em grande espanto, e jamais
subsistirá.

\medskip

\lettrine{28} E veio a mim a palavra do Senhor, dizendo:
Filho do homem, dize ao príncipe de Tiro: Assim diz o Senhor
Deus: Porquanto o teu coração se elevou e disseste: Eu sou Deus,
sobre a cadeira de Deus me assento no meio dos mares; e não passas
de homem, e não és Deus, ainda que estimas o teu coração como se
fora o coração de Deus; eis que tu és mais sábio que Daniel; e
não há segredo algum que se possa esconder de ti. Pela tua
sabedoria e pelo teu entendimento alcançaste para ti riquezas, e
adquiriste ouro e prata nos teus tesouros. Pela extensão da tua
sabedoria no teu comércio aumentaste as tuas riquezas; e eleva-se o
teu coração por causa das tuas riquezas; portanto, assim diz o
Senhor Deus: Porquanto estimas o teu coração, como se fora o coração
de Deus, por isso eis que eu trarei sobre ti estrangeiros, os
mais terríveis dentre as nações, os quais desembainharão as suas
espadas contra a formosura da tua sabedoria, e mancharão o teu
resplendor. Eles te farão descer à cova e morrerás da morte dos
traspassados no meio dos mares. Acaso dirás ainda diante daquele
que te matar: Eu sou Deus? mas tu és homem, e não Deus, na mão do
que te traspassa. Da morte dos incircuncisos morrerás, por
mão de estrangeiros, porque eu o falei, diz o Senhor Deus.

Veio a mim a palavra do Senhor, dizendo: Filho do homem,
levanta uma lamentação sobre o rei de Tiro, e dize-lhe: Assim diz o
Senhor Deus: Tu eras o selo da medida, cheio de sabedoria e perfeito
em formosura. Estiveste no Éden, jardim de Deus; de toda a
pedra preciosa era a tua cobertura: sardônia, topázio, diamante,
turquesa, ônix, jaspe, safira, carbúnculo, esmeralda e ouro; em ti
se faziam os teus tambores e os teus pífaros; no dia em que foste
criado foram preparados. Tu eras o querubim, ungido para
cobrir, e te estabeleci; no monte santo de Deus estavas, no meio das
pedras afogueadas\footnote{Muito quente, escaldante.} andavas.
Perfeito eras nos teus caminhos, desde o dia em que foste
criado, até que se achou iniqüidade em ti. Na multiplicação
do teu comércio encheram o teu interior de violência, e pecaste; por
isso te lancei, profanado, do monte de Deus, e te fiz perecer, ó
querubim cobridor, do meio das pedras afogueadas. Elevou-se o
teu coração por causa da tua formosura, corrompeste a tua sabedoria
por causa do teu resplendor; por terra te lancei, diante dos reis te
pus, para que olhem para ti. Pela multidão das tuas
iniqüidades, pela injustiça do teu comércio profanaste os teus
santuários; eu, pois, fiz sair do meio de ti um fogo, que te
consumiu e te tornei em cinza sobre a terra, aos olhos de todos os
que te vêem. Todos os que te conhecem entre os povos estão
espantados de ti; em grande espanto te tornaste, e nunca mais
subsistirá.

E veio a mim a palavra do Senhor, dizendo: Filho do homem,
dirige o teu rosto contra Sidom, e profetiza contra ela, e
dize: Assim diz o Senhor Deus: Eis-me contra ti, ó Sidom, e serei
glorificado no meio de ti; e saberão que eu sou o Senhor, quando
nela executar juízos e nela me santificar. Porque enviarei
contra ela a peste, e o sangue nas suas ruas, e os traspassados
cairão no meio dela, estando a espada contra ela por todos os lados;
e saberão que eu sou o Senhor. E a casa de Israel nunca mais
terá espinho que a fira, nem espinho que cause dor, entre os que se
acham ao redor deles e que os desprezam; e saberão que eu sou o
Senhor Deus. Assim diz o Senhor Deus: Quando eu congregar a
casa de Israel dentre os povos entre os quais estão espalhados, e eu
me santificar entre eles, perante os olhos dos gentios, então
habitarão na sua terra que dei a meu servo, a Jacó. E
habitarão nela seguros, e edificarão casas, e plantarão vinhas, e
habitarão seguros, quando eu executar juízos contra todos os que
estão ao seu redor e que os desprezam; e saberão que eu sou o Senhor
seu Deus.

\medskip

\lettrine{29} No décimo ano, no décimo mês, no dia doze do
mês, veio a mim a palavra do Senhor, dizendo: Filho do homem,
dirige o teu rosto contra Faraó, rei do Egito, e profetiza contra
ele e contra todo o Egito. Fala, e dize: Assim diz o Senhor
Deus: Eis-me contra ti, ó Faraó, rei do Egito, grande dragão, que
pousas no meio dos teus rios, e que dizes: O meu rio é meu, e eu o
fiz para mim. Mas eu porei anzóis em teus queixos, e farei que
os peixes dos teus rios se apeguem às tuas escamas; e tirar-te-ei do
meio dos teus rios, e todos os peixes dos teus rios se apegarem às
tuas escamas. E te deixarei no deserto, a ti e a todo o peixe
dos teus rios; sobre a face do campo cairás; não serás recolhido nem
ajuntado; aos animais da terra e às aves do céu te dei por
mantimento. E saberão todos os moradores do Egito que eu sou o
Senhor, porquanto se tornaram um bordão de cana para a casa de
Israel. Tomando-te eles pela mão, te quebraste, e lhes rasgaste
todo o ombro; e quando se apoiaram em ti, te quebraste, e lhes
fazias tremer todos os seus lombos.

Portanto, assim diz o Senhor Deus: Eis que eu trarei sobre ti a
espada, e de ti destruirei homem e animal, e a terra do Egito se
tornará em desolação e deserto; e saberão que eu sou o Senhor,
porquanto disse: O rio é meu, e eu o fiz. Portanto, eis que
eu estou contra ti, e contra os teus rios; e tornarei a terra do
Egito deserta, em completa desolação, desde a torre de Syene até aos
confins da Etiópia. Não passará por ela pé de homem, nem pé
de animal passará por ela, nem será habitada quarenta anos.
Porque tornarei a terra do Egito em desolação no meio das
terras desoladas; e as suas cidades entre as cidades desertas se
tornarão em desolação por quarenta anos; e espalharei os egípcios
entre as nações, e os dispersarei pelas terras. Porém, assim
diz o Senhor Deus: Ao fim de quarenta anos ajuntarei os egípcios
dentre os povos entre os quais foram espalhados. E removerei
o cativeiro dos egípcios, e os farei voltar à terra de Patros, à
terra de sua origem; e serão ali um reino humilde; mais
humilde se fará do que os outros reinos, e nunca mais se
exalçará\footnote{Exalçar: Atribuir grandeza a; engrandecer,
exaltar. Conduzir para o alto; elevar, guindar.} sobre as nações;
porque os diminuirei, para que não dominem sobre as nações. E
não será mais a confiança da casa de Israel, para lhes trazer à
lembrança a sua iniqüidade, quando olharem para trás deles; antes
saberão que eu sou o Senhor Deus.

E sucedeu que, no ano vinte e sete, no primeiro mês, no primeiro
dia do mês, veio a mim a palavra do Senhor, dizendo: Filho do
homem, Nabucodonosor, rei de Babilônia, fez com que o seu exército
prestasse um grande serviço contra Tiro; toda a cabeça se tornou
calva, e todo o ombro se pelou; e não houve paga de Tiro para ele,
nem para o seu exército, pelo serviço que prestou contra ela.
Portanto, assim diz o Senhor Deus: Eis que eu darei a
Nabucodonosor, rei de Babilônia, a terra do Egito; e levará a sua
multidão, e tomará o seu despojo, e roubará a sua presa, e isto será
a recompensa para o seu exército. Como recompensa do seu
trabalho, com que serviu contra ela, lhe dei a terra do Egito;
porquanto trabalharam por mim, diz o Senhor Deus. Naquele dia
farei brotar o poder na casa de Israel, e abrirei a tua boca no meio
deles; e saberão que eu sou o Senhor.

\medskip

\lettrine{30} E veio a mim a palavra do Senhor, dizendo:
Filho do homem, profetiza, e dize: Assim diz o Senhor Deus:
Gemei: Ah! Aquele dia! Porque está perto o dia, sim, está perto
o dia do Senhor; dia nublado; será o tempo dos gentios. A espada
virá ao Egito, e haverá grande dor na Etiópia, quando caírem os
traspassados no Egito; e tomarão a sua multidão, e serão destruídos
os seus fundamentos. Etiópia, Pute e Lude, e toda a mistura de
gente, e Cube, e os homens da terra da liga, juntamente com eles
cairão à espada. Assim diz o Senhor: Também cairão os que sustém
o Egito, e descerá a soberba de seu poder; desde a torre de Syene
ali cairão à espada, diz o Senhor Deus. E serão desolados no
meio das terras assoladas; e as suas cidades estarão no meio das
cidades desertas. E saberão que eu sou o Senhor, quando eu puser
fogo no Egito, e forem destruídos todos os que lhe davam auxílio.
Naquele dia sairão mensageiros de diante de mim em navios, para
espantarem a Etiópia descuidada; e haverá neles grandes dores, como
no dia do Egito; pois, eis que já vem. Assim diz o Senhor
Deus: Eu, pois, farei cessar a multidão do Egito, por mão de
Nabucodonosor, rei de Babilônia. Ele e o seu povo com ele, os
mais terríveis das nações, serão levados para destruírem a terra; e
desembainharão as suas espadas contra o Egito, e encherão a terra de
mortos. E secarei os rios, e venderei a terra entregando-a na
mão dos maus, e assolarei a terra e a sua plenitude pela mão dos
estrangeiros; eu, o Senhor, o disse. Assim diz o Senhor Deus:
Também destruirei os ídolos, e farei cessar as imagens de Nofe; e
não haverá mais um príncipe da terra do Egito; e porei o temor na
terra do Egito. E assolarei a Patros, e porei fogo a Zoã, e
executarei juízos em Nô. E derramarei o meu furor sobre Sim,
a fortaleza do Egito, e exterminarei a multidão de Nô. E
porei fogo no Egito; Sim terá grande dor, e Nô será fendida, e Nofe
terá angústias cotidianas. Os jovens de Áven e Pi-Besete
cairão à espada, e as cidades irão em cativeiro. E em Tafnes
se escurecerá o dia, quando eu quebrar ali os jugos do Egito, e nela
cessar a soberba do seu poder; uma nuvem a cobrirá, e suas filhas
irão em cativeiro. Assim executarei juízos no Egito, e
saberão que eu sou o Senhor.

E sucedeu que, no ano undécimo, no primeiro mês, aos sete do mês,
veio a mim a palavra do Senhor, dizendo: Filho do homem, eu
quebrei o braço de Faraó, rei do Egito, e eis que não foi atado para
se lhe aplicar remédios, nem lhe colocarão ligaduras para o atar, a
fim de torná-lo forte, para pegar na espada. Portanto assim
diz o Senhor Deus: Eis que eu estou contra Faraó, rei do Egito, e
quebrarei os seus braços, assim o forte como o que está quebrado, e
farei cair da sua mão a espada. E espalharei os egípcios
entre as nações, e os dispersarei pelas terras. E
fortalecerei os braços do rei de Babilônia, e porei a minha espada
na sua mão; mas quebrarei os braços de Faraó, e diante dele gemerá
como geme o traspassado. Eu fortalecerei os braços do rei de
Babilônia, mas os braços de Faraó cairão; e saberão que eu sou o
Senhor, quando eu puser a minha espada na mão do rei de Babilônia, e
ele a estender sobre a terra do Egito. E espalharei os
egípcios entre as nações, e os dispersarei entre as terras; assim
saberão que eu sou o Senhor.

\medskip

\lettrine{31} E sucedeu, no ano undécimo, no terceiro mês, ao
primeiro do mês, que veio a mim a palavra do Senhor, dizendo:
Filho do homem, dize a Faraó, rei do Egito, e à sua multidão: A
quem és semelhante na tua grandeza? Eis que a Assíria era um
cedro no Líbano, de ramos formosos, de sombrosa ramagem e de alta
estatura, e a sua copa estava entre os ramos espessos. As águas
o fizeram crescer, o abismo o exalçou; as suas correntes corriam em
torno da sua plantação, e ele enviava os regatos a todas as árvores
do campo. Por isso se elevou a sua estatura sobre todas as
árvores do campo, e se multiplicaram os seus ramos, e se alongaram
as suas varas, por causa das muitas águas quando brotava. Todas
as aves do céu se aninhavam nos seus ramos, e todos os animais do
campo geravam debaixo dos seus ramos, e todas as grandes nações
habitavam à sua sombra. Assim era ele formoso na sua grandeza,
na extensão dos seus ramos, porque a sua raiz estava junto às muitas
águas. Os cedros, no jardim de Deus, não o podiam obscurecer; as
faias não igualavam os seus ramos, e os castanheiros não eram como
os seus renovos; nenhuma árvore no jardim de Deus se assemelhou a
ele na sua formosura. Formoso o fiz com a multidão dos seus
ramos; e todas as árvores do Éden, que estavam no jardim de Deus,
tiveram inveja dele.

Portanto assim diz o Senhor Deus: Porquanto te elevaste na tua
estatura, e se levantou a sua copa no meio dos espessos ramos, e o
seu coração se exalçou na sua altura, eu o entregarei na mão
do mais poderoso dos gentios, que lhe dará o tratamento merecido;
pela sua impiedade o lançarei fora. E estrangeiros, das mais
terríveis nações o cortarão, e deixá-lo-ão; cairão os seus ramos
sobre os montes e por todos os vales, e os seus renovos serão
quebrados por todos os rios da terra; e todos os povos da terra se
retirarão da sua sombra, e o deixarão. Todas as aves do céu
habitarão sobre a sua ruína, e todos os animais do campo se
acolherão sob os seus renovos; para que todas as árvores
junto às águas não se exaltem na sua estatura, nem levantem a sua
copa no meio dos ramos espessos, nem as que bebem as águas venham a
confiar em si, por causa da sua altura; porque todos estão entregues
à morte, até à terra mais baixa, no meio dos filhos dos homens, com
os que descem à cova. Assim diz o Senhor Deus: No dia em que
ele desceu ao inferno, fiz eu que houvesse luto; fiz cobrir o
abismo, por sua causa, e retive as suas correntes, e detiveram-se as
muitas águas; e cobri o Líbano de preto por causa dele, e todas as
árvores do campo por causa dele desfaleceram. Ao som da sua
queda fiz tremer as nações, quando o fiz descer ao inferno, com os
que descem à cova; e todas as árvores do Éden, a flor e o melhor do
Líbano, todas as árvores que bebem águas, se consolavam nas partes
mais baixas da terra. Também estes com ele descerão ao
inferno a juntar-se aos que foram traspassados à espada, sim, aos
que foram seu braço, e que habitavam à sombra no meio dos gentios.
A quem, pois, és semelhante em glória e em grandeza entre as
árvores do Éden? Todavia serás precipitado com as árvores do Éden às
partes mais baixas da terra; no meio dos incircuncisos jazerás com
os que foram traspassados à espada; este é Faraó e toda a sua
multidão, diz o Senhor Deus.

\medskip

\lettrine{32} E sucedeu que, no ano duodécimo, no duodécimo
mês, ao primeiro do mês, veio a mim a palavra do Senhor, dizendo:
Filho do homem, levanta uma lamentação sobre Faraó, rei do
Egito, e dize-lhe: Eras semelhante a um filho do leão entre as
nações, mas tu és como uma baleia nos mares, e rompias os teus rios,
e turbavas as águas com os teus pés, e pisavas os teus rios.
Assim diz o Senhor Deus: Portanto, estenderei sobre ti a minha
rede com reunião de muitos povos, e te farão subir na minha rede.
Então te deixarei em terra; sobre a face do campo te lançarei, e
farei pousar sobre ti todas as aves do céu, e fartarei de ti os
animais de toda a terra. E porei as tuas carnes sobre os montes,
e encherei os vales da tua altura. E regarei com o teu sangue a
terra onde nadas, até aos montes; e os rios se encherão de ti.
E, apagando-te eu, cobrirei os céus, e enegrecerei as suas
estrelas; ao sol encobrirei com uma nuvem, e a lua não fará
resplandecer a sua luz. Todas as brilhantes luzes do céu
enegrecerei sobre ti, e trarei trevas sobre a tua terra, diz o
Senhor Deus. E afligirei os corações de muitos povos, quando eu
levar a tua destruição entre as nações, às terras que não
conheceste. E farei com que muitos povos fiquem pasmados de
ti, e os seus reis tremam sobremaneira, quando eu brandir a minha
espada ante os seus rostos; e estremecerão a cada momento, cada um
pela sua vida, no dia da tua queda. Porque assim diz o Senhor
Deus: A espada do rei de Babilônia virá sobre ti. Farei cair
a tua multidão pelas espadas dos poderosos, que são todos os mais
terríveis das nações; e destruirão a soberba do Egito, e toda a sua
multidão será destruída. E exterminarei todos os seus animais
sobre as muitas águas; nem as turbará mais pé de homem, nem as
turbarão unhas de animais. Então farei assentar as suas
águas, e farei correr os seus rios como o azeite, diz o Senhor Deus.
Quando eu tornar a terra do Egito em desolação, e ela for
despojada da sua plenitude, e quando ferir a todos os que habitam
nela, então saberão que eu sou o Senhor. Esta é a lamentação
que se fará; que as filhas das nações farão; sobre o Egito e sobre
toda a sua multidão, diz o Senhor Deus.

E sucedeu que, no ano duodécimo, aos quinze do mês, veio a mim a
palavra do Senhor, dizendo: Filho do homem, pranteia sobre a
multidão do Egito, e faze-a descer, a ela e às filhas das nações
magníficas, às partes mais baixas da terra, juntamente com os que
descem à cova. A quem sobrepujas tu em formosura? Desce, e
deita-te com os incircuncisos. No meio daqueles que foram
mortos à espada cairão; à espada ela está entregue; arrastai-a e a
toda a sua multidão. Os mais poderosos dos fortes lhe falarão
desde o meio do inferno, com os que a socorrem; desceram, jazeram
com os incircuncisos mortos à espada. Ali está Assur com toda
a sua multidão; em redor dele estão os seus sepulcros; todos eles
mortos, abatidos à espada. Os seus sepulcros foram postos nas
extremidades da cova, e a sua multidão está em redor do seu
sepulcro; todos eles mortos, abatidos à espada; os que tinham
causado espanto na terra dos viventes. Ali está Elão com toda
a sua multidão em redor do seu sepulcro; todos eles mortos, abatidos
à espada; desceram incircuncisos às partes mais baixas da terra,
causaram terror na terra dos viventes e levaram a sua vergonha com
os que desceram à cova. No meio dos mortos lhe puseram uma
cama, entre toda a sua multidão; ao redor dele estão os seus
sepulcros; todos eles são incircuncisos, mortos à espada; porque
causaram terror na terra dos viventes, e levaram a sua vergonha com
os que desceram à cova; foi posto no meio dos mortos. Ali
estão Meseque, Tubal e toda a sua multidão; ao redor deles estão os
seus sepulcros; todos eles são incircuncisos, e mortos à espada,
porquanto causaram terror na terra dos viventes. Porém não
jazerão com os poderosos que caíram dos incircuncisos, os quais
desceram ao inferno com as suas armas de guerra e puseram as suas
espadas debaixo das suas cabeças; e a sua iniqüidade está sobre os
seus ossos, porquanto eram o terror dos fortes na terra dos
viventes. Também tu serás quebrado no meio dos incircuncisos,
e jazerás com os que foram mortos à espada. Ali está Edom, os
seus reis e todos os seus príncipes, que com o seu poder foram
postos com os que foram mortos à espada; estes jazem com os
incircuncisos e com os que desceram à cova. Ali estão os
príncipes do norte, todos eles, e todos os sidônios, que desceram
com os mortos, envergonhados com o terror causado pelo seu poder; e
jazem incircuncisos com os que foram mortos à espada, e levam a sua
vergonha com os que desceram à cova. Faraó os verá, e se
consolará com toda a sua multidão; sim, o próprio Faraó, e todo o
seu exército, mortos à espada, diz o Senhor Deus. Porque
também eu pus o meu espanto na terra dos viventes; por isso jazerá
no meio dos incircuncisos, com os mortos à espada, Faraó e toda a
sua multidão, diz o Senhor Deus.

\medskip

\lettrine{33} E veio a mim a palavra do Senhor, dizendo:
Filho do homem, fala aos filhos do teu povo, e dize-lhes: Quando
eu fizer vir a espada sobre a terra, e o povo da terra tomar um
homem dos seus termos, e o constituir por seu atalaia; e, vendo
ele que a espada vem sobre a terra, tocar a trombeta e avisar o
povo; se aquele que ouvir o som da trombeta, não se der por
avisado, e vier a espada, e o alcançar, o seu sangue será sobre a
sua cabeça. Ele ouviu o som da trombeta, e não se deu por
avisado, o seu sangue será sobre ele; mas o que se dá por avisado
salvará a sua vida. Mas, se quando o atalaia vir que vem a
espada, e não tocar a trombeta, e não for avisado o povo, e a espada
vier, e levar uma vida dentre eles, este tal foi levado na sua
iniqüidade, porém o seu sangue requererei da mão do atalaia. A
ti, pois, ó filho do homem, te constituí por atalaia sobre a casa de
Israel; tu, pois, ouvirás a palavra da minha boca, e lha anunciarás
da minha parte. Se eu disser ao ímpio: Ó ímpio, certamente
morrerás; e tu não falares, para dissuadir ao ímpio do seu caminho,
morrerá esse ímpio na sua iniqüidade, porém o seu sangue eu o
requererei da tua mão. Mas, se advertires o ímpio do seu
caminho, para que dele se converta, e ele não se converter do seu
caminho, ele morrerá na sua iniqüidade; mas tu livraste a tua alma.

Tu, pois, filho do homem, dize à casa de Israel: Assim falais
vós, dizendo: Visto que as nossas transgressões e os nossos pecados
estão sobre nós, e nós desfalecemos neles, como viveremos então?
Dize-lhes: Vivo eu, diz o Senhor Deus, que não tenho prazer
na morte do ímpio, mas em que o ímpio se converta do seu caminho, e
viva. Convertei-vos, convertei-vos dos vossos maus caminhos; pois,
por que razão morrereis, ó casa de Israel? Tu, pois, filho do
homem, dize aos filhos do teu povo: A justiça do justo não o livrará
no dia da sua transgressão; e, quanto à impiedade do ímpio, não
cairá por ela, no dia em que se converter da sua impiedade; nem o
justo poderá viver pela sua justiça no dia em que pecar.
Quando eu disser ao justo que certamente viverá, e ele,
confiando na sua justiça, praticar a iniqüidade, não virão à memória
todas as suas justiças, mas na sua iniqüidade, que pratica, ele
morrerá. Quando eu também disser ao ímpio: Certamente
morrerás; se ele se converter do seu pecado, e praticar juízo e
justiça, restituindo esse ímpio o penhor, indenizando o que
furtou, andando nos estatutos da vida, e não praticando iniqüidade,
certamente viverá, não morrerá. De todos os seus pecados que
cometeu não se terá memória contra ele; juízo e justiça fez,
certamente viverá. Todavia os filhos do teu povo dizem: Não é
justo o caminho do Senhor; mas o próprio caminho deles é que não é
justo. Desviando-se o justo da sua justiça, e praticando
iniqüidade, morrerá nela. E, convertendo-se o ímpio da sua
impiedade, e praticando juízo e justiça, ele viverá por eles.
Todavia, vós dizeis: Não é justo o caminho do Senhor;
julgar-vos-ei a cada um conforme os seus caminhos, ó casa de Israel.

E sucedeu que, no ano duodécimo do nosso cativeiro, no décimo
mês, aos cinco do mês, veio a mim um que tinha escapado de
Jerusalém, dizendo: A cidade está ferida. Ora, a mão do
Senhor estivera sobre mim pela tarde, antes que viesse o que tinha
escapado; e ele abrira a minha boca antes que esse homem viesse ter
comigo pela manhã; e abriu-se a minha boca, e não fiquei mais
calado. Então veio a mim a palavra do Senhor, dizendo:
Filho do homem, os moradores destes lugares desertos da terra
de Israel falam, dizendo: Abraão era um só, e possuiu esta terra;
mas nós somos muitos, esta terra nos foi dada em possessão.
Dize-lhes portanto: Assim diz o Senhor Deus: Comeis a carne
com o sangue, e levantais os vossos olhos para os vossos ídolos, e
derramais o sangue! Porventura possuireis a terra? Vós vos
estribais sobre a vossa espada, cometeis abominação, e cada um
contamina a mulher do seu próximo! E possuireis a terra?
Assim lhes dirás: Assim disse o Senhor Deus: Vivo eu, que os
que estiverem em lugares desertos, cairão à espada, e o que estiver
em campo aberto o entregarei às feras, para que o devorem, e os que
estiverem em lugares fortes e em cavernas morrerão de peste.
E tornarei a terra em desolação e espanto e cessará a soberba
do seu poder; e os montes de Israel ficarão tão desolados que
ninguém passará por eles. Então saberão que eu sou o Senhor,
quando eu tornar a terra em desolação e espanto, por causa de todas
as abominações que cometeram.

Quanto a ti, ó filho do homem, os filhos do teu povo falam de ti
junto às paredes e nas portas das casas; e fala um com o outro, cada
um a seu irmão, dizendo: Vinde, peço-vos, e ouvi qual seja a palavra
que procede do Senhor. E eles vêm a ti, como o povo costumava
vir, e se assentam diante de ti, como meu povo, e ouvem as tuas
palavras, mas não as põem por obra; pois lisonjeiam com a sua boca,
mas o seu coração segue a sua avareza. E eis que tu és para
eles como uma canção de amores, de quem tem voz suave, e que bem
tange; porque ouvem as tuas palavras, mas não as põem por obra.
Mas, quando vier isto (eis que está para vir), então saberão
que houve no meio deles um profeta.

\medskip

\lettrine{34} E veio a mim a palavra do Senhor, dizendo:
Filho do homem, profetiza contra os pastores de Israel;
profetiza, e dize aos pastores: Assim diz o Senhor Deus: Ai dos
pastores de Israel que se apascentam a si mesmos! Não devem os
pastores apascentar as ovelhas? Comeis a gordura, e vos vestis
da lã; matais o cevado; mas não apascentais as ovelhas. As
fracas não fortalecestes, e a doente não curastes, e a quebrada não
ligastes, e a desgarrada não tornastes a trazer, e a perdida não
buscastes; mas dominais sobre elas com rigor e dureza. Assim se
espalharam, por não haver pastor, e tornaram-se pasto para todas as
feras do campo, porquanto se espalharam. As minhas ovelhas
andaram desgarradas por todos os montes, e por todo o alto outeiro;
sim, as minhas ovelhas andaram espalhadas por toda a face da terra,
sem haver quem perguntasse por elas, nem quem as buscasse.

Portanto, ó pastores, ouvi a palavra do Senhor: Vivo eu, diz o
Senhor Deus, que, porquanto as minhas ovelhas foram entregues à
rapina, e as minhas ovelhas vieram a servir de pasto a todas as
feras do campo, por falta de pastor, e os meus pastores não
procuraram as minhas ovelhas; e os pastores apascentaram a si
mesmos, e não apascentaram as minhas ovelhas; portanto, ó
pastores, ouvi a palavra do Senhor: Assim diz o Senhor Deus:
Eis que eu estou contra os pastores; das suas mãos demandarei as
minhas ovelhas, e eles deixarão de apascentar as ovelhas; os
pastores não se apascentarão mais a si mesmos; e livrarei as minhas
ovelhas da sua boca, e não lhes servirão mais de pasto.
Porque assim diz o Senhor Deus: Eis que eu, eu mesmo,
procurarei pelas minhas ovelhas, e as buscarei. Como o pastor
busca o seu rebanho, no dia em que está no meio das suas ovelhas
dispersas, assim buscarei as minhas ovelhas; e livrá-las-ei de todos
os lugares por onde andam espalhadas, no dia nublado e de escuridão.
E tirá-las-ei dos povos, e as congregarei dos países, e as
trarei à sua própria terra, e as apascentarei nos montes de Israel,
junto aos rios, e em todas as habitações da terra. Em bons
pastos as apascentarei, e nos altos montes de Israel será o seu
aprisco; ali se deitarão num bom redil, e pastarão em pastos gordos
nos montes de Israel. Eu mesmo apascentarei as minhas
ovelhas, e eu as farei repousar, diz o Senhor Deus. A perdida
buscarei, e a desgarrada tornarei a trazer, e a quebrada ligarei, e
a enferma fortalecerei; mas a gorda e a forte destruirei;
apascentá-las-ei com juízo.

E quanto a vós, ó ovelhas minhas, assim diz o Senhor Deus: Eis
que eu julgarei entre ovelhas e ovelhas, entre carneiros e bodes.
Acaso não vos basta pastar os bons pastos, senão que pisais o
resto de vossos pastos aos vossos pés? E não vos basta beber as
águas claras, senão que sujais o resto com os vossos pés? E
quanto às minhas ovelhas elas pastarão o que haveis pisado com os
vossos pés, e beberão o que haveis sujado com os vossos pés.
Por isso o Senhor Deus assim lhes diz: Eis que eu, eu mesmo,
julgarei entre a ovelha gorda e a ovelha magra. Porquanto com
o lado e com o ombro dais empurrões, e com os vossos chifres
escorneais todas as fracas, até que as espalhais para fora.
Portanto livrarei as minhas ovelhas, para que não sirvam mais
de rapina, e julgarei entre ovelhas e ovelhas. E suscitarei
sobre elas um só pastor, e ele as apascentará; o meu servo Davi é
que as apascentará; ele lhes servirá de pastor. E eu, o
Senhor, lhes serei por Deus, e o meu servo Davi será príncipe no
meio delas; eu, o Senhor, o disse. E farei com elas uma
aliança de paz, e acabarei com as feras da terra, e habitarão em
segurança no deserto, e dormirão nos bosques. E delas e dos
lugares ao redor do meu outeiro, farei uma bênção; e farei descer a
chuva a seu tempo; chuvas de bênção serão. E as árvores do
campo darão o seu fruto, e a terra dará a sua novidade, e estarão
seguras na sua terra; e saberão que eu sou o Senhor, quando eu
quebrar as ataduras do seu jugo e as livrar da mão dos que se
serviam delas. E não servirão mais de rapina aos gentios, as
feras da terra nunca mais as devorarão; e habitarão seguramente, e
ninguém haverá que as espante. E lhes levantarei uma
plantação de renome, e nunca mais serão consumidas pela fome na
terra, nem mais levarão sobre si o opróbrio dos gentios.
Saberão, porém, que eu, o Senhor seu Deus, estou com elas, e
que elas são o meu povo, a casa de Israel, diz o Senhor Deus.
Vós, pois, ó ovelhas minhas, ovelhas do meu pasto; homens
sois; porém eu sou o vosso Deus, diz o Senhor Deus.

\medskip

\lettrine{35} E veio a mim a palavra do Senhor, dizendo:
Filho do homem, dirige o teu rosto contra o monte Seir, e
profetiza contra ele. E dize-lhe: Assim diz o Senhor Deus: Eis
que eu estou contra ti, ó monte Seir, e estenderei a minha mão
contra ti, e te farei maior desolação. As tuas cidades farei
desertas, e tu serás desolado; e saberás que eu sou o Senhor.
Porquanto guardaste inimizade perpétua, e espalhaste os filhos
de Israel pelo poder da espada no tempo da sua calamidade e no tempo
da iniqüidade final. Por isso vivo eu, diz o Senhor Deus, que te
preparei para sangue, e o sangue te perseguirá; visto que não
odiaste o sangue, o sangue te perseguirá. E farei do monte Seir
uma extrema desolação, e exterminarei dele o que por ele passar, e o
que por ele voltar. E encherei os seus montes dos seus mortos;
nos teus outeiros, e nos teus vales, e em todos os teus rios cairão
os mortos à espada. Em desolações perpétuas te porei, e as tuas
cidades nunca mais serão habitadas; assim sabereis que eu sou o
Senhor.

Porquanto disseste: As duas nações e as duas terras serão minhas,
e as possuiremos, sendo que o Senhor se achava ali. Portanto,
vivo eu, diz o Senhor Deus, que procederei conforme a tua ira, e
conforme a tua inveja, de que usaste, no teu ódio contra eles; e me
farei conhecer entre eles, quando te julgar. E saberás que
eu, o Senhor, ouvi todas as tuas blasfêmias, que proferiste contra
os montes de Israel, dizendo: Já estão assolados, a nós nos são
entregues por pasto. Assim vos engrandecestes contra mim com
a vossa boca, e multiplicastes as vossas palavras contra mim. Eu o
ouvi. Assim diz o Senhor Deus: Quando toda a terra se alegrar
eu te porei em desolação. Como te alegraste da herança da
casa de Israel, porque foi assolada, assim te farei a ti; assolado
serás, ó monte Seir, e todo o Edom, sim, todo ele; e saberão que eu
sou o Senhor.

\medskip

\lettrine{36} E tu, ó filho do homem, profetiza aos montes de
Israel, e dize: Montes de Israel, ouvi a palavra do Senhor.
Assim diz o Senhor Deus: Pois que disse o inimigo contra vós:
Ah! ah! até as alturas eternas serão nossa herança; portanto,
profetiza, e dize: Assim diz o Senhor Deus: Porquanto vos assolaram
e devoraram de todos os lados, para que ficásseis feitos herança do
restante dos gentios, e tendes andado em lábios paroleiros, e em
infâmia do povo, portanto, ouvi, ó montes de Israel, a palavra
do Senhor Deus: Assim diz o Senhor Deus aos montes e aos outeiros,
aos rios e aos vales, aos lugares assolados e solitários, e às
cidades desamparadas que se tornaram em rapina e em escárnio para o
restante dos gentios que lhes estão em redor; portanto, assim
diz o Senhor Deus: Certamente no fogo do meu zelo falei contra o
restante dos gentios, e contra todo o Edom, que se apropriaram da
minha terra, com toda a alegria de seu coração, e com menosprezo da
alma, para a lançarem fora à rapina. Portanto, profetiza sobre a
terra de Israel, e dize aos montes, e aos outeiros, aos rios e aos
vales: Assim diz o Senhor Deus: Eis que falei no meu zelo e no meu
furor, porque levastes sobre vós o opróbrio dos gentios.
Portanto, assim diz o Senhor Deus: Eu levantei a minha mão, para
que os gentios, que estão ao redor de vós, levem o seu opróbrio.
Mas vós, ó montes de Israel, produzireis os vossos ramos, e
dareis o vosso fruto para o meu povo de Israel; porque estão prestes
a vir. Porque eis que eu estou convosco, e eu me voltarei para
vós, e sereis lavrados e semeados. E multiplicarei homens
sobre vós, a toda a casa de Israel, a toda ela; e as cidades serão
habitadas, e os lugares devastados serão edificados. E
multiplicarei homens e animais sobre vós, e eles se multiplicarão, e
frutificarão. E farei com que sejais habitados como dantes e vos
tratarei melhor que nos vossos princípios; e sabereis que eu sou o
Senhor. E farei andar sobre vós homens, o meu povo de Israel;
eles te possuirão, e serás a sua herança, e nunca mais os
desfilharás. Assim diz o Senhor Deus: Porquanto vos dizem: Tu
és uma terra que devora os homens, e és uma terra que desfilha as
suas nações; por isso tu não devorarás mais os homens, nem
desfilharás mais as tuas nações, diz o Senhor Deus. E farei
que nunca mais tu ouças a afronta dos gentios; nem levarás mais
sobre ti o opróbrio das gentes, nem mais desfilharás a tua nação,
diz o Senhor Deus.

E veio a mim a palavra do Senhor, dizendo: Filho do homem,
quando a casa de Israel habitava na sua terra, então a contaminaram
com os seus caminhos e com as suas ações. Como a imundícia de uma
mulher em sua separação, tal era o seu caminho perante o meu rosto.
Derramei, pois, o meu furor sobre eles, por causa do sangue
que derramaram sobre a terra, e dos seus ídolos, com que a
contaminaram. E espalhei-os entre os gentios, e foram
dispersos pelas terras; conforme os seus caminhos, e conforme os
seus feitos, eu os julguei. E, chegando aos gentios para onde
foram, profanaram o meu santo nome, porquanto se dizia deles: Estes
são o povo do Senhor, e saíram da sua terra. Mas eu os poupei
por amor do meu santo nome, que a casa de Israel profanou entre os
gentios para onde foi. Dize portanto à casa de Israel: Assim
diz o Senhor Deus: Não é por respeito a vós que eu faço isto, ó casa
de Israel, mas pelo meu santo nome, que profanastes entre as nações
para onde fostes. E eu santificarei o meu grande nome, que
foi profanado entre os gentios, o qual profanastes no meio deles; e
os gentios saberão que eu sou o Senhor, diz o Senhor Deus, quando eu
for santificado aos seus olhos. E vos tomarei dentre os
gentios, e vos congregarei de todas as terras, e vos trarei para a
vossa terra.

Então aspergirei água pura sobre vós, e ficareis purificados; de
todas as vossas imundícias e de todos os vossos ídolos vos
purificarei. E dar-vos-ei um coração novo, e porei dentro de
vós um espírito novo; e tirarei da vossa carne o coração de pedra, e
vos darei um coração de carne. E porei dentro de vós o meu
Espírito, e farei que andeis nos meus estatutos, e guardeis os meus
juízos, e os observeis. E habitareis na terra que eu dei a
vossos pais e vós sereis o meu povo, e eu serei o vosso Deus.
E livrar-vos-ei de todas as vossas imundícias; e chamarei o
trigo, e o multiplicarei, e não trarei fome sobre vós. E
multiplicarei o fruto das árvores, e a novidade do campo, para que
nunca mais recebais o opróbrio da fome entre os gentios.
Então vos lembrareis dos vossos maus caminhos, e dos vossos
feitos, que não foram bons; e tereis nojo em vós mesmos das vossas
iniqüidades e das vossas abominações. Não é por amor de vós
que eu faço isto, diz o Senhor Deus; notório vos seja;
envergonhai-vos, e confundi-vos por causa dos vossos caminhos, ó
casa de Israel. Assim diz o Senhor Deus: No dia em que eu vos
purificar de todas as vossas iniqüidades, então farei com que sejam
habitadas as cidades e sejam edificados os lugares devastados.
E a terra assolada será lavrada, em lugar de estar assolada
aos olhos de todos os que passavam. E dirão: Esta terra
assolada ficou como jardim do Éden: e as cidades solitárias, e
assoladas, e destruídas, estão fortalecidas e habitadas.
Então saberão os gentios, que tiverem ficado ao redor de vós,
que eu, o Senhor, tenho reedificado as cidades destruídas, e
plantado o que estava devastado. Eu, o Senhor, o disse e o farei.
Assim diz o Senhor Deus: Ainda por isso serei solicitado pela
casa de Israel, que lho faça; multiplicar-lhes-ei os homens, como a
um rebanho. Como o rebanho santificado, como o rebanho de
Jerusalém nas suas solenidades, assim as cidades desertas se
encherão de rebanhos de homens; e saberão que eu sou o Senhor.

\medskip

\lettrine{37} Veio sobre mim a mão do Senhor, e ele me fez
sair no Espírito do Senhor, e me pôs no meio de um vale que estava
cheio de ossos. E me fez passar em volta deles; e eis que eram
mui numerosos sobre a face do vale, e eis que estavam sequíssimos.
E me disse: Filho do homem, porventura viverão estes ossos? E eu
disse: Senhor Deus, tu o sabes. Então me disse: Profetiza sobre
estes ossos, e dize-lhes: Ossos secos, ouvi a palavra do Senhor.
Assim diz o Senhor Deus a estes ossos: Eis que farei entrar em
vós o espírito, e vivereis. E porei nervos sobre vós e farei
crescer carne sobre vós, e sobre vós estenderei pele, e porei em vós
o espírito, e vivereis, e sabereis que eu sou o Senhor. Então
profetizei como se me deu ordem. E houve um ruído, enquanto eu
profetizava; e eis que se fez um rebuliço, e os ossos se achegaram,
cada osso ao seu osso. E olhei, e eis que vieram nervos sobre
eles, e cresceu a carne, e estendeu-se a pele sobre eles por cima;
mas não havia neles espírito. E ele me disse: Profetiza ao
espírito, profetiza, ó filho do homem, e dize ao espírito: Assim diz
o Senhor Deus: Vem dos quatro ventos, ó espírito, e assopra sobre
estes mortos, para que vivam. E profetizei como ele me deu
ordem; então o espírito entrou neles, e viveram, e se puseram em pé,
um exército grande em extremo. Então me disse: Filho do
homem, estes ossos são toda a casa de Israel. Eis que dizem: Os
nossos ossos se secaram, e pereceu a nossa esperança; nós mesmos
estamos cortados. Portanto profetiza, e dize-lhes: Assim diz
o Senhor Deus: Eis que eu abrirei os vossos sepulcros, e vos farei
subir das vossas sepulturas, ó povo meu, e vos trarei à terra de
Israel. E sabereis que eu sou o Senhor, quando eu abrir os
vossos sepulcros, e vos fizer subir das vossas sepulturas, ó povo
meu. E porei em vós o meu Espírito, e vivereis, e vos porei
na vossa terra; e sabereis que eu, o Senhor, disse isto, e o fiz,
diz o Senhor.

E outra vez veio a mim a palavra do Senhor, dizendo: Tu,
pois, ó filho do homem, toma um pedaço de madeira, e escreve nele:
Por Judá e pelos filhos de Israel, seus companheiros. E toma outro
pedaço de madeira, e escreve nele: Por José, vara de Efraim, e por
toda a casa de Israel, seus companheiros. E ajunta um ao
outro, para que se unam, e se tornem uma só vara na tua mão.
E quando te falarem os filhos do teu povo, dizendo:
Porventura não nos declararás o que significam estas coisas?
Tu lhes dirás: Assim diz o Senhor Deus: Eis que eu tomarei a
vara de José que esteve na mão de Efraim, e a das tribos de Israel,
suas companheiras, e as ajuntarei à vara de Judá, e farei delas uma
só vara, e elas se farão uma só na minha mão. E as varas,
sobre que houveres escrito, estarão na tua mão, perante os olhos
deles. Dize-lhes pois: Assim diz o Senhor Deus: Eis que eu
tomarei os filhos de Israel dentre os gentios, para onde eles foram,
e os congregarei de todas as partes, e os levarei à sua terra.
E deles farei uma nação na terra, nos montes de Israel, e um
rei será rei de todos eles, e nunca mais serão duas nações; nunca
mais para o futuro se dividirão em dois reinos. E nunca mais
se contaminarão com os seus ídolos, nem com as suas abominações, nem
com as suas transgressões, e os livrarei de todas as suas
habitações, em que pecaram, e os purificarei. Assim eles serão o meu
povo, e eu serei o seu Deus. E meu servo Davi será rei sobre
eles, e todos eles terão um só pastor; e andarão nos meus juízos e
guardarão os meus estatutos, e os observarão. E habitarão na
terra que dei a meu servo Jacó, em que habitaram vossos pais; e
habitarão nela, eles e seus filhos, e os filhos de seus filhos, para
sempre, e Davi, meu servo, será seu príncipe eternamente. E
farei com eles uma aliança de paz; e será uma aliança perpétua. E os
estabelecerei, e os multiplicarei, e porei o meu santuário no meio
deles para sempre. E o meu tabernáculo estará com eles, e eu
serei o seu Deus e eles serão o meu povo. E os gentios
saberão que eu sou o Senhor que santifico a Israel, quando estiver o
meu santuário no meio deles para sempre.

\medskip

\lettrine{38} Veio a mim a palavra do Senhor, dizendo:
Filho do homem, dirige o teu rosto contra Gogue, terra de
Magogue, príncipe e chefe de Meseque, e Tubal, e profetiza contra
ele. E dize: Assim diz o Senhor Deus: Eis que eu sou contra ti,
ó Gogue, príncipe e chefe de Meseque e de Tubal; e te farei
voltar, e porei anzóis nos teus queixos, e te levarei a ti, com todo
o teu exército, cavalos e cavaleiros, todos vestidos com primor,
grande multidão, com escudo e rodela, manejando todos a espada;
persas, etíopes, e os de Pute com eles, todos com escudo e
capacete; Gômer e todas as suas tropas; a casa de Togarma, do
extremo norte, e todas as suas tropas, muitos povos contigo.
Prepara-te, e dispõe-te, tu e todas as multidões do teu povo que
se reuniram a ti, e serve-lhes tu de guarda. Depois de muitos
dias serás visitado. No fim dos anos virás à terra que se recuperou
da espada, e que foi congregada dentre muitos povos, junto aos
montes de Israel, que sempre se faziam desertos; mas aquela terra
foi tirada dentre as nações, e todas elas habitarão seguramente.
Então subirás, virás como uma tempestade, far-te-ás como uma
nuvem para cobrir a terra, tu e todas as tuas tropas, e muitos povos
contigo. Assim diz o Senhor Deus: E acontecerá naquele dia
que subirão palavras no teu coração, e maquinarás um mau desígnio,
e dirás: Subirei contra a terra das aldeias não muradas;
virei contra os que estão em repouso, que habitam seguros; todos
eles habitam sem muro, e não têm ferrolhos nem portas; a fim
de tomar o despojo, e para arrebatar a presa, e tornar a tua mão
contra as terras desertas que agora se acham habitadas, e contra o
povo que se congregou dentre as nações, o qual adquiriu gado e bens,
e habita no meio da terra. Sebá e Dedã, e os mercadores de
Társis, e todos os seus leõezinhos te dirão: Vens tu para tomar o
despojo? Ajuntaste a tua multidão para arrebatar a tua presa? Para
levar a prata e o ouro, para tomar o gado e os bens, para saquear o
grande despojo?

Portanto, profetiza, ó filho do homem, e dize a Gogue: Assim diz
o Senhor Deus: Porventura não o saberás naquele dia, quando o meu
povo Israel habitar em segurança? Virás, pois, do teu lugar,
do extremo norte, tu e muitos povos contigo, montados todos a
cavalo, grande ajuntamento, e exército poderoso, e subirás
contra o meu povo Israel, como uma nuvem, para cobrir a terra. Nos
últimos dias sucederá que hei de trazer-te contra a minha terra,
para que os gentios me conheçam a mim, quando eu me houver
santificado em ti, ó Gogue, diante dos seus olhos. Assim diz
o Senhor Deus: Não és tu aquele de quem eu disse nos dias antigos,
por intermédio dos meus servos, os profetas de Israel, os quais
naqueles dias profetizaram largos anos, que te traria contra eles?
Sucederá, porém, naquele dia, no dia em que vier Gogue contra
a terra de Israel, diz o Senhor Deus, que a minha indignação subirá
à minha face. Porque disse no meu zelo, no fogo do meu furor,
que, certamente, naquele dia haverá grande tremor sobre a terra de
Israel; de tal modo que tremerão diante da minha face os
peixes do mar, e as aves do céu, e os animais do campo, e todos os
répteis que se arrastam sobre a terra, e todos os homens que estão
sobre a face da terra; e os montes serão deitados abaixo, e os
precipícios se desfarão, e todos os muros desabarão por terra.
Porque chamarei contra ele a espada sobre todos os meus
montes, diz o Senhor Deus; a espada de cada um se voltará contra seu
irmão. E contenderei com ele por meio da peste e do sangue; e
uma chuva inundante, e grandes pedras de saraiva, fogo, e enxofre
farei chover sobre ele, e sobre as suas tropas, e sobre os muitos
povos que estiverem com ele. Assim eu me engrandecerei e me
santificarei, e me darei a conhecer aos olhos de muitas nações; e
saberão que eu sou o Senhor.

\medskip

\lettrine{39} Tu, pois, ó filho do homem, profetiza ainda
contra Gogue, e dize: Assim diz o Senhor Deus: Eis que eu sou contra
ti, ó Gogue, príncipe e chefe de Meseque e de Tubal. E te farei
voltar, mas deixarei uma sexta parte de ti, e far-te-ei subir do
extremo norte, e te trarei aos montes de Israel. E, com um
golpe, tirarei o teu arco da tua mão esquerda, e farei cair as tuas
flechas da tua mão direita. Nos montes de Israel cairás, tu e
todas as tuas tropas, e os povos que estão contigo; e às aves de
rapina, de toda espécie, e aos animais do campo, te darei por
comida. Sobre a face do campo cairás, porque eu o falei, diz o
Senhor Deus. E enviarei um fogo sobre Magogue e entre os que
habitam seguros nas ilhas; e saberão que eu sou o Senhor. E
farei conhecido o meu santo nome no meio do meu povo Israel, e nunca
mais deixarei profanar o meu santo nome; e os gentios saberão que eu
sou o Senhor, o Santo em Israel.

Eis que vem, e se cumprirá, diz o Senhor Deus; este é o dia de que
tenho falado. E os habitantes das cidades de Israel sairão, e
acenderão o fogo, e queimarão as armas, e os escudos e as rodelas,
com os arcos, e com as flechas, e com os bastões de mão, e com as
lanças; e acenderão fogo com elas por sete anos. E não trarão
lenha do campo, nem a cortarão dos bosques, mas com as armas
acenderão fogo; e roubarão aos que os roubaram, e despojarão aos que
os despojaram, diz o Senhor Deus. E sucederá que, naquele
dia, darei ali a Gogue um lugar de sepultura em Israel, o vale dos
que passam ao oriente do mar; e pararão os que por ele passarem; e
ali sepultarão a Gogue, e a toda a sua multidão, e lhe chamarão o
vale da multidão de Gogue. E a casa de Israel os enterrará
durante sete meses, para purificar a terra. Sim, todo o povo
da terra os enterrará, e será para eles memorável dia em que eu for
glorificado, diz o Senhor Deus. E separarão homens que
incessantemente percorrerão a terra, para que eles, juntamente com
os que passam, sepultem os que tiverem ficado sobre a face da terra,
para a purificarem; durante sete meses farão esta busca. E os
que percorrerem a terra, a qual atravessarão, vendo algum osso de
homem, porão ao lado um sinal; até que os enterradores o tenham
enterrado no vale da multidão de Gogue. E também o nome da
cidade será Hamona; assim purificarão a terra. Tu, pois, ó
filho do homem, assim diz o Senhor Deus, dize às aves de toda
espécie, e a todos os animais do campo: Ajuntai-vos e vinde,
congregai-vos de toda parte para o meu sacrifício, que eu ofereci
por vós, um sacrifício grande, nos montes de Israel, e comei carne e
bebei sangue. Comereis a carne dos poderosos e bebereis o
sangue dos príncipes da terra; dos carneiros, dos cordeiros, e dos
bodes, e dos bezerros, todos cevados de Basã. E comereis a
gordura até vos fartardes e bebereis o sangue até vos embebedardes,
do meu sacrifício que ofereci por vós. E, à minha mesa,
fartar-vos-ei de cavalos, de carros, de poderosos, e de todos os
homens de guerra, diz o Senhor Deus. E eu porei a minha
glória entre os gentios e todos os gentios verão o meu juízo, que eu
tiver executado, e a minha mão, que sobre elas tiver descarregado.
E saberão os da casa de Israel que eu sou o Senhor seu Deus,
desde aquele dia em diante.

E os gentios saberão que os da casa de Israel, por causa da sua
iniqüidade, foram levados em cativeiro, porque se rebelaram contra
mim, e eu escondi deles a minha face, e os entreguei nas mãos de
seus adversários, e todos caíram à espada. Conforme a sua
imundícia e conforme as suas transgressões me houve com eles, e
escondi deles a minha face. Portanto assim diz o Senhor Deus:
Agora tornarei a trazer os cativos de Jacó, e me compadecerei de
toda a casa de Israel; zelarei pelo meu santo nome. E levarão
sobre si a sua vergonha, e toda a sua rebeldia, com que se rebelaram
contra mim, quando eles habitarem seguros na sua terra, sem haver
quem os espante. Quando eu os tornar a trazer de entre os
povos, e os houver ajuntado das terras de seus inimigos, e eu for
santificado neles aos olhos de muitas nações, então saberão
que eu sou o Senhor seu Deus, vendo que eu os fiz ir em cativeiro
entre os gentios, e os ajuntarei para voltarem a sua terra, e não
mais deixarei lá nenhum deles. Nem lhes esconderei mais a
minha face, pois derramarei o meu Espírito sobre a casa de Israel,
diz o Senhor Deus.

\medskip

\lettrine{40} No ano vinte e cinco do nosso cativeiro, no
princípio do ano, no décimo dia do mês, catorze anos depois que a
cidade foi conquistada, naquele mesmo dia veio sobre mim a mão do
Senhor, e me levou para lá. Em visões de Deus me levou à terra
de Israel, e me pôs sobre um monte muito alto, sobre o qual havia
como que um edifício de cidade para o lado sul. E, havendo-me
levado ali, eis que um homem cuja aparência era como a do bronze,
tendo um cordel de linho na sua mão e uma cana de medir, e estava em
pé na porta. E disse-me o homem: Filho do homem, vê com os teus
olhos, e ouve com os teus ouvidos, e põe no teu coração tudo quanto
eu te fizer ver; porque para to mostrar foste tu aqui trazido;
anuncia, pois, à casa de Israel tudo quanto vires.

E havia um muro fora da casa, em seu redor, e na mão do homem uma
cana de medir, de seis côvados, cada um dos quais tinha um côvado e
um palmo; e ele mediu a largura do edifício, uma cana, e a altura,
uma cana. Então veio à porta que olhava para o caminho do
oriente, e subiu pelos seus degraus; mediu o umbral da porta, uma
cana de largo, e o outro umbral, uma cana de largo. E cada
câmara tinha uma cana de comprido, e uma cana de largo, e o espaço
entre os aposentos era de cinco côvados; e o umbral da porta, ao pé
do vestíbulo da porta, por dentro, era de uma cana. Também mediu
o vestíbulo da porta, por dentro, uma cana. Então mediu o
vestíbulo da porta, que tinha oito côvados, e os seus pilares, dois
côvados, e este vestíbulo da porta, estava por dentro. As
câmaras da porta para o lado do oriente eram três de um lado e três
do outro; a mesma medida era a dos três; também os pilares de um
lado e do outro tinham a mesma medida. Mediu mais a largura
da entrada da porta, que era de dez côvados; e o comprimento da
porta, treze côvados. E o espaço em frente das câmaras era de
um côvado, e de um côvado o espaço do outro lado; e cada câmara
tinha seis côvados de um lado e seis côvados do outro. Então
mediu a porta desde o telhado de uma câmara até ao telhado da outra,
vinte e cinco côvados de largo, porta contra porta. Fez
também os pilares, de sessenta côvados, cada pilar, do átrio, em
redor da porta. E, desde a face da porta da entrada até à
face do vestíbulo da porta interior, havia cinqüenta côvados.
Havia também janelas estreitas nas câmaras, e nos seus
pilares, dentro da porta ao redor, e da mesma sorte nos vestíbulos;
e as janelas estavam ao redor, na parte de dentro, e nos pilares
havia palmeiras. E ele me levou ao átrio exterior, e eis que
havia nele câmaras, e um pavimento que estava feito no átrio em
redor; trinta câmaras havia naquele pavimento. E o pavimento
do lado das portas era proporcional ao comprimento das portas; o
pavimento estava mais baixo. E mediu a largura desde a
dianteira da porta inferior até a dianteira do átrio interior, por
fora, cem côvados, do lado do oriente e do norte. E, quanto à
porta que olhava para o caminho do norte, no átrio exterior, ele
mediu o seu comprimento e a sua largura. E as suas câmaras
eram três de um lado, e três do outro, e os seus pilares e os seus
arcos eram da medida da primeira porta: cinqüenta côvados era o seu
comprimento, e a largura vinte e cinco côvados. E as suas
janelas, e os seus arcos, e as suas palmeiras, eram da medida da
porta que olhava para o caminho do oriente; e subia-se para ela por
sete degraus, e os seus arcos estavam diante dela. E a porta
do átrio interior estava defronte da porta do norte bem como da do
oriente; e mediu de porta a porta cem côvados. Então ele me
levou ao caminho do sul, e eis que havia ali uma porta que olhava
para o caminho do sul, e mediu os seus pilares e os seus arcos
conforme estas medidas. E havia também janelas em redor dos
seus arcos, como as outras janelas; cinqüenta côvados era o
comprimento, e a largura vinte e cinco côvados. E de sete
degraus eram as suas subidas, e os seus arcos estavam diante delas;
e tinha palmeiras, uma de um lado e outra do outro, nos seus
pilares.

Também havia uma porta no átrio interior para o caminho do sul; e
mediu de porta a porta, para o caminho do sul, cem côvados.
Então me levou ao átrio interior pela porta do sul; e mediu a
porta do sul, conforme estas medidas. E as suas câmaras, e os
seus pilares, e os seus arcos eram conforme estas medidas; e tinham
também janelas ao redor dos seus arcos; o comprimento era de
cinqüenta côvados, e a largura de vinte e cinco côvados. E
havia arcos em redor; o comprimento era de vinte e cinco côvados, e
a largura de cinco côvados. E os seus arcos estavam na
direção do átrio exterior, e havia palmeiras nos seus pilares; e de
oito degraus eram as suas subidas. Depois me levou ao átrio
interior, para o caminho do oriente, e mediu a porta conforme estas
medidas; e também as suas câmaras, e os seus pilares, e os
seus arcos, conforme estas medidas; e havia também janelas em redor
dos seus arcos; o comprimento de cinqüenta côvados, e a largura de
vinte e cinco côvados. E os seus arcos estavam no átrio de
fora; também havia palmeiras nos seus pilares de um e de outro lado;
e eram as suas subidas de oito degraus. Então me levou à
porta do norte, e mediu conforme estas medidas; as suas
câmaras, os seus pilares, e os seus arcos; também tinha janelas em
redor; o comprimento era de cinqüenta côvados, e a largura de vinte
e cinco côvados. E os seus pilares estavam no átrio exterior;
também havia palmeiras nos seus pilares de um e de outro lado; e
eram as suas subidas de oito degraus. E as suas câmaras e as
suas entradas estavam junto aos pilares das portas onde lavavam o
holocausto.

E no vestíbulo da porta havia duas mesas de um lado, e duas mesas
do outro, para nelas se matar o holocausto e a oferta pelo pecado e
pela culpa. Também do lado de fora da subida para a entrada
da porta do norte havia duas mesas; e do outro lado, que estava no
vestíbulo da porta, havia duas mesas. Quatro mesas de um
lado, e quatro mesas do outro; aos lados da porta oito mesas, sobre
as quais imolavam. E as quatro mesas para o holocausto eram
de pedras lavradas; o comprimento era de um côvado e meio, e a
largura de um côvado e meio, e a altura de um côvado; e sobre elas
se punham os instrumentos com que imolavam o holocausto e o
sacrifício. E os ganchos de um palmo de comprimento, estavam
fixos por dentro em redor, e sobre as mesas estava a carne da
oferta. E fora da porta interior estavam as câmaras dos
cantores, no átrio de dentro, que estava ao lado da porta do norte e
olhava para o caminho do sul; uma estava ao lado da porta do
oriente, e olhava para o caminho do norte. E ele me disse:
Esta câmara que olha para o caminho do sul é para os sacerdotes que
têm a guarda da casa. Mas a câmara que olha para o caminho do
norte é para os sacerdotes que têm a guarda do altar; são estes os
filhos de Zadoque, que se chegam ao Senhor, dentre os filhos de
Levi, para o servir. E mediu o átrio; o comprimento de cem
côvados e a largura de cem côvados, um quadrado; e o altar estava
diante da casa. Então me levou ao vestíbulo da casa, e mediu
a cada pilar do vestíbulo, cinco côvados de um lado, e cinco côvados
do outro; e a largura da porta, três côvados de um lado, e três
côvados do outro. O comprimento do vestíbulo era de vinte
côvados, e a largura de onze côvados, e era por degraus, que se
subia a ele; e havia colunas junto aos pilares, uma de um lado e
outra do outro.

\medskip

\lettrine{41} Então me levou ao templo, e mediu os pilares,
seis côvados de largura de um lado, e seis côvados de largura do
outro, que era a largura da tenda. E a largura da entrada, dez
côvados; e os lados da entrada, cinco côvados de um lado e cinco
côvados do outro; também mediu o seu comprimento, de quarenta
côvados, e a largura, de vinte côvados. E entrou no interior, e
mediu o pilar da entrada, dois côvados, e a entrada, seis côvados, e
a largura da entrada, sete côvados. Também mediu o seu
comprimento, vinte côvados, e a largura, vinte côvados, diante do
templo, e disse-me: Este é o Santo dos Santos. E mediu a parede
da casa, seis côvados, e a largura das câmaras laterais, quatro
côvados, por todo o redor da casa. E as câmaras laterais,
estavam em três andares, câmara sobre câmara, trinta em cada andar,
e elas entravam na parede que tocava na casa pelas câmaras laterais
em redor, para prenderem nela, e não travavam na parede da casa.
E havia maior largura nas câmaras laterais superiores, porque o
caracol da casa ia subindo muito alto por todo o redor da casa, por
isso que a casa tinha mais largura para cima; e assim da câmara
baixa se subia à mais alta pelo meio. E olhei para a altura da
casa ao redor; e eram os fundamentos das câmaras laterais da medida
de uma cana inteira, seis côvados grandes. A grossura da parede
das câmaras laterais de fora era de cinco côvados; e o que foi
deixado vazio era o lugar das câmaras laterais, que estavam por
dentro. E entre as câmaras havia a largura de vinte côvados
por todo o redor da casa. E as entradas das câmaras laterais
estavam voltadas para o lugar vazio; uma entrada para o caminho do
norte, e outra entrada para o do sul; e a largura do lugar vazio era
de cinco côvados em redor.

Era também o edifício que estava diante do lugar separado, do
lado do ocidente, da largura de setenta côvados; e a parede do
edifício de cinco côvados de largura em redor, e o seu comprimento
era de noventa côvados. Assim mediu a casa, do comprimento de
cem côvados, como também o lugar separado, e o edifício, e as suas
paredes, cem côvados de comprimento. E a largura da frente da
casa, e do lugar separado para o oriente, de uma e de outra parte,
de cem côvados. Também mediu o comprimento do edifício,
diante do lugar separado, que estava por detrás, e as suas galerias
de uma e de outra parte, cem côvados, com o templo de dentro e os
vestíbulos do átrio. Os umbrais e as janelas estreitas, e as
galerias em redor nos três andares, defronte do umbral, estavam
cobertas de madeira em redor; e isto desde o chão até às janelas; e
as janelas estavam cobertas. No espaço em cima da porta, e
até na casa, no seu interior e na parte de fora, e até toda a parede
em redor, por dentro e por fora, tudo por medida. E foi feito
com querubins e palmeiras, de maneira que cada palmeira estava entre
querubim e querubim, e cada querubim tinha dois rostos, a
saber: um rosto de homem olhava para a palmeira de um lado, e um
rosto de leãozinho para a palmeira do outro lado; assim foi feito
por toda a casa em redor. Desde o chão até acima da entrada
estavam feitos os querubins e as palmeiras, como também pela parede
do templo. As ombreiras do templo eram quadradas e, no
tocante à frente do santuário, a aparência de uma era como a
aparência da outra, o altar de madeira era de três côvados de
altura, e o seu comprimento de dois côvados; os seus cantos, o seu
comprimento e as suas paredes eram de madeira; e disse-me: Esta é a
mesa que está perante a face do Senhor. E o templo e o
santuário, ambos tinham duas portas. E as portas tinham duas
folhas; duas folhas que viravam; duas para uma porta e duas para a
outra. E nelas, isto é, nas portas do templo, foram feitos
querubins e palmeiras, como estavam feitos nas paredes, e havia uma
trave grossa de madeira na frente do vestíbulo por fora. E
havia janelas estreitas, e palmeiras, de um e de outro lado, pelos
lados do vestíbulo, como também nas câmaras da casa e nas grossas
traves.

\medskip

\lettrine{42} Depois disto fez-me sair para fora, ao átrio
exterior, para o lado do caminho do norte; e me levou às câmaras que
estavam defronte do lugar separado, e que estavam defronte do
edifício, do lado norte. Do comprimento de cem côvados, era a
entrada do norte; e a largura era de cinqüenta côvados. Em
frente dos vinte côvados, que tinha o átrio interior, e em frente do
pavimento que tinha o átrio exterior, havia galeria contra galeria
em três andares. E diante das câmaras havia um passeio de dez
côvados de largo, do lado de dentro, e um caminho de um côvado, e as
suas entradas eram para o lado do norte. E as câmaras superiores
eram mais estreitas; porque as galerias tomavam aqui mais espaço do
que as de baixo e as do meio do edifício. Porque elas eram de
três andares, e não tinham colunas como as colunas dos átrios; por
isso desde o chão se iam estreitando, mais do que as de baixo e as
do meio. E o muro que estava de fora, defronte das câmaras, no
caminho do átrio exterior, diante das câmaras, tinha cinqüenta
côvados de comprimento. Pois o comprimento das câmaras, que
estavam no átrio exterior, era de cinqüenta côvados; e eis que
defronte do templo havia cem côvados. Por baixo destas câmaras
estava a entrada do lado do oriente, quando se entra nelas pelo
átrio exterior. Na largura do muro do átrio para o lado do
oriente, diante do lugar separado, e diante do edifício, havia
também câmaras. E o caminho que havia diante delas era da
aparência das câmaras, que davam para o norte; conforme o seu
comprimento, assim era a sua largura; e todas as suas saídas eram
também conforme os seus padrões, e conforme as suas entradas.
E conforme as portas das câmaras, que olhavam para o caminho
do sul, havia também uma entrada no topo do caminho, isto é, do
caminho em frente do muro direito, para o caminho do oriente, quando
se entra por elas. Então me disse: As câmaras do norte, e as
câmaras do sul, que estão diante do lugar separado, elas são câmaras
santas, em que os sacerdotes, que se chegam ao Senhor, comerão as
coisas mais santas; ali porão as coisas mais santas, e a oferta de
manjar, a oferta pelo pecado, e a oferta pela culpa; porque o lugar
é santo. Quando os sacerdotes entrarem, não sairão do
santuário para o átrio exterior, mas porão ali as suas vestiduras
com que ministraram, porque elas são santas; e vestir-se-ão de
outras vestiduras, e assim se aproximarão do lugar pertencente ao
povo.

E, acabando ele de medir a casa interior, ele me fez sair pelo
caminho da porta, cuja face olha para o caminho do oriente; e a
mediu em redor. Mediu o lado oriental com a cana de medir,
quinhentas canas, com a cana de medir, ao redor. Mediu o lado
do norte, com a cana de medir, quinhentas canas ao redor.
Mediu também o lado do sul, com a cana de medir, quinhentas
canas. Deu uma volta para o lado do ocidente, e mediu, com a
cana de medir, quinhentas canas. Mediu pelos quatro lados; e
havia um muro em redor, de quinhentas canas de comprimento, e
quinhentas de largura, para fazer separação entre o santo e o
profano.

\medskip

\lettrine{43} Então me levou à porta, à porta que olha para o
caminho do oriente. E eis que a glória do Deus de Israel vinha
do caminho do oriente; e a sua voz era como a voz de muitas águas, e
a terra resplandeceu por causa da sua glória. E o aspecto da
visão que tive era como o da visão que eu tivera quando vim destruir
a cidade; e eram as visões como as que tive junto ao rio Quebar; e
caí sobre o meu rosto. E a glória do Senhor entrou na casa pelo
caminho da porta, cuja face está para o lado do oriente. E
levantou-me o Espírito, e me levou ao átrio interior; e eis que a
glória do Senhor encheu a casa. E ouvi alguém que falava comigo
de dentro da casa, e um homem se pôs em pé junto de mim.

E disse-me: Filho do homem, este é o lugar do meu trono, e o lugar
das plantas dos meus pés, onde habitarei no meio dos filhos de
Israel para sempre; e os da casa de Israel não contaminarão mais o
meu nome santo, nem eles nem os seus reis, com suas prostituições e
com os cadáveres dos seus reis, nos seus altos, pondo o seu
limiar ao pé do meu limiar, e o seu umbral junto ao meu umbral, e
havendo uma parede entre mim e eles; e contaminaram o meu santo nome
com as suas abominações que cometiam; por isso eu os consumi na
minha ira. Agora lancem eles para longe de mim a sua
prostituição, e os cadáveres dos seus reis, e habitarei no meio
deles para sempre. Tu, pois, ó filho do homem, mostra à casa
de Israel esta casa, para que se envergonhe das suas maldades, e
meça o modelo. E, envergonhando-se eles de tudo quanto
fizeram, faze-lhes saber a forma desta casa, e a sua figura, e as
suas saídas, e as suas entradas, e todas as suas formas, e todos os
seus estatutos, todas as suas formas, e todas as suas leis; e
escreve isto aos seus olhos, para que guardem toda a sua forma, e
todos os seus estatutos, e os cumpram. Esta é a lei da casa:
Sobre o cume do monte todo o seu contorno em redor será santíssimo;
eis que esta é a lei da casa.

E estas são as medidas do altar, em côvados (o côvado é um côvado
e um palmo): e o fundo será de um côvado de altura, e um côvado de
largura, e a sua borda em todo o seu contorno, de um palmo; e esta é
a base do altar. E do fundo, desde a terra até a armação
inferior, dois côvados, e de largura um côvado, e desde a pequena
armação até a grande, quatro côvados, e a largura de um côvado.
E o altar, de quatro côvados; e desde o altar e para cima
havia quatro pontas. E o altar terá doze côvados de
comprimento, e doze de largura, quadrado nos quatro lados. E
a armação, catorze côvados de comprimento, e catorze de largura, nos
seus quatro lados; e o contorno, ao redor dela, de meio côvado, e o
fundo dela de um côvado, ao redor; e os seus degraus davam para o
oriente. E disse-me: Filho do homem, assim diz o Senhor Deus:
Estes são os estatutos do altar, no dia em que o fizerem, para
oferecerem sobre ele holocausto e para aspergirem sobre ele sangue.
E aos sacerdotes levitas, que são da descendência de Zadoque,
que se chegam a mim (diz o Senhor Deus) para me servirem, darás um
bezerro, para oferta pelo pecado. E tomarás do seu sangue, e
o porás sobre as suas quatro pontas, e sobre os quatro cantos da
armação, e no contorno ao redor; assim o purificarás e o expiarás.
Então tomarás o bezerro da oferta pelo pecado, e o queimará
no lugar da casa para isso designado, fora do santuário. E no
segundo dia oferecerás um bode, sem mancha, como oferta pelo pecado;
e purificarão o altar, como o purificaram com o bezerro. E,
acabando tu de purificá-lo, oferecerás um bezerro, sem mancha, e um
carneiro do rebanho, sem mancha. E oferecê-los-ás perante a
face do Senhor; e os sacerdotes deitarão sal sobre eles, e
oferecê-los-ão em holocausto ao Senhor. Por sete dias
prepararás, cada dia um bode como oferta pelo pecado; também
prepararão um bezerro, e um carneiro do rebanho, sem mancha.
Por sete dias expiarão o altar, e o purificarão; e assim
consagrar-se-ão. E, cumprindo eles estes dias, será que, ao
oitavo dia, e dali em diante, os sacerdotes oferecerão sobre o altar
os vossos holocaustos e as vossas ofertas pacíficas; e eu me
deleitarei em vós, diz o Senhor Deus.

\medskip

\lettrine{44} Então me fez voltar para o caminho da porta
exterior do santuário, que olha para o oriente, a qual estava
fechada. E disse-me o Senhor: Esta porta permanecerá fechada,
não se abrirá; ninguém entrará por ela, porque o Senhor, o Deus de
Israel entrou por ela; por isso permanecerá fechada. Quanto ao
príncipe, por ser príncipe, se assentará nela para sempre, para
comer o pão diante do Senhor; pelo caminho do vestíbulo da porta
entrará e por esse mesmo caminho sairá.

Depois me levou pelo caminho da porta do norte, diante da casa; e
olhei, e eis que a glória do Senhor encheu a casa do Senhor; então
caí sobre o meu rosto. E disse-me o Senhor: Filho do homem,
pondera no teu coração, e vê com os teus olhos, e ouve com os teus
ouvidos, tudo quanto eu te disser de todos os estatutos da casa do
Senhor, e de todas as suas leis; e considera no teu coração a
entrada da casa, com todas as saídas do santuário. E dize ao
rebelde, à casa de Israel: Assim diz o Senhor Deus: Bastem-vos todas
as vossas abominações, ó casa de Israel! Porque introduzistes
estrangeiros, incircuncisos de coração e incircuncisos de carne,
para estarem no meu santuário, para o profanarem em minha casa,
quando ofereceis o meu pão, a gordura, e o sangue; e eles
invalidaram a minha aliança, por causa de todas as vossas
abominações. E não guardastes a ordenança a respeito das minhas
coisas sagradas; antes vos constituístes, a vós mesmos, guardas da
minha ordenança no meu santuário. Assim diz o Senhor Deus:
Nenhum estrangeiro, incircunciso de coração ou incircunciso de
carne, entrará no meu santuário, dentre os estrangeiros que se
acharem no meio dos filhos de Israel.

Mas os levitas que se apartaram para longe de mim, quando Israel
andava errado; os quais andavam transviados, desviados de mim, para
irem atrás dos seus ídolos, levarão sobre si a sua iniqüidade.
Contudo serão ministros no meu santuário, nos ofícios das
portas da casa, e servirão à casa; eles matarão o holocausto, e o
sacrifício para o povo, e estarão perante eles, para os servir.
Porque lhes ministraram diante dos seus ídolos, e fizeram a
casa de Israel cair em iniqüidade; por isso eu levantei a minha mão
contra eles, diz o Senhor Deus, e levarão sobre si a sua iniqüidade.
E não se chegarão a mim, para me servirem no sacerdócio, nem
para se chegarem a alguma de todas as minhas coisas sagradas, às
coisas que são santíssimas, mas levarão sobre si a sua vergonha e as
suas abominações que cometeram. Contudo, eu os constituirei
guardas da ordenança da casa, em todo o seu serviço, e em tudo o que
nela se fizer. Mas os sacerdotes levíticos, os filhos de
Zadoque, que guardaram a ordenança do meu santuário quando os filhos
de Israel se extraviaram de mim, eles se chegarão a mim, para me
servirem, e estarão diante de mim, para me oferecerem a gordura e o
sangue, diz o Senhor Deus. Eles entrarão no meu santuário, e
se chegarão à minha mesa, para me servirem, e guardarão a minha
ordenança.

E será que, quando entrarem pelas portas do átrio interior, se
vestirão com vestes de linho; e não se porá lã sobre eles, quando
servirem nas portas do átrio interior, e dentro. Gorros de
linho estarão sobre as suas cabeças, e calções de linho sobre os
seus lombos; não se cingirão de modo que lhes venha suor. E,
saindo eles ao átrio exterior, ao átrio de fora, ao povo, despirão
as suas vestiduras com que ministraram, e as porão nas santas
câmaras, e se vestirão de outras vestes, para que não santifiquem o
povo estando com as suas vestiduras. E não raparão a sua
cabeça, nem deixarão crescer o cabelo; antes, como convém,
tosquiarão as suas cabeças. E nenhum sacerdote beberá vinho
quando entrar no átrio interior. E eles não se casarão nem
com viúva nem com repudiada, mas tomarão virgens da linhagem da casa
de Israel, ou viúva que for viúva de sacerdote. E a meu povo
ensinarão a distinguir entre o santo e o profano, e o farão
discernir entre o impuro e o puro. E, quando houver disputa,
eles assistirão a ela para a julgarem; pelos meus juízos as
julgarão; e as minhas leis e os meus estatutos guardarão em todas as
minhas solenidades, e santificarão os meus sábados. E eles
não se aproximarão de nenhum homem morto, para se contaminarem; mas
por pai, ou por mãe, ou por filho, ou por filha, ou por irmão, ou
por irmã que não tiver marido, se poderão contaminar. E,
depois da sua purificação, contar-se-lhe-ão sete dias. E, no
dia em que ele entrar no lugar santo, no átrio interior, para
ministrar no lugar santo, oferecerá a sua expiação pelo pecado, diz
o Senhor Deus. Eles terão uma herança: eu serei a sua
herança. Não lhes dareis, portanto, possessão em Israel; eu sou a
sua possessão. Eles comerão a oferta de alimentos, e a oferta
pelo pecado e a oferta pela culpa; e toda a coisa consagrada em
Israel será deles. E as primícias de todos os primeiros
frutos de tudo, e toda a oblação de tudo, de todas as vossas
oblações, serão dos sacerdotes; também as primeiras das vossas
massas dareis ao sacerdote, para que faça repousar a bênção sobre a
tua casa. Nenhuma coisa, que tenha morrido ou tenha sido
despedaçada, de aves e de animais, comerão os sacerdotes.

\medskip

\lettrine{45} Quando, pois, repartirdes a terra em herança,
oferecereis uma oferta ao Senhor, uma porção santa da terra; o seu
comprimento será de vinte e cinco mil canas e a largura de dez mil.
Esta será santa em toda a sua extensão ao redor. Desta porção o
santuário ocupará quinhentas canas de comprimento, e quinhentas de
largura, em quadrado, e terá em redor um espaço vazio de cinqüenta
côvados. E desta porção medirás vinte e cinco mil côvados de
comprimento, e a largura de dez mil; e ali estará o santuário, o
lugar santíssimo. Esta será a porção santa da terra; ela será
para os sacerdotes, ministros do santuário, que dele se aproximam
para servir ao Senhor; e lhes servirá de lugar para suas casas, e de
lugar santo para o santuário. E os levitas, ministros da casa,
terão em sua possessão, vinte e cinco mil canas de comprimento, para
vinte câmaras. E para possessão da cidade, de largura dareis
cinco mil canas, e de comprimento vinte e cinco mil, defronte da
oferta santa; o que será para toda a casa de Israel. O príncipe,
porém, terá a sua parte deste e do outro lado da área santa, e da
possessão da cidade, diante da santa oferta, e em frente da
possessão da cidade, desde o extremo ocidental até o extremo
oriental, e de comprimento, corresponderá a uma das porções, desde o
termo ocidental até ao termo oriental. E esta terra será a sua
possessão em Israel; e os meus príncipes nunca mais oprimirão o meu
povo, antes deixarão a terra à casa de Israel, conforme as suas
tribos.

Assim diz o Senhor Deus: Basta já, ó príncipes de Israel; afastai
a violência e a assolação e praticai juízo e justiça; tirai as
vossas imposições do meu povo, diz o Senhor Deus. Tereis
balanças justas, efa justo e bato justo. O efa e o bato serão
de uma mesma medida, de modo que o bato contenha a décima parte do
ômer, e o efa a décima parte do ômer; conforme o ômer será a sua
medida. E o siclo será de vinte geras; vinte siclos, vinte e
cinco siclos, e quinze siclos terá a vossa mina.

Esta será a oferta que haveis de oferecer: a sexta parte de um
efa de cada ômer de trigo; também dareis a sexta parte de um efa de
cada ômer de cevada. Quanto à ordenança do azeite, de cada
bato de azeite oferecereis a décima parte de um bato tirado de um
coro, que é um ômer de dez batos; porque dez batos fazem um ômer.
E um cordeiro do rebanho, de cada duzentos, da terra mais
regada de Israel, para oferta de alimentos, e para holocausto, e
para sacrifício pacífico; para que façam expiação por eles, diz o
Senhor Deus. Todo o povo da terra concorrerá com esta oferta,
para o príncipe em Israel. E estarão a cargo do príncipe os
holocaustos, e as ofertas de alimentos, e as libações, nas festas, e
nas luas novas, e nos sábados, em todas as solenidades da casa de
Israel. Ele preparará a oferta pelo pecado, e a oferta de alimentos,
e o holocausto, e os sacrifícios pacíficos, para fazer expiação pela
casa de Israel. Assim diz o Senhor Deus: No primeiro mês, no
primeiro dia do mês, tomarás um bezerro sem mancha e purificarás o
santuário. E o sacerdote tomará do sangue do sacrifício pelo
pecado, e porá dele nas ombreiras da casa, e nos quatro cantos da
armação do altar, e nas ombreiras da porta do átrio interior.
Assim também farás no sétimo dia do mês, pelos que erram, e
pelos símplices; assim expiareis a casa. No primeiro mês, no
dia catorze do mês, tereis a páscoa, uma festa de sete dias; pão
ázimo se comerá. E no mesmo dia o príncipe preparará por si e
por todo o povo da terra, um bezerro como oferta pelo pecado.
E durante os sete dias da festa preparará um holocausto ao
Senhor, de sete bezerros e sete carneiros sem mancha, cada dia,
durante os sete dias; e em sacrifício pelo pecado um bode cada dia.
Também preparará uma oferta de alimentos, a saber, um efa,
para cada bezerro, e um efa para cada carneiro, e um him de azeite
para cada efa. No sétimo mês, no dia quinze do mês, na festa,
fará o mesmo por sete dias, tanto o sacrifício pelo pecado, como o
holocausto, e como a oferta de alimentos, e como o azeite.

\medskip

\lettrine{46} Assim diz o Senhor Deus: A porta do átrio
interior que dá para o oriente, estará fechada durante os seis dias
que são de trabalho; mas no dia de sábado ela se abrirá; também no
dia da lua nova se abrirá. E o príncipe entrará pelo caminho do
vestíbulo da porta, por fora, e permanecerá junto da ombreira da
porta; e os sacerdotes prepararão o holocausto, e os sacrifícios
pacíficos dele; e ele adorará junto ao umbral da porta, e sairá; mas
a porta não se fechará até à tarde. E o povo da terra adorará à
entrada da mesma porta, nos sábados e nas luas novas, diante do
Senhor. E o holocausto, que o príncipe oferecer ao Senhor, será,
no dia de sábado, seis cordeiros sem mancha e um carneiro sem
mancha. E a oferta de alimentos será um efa para o carneiro; e
para o cordeiro, a oferta de alimentos será o que puder dar; e de
azeite um him para cada efa. Mas no dia da lua nova será um
bezerro sem mancha, e seis cordeiros e um carneiro; eles serão sem
mancha. E preparará por oferta de manjares um efa para o bezerro
e um efa para o carneiro, mas para os cordeiros, o que a sua mão
puder dar; e um him de azeite para um efa. E, quando entrar o
príncipe, entrará pelo caminho do vestíbulo da porta, e sairá pelo
mesmo caminho. Mas, quando vier o povo da terra perante a face
do Senhor nas solenidades, aquele que entrar pelo caminho da porta
do norte, para adorar, sairá pelo caminho da porta do sul; e aquele
que entrar pelo caminho da porta do sul sairá pelo caminho da porta
do norte; não tornará pelo caminho da porta por onde entrou, mas
sairá pela outra que está oposta. E o príncipe entrará no
meio deles; quando eles entrarem e, saindo eles, sairão todos.
E nas festas e nas solenidades a oferta de alimentos será um
efa para o bezerro, e um efa para o carneiro, mas para os cordeiros
o que puder dar; e de azeite um him para um efa. E, quando o
príncipe fizer oferta voluntária de holocaustos, ou de sacrifícios
pacíficos, uma oferta voluntária ao Senhor, então lhe abrirão a
porta que dá para o oriente, e fará o seu holocausto e os seus
sacrifícios pacíficos, como houver feito no dia de sábado; e sairá,
e se fechará a porta depois dele sair. E prepararás um
cordeiro de um ano sem mancha, em holocausto ao Senhor, cada dia;
todas as manhãs o prepararás. E, juntamente com ele
prepararás uma oferta de alimentos, todas as manhãs, a sexta parte
de um efa, e de azeite a terça parte de um him, para misturar com a
flor de farinha; por oferta de alimentos para o Senhor, em estatutos
perpétuos e contínuos. Assim prepararão o cordeiro, e a
oferta de alimentos, e o azeite, todas as manhãs, em holocausto
contínuo.

Assim diz o Senhor Deus: Quando o príncipe der um presente a
algum de seus filhos, é sua herança, pertencerá a seus filhos; será
possessão deles por herança. Mas, dando ele um presente da
sua herança a algum dos seus servos, será deste até ao ano da
liberdade; então tornará para o príncipe, porque herança dele é;
seus filhos a herdarão. E o príncipe não tomará nada da
herança do povo por opressão, defraudando-os da sua possessão; da
sua própria possessão deixará herança a seus filhos, para que o meu
povo não seja separado, cada um da sua possessão.

Depois disto me trouxe pela entrada que estava ao lado da porta,
às câmaras santas dos sacerdotes, que olhavam para o norte; e eis
que ali havia um lugar nos fundos extremos, para o lado do ocidente.
E ele me disse: Este é o lugar onde os sacerdotes cozerão a
oferta pela culpa, e a oferta pelo pecado, e onde cozerão a oferta
de alimentos, para que não as tragam ao átrio exterior para
santificarem o povo. Então me levou para fora, para o átrio
exterior, e me fez passar pelos quatro cantos do átrio; e eis que em
cada canto do átrio havia outro átrio. Nos quatro cantos do
átrio havia outros átrios juntos, de quarenta côvados de comprimento
e de trinta de largura; estes quatro cantos tinham uma mesma medida.
E havia uma fileira construída ao redor deles, ao redor dos
quatro; e havia cozinhas feitas por baixo das fileiras ao redor.
E me disse: Estas são as cozinhas, onde os ministros da casa
cozerão o sacrifício do povo.

\medskip

\lettrine{47} Depois disto me fez voltar à porta da casa, e
eis que saíam águas por debaixo do umbral da casa para o oriente;
porque a face da casa dava para o oriente, e as águas desciam de
debaixo, desde o lado direito da casa, ao sul do altar. E ele me
fez sair pelo caminho da porta do norte, e me fez dar uma volta pelo
caminho de fora, até à porta exterior, pelo caminho que dá para o
oriente e eis que corriam as águas do lado direito. E saiu
aquele homem para o oriente, tendo na mão um cordel de medir; e
mediu mil côvados, e me fez passar pelas águas, águas que me davam
pelos artelhos. E mediu mais mil côvados, e me fez passar pelas
águas, águas que me davam pelos joelhos; e outra vez mediu mil, e me
fez passar pelas águas que me davam pelos lombos. E mediu mais
mil, e era um rio, que eu não podia atravessar, porque as águas eram
profundas, águas que se deviam passar a nado, rio pelo qual não se
podia passar. E disse-me: Viste isto, filho do homem? Então
levou-me, e me fez voltar para a margem do rio. E, tendo eu
voltado, eis que à margem do rio havia uma grande abundância de
árvores, de um e de outro lado. Então disse-me: Estas águas saem
para a região oriental, e descem ao deserto, e entram no mar; e,
sendo levadas ao mar, as águas tornar-se-ão saudáveis. E será
que toda a criatura vivente que passar por onde quer que entrarem
estes rios viverá; e haverá muitíssimo peixe, porque lá chegarão
estas águas, e serão saudáveis, e viverá tudo por onde quer que
entrar este rio. Será também que os pescadores estarão em pé
junto dele; desde En-Gedi até En-Eglaim haverá lugar para estender
as redes; o seu peixe, segundo a sua espécie, será como o peixe do
mar grande, em multidão excessiva. Mas os seus charcos e os
seus pântanos não tornar-se-ão saudáveis; serão deixados para sal.
E junto ao rio, à sua margem, de um e de outro lado, nascerá
toda a sorte de árvore que dá fruto para se comer; não cairá a sua
folha, nem acabará o seu fruto; nos seus meses produzirá novos
frutos, porque as suas águas saem do santuário; e o seu fruto
servirá de comida e a sua folha de remédio.

Assim diz o Senhor Deus: Este será o termo conforme o qual
repartireis a terra em herança, segundo as doze tribos de Israel;
José terá duas partes. E vós a herdareis, tanto um como o
outro; terra sobre a qual levantei a minha mão, para dá-la a vossos
pais; assim esta mesma terra vos cairá a vós em herança. E
este será o termo da terra; do lado do norte, desde o mar grande,
caminho de Hetlom, até à entrada de Zedade; Hamate, Berota,
Sibraim, que estão entre o termo de Damasco e o termo de Hamate;
Hazer-Haticom, que está junto ao termo de Haurã. E o termo
será desde o mar até Hazar-Enom, o termo de Damasco, e na direção do
norte, para o norte, está o termo de Hamate. Este será o lado do
norte. E o lado do oriente, entre Haurã, e Damasco, e
Gileade, e a terra de Israel será o Jordão; desde o termo do norte
até ao mar do oriente medireis. Este será o lado do oriente.
E o lado do sul, para o sul, será desde Tamar até às águas da
contenda de Cades, junto ao ribeiro, até ao mar grande. Este será o
lado do sul. E o lado do ocidente será o mar grande, desde o
termo do sul até a entrada de Hamate. Este será o lado do ocidente.
Repartireis, pois, esta terra entre vós, segundo as tribos de
Israel. Será, porém, que a sorteareis para vossa herança, e
para a dos estrangeiros que habitam no meio de vós, que gerarão
filhos no meio de vós; e vos serão como naturais entre os filhos de
Israel; convosco entrarão em herança, no meio das tribos de Israel.
E será que na tribo em que habitar o estrangeiro, ali lhe
dareis a sua herança, diz o Senhor Deus.

\medskip

\lettrine{48} E estes são os nomes das tribos: desde o extremo
norte, ao longo do caminho de Hetlom, indo para Hamate, até
Hazar-Enom, termo de Damasco para o norte, ao pé de Hamate, terá Dã
uma parte, desde o lado oriental até o ocidental. E junto ao
termo de Dã, desde o lado oriental até o ocidental, Aser terá uma
porção. E junto ao termo de Aser, desde o lado oriental até o
ocidental, Naftali, uma porção. E junto ao termo de Naftali,
desde o lado oriental até o lado ocidental, Manassés, uma porção.
E junto ao termo de Manassés, desde o lado oriental até o lado
ocidental, Efraim, uma porção. E junto ao termo de Efraim, desde
o lado oriental até o lado ocidental, Rúben, uma porção. E junto
ao termo de Rúben, desde o lado oriental até o lado ocidental, Judá,
uma porção. E junto ao termo de Judá, desde o lado oriental até
o lado ocidental, será a oferta que haveis de fazer de vinte e cinco
mil canas de largura, e de comprimento de cada uma das porções,
desde o lado oriental até o lado ocidental; e o santuário estará no
meio dela. A oferta que haveis de oferecer ao Senhor será do
comprimento de vinte e cinco mil canas, e da largura de dez mil.
E ali será a oferta santa para os sacerdotes, medindo para o
norte vinte e cinco mil canas de comprimento, e para o ocidente dez
mil de largura, e para o oriente dez mil de largura, e para o sul
vinte e cinco mil de comprimento; e o santuário do Senhor estará no
meio dela. E será para os sacerdotes santificados dentre os
filhos de Zadoque, que guardaram a minha ordenança, que não se
desviaram, quando os filhos de Israel se extraviaram, como se
extraviaram os outros levitas. E eles terão uma oferta, da
oferta da terra, lugar santíssimo, junto ao limite dos levitas.
E os levitas terão, consoante ao termo dos sacerdotes, vinte
e cinco mil canas de comprimento, e de largura dez mil; todo o
comprimento será vinte e cinco mil, e a largura dez mil. E
não venderão disto, nem trocarão, nem transferirão as primícias da
terra, porque é santidade ao Senhor. Mas as cinco mil canas,
as que restaram da largura, diante das vinte e cinco mil, ficarão
para uso comum, para a cidade, para habitação e para arrabaldes; e a
cidade estará no meio delas. E estas serão as suas medidas: o
lado do norte de quatro mil e quinhentas canas, o lado do sul de
quatro mil e quinhentas, o lado oriental de quatro mil e quinhentas
e o lado ocidental de quatro mil e quinhentas. E os
arrabaldes da cidade serão para o norte de duzentas e cinqüenta
canas, para o sul de duzentas e cinqüenta, para o oriente de
duzentas e cinqüenta e para o ocidente de duzentas e cinqüenta.
E, quanto ao que restou do comprimento, consoante com a santa
oferta, será dez mil para o oriente, e dez mil para o ocidente; e
corresponderá à santa oferta; e a sua novidade será para sustento
daqueles que servem a cidade. E os que servem à cidade,
servi-la-ão dentre todas as tribos de Israel. Toda a oferta
será de vinte e cinco mil canas com mais vinte e cinco mil; em
quadrado oferecereis a oferta santa, com a possessão da cidade.
E o que restou será para o príncipe; deste e do outro lado da
oferta santa, e da possessão da cidade, diante das vinte e cinco mil
canas da oferta, até ao termo do oriente e do ocidente, diante das
vinte e cinco mil, até ao termo do ocidente, correspondente às
porções, será para o príncipe; e a santa oferta e o santuário da
casa estarão no meio dela. E desde a possessão dos levitas, e
desde a possessão da cidade, no meio do que pertencer ao príncipe,
entre o termo de Judá, e o termo de Benjamim, será isso para o
príncipe. E, quanto ao restante das tribos, desde o lado
oriental até o lado ocidental, Benjamim terá uma porção. E
junto ao termo de Benjamim, desde o lado oriental até o lado
ocidental, Simeão terá uma porção. E junto ao termo de
Simeão, desde o lado oriental até o lado ocidental, Issacar terá uma
porção. E junto ao termo de Issacar, desde o lado oriental
até o lado ocidental, Zebulom terá uma porção. E junto ao
termo de Zebulom, desde o lado oriental até o lado ocidental, Gade
terá uma porção. E junto ao termo de Gade, ao sul, do lado
sul, será o termo desde Tamar até às águas da contenda de Cades,
junto ao rio até ao mar grande. Esta é a terra que sorteareis
em herança às tribos de Israel; e estas são as suas porções, diz o
Senhor Deus. E estas são as saídas da cidade, desde o lado
norte: quatro mil e quinhentas canas por medida.

E as portas da cidade serão conforme os nomes das tribos de
Israel; três portas para o norte: a porta de Rúben uma, a porta de
Judá outra, a porta de Levi outra. E do lado oriental quatro
mil e quinhentas canas, e três portas, a saber: a porta de José uma,
a porta de Benjamim outra, a porta de Dã outra. E do lado sul
quatro mil e quinhentas canas por medida, e três portas: a porta de
Simeão uma, a porta de Issacar outra, a porta de Zebulom outra.
Do lado ocidental quatro mil e quinhentas canas, e as suas
três portas: a porta de Gade uma, a porta de Aser outra, a porta de
Naftali outra. Dezoito mil canas por medida terá ao redor; e
o nome da cidade desde aquele dia será: O SENHOR ESTÁ ALI.

