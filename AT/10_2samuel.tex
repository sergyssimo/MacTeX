\addchap{Segundo livro de Samuel}

\lettrine{1} E sucedeu que, depois da morte de Saul, voltando
Davi da derrota dos amalequitas, ficou dois dias em Ziclague; ao
terceiro dia um homem veio do arraial de Saul, com as vestes rotas e
com terra sobre a cabeça; e, chegando ele a Davi, se lançou no chão,
e se inclinou. E Davi lhe disse: Donde vens? E ele lhe disse:
Escapei do arraial de Israel. E disse-lhe Davi: Como foi lá
isso? peço-te, dize-mo. E ele lhe respondeu: O povo fugiu da
batalha, e muitos do povo caíram, e morreram; assim como também Saul
e Jônatas, seu filho, foram mortos. E disse Davi ao moço que lhe
trazia as novas: Como sabes tu que Saul e Jônatas, seu filho, foram
mortos? Então disse o moço que lhe dava a notícia: Cheguei por
acaso à montanha de Gilboa, e eis que Saul estava encostado sobre a
sua lança, e eis que os carros e a cavalaria apertavam-no. E,
olhando ele para trás de si, viu-me, e chamou-me; e eu disse: Eis-me
aqui. E ele me disse: Quem és tu? E eu lhe disse: Sou
amalequita. Então ele me disse: Peço-te, arremessa-te sobre mim,
e mata-me, porque angústias me têm cercado, pois toda a minha vida
está ainda em mim. Arremessei-me, pois, sobre ele, e o matei,
porque bem sabia eu que não viveria depois da sua queda, e tomei a
coroa que tinha na cabeça, e o bracelete que trazia no braço, e os
trouxe aqui a meu senhor.

Então apanhou Davi as suas vestes, e as rasgou; assim fizeram
todos os homens que estavam com ele. E prantearam, e
choraram, e jejuaram até à tarde por Saul, e por Jônatas, seu filho,
e pelo povo do Senhor, e pela casa de Israel, porque tinham caído à
espada. Disse então Davi ao moço que lhe trouxera a nova:
Donde és tu? E disse ele: Sou filho de um estrangeiro, amalequita.
E Davi lhe disse: Como não temeste tu estender a mão para
matares ao ungido do Senhor? Então chamou Davi a um dos
moços, e disse: Chega, e lança-te sobre ele. E ele o feriu, e
morreu. Pois Davi lhe dissera: O teu sangue seja sobre a tua
cabeça, porque a tua própria boca testificou contra ti, dizendo: Eu
matei o ungido do Senhor.

E lamentou Davi a Saul e a Jônatas, seu filho, com esta
lamentação (dizendo ele que ensinassem aos filhos de Judá o
uso do arco. Eis que está escrito no livro de Jasher): Ah,
ornamento de Israel! Nos teus altos foi ferido, como caíram os
poderosos! Não o noticieis em Gate, não o publiqueis nas ruas
de Ascalom, para que não se alegrem as filhas dos filisteus, para
que não saltem de contentamento as filhas dos incircuncisos.
Vós, montes de Gilboa, nem orvalho, nem chuva caia sobre vós,
nem haja campos de ofertas alçadas, pois aí desprezivelmente foi
arrojado o escudo dos poderosos, o escudo de Saul, como se não fora
ungido com óleo. Do sangue dos feridos, da gordura dos
valentes, nunca se retirou para trás o arco de Jônatas, nem voltou
vazia a espada de Saul. Saul e Jônatas, tão amados e queridos
na sua vida, também na sua morte não se separaram; eram mais
ligeiros do que as águias, mais fortes do que os leões. Vós,
filhas de Israel, chorai por Saul, que vos vestia de escarlata em
delícias, que vos fazia trazer ornamentos de ouro sobre as vossas
vestes. Como caíram os poderosos, no meio da peleja! Jônatas
nos teus altos foi morto. Angustiado estou por ti, meu irmão
Jônatas; quão amabilíssimo me eras! Mais maravilhoso me era o teu
amor do que o amor das mulheres. Como caíram os poderosos, e
pereceram as armas de guerra!

\medskip

\lettrine{2} E sucedeu depois disto que Davi consultou ao
Senhor, dizendo: Subirei a alguma das cidades de Judá? E disse-lhe o
Senhor: Sobe. E falou Davi: Para onde subirei? E disse: Para Hebrom.
E subiu Davi para lá, e também as suas duas mulheres, Ainoã, a
jizreelita, e Abigail, a mulher de Nabal, o carmelita. Fez
também Davi subir os homens que estavam com ele, cada um com a sua
família; e habitaram nas cidades de Hebrom. Então vieram os
homens de Judá, e ungiram ali a Davi rei sobre a casa de Judá. E
deram avisos a Davi, dizendo: Os homens de Jabes-Gileade foram os
que sepultaram a Saul. Então enviou Davi mensageiros aos homens
de Jabes-Gileade, para dizer-lhes: Benditos sejais vós do Senhor,
que fizestes tal beneficência a vosso senhor, a Saul, e o
sepultastes! Agora, pois, o Senhor use convosco de beneficência
e fidelidade; e também eu vos farei este bem, porquanto fizestes
isto. Esforcem-se, pois, agora as vossas mãos, e sede homens
valentes, pois Saul, vosso senhor, é morto, mas também os da casa de
Judá já me ungiram a mim por seu rei.

Porém Abner, filho de Ner, capitão do exército de Saul, tomou a
Is-Bosete, filho de Saul, e o fez passar a Maanaim, e o
constituiu rei sobre Gileade, e sobre os assuritas, e sobre Jizreel,
e sobre Efraim, e sobre Benjamim, e sobre todo o Israel. Da
idade de quarenta anos era Is-Bosete, filho de Saul, quando começou
a reinar sobre Israel, e reinou dois anos; mas os da casa de Judá
seguiam a Davi. E foi o número dos dias que Davi reinou em
Hebrom, sobre a casa de Judá, sete anos e seis meses. Então
saiu Abner, filho de Ner, com os servos de Is-Bosete, filho de Saul,
de Maanaim a Gibeom. Saíram também Joabe, filho de Zeruia, e
os servos de Davi, e se encontraram uns com os outros perto do
tanque de Gibeom; e pararam estes deste lado do tanque, e os outros
do outro lado do tanque. E disse Abner a Joabe: Deixa
levantar os moços, e joguem diante de nós. E disse Joabe:
Levantem-se. Então se levantaram, e passaram, em número de
doze de Benjamim, da parte de Is-Bosete, filho de Saul, e doze dos
servos de Davi. E cada um lançou mão da cabeça do outro,
cravou-lhe a espada no lado, e caíram juntos, por isso se chamou
àquele lugar Helcate-Hazurim, que está junto a Gibeom. E
seguiu-se naquele dia uma crua peleja; porém Abner e os homens de
Israel foram feridos diante dos servos de Davi.

E estavam ali os três filhos de Zeruia, Joabe, Abisai, e Asael; e
Asael era ligeiro de pés, como as gazelas do campo. E Asael
perseguiu a Abner; e não se desviou de detrás de Abner, nem para a
direita nem para a esquerda. E Abner, olhando para trás,
perguntou: És tu Asael? E ele falou: Eu sou. Então lhe disse
Abner: Desvia-te para a direita, ou para a esquerda, e lança mão de
um dos moços, e toma os seus despojos. Porém Asael não quis
desviar-se de detrás dele. Então Abner tornou a dizer a
Asael: Desvia-te de detrás de mim; por que hei de eu ferir-te e dar
contigo em terra? E como levantaria eu o meu rosto diante de Joabe,
teu irmão? Porém, não querendo ele se desviar, Abner o feriu
com a ponta da lança pela quinta costela, e a lança lhe saiu por
detrás, e caiu ali, e morreu naquele mesmo lugar; e sucedeu que,
todos os que chegavam ao lugar onde Asael caiu e morreu, paravam.
Porém Joabe e Abisai perseguiram a Abner; e pôs-se o sol,
chegando eles ao outeiro de Amá, que está diante de Gia, junto ao
caminho do deserto de Gibeão.

E os filhos de Benjamim se ajuntaram atrás de Abner, e fizeram um
batalhão, e puseram-se no cume de um outeiro. Então Abner
gritou a Joabe, e disse: Consumirá a espada para sempre? Não sabes
tu que por fim haverá amargura? E até quando não hás de dizer ao
povo que deixe de perseguir a seus irmãos? E disse Joabe:
Vive Deus, que, se não tivesses falado, só pela manhã o povo teria
cessado, cada um, de perseguir a seu irmão. Então Joabe tocou
a buzina, e todo o povo parou, e não perseguiram mais a Israel; e
tampouco pelejaram mais. E caminharam Abner e os seus homens
toda aquela noite pela planície; e, passando o Jordão, caminharam
por todo o Bitrom, e chegaram a Maanaim. Também Joabe voltou
de perseguir a Abner, e ajuntou todo o povo; e dos servos de Davi
faltaram dezenove homens, e Asael. Porém os servos de Davi
feriram dentre os de Benjamim, e dentre os homens de Abner, a
trezentos e sessenta homens, que ali ficaram mortos. E
levantaram a Asael, e sepultaram-no na sepultura de seu pai, que
estava em Belém; e Joabe e seus homens caminharam toda aquela noite,
e amanheceu-lhes o dia em Hebrom.

\medskip

\lettrine{3} E houve uma longa guerra entre a casa de Saul e a
casa de Davi; porém Davi ia se fortalecendo, mas os da casa de Saul
se iam enfraquecendo. E a Davi nasceram filhos em Hebrom; e foi
o seu primogênito Amnom, de Ainoã a jizreelita; e seu segundo,
Quileabe, de Abigail, mulher de Nabal, o carmelita; e o terceiro
Absalão, filho de Maaca, filha de Talmai, rei de Gesur; e o
quarto, Adonias, filho de Hagite; e o quinto, Sefatias, filho de
Abital; e o sexto, Itreão, de Eglá, também mulher de Davi; estes
nasceram a Davi em Hebrom. E, havendo guerra entre a casa de
Saul e a casa de Davi, sucedeu que Abner se fez poderoso na casa de
Saul.

E tinha tido Saul uma concubina, cujo nome era Rispa, filha de
Aiá; e disse Is-Bosete a Abner: Por que possuíste a concubina de meu
pai? Então se irou muito Abner pelas palavras de Is-Bosete, e
disse: Sou eu cabeça de cão, que pertença a Judá? Ainda hoje faço
beneficência à casa de Saul, teu pai, a seus irmãos, e a seus
amigos, e não te entreguei nas mãos de Davi, e tu hoje buscas motivo
para me argüires por causa da maldade de uma mulher. Assim faça
Deus a Abner, e outro tanto, se, como o Senhor jurou a Davi, assim
eu não lhe fizer, transferindo o reino da casa de Saul, e
confirmando o trono de Davi sobre Israel, e sobre Judá, desde Dã até
Berseba. E nenhuma palavra podia ele responder a Abner,
porque o temia. Então enviou Abner da sua parte mensageiros a
Davi, dizendo: De quem é a terra? E disse mais: Comigo faze o teu
acordo, e eis que a minha mão será contigo, para tornar a ti todo o
Israel. E disse Davi: Bem, eu farei contigo acordo, porém uma
coisa te peço: não verás a minha face, se primeiro não me trouxeres
a Mical, filha de Saul, quando vieres ver a minha face.
Também enviou Davi mensageiros a Is-Bosete, filho de Saul,
dizendo: Dá-me minha mulher Mical, que eu desposei por cem prepúcios
de filisteus. E enviou Is-Bosete, e tirou-a de seu marido, a
Paltiel, filho de Laís. E ia com ela seu marido, caminhando,
e chorando atrás dela, até Baurim. Então lhe disse Abner: Vai-te,
agora volta. E ele voltou. E falou Abner com os anciãos de
Israel, dizendo: Já há muito tempo que procuráveis que Davi reinasse
sobre vós. Fazei-o, pois, agora, porque o Senhor falou a
Davi, dizendo: Pela mão de Davi meu servo livrarei o meu povo das
mãos dos filisteus e das mãos de todos os seus inimigos. E
falou também Abner aos de Benjamim; e foi também Abner dizer aos de
Davi, em Hebrom, tudo o que era bom aos olhos de Israel e aos olhos
de toda a casa de Benjamim. E foi Abner a Davi, em Hebrom, e
vinte homens com ele; e Davi fez um banquete a Abner e aos homens
que com ele estavam. Então disse Abner a Davi: Eu me
levantarei, e irei, e ajuntarei ao rei meu senhor todo o Israel,
para fazer acordo contigo; e tu reinarás sobre tudo o que desejar a
tua alma. Assim despediu Davi a Abner, e ele foi em paz.

E eis que os servos de Davi e Joabe vieram de uma batalha, e
traziam consigo grande despojo; e já Abner não estava com Davi em
Hebrom, porque o tinha despedido, e se tinha ido em paz.
Chegando, pois, Joabe, e todo o exército que vinha com ele,
deram aviso a Joabe, dizendo: Abner, filho de Ner, veio ao rei, e o
despediu, e foi em paz. Então Joabe foi ao rei, e disse: Que
fizeste? Eis que Abner veio ter contigo; por que pois o despediste,
de maneira que se fosse assim livremente? Bem conheces a
Abner, filho de Ner, que te veio enganar, e saber a tua saída e a
tua entrada, e entender tudo quanto fazes. E Joabe,
retirando-se de Davi, enviou mensageiros atrás de Abner, e o fizeram
voltar desde o poço de Sirá, sem que Davi o soubesse.
Voltando, pois, Abner a Hebrom, Joabe o levou à parte, à
entrada da porta, para lhe falar em segredo; e feriu-o ali pela
quinta costela, e morreu, por causa do sangue de Asael seu irmão.
O que Davi depois ouvindo, disse: Inocente sou eu, e o meu
reino, para com o Senhor, para sempre, do sangue de Abner, filho de
Ner. Caia sobre a cabeça de Joabe e sobre toda a casa de seu
pai, e nunca na casa de Joabe falte quem tenha fluxo, ou quem seja
leproso, ou quem se atenha a bordão, ou quem caia à espada, ou quem
necessite de pão. Joabe, pois, e Abisai, seu irmão, mataram a
Abner, por ter morto a Asael, seu irmão, na peleja em Gibeão.
Disse, pois, Davi a Joabe, e a todo o povo que com ele
estava: Rasgai as vossas vestes; e cingi-vos de sacos e ide
pranteando diante de Abner. E o rei Davi ia seguindo o
féretro\footnote{Tumba; ataúde; esquife; andor (padiola portátil e
ornamentada, sobre a qual se conduzem imagens nas procissões;
charola, andas) em que nos triunfos romanos se levavam os despojos
dos vencidos.}. E, sepultando a Abner em Hebrom, o rei
levantou a sua voz, e chorou junto da sepultura de Abner; e chorou
todo o povo. E o rei, pranteando Abner, disse: Havia de
morrer Abner como morre o vilão? As tuas mãos não estavam
atadas, nem os teus pés carregados de grilhões, mas caíste como os
que caem diante dos filhos da maldade! Então todo o povo chorou
muito mais por ele. Depois todo o povo veio fazer com que
Davi comesse pão, sendo ainda dia; porém Davi jurou, dizendo: Assim
Deus me faça, e outro tanto, se, antes que o sol se ponha, eu provar
pão ou alguma coisa. O que todo o povo entendendo, pareceu
bem aos seus olhos; assim como tudo quanto o rei fez pareceu bem aos
olhos de todo o povo. E todo o povo e todo o Israel
entenderam naquele mesmo dia que não procedera do rei que matasse a
Abner, filho de Ner. Então disse o rei aos seus servos: Não
sabeis que hoje caiu em Israel um príncipe e um grande? Que
eu hoje estou fraco, ainda que ungido rei; estes homens, filhos de
Zeruia, são mais duros do que eu; o Senhor pagará ao malfeitor,
conforme a sua maldade.


\medskip

\lettrine{4} Ouvindo, pois, o filho de Saul, que Abner morrera
em Hebrom, as mãos se lhe afrouxaram; e todo o Israel pasmou. E
tinha o filho de Saul dois homens capitães de tropas; e era o nome
de um Baaná, e o nome do outro Recabe, filhos de Rimom, o beerotita,
dos filhos de Benjamim, porque também Beerote se reputava de
Benjamim. E tinham fugido os beerotitas para Gitaim, e ali têm
peregrinado até ao dia de hoje. E Jônatas, filho de Saul, tinha
um filho aleijado de ambos os pés; era da idade de cinco anos quando
as novas de Saul e Jônatas vieram de Jizreel, e sua ama o tomou, e
fugiu; e sucedeu que, apressando-se ela a fugir, ele caiu, e ficou
coxo; e o seu nome era Mefibosete. E foram os filhos de Rimom, o
beerotita, Recabe e Baaná, e entraram em casa de Is-Bosete no maior
calor do dia, estando ele deitado a dormir, ao meio dia. E ali
entraram até ao meio da casa, como que vindo buscar trigo, e o
feriram na quinta costela; e Recabe e Baaná, seu irmão, escaparam.
Porque entraram na sua casa, estando ele na cama deitado, no seu
quarto, e o feriram, e o mataram, e lhe cortaram a cabeça; e,
tomando a sua cabeça, andaram toda a noite caminhando pela planície.
E trouxeram a cabeça de Is-Bosete a Davi, a Hebrom, e disseram
ao rei: Eis aqui a cabeça de Is-Bosete, filho de Saul, teu inimigo,
que procurava a tua morte; assim o Senhor vingou hoje ao rei meu
senhor, de Saul e da sua descendência.

Porém Davi, respondendo a Recabe e a Baaná, seu irmão, filhos de
Rimom, o beerotita, disse-lhes: Vive o Senhor, que remiu a minha
alma de toda a angústia, se aquele que me trouxe novas,
dizendo: Eis que Saul é morto, parecendo-lhe, porém, aos olhos que
era como quem trazia boas novas, eu logo lancei mão dele, e o matei
em Ziclague, cuidando ele que eu por isso lhe desse recompensa.
Quanto mais a ímpios homens, que mataram um homem justo em
sua casa, sobre a sua cama; agora, pois, não requereria eu o seu
sangue de vossas mãos, e não vos exterminaria da terra? E deu
Davi ordem aos seus moços que os matassem; e cortaram-lhes os pés e
as mãos, e os penduraram sobre o tanque de Hebrom; tomaram, porém, a
cabeça de Is-Bosete, e a sepultaram na sepultura de Abner, em
Hebrom.

\medskip

\lettrine{5} Então todas as tribos de Israel vieram a Davi, em
Hebrom, e falaram, dizendo: Eis-nos aqui, somos teus ossos e tua
carne. E também outrora, sendo Saul ainda rei sobre nós, eras tu
o que saías e entravas com Israel; e também o Senhor te disse: Tu
apascentarás o meu povo de Israel, e tu serás príncipe sobre Israel.
Assim, pois, todos os anciãos de Israel vieram ao rei, em
Hebrom; e o rei Davi fez com eles acordo em Hebrom, perante o
Senhor; e ungiram a Davi rei sobre Israel. Da idade de trinta
anos era Davi quando começou a reinar; quarenta anos reinou. Em
Hebrom reinou sobre Judá sete anos e seis meses, e em Jerusalém
reinou trinta e três anos sobre todo o Israel e Judá.

E partiu o rei com os seus homens a Jerusalém, contra os jebuseus
que habitavam naquela terra; e falaram a Davi, dizendo: Não entrarás
aqui, pois os cegos e os coxos te repelirão, querendo dizer: Não
entrará Davi aqui. Porém Davi tomou a fortaleza de Sião; esta é
a cidade de Davi. Porque Davi disse naquele dia: Qualquer que
ferir aos jebuseus, suba ao canal e fira aos coxos e aos cegos, a
quem a alma de Davi odeia. Por isso se diz: Nem cego nem coxo
entrará nesta casa. Assim habitou Davi na fortaleza, e a chamou
a cidade de Davi; e Davi foi edificando em redor, desde Milo para
dentro. E Davi ia, cada vez mais, aumentando e crescendo,
porque o Senhor Deus dos Exércitos era com ele.

E Hirão, rei de Tiro, enviou mensageiros a Davi, e madeira de
cedro, e carpinteiros, e pedreiros que edificaram a Davi uma casa.
E entendeu Davi que o Senhor o confirmara rei sobre Israel, e
que exaltara o seu reino por amor do seu povo. E tomou Davi
mais concubinas e mulheres de Jerusalém, depois que viera de Hebrom;
e nasceram a Davi mais filhos e filhas. E estes são os nomes
dos que lhe nasceram em Jerusalém: Samua, e Sobabe, e Natã, e
Salomão, e Ibar, e Elisua, e Nefegue, e Jafia, e
Elisama, e Eliada, e Elifelete.

Ouvindo, pois, os filisteus que haviam ungido a Davi rei sobre
Israel, todos os filisteus subiram em busca de Davi; o que ouvindo
Davi, desceu à fortaleza. E os filisteus vieram, e se
estenderam pelo vale de Refaim. E Davi consultou ao Senhor,
dizendo: Subirei contra os filisteus? Entregar-mos-ás nas minhas
mãos? E disse o Senhor a Davi: Sobe, porque certamente entregarei os
filisteus nas tuas mãos. Então foi Davi a Baal-Perazim; e
feriu-os ali Davi, e disse: Rompeu o Senhor a meus inimigos diante
de mim, como quem rompe águas. Por isso chamou o nome daquele lugar
Baal-Perazim. E deixaram ali os seus ídolos; e Davi e os seus
homens os tomaram. E os filisteus tornaram a subir, e se
estenderam pelo vale de Refaim. E Davi consultou ao Senhor, o
qual disse: Não subirás; mas rodeia por detrás deles, e virás a eles
por defronte das amoreiras. E há de ser que, ouvindo tu um
estrondo de marcha pelas copas das amoreiras, então te apressarás;
porque o Senhor saiu então diante de ti, a ferir o arraial dos
filisteus. E fez Davi assim como o Senhor lhe tinha ordenado;
e feriu os filisteus desde Gibeá, até chegar a Gezer.

\medskip

\lettrine{6} E tornou Davi a ajuntar todos os escolhidos de
Israel, em número de trinta mil. E levantou-se Davi, e partiu,
com todo o povo que tinha consigo, para Baalim de Judá, para levarem
dali para cima a arca de Deus, sobre a qual se invoca o nome, o nome
do Senhor dos Exércitos, que se assenta entre os querubins. E
puseram a arca de Deus em um carro novo, e a levaram da casa de
Abinadabe, que está em Gibeá; e Uzá e Aiô, filhos de Abinadabe,
guiavam o carro novo. E levando-o da casa de Abinadabe, que está
em Gibeá, com a arca de Deus, Aiô ia adiante da arca. E Davi, e
toda a casa de Israel, festejavam perante o Senhor, com toda a sorte
de instrumentos de pau de faia\footnote{Planta (árvore) européia da
família das Fagáceas; madeira desta árvore.}, como também com
harpas, e com saltérios, e com tamboris\footnote{Instrumento de
percussão, espécie de cítara com seis cordas percutíveis com uma
baqueta.}, e com pandeiros, e com címbalos.

E, chegando à eira de Nacom, estendeu Uzá a mão à arca de Deus, e
pegou nela; porque os bois a deixavam pender. Então a ira do
Senhor se acendeu contra Uzá, e Deus o feriu ali por esta
imprudência; e morreu ali junto à arca de Deus. E Davi se
contristou, porque o Senhor abrira rotura em Uzá; e chamou àquele
lugar Perez-Uzá, até ao dia de hoje. E temeu Davi ao Senhor
naquele dia; e disse: Como virá a mim a arca do Senhor? E não
quis Davi retirar para junto de si a arca do Senhor, à cidade de
Davi; mas Davi a fez levar à casa de Obede-Edom, o giteu. E
ficou a arca do Senhor em casa de Obede-Edom, o giteu, três meses; e
abençoou o Senhor a Obede-Edom, e a toda a sua casa.

Então avisaram a Davi, dizendo: Abençoou o Senhor a casa de
Obede-Edom, e tudo quanto tem, por causa da arca de Deus; foi pois
Davi, e trouxe a arca de Deus para cima, da casa de Obede-Edom, à
cidade de Davi, com alegria. E sucedeu que, quando os que
levavam a arca do Senhor tinham dado seis passos, sacrificava bois e
carneiros cevados. E Davi saltava com todas as suas forças
diante do Senhor; e estava Davi cingido de um éfode de linho.
Assim subindo, levavam Davi e todo o Israel a arca do Senhor,
com júbilo, e ao som das trombetas. E sucedeu que, entrando a
arca do Senhor na cidade de Davi, Mical, a filha de Saul, estava
olhando pela janela; e, vendo ao rei Davi, que ia bailando e
saltando diante do Senhor, o desprezou no seu coração. E
introduzindo a arca do Senhor, a puseram no seu lugar, na tenda que
Davi lhe armara; e ofereceu Davi holocaustos e ofertas pacíficas
perante o Senhor. E acabando Davi de oferecer os holocaustos
e ofertas pacíficas, abençoou o povo em nome do Senhor dos
Exércitos. E repartiu a todo o povo, e a toda a multidão de
Israel, desde os homens até às mulheres, a cada um, um bolo de pão,
e um bom pedaço de carne, e um frasco de vinho; então retirou-se
todo o povo, cada um para sua casa.

E, voltando Davi para abençoar a sua casa, Mical, a filha de
Saul, saiu a encontrar-se com Davi, e disse: Quão honrado foi o rei
de Israel, descobrindo-se hoje aos olhos das servas de seus servos,
como sem pejo\footnote{Pudor.} se descobre qualquer dos vadios.
Disse, porém, Davi a Mical: Perante o Senhor, que me escolheu
preferindo-me a teu pai, e a toda a sua casa, mandando-me que fosse
soberano sobre o povo do Senhor, sobre Israel, perante o Senhor
tenho me alegrado. E ainda mais do que isto me
envilecerei\footnote{Tornar-se vil, desprezível; aviltar-se,
rebaixar-se. Perder o valor, a valia.}, e me humilharei aos meus
olhos; mas das servas, de quem falaste, delas serei honrado.
E Mical, a filha de Saul, não teve filhos, até o dia da sua
morte.

\medskip

\lettrine{7} E sucedeu que, estando o rei Davi em sua casa, e
tendo o Senhor lhe dado descanso de todos os seus inimigos em redor,
disse o rei ao profeta Natã: Eis que eu moro em casa de cedro, e
a arca de Deus mora dentro de cortinas. E disse Natã ao rei:
Vai, e faze tudo quanto está no teu coração; porque o Senhor é
contigo.

Porém sucedeu naquela mesma noite, que a palavra do Senhor veio a
Natã, dizendo: Vai, e dize a meu servo Davi: Assim diz o Senhor:
Edificar-me-ás tu uma casa para minha habitação? Porque em casa
nenhuma habitei desde o dia em que fiz subir os filhos de Israel do
Egito até ao dia de hoje; mas andei em tenda e em tabernáculo. E
em todo o lugar em que andei com todos os filhos de Israel, falei
porventura alguma palavra a alguma das tribos de Israel, a quem
mandei apascentar o meu povo de Israel, dizendo: Por que não me
edificais uma casa de cedro? Agora, pois, assim dirás ao meu
servo Davi: Assim diz o Senhor dos Exércitos: Eu te tomei da
malhada, de detrás das ovelhas, para que fosses o soberano sobre o
meu povo, sobre Israel. E fui contigo, por onde quer que foste,
e destruí a teus inimigos diante de ti; e fiz grande o teu nome,
como o nome dos grandes que há na terra. E prepararei lugar
para o meu povo, para Israel, e o plantarei, para que habite no seu
lugar, e não mais seja removido, e nunca mais os filhos da
perversidade o aflijam, como dantes, e desde o dia em que
mandei que houvesse juízes sobre o meu povo Israel; a ti, porém, te
dei descanso de todos os teus inimigos; também o Senhor te faz saber
que te fará casa. Quando teus dias forem completos, e vieres
a dormir com teus pais, então farei levantar depois de ti um dentre
a tua descendência, o qual sairá das tuas entranhas, e estabelecerei
o seu reino. Este edificará uma casa ao meu nome, e
confirmarei o trono do seu reino para sempre. Eu lhe serei
por pai, e ele me será por filho; e, se vier a transgredir,
castigá-lo-ei com vara de homens, e com açoites de filhos de homens.
Mas a minha benignidade não se apartará dele; como a tirei de
Saul, a quem tirei de diante de ti. Porém a tua casa e o teu
reino serão firmados para sempre diante de ti; teu trono será firme
para sempre. Conforme a todas estas palavras, e conforme a
toda esta visão, assim falou Natã a Davi.

Então entrou o rei Davi, e ficou perante o Senhor, e disse: Quem
sou eu, Senhor Deus, e qual é a minha casa, para que me tenhas
trazido até aqui? E ainda foi isto pouco aos teus olhos,
Senhor Deus, senão que também falaste da casa de teu servo para
tempos distantes; é este o procedimento dos homens, ó Senhor Deus?
E que mais te pode dizer ainda Davi? Pois tu conheces bem a
teu servo, ó Senhor Deus. Por causa da tua palavra, e segundo
o teu coração, fizeste toda esta grandeza; fazendo-a saber a teu
servo. Portanto, grandioso és, ó Senhor Deus, porque não há
semelhante a ti, e não há outro Deus senão tu só, segundo tudo o que
temos ouvido com os nossos ouvidos. E quem há como o teu
povo, como Israel, gente única na terra, a quem Deus foi resgatar
para seu povo, para fazer-te nome, e para fazer-vos estas grandes e
terríveis coisas à tua terra, diante do teu povo, que tu resgataste
do Egito, desterrando as nações e a seus deuses? E
confirmaste a teu povo Israel por teu povo para sempre, e tu,
Senhor, te fizeste o seu Deus. Agora, pois, ó Senhor Deus,
esta palavra que falaste acerca de teu servo e acerca da sua casa,
confirma-a para sempre, e faze como tens falado. E
engrandeça-se o teu nome para sempre, para que se diga: O Senhor dos
Exércitos é Deus sobre Israel; e a casa de teu servo será confirmada
diante de ti. Pois tu, Senhor dos Exércitos, Deus de Israel,
revelaste aos ouvidos de teu servo, dizendo: Edificar-te-ei uma
casa. Portanto o teu servo se animou para fazer-te esta oração.
Agora, pois, Senhor Deus, tu és o mesmo Deus, e as tuas
palavras são verdade, e tens falado a teu servo este bem. Sê,
pois, agora servido de abençoar a casa de teu servo, para permanecer
para sempre diante de ti, pois tu, ó Senhor Deus, o disseste; e com
a tua bênção será para sempre bendita a casa de teu servo.

\medskip

\lettrine{8} E sucedeu depois disto que Davi feriu os
filisteus, e os sujeitou; e Davi tomou a Metegue-Ama das mãos dos
filisteus. Também derrotou os moabitas, e os mediu com cordel,
fazendo-os deitar por terra; e os mediu com dois cordéis para os
matar, e com um cordel inteiro para os deixar com vida. Ficaram
assim os moabitas por servos de Davi, pagando-lhe tributos.
Feriu também Davi a Hadadezer, filho de Reobe, rei de Zobá,
quando ele ia recuperar o seu domínio sobre o rio Eufrates. E
tomou-lhe Davi mil carros e setecentos cavaleiros e vinte mil homens
de pé; e Davi jarretou a todos os cavalos dos carros, e reservou
deles cem carros. E vieram os sírios de Damasco a socorrer a
Hadadezer, rei de Zobá; porém Davi feriu dos sírios vinte e dois mil
homens. E Davi pôs guarnições na Síria de Damasco, e os sírios
ficaram por servos de Davi, pagando-lhe tributos; e o Senhor guardou
a Davi por onde quer que ia. E Davi tomou os escudos de ouro que
havia com os servos de Hadadezer, e os trouxe a Jerusalém. Tomou
mais o rei Davi uma quantidade muito grande de bronze de Betá e de
Berotai, cidades de Hadadezer.

Então ouvindo Toí, rei de Hamate, que Davi ferira a todo o
exército de Hadadezer, mandou Toí, seu filho Jorão, ao rei
Davi, para lhe perguntar como estava, e para lhe dar os parabéns por
haver pelejado contra Hadadezer, e por o haver ferido (porque
Hadadezer de contínuo fazia guerra a Toí); e na sua mão trazia vasos
de prata, e vasos de ouro, e vasos de bronze, os quais também
o rei Davi consagrou ao Senhor, juntamente com a prata e ouro que já
havia consagrado de todas as nações que sujeitara. Da Síria,
e de Moabe, e dos filhos de Amom, e dos filisteus, e de Amaleque, e
dos despojos de Hadadezer, filho de Reobe, rei de Zobá.
Também Davi ganhou nome, voltando ele de ferir os sírios no
vale do Sal, a saber, a dezoito mil. E pôs guarnições, em
Edom, em todo o Edom pôs guarnições, e todos os edomeus ficaram por
servos de Davi; e o Senhor ajudava a Davi por onde quer que ia.

Reinou, pois, Davi sobre todo o Israel; e Davi fazia direito e
justiça a todo o seu povo. E Joabe, filho de Zeruia, era
sobre o exército; e Jeosafá, filho de Ailude, era cronista. E
Zadoque, filho de Aitube, e Aimeleque, filho de Abiatar, eram
sacerdotes, e Seraías escrivão. Também Benaia, filho de
Jeoiada, estava sobre os quereteus e peleteus; porém os filhos de
Davi eram ministros.

\medskip

\lettrine{9} E disse Davi: Há ainda alguém que tenha ficado da
casa de Saul, para que lhe faça benevolência por amor de Jônatas?
E havia um servo na casa de Saul cujo nome era Ziba; e o
chamaram à presença de Davi. Disse-lhe o rei: És tu Ziba? E ele
disse: Servo teu. E disse o rei: Não há ainda alguém da casa de
Saul para que eu use com ele da benevolência de Deus? Então disse
Ziba ao rei: Ainda há um filho de Jônatas, aleijado de ambos os pés.
E disse-lhe o rei: Onde está? E disse Ziba ao rei: Eis que está
em casa de Maquir, filho de Amiel, em Lo-Debar. Então mandou o
rei Davi, e o tomou da casa de Maquir, filho de Amiel, de Lo-Debar.
E Mefibosete, filho de Jônatas, o filho de Saul, veio a Davi, e
se prostrou com o rosto por terra e inclinou-se; e disse Davi:
Mefibosete! E ele disse: Eis aqui teu servo. E disse-lhe Davi:
Não temas, porque decerto usarei contigo de benevolência por amor de
Jônatas, teu pai, e te restituirei todas as terras de Saul, teu pai,
e tu sempre comerás pão à minha mesa. Então se inclinou, e
disse: Quem é teu servo, para teres olhado para um cão morto tal
como eu?

Então chamou Davi a Ziba, moço de Saul, e disse-lhe: Tudo o que
pertencia a Saul, e a toda a sua casa, tenho dado ao filho de teu
senhor. Trabalhar-lhe-ás, pois, a terra, tu e teus filhos, e
teus servos, e recolherás os frutos, para que o filho de teu senhor
tenha pão para comer; mas Mefibosete, filho de teu senhor, sempre
comerá pão à minha mesa. E tinha Ziba quinze filhos e vinte servos.
E disse Ziba ao rei: Conforme a tudo quanto meu senhor, o
rei, manda a seu servo, assim fará teu servo. Quanto a Mefibosete,
disse o rei, comerá à minha mesa como um dos filhos do rei. E
tinha Mefibosete um filho pequeno, cujo nome era Mica; e todos
quantos moravam em casa de Ziba eram servos de Mefibosete.
Morava, pois, Mefibosete em Jerusalém, porquanto sempre comia
à mesa do rei, e era coxo de ambos os pés.

\medskip

\lettrine{10} E aconteceu depois disto que morreu o rei dos
filhos de Amom, e seu filho Hanum reinou em seu lugar. Então
disse Davi: Usarei de benevolência com Hanum, filho de Naás, como
seu pai usou de benevolência comigo. E enviou Davi os seus servos
para consolá-lo acerca de seu pai; e foram os servos de Davi à terra
dos filhos de Amom. Então disseram os príncipes dos filhos de
Amom a seu Senhor, Hanum: Porventura honra Davi a teu pai aos teus
olhos, porque te enviou consoladores? Não te enviou antes Davi os
seus servos para reconhecerem esta cidade, e para espiá-la, e para
transtorná-la? Então tomou Hanum os servos de Davi, e lhes
raspou metade da barba, e lhes cortou metade das vestes, até às
nádegas, e os despediu. Quando isso foi informado a Davi, enviou
ele mensageiros a encontrá-los, porque estavam aqueles homens
sobremaneira envergonhados. Mandou o rei dizer-lhes: Deixai-vos
estar em Jericó, até que vos torne a crescer a barba, e então
voltai.

Vendo, pois, os filhos de Amom que se tinham feito abomináveis
para com Davi, enviaram os filhos de Amom, e alugaram dos sírios de
Bete-Reobe e dos sírios de Zobá vinte mil homens de pé, e do rei de
Maaca mil homens e dos homens de Tobe doze mil homens. E ouvindo
Davi, enviou a Joabe e a todo o exército dos valentes. E saíram
os filhos de Amom, e ordenaram a batalha à entrada da porta; mas os
sírios de Zobá e Reobe, e os homens de Tobe e Maaca estavam à parte
no campo. Vendo, pois, Joabe que a batalha estava preparada
contra ele pela frente e pela retaguarda, escolheu dentre todos os
homens de Israel, e formou-os em linha contra os sírios. E o
restante do povo entregou na mão de Abisai seu irmão, o qual formou
em linha contra os filhos de Amom. E disse: Se os sírios
forem mais fortes do que eu, tu me virás em socorro; e, se os filhos
de Amom forem mais fortes do que tu, irei a socorrer-te.
Esforça-te, pois, e esforcemo-nos pelo nosso povo, e pelas
cidades de nosso Deus; e faça o Senhor o que bem parecer aos seus
olhos. Então se achegou Joabe, e o povo que estava com ele, à
peleja contra os sírios; e fugiram de diante dele. E, vendo
os filhos de Amom que os sírios fugiam, também eles fugiram de
diante de Abisai, e entraram na cidade; e voltou Joabe dos filhos de
Amom, e veio para Jerusalém.

Vendo, pois, os sírios que foram feridos diante de Israel,
tornaram a refazer-se. E mandou Hadadezer, e fez sair os
sírios que estavam do outro lado do rio, e vieram a Helã; e Sobaque,
capitão do exército de Hadadezer, marchava diante deles. Do
que informado Davi, ajuntou a todo o Israel, e passou o Jordão, e
foi a Helã; e os sírios se puseram em ordem contra Davi, e pelejaram
com ele. Porém os sírios fugiram de diante de Israel, e Davi
feriu dentre os sírios aos homens de setecentos carros, e quarenta
mil homens de cavalaria; feriu também a Sobaque, capitão do
exército, que morreu ali. Vendo, pois, todos os reis, servos
de Hadadezer, que foram feridos diante de Israel, fizeram paz com
Israel, e o serviram; e temeram os sírios de socorrer aos filhos de
Amom.

\medskip

\lettrine{11} E aconteceu que, tendo decorrido um ano, no
tempo em que os reis saem à guerra, enviou Davi a Joabe, e com ele
os seus servos, e a todo o Israel; e eles destruíram os filhos de
Amom, e cercaram a Rabá; porém Davi ficou em Jerusalém. E
aconteceu que numa tarde Davi se levantou do seu leito, e andava
passeando no terraço da casa real, e viu do terraço a uma mulher que
se estava lavando; e era esta mulher mui formosa à vista. E
mandou Davi indagar quem era aquela mulher; e disseram: Porventura
não é esta Bate-Seba, filha de Eliã, mulher de Urias, o heteu?
Então enviou Davi mensageiros, e mandou trazê-la; e ela veio, e
ele se deitou com ela (pois já estava purificada da sua imundícia);
então voltou ela para sua casa. E a mulher concebeu; e mandou
dizer a Davi: Estou grávida.

Então Davi mandou dizer a Joabe: Envia-me Urias o heteu. E Joabe
enviou Urias a Davi. Vindo, pois, Urias a ele, perguntou Davi
como passava Joabe, e como estava o povo, e como ia a guerra.
Depois disse Davi a Urias: Desce à tua casa, e lava os teus pés.
E, saindo Urias da casa real, logo lhe foi mandado um presente da
mesa do rei. Porém Urias se deitou à porta da casa real, com
todos os servos do seu senhor; e não desceu à sua casa. E
fizeram saber isto a Davi, dizendo: Urias não desceu a sua casa.
Então disse Davi a Urias: Não vens tu duma jornada? Por que não
desceste à tua casa? E disse Urias a Davi: A arca, e Israel,
e Judá ficaram em tendas; e Joabe, meu senhor, e os servos de meu
Senhor estão acampados no campo; e hei de eu entrar na minha casa,
para comer e beber, e para me deitar com minha mulher? Pela tua
vida, e pela vida da tua alma, não farei tal coisa. Então
disse Davi a Urias: Demora-te aqui ainda hoje, e amanhã te
despedirei. Urias, pois, ficou em Jerusalém aquele dia e o seguinte.
E Davi o convidou, e comeu e bebeu diante dele, e o
embebedou; e à tarde saiu a deitar-se na sua cama com os servos de
seu senhor; porém não desceu à sua casa.

E sucedeu que pela manhã Davi escreveu uma carta a Joabe; e
mandou-lha por mão de Urias. Escreveu na carta, dizendo:
Ponde a Urias na frente da maior força da peleja; e retirai-vos de
detrás dele, para que seja ferido e morra. Aconteceu, pois,
que, tendo Joabe observado bem a cidade, pôs a Urias no lugar onde
sabia que havia homens valentes. E, saindo os homens da
cidade, e pelejando com Joabe, caíram alguns do povo, dos servos de
Davi; e morreu também Urias, o heteu. Então enviou Joabe, e
fez saber a Davi todo o sucesso daquela peleja. E deu ordem
ao mensageiro, dizendo: Acabando tu de contar ao rei todo o sucesso
desta peleja, e sucedendo que o rei se encolerize, e te diga:
Por que vos chegastes tão perto da cidade a pelejar? Não sabíeis vós
que haviam de atirar do muro? Quem feriu a Abimeleque, filho
de Jerubesete? Não lançou uma mulher sobre ele do muro um pedaço de
uma mó corredora, de que morreu em Tebes? Por que vos chegastes ao
muro? Então dirás: Também morreu teu servo Urias, o heteu. E
foi o mensageiro, e entrou, e fez saber a Davi tudo o que Joabe o
enviara a dizer. E disse o mensageiro a Davi: Na verdade que
mais poderosos foram aqueles homens do que nós, e saíram a nós ao
campo; porém nós fomos contra eles, até à entrada da porta.
Então os flecheiros atiraram contra os teus servos desde o
alto do muro, e morreram alguns dos servos do rei; e também morreu o
teu servo Urias, o heteu. E disse Davi ao mensageiro: Assim
dirás a Joabe: Não te pareça isto mal aos teus olhos; pois a espada
tanto consome este como aquele; esforça a tua peleja contra a
cidade, e a derrota; esforça-o tu assim. Ouvindo, pois, a
mulher de Urias que seu marido era morto, lamentou a seu senhor.
E, passado o luto, enviou Davi, e a recolheu em sua casa, e
lhe foi por mulher, e deu-lhe à luz um filho. Porém esta coisa que
Davi fez pareceu mal aos olhos do Senhor.

\medskip

\lettrine{12} E O Senhor enviou Natã a Davi; e,
apresentando-se ele a Davi, disse-lhe: Havia numa cidade dois
homens, um rico e outro pobre. O rico possuía muitíssimas
ovelhas e vacas. Mas o pobre não tinha coisa nenhuma, senão uma
pequena cordeira que comprara e criara; e ela tinha crescido com ele
e com seus filhos; do seu bocado comia, e do seu copo bebia, e
dormia em seu regaço, e a tinha como filha. E, vindo um viajante
ao homem rico, deixou este de tomar das suas ovelhas e das suas
vacas para assar para o viajante que viera a ele; e tomou a cordeira
do homem pobre, e a preparou para o homem que viera a ele. Então
o furor de Davi se acendeu em grande maneira contra aquele homem, e
disse a Natã: Vive o Senhor, que digno de morte é o homem que fez
isso. E pela cordeira tornará a dar o quadruplicado, porque fez
tal coisa, e porque não se compadeceu. Então disse Natã a Davi:
Tu és este homem. Assim diz o Senhor Deus de Israel: Eu te ungi rei
sobre Israel, e eu te livrei das mãos de Saul; e te dei a casa
de teu senhor, e as mulheres de teu senhor em teu seio, e também te
dei a casa de Israel e de Judá, e, se isto é pouco, mais te
acrescentaria tais e tais coisas. Porque, pois, desprezaste a
palavra do Senhor, fazendo o mal diante de seus olhos? A Urias, o
heteu, feriste à espada, e a sua mulher tomaste por tua mulher; e a
ele mataste com a espada dos filhos de Amom. Agora, pois, não
se apartará a espada jamais da tua casa, porquanto me desprezaste, e
tomaste a mulher de Urias, o heteu, para ser tua mulher.
Assim diz o Senhor: Eis que suscitarei da tua própria casa o
mal sobre ti, e tomarei tuas mulheres perante os teus olhos, e as
darei a teu próximo, o qual se deitará com tuas mulheres perante
este sol. Porque tu o fizeste em oculto, mas eu farei este
negócio perante todo o Israel e perante o sol. Então disse
Davi a Natã: Pequei contra o Senhor. E disse Natã a Davi: Também o
Senhor perdoou o teu pecado; não morrerás. Todavia, porquanto
com este feito deste lugar sobremaneira a que os inimigos do Senhor
blasfemem, também o filho que te nasceu certamente morrerá.

Então Natã foi para sua casa; e o Senhor feriu a criança que a
mulher de Urias dera a Davi, e adoeceu gravemente. E buscou
Davi a Deus pela criança; e jejuou Davi, e entrou, e passou a noite
prostrado sobre a terra. Então os anciãos da sua casa se
levantaram e foram a ele, para o levantar da terra; porém ele não
quis, e não comeu pão com eles. E sucedeu que ao sétimo dia
morreu a criança; e temiam os servos de Davi dizer-lhe que a criança
estava morta, porque diziam: Eis que, sendo a criança ainda viva,
lhe falávamos, porém não dava ouvidos à nossa voz; como, pois, lhe
diremos que a criança está morta? Porque mais lhe afligiria.
Viu, porém, Davi que seus servos falavam baixo, e entendeu
Davi que a criança estava morta, pelo que disse Davi a seus servos:
Está morta a criança? E eles disseram: Está morta. Então Davi
se levantou da terra, e se lavou, e se ungiu, e mudou de roupas, e
entrou na casa do Senhor, e adorou. Então foi à sua casa, e pediu
pão; e lhe puseram pão, e comeu. E disseram-lhe seus servos:
Que é isto que fizeste? Pela criança viva jejuaste e choraste; porém
depois que morreu a criança te levantaste e comeste pão. E
disse ele: Vivendo ainda a criança, jejuei e chorei, porque dizia:
Quem sabe se Deus se compadecerá de mim, e viverá a criança?
Porém, agora que está morta, porque jejuaria eu? Poderei eu
fazê-la voltar? Eu irei a ela, porém ela não voltará para mim.
Então consolou Davi a Bate-Seba, sua mulher, e entrou a ela,
e se deitou com ela, e ela deu à luz um filho, e deu-lhe o nome de
Salomão; e o Senhor o amou. E enviou pela mão do profeta
Natã, dando-lhe o nome de Jedidias, por amor ao Senhor.

Ora pelejou Joabe contra Rabá, dos filhos de Amom, e tomou a
cidade real. Então mandou Joabe mensageiros a Davi, e disse:
Pelejei contra Rabá, e também tomei a cidade das águas.
Ajunta, pois, agora o restante do povo, e cerca a cidade, e
toma-a, para que tomando eu a cidade, não se aclame sobre ela o meu
nome. Então ajuntou Davi a todo o povo, e marchou para Rabá,
e pelejou contra ela, e a tomou. E tirou a coroa da cabeça do
seu rei, cujo peso era de um talento de ouro, e havia nela pedras
preciosas, e foi posta sobre a cabeça de Davi; e da cidade levou mui
grande despojo. E, trazendo o povo que havia nela, o pôs às
serras, e às talhadeiras de ferro, e aos machados de ferro, e os fez
passar por forno de tijolos; e assim fez a todas as cidades dos
filhos de Amom; e voltou Davi e todo o povo para Jerusalém.

\medskip

\lettrine{13} E aconteceu depois disto que, tendo Absalão,
filho de Davi, uma irmã formosa, cujo nome era Tamar, Amnom, filho
de Davi, amou-a. E angustiou-se Amnom, até adoecer, por Tamar,
sua irmã, porque era virgem; e parecia aos olhos de Amnom
dificultoso fazer-lhe coisa alguma. Tinha, porém, Amnom um
amigo, cujo nome era Jonadabe, filho de Siméia, irmão de Davi; e era
Jonadabe homem mui sagaz. O qual lhe disse: Por que tu de dia em
dia tanto emagreces, sendo filho do rei? Não mo farás saber a mim?
Então lhe disse Amnom: Amo a Tamar, irmã de Absalão, meu irmão.
E Jonadabe lhe disse: Deita-te na tua cama, e finge-te doente;
e, quando teu pai te vier visitar, dize-lhe: Peço-te que minha irmã
Tamar venha, e me dê de comer pão, e prepare a comida diante dos
meus olhos, para que eu a veja e coma da sua mão. Deitou-se,
pois, Amnom, e fingiu-se doente; e, vindo o rei visitá-lo, disse
Amnom, ao rei: Peço-te que minha irmã Tamar venha, e prepare dois
bolos diante dos meus olhos, para que eu coma de sua mão. Mandou
então Davi à casa, a Tamar, dizendo: Vai à casa de Amnom, teu irmão,
e faze-lhe alguma comida. E foi Tamar à casa de Amnom, seu irmão
(ele porém estava deitado), e tomou massa, e a amassou, e fez bolos
diante dos seus olhos, e cozeu os bolos. E tomou a frigideira, e
os tirou diante dele; porém ele recusou comer. E disse Amnom: Fazei
retirar a todos da minha presença. E todos se retiraram dele.
Então disse Amnom a Tamar: Traze a comida ao quarto, e
comerei da tua mão. E tomou Tamar os bolos que fizera, e levou-os a
Amnom, seu irmão, no quarto. E chegando-lhos, para que
comesse, pegou dela, e disse-lhe: Vem, deita-te comigo, minha irmã.
Porém ela lhe disse: Não, meu irmão, não me forces, porque
não se faz assim em Israel; não faças tal loucura. Porque,
aonde iria eu com a minha vergonha? E tu serias como um dos loucos
de Israel. Agora, pois, peço-te que fales ao rei, porque não me
negará a ti. Porém ele não quis dar ouvidos à sua voz; antes,
sendo mais forte do que ela, a forçou, e se deitou com ela.
Depois Amnom sentiu grande aversão por ela, pois maior era o
ódio que sentiu por ela do que o amor com que a amara. E disse-lhe
Amnom: Levanta-te, e vai-te. Então ela lhe disse: Não há
razão de me despedires assim; maior seria este mal do que o outro
que já me tens feito. Porém não lhe quis dar ouvidos. E
chamou a seu moço que o servia, e disse: Ponha fora a esta, e fecha
a porta após ela. E trazia ela uma roupa de muitas cores
(porque assim se vestiam as filhas virgens dos reis); e seu servo a
pôs para fora, e fechou a porta após ela. Então Tamar tomou
cinza sobre a sua cabeça, e a roupa de muitas cores que trazia
rasgou; e pôs as mãos sobre a cabeça, e foi andando e clamando.
E Absalão, seu irmão, lhe disse: Esteve Amnom, teu irmão,
contigo? Ora, pois, minha irmã, cala-te; é teu irmão. Não se
angustie o teu coração por isto. Assim ficou Tamar, e esteve
solitária em casa de Absalão seu irmão.

E, ouvindo o rei Davi todas estas coisas, muito se lhe acendeu a
ira. Porém Absalão não falou com Amnom, nem mal nem bem;
porque Absalão odiava a Amnom, por ter forçado a Tamar sua irmã.
E aconteceu que, passados dois anos inteiros, Absalão tinha
tosquiadores em Baal-Hazor, que está junto a Efraim; e convidou
Absalão a todos os filhos do rei. E foi Absalão ao rei, e
disse: Eis que teu servo tem tosquiadores; peço que o rei e os seus
servos venham com o teu servo. O rei, porém, disse a Absalão:
Não, filho meu, não vamos todos juntos, para não te sermos pesados.
E instou com ele; porém não quis ir, mas o abençoou. Então
disse Absalão: Quando não, deixa ir conosco Amnom, meu irmão. Porém
o rei disse: Para que iria contigo? E, instando Absalão com
ele, deixou ir com ele a Amnom, e a todos os filhos do rei. E
Absalão deu ordem aos seus servos, dizendo: Tomai sentido; quando o
coração de Amnom estiver alegre do vinho, e eu vos disser: Feri a
Amnom, então o matareis; não temais: porque porventura não sou eu
quem vo-lo ordenei? Esforçai-vos, e sede valentes. E os
servos de Absalão fizeram a Amnom como Absalão lho havia ordenado.
Então todos os filhos do rei se levantaram, e montaram cada um no
seu mulo, e fugiram.

E aconteceu que, estando eles ainda no caminho, chegou a nova a
Davi, dizendo-se: Absalão feriu a todos os filhos do rei, e nenhum
deles ficou. Então o rei se levantou, e rasgou as suas
vestes, e se lançou por terra; da mesma maneira todos os seus servos
estavam com vestes rotas. Mas Jonadabe, filho de Siméia,
irmão de Davi, respondeu, e disse: Não diga o meu senhor que mataram
a todos os moços filhos do rei, porque só morreu Amnom; porque assim
tinha resolvido fazer Absalão, desde o dia em que forçou a Tamar sua
irmã. Não se lhe ponha, pois, agora no coração do rei meu
senhor tal coisa, dizendo: Morreram todos os filhos do rei; porque
só morreu Amnom. E Absalão fugiu; e o moço que estava de
guarda, levantou os seus olhos, e olhou; e eis que muito povo vinha
pelo caminho por detrás dele, pelo lado do monte. Então disse
Jonadabe ao rei: Eis aqui vêm os filhos do rei; conforme à palavra
de teu servo, assim sucedeu. E aconteceu que, como acabou de
falar, os filhos do rei vieram, e levantaram a sua voz, e choraram;
e também o rei e todos os seus servos choraram amargamente.
Assim Absalão fugiu, e foi a Talmai, filho de Amiur, rei de
Gesur. E Davi pranteava por seu filho todos aqueles dias.
Assim Absalão fugiu, e foi para Gesur; esteve ali três anos.
Então tinha o rei Davi saudades de Absalão; porque já se
tinha consolado acerca da morte de Amnom.

\medskip

\lettrine{14} Conhecendo, pois, Joabe, filho de Zeruia, que o
coração do rei estava inclinado para Absalão, enviou Joabe a
Tecoa, e tomou de lá uma mulher e disse-lhe: Ora, finge que estás de
luto; veste roupas de luto, e não te unjas com óleo, e sê como uma
mulher que há já muitos dias está de luto por algum morto. E vai
ao rei, e fala-lhe conforme a esta palavra. E Joabe lhe pôs as
palavras na boca. E a mulher tecoíta falou ao rei, e,
deitando-se com o rosto em terra, se prostrou e disse: Salva-me, ó
rei. E disse-lhe o rei: Que tens? E disse ela: Na verdade sou
mulher viúva; morreu meu marido. Tinha, pois, a tua serva dois
filhos, e estes brigaram entre si no campo, e não houve quem os
apartasse; assim um feriu ao outro, e o matou. E eis que toda a
linhagem se levantou contra a tua serva, e disseram: Dá-nos aquele
que feriu a seu irmão, para que o matemos, por causa da vida de seu
irmão, a quem matou, e para que destruamos também ao herdeiro. Assim
apagarão a brasa que me ficou, de sorte que não deixam a meu marido
nome, nem remanescente sobre a terra. E disse o rei à mulher:
Vai para tua casa; e eu mandarei ordem acerca de ti. E disse a
mulher tecoíta ao rei: A injustiça, rei meu Senhor, venha sobre mim
e sobre a casa de meu pai; e o rei e o seu trono fique inculpável.
E disse o rei: Quem falar contra ti, traze-mo a mim; e nunca
mais te tocará. E disse ela: Ora, lembre-se o rei do Senhor
seu Deus, para que os vingadores do sangue não prossigam na
destruição, e não exterminem a meu filho. Então disse ele: Vive o
Senhor, que não há de cair no chão nem um dos cabelos de teu filho.
Então disse a mulher: Peço-te que a tua serva fale uma
palavra ao rei meu senhor. E disse ele: Fala. E disse a
mulher: Por que, pois, pensaste tu uma tal coisa contra o povo de
Deus? Porque, falando o rei tal palavra, fica como culpado; visto
que o rei não torna a trazer o seu desterrado. Porque
certamente morreremos, e seremos como águas derramadas na terra que
não se ajuntam mais; Deus, pois, lhe não tirará a vida, mas cogita
meios, para que não fique banido dele o seu desterrado. E se
eu agora vim falar esta palavra ao rei, meu senhor, é porque o povo
me atemorizou; dizia, pois, a tua serva: Falarei, pois, ao rei;
porventura fará o rei segundo a palavra da sua serva. Porque
o rei ouvirá, para livrar a sua serva da mão do homem que intenta
destruir juntamente a mim e a meu filho da herança de Deus.
Dizia mais a tua serva: Seja agora a palavra do rei meu
senhor para descanso; porque como um anjo de Deus, assim é o rei,
meu Senhor, para ouvir o bem e o mal; e o Senhor teu Deus será
contigo. Então respondeu o rei, e disse à mulher: Peço-te que
não me encubras o que eu te perguntar. E disse a mulher: Ora fale o
rei, meu senhor. E disse o rei: Não é verdade que a mão de
Joabe anda contigo em tudo isto? E respondeu a mulher, e disse: Vive
a tua alma, ó rei meu senhor, que ninguém se poderá desviar, nem
para a direita nem para a esquerda, de tudo quanto o rei, meu
senhor, tem falado: Porque Joabe, teu servo, é quem me deu ordem, e
foi ele que pôs na boca da tua serva todas estas palavras:
para mudar o aspecto deste caso foi que o teu servo Joabe fez
isto; porém sábio é meu senhor, conforme à sabedoria de um anjo de
Deus, para entender tudo o que há na terra.

Então o rei disse a Joabe: Eis que fiz isto; vai, pois, e torna a
trazer o jovem Absalão. Então Joabe se prostrou sobre o seu
rosto em terra, e se inclinou, e agradeceu ao rei; e disse Joabe:
Hoje conhece o teu servo que achei graça aos teus olhos, ó rei meu
senhor, porque o rei fez segundo a palavra do teu servo.
Levantou-se, pois, Joabe, e foi a Gesur, e trouxe Absalão a
Jerusalém. E disse o rei: Torne para a sua casa, e não veja a
minha face. Tornou, pois, Absalão para sua casa, e não viu a face do
rei. Não havia, porém, em todo o Israel homem tão belo e tão
aprazível como Absalão; desde a planta do pé até à cabeça não havia
nele defeito algum. E, quando tosquiava a sua cabeça (e
sucedia que no fim de cada ano a tosquiava, porquanto muito lhe
pesava, e por isso a tosquiava), pesava o cabelo da sua cabeça
duzentos siclos, segundo o peso real. Também nasceram a
Absalão três filhos e uma filha, cujo nome era Tamar; e esta era
mulher formosa à vista.

Assim ficou Absalão dois anos inteiros em Jerusalém, e não viu a
face do rei. Mandou, pois, Absalão chamar a Joabe, para o
enviar ao rei; porém não quis vir a ele; e enviou ainda segunda vez e, contudo, não quis vir. Então disse aos seus servos: Vedes
ali o pedaço de campo de Joabe pegado ao meu, e tem cevada nele;
ide, e ponde-lhe fogo. E os servos de Absalão puseram fogo ao pedaço de campo. Então Joabe se levantou, e veio a Absalão, em casa,
e disse-lhe: Por que puseram os teus servos fogo ao pedaço de campo que é meu? E disse Absalão a Joabe: Eis que enviei a ti,
dizendo: Vem cá, para que te envie ao rei, a dizer-lhe: Para que vim de Gesur? Melhor me fora estar ainda lá. Agora, pois, veja eu a face do rei; e, se há ainda em mim alguma culpa, que me mate.
Então foi Joabe ao rei, e assim lho disse. Então chamou a
Absalão, e ele se apresentou ao rei, e se inclinou sobre o seu rosto
em terra diante do rei; e o rei beijou a Absalão.

\medskip

\lettrine{15}{} E aconteceu depois disto que Absalão fez
aparelhar carros e cavalos, e cin\-quen\-ta homens que corressem adiante
dele. Também Absalão se levantou pela manhã, e parava a um lado
do caminho da porta. E sucedia que a todo o homem que tinha alguma
demanda para vir ao rei a juízo, o chamava Absalão a si, e lhe
dizia: De que cidade és tu? E, dizendo ele: De uma das tribos de
Israel é teu servo; então Absalão lhe dizia: Olha, os teus
negócios são bons e retos, porém não tens quem te ouça da parte do
rei. Dizia mais Absalão: Ah, quem me dera ser juiz na terra,
para que viesse a mim todo o homem que tivesse demanda ou questão,
para que lhe fizesse justiça! Sucedia também que, quando alguém
se chegava a ele para se inclinar diante dele, ele estendia a sua
mão, e pegava dele, e o beijava. E desta maneira fazia Absalão a
todo o Israel que vinha ao rei para juízo; assim furtava Absalão o
coração dos homens de Israel.

Aconteceu, pois, ao cabo de quarenta anos, que Absalão disse ao
rei: Deixa-me ir pagar em Hebrom o meu voto que fiz ao Senhor.
Porque, morando eu em Gesur, na Síria, fez o teu servo um voto,
dizendo: Se o Senhor outra vez me fizer tornar a Jerusalém, servirei
ao Senhor. Então lhe disse o rei: Vai em paz. Levantou-se, pois,
e foi para Hebrom. E enviou Absalão espias por todas as
tribos de Israel, dizendo: Quando ouvirdes o som das trombetas,
direis: Absalão reina em Hebrom. E de Jerusalém foram com
Absalão duzentos homens convidados, porém iam na sua simplicidade,
porque nada sabiam daquele negócio. Também Absalão mandou vir
Aitofel, o gilonita, do conselho de Davi, à sua cidade de Giló,
estando ele oferecendo os seus sacrifícios; e a conjuração se
fortificava, e vinha o povo, e ia crescendo com Absalão.

Então veio um mensageiro a Davi, dizendo: O coração de cada um em
Israel segue a Absalão. Disse, pois, Davi a todos os seus
servos que estavam com ele em Jerusalém: Levantai-vos, e fujamos,
porque não poderíamos escapar diante de Absalão. Dai-vos pressa a
caminhar, para que porventura não se apresse ele, e nos alcance, e
lance sobre nós algum mal, e fira a cidade a fio de espada.
Então os servos do rei disseram ao rei: Eis aqui os teus
servos, para tudo quanto determinar o rei, nosso senhor. E
saiu o rei, com toda a sua casa, a pé; deixou, porém, o rei dez
mulheres concubinas, para guardarem a casa. Tendo, pois,
saído o rei com todo o povo a pé, pararam num lugar distante.
E todos os seus servos iam a seu lado, como também todos os
quereteus e todos os peleteus; e todos os giteus, seiscentos homens
que vieram de Gate a pé, caminhavam diante do rei. Disse,
pois, o rei a Itai, o giteu: Por que irias tu também conosco?
Volta-te, e fica-te com o rei, porque és estrangeiro, e também
desterrado de teu lugar. Ontem vieste, e te levaria eu hoje
conosco a caminhar? Pois eu vou para onde puder ir; volta, pois, e
torna a levar teus irmãos contigo, com beneficência e fidelidade.
Respondeu, porém, Itai ao rei, e disse: Vive o Senhor, e vive
o rei meu senhor, que no lugar em que estiver o rei meu senhor, seja
para morte seja para vida, aí certamente estará também o teu
servidor. Então Davi disse a Itai: Vem, pois, e passa
adiante. Assim passou Itai, o giteu, e todos os seus homens, e todas
as crianças que havia com ele. E toda a terra chorava a
grandes vozes, passando todo o povo; também o rei passou o ribeiro
de Cedrom, e passou todo o povo na direção do caminho do deserto.

Eis que também Zadoque ali estava, e com ele todos os levitas que
levavam a arca da aliança de Deus; e puseram ali a arca de Deus, e
subiu Abiatar, até que todo o povo acabou de passar da cidade.
Então disse o rei a Zadoque: Torna a levar a arca de Deus à
cidade; que, se achar graça nos olhos do Senhor, ele me tornará a
trazer para lá e me deixará ver a ela e a sua habitação. Se,
porém, disser assim: Não tenho prazer em ti; eis-me aqui, faça de
mim como parecer bem aos seus olhos. Disse mais o rei a
Zadoque, o sacerdote: Não és tu porventura vidente? Torna, pois, em
paz para a cidade, e convosco também vossos dois filhos, Aimás, teu
filho, e Jônatas, filho de Abiatar. Olhai que me demorarei
nas campinas do deserto até que tenha notícias vossas.
Zadoque, pois, e Abiatar, tornaram a levar para Jerusalém a
arca de Deus; e ficaram ali. E seguiu Davi pela encosta do
monte das Oliveiras, subindo e chorando, e com a cabeça coberta; e
caminhava com os pés descalços; e todo o povo que ia com ele cobria
cada um a sua cabeça, e subiam chorando sem cessar.

Então fizeram saber a Davi, dizendo: Também Aitofel está entre os
que se conjuraram com Absalão. Pelo que disse Davi: Ó Senhor,
peço-te que torne em loucura o conselho de Aitofel. E
aconteceu que, chegando Davi ao cume, para adorar ali a Deus, eis
que Husai, o arquita, veio encontrar-se com ele com a roupa rasgada
e terra sobre a cabeça. E disse-lhe Davi: Se passares comigo,
ser-me-ás pesado. Porém se voltares para a cidade, e disseres
a Absalão: Eu serei, ó rei, teu servo; bem fui antes servo de teu
pai, mas agora serei teu servo; dissipar-me-ás então o conselho de
Aitofel. E não estão ali contigo Zadoque e Abiatar,
sacerdotes? E será que todas as coisas que ouvires da casa do rei,
farás saber a Zadoque, e a Abiatar, sacerdotes. Eis que estão
também ali com eles seus dois filhos, Aimaás filho de Zadoque, e
Jônatas filho de Abiatar; pela mão deles aviso me mandareis, de
todas as coisas que ouvirdes. Husai, pois, amigo de Davi,
veio para a cidade; e Absalão entrou em Jerusalém.

\medskip

\lettrine{16} E passando Davi um pouco mais adiante do cume,
eis que Ziba, o servo de Mefibosete, veio encontrar-se com ele, com
um par de jumentos albardados, e sobre eles duzentos pães, com cem
cachos de passas, e cem de frutas de verão e um odre de vinho. E
disse o rei a Ziba: Que pretendes com isto? E disse Ziba: Os
jumentos são para a casa do rei, para se montarem neles; e o pão e
as frutas de verão para comerem os moços; e o vinho para beberem os
cansados no deserto. Então disse o rei: Ora, onde está o filho
de teu senhor? E disse Ziba ao rei: Eis que ficou em Jerusalém;
porque disse: Hoje me restituirá a casa de Israel o reino de meu
pai. Então disse o rei a Ziba: Eis que teu é tudo quanto tem
Mefibosete. E disse Ziba: Eu me inclino, que eu ache graça em teus
olhos, ó rei meu senhor.

E, chegando o rei Davi a Baurim, eis que dali saiu um homem da
linhagem da casa de Saul, cujo nome era Simei, filho de Gera, e,
saindo, ia amaldiçoando. E atirava pedras contra Davi, e contra
todos os servos do rei Davi; ainda que todo o povo e todos os
valentes iam à sua direita e à sua esquerda. E, amaldiçoando-o
Simei, assim dizia: Sai, sai, homem de sangue, e homem de Belial.
O Senhor te deu agora a paga de todo o sangue da casa de Saul,
em cujo lugar tens reinado; já deu o Senhor o reino na mão de
Absalão teu filho; e eis-te agora na tua desgraça, porque és um
homem de sangue. Então disse Abisai, filho de Zeruia, ao rei:
Por que amaldiçoaria este cão morto ao rei meu senhor? Deixa-me
passar, e lhe tirarei a cabeça. Disse, porém, o rei: Que
tenho eu convosco, filhos de Zeruia? Ora deixai-o amaldiçoar; pois o
Senhor lhe disse: Amaldiçoa a Davi; quem pois diria: Por que assim
fizeste? Disse mais Davi a Abisai, e a todos os seus servos:
Eis que meu filho, que saiu das minhas entranhas, procura a minha
morte; quanto mais ainda este benjamita? Deixai-o, que amaldiçoe;
porque o Senhor lho disse. Porventura o Senhor olhará para a
minha miséria; e o Senhor me pagará com bem a sua maldição deste
dia. Prosseguiram, pois, o seu caminho, Davi e os seus
homens; e também Simei ia ao longo do monte, defronte dele,
caminhando e amaldiçoando, e atirava pedras contra ele, e levantava
poeira. E o rei e todo o povo que ia com ele chegaram
cansados, e refrescaram-se ali.

Absalão, pois, e todo o povo, os homens de Israel, vieram a
Jerusalém; e Aitofel com ele. E sucedeu que, chegando Husai,
o arquita, amigo de Davi, a Absalão, disse Husai a Absalão: Viva o
rei, viva o rei! Porém Absalão disse a Husai: É esta a tua
beneficência para com o teu amigo? Por que não foste com o teu
amigo? E disse Husai a Absalão: Não, porém daquele que eleger
o Senhor, e todo este povo, e todos os homens de Israel, dele serei
e com ele ficarei. E, demais disto, a quem serviria eu?
Porventura não seria diante de seu filho? Como servi diante de teu
pai, assim serei diante de ti. Então disse Absalão a Aitofel:
Dai conselho entre vós sobre o que devemos fazer. E disse
Aitofel a Absalão: Possue as concubinas de teu pai, que deixou para
guardarem a casa; e assim todo o Israel ouvirá que te fizeste
aborrecível para com teu pai; e se fortalecerão as mãos de todos os
que estão contigo. Estenderam, pois, para Absalão uma tenda
no terraço; e Absalão possuiu as concubinas de seu pai, perante os
olhos de todo o Israel. E era o conselho de Aitofel, que
aconselhava naqueles dias, como se a palavra de Deus se consultara;
tal era todo o conselho de Aitofel, assim para com Davi como para
com Absalão.

\medskip

\lettrine{17} Disse mais Aitofel a Absalão: Deixa-me escolher
doze mil homens, e me levantarei, e perseguirei a Davi esta noite.
E irei sobre ele, pois está cansado e frouxo de mãos; e o
espantarei, e fugirá todo o povo que está com ele; e então ferirei
somente o rei. E farei tornar a ti todo o povo; pois o homem a
quem tu buscas é como se tornassem todos; assim todo o povo estará
em paz. E esta palavra pareceu boa aos olhos de Absalão, e aos
olhos de todos os anciãos de Israel. Disse, porém, Absalão:
Chamai agora também a Husai o arquita; e ouçamos também o que ele
dirá. E, chegando Husai a Absalão, lhe falou Absalão, dizendo:
Desta maneira falou Aitofel; faremos conforme à sua palavra? Se não,
fala tu. Então disse Husai a Absalão: O conselho que Aitofel deu
desta vez não é bom. Disse mais Husai: Bem conheces tu a teu
pai, e a seus homens, que são valorosos, e que estão com o espírito
amargurado, como a ursa no campo, roubada dos cachorros; e também
teu pai é homem de guerra, e não passará a noite com o povo. Eis
que agora estará escondido nalguma cova, ou em qualquer outro lugar;
e será que, caindo no princípio alguns dentre eles, cada um que o
ouvir então dirá: Houve derrota no povo que segue a Absalão.
Então até o homem valente, cujo coração é como coração de
leão, sem dúvida desmaiará; porque todo o Israel sabe que teu pai é
valoroso, e homens valentes os que estão com ele. Eu, porém,
aconselho que com toda a pressa se ajunte a ti todo o Israel desde
Dã até Berseba, em multidão como a areia do mar; e tu em pessoa vás
com eles à peleja. Então iremos a ele, em qualquer lugar que
se achar, facilmente cairemos sobre ele, como o orvalho cai sobre a
terra; e não ficará dele e de todos os homens que estão com ele nem
ainda um só. E, se ele se retirar para alguma cidade, todo o
Israel levará cordas àquela cidade; e arrastá-la-emos até ao
ribeiro, até que não se ache ali nem uma só pedrinha. Então
disse Absalão e todos os homens de Israel: Melhor é o conselho de
Husai, o arquita, do que o conselho de Aitofel (porém assim o Senhor
o ordenara, para aniquilar o bom conselho de Aitofel, para que o
Senhor trouxesse o mal sobre Absalão).

E disse Husai a Zadoque e a Abiatar, sacerdotes: assim e assim
aconselhou Aitofel a Absalão e aos anciãos de Israel; porém assim e
assim aconselhei eu. Agora, pois, enviai apressadamente, e
avisai a Davi, dizendo: Não passes esta noite nas campinas do
deserto; logo também passa ao outro lado, para que o rei e todo o
povo que com ele está não seja devorado. Estavam, pois,
Jônatas e Aimaás junto à fonte de Rogel; e foi uma criada, e lho
disse, e eles foram e o disseram ao rei Davi, porque não podiam ser
vistos entrar na cidade. Mas viu-os todavia um moço, e avisou
a Absalão; porém ambos logo partiram apressadamente, e entraram em
casa de um homem, em Baurim, o qual tinha um poço no seu pátio, e
ali dentro desceram. E tomou a mulher a tampa, e a estendeu
sobre a boca do poço, e espalhou grão descascado sobre ela; assim
nada se soube. Chegando, pois, os servos de Absalão à mulher,
àquela casa, disseram: Onde estão Aimaás e Jônatas? E a mulher lhes
disse: Já passaram o vau das águas. E havendo-os buscado, e não os
achando, voltaram para Jerusalém. E sucedeu que, depois que
se retiraram, Aimaás e Jônatas saíram do poço, e foram, e anunciaram
a Davi; e disseram a Davi: Levantai-vos, e passai depressa as águas,
porque assim aconselhou contra vós Aitofel.

Então Davi e todo o povo que com ele estava se levantou, e
passaram o Jordão; e já pela luz da manhã nem ainda faltava um só
que não tivesse passado o Jordão. Vendo, pois, Aitofel que se
não tinha seguido o seu conselho, albardou o jumento, e levantou-se,
e foi para sua casa e para a sua cidade, e deu ordem à sua casa, e
se enforcou e morreu, e foi sepultado na sepultura de seu pai.
E Davi foi a Maanaim; e Absalão passou o Jordão, ele e todo o
homem de Israel com ele. E Absalão constituiu a Amasa em
lugar de Joabe sobre o arraial; e era Amasa filho de um homem cujo
nome era Itra, o israelita, o qual possuíra a Abigail, filha de
Naás, irmã de Zeruia, mãe de Joabe. Israel, pois, e Absalão
acamparam na terra de Gileade. E sucedeu que, chegando Davi a
Maanaim, Sobi, filho de Naás, de Rabá, dos filhos de Amom, e Maquir,
filho de Amiel, de Lo-Debar, e Barzilai, o gileadita, de Rogelim,
tomaram camas e bacias, e vasilhas de barro, e trigo, e
cevada, e farinha, e grão torrado, e favas, e lentilhas, também
torradas, e mel, e manteiga, e ovelhas, e queijos de vacas, e
os trouxeram a Davi e ao povo que com ele estava, para comerem,
porque disseram: Este povo no deserto está faminto, cansado e
sedento.

\medskip

\lettrine{18} E Davi contou o povo que tinha consigo, e pôs
sobre eles capitães de mil e capitães de cem. E Davi enviou o
povo, um terço sob o mando de Joabe, e outro terço sob o mando de
Abisai, filho de Zeruia, irmão de Joabe, e outro terço sob o mando
de Itai, o giteu; e disse o rei ao povo: Eu também sairei convosco.
Porém o povo disse: Não sairás, porque, se formos obrigados a
fugir, não se importarão conosco; e, ainda que metade de nós morra,
não farão caso de nós, porque ainda, tais como nós somos, ajuntarás
dez mil; melhor será, pois, que da cidade nos sirvas de socorro.
Então disse-lhe Davi: O que bem parecer aos vossos olhos, farei.
E o rei se pôs do lado da porta, e todo o povo saiu em centenas e em
milhares. E o rei deu ordem a Joabe, e a Abisai, e a Itai,
dizendo: Brandamente tratai, por amor de mim, ao jovem Absalão. E
todo o povo ouviu quando o rei deu ordem a todos os capitães acerca
de Absalão. Saiu, pois, o povo ao campo, a encontrar-se com
Israel, e deu-se a batalha no bosque de Efraim. E ali foi ferido
o povo de Israel, diante dos servos de Davi; e naquele mesmo dia
houve ali uma grande derrota de vinte mil. Porque ali se
derramou a batalha sobre a face de toda aquela terra; e foram mais
os do povo que o bosque consumiu do que os que a espada consumiu
naquele dia.

E Absalão se encontrou com os servos de Davi; e Absalão ia montado
num mulo; e, entrando o mulo debaixo dos espessos ramos de um grande
carvalho, pegou-se-lhe a cabeça no carvalho, e ficou pendurado entre
o céu e a terra; e o mulo, que estava debaixo dele, passou adiante.
O que vendo um homem, fez saber a Joabe, e disse: Eis que vi
a Absalão pendurado num carvalho. Então disse Joabe ao homem
que lho fizera saber: Pois que o viste, por que o não feriste logo
ali em terra? E forçoso seria o eu dar-te dez moedas de prata e um
cinto. Disse, porém, aquele homem a Joabe: Ainda que eu
pudesse pesar nas minhas mãos mil moedas de prata, não estenderia a
minha mão contra o filho do rei, pois bem ouvimos que o rei te deu
ordem a ti, e a Abisai, e a Itai, dizendo: Guardai-vos, cada um de
vós, de tocar no jovem Absalão. Ainda que cometesse mentira a
risco da minha vida, nem por isso coisa nenhuma se esconderia ao
rei; e tu mesmo te oporias. Então disse Joabe: Não me
demorarei assim contigo aqui. E tomou três dardos, e traspassou com
eles o coração de Absalão, estando ele ainda vivo no meio do
carvalho. E o cercavam dez moços, que levaram as armas de
Joabe. E feriram a Absalão, e o mataram. Então tocou Joabe a
buzina, e voltou o povo de perseguir a Israel, porque Joabe deteve o
povo. E tomaram a Absalão, e o lançaram no bosque, numa
grande cova, e levantaram sobre ele um mui grande montão de pedras;
e todo o Israel fugiu, cada um para a sua tenda. Ora,
Absalão, quando ainda vivia, tinha tomado e levantado para si uma
coluna, que está no vale do rei, porque dizia: Filho nenhum tenho
para conservar a memória do meu nome. E chamou aquela coluna pelo
seu próprio nome; por isso até ao dia de hoje se chama o Pilar de
Absalão.

Então disse Aimaás, filho de Zadoque: Deixa-me correr, e
denunciarei ao rei que já o Senhor o vingou da mão de seus inimigos.
Mas Joabe lhe disse: Tu não serás hoje o portador de novas,
porém outro dia as levarás; mas hoje não darás a nova, porque é
morto o filho do rei. E disse Joabe a Cusi: Vai tu, e dize ao
rei o que viste. E Cusi se inclinou a Joabe, e correu. E
prosseguiu Aimaás, filho de Zadoque, e disse a Joabe: Seja o que for
deixa-me também correr após Cusi. E disse Joabe: Para que agora
correrias tu, meu filho, pois não tens mensagem conveniente?
Seja o que for, disse Aimaás, correrei. E Joabe lhe disse:
Corre. E Aimaás correu pelo caminho da planície, e passou a Cusi.
E Davi estava assentado entre as duas portas; e a sentinela
subiu ao terraço da porta junto ao muro; e levantou os olhos, e
olhou, e eis que um homem corria só. Gritou, pois, a
sentinela, e o disse ao rei: Se vem só, há novas em sua boca. E
vinha andando e chegando. Então viu a sentinela outro homem
que corria, e a sentinela gritou ao porteiro, e disse: Eis que lá
vem outro homem correndo só. Então disse o rei: Também traz este
novas. Disse mais a sentinela: Vejo o correr do primeiro, que
parece ser o correr de Aimaás, filho de Zadoque. Então disse o rei:
Este é homem de bem, e virá com boas novas. Gritou, pois,
Aimaás, e disse ao rei: Paz. E inclinou-se ao rei com o rosto em
terra, e disse: Bendito seja o Senhor, que entregou os homens que
levantaram a mão contra o rei meu senhor. Então disse o rei:
Vai bem com o jovem, com Absalão? E disse Aimaás: Vi um grande
alvoroço, quando Joabe mandou o servo do rei, e a mim teu servo;
porém não sei o que era. E disse o rei: Vira-te, e põe-te
aqui. E virou-se, e parou. E eis que vinha Cusi; e disse
Cusi: Anunciar-se-á ao rei meu Senhor que hoje o Senhor te vingou da
mão de todos os que se levantaram contra ti. Então disse o
rei a Cusi: Vai bem com o jovem, com Absalão? E disse Cusi: Sejam
como aquele jovem os inimigos do rei meu senhor, e todos os que se
levantam contra ti para mal. Então o rei se perturbou, e
subiu à sala que estava por cima da porta, e chorou; e andando,
dizia assim: Meu filho Absalão, meu filho, meu filho, Absalão! Quem
me dera que eu morrera por ti, Absalão, meu filho, meu filho!

\medskip

\lettrine{19} E disseram a Joabe: Eis que o rei anda chorando,
e lastima-se por Absalão. Então a vitória se tornou naquele
mesmo dia em tristeza por todo o povo; porque naquele mesmo dia o
povo ouvira dizer: Mui triste está o rei por causa de seu filho.
E naquele mesmo dia o povo entrou às furtadelas na cidade, como
o faz quando, envergonhado, foge da peleja. 4 Estava, pois, o rei
com o rosto coberto; e o rei gritava a alta voz: Meu filho Absalão,
Absalão meu filho, meu filho! Então entrou Joabe na casa do rei,
e disse: Hoje envergonhaste o rosto de todos os teus servos, que
livraram hoje a tua vida, e a vida de teus filhos, e de tuas filhas,
e a vida de tuas mulheres, e a vida de tuas concubinas; amando
tu aos teus inimigos, e odiando aos teus amigos. Porque hoje dás a
entender que nada valem para contigo príncipes e servos; porque
entendo hoje que se Absalão vivesse, e todos nós hoje fôssemos
mortos, estarias bem contente. Levanta-te, pois, agora; sai, e
fala conforme ao coração de teus servos; porque pelo Senhor te juro
que, se não saíres, nem um só homem ficará contigo esta noite; e
maior mal te será isto do que todo o mal que tem vindo sobre ti
desde a tua mocidade até agora. Então o rei se levantou, e se
assentou à porta; e fizeram saber a todo o povo dizendo: Eis que o
rei está assentado à porta. Então todo o povo veio apresentar-se
diante do rei; porém Israel havia fugido cada um para a sua tenda.

E todo o povo, em todas as tribos de Israel, andava porfiando
entre si, dizendo: O rei nos tirou das mãos de nossos inimigos, e
ele nos livrou das mãos dos filisteus; e agora fugiu da terra por
causa de Absalão. E Absalão, a quem ungimos sobre nós, já
morreu na peleja; agora, pois, por que vos calais, e não fazeis
voltar o rei? Então o rei Davi mandou dizer a Zadoque e a
Abiatar, sacerdotes: Falai aos anciãos de Judá, dizendo: Por que
seríeis vós os últimos em tornar a trazer o rei para a sua casa?
Porque as palavras de todo o Israel chegaram ao rei, até à sua casa.
Vós sois meus irmãos, meus ossos e minha carne sois vós; por
que, pois, seríeis os últimos em tornar a trazer o rei? E a
Amasa direis: Porventura não és tu meu osso e minha carne? Assim me
faça Deus, e outro tanto, se não fores capitão do arraial diante de
mim para sempre, em lugar de Joabe. Assim moveu ele o coração
de todos os homens de Judá, como o de um só homem; e
enviaram\footnote{King James: ``And he bowed the heart of all the
men of Judah, even as the heart of one man; so that they \emph{sent
this word} unto the king, return thou, and all thy servants.'' RA:
Com isto moveu o rei o coração de todos os homens de Judá, como se
fora um só homem, e mandaram dizer-lhe: Volta, ó rei, tu e todos os
teus servos.} ao rei, dizendo: Volta tu com todos os teus servos.
Então o rei voltou, e chegou até ao Jordão; e Judá veio a
Gilgal, para ir encontrar-se com o rei, ao outro lado do Jordão.

E apressou-se Simei, filho de Gera, benjamita, que era de Baurim;
e desceu com os homens de Judá a encontrar-se com o rei Davi.
E com ele mil homens de Benjamim, como também Ziba, servo da
casa de Saul, e seus quinze filhos, e seus vinte servos com ele; e
prontamente passaram o Jordão adiante do rei. E, atravessando
a barca, para fazer passar a casa do rei e para fazer o que bem
parecesse aos seus olhos, então Simei, filho de Gera, se prostrou
diante do rei, quando ele passava o Jordão. E disse ao rei:
Não me impute meu senhor a minha culpa, e não te lembres do que tão
perversamente fez teu servo, no dia em que o rei meu senhor saiu de
Jerusalém; não conserve o rei isso no coração. Porque teu
servo deveras confessa que pecou; porém eis que eu sou o primeiro
que de toda a casa de José desci a encontrar-me com o rei meu
Senhor. Então respondeu Abisai, filho de Zeruia, e disse: Não
morreria, pois, Simei por isto, havendo amaldiçoado ao ungido do
Senhor? Porém Davi disse: Que tenho eu convosco, filhos de
Zeruia, para que hoje me sejais adversários? Morreria alguém hoje em
Israel? Pois porventura não sei que hoje fui feito rei sobre Israel?
E disse o rei a Simei: Não morrerás. E o rei lho jurou.

Também Mefibosete, filho de Saul, desceu a encontrar-se com o
rei, e não tinha lavado os pés, nem tinha feito a barba, nem tinha
lavado as suas vestes desde o dia em que o rei tinha saído até ao
dia em que voltou em paz. E sucedeu que, vindo ele a
Jerusalém a encontrar-se com o rei, disse-lhe o rei: Por que não
foste comigo, Mefibosete? E disse ele: Ó rei meu senhor, o
meu servo me enganou; porque o teu servo dizia: Albardarei um
jumento, e nele montarei, e irei com o rei; pois o teu servo é coxo.
Demais disto, falsamente acusou a teu servo diante do rei meu
senhor; porém o rei meu senhor é como um anjo de Deus; faze, pois, o
que parecer bem aos teus olhos. Porque toda a casa de meu pai
não era senão de homens dignos de morte diante do rei meu senhor; e
contudo puseste a teu servo entre os que comem à tua mesa; e que
mais direito tenho eu de clamar ao rei? E disse-lhe o rei:
Por que ainda mais falas de teus negócios? Já disse eu: Tu e Ziba
reparti as terras. E disse Mefibosete ao rei: Tome ele também
tudo; pois já veio o rei meu senhor em paz à sua casa.

Também Barzilai, o gileadita, desceu de Rogelim, e passou com o
rei o Jordão, para o acompanhar ao outro lado do Jordão. E
era Barzilai muito velho, da idade de oitenta anos; e ele tinha
sustentado o rei, quando tinha a sua morada em Maanaim, porque era
grande homem. E disse o rei a Barzilai: Passa tu comigo, e
sustentar-te-ei comigo em Jerusalém. Porém Barzilai disse ao
rei: Quantos serão os dias dos anos da minha vida, para que suba com
o rei a Jerusalém? Da idade de oitenta anos sou eu hoje;
poderia eu discernir entre o bom e o mau? Poderia o teu servo ter
gosto no que comer e beber? Poderia eu mais ouvir a voz dos cantores
e cantoras? E por que será o teu servo ainda pesado ao rei meu
senhor? Com o rei passará teu servo ainda um pouco mais além
do Jordão; e por que me recompensará o rei com tal recompensa?
Deixa voltar o teu servo, e morrerei na minha cidade, junto à
sepultura de meu pai e de minha mãe; mas eis aí está o teu servo
Quimã; passe ele com o rei meu senhor, e faze-lhe o que bem parecer
aos teus olhos. Então disse o rei: Quimã passará comigo, e eu
lhe farei como bem parecer aos teus olhos, e tudo quanto me pedires
te farei. Havendo, pois, todo o povo passado o Jordão, e
passando também o rei, beijou o rei a Barzilai, e o abençoou; e ele
voltou para o seu lugar.

E dali passou o rei a Gilgal, e Quimã passou com ele; e todo o
povo de Judá conduziu o rei, como também a metade do povo de Israel.
E eis que todos os homens de Israel vieram ao rei, e disseram
ao rei: Por que te furtaram nossos irmãos, os homens de Judá, e
conduziram o rei e a sua casa dalém do Jordão, e todos os homens de
Davi com eles? Então responderam todos os homens de Judá aos
homens de Israel: Porquanto o rei é nosso parente; e por que vos
irais por isso? Porventura comemos às custas do rei, ou nos deu
algum presente? E responderam os homens de Israel aos homens
de Judá, e disseram: Dez partes temos no rei, e até em Davi mais
temos nós do que vós; por que, pois, não fizestes conta de nós, para
que a nossa palavra não fosse a primeira, para tornar a trazer o
nosso rei? Porém a palavra dos homens de Judá foi mais forte do que
a palavra dos homens de Israel.

\medskip

\lettrine{20} Então se achou ali por acaso um homem de Belial,
cujo nome era Seba, filho de Bicri, homem de Benjamim, o qual tocou
a buzina, e disse: Não temos parte em Davi, nem herança no filho de
Jessé; cada um às suas tendas, ó Israel. Então todos os homens
de Israel se separaram de Davi, e seguiram Seba, filho de Bicri;
porém os homens de Judá se uniram ao seu rei desde o Jordão até
Jerusalém. Vindo, pois, Davi para sua casa, em Jerusalém, tomou
o rei as dez mulheres, suas concubinas, que deixara para guardarem a
casa, e as pôs numa casa sob guarda, e as sustentava; porém não as
possuiu; e estiveram encerradas até ao dia da sua morte, vivendo
como viúvas.

Disse mais o rei a Amasa: Convoca-me os homens de Judá para o
terceiro dia; e tu então apresenta-te aqui. E foi Amasa para
convocar a Judá; porém demorou-se além do tempo que lhe tinha sido
designado. Então disse Davi a Abisai: Mais mal agora nos fará
Seba, o filho de Bicri, do que Absalão; por isso toma tu os servos
de teu senhor, e persegue-o, para que não ache para si cidades
fortes; e escape dos nossos olhos. Então saíram atrás dele os
homens de Joabe, e os quereteus, e os peleteus, e todos os valentes;
estes saíram de Jerusalém para irem atrás de Seba, filho de Bicri.
Chegando eles, pois, à pedra grande, que está junto a Gibeom,
Amasa veio diante deles; e estava Joabe cingido da sua roupa que
vestira, e sobre ela um cinto, ao qual estava presa a espada a seus
lombos, na sua bainha; e, adiantando-se ele, lhe caiu a espada.
E disse Joabe a Amasa: Vai bem, meu irmão? E Joabe, com a mão
direita, pegou da barba de Amasa, para o beijar. E Amasa não
se resguardou da espada que estava na mão de Joabe, de sorte que
este o feriu com ela na quinta costela, e lhe derramou por terra as
entranhas, e não o feriu segunda vez, e morreu; então Joabe e
Abisai, seu irmão, foram atrás de Seba, filho de Bicri. Mas
um dentre os homens de Joabe parou junto a ele, e disse: Quem há que
queira bem a Joabe, e quem seja por Davi, siga Joabe. E Amasa
estava envolto no seu sangue no meio do caminho; e, vendo aquele
homem, que todo o povo parava, removeu a Amasa do caminho para o
campo, e lançou sobre ele um manto; porque via que todo aquele que
chegava a ele parava. E, como estava removido do caminho,
todos os homens seguiram a Joabe, para perseguirem a Seba, filho de
Bicri.

E ele passou por todas as tribos de Israel até Abel, e Bete-Maaca
e a todos os beritas; e ajuntaram-se, e também o seguiram. E
vieram, e o cercaram em Abel de Bete-Maaca, e levantaram uma
barragem contra a cidade, e isto colocado na trincheira; e todo o
povo que estava com Joabe batia no muro, para derrubá-lo.
Então uma mulher sábia gritou de dentro da cidade: Ouvi,
ouvi, peço-vos que digais a Joabe: Chega-te aqui, para que eu te
fale. Chegando-se a ela, a mulher lhe disse: Tu és Joabe? E
disse ele: Eu sou. E ela lhe disse: Ouve as palavras da tua serva. E
disse ele: Ouço. Então falou ela, dizendo: Antigamente
costumava-se dizer: Certamente pediram conselho a Abel; e assim
resolveram. Sou eu uma das pacíficas e das fiéis em Israel; e
tu procuras matar uma cidade que é mãe em Israel; por que, pois,
devorarias a herança do Senhor? Então respondeu Joabe, e
disse: Longe, longe de mim que eu tal faça, que eu devore ou
arruíne! A coisa não é assim; porém um só homem do monte de
Efraim, cujo nome é Seba, filho de Bicri, levantou a mão contra o
rei, contra Davi; entregai-me só este, e retirar-me-ei da cidade.
Então disse a mulher a Joabe: Eis que te será lançada a sua cabeça
pelo muro. E a mulher, na sua sabedoria, foi a todo o povo, e
cortaram a cabeça de Seba, filho de Bicri, e a lançaram a Joabe;
então este tocou a buzina, e se retiraram da cidade, cada um para a
sua tenda, e Joabe voltou a Jerusalém, ao rei.

E Joabe estava sobre todo o exército de Israel; e Benaia, filho
de Joiada, sobre os quereteus e sobre os peleteus; e Adorão
sobre os tributos; e Jeosafá, filho de Ailude, era o cronista;
e Seva, o escrivão; e Zadoque e Abiatar, os sacerdotes;
e também Ira, o jairita, era o oficial-mor de Davi.

\medskip

\lettrine{21} E houve nos dias de Davi uma fome de três anos
consecutivos; e Davi consultou ao Senhor, e o Senhor lhe disse: É
por causa de Saul e da sua casa sanguinária, porque matou os
gibeonitas. Então chamou o rei aos gibeonitas, e lhes falou (ora
os gibeonitas não eram dos filhos de Israel, mas do restante dos
amorreus, e os filhos de Israel lhes tinham jurado, porém Saul, no
seu zelo à causa dos filhos de Israel e de Judá, procurou feri-los).
Disse, pois, Davi aos gibeonitas: Que quereis que eu vos faça? E
que satisfação vos darei, para que abençoeis a herança do Senhor?
Então os gibeonitas lhe disseram: Não é por prata nem ouro que
temos questão com Saul e com sua casa; nem tampouco pretendemos
matar pessoa alguma em Israel. E disse ele: Que é, pois, que quereis
que vos faça? E disseram ao rei: O homem que nos destruiu, e
intentou contra nós de modo que fôssemos assolados, sem que
pudéssemos subsistir em termo algum de Israel, de seus filhos se
nos dêem sete homens, para que os enforquemos ao Senhor em Gibeá de
Saul, o eleito do Senhor. E disse o rei: Eu os darei. Porém o
rei poupou a Mefibosete, filho de Jônatas, filho de Saul, por causa
do juramento do Senhor, que entre eles houvera, entre Davi e
Jônatas, filho de Saul. Mas tomou o rei os dois filhos de Rispa,
filha de Aiá, que tinha tido de Saul, a Armoni e a Mefibosete; como
também os cinco filhos da irmã de Mical, filha de Saul, que tivera
de Adriel, filho de Barzilai, meolatita, e os entregou na mão
dos gibeonitas, os quais os enforcaram no monte, perante o Senhor; e
caíram estes sete juntamente; e foram mortos nos dias da sega, nos
dias primeiros, no princípio da sega das cevadas.

Então Rispa, filha de Aiá, tomou um pano de
cilício\footnote{Pequena túnica ou cinto ou cordão, de crina, de lã
áspera, às vezes com farpas de madeira, que, por penitência, se
trazia vestido diretamente sobre a pele.}, e estendeu-lho sobre uma
penha, desde o princípio da sega até que a água do céu caiu sobre
eles; e não deixou as aves do céu pousar sobre eles de dia, nem os
animais do campo de noite. E foi contado a Davi o que fizera
Rispa, filha de Aiá, concubina de Saul. Então foi Davi, e
tomou os ossos de Saul, e os ossos de Jônatas seu filho, dos
moradores de Jabes-Gileade, os quais os furtaram da rua de Bete-Sã,
onde os filisteus os tinham pendurado, quando feriram a Saul em
Gilboa. E fez subir dali os ossos de Saul, e os ossos de
Jônatas seu filho; e ajuntaram também os ossos dos enforcados.
Enterraram os ossos de Saul, e de Jônatas seu filho na terra
de Benjamim, em Zela, na sepultura de seu pai Quis, e fizeram tudo o
que o rei ordenara; e depois disto Deus se aplacou com a terra.

Tiveram mais os filisteus uma peleja contra Israel; e desceu
Davi, e com ele os seus servos; e tanto pelejaram contra os
filisteus, que Davi se cansou. E Isbi-Benobe, que era dos
filhos do gigante, cuja lança pesava trezentos siclos de cobre, e
que cingia uma espada nova, intentou ferir a Davi. Porém,
Abisai, filho de Zeruia, o socorreu, e feriu o filisteu, e o matou.
Então os homens de Davi lhe juraram, dizendo: Nunca mais sairás
conosco à peleja, para que não apagues a lâmpada de Israel. E
aconteceu depois disto que houve em Gobe ainda outra peleja contra
os filisteus; então Sibecai, o husatita, feriu a Safe, que era dos
filhos do gigante. Houve mais outra peleja contra os
filisteus em Gobe; e El-Hanã, filho de Jaaré-Oregim, o belemita,
feriu Golias, o giteu, de cuja lança era a haste como órgão de
tecelão. Houve ainda também outra peleja em Gate, onde estava
um homem de alta estatura, que tinha em cada mão seis dedos, e em
cada pé outros seis, vinte e quatro ao todo, e também este nascera
do gigante. E injuriava a Israel; porém Jônatas, filho de
Simei, irmão de Davi, o feriu. Estes quatro nasceram ao
gigante em Gate; e caíram pela mão de Davi e pela mão de seus
servos.

\medskip

\lettrine{22} E falou Davi ao Senhor as palavras deste
cântico, no dia em que o Senhor o livrou das mãos de todos os seus
inimigos e das mãos de Saul.

Disse pois: O Senhor é o meu rochedo, e o meu lugar forte, e o meu
libertador. Deus é o meu rochedo, nele confiarei; o meu escudo,
e a força da minha salvação, o meu alto retiro, e o meu refúgio. Ó
meu Salvador, da violência me salvas. O Senhor, digno de louvor,
invocarei, e de meus inimigos ficarei livre, porque me cercaram
as ondas de morte; as torrentes dos homens ímpios me assombraram.
Cordas do inferno me cingiram; encontraram-me laços de morte.
Estando em angústia, invoquei ao Senhor, e a meu Deus clamei; do
seu templo ouviu ele a minha voz, e o meu clamor chegou aos seus
ouvidos. Então se abalou e tremeu a terra, os fundamentos dos
céus se moveram e abalaram, porque ele se irou. Subiu fumaça de
suas narinas, e da sua boca um fogo devorador; carvões se incenderam
dele. E abaixou os céus, e desceu; e uma escuridão havia
debaixo de seus pés. E subiu sobre um querubim, e voou; e foi
visto sobre as asas do vento. E por tendas pôs as trevas ao
redor de si; ajuntamento de águas, nuvens dos céus. Pelo
resplendor da sua presença brasas de fogo se acenderam.
Trovejou desde os céus o Senhor; e o Altíssimo fez soar a sua
voz. E disparou flechas, e os dissipou; raios, e os
perturbou. E apareceram as profundezas do mar, e os
fundamentos do mundo se descobriram; pela repreensão do Senhor, pelo
sopro do vento das suas narinas. Desde o alto enviou, e me
tomou; tirou-me das muitas águas. Livrou-me do meu poderoso
inimigo, e daqueles que me tinham ódio, porque eram mais fortes do
que eu. Encontraram-me no dia da minha calamidade; porém o
Senhor se fez o meu amparo. E tirou-me para um lugar
espaçoso, e livrou-me, porque tinha prazer em mim.
Recompensou-me o Senhor conforme a minha justiça; conforme a
pureza de minhas mãos me retribuiu. Porque guardei os
caminhos do Senhor; e não me apartei impiamente do meu Deus.
Porque todos os seus juízos estavam diante de mim; e de seus
estatutos não me desviei. Porém fui sincero perante ele; e
guardei-me da minha iniqüidade. E me retribuiu o Senhor
conforme a minha justiça, conforme a minha pureza diante dos seus
olhos. Com o benigno, te mostras benigno; com o homem íntegro
te mostras perfeito. Com o puro te mostras puro; mas com o
perverso te mostras rígido. E o povo aflito livras; mas teus
olhos são contra os altivos, e tu os abaterás. Porque tu,
Senhor, és a minha lâmpada; e o Senhor ilumina as minhas trevas.
Porque contigo passo pelo meio de um esquadrão; pelo meu Deus
salto um muro. O caminho de Deus é perfeito, e a palavra do
Senhor refinada; e é o escudo de todos os que nele confiam.
Por que, quem é Deus, senão o Senhor? E quem é rochedo, senão
o nosso Deus? Deus é a minha fortaleza e a minha força, e ele
perfeitamente desembaraça o meu caminho. Faz ele os meus pés
como os das cervas, e me põe sobre as minhas alturas. Instrui
as minhas mãos para a peleja, de maneira que um arco de cobre se
quebra pelos meus braços. Também me deste o escudo da tua
salvação, e pela tua brandura me vieste a engrandecer.
Alargaste os meus passos debaixo de mim, e não vacilaram os
meus artelhos. Persegui os meus inimigos, e os derrotei, e
nunca me tornei até que os consumisse. E os consumi, e os
atravessei, de modo que nunca mais se levantaram, mas caíram debaixo
dos meus pés. Porque me cingiste de força para a peleja;
fizeste abater-se debaixo de mim os que se levantaram contra mim,
e deste-me o pescoço de meus inimigos, daqueles que me tinham
ódio, e os destruí. Olharam, porém não houve libertador; sim,
para o Senhor, porém não lhes respondeu. Então os moí como o
pó da terra; como a lama das ruas os trilhei e dissipei.
Também me livraste das contendas do meu povo; guardaste-me
para cabeça das nações; o povo que não conhecia me servirá.
Os filhos de estranhos se me sujeitaram; ouvindo a minha voz,
me obedeceram. Os filhos de estranhos desfaleceram; e,
cingindo-se, saíram dos seus esconderijos. Vive o Senhor, e
bendito seja o meu rochedo; e exaltado seja Deus, a rocha da minha
salvação, o Deus que me dá inteira vingança, e sujeita os
povos debaixo de mim. e o que me tira dentre os meus
inimigos; e tu me exaltas sobre os que contra mim se levantam; do
homem violento me livras. Por isso, ó Senhor, te louvarei
entre os gentios, e entoarei louvores ao teu nome. Ele é a
torre das salvações do seu rei, e usa de benignidade com o seu
ungido, com Davi, e com a sua descendência para sempre.

\medskip

\lettrine{23} E estas são as últimas palavras de Davi: Diz
Davi, filho de Jessé, e diz o homem que foi levantado em altura, o
ungido do Deus de Jacó, e o suave em salmos de Israel. O
Espírito do Senhor falou por mim, e a sua palavra está na minha
boca. Disse o Deus de Israel, a Rocha de Israel a mim me falou:
Haverá um justo que domine sobre os homens, que domine no temor de
Deus. E será como a luz da manhã, quando sai o sol, da manhã sem
nuvens, quando pelo seu resplendor e pela chuva a erva brota da
terra. Ainda que a minha casa não seja tal para com Deus,
contudo estabeleceu comigo uma aliança eterna, que em tudo será bem
ordenado e guardado, pois toda a minha salvação e todo o meu prazer
está nele, apesar de que ainda não o faz brotar. Porém os filhos
de Belial todos serão como os espinhos que se lançam fora, porque
não podem ser tocados com a mão. Mas qualquer que os tocar se
armará de ferro e da haste de uma lança; e a fogo serão totalmente
queimados no mesmo lugar.

Estes são os nomes dos poderosos que Davi teve: Josebe-Bassebete,
filho de Taquemoni, o principal dos capitães; este era Adino, o
eznita, que se opusera a oitocentos, e os feriu de uma vez. E
depois dele Eleazar, filho de Dodó, filho de Aoí, entre os três
valentes que estavam com Davi quando provocaram os filisteus que ali
se ajuntaram à peleja, e quando se retiraram os homens de Israel.
Este se levantou, e feriu os filisteus, até lhe cansar a mão
e ficar a mão pegada à espada; e naquele dia o Senhor efetuou um
grande livramento; e o povo voltou junto dele, somente a tomar o
despojo. E depois dele Samá, filho de Agé, o hararita, quando
os filisteus se ajuntaram numa multidão, onde havia um pedaço de
terra cheio de lentilhas, e o povo fugira de diante dos filisteus.
Este, pois, se pôs no meio daquele pedaço de terra, e o
defendeu, e feriu os filisteus; e o Senhor efetuou um grande
livramento. Também três dos trinta chefes desceram, e no
tempo da sega foram a Davi, à caverna de Adulão; e a multidão dos
filisteus acampara no vale de Refaim. Davi estava então num
lugar forte, e a guarnição dos filisteus em Belém. E teve
Davi desejo, e disse: Quem me dera beber da água da cisterna de
Belém, que está junto à porta! Então aqueles três poderosos
romperam pelo arraial dos filisteus, e tiraram água da cisterna de
Belém, que está junto à porta, e a tomaram, e a trouxeram a Davi;
porém ele não a quis beber, mas derramou-a perante o Senhor.
E disse: Guarda-me, ó Senhor, de que tal faça; beberia eu o
sangue dos homens que foram com risco da sua vida? De maneira que
não a quis beber; isto fizeram aqueles três poderosos. Também
Abisai, irmão de Joabe, filho de Zeruia, era chefe de três; e este
alçou a sua lança contra trezentos e os feriu; e tinha nome entre os
três. Porventura este não era o mais nobre dentre estes três?
Pois era o primeiro deles; porém aos primeiros três não chegou.
Também Benaia, filho de Joiada, filho de um homem valoroso de
Cabzeel, grande em obras, este feriu dois fortes heróis de Moabe; e
desceu ele, e feriu um leão no meio duma cova, no tempo da neve.
Também este feriu um egípcio, homem de respeito; e na mão do
egípcio havia uma lança, porém ele desceu a ele com um cajado, e
arrancou a lança da mão do egípcio, e com ela o matou. Estas
coisas fez Benaia, filho de Joiada, pelo que teve nome entre três
poderosos. Dentre os trinta ele era o mais nobre, porém aos
três primeiros não chegou; e Davi o pôs sobre os seus guardas.
Asael, irmão de Joabe, estava entre os trinta; El-Hanã, filho
de Dodó, de Belém; Samá, harodita; Elica, harodita;
Helez, paltita; Ira, filho de Iques, tecoíta; Abiezer,
anatotita; Mebunai, husatita; Zalmom, aoíta; Maarai,
netofatita; Elebe, filho de Baaná, netofatita; Itai, filho de
Ribai, de Gibeá dos filhos de Benjamim; Benaia, piratonita;
Hidai, do ribeiro de Gaás; Abi-Albom, arbatita; Azmavete,
barumita; Eliaba, saalbonita; os filhos de Jásen e Jônatas;
Samá, hararita, Aião, filho de Sarar, ararita;\footnote{KJ:
Shammah the Hararite, Ahiam the son of Sharar the Hararite. RA:
Sama, hararita; Aião, filho de Sarar, ararita. RC: Sama, hararita,
Aião, filho de Sarar, ararita.} Elifelete, filho de Aasbai,
filho de um maacatita; Eliã, filho de Aitofel, gilonita;
Hesrai, carmelita; Paarai, arbita; Igal, filho de
Natã, de Zobá; Bani, gadita; Zeleque, amonita; Naarai,
beerotita, o que trazia as armas de Joabe, filho de Zeruia;
Ira, itrita; Garebe, itrita; Urias, heteu; trinta e
sete ao todo.

\medskip

\lettrine{24} E a ira do Senhor se tornou a acender contra
Israel; e incitou a Davi contra eles, dizendo: Vai, numera a Israel
e a Judá. Disse, pois, o rei a Joabe, capitão do exército, o
qual tinha consigo: Agora percorre todas as tribos de Israel, desde
Dã até Berseba, e numera o povo, para que eu saiba o número do povo.
Então disse Joabe ao rei: Ora, multiplique o Senhor teu Deus a
este povo cem vezes tanto quanto agora é, e os olhos do rei meu
senhor o vejam; mas, por que deseja o rei meu Senhor este negócio?
Porém a palavra do rei prevaleceu contra Joabe, e contra os
capitães do exército; Joabe, pois, saiu com os capitães do exército
da presença do rei, para numerar o povo de Israel. E passaram o
Jordão; e acamparam-se em Aroer, à direita da cidade que está no
meio do ribeiro de Gade, junto a Jazer. E foram a Gileade, e à
terra baixa de Hodsi; também foram até Dã-Jaã, e ao redor de Sidom.
E foram à fortaleza de Tiro, e a todas as cidades dos heveus e
dos cananeus; e saíram para o lado do sul de Judá, a Berseba.
Assim percorreram toda a terra; e ao cabo de nove meses e vinte
dias voltaram a Jerusalém. E Joabe deu ao rei a soma do número
do povo contado; e havia em Israel oitocentos mil homens de guerra,
que arrancavam da espada; e os homens de Judá eram quinhentos mil
homens.

E pesou o coração de Davi, depois de haver numerado o povo; e
disse Davi ao Senhor: Muito pequei no que fiz; porém agora ó Senhor,
peço-te que perdoes a iniqüidade do teu servo; porque tenho
procedido mui loucamente. Levantando-se, pois, Davi pela
manhã, veio a palavra do Senhor ao profeta Gade, vidente de Davi,
dizendo: Vai, e dize a Davi: Assim diz o Senhor: Três coisas
te ofereço; escolhe uma delas, para que ta faça. Foi, pois,
Gade a Davi, e fez-lho saber; e disse-lhe: Queres que sete anos de
fome te venham à tua terra; ou que por três meses fujas de teus
inimigos, e eles te persigam; ou que por três dias haja peste na tua
terra? Delibera agora, e vê que resposta hei de dar ao que me
enviou. Então disse Davi a Gade: Estou em grande angústia;
porém caiamos nas mãos do Senhor, porque muitas são as suas
misericórdias; mas nas mãos dos homens não caia eu. Então
enviou o Senhor a peste a Israel, desde a manhã até ao tempo
determinado; e desde Dã até Berseba, morreram setenta mil homens do
povo. Estendendo, pois, o anjo a sua mão sobre Jerusalém,
para a destruir, o Senhor se arrependeu daquele mal; e disse ao anjo
que fazia a destruição entre o povo: Basta, agora retira a tua mão.
E o anjo do Senhor estava junto à eira de Araúna, o jebuseu.
E, vendo Davi ao anjo que feria o povo, falou ao Senhor,
dizendo: Eis que eu sou o que pequei, e eu que iniquamente procedi;
porém estas ovelhas que fizeram? Seja, pois, a tua mão contra mim, e
contra a casa de meu pai.

E Gade veio naquele mesmo dia a Davi, e disse-lhe: Sobe, levanta
ao Senhor um altar na eira de Araúna, o jebuseu. Davi subiu
conforme à palavra de Gade, como o Senhor lhe tinha ordenado.
E olhou Araúna, e viu que vinham para ele o rei e os seus
servos; saiu, pois, Araúna e inclinou-se diante do rei com o rosto
em terra. E disse Araúna: Por que vem o rei meu Senhor ao seu
servo? E disse Davi: Para comprar de ti esta eira, a fim de edificar
nela um altar ao Senhor, para que este castigo cesse de sobre o
povo. Então disse Araúna a Davi: Tome, e ofereça o rei meu
senhor o que bem parecer aos seus olhos; eis aí bois para o
holocausto, e os trilhos\footnote{Utensílio de lavoura para debulhar
cereais.}, e o aparelho dos bois para a lenha. Tudo isto deu
Araúna ao rei; disse mais Araúna ao rei: O Senhor teu Deus tome
prazer em ti. Porém o rei disse a Araúna: Não, mas por preço
justo to comprarei, porque não oferecerei ao Senhor meu Deus
holocaustos que não me custem nada. Assim Davi comprou a eira e os
bois por cinqüenta siclos de prata. E edificou ali Davi ao
Senhor um altar, e ofereceu holocaustos, e ofertas pacíficas. Assim
o Senhor se aplacou para com a terra e cessou aquele castigo de
sobre Israel.

