\addchap{Esdras}

\lettrine{1} No primeiro ano de Ciro, rei da Pérsia (para que
se cumprisse a palavra do Senhor, pela boca de Jeremias), despertou
o Senhor o espírito de Ciro, rei da Pérsia, o qual fez passar pregão
por todo o seu reino, como também por escrito, dizendo: Assim
diz Ciro, rei da Pérsia: O Senhor Deus dos céus me deu todos os
reinos da terra, e me encarregou de lhe edificar uma casa em
Jerusalém, que está em Judá. Quem há entre vós, de todo o seu
povo, seja seu Deus com ele, e suba a Jerusalém, que está em Judá, e
edifique a casa do Senhor Deus de Israel (ele é o Deus) que está em
Jerusalém. E todo aquele que ficar atrás em algum lugar em que
andar peregrinando, os homens do seu lugar o ajudarão com prata, com
ouro, com bens, e com gados, além das dádivas voluntárias para a
casa de Deus, que está em Jerusalém.

Então se levantaram os chefes dos pais de Judá e Benjamim, e os
sacerdotes e os levitas, com todos aqueles cujo espírito Deus
despertou, para subirem a edificar a casa do Senhor, que está em
Jerusalém. E todos os que habitavam nos arredores lhes firmaram
as mãos com vasos de prata, com ouro, com bens e com gado, e com
coisas preciosas; além de tudo o que voluntariamente se deu.
Também o rei Ciro tirou os utensílios da casa do Senhor, que
Nabucodonosor tinha trazido de Jerusalém, e que tinha posto na casa
de seus deuses. Estes tirou Ciro, rei da Pérsia, pela mão de
Mitredate, o tesoureiro, que os entregou contados a Sesbazar,
príncipe de Judá. E este é o número deles: trinta travessas de
ouro, mil travessas de prata, vinte e nove facas, trinta
bacias de ouro, mais outras quatrocentas e dez bacias de prata, e
mil outros utensílios. Todos os utensílios de ouro e de prata
foram cinco mil e quatrocentos; todos estes levou Sesbazar, quando
os do cativeiro subiram de Babilônia para Jerusalém.

\medskip

\lettrine{2} Estes são os filhos da província, que subiram do
cativeiro, dentre os exilados, que Nabucodonosor, rei de Babilônia,
tinha transportado a Babilônia, e tornaram a Jerusalém e a Judá,
cada um para a sua cidade; os quais vieram com Zorobabel, Jesuá,
Neemias, Seraías, Reelaías, Mardoqueu, Bilsã, Mizpar, Bigvai, Reum e
Baaná. O número dos homens do povo de Israel: Os filhos de
Parós, dois mil cento e setenta e dois. Os filhos de Sefatias,
trezentos e setenta e dois. Os filhos de Ará, setecentos e
setenta e cinco. Os filhos de Paate-Moabe, dos filhos de
Jesuá-Joabe, dois mil oitocentos e doze. Os filhos de Elão, mil
duzentos e cinqüenta e quatro. Os filhos de Zatu, novecentos e
quarenta e cinco. Os filhos de Zacai, setecentos e sessenta.
Os filhos de Bani, seiscentos e quarenta e dois. Os
filhos de Bebai, seiscentos e vinte e três. Os filhos de
Azgade, mil duzentos e vinte e dois. Os filhos de Adonicão,
seiscentos e sessenta e seis. Os filhos de Bigvai, dois mil e
cinqüenta e seis. Os filhos de Adim, quatrocentos e cinqüenta
e quatro. Os filhos de Ater, de Ezequias, noventa e oito.
Os filhos de Bezai, trezentos e vinte e três. Os
filhos de Jora, cento e doze. Os filhos de Hasum, duzentos e
vinte e três. Os filhos de Gibar, noventa e cinco. Os
filhos de Belém, cento e vinte e três. Os homens de Netofá,
cinqüenta e seis. Os homens de Anatote, cento e vinte e oito.
Os filhos de Azmavete, quarenta e dois. Os filhos de
Quiriate-Arim, Quefira e Beerote, setecentos e quarenta e três.
Os filhos de Ramá, e de Geba, seiscentos e vinte e um.
Os homens de Micmás, cento e vinte e dois. Os homens
de Betel e de Ai, duzentos e vinte e três. Os filhos de Nebo,
cinqüenta e dois. Os filhos de Magbis, cento e cinqüenta e
seis. Os filhos do outro Elão, mil duzentos e cinqüenta e
quatro. Os filhos de Harim, trezentos e vinte. Os
filhos de Lode, de Hadide e de Ono, setecentos e vinte e cinco.
Os filhos de Jericó, trezentos e quarenta e cinco. Os
filhos de Senaá, três mil seiscentos e trinta.

Os sacerdotes: os filhos de Jedaías, da casa de Jesuá, novecentos
e setenta e três. Os filhos de Imer, mil e cinqüenta e dois.
Os filhos de Pasur, mil duzentos e quarenta e sete. Os
filhos de Harim, mil e dezessete. Os levitas: os filhos de
Jesuá e Cadmiel, dos filhos de Hodavias, setenta e quatro. Os
cantores: os filhos de Asafe, cento e vinte e oito. Os filhos
dos porteiros: os filhos de Salum, os filhos de Ater, os filhos de
Talmom, os filhos de Acube, os filhos de Hatita, os filhos de Sobai;
ao todo, cento e trinta e nove. Os netinins: os filhos de
Zia, os filhos de Hasufa, os filhos de Tabaote, os filhos de
Querós, os filhos de Siá, os filhos de Padom, os filhos de
Lebaná, os filhos de Hagaba, os filhos de Acube, os filhos de
Hagabe, os filhos de Sanlai, os filhos de Hanã, os filhos de
Gidel, os filhos de Gaar, os filhos de Reaías, os filhos de
Rezim, os filhos de Necoda, os filhos de Gazão, os filhos de
Uzá, os filhos de Paseá, os filhos de Besai, os filhos de
Asna, os filhos dos meunitas, os filhos dos nefuseus, os
filhos de Bacbuque, os filhos de Hacufa, os filhos de Harur,
os filhos de Bazlute, os filhos de Meída, os filhos de Harsa,
os filhos de Barcos, os filhos de Sísera, os filhos de Tama.
Os filhos de Neziá, os filhos de Hatifa, os filhos dos
servos de Salomão; os filhos de Sotai, os filhos de Soferete, os
filhos de Peruda, os filhos de Jaalá, os filhos de Darcom, os
filhos de Gidel, os filhos de Sefatias, os filhos de Hatil,
os filhos de Poquerete-Hazebaim, os filhos de Ami. Todos os
netinins, e os filhos dos servos de Salomão, trezentos e noventa e
dois. Também estes subiram de Tel-Melá e Tel-Harsa, Querube,
Adã e Imer; porém não puderam provar que as suas famílias e a sua
linhagem eram de Israel. Os filhos de Delaías, os filhos de
Tobias, os filhos de Necoda, seiscentos e cinqüenta e dois. E
dos filhos dos sacerdotes: os filhos de Habaías, os filhos de Coz,
os filhos de Barzilai, que tomou mulher das filhas de Barzilai, o
gileadita, e que foi chamado do seu nome. Estes procuraram o
seu registro entre os que estavam arrolados nas genealogias, mas não
se acharam nelas; assim, por imundos, foram excluídos do sacerdócio.
E o governador lhes disse que não comessem das coisas
consagradas, até que houvesse sacerdote com Urim e com Tumim.

Toda esta congregação junta foi de quarenta e dois mil trezentos
e sessenta, afora os seus servos e as suas servas, que foram
sete mil trezentos e trinta e sete; também tinha duzentos cantores e
cantoras. Os seus cavalos, setecentos e trinta e seis; os
seus mulos, duzentos e quarenta e cinco; os seus camelos,
quatrocentos e trinta e cinco; os jumentos, seis mil setecentos e
vinte. E alguns dos chefes dos pais, vindo à casa do Senhor,
que habita em Jerusalém, deram ofertas voluntárias para a casa de
Deus, para a estabelecerem no seu lugar. Conforme as suas
posses, deram para o tesouro da obra, em ouro, sessenta e uma mil
dracmas, e em prata cinco mil libras, e cem vestes sacerdotais.
E habitaram os sacerdotes e os levitas, e alguns do povo,
tanto os cantores, como os porteiros, e os netinins, nas suas
cidades; como também todo o Israel nas suas cidades.

\medskip

\lettrine{3} Chegando, pois, o sétimo mês, e estando os filhos
de Israel já nas cidades, ajuntou-se o povo, como um só homem, em
Jerusalém. E levantou-se Jesuá, filho de Jozadaque, e seus
irmãos, os sacerdotes, e Zorobabel, filho de Sealtiel, e seus
irmãos, e edificaram o altar do Deus de Israel, para oferecerem
sobre ele holocaustos, como está escrito na lei de Moisés, o homem
de Deus. E firmaram o altar sobre as suas bases, porque o terror
estava sobre eles, por causa dos povos das terras; e ofereceram
sobre ele holocaustos ao Senhor, holocaustos pela manhã e à tarde.
E celebraram a festa dos tabernáculos, como está escrito;
ofereceram holocaustos cada dia, por ordem, conforme ao rito, cada
coisa em seu dia. E depois disto o holocausto contínuo, e os das
luas novas e de todas as solenidades consagradas ao Senhor; como
também de qualquer que oferecia oferta voluntária ao Senhor;
desde o primeiro dia do sétimo mês começaram a oferecer
holocaustos ao Senhor; porém ainda não estavam postos os fundamentos
do templo do Senhor. Deram, pois, o dinheiro aos pedreiros e
carpinteiros, como também comida e bebida, e azeite aos sidônios, e
aos tírios, para trazerem do Líbano madeira de cedro ao mar, para
Jope, segundo a concessão que lhes tinha feito Ciro, rei da Pérsia.

E no segundo ano da sua vinda à casa de Deus em Jerusalém, no
segundo mês, Zorobabel, filho de Sealtiel, e Jesuá, filho de
Jozadaque, e os outros seus irmãos, os sacerdotes e os levitas, e
todos os que vieram do cativeiro a Jerusalém, começaram a obra da
casa do Senhor, e constituíram os levitas da idade de vinte anos
para cima, para que a dirigissem. Então se levantou Jesuá, seus
filhos, e seus irmãos, Cadmiel e seus filhos, os filhos de Judá,
como um só homem, para dirigirem os que faziam a obra na casa de
Deus, bem como os filhos de Henadade, seus filhos e seus irmãos, os
levitas. Quando, pois, os edificadores lançaram os alicerces
do templo do Senhor, então apresentaram-se os sacerdotes, já
vestidos e com trombetas, e os levitas, filhos de Asafe, com
címbalos, para louvarem ao Senhor conforme à instituição de Davi,
rei de Israel. E cantavam juntos por grupo, louvando e
rendendo graças ao Senhor, dizendo: porque é bom; porque a sua
benignidade dura para sempre sobre Israel. E todo o povo jubilou com
altas vozes, quando louvaram ao Senhor, pela fundação da casa do
Senhor. Porém muitos dos sacerdotes, e levitas e chefes dos
pais, já idosos, que viram a primeira casa, choraram em altas vozes
quando à sua vista foram lançados os fundamentos desta casa; mas
muitos levantaram as vozes com júbilo e com alegria. De
maneira que não discernia o povo as vozes do júbilo de alegria das
vozes do choro do povo; porque o povo jubilava com tão altas vozes,
que o som se ouvia de muito longe.

\medskip

\lettrine{4} Ouvindo, pois, os adversários de Judá e Benjamim
que os que voltaram do cativeiro edificavam o templo ao Senhor Deus
de Israel, chegaram-se a Zorobabel e aos chefes dos pais, e
disseram-lhes: Deixai-nos edificar convosco, porque, como vós,
buscaremos a vosso Deus; como também já lhe sacrificamos desde os
dias de Esar-Hadom, rei da Assíria, que nos fez subir aqui.
Porém Zorobabel, e Jesuá, e os outros chefes dos pais de Israel
lhes disseram: Não convém que nós e vós edifiquemos casa a nosso
Deus; mas nós sozinhos a edificaremos ao Senhor Deus de Israel, como
nos ordenou o rei Ciro, rei da Pérsia. Todavia o povo da terra
debilitava as mãos do povo de Judá, e inquietava-os no edificar.
E alugaram contra eles conselheiros, para frustrarem o seu
plano, todos os dias de Ciro, rei da Pérsia, até ao reinado de
Dario, rei da Pérsia.

No reinado de Assuero, no princípio do seu reinado, escreveram uma
acusação contra os habitantes de Judá e de Jerusalém. E nos dias
de Artaxerxes escreveram Bislão, Mitredate, Tabeel, e os outros seus
companheiros, a Artaxerxes, rei da Pérsia; e a carta estava escrita
em caracteres siríacos, e na língua siríaca. Escreveram, pois,
Reum, o chanceler, e Sinsai, o escrivão, uma carta contra Jerusalém,
ao rei Artaxerxes do teor seguinte: Então escreveu Reum, o
chanceler, e Sinsai, o escrivão, e os outros seus companheiros, os
dinaítas, afarsaquitas, tarpelitas, afarsitas, arquevitas,
babilônios, susanquitas, deavitas, elamitas, e os outros
povos, que o grande e afamado Asnapar transportou, e que fez habitar
na cidade de Samaria, e nas demais províncias dalém do rio.
Este, pois, é o teor da carta que mandaram ao rei Artaxerxes:
Teus servos, os homens dalém do rio, em tal tempo. Saiba o
rei que os judeus, que subiram de ti, vieram a nós em Jerusalém, e
reedificam aquela rebelde e malvada cidade, e vão restaurando os
seus muros, e reparando os seus fundamentos. Agora saiba o
rei que, se aquela cidade se reedificar, e os muros se restaurarem,
eles não pagarão os direitos, os tributos e os pedágios; e assim se
danificará a fazenda dos reis. Agora, pois, porquanto somos
assalariados do palácio, e não nos convém ver a desonra do rei, por
isso mandamos avisar ao rei, para que se busque no livro das
crônicas de teus pais. E acharás no livro das crônicas, e saberás
que aquela foi uma cidade rebelde, e danosa aos reis e províncias, e
que nela houve rebelião em tempos antigos; por isso foi aquela
cidade destruída. Nós, pois, fazemos notório ao rei que, se
aquela cidade se reedificar, e os seus muros se restaurarem,
sucederá que não terás porção alguma deste lado do rio.

E o rei enviou esta resposta a Reum, o chanceler, e a Sinsai, o
escrivão, e aos demais seus companheiros, que habitavam em Samaria;
como também aos demais que estavam dalém do rio: Paz! em tal tempo.
A carta que nos enviastes foi explicitamente lida diante de
mim. E, ordenando-o eu, buscaram e acharam, que de tempos
antigos aquela cidade se levantou contra os reis, e nela se têm
feito rebelião e sedição. Também houve reis poderosos sobre
Jerusalém que dalém do rio dominaram em todo o lugar, e se lhes
pagaram direitos, tributos e pedágios. Agora, pois, dai ordem
para impedirdes aqueles homens, a fim de que não se edifique aquela
cidade, até que eu dê uma ordem. E guardai-vos de serdes
remissos nisto; por que cresceria o dano para prejuízo dos reis?
Então, depois que a cópia da carta do rei Artaxerxes foi lida
perante Reum, e Sinsai, o escrivão, e seus companheiros,
apressadamente foram eles a Jerusalém, aos judeus, e os impediram à
força e com violência. Então cessou a obra da casa de Deus,
que estava em Jerusalém; e cessou até ao ano segundo do reinado de
Dario, rei da Pérsia.

\medskip

\lettrine{5} E os profetas Ageu e Zacarias, filho de Ido,
profetizaram aos judeus que estavam em Judá, e em Jerusalém; em nome
do Deus de Israel lhes profetizaram. Então se levantaram
Zorobabel, filho de Sealtiel, e Jesuá, filho de Jozadaque, e
começaram a edificar a casa de Deus, que está em Jerusalém; e com
eles os profetas de Deus, que os ajudavam.

Naquele tempo vieram a eles Tatenai, governador dalém do rio, e
Setar-Bozenai, e os seus companheiros, e disseram-lhes assim: Quem
vos deu ordem para reedificardes esta casa, e restaurardes este
muro? Disseram-lhes, mais: E quais são os nomes dos homens que
construíram este edifício? Porém os olhos de Deus estavam sobre
os anciãos dos judeus, e não os impediram, até que o negócio
chegasse a Dario, e viesse resposta por carta sobre isso. Cópia
da carta que Tatenai, o governador dalém do rio, com Setar-Bozenai e
os seus companheiros, os afarsaquitas, que estavam dalém do rio,
enviaram ao rei Dario. Enviaram-lhe uma carta, na qual estava
escrito: Toda a paz ao rei Dario. Seja notório ao rei, que nós
fomos à província de Judá, à casa do grande Deus, a qual se edifica
com grandes pedras, e a madeira já está sendo posta nas paredes; e
esta obra vai sendo feita com diligência, e se adianta em suas mãos.
Então perguntamos aos anciãos, e assim lhes dissemos: Quem vos
deu ordem para reedificardes esta casa, e restaurardes este muro?
Além disso, lhes perguntamos também pelos seus nomes, para
tos declararmos; para que te pudéssemos escrever os nomes dos homens
que entre eles são os chefes. E esta foi a resposta que nos
deram: Nós somos servos do Deus dos céus e da terra, e reedificamos
a casa que há muitos anos foi edificada; porque um grande rei de
Israel a edificou e a terminou. Mas depois que nossos pais
provocaram à ira o Deus dos céus, ele os entregou nas mãos de
Nabucodonosor, rei de Babilônia, o caldeu, o qual destruiu esta
casa, e transportou o povo para Babilônia. Porém, no primeiro
ano de Ciro, rei de Babilônia, o rei Ciro deu ordem para que esta
casa de Deus se reedificasse. E até os utensílios de ouro e
prata, da casa de Deus, que Nabucodonosor tomou do templo que estava
em Jerusalém e os levou para o templo de Babilônia, o rei Ciro os
tirou do templo de Babilônia, e foram dados a um homem cujo nome era
Sesbazar, a quem nomeou governador. E disse-lhe: Toma estes
utensílios, vai e leva-os ao templo que está em Jerusalém, e faze
reedificar a casa de Deus, no seu lugar. Então veio este
Sesbazar, e pôs os fundamentos da casa de Deus, que está em
Jerusalém e desde então para cá se está reedificando, e ainda não
está acabada. Agora, pois, se parece bem ao rei, busque-se na
casa dos tesouros do rei, que está em Babilônia, se é verdade que se
deu uma ordem pelo rei Ciro para reedificar esta casa de Deus em
Jerusalém; e sobre isto nos faça saber a vontade do rei.

\medskip

\lettrine{6} Então o rei Dario deu ordem, e buscaram nos
arquivos, onde se guardavam os tesouros em Babilônia. E em
Acmeta, no palácio, que está na província de Média, se achou um
rolo, e nele estava escrito um memorial que dizia assim: No
primeiro ano do rei Ciro, este baixou o seguinte decreto: A casa de
Deus, em Jerusalém, se reedificará para lugar em que se ofereçam
sacrifícios, e seus fundamentos serão firmes; a sua altura de
sessenta côvados, e a sua largura de sessenta côvados; com três
carreiras de grandes pedras, e uma carreira de madeira nova; e a
despesa se fará da casa do rei. Além disso, os utensílios de
ouro e de prata da casa de Deus, que Nabucodonosor transportou do
templo que estava em Jerusalém, e levou para Babilônia, serão
restituídos, para que voltem ao seu lugar, ao templo que está em
Jerusalém, e serão postos na casa de Deus. Agora, pois, Tatenai,
governador dalém do rio, Setar-Bozenai, e os seus companheiros, os
afarsaquitas, que habitais dalém do rio, apartai-vos dali.
Deixai que se faça a obra desta casa de Deus; que o governador
dos judeus e os seus anciãos reedifiquem esta casa de Deus no seu
lugar. Também por mim se decreta o que haveis de fazer com os
anciãos dos judeus, para a reedificação desta casa de Deus, a saber:
que da fazenda do rei, dos tributos dalém do rio se pague
prontamente a despesa a estes homens, para que não interrompam a
obra. E o que for necessário, como bezerros, carneiros, e
cordeiros, para holocaustos ao Deus dos céus, trigo, sal, vinho e
azeite, segundo o rito dos sacerdotes que estão em Jerusalém,
dê-se-lhes, de dia em dia, para que não haja falta. Para que
ofereçam sacrifícios de cheiro suave ao Deus dos céus, e orem pela
vida do rei e de seus filhos. Também por mim se decreta que
todo o homem que mudar este decreto, se arrancará um madeiro da sua
casa, e, levantado, o pendurarão nele, e da sua casa se fará por
isso um monturo. O Deus, pois, que fez habitar ali o seu nome
derrube a todos os reis e povos que estenderem a sua mão para mudar
o decreto e para destruir esta casa de Deus, que está em Jerusalém.
Eu, Dario, baixei o decreto; com diligência se faça.

Então Tatenai, o governador dalém do rio, Setar-Bozenai e os seus
companheiros, assim fizeram diligentemente, conforme ao que
decretara o rei Dario. E os anciãos dos judeus iam edificando
e prosperando pela profecia do profeta Ageu, e de Zacarias, filho de
Ido. E edificaram e terminaram a obra conforme ao mandado do Deus de
Israel, e conforme ao decreto de Ciro e Dario, e de Artaxerxes, rei
da Pérsia. E acabou-se esta casa no terceiro dia do mês de
Adar, no sexto ano do reinado do rei Dario. E os filhos de
Israel, os sacerdotes, os levitas, e o restante dos filhos do
cativeiro, fizeram a dedicação desta casa de Deus com alegria.
E ofereceram para a dedicação desta casa de Deus cem
novilhos, duzentos carneiros, quatrocentos cordeiros, e doze
cabritos por expiação do pecado de todo o Israel; segundo o número
das tribos de Israel. E puseram os sacerdotes nas suas turmas
e os levitas nas suas divisões, para o ministério de Deus, em
Jerusalém, conforme ao que está escrito no livro de Moisés. E
os filhos do cativeiro celebraram a páscoa no dia catorze do
primeiro mês. Porque os sacerdotes e levitas se purificaram
como se fossem um só homem, todos estavam limpos; e mataram o
cordeiro da páscoa para todos os filhos do cativeiro, e para seus
irmãos, os sacerdotes, e para si mesmos. Assim comeram a
páscoa os filhos de Israel que tinham voltado do cativeiro, com
todos os que com eles se apartaram da imundícia dos gentios da
terra, para buscarem o Senhor Deus de Israel; e celebraram a
festa dos pães ázimos por sete dias com alegria; porque o Senhor os
tinha alegrado, e tinha mudado o coração do rei da Assíria a favor
deles, para lhes fortalecer as mãos na obra da casa de Deus, o Deus
de Israel.

\medskip

\lettrine{7} E passadas estas coisas no reinado de Artaxerxes,
rei da Pérsia, Esdras, filho de Seraías, filho de Azarias, filho de
Hilquias, filho de Salum, filho de Zadoque, filho de Aitube,
filho de Amarias, filho de Azarias, filho de Meraiote, filho
de Zeraquias, filho de Uzi, filho de Buqui, filho de Abisua,
filho de Finéias, filho de Eleazar, filho de Arão, o sumo sacerdote;
este Esdras subiu de Babilônia; e era escriba hábil na lei de
Moisés, que o Senhor Deus de Israel tinha dado; e, segundo a mão do
Senhor seu Deus, que estava sobre ele, o rei lhe deu tudo quanto lhe
pedira. Também subiram a Jerusalém alguns dos filhos de Israel,
dos sacerdotes, dos levitas, dos cantores, dos porteiros e dos
servidores do templo, no sétimo ano do rei Artaxerxes. E no
quinto mês chegou a Jerusalém, no sétimo ano deste rei. Pois no
primeiro dia do primeiro mês foi o princípio da partida de
Babilônia; e no primeiro dia do quinto mês chegou a Jerusalém,
segundo a boa mão do seu Deus sobre ele. Porque Esdras tinha
preparado o seu coração para buscar a lei do Senhor e para cumpri-la
e para ensinar em Israel os seus estatutos e os seus juízos.

Esta é, pois, a cópia da carta que o rei Artaxerxes deu ao
sacerdote Esdras, o escriba das palavras dos mandamentos do Senhor,
e dos seus estatutos sobre Israel: Artaxerxes, rei dos reis,
ao sacerdote Esdras, escriba da lei do Deus do céu, paz perfeita,
etc. Por mim se decreta que no meu reino todo aquele do povo
de Israel, e dos seus sacerdotes e levitas, que quiser ir contigo a
Jerusalém, vá. Porquanto és enviado da parte do rei e dos
seus sete conselheiros para fazeres inquirição a respeito de Judá e
de Jerusalém, conforme à lei do teu Deus, que está na tua mão;
e para levares a prata e o ouro que o rei e os seus
conselheiros voluntariamente deram ao Deus de Israel, cuja habitação
está em Jerusalém; e toda a prata e o ouro que achares em
toda a província de Babilônia, com as ofertas voluntárias do povo e
dos sacerdotes, que voluntariamente oferecerem, para a casa de seu
Deus, que está em Jerusalém. Portanto diligentemente
comprarás com este dinheiro novilhos, carneiros, cordeiros, com as
suas ofertas de alimentos, e as suas libações, e as oferecerás sobre
o altar da casa de vosso Deus, que está em Jerusalém. Também
o que a ti e a teus irmãos bem parecer fazerdes do restante da prata
e do ouro, o fareis conforme a vontade do vosso Deus. E os
utensílios que te foram dados para o serviço da casa de teu Deus,
restitui-os perante o Deus de Jerusalém. E tudo mais que for
necessário para a casa de teu Deus, que te convenha dar, dá-lo-ás da
casa dos tesouros do rei. E por mim mesmo, o rei Artaxerxes,
se decreta a todos os tesoureiros que estão dalém do rio que tudo
quanto vos pedir o sacerdote Esdras, escriba da lei do Deus dos
céus, prontamente se faça. Até cem talentos de prata, e até
cem coros de trigo, e até cem batos de vinho, e até cem batos de
azeite; e sal à vontade. Tudo quanto se ordenar, segundo o
mandado do Deus do céu, prontamente se faça para a casa do Deus do
céu; pois, para que haveria grande ira sobre o reino do rei e de
seus filhos? Também vos fazemos saber acerca de todos os
sacerdotes e levitas, cantores, porteiros, servidores do templo e
ministros desta casa de Deus, que não será lícito impor-lhes, nem
tributo, nem contribuição, nem renda. E tu, Esdras, conforme
a sabedoria do teu Deus, que possues, nomeia magistrados e juízes,
que julguem a todo o povo que está dalém do rio, a todos os que
sabem as leis do teu Deus; e ao que não as sabe, lhe ensinarás.
E todo aquele que não observar a lei do teu Deus e a lei do
rei, seja julgado prontamente; quer seja morte, quer desterro, quer
multa sobre os seus bens, quer prisão.

Bendito seja o Senhor Deus de nossos pais, que tal inspirou ao
coração do rei, para ornar a casa do Senhor, que está em Jerusalém.
E que estendeu para mim a sua benignidade perante o rei e os
seus conselheiros e todos os príncipes poderosos do rei. Assim me
animei, segundo a mão do Senhor meu Deus sobre mim, e ajuntei dentre
Israel alguns chefes para subirem comigo.

\medskip

\lettrine{8} Estes, pois, são os chefes das casas paternas e
esta a genealogia dos que subiram comigo de Babilônia no reinado do
rei Artaxerxes: Dos filhos de Finéias, Gérson; dos filhos de
Itamar, Daniel; dos filhos de Davi, Hatus; dos filhos de
Secanias, e dos filhos de Parós, Zacarias, e com ele, segundo a
genealogia, se contaram até cento e cinqüenta homens. Dos filhos
de Paate-Moabe, Elioenai, filho de Zacarias, e com ele duzentos
homens. Dos filhos de Secanias, o filho de Jeaziel, e com ele
trezentos homens. E dos filhos de Adim, Ebede, filho de Jônatas,
e com ele cinqüenta homens. E dos filhos de Elão, Jesaías, filho
de Atalias, e com ele setenta homens. E dos filhos de Sefatias,
Zebadias, filho de Micael, e com ele oitenta homens. Dos filhos
de Joabe, Obadias, filho de Jeiel, e com ele duzentos e dezoito
homens. E dos filhos de Selomite, o filho de Josifias, e com
ele cento e sessenta homens. E dos filhos de Bebai, Zacarias,
o filho de Bebai, e com ele vinte e oito homens. E dos filhos
de Azgade, Joanã, o filho de Hacatã, e com ele cento e dez homens.
E dos últimos filhos de Adonicão, cujos nomes eram estes:
Elifelete, Jeiel e Semaías, e com eles sessenta homens. E dos
filhos de Bigvai, Utai e Zabude, e com eles setenta homens. E
ajuntei-os perto do rio que vai a Aava, e ficamos ali acampados três
dias. Então atentei para o povo e para os sacerdotes, e não achei
ali nenhum dos filhos de Levi. Enviei, pois, Eliezer, Ariel,
Semaías, Elnatã, Jaribe, Elnatã, Natã, Zacarias e Mesulão, os
chefes; como também a Joiaribe, e a Elnatã, que eram entendidos.
E enviei-os com mandado a Ido, chefe em Casifia; e falei a
eles o que deveriam dizer a Ido e aos seus irmãos, servidores do
templo, em Casifia, que nos trouxessem ministros para a casa do
nosso Deus. E trouxeram-nos, segundo a boa mão de Deus sobre
nós, um homem entendido, dos filhos de Mali, filho de Levi, filho de
Israel, a saber: Serebias, com os seus filhos e irmãos, dezoito;
e a Hasabias, e com ele Jesaías, dos filhos de Merari, com
seus irmãos e os filhos deles, vinte; e dos servidores do
templo que Davi e os príncipes deram para o ministério dos levitas,
duzentos e vinte servidores do templo; que foram todos mencionados
por seus nomes.

Então apregoei\footnote{Apregoar: declarar em público; proclamar,
divulgar, publicar.} ali um jejum junto ao rio Aava, para nos
humilharmos diante da face de nosso Deus, para lhe pedirmos caminho
seguro para nós, para nossos filhos e para todos os nossos bens.
Porque tive vergonha de pedir ao rei, exército e cavaleiros
para nos defenderem do inimigo pelo caminho; porquanto tínhamos
falado ao rei, dizendo: A mão do nosso Deus é sobre todos os que o
buscam, para o bem deles; mas o seu poder e a sua ira contra todos
os que o deixam. Nós, pois, jejuamos, e pedimos isto ao nosso
Deus, e moveu-se pelas nossas orações.

Então separei doze dos chefes dos sacerdotes: Serebias, Hasabias,
e com eles dez dos seus irmãos. E pesei-lhes a prata, o ouro
e os vasos; que eram a oferta para a casa de nosso Deus, que
ofereceram o rei, os seus conselheiros, os seus príncipes e todo o
Israel que ali se achou. E pesei em suas mãos seiscentos e
cinqüenta talentos de prata, e em vasos de prata cem talentos, e cem
talentos de ouro, e vinte bacias de ouro, de mil dracmas, e
dois vasos de bom metal lustroso, tão precioso como ouro. E
disse-lhes: Vós sois santos ao Senhor, e são santos estes
utensílios, como também esta prata e este ouro, oferta voluntária,
oferecida ao Senhor Deus de vossos pais. Vigiai, pois, e
guardai-os até que os peseis na presença dos chefes dos sacerdotes e
dos levitas, e dos chefes dos pais de Israel, em Jerusalém, nas
câmaras da casa do Senhor. Então os sacerdotes e os levitas
receberam o peso da prata, do ouro e dos utensílios, para os
trazerem a Jerusalém, à casa de nosso Deus.

E partimos do rio Aava, no dia doze do primeiro mês, para irmos a
Jerusalém; e a mão do nosso Deus estava sobre nós, e livrou-nos da
mão dos inimigos, e dos que nos armavam ciladas pelo caminho.
E chegamos a Jerusalém, e repousamos ali três dias. E
no quarto dia se pesou a prata, o ouro e os utensílios, na casa do
nosso Deus, por mão de Meremote, filho do sacerdote Urias; e com ele
Eleazar, filho de Finéias, e com eles Jozabade, filho de Jesuá, e
Noadias, filho de Binui, levitas. Tudo foi contado e pesado;
e todo o peso foi registrado na mesma ocasião. E os exilados,
que vieram do cativeiro, ofereceram holocaustos ao Deus de Israel,
doze novilhos por todo o Israel, noventa e seis carneiros, setenta e
sete cordeiros, e doze bodes em sacrifício pelo pecado; tudo em
holocausto ao Senhor. Então deram as ordens do rei aos seus
sátrapas\footnote{Governador de província, na Pérsia antiga.}, e aos
governadores dalém do rio; e estes ajudaram o povo e a casa de Deus.

\medskip

\lettrine{9} Acabadas, pois, estas coisas, chegaram-se a mim
os príncipes, dizendo: O povo de Israel, os sacerdotes e os levitas,
não se têm separado dos povos destas terras, seguindo as abominações
dos cananeus, dos heteus, dos perizeus, dos jebuseus, dos amonitas,
dos moabitas, dos egípcios, e dos amorreus. Porque tomaram das
suas filhas para si e para seus filhos, e assim se misturou a
linhagem santa com os povos dessas terras; e até os príncipes e
magistrados foram os primeiros nesta transgressão. E, ouvindo eu
tal coisa, rasguei as minhas vestes e o meu manto, e arranquei os
cabelos da minha cabeça e da minha barba, e sentei-me atônito.
Então se ajuntaram a mim todos os que tremiam das palavras do
Deus de Israel por causa da transgressão dos do cativeiro; porém eu
permaneci sentado atônito até ao sacrifício da tarde.

E perto do sacrifício da tarde me levantei da minha aflição,
havendo já rasgado as minhas vestes e o meu manto, e me pus de
joelhos, e estendi as minhas mãos para o Senhor meu Deus; e
disse: Meu Deus! Estou confuso e envergonhado, para levantar a ti a
minha face, meu Deus; porque as nossas iniqüidades se multiplicaram
sobre a nossa cabeça, e a nossa culpa tem crescido até aos céus.
Desde os dias de nossos pais até ao dia de hoje estamos em
grande culpa, e por causa das nossas iniqüidades somos entregues,
nós e nossos reis e os nossos sacerdotes, na mão dos reis das
terras, à espada, ao cativeiro, e ao roubo, e à confusão do rosto,
como hoje se vê. E agora, por um pequeno momento, se manifestou
a graça da parte do Senhor, nosso Deus, para nos deixar alguns que
escapem, e para dar-nos uma estaca no seu santo lugar; para nos
iluminar os olhos, ó Deus nosso, e para nos dar um pouco de vida na
nossa servidão. Porque somos servos; porém na nossa servidão não
nos desamparou o nosso Deus; antes estendeu sobre nós a sua
benignidade perante os reis da Pérsia, para que nos desse vida, para
levantarmos a casa do nosso Deus, e para restaurarmos as suas
assolações; e para que nos desse uma parede de proteção em Judá e em
Jerusalém. Agora, pois, ó nosso Deus, que diremos depois
disto? Pois deixamos os teus mandamentos, os quais mandaste
pelo ministério de teus servos, os profetas, dizendo: A terra em que
entrais para a possuir, terra imunda é pelas imundícias dos povos
das terras, pelas suas abominações com que, na sua corrupção a
encheram, de uma extremidade à outra. Agora, pois, vossas
filhas não dareis a seus filhos, e suas filhas não tomareis para
vossos filhos, e nunca procurareis a sua paz e o seu bem; para que
sejais fortes, e comais o bem da terra, e a deixeis por herança a
vossos filhos para sempre. E depois de tudo o que nos tem
sucedido por causa das nossas más obras, e da nossa grande culpa,
porquanto tu, ó nosso Deus, impediste que fôssemos destruídos, por
causa da nossa iniqüidade, e ainda nos deste um remanescente como
este, tornaremos, pois, agora a violar os teus mandamentos e
a aparentar-nos com os povos destas abominações? Não te indignarias
tu assim contra nós até de todo nos consumir, até que não ficasse
remanescente nem quem escapasse? Ah! Senhor Deus de Israel,
justo és, pois ficamos qual um remanescente que escapou, como hoje
se vê; eis que estamos diante de ti, na nossa culpa, porque ninguém
há que possa estar na tua presença, por causa disto.

\medskip

\lettrine{10} E enquanto Esdras orava, e fazia confissão,
chorando e prostrando-se diante da casa de Deus, ajuntou-se a ele,
de Israel, uma grande congregação, de homens, mulheres e crianças;
pois o povo chorava com grande choro. Então Secanias, filho de
Jeiel, um dos filhos de Elão, tomou a palavra e disse a Esdras: Nós
temos transgredido contra o nosso Deus, e casamos com mulheres
estrangeiras dentre os povos da terra, mas, no tocante a isto, ainda
há esperança para Israel. Agora, pois, façamos aliança com o
nosso Deus de que despediremos todas as mulheres, e os que delas são
nascidos, conforme ao conselho do meu senhor, e dos que tremem ao
mandado do nosso Deus; e faça-se conforme a lei. Levanta-te,
pois, porque te pertence este negócio, e nós seremos contigo;
esforça-te, e age. Então Esdras se levantou, e ajuramentou os
chefes dos sacerdotes e dos levitas, e a todo o Israel, de que
fariam conforme a esta palavra; e eles juraram.

E Esdras se levantou de diante da casa de Deus, e entrou na câmara
de Joanã, filho de Eliasibe; e, chegando lá, não comeu pão, e nem
bebeu água; porque lamentava pela transgressão dos do cativeiro.
E fizeram passar pregão por Judá e Jerusalém, a todos os que
vieram do cativeiro, para que se ajuntassem em Jerusalém. E que
todo aquele que em três dias não viesse, segundo o conselho dos
príncipes e dos anciãos, toda a sua fazenda se poria em interdito, e
ele seria separado da congregação dos do cativeiro. Então todos
os homens de Judá e Benjamim em três dias se ajuntaram em Jerusalém;
era o nono mês, aos vinte dias do mês; e todo o povo se assentou na
praça da casa de Deus, tremendo por este negócio e por causa das
grandes chuvas. Então se levantou Esdras, o sacerdote, e
disse-lhes: Vós tendes transgredido, e casastes com mulheres
estrangeiras, aumentando a culpa de Israel. Agora, pois,
fazei confissão ao Senhor Deus de vossos pais, e fazei a sua
vontade; e apartai-vos dos povos das terras, e das mulheres
estrangeiras. E respondeu toda a congregação, e disse em
altas vozes: Assim seja, conforme às tuas palavras nos convém fazer.
Porém o povo é muito, e também é tempo de grandes chuvas, e
não se pode estar aqui fora; nem é obra de um dia nem de dois,
porque somos muitos os que transgredimos neste negócio. Ora,
ponham-se os nossos líderes, por toda a congregação sobre este
negócio; e todos os que em nossas cidades casaram com mulheres
estrangeiras venham em tempos apontados, e com eles os anciãos de
cada cidade, e os seus juízes, até que desviemos de nós o ardor da
ira do nosso Deus, por esta causa.

Porém, somente Jônatas, filho de Asael, e Jaseías, filho de
Ticva, se opuseram a isto; e Mesulão, e Sabetai, levita, os
ajudaram. E assim o fizeram os que voltaram do cativeiro; e o
sacerdote Esdras e os homens, chefes dos pais, segundo a casa de
seus pais, e todos pelos seus nomes, foram apontados; e
assentaram-se no primeiro dia do décimo mês, para inquirirem neste
negócio. E no primeiro dia do primeiro mês acabaram de tratar
com todos os homens que casaram com mulheres estrangeiras. E
acharam-se dos filhos dos sacerdotes que casaram com mulheres
estrangeiras: Dos filhos de Jesuá, filho de Jozadaque, e seus
irmãos, Maaséias, e Eliezer, e Jaribe, e Gedalias. E deram as
suas mãos prometendo que despediriam suas mulheres; e, achando-se
culpados, ofereceram um carneiro do rebanho pelo seu delito.
E dos filhos de Imer: Hanani e Zebadias. E dos filhos
de Harim: Maaséias, Elias, Semaías, Jeiel e Uzias. E dos
filhos de Pasur: Elioenai, Maaséias, Ismael, Netanel, Jozabade e
Eleasa. E dos levitas: Jozabade, Simei, Quelaías (este é
Quelita), Petaías, Judá e Eliezer. E dos cantores: Eliasibe;
e dos porteiros: Salum, Telém e Uri. E de Israel, dos filhos
de Parós: Ramias, Jezias, Malquias, Miamim, Eleazar, Malquias e
Benaia. E dos filhos de Elão: Matanias, Zacarias, Jeiel,
Abdi, Jeremote e Elias. E dos filhos de Zatu: Elioenai,
Eliasibe, Matanias, Jeremote, Zabade e Aziza. E dos filhos de
Bebai: Joanã, Hananias, Zabai e Atlai. E dos filhos de Bani:
Mesulão, Maluque, Adaías, Jasube, Seal, Jeremote. E dos
filhos de Paate-Moabe: Adna, Quelal, Benaia, Maséias, Matanias,
Bezalel, Binui e Manassés. E dos filhos de Harim: Eliezer,
Josias, Malquias, Semaías, Simeão, Benjamim, Maluque,
Semarias. Dos filhos de Hasum: Matenai, Matatá, Zabade,
Elifelete, Jeremai, Manassés e Simei. Dos filhos de Bani:
Maadai, Anrão, Uel, Benaia, Bedias, Queluí, Vanias,
Meremote, Eliasibe, Matanias, Matnai e Jaasai, e Bani,
Binui, Simei, e Selemias, Natã e Adaías, Macnadbai,
Sasai, Sarai, Azareel, Selemias, Semarias, Salum,
Amarias e José. Dos filhos de Nebo: Jeiel, Matitias, Zabade,
Zebina, Jadai, Joel e Benaia. Todos estes tomaram mulheres
estrangeiras; e alguns deles tinham mulheres de quem tiveram filhos.

