\addchap{Eclesiastes}

\lettrine{1} Palavras do pregador, filho de Davi, rei em
Jerusalém. Vaidade de vaidades, diz o pregador, vaidade de
vaidades! Tudo é vaidade. Que proveito tem o homem de todo o seu
trabalho, que faz debaixo do sol?

Uma geração vai, e outra geração vem; mas a terra para sempre
permanece. Nasce o sol, e o sol se põe, e apressa-se e volta ao
seu lugar de onde nasceu. O vento vai para o sul, e faz o seu
giro para o norte; continuamente vai girando o vento, e volta
fazendo os seus circuitos. Todos os rios vão para o mar, e
contudo o mar não se enche; ao lugar para onde os rios vão, para ali
tornam eles a correr. Todas as coisas são trabalhosas; o homem
não o pode exprimir; os olhos não se fartam de ver, nem os ouvidos
se enchem de ouvir.

O que foi, isso é o que há de ser; e o que se fez, isso se fará;
de modo que nada há de novo debaixo do sol. Há alguma coisa
de que se possa dizer: Vê, isto é novo? Já foi nos séculos passados,
que foram antes de nós. Já não há lembrança das coisas que
precederam, e das coisas que hão de ser também delas não haverá
lembrança, entre os que hão de vir depois.

Eu, o pregador, fui rei sobre Israel em Jerusalém. E
apliquei o meu coração a esquadrinhar, e a informar-me com sabedoria
de tudo quanto sucede debaixo do céu; esta enfadonha ocupação deu
Deus aos filhos dos homens, para nela os exercitar. Atentei
para todas as obras que se fazem debaixo do sol, e eis que tudo era
vaidade e aflição de espírito. Aquilo que é torto não se pode
endireitar; aquilo que falta não se pode calcular. Falei eu
com o meu coração, dizendo: Eis que eu me engrandeci, e sobrepujei
em sabedoria a todos os que houve antes de mim em Jerusalém; e o meu
coração contemplou abundantemente a sabedoria e o conhecimento.
E apliquei o meu coração a conhecer a sabedoria e a conhecer
os desvarios e as loucuras, e vim a saber que também isto era
aflição de espírito. Porque na muita sabedoria há muito
enfado; e o que aumenta em conhecimento, aumenta em dor.

\medskip

\lettrine{2} Disse eu no meu coração: Ora vem, eu te provarei
com alegria; portanto goza o prazer; mas eis que também isso era
vaidade. Ao riso disse: Está doido; e da alegria: De que serve
esta? Busquei no meu coração como estimular com vinho a minha
carne (regendo porém o meu coração com sabedoria), e entregar-me à
loucura, até ver o que seria melhor que os filhos dos homens
fizessem debaixo do céu durante o número dos dias de sua vida.
Fiz para mim obras magníficas; edifiquei para mim casas; plantei
para mim vinhas. Fiz para mim hortas e jardins, e plantei neles
árvores de toda a espécie de fruto. Fiz para mim tanques de
águas, para regar com eles o bosque em que reverdeciam as árvores.
Adquiri servos e servas, e tive servos nascidos em casa; também
tive grandes possessões de gados e ovelhas, mais do que todos os que
houve antes de mim em Jerusalém. Amontoei também para mim prata
e ouro, e tesouros dos reis e das províncias; provi-me de cantores e
cantoras, e das delícias dos filhos dos homens; e de instrumentos de
música de toda a espécie. E fui engrandecido, e aumentei mais do
que todos os que houve antes de mim em Jerusalém; perseverou também
comigo a minha sabedoria. E tudo quanto desejaram os meus
olhos não lhes neguei, nem privei o meu coração de alegria alguma;
mas o meu coração se alegrou por todo o meu trabalho, e esta foi a
minha porção de todo o meu trabalho. E olhei eu para todas as
obras que fizeram as minhas mãos, como também para o trabalho que
eu, trabalhando, tinha feito, e eis que tudo era vaidade e aflição
de espírito, e que proveito nenhum havia debaixo do sol.

Então passei a contemplar a sabedoria, e a loucura e a
estultícia. Pois que fará o homem que seguir ao rei? O mesmo que
outros já fizeram. Então vi eu que a sabedoria é mais
excelente do que a estultícia, quanto a luz é mais excelente do que
as trevas. Os olhos do homem sábio estão na sua cabeça, mas o
louco anda em trevas; então também entendi eu que o mesmo lhes
sucede a ambos. Assim eu disse no meu coração: Como acontece
ao tolo, assim me sucederá a mim; por que então busquei eu mais a
sabedoria? Então disse no meu coração que também isto era vaidade.
Porque nunca haverá mais lembrança do sábio do que do tolo;
porquanto de tudo, nos dias futuros, total esquecimento haverá. E
como morre o sábio, assim morre o tolo!

Por isso odiei esta vida, porque a obra que se faz debaixo do sol
me era penosa; sim, tudo é vaidade e aflição de espírito.
Também eu odiei todo o meu trabalho, que realizei debaixo do
sol, visto que eu havia de deixá-lo ao homem que viesse depois de
mim. E quem sabe se será sábio ou tolo? Todavia, se
assenhoreará de todo o meu trabalho que realizei e em que me houve
sabiamente debaixo do sol; também isto é vaidade. Então eu me
volvi e entreguei o meu coração ao desespero no tocante ao trabalho,
o qual realizei debaixo do sol. Porque há homem cujo trabalho
é feito com sabedoria, conhecimento, e destreza; contudo deixará o
seu trabalho como porção de quem nele não trabalhou; também isto é
vaidade e grande mal. Porque, que mais tem o homem de todo o
seu trabalho, e da aflição do seu coração, em que ele anda
trabalhando debaixo do sol? Porque todos os seus dias são
dores, e a sua ocupação é aflição; até de noite não descansa o seu
coração; também isto é vaidade. Não há nada melhor para o
homem do que comer e beber, e fazer com que sua alma goze do bem do
seu trabalho. Também vi que isto vem da mão de Deus. Pois
quem pode comer, ou quem pode gozar melhor do que eu? Porque
ao homem que é bom diante dele, dá Deus sabedoria e conhecimento e
alegria; mas ao pecador dá trabalho, para que ele ajunte, e amontoe,
para dá-lo ao que é bom perante Deus. Também isto é vaidade e
aflição de espírito.

\medskip

\lettrine{3} Tudo tem o seu tempo determinado, e há tempo para
todo o propósito debaixo do céu. Há tempo de nascer, e tempo de
morrer; tempo de plantar, e tempo de arrancar o que se plantou;
tempo de matar, e tempo de curar; tempo de derrubar, e tempo de
edificar; tempo de chorar, e tempo de rir; tempo de prantear, e
tempo de dançar; tempo de espalhar pedras, e tempo de ajuntar
pedras; tempo de abraçar, e tempo de afastar-se de abraçar;
tempo de buscar, e tempo de perder; tempo de guardar, e tempo de
lançar fora; tempo de rasgar, e tempo de coser; tempo de estar
calado, e tempo de falar; tempo de amar, e tempo de odiar; tempo
de guerra, e tempo de paz. Que proveito tem o trabalhador
naquilo em que trabalha? Tenho visto o trabalho que Deus deu
aos filhos dos homens, para com ele os exercitar.

Tudo fez formoso em seu tempo; também pôs o mundo no coração do
homem, sem que este possa descobrir a obra que Deus fez desde o
princípio até ao fim. Já tenho entendido que não há coisa
melhor para eles do que alegrar-se e fazer bem na sua vida; e
também que todo o homem coma e beba, e goze do bem de todo o seu
trabalho; isto é um dom de Deus. Eu sei que tudo quanto Deus
faz durará eternamente; nada se lhe deve acrescentar, e nada se lhe
deve tirar; e isto faz Deus para que haja temor diante dele.
O que é, já foi; e o que há de ser, também já foi; e Deus
pede conta do que passou.

Vi mais debaixo do sol que no lugar do juízo havia impiedade, e
no lugar da justiça havia iniqüidade. Eu disse no meu
coração: Deus julgará o justo e o ímpio; porque há um tempo para
todo o propósito e para toda a obra. Disse eu no meu coração:
quanto à condição dos filhos dos homens, que Deus os provaria, para
que assim pudessem ver que são em si mesmos como os animais.
Porque o que sucede aos filhos dos homens, isso mesmo também
sucede aos animais, e lhes sucede a mesma coisa; como morre um,
assim morre o outro; e todos têm o mesmo fôlego, e a vantagem dos
homens sobre os animais não é nenhuma, porque todos são vaidade.
Todos vão para um lugar; todos foram feitos do pó, e todos
voltarão ao pó. Quem sabe que o fôlego do homem vai para
cima, e que o fôlego dos animais vai para baixo da terra?
Assim que tenho visto que não há coisa melhor do que
alegrar-se o homem nas suas obras, porque essa é a sua porção; pois
quem o fará voltar para ver o que será depois dele?

\medskip

\lettrine{4} Depois voltei-me, e atentei para todas as
opressões que se fazem debaixo do sol; e eis que vi as lágrimas dos
que foram oprimidos e dos que não têm consolador, e a força estava
do lado dos seus opressores; mas eles não tinham consolador.

Por isso eu louvei os que já morreram, mais do que os que vivem
ainda. E melhor que uns e outros é aquele que ainda não é; que
não viu as más obras que se fazem debaixo do sol.

Também vi eu que todo o trabalho, e toda a destreza em obras, traz
ao homem a inveja do seu próximo. Também isto é vaidade e aflição de
espírito. O tolo cruza as suas mãos, e come a sua própria carne.
Melhor é a mão cheia com descanso do que ambas as mãos cheias
com trabalho, e aflição de espírito.

Outra vez me voltei, e vi vaidade debaixo do sol. Há um que é
só, e não tem ninguém, nem tampouco filho nem irmão; e contudo não
cessa do seu trabalho, e também seus olhos não se satisfazem com
riqueza; nem diz: Para quem trabalho eu, privando a minha alma do
bem? Também isto é vaidade e enfadonha ocupação. Melhor é serem
dois do que um, porque têm melhor paga do seu trabalho.
Porque se um cair, o outro levanta o seu companheiro; mas ai
do que estiver só; pois, caindo, não haverá outro que o levante.
Também, se dois dormirem juntos, eles se aquentarão; mas um
só, como se aquentará? E, se alguém prevalecer contra um, os
dois lhe resistirão; e o cordão de três dobras não se quebra tão
depressa.

Melhor é a criança pobre e sábia do que o rei velho e insensato,
que não se deixa mais admoestar. Porque um sai do cárcere
para reinar; enquanto outro, que nasceu em seu reino, torna-se
pobre. Vi a todos os viventes andarem debaixo do sol com a
criança, a sucessora, que ficará no seu lugar. Não tem fim
todo o povo que foi antes dele; tampouco os que lhe sucederem se
alegrarão dele. Na verdade que também isto é vaidade e aflição de
espírito.

\medskip

\lettrine{5} Guarda o teu pé, quando entrares na casa de Deus;
porque chegar-se para ouvir é melhor do que oferecer sacrifícios de
tolos, pois não sabem que fazem mal. Não te precipites com a tua
boca, nem o teu coração se apresse a pronunciar palavra alguma
diante de Deus; porque Deus está nos céus, e tu estás sobre a terra;
assim sejam poucas as tuas palavras. Porque, da muita ocupação
vêm os sonhos, e a voz do tolo da multidão das palavras.

Quando a Deus fizeres algum voto, não tardes em cumpri-lo; porque
não se agrada de tolos; o que votares, paga-o. Melhor é que não
votes do que votares e não cumprires. Não consintas que a tua
boca faça pecar a tua carne, nem digas diante do anjo que foi erro;
por que razão se iraria Deus contra a tua voz, e destruiria a obra
das tuas mãos? Porque, como na multidão dos sonhos há vaidades,
assim também nas muitas palavras; mas tu teme a Deus. Se vires
em alguma província opressão do pobre, e violência do direito e da
justiça, não te admires de tal procedimento; pois quem está
altamente colocado tem superior que o vigia; e há mais altos do que
eles.

O proveito da terra é para todos; até o rei se serve do campo.
Quem amar o dinheiro jamais dele se fartará; e quem amar a
abundância nunca se fartará da renda; também isto é vaidade.
Onde os bens se multiplicam, ali se multiplicam também os que
deles comem; que mais proveito, pois, têm os seus donos do que os
ver com os seus olhos? Doce é o sono do trabalhador, quer
coma pouco quer muito; mas a fartura do rico não o deixa dormir.
Há um grave mal que vi debaixo do sol, e atrai enfermidades:
as riquezas que os seus donos guardam para o seu próprio dano;
porque as mesmas riquezas se perdem por qualquer má ventura,
e havendo algum filho nada lhe fica na sua mão. Como saiu do
ventre de sua mãe, assim nu tornará, indo-se como veio; e nada
tomará do seu trabalho, que possa levar na sua mão. Assim que
também isto é um grave mal que, justamente como veio, assim há de
ir; e que proveito lhe vem de trabalhar para o vento, e de
haver comido todos os seus dias nas trevas, e de haver padecido
muito enfado, e enfermidade, e furor?

Eis aqui o que eu vi, uma boa e bela coisa: comer e beber, e
gozar cada um do bem de todo o seu trabalho, em que trabalhou
debaixo do sol, todos os dias de vida que Deus lhe deu, porque esta
é a sua porção. E a todo o homem, a quem Deus deu riquezas e
bens, e lhe deu poder para delas comer e tomar a sua porção, e gozar
do seu trabalho, isto é dom de Deus. Porque não se lembrará
muito dos dias da sua vida; porquanto Deus lhe enche de alegria o
seu coração.

\medskip

\lettrine{6} Há um mal que tenho visto debaixo do sol, e é mui
freqüente entre os homens: Um homem a quem Deus deu riquezas,
bens e honra, e nada lhe falta de tudo quanto a sua alma deseja, e
Deus não lhe dá poder para daí comer, antes o estranho lho come;
também isto é vaidade e má enfermidade. Se o homem gerar cem
filhos, e viver muitos anos, e os dias dos seus anos forem muitos, e
se a sua alma não se fartar do bem, e além disso não tiver
sepultura, digo que um aborto é melhor do que ele. Porquanto
debalde veio, e em trevas se vai, e de trevas se cobre o seu nome.
E ainda que nunca viu o sol, nem conheceu nada, mais descanso
tem este do que aquele. E, ainda que vivesse duas vezes mil anos
e não gozasse o bem, não vão todos para um mesmo lugar?

Todo o trabalho do homem é para a sua boca, e contudo nunca se
satisfaz o seu espírito. Porque, que mais tem o sábio do que o
tolo? E que mais tem o pobre que sabe andar perante os vivos?
Melhor é a vista dos olhos do que o vaguear da cobiça; também
isto é vaidade e aflição de espírito. Seja qualquer o que
for, já o seu nome foi nomeado, e sabe-se que é homem, e que não
pode contender com o que é mais forte do que ele.

Na verdade que há muitas coisas que multiplicam a vaidade; que
mais tem o homem de melhor? Pois, quem sabe o que é bom nesta
vida para o homem, por todos os dias da sua vida de vaidade, os
quais gasta como sombra? Quem declarará ao homem o que será depois
dele debaixo do sol?

\medskip

\lettrine{7} Melhor é a boa fama do que o melhor ungüento, e o
dia da morte do que o dia do nascimento de alguém. Melhor é ir à
casa onde há luto do que ir à casa onde há banquete, porque naquela
está o fim de todos os homens, e os vivos o aplicam ao seu coração.
Melhor é a mágoa do que o riso, porque com a tristeza do rosto
se faz melhor o coração. O coração dos sábios está na casa do
luto, mas o coração dos tolos na casa da alegria. Melhor é ouvir
a repreensão do sábio, do que ouvir alguém a canção do tolo.
Porque qual o crepitar dos espinhos debaixo de uma panela, tal é
o riso do tolo; também isto é vaidade.

Verdadeiramente que a opressão faria endoidecer até ao sábio, e o
suborno corrompe o coração. Melhor é o fim das coisas do que o
princípio delas; melhor é o paciente de espírito do que o altivo de
espírito. Não te apresses no teu espírito a irar-te, porque a
ira repousa no íntimo dos tolos. Nunca digas: Por que foram
os dias passados melhores do que estes? Porque não provém da
sabedoria esta pergunta.

Tão boa é a sabedoria como a herança, e dela tiram proveito os
que vêem o sol. Porque a sabedoria serve de defesa, como de
defesa serve o dinheiro; mas a excelência do conhecimento é que a
sabedoria dá vida ao seu possuidor. Atenta para a obra de
Deus; porque quem poderá endireitar o que ele fez torto? No
dia da prosperidade goza do bem, mas no dia da adversidade
considera; porque também Deus fez a este em oposição àquele, para
que o homem nada descubra do que há de vir depois dele. Tudo
isto vi nos dias da minha vaidade: há justo que perece na sua
justiça, e há ímpio que prolonga os seus dias na sua maldade.
Não sejas demasiadamente justo, nem demasiadamente sábio; por
que te destruirias a ti mesmo? Não sejas demasiadamente
ímpio, nem sejas louco; por que morrerias fora de teu tempo?
Bom é que retenhas isto, e também daquilo não retires a tua
mão; porque quem teme a Deus escapa de tudo isso. A sabedoria
fortalece ao sábio, mais do que dez poderosos que haja na cidade.
Na verdade que não há homem justo sobre a terra, que faça o
bem, e nunca peque. Tampouco apliques o teu coração a todas
as palavras que se disserem, para que não venhas a ouvir o teu servo
amaldiçoar-te. Porque o teu coração também já confessou que
muitas vezes tu amaldiçoaste a outros.

Tudo isto provei-o pela sabedoria; eu disse: Sabedoria
adquirirei; mas ela ainda estava longe de mim. O que já
sucedeu é remoto e profundíssimo; quem o achará? Eu apliquei
o meu coração para saber, e inquirir, e buscar a sabedoria e a razão
das coisas, e para conhecer que a impiedade é insensatez e que a
estultícia é loucura. E eu achei uma coisa mais amarga do que
a morte, a mulher cujo coração são redes e laços, e cujas mãos são
ataduras; quem for bom diante de Deus escapará dela, mas o pecador
virá a ser preso por ela. Vedes aqui, isto achei, diz o
pregador, conferindo uma coisa com a outra para achar a razão delas;
a qual ainda busca a minha alma, porém ainda não a achei; um
homem entre mil achei eu, mas uma mulher entre todas estas não
achei. Eis aqui, o que tão-somente achei: que Deus fez ao
homem reto, porém eles buscaram muitas astúcias.

\medskip

\lettrine{8} Quem é como o sábio? E quem sabe a interpretação
das coisas? A sabedoria do homem faz brilhar o seu rosto, e a dureza
do seu rosto se muda. Eu digo: Observa o mandamento do rei, e
isso em consideração ao juramento que fizeste a Deus. Não te
apresses a sair da presença dele, nem persistas em alguma coisa má,
porque ele faz tudo o que quer. Porque a palavra do rei tem
poder; e quem lhe dirá: Que fazes? Quem guardar o mandamento não
experimentará nenhum mal; e o coração do sábio discernirá o tempo e
o juízo.

Porque para todo o propósito há seu tempo e juízo; porquanto a
miséria do homem pesa sobre ele. Porque não sabe o que há de
suceder, e quando há de ser, quem lho dará a entender? Nenhum
homem há que tenha domínio sobre o espírito, para o reter; nem
tampouco tem ele poder sobre o dia da morte; como também não há
licença nesta peleja; nem tampouco a impiedade livrará aos ímpios.

Tudo isto vi quando apliquei o meu coração a toda a obra que se
faz debaixo do sol; tempo há em que um homem tem domínio sobre outro
homem, para desgraça sua. Assim também vi os ímpios, quando
os sepultavam; e eles entravam, e saíam do lugar santo; e foram
esquecidos na cidade, em que assim fizeram; também isso é vaidade.
Porquanto não se executa logo o juízo sobre a má obra, por
isso o coração dos filhos dos homens está inteiramente disposto para
fazer o mal. Ainda que o pecador faça o mal cem vezes, e os
dias se lhe prolonguem, contudo eu sei com certeza, que bem sucede
aos que temem a Deus, aos que temem diante dele. Porém o
ímpio não irá bem, e ele não prolongará os seus dias, que são como a
sombra; porque ele não teme diante de Deus.

Ainda há outra vaidade que se faz sobre a terra: que há justos a
quem sucede segundo as obras dos ímpios, e há ímpios a quem sucede
segundo as obras dos justos. Digo que também isto é vaidade.
Então louvei eu a alegria, porquanto para o homem nada há
melhor debaixo do sol do que comer, beber e alegrar-se; porque isso
o acompanhará no seu trabalho nos dias da sua vida que Deus lhe dá
debaixo do sol. Aplicando eu o meu coração a conhecer a
sabedoria, e a ver o trabalho que há sobre a terra (que nem de dia
nem de noite vê o homem sono nos seus olhos); então vi toda a
obra de Deus, que o homem não pode perceber, a obra que se faz
debaixo do sol, por mais que trabalhe o homem para a descobrir, não
a achará; e, ainda que diga o sábio que a conhece, nem por isso a
poderá compreender.

\medskip

\lettrine{9} Deveras todas estas coisas considerei no meu
coração, para declarar tudo isto: que os justos, e os sábios, e as
suas obras, estão nas mãos de Deus, e também o homem não conhece nem
o amor nem o ódio; tudo passa perante ele. Tudo sucede
igualmente a todos; o mesmo sucede ao justo e ao ímpio, ao bom e ao
puro, como ao impuro; assim ao que sacrifica como ao que não
sacrifica; assim ao bom como ao pecador; ao que jura como ao que
teme o juramento. Este é o mal que há entre tudo quanto se faz
debaixo do sol: a todos sucede o mesmo; e que também o coração dos
filhos dos homens está cheio de maldade, e que há desvarios no seu
coração enquanto vivem, e depois se vão aos mortos.

Ora, para aquele que está entre os vivos há esperança (porque
melhor é o cão vivo do que o leão morto). Porque os vivos sabem
que hão de morrer, mas os mortos não sabem coisa nenhuma, nem
tampouco terão eles recompensa, mas a sua memória fica entregue ao
esquecimento. Também o seu amor, o seu ódio, e a sua inveja já
pereceram, e já não têm parte alguma para sempre, em coisa alguma do
que se faz debaixo do sol. Vai, pois, come com alegria o teu pão
e bebe com coração contente o teu vinho, pois já Deus se agrada das
tuas obras. Em todo o tempo sejam alvas as tuas roupas, e nunca
falte o óleo sobre a tua cabeça. Goza a vida com a mulher que
amas, todos os dias da tua vida vã, os quais Deus te deu debaixo do
sol, todos os dias da tua vaidade; porque esta é a tua porção nesta
vida, e no teu trabalho, que tu fizeste debaixo do sol. Tudo
quanto te vier à mão para fazer, faze-o conforme as tuas forças,
porque na sepultura, para onde tu vais, não há obra nem projeto, nem
conhecimento, nem sabedoria alguma.

Voltei-me, e vi debaixo do sol que não é dos ligeiros a carreira,
nem dos fortes a batalha, nem tampouco dos sábios o pão, nem
tampouco dos prudentes as riquezas, nem tampouco dos entendidos o
favor, mas que o tempo e a oportunidade ocorrem a todos. Que
também o homem não sabe o seu tempo; assim como os peixes que se
pescam com a rede maligna, e como os passarinhos que se prendem com
o laço, assim se enlaçam também os filhos dos homens no mau tempo,
quando cai de repente sobre eles.

Também vi esta sabedoria debaixo do sol, que para mim foi grande:
Houve uma pequena cidade em que havia poucos homens, e veio
contra ela um grande rei, e a cercou e levantou contra ela grandes
baluartes\footnote{Bastião: obra de fortificação constituída de um
avançado para artilharia com dois flancos e duas faces ligadas às
cortinas da fortaleza ou praça por dois dos seus lados. Fortaleza
inexpugnável; local absolutamente seguro. Sustentáculo, alicerce,
base.}; e encontrou-se nela um sábio pobre, que livrou aquela
cidade pela sua sabedoria, e ninguém se lembrava daquele pobre
homem. Então disse eu: Melhor é a sabedoria do que a força,
ainda que a sabedoria do pobre foi desprezada, e as suas palavras
não foram ouvidas. As palavras dos sábios devem em silêncio
ser ouvidas, mais do que o clamor do que domina entre os tolos.
Melhor é a sabedoria do que as armas de guerra, porém um só
pecador destrói muitos bens.

\medskip

\lettrine{10} Assim como as moscas mortas fazem exalar mau
cheiro e inutilizar o ungüento do perfumador, assim é, para o famoso
em sabedoria e em honra, um pouco de estultícia. O coração do
sábio está à sua direita, mas o coração do tolo está à sua esquerda.
E, até quando o tolo vai pelo caminho, falta-lhe o seu
entendimento e diz a todos que é tolo.

Levantando-se contra ti o espírito do governador, não deixes o teu
lugar, porque a submissão é um remédio que aplaca grandes ofensas.
Ainda há um mal que vi debaixo do sol, como o erro que procede
do governador. A estultícia está posta em grandes alturas, mas
os ricos estão assentados em lugar baixo. Vi os servos a cavalo,
e os príncipes andando sobre a terra como servos. Quem abrir uma
cova, nela cairá, e quem romper um muro, uma cobra o morderá.
Aquele que transporta pedras, será maltratado por elas, e o que
rachar lenha expõe-se ao perigo. Se estiver embotado o ferro,
e não se afiar o corte, então se deve redobrar a força; mas a
sabedoria é excelente para dirigir. Seguramente a serpente
morderá antes de estar encantada, e o falador não é melhor.

Nas palavras da boca do sábio há favor, porém os lábios do tolo o
devoram. O princípio das palavras da sua boca é a estultícia,
e o fim do seu falar um desvario\footnote{Insanidade mental;
demência, loucura. Falta de acerto; delírio. Comportamento insensato
e extravagante; excesso.} péssimo. O tolo multiplica as
palavras, porém, o homem não sabe o que será; e quem lhe fará saber
o que será depois dele? O trabalho dos tolos a cada um deles
fatiga, porque não sabem como ir à cidade.

Ai de ti, ó terra, quando seu rei é uma criança, e cujos
príncipes comem de manhã. Bem-aventurada tu, ó terra, quando
seu rei é filho dos nobres, e seus príncipes comem a tempo, para se
fortalecerem, e não para bebedice. Por muita preguiça se
enfraquece o teto, e pela frouxidão das mãos a casa goteja.
Para rir se fazem banquetes, e o vinho produz alegria, e por
tudo o dinheiro responde. Nem ainda no teu pensamento
amaldiçoes ao rei, nem tampouco no mais interior da tua
recâmara\footnote{Quarto oculto; local onde se guardam roupas;
recanto.} amaldiçoes ao rico; porque as aves dos céus levariam a
voz, e os que têm asas dariam notícia do assunto.

\medskip

\lettrine{11} Lança o teu pão sobre as águas, porque depois de
muitos dias o acharás. Reparte com sete, e ainda até com oito,
porque não sabes que mal haverá sobre a terra. Estando as nuvens
cheias, derramam a chuva sobre a terra, e caindo a árvore para o
sul, ou para o norte, no lugar em que a árvore cair ali ficará.
Quem observa o vento, nunca semeará, e o que olha para as nuvens
nunca segará. Assim como tu não sabes qual o caminho do vento,
nem como se formam os ossos no ventre da mulher grávida, assim
também não sabes as obras de Deus, que faz todas as coisas. Pela
manhã semeia a tua semente, e à tarde não retires a tua mão, porque
tu não sabes qual prosperará, se esta, se aquela, ou se ambas serão
igualmente boas.

Certamente suave é a luz, e agradável é aos olhos ver o sol.
Porém, se o homem viver muitos anos, e em todos eles se alegrar,
também se deve lembrar dos dias das trevas, porque hão de ser
muitos. Tudo quanto sucede é vaidade. Alegra-te, jovem, na tua
mocidade, e recreie-se o teu coração nos dias da tua mocidade, e
anda pelos caminhos do teu coração, e pela vista dos teus olhos;
sabe, porém, que por todas estas coisas te trará Deus a juízo.
Afasta, pois, a ira do teu coração, e remove da tua carne o
mal, porque a adolescência e a juventude são vaidade.

\medskip

\lettrine{12} Lembra-te também do teu Criador nos dias da tua
mocidade, antes que venham os maus dias, e cheguem os anos dos quais
venhas a dizer: Não tenho neles contentamento; antes que se
escureçam o sol, e a luz, e a lua, e as estrelas, e tornem a vir as
nuvens depois da chuva; no dia em que tremerem os guardas da
casa, e se encurvarem os homens fortes, e cessarem os moedores, por
já serem poucos, e se escurecerem os que olham pelas janelas; e
as portas da rua se fecharem por causa do baixo ruído da moedura, e
se levantar à voz das aves, e todas as filhas da música se abaterem.
Como também quando temerem o que é alto, e houver espantos no
caminho, e florescer a amendoeira, e o gafanhoto for um peso, e
perecer o apetite; porque o homem se vai à sua casa eterna, e os
pranteadores andarão rodeando pela praça; antes que se rompa o
cordão de prata, e se quebre o copo de ouro, e se despedace o
cântaro junto à fonte, e se quebre a roda junto ao poço, e o pó
volte à terra, como o era, e o espírito volte a Deus, que o deu.

Vaidade de vaidades, diz o  Pregador, tudo é vaidade. E,
quanto mais sábio foi o pregador, tanto mais ensinou ao povo
sabedoria; e atentando, e esquadrinhando, compôs muitos provérbios.
Procurou o pregador achar palavras agradáveis; e escreveu-as
com retidão, palavras de verdade. As palavras dos sábios são
como aguilhões, e como pregos, bem fixados pelos mestres das
assembléias, que nos foram dadas pelo único Pastor. E, demais
disto, filho meu, atenta: não há limite para fazer livros, e o muito
estudar é enfado da carne.

De tudo o que se tem ouvido, o fim é: Teme a Deus, e guarda os
seus mandamentos; porque isto é o dever de todo o homem.
Porque Deus há de trazer a juízo toda a obra, e até tudo o
que está encoberto, quer seja bom, quer seja mau.

