\addchap{Lamentações de Jeremias}

\lettrine{1} Como está sentada solitária aquela cidade, antes
tão populosa! Tornou-se como viúva, a que era grande entre as
nações! A que era princesa entre as províncias, tornou-se
tributária! Chora amargamente de noite, e as suas lágrimas lhe
correm pelas faces; não tem quem a console entre todos os seus
amantes; todos os seus amigos se houveram
aleivosamente\footnote{Aleivosia: Traição, perfídia, deslealdade.
Dolo, fraude.  Falsa acusação; calúnia.} com ela, tornaram-se seus
inimigos. Judá passou em cativeiro por causa da aflição, e por
causa da grande servidão; ela habita entre os gentios, não acha
descanso; todos os seus perseguidores a alcançam entre as suas
dificuldades. Os caminhos de Sião pranteiam, porque não há quem
venha à festa solene; todas as suas portas estão desoladas; os seus
sacerdotes suspiram; as suas virgens estão tristes, e ela mesma tem
amargura. Os seus adversários têm sido feitos chefes, os seus
inimigos prosperam; porque o Senhor  a afligiu, por causa da
multidão das suas transgressões; os seus filhinhos foram para o
cativeiro na frente do adversário. E da filha de Sião já se foi
toda a sua formosura; os seus príncipes ficaram sendo como corços
que não acham pasto e caminham sem força adiante do perseguidor.
Lembra-se Jerusalém, nos dias da sua aflição e dos seus exílios,
de todas as suas mais queridas coisas, que tivera desde os tempos
antigos; quando caía o seu povo na mão do adversário, e não havia
quem a socorresse; os adversários a viram, e fizeram escárnio da sua
ruína. Jerusalém gravemente pecou, por isso se fez errante;
todos os que a honravam, a desprezaram, porque viram a sua nudez;
ela também suspira e volta para trás. A sua imundícia está nas
suas saias; nunca se lembrou do seu fim; por isso foi pasmosamente
abatida, não tem consolador; vê, Senhor, a minha aflição, porque o
inimigo se tem engrandecido. Estendeu o adversário a sua mão
a todas as coisas mais preciosas dela; pois ela viu entrar no seu
santuário os gentios, acerca dos quais mandaste que não entrassem na
tua congregação. Todo o seu povo anda suspirando, buscando o
pão; deram as suas coisas mais preciosas a troco de mantimento para
restaurarem a alma; vê, Senhor, e contempla, que sou desprezível.

Não vos comove isto a todos vós que passais pelo caminho?
Atendei, e vede, se há dor como a minha dor, que veio sobre mim, com
que o Senhor  me afligiu, no dia do furor da sua ira. Desde o
alto enviou fogo a meus ossos, o qual se assenhoreou deles; estendeu
uma rede aos meus pés, fez-me voltar para trás, fez-me assolada e
enferma todo o dia. O jugo das minhas transgressões está
atado pela sua mão; elas estão entretecidas, subiram sobre o meu
pescoço, e ele abateu a minha força; entregou-me o Senhor nas mãos
daqueles a quem não posso resistir. O Senhor atropelou todos
os meus poderosos no meio de mim; convocou contra mim uma
assembléia, para esmagar os meus jovens; o Senhor pisou como num
lagar a virgem filha de Judá. Por estas coisas eu ando
chorando; os meus olhos, os meus olhos se desfazem em águas; porque
se afastou de mim o consolador que devia restaurar a minha alma; os
meus filhos estão assolados, porque prevaleceu o inimigo.
Estende Sião as suas mãos, não há quem a console; mandou o
Senhor acerca de Jacó que lhe fossem inimigos os que estão em redor
dele; Jerusalém é entre eles como uma mulher imunda. Justo é
o Senhor, pois me rebelei contra o seu mandamento; ouvi, pois, todos
os povos, e vede a minha dor; as minhas virgens e os meus jovens
foram levados para o cativeiro. Chamei os meus amantes, mas
eles me enganaram; os meus sacerdotes e os meus anciãos expiraram na
cidade; enquanto buscavam para si mantimento, para restaurarem a sua
alma. Olha, Senhor, porque estou angustiada; turbadas estão
as minhas entranhas; o meu coração está transtornado dentro de mim,
porque gravemente me rebelei; fora me desfilhou a espada, em casa
está a morte. Ouviram que eu suspiro, mas não tenho quem me
console; todos os meus inimigos que souberam do meu mal folgam,
porque tu o fizeste; mas, em trazendo tu o dia que apregoaste, serão
como eu. Venha toda a sua maldade diante de ti, e faze-lhes
como me fizeste a mim por causa de todas as minhas transgressões;
porque os meus suspiros são muitos, e o meu coração está
desfalecido.

\medskip

\lettrine{2} Como cobriu o Senhor de nuvens na sua ira a filha
de Sião! Derrubou do céu à terra a glória de Israel, e não se
lembrou do escabelo de seus pés, no dia da sua ira. Devorou o
Senhor todas as moradas de Jacó, e não se apiedou; derrubou no seu
furor as fortalezas da filha de Judá, e abateu-as até à terra;
profanou o reino e os seus príncipes. No furor da sua ira cortou
toda a força de Israel; retirou para trás a sua destra de diante do
inimigo; e ardeu contra Jacó, como labareda de fogo que consome em
redor. Armou o seu arco como inimigo, firmou a sua destra como
adversário, e matou tudo o que era formoso à vista; derramou a sua
indignação como fogo na tenda da filha de Sião. Tornou-se o
Senhor como inimigo; devorou a Israel, devorou a todos os seus
palácios, destruiu as suas fortalezas; e multiplicou na filha de
Judá a lamentação e a tristeza. E arrancou o seu tabernáculo com
violência, como se fosse a de uma horta; destruiu o lugar da sua
congregação; o Senhor, em Sião, pôs em esquecimento a festa solene e
o sábado, e na indignação da sua ira rejeitou com desprezo o rei e o
sacerdote. Rejeitou o Senhor o seu altar, detestou o seu
santuário; entregou na mão do inimigo os muros dos seus palácios;
deram gritos na casa do Senhor, como em dia de festa solene.
Intentou o Senhor  destruir o muro da filha de Sião; estendeu o
cordel sobre ele, não retirou a sua mão destruidora; fez gemer o
antemuro e o muro; estão eles juntamente enfraquecidos. As suas
portas caíram por terra; ele destruiu e quebrou os seus ferrolhos; o
seu rei e os seus príncipes estão entre os gentios, onde não há lei,
nem os seus profetas acham visão alguma do Senhor.

Estão sentados na terra, silenciosos, os anciãos da filha de
Sião; lançam pó sobre as suas cabeças, cingiram sacos; as virgens de
Jerusalém abaixam as suas cabeças até à terra. Já se
consumiram os meus olhos com lágrimas, turbadas estão as minhas
entranhas, o meu fígado se derramou pela terra por causa do
quebrantamento da filha do meu povo; pois desfalecem o menino e a
criança de peito pelas ruas da cidade. Ao desfalecerem, como
feridos, pelas ruas da cidade, ao exalarem as suas almas no
regaço\footnote{Cavidade formada por veste comprida entre a cintura
e os joelhos de quem está sentado; colo. Fig.  Lugar de repouso ou
abrigo.} de suas mães, perguntam a elas: Onde está o trigo e o
vinho? Que testemunho te trarei? A quem te compararei, ó
filha de Jerusalém? A quem te assemelharei, para te consolar, ó
virgem filha de Sião? Porque grande como o mar é a tua
quebradura\footnote{Ato ou efeito de quebrar(-se); quebra,
quebramento. Qualquer abertura ou ruptura. Ex.: q. da rocha. Uso
informal: hérnia.}; quem te sarará? Os teus profetas viram
para ti, vaidade e loucura, e não manifestaram a tua maldade, para
impedirem o teu cativeiro; mas viram para ti cargas vãs e motivos de
expulsão. Todos os que passam pelo caminho batem palmas,
assobiam e meneiam as suas cabeças sobre a filha de Jerusalém,
dizendo: É esta a cidade que denominavam: perfeita em formosura,
gozo de toda a terra? Todos os teus inimigos abrem as suas
bocas contra ti, assobiam, e rangem os dentes; dizem: Devoramo-la;
certamente este é o dia que esperávamos; achamo-lo, vimo-lo.
Fez o Senhor  o que intentou; cumpriu a sua palavra, que
ordenou desde os dias da antiguidade; derrubou, e não se apiedou;
fez que o inimigo se alegrasse por tua causa, exaltou o poder dos
teus adversários. O coração deles clamou ao Senhor: Ó muralha
da filha de Sião, corram as tuas lágrimas como um ribeiro, de dia e
de noite; não te dês descanso, nem parem as meninas de teus olhos.
Levanta-te, clama de noite no princípio das vigias; derrama o
teu coração como águas diante da presença do Senhor; levanta a ele
as tuas mãos, pela vida de teus filhinhos, que desfalecem de fome à
entrada de todas as ruas. Vê, ó Senhor, e considera a quem
fizeste assim! Hão de comer as mulheres o fruto de si mesmas, as
crianças que trazem nos braços? Ou matar-se-á no santuário do Senhor
o sacerdote e o profeta? Jazem por terra pelas ruas o moço e
o velho, as minhas virgens e os meus jovens vieram a cair à espada;
tu os mataste no dia da tua ira; mataste e não te apiedaste.
Convocaste os meus temores em redor como num dia de
solenidade; não houve no dia da ira do Senhor  quem escapasse, ou
ficasse; aqueles que eu trouxe nas mãos e sustentei, o meu inimigo
os consumiu.

\medskip

\lettrine{3} Eu sou aquele homem que viu a aflição pela vara
do seu furor. Ele me guiou e me fez andar em trevas e não na
luz. Deveras fez virar e revirar a sua mão contra mim o dia
todo. Fez envelhecer a minha carne e a minha pele, quebrou os
meus ossos. Edificou contra mim, e me cercou de fel e trabalho.
Assentou-me em lugares tenebrosos, como os que estavam mortos há
muito. Cercou-me de uma sebe, e não posso sair; agravou os meus
grilhões. Ainda quando clamo e grito, ele exclui a minha oração.
Fechou os meus caminhos com pedras lavradas, fez tortuosas as
minhas veredas. Fez-se-me como urso de emboscada, um leão em
esconderijos. Desviou os meus caminhos, e fez-me em pedaços;
deixou-me assolado. Armou o seu arco, e me pôs como alvo à
flecha. Fez entrar nos meus rins as flechas da sua aljava.
Fui feito um objeto de escárnio para todo o meu povo, e a sua
canção todo o dia. Fartou-me de amarguras, embriagou-me de
absinto. Quebrou com cascalho os meus dentes, abaixou-me na
cinza. E afastaste da paz a minha alma; esqueci-me do bem.
Então disse eu: Já pereceu a minha força, como também a minha
esperança no Senhor. Lembra-te da minha aflição e do meu
pranto, do absinto e do fel. Minha alma certamente disto se
lembra, e se abate dentro de mim.

Disto me recordarei na minha mente; por isso esperarei. As
misericórdias do Senhor  são a causa de não sermos consumidos,
porque as suas misericórdias não têm fim; novas são cada
manhã; grande é a tua fidelidade. A minha porção é o Senhor,
diz a minha alma; portanto esperarei nele. Bom é o Senhor
para os que esperam por ele, para a alma que o busca. Bom é
ter esperança, e aguardar em silêncio a salvação do Senhor.
Bom é para o homem suportar o jugo na sua mocidade.
Assente-se solitário e fique em silêncio; porquanto Deus o
pôs sobre ele. Ponha a sua boca no pó; talvez ainda haja
esperança. Dê a sua face ao que o fere; farte-se de afronta.
Pois o Senhor não rejeitará para sempre. Pois, ainda
que entristeça a alguém, usará de compaixão, segundo a grandeza das
suas misericórdias. Porque não aflige nem entristece de bom
grado aos filhos dos homens. Pisar debaixo dos seus pés a
todos os presos da terra, perverter o direito do homem
perante a face do Altíssimo; subverter ao homem no seu
pleito, não o veria o Senhor?

Quem é aquele que diz, e assim acontece, quando o Senhor o não
mande? Porventura da boca do Altíssimo não sai tanto o mal
como o bem? De que se queixa, pois, o homem vivente?
Queixe-se cada um dos seus pecados. Esquadrinhemos os nossos
caminhos, e provemo-los, e voltemos para o Senhor. Levantemos
os nossos corações com as mãos para Deus nos céus, dizendo:

Nós transgredimos, e fomos rebeldes; por isso tu não perdoaste.
Cobriste-te de ira, e nos perseguiste; mataste, não
perdoaste. Cobriste-te de nuvens, para que não passe a nossa
oração. Como escória e refugo nos puseste no meio dos povos.
Todos os nossos inimigos abriram contra nós a sua boca.
Temor e laço vieram sobre nós, assolação e destruição.
Torrentes de água derramaram os meus olhos, por causa da
destruição da filha do meu povo. Os meus olhos choram, e não
cessam, porque não há descanso, até que o Senhor  atente e
veja desde os céus. Os meus olhos entristecem a minha alma,
por causa de todas as filhas da minha cidade. Como ave me
caçam os que, sem causa, são meus inimigos. Cortaram-me a
vida na masmorra, e lançaram pedras sobre mim. Águas correram
sobre a minha cabeça; eu disse: Estou cortado.

Invoquei o teu nome, Senhor, desde a mais profunda masmorra.
Ouviste a minha voz; não escondas o teu ouvido ao meu
suspiro, ao meu clamor. Tu te aproximaste no dia em que te
invoquei; disseste: Não temas. Pleiteaste, Senhor, as causas
da minha alma, remiste a minha vida. Viste, Senhor, a
injustiça que me fizeram; julga a minha causa. Viste toda a
sua vingança, todos os seus pensamentos contra mim. Ouviste a
sua afronta, Senhor, todos os seus pensamentos contra mim, os
lábios dos que se levantam contra mim e os seus desígnios me são
contrários todo o dia. Observa-os ao assentarem-se e ao
levantarem-se; eu sou a sua música. Tu lhes darás recompensa,
Senhor, conforme a obra das suas mãos. Tu lhes darás ânsia de
coração, maldição tua sobre eles. Na tua ira os perseguirás,
e os destruirás de debaixo dos céus do Senhor.

\medskip

\lettrine{4} Como se escureceu o ouro! Como se mudou o ouro
puro e bom! Como estão espalhadas as pedras do santuário sobre cada
rua! Os preciosos filhos de Sião, avaliados a puro ouro, como
são agora reputados por vasos de barro, obra das mãos do oleiro!
Até os chacais abaixam o peito, dão de mamar aos seus filhos;
mas a filha do meu povo tornou-se cruel como os avestruzes no
deserto. A língua do que mama fica pegada pela sede ao seu
paladar; os meninos pedem pão, e ninguém lho reparte. Os que
comiam comidas finas agora desfalecem nas ruas; os que se criaram em
carmesim abraçam monturos. Porque maior é a iniqüidade da filha
do meu povo do que o pecado de Sodoma, a qual foi subvertida como
num momento, sem que mãos lhe tocassem. Os seus nobres eram mais
puros do que a neve, mais brancos do que o leite, mais vermelhos de
corpo do que os rubis, e mais polidos do que a safira. Mas agora
escureceu-se o seu aspecto mais do que o negrume; não são conhecidos
nas ruas; a sua pele se lhes pegou aos ossos, secou-se, tornou-se
como um pau. Os mortos à espada foram mais ditosos do que os
mortos à fome; porque estes morreram lentamente, por falta dos
frutos dos campos. As mãos das mulheres compassivas cozeram
seus próprios filhos; serviram-lhes de alimento na destruição da
filha do meu povo. Deu o Senhor  cumprimento ao seu furor;
derramou o ardor da sua ira, e acendeu fogo em Sião, que consumiu os
seus fundamentos. Não creram os reis da terra, nem todos os
moradores do mundo, que entrasse o adversário e o inimigo pelas
portas de Jerusalém.

Foi por causa dos pecados dos profetas, das maldades dos seus
sacerdotes, que derramaram o sangue dos justos no meio dela.
Vagueiam como cegos nas ruas, andam contaminados de sangue;
de tal sorte que ninguém pode tocar nas suas roupas.
Desviai-vos, imundos! gritavam-lhes; desviai-vos,
desviai-vos, não toqueis! quando fugiram e também andaram errantes,
dizia-se entre os gentios: Nunca mais morarão aqui. A face
indignada do Senhor  os espalhou, ele nunca mais tornará a olhar
para eles; não respeitaram a pessoa dos sacerdotes, nem se
compadeceram dos velhos. Os nossos olhos desfaleciam,
esperando o nosso vão socorro; olhávamos atentamente para uma nação
que não nos podia livrar. Espiaram os nossos passos, de
maneira que não podíamos andar pelas nossas ruas; está chegado o
nosso fim, estão cumpridos os nossos dias, porque é vindo o nosso
fim. Os nossos perseguidores foram mais ligeiros do que as
águias dos céus; sobre os montes nos perseguiram, no deserto nos
armaram ciladas. O fôlego das nossas narinas, o ungido do
Senhor, foi preso nas suas covas; dele dizíamos: Debaixo da sua
sombra viveremos entre os gentios.

Regozija-te e alegra-te, ó filha de Edom, que habitas na terra de
Uz; o cálice passará também para ti; embebedar-te-ás, e te
descobrirás. O castigo da tua maldade está consumado, ó filha
de Sião; ele nunca mais te levará para o cativeiro; ele visitará a
tua maldade, ó filha de Edom, descobrirá os teus pecados.

\medskip

\lettrine{5} Lembra-te, Senhor, do que nos tem sucedido;
considera, e olha o nosso opróbrio. A nossa herança passou a
estrangeiros, e as nossas casas a forasteiros. Órfãos somos sem
pai, nossas mães são como viúvas. A nossa água por dinheiro a
bebemos, por preço vem a nossa lenha. Os nossos perseguidores
estão sobre os nossos pescoços; estamos cansados, e não temos
descanso. Aos egípcios e aos assírios estendemos as mãos, para
nos fartarem de pão. Nossos pais pecaram, e já não existem; e
nós levamos as suas maldades. Servos dominam sobre nós; ninguém
há que nos livre da sua mão. Com perigo de nossas vidas trazemos
o nosso pão, por causa da espada do deserto. Nossa pele se
queimou como um forno, por causa do ardor da fome. Forçaram
as mulheres em Sião, as virgens nas cidades de Judá. Os
príncipes foram enforcados pelas mãos deles; as faces dos velhos não
foram reverenciadas. Aos jovens obrigaram a moer, e os
meninos caíram debaixo das cargas de lenha. Os velhos já não
estão mais às portas, os jovens já deixaram a sua música.
Cessou o gozo de nosso coração; converteu-se em lamentação a
nossa dança. Caiu a coroa da nossa cabeça; ai de nós! porque
pecamos.

Por isso desmaiou o nosso coração; por isso se escureceram os
nossos olhos. Pelo monte de Sião, que está assolado, andam as
raposas. Tu, Senhor, permaneces eternamente, e o teu trono
subsiste de geração em geração. Por que te esquecerias de nós
para sempre? Por que nos desampararias por tanto tempo?
Converte-nos a ti, Senhor, e seremos convertidos; renova os
nossos dias como dantes. Mas tu nos rejeitaste totalmente. Tu
estás muito enfurecido contra nós.

