\addchap{Daniel}

\lettrine{1} No ano terceiro do reinado de Jeoiaquim, rei de
Judá, veio Nabucodonosor, rei de Babilônia, a Jerusalém, e a sitiou.
E o Senhor entregou nas suas mãos a Jeoiaquim, rei de Judá, e
uma parte dos utensílios da casa de Deus, e ele os levou para a
terra de Sinar, para a casa do seu deus, e pôs os utensílios na casa
do tesouro do seu deus. E disse o rei a Aspenaz, chefe dos seus
eunucos, que trouxesse alguns dos filhos de Israel, e da linhagem
real e dos príncipes, jovens em quem não houvesse defeito algum,
de boa aparência, e instruídos em toda a sabedoria, e doutos em
ciência, e entendidos no conhecimento, e que tivessem habilidade
para assistirem no palácio do rei, e que lhes ensinassem as letras e
a língua dos caldeus. E o rei lhes determinou a porção diária,
das iguarias do rei, e do vinho que ele bebia, e que assim fossem
mantidos por três anos, para que no fim destes pudessem estar diante
do rei. E entre eles se achavam, dos filhos de Judá, Daniel,
Hananias, Misael e Azarias; e o chefe dos eunucos lhes pôs
outros nomes, a saber: a Daniel pôs o de Beltessazar, e a Hananias o
de Sadraque, e a Misael o de Mesaque, e a Azarias o de Abednego.

E Daniel propôs no seu coração não se contaminar com a porção das
iguarias do rei, nem com o vinho que ele bebia; portanto pediu ao
chefe dos eunucos que lhe permitisse não se contaminar. Ora,
Deus fez com que Daniel achasse graça e misericórdia diante do chefe
dos eunucos. E disse o chefe dos eunucos a Daniel: Tenho medo
do meu senhor, o rei, que determinou a vossa comida e a vossa
bebida; pois por que veria ele os vossos rostos mais tristes do que
os dos outros jovens da vossa idade? Assim porias em perigo a minha
cabeça para com o rei. Então disse Daniel ao despenseiro a
quem o chefe dos eunucos havia constituído sobre Daniel, Hananias,
Misael e Azarias: Experimenta, peço-te, os teus servos dez
dias, e que se nos dêem legumes a comer, e água a beber.
Então se examine diante de ti a nossa aparência, e a
aparência dos jovens que comem a porção das iguarias do rei; e,
conforme vires, procederás para com os teus servos. E ele
consentiu isto, e os experimentou dez dias. E, ao fim dos dez
dias, apareceram os seus semblantes melhores, e eles estavam mais
gordos de carne do que todos os jovens que comiam das iguarias do
rei. Assim o despenseiro tirou-lhes a porção das iguarias, e
o vinho de que deviam beber, e lhes dava legumes.

Quanto a estes quatro jovens, Deus lhes deu o conhecimento e a
inteligência em todas as letras, e sabedoria; mas a Daniel deu
entendimento em toda a visão e sonhos. E ao fim dos dias, em
que o rei tinha falado que os trouxessem, o chefe dos eunucos os
trouxe diante de Nabucodonosor. E o rei falou com eles; e
entre todos eles não foram achados outros tais como Daniel,
Hananias, Misael e Azarias; portanto ficaram assistindo diante do
rei. E em toda a matéria de sabedoria e de discernimento,
sobre o que o rei lhes perguntou, os achou dez vezes mais doutos do
que todos os magos astrólogos que havia em todo o seu reino.
E Daniel permaneceu até ao primeiro ano do rei Ciro.

\medskip

\lettrine{2} E no segundo ano do reinado de Nabucodonosor,
Nabucodonosor teve sonhos; e o seu espírito se perturbou, e
passou-se-lhe o sono. Então o rei mandou chamar os magos, os
astrólogos, os encantadores e os caldeus, para que declarassem ao
rei os seus sonhos; e eles vieram e se apresentaram diante do rei.
E o rei lhes disse: Tive um sonho; e para saber o sonho está
perturbado o meu espírito. E os caldeus disseram ao rei em
aramaico\footnote{Outras versões, como a RC e a King James registram
\emph{siríaco}: Then spake the Chaldeans to the king in Syriack, O
king, live for ever: tell thy servants the dream, and we will shew
the interpretation.}: Ó rei, vive eternamente! Dize o sonho a teus
servos, e daremos a interpretação. Respondeu o rei, e disse aos
caldeus: O assunto me tem escapado; se não me fizerdes saber o sonho
e a sua interpretação, sereis despedaçados, e as vossas casas serão
feitas um monturo; mas se vós me declarardes o sonho e a sua
interpretação, recebereis de mim dádivas, recompensas e grande
honra; portanto declarai-me o sonho e a sua interpretação.
Responderam segunda vez, e disseram: Diga o rei o sonho a seus
servos, e daremos a sua interpretação. Respondeu o rei, e disse:
Percebo muito bem que vós quereis ganhar tempo; porque vedes que o
assunto me tem escapado. De modo que, se não me fizerdes saber o
sonho, uma só sentença será a vossa; pois vós preparastes palavras
mentirosas e perversas para as proferirdes na minha presença, até
que se mude o tempo; portanto dizei-me o sonho, para que eu entenda
que me podeis dar a sua interpretação. Responderam os caldeus
na presença do rei, e disseram: Não há ninguém sobre a terra que
possa declarar a palavra ao rei; pois nenhum rei há, grande ou
dominador, que requeira coisas semelhantes de algum mago, ou
astrólogo, ou caldeu. Porque o assunto que o rei requer é
difícil; e ninguém há que o possa declarar diante do rei, senão os
deuses, cuja morada não é com a carne. Por isso o rei muito
se irou e enfureceu; e ordenou que matassem a todos os sábios de
Babilônia. E saiu o decreto, segundo o qual deviam ser mortos
os sábios; e buscaram a Daniel e aos seus companheiros, para que
fossem mortos.

Então Daniel falou avisada e prudentemente a Arioque, capitão da
guarda do rei, que tinha saído para matar os sábios de Babilônia.
Respondeu, e disse a Arioque, capitão do rei: Por que se
apressa tanto o decreto da parte do rei? Então Arioque explicou o
caso a Daniel. E Daniel entrou; e pediu ao rei que lhe desse
tempo, para que lhe pudesse dar a interpretação. Então Daniel
foi para a sua casa, e fez saber o caso a Hananias, Misael e
Azarias, seus companheiros; para que pedissem misericórdia ao
Deus do céu, sobre este mistério, a fim de que Daniel e seus
companheiros não perecessem, juntamente com o restante dos sábios de
Babilônia. Então foi revelado o mistério a Daniel numa visão
de noite; então Daniel louvou o Deus do céu. Falou Daniel,
dizendo: Seja bendito o nome de Deus de eternidade a eternidade,
porque dele são a sabedoria e a força; e ele muda os tempos e
as estações; ele remove os reis e estabelece os reis; ele dá
sabedoria aos sábios e conhecimento aos entendidos. Ele
revela o profundo e o escondido; conhece o que está em trevas, e com
ele mora a luz. Ó Deus de meus pais, eu te dou graças e te
louvo, porque me deste sabedoria e força; e agora me fizeste saber o
que te pedimos, porque nos fizeste saber este assunto do rei.

Por isso Daniel foi ter com Arioque, ao qual o rei tinha
constituído para matar os sábios de Babilônia; entrou, e disse-lhe
assim: Não mates os sábios de Babilônia; introduze-me na presença do
rei, e declararei ao rei a interpretação. Então Arioque
depressa introduziu a Daniel na presença do rei, e disse-lhe assim:
Achei um homem dentre os cativos de Judá, o qual fará saber ao rei a
interpretação. Respondeu o rei, e disse a Daniel (cujo nome
era Beltessazar): Podes tu fazer-me saber o sonho que tive e a sua
interpretação? Respondeu Daniel na presença do rei, dizendo:
O segredo que o rei requer, nem sábios, nem astrólogos, nem magos,
nem adivinhos o podem declarar ao rei; mas há um Deus no céu,
o qual revela os mistérios; ele, pois, fez saber ao rei
Nabucodonosor o que há de acontecer nos últimos dias; o teu sonho e
as visões da tua cabeça que tiveste na tua cama são estes:
Estando tu, ó rei, na tua cama, subiram os teus pensamentos,
acerca do que há de ser depois disto. Aquele, pois, que revela os
mistérios te fez saber o que há de ser. E a mim me foi
revelado esse mistério, não porque haja em mim mais sabedoria que em
todos os viventes, mas para que a interpretação se fizesse saber ao
rei, e para que entendesses os pensamentos do teu coração.

Tu, ó rei, estavas vendo, e eis aqui uma grande estátua; esta
estátua, que era imensa, cujo esplendor era excelente, e estava em
pé diante de ti; e a sua aparência era terrível. A cabeça
daquela estátua era de ouro fino; o seu peito e os seus braços de
prata; o seu ventre e as suas coxas de cobre; as pernas de
ferro; os seus pés em parte de ferro e em parte de barro.
Estavas vendo isto, quando uma pedra foi cortada, sem auxílio
de mão, a qual feriu a estátua nos pés de ferro e de barro, e os
esmiuçou. Então foi juntamente esmiuçado o ferro, o barro, o
bronze, a prata e o ouro, os quais se fizeram como
pragana\footnote{Barba de espiga de cereais; aresta.} das
eiras\footnote{Área de terra batida, lajeada ou cimentada, onde se
malham, trilham, secam e limpam cereais e legumes; almanxar. Terreno
onde se junta o sal, ao lado das marinhas. Pátio, em algumas
fábricas de tecido. Bras. Lugar anexo às fábricas de açúcar, onde se
guardam as canas antes de serem utilizadas.} do
estio\footnote{Verão.}, e o vento os levou, e não se achou lugar
algum para eles; mas a pedra, que feriu a estátua, se tornou grande
monte, e encheu toda a terra. Este é o sonho; também a sua
interpretação diremos na presença do rei. Tu, ó rei, és rei
de reis; a quem o Deus do céu tem dado o reino, o poder, a força, e
a glória. E onde quer que habitem os filhos de homens, na tua
mão entregou os animais do campo, e as aves do céu, e fez que
reinasse sobre todos eles; tu és a cabeça de ouro. E depois
de ti se levantará outro reino, inferior ao teu; e um terceiro
reino, de bronze, o qual dominará sobre toda a terra. E o
quarto reino será forte como ferro; pois, como o ferro, esmiúça e
quebra tudo; como o ferro que quebra todas as coisas, assim ele
esmiuçará e fará em pedaços. E, quanto ao que viste dos pés e
dos dedos, em parte de barro de oleiro, e em parte de ferro, isso
será um reino dividido; contudo haverá nele alguma coisa da firmeza
do ferro, pois viste o ferro misturado com barro de lodo. E
como os dedos dos pés eram em parte de ferro e em parte de barro,
assim por uma parte o reino será forte, e por outra será frágil.
Quanto ao que viste do ferro misturado com barro de lodo,
misturar-se-ão com semente humana, mas não se ligarão um ao outro,
assim como o ferro não se mistura com o barro. Mas, nos dias
desses reis, o Deus do céu levantará um reino que não será jamais
destruído; e este reino não passará a outro povo; esmiuçará e
consumirá todos esses reinos, mas ele mesmo subsistirá para sempre,
da maneira que viste que do monte foi cortada uma pedra, sem
auxílio de mãos, e ela esmiuçou o ferro, o bronze, o barro, a prata
e o ouro; o grande Deus fez saber ao rei o que há de ser depois
disto. Certo é o sonho, e fiel a sua interpretação.

Então o rei Nabucodonosor caiu sobre a sua face, e adorou a
Daniel, e ordenou que lhe oferecessem uma oblação e perfumes suaves.
Respondeu o rei a Daniel, e disse: Certamente o vosso Deus é
Deus dos deuses, e o Senhor dos reis e revelador de mistérios, pois
pudeste revelar este mistério. Então o rei engrandeceu a
Daniel, e lhe deu muitas e grandes dádivas, e o pôs por governador
de toda a província de Babilônia, como também o fez chefe dos
governadores sobre todos os sábios de Babilônia. E pediu
Daniel ao rei, e constituiu ele sobre os negócios da província de
Babilônia a Sadraque, Mesaque e Abednego; mas Daniel permaneceu na
porta do rei.

\medskip

\lettrine{3} O rei Nabucodonosor fez uma estátua de ouro, cuja
altura era de sessenta côvados, e a sua largura de seis côvados;
levantou-a no campo de Dura, na província de Babilônia. Então o
rei Nabucodonosor mandou reunir os príncipes, os prefeitos, os
governadores, os conselheiros, os tesoureiros, os juízes, os
capitães, e todos os oficiais das províncias, para que viessem à
consagração da estátua que o rei Nabucodonosor tinha levantado.
Então se reuniram os príncipes, os prefeitos e governadores, os
capitães, os juízes, os tesoureiros, os conselheiros, e todos os
oficiais das províncias, à consagração da estátua que o rei
Nabucodonosor tinha levantado; e estavam em pé diante da imagem que
Nabucodonosor tinha levantado. E o arauto apregoava em alta voz:
Ordena-se a vós, ó povos, nações e línguas: quando ouvirdes o
som da buzina, da flauta, da harpa, da sambuca\footnote{Pequena
harpa triangular. Instrumento de sopro do tipo da museta e da
sacabuxa.}, do saltério\footnote{Entre os gregos, designação comum
aos instrumentos de cordas que se feriam com os dedos e não com o
plectro. Na Idade Média, instrumento de origem oriental, de forma
triangular ou trapezoidal, composto de uma caixa de ressonância com
uma ou várias rosáceas, e de um número variável de cordas simples ou
duplas, retesadas sobre a caixa por meio de cravelhas, e que eram
feridas com os dedos ou com o plectro, ou percutidas com duas
baquetas; címbalo.}, da gaita de foles, e de toda a espécie de
música, prostrar-vos-eis, e adorareis a estátua de ouro que o rei
Nabucodonosor tem levantado. E qualquer que não se prostrar e
não a adorar, será na mesma hora lançado dentro da fornalha de fogo
ardente. Portanto, no mesmo instante em que todos os povos
ouviram o som da buzina, da flauta, da harpa, da sambuca, do
saltério e de toda a espécie de música, prostraram-se todos os
povos, nações e línguas, e adoraram a estátua de ouro que o rei
Nabucodonosor tinha levantado.

Por isso, no mesmo instante chegaram perto alguns caldeus, e
acusaram os judeus. E responderam, dizendo ao rei Nabucodonosor:
Ó rei, vive eternamente! Tu, ó rei, fizeste um decreto, pelo
qual todo homem que ouvisse o som da buzina, da flauta, da harpa, da
sambuca, do saltério, e da gaita de foles, e de toda a espécie de
música, se prostrasse e adorasse a estátua de ouro; e,
qualquer que não se prostrasse e adorasse, seria lançado dentro da
fornalha de fogo ardente. Há uns homens judeus, os quais
constituíste sobre os negócios da província de Babilônia: Sadraque,
Mesaque e Abednego; estes homens, ó rei, não fizeram caso de ti; a
teus deuses não servem, nem adoram a estátua de ouro que levantaste.
Então Nabucodonosor, com ira e furor, mandou trazer a
Sadraque, Mesaque e Abednego. E trouxeram a estes homens perante o
rei. Falou Nabucodonosor, e lhes disse: É de propósito, ó
Sadraque, Mesaque e Abednego, que vós não servis a meus deuses nem
adorais a estátua de ouro que levantei? Agora, pois, se
estais prontos, quando ouvirdes o som da buzina, da flauta, da
harpa, da sambuca, do saltério, da gaita de foles, e de toda a
espécie de música, para vos prostrardes e adorardes a estátua que
fiz, bom é; mas, se não a adorardes, sereis lançados, na mesma hora,
dentro da fornalha de fogo ardente. E quem é o Deus que vos poderá
livrar das minhas mãos? Responderam Sadraque, Mesaque e
Abednego, e disseram ao rei Nabucodonosor: Não necessitamos de te
responder sobre este negócio. Eis que o nosso Deus, a quem
nós servimos, é que nos pode livrar; ele nos livrará da fornalha de
fogo ardente, e da tua mão, ó rei. E, se não, fica sabendo ó
rei, que não serviremos a teus deuses nem adoraremos a estátua de
ouro que levantaste.

Então Nabucodonosor se encheu de furor, e mudou-se o aspecto do
seu semblante contra Sadraque, Mesaque e Abednego; falou, e ordenou
que a fornalha se aquecesse sete vezes mais do que se costumava
aquecer. E ordenou aos homens mais poderosos, que estavam no
seu exército, que atassem a Sadraque, Mesaque e Abednego, para
lançá-los na fornalha de fogo ardente. Então estes homens
foram atados, vestidos com as suas capas, suas túnicas, e seus
chapéus, e demais roupas, e foram lançados dentro da fornalha de
fogo ardente. E, porque a palavra do rei era urgente, e a
fornalha estava sobremaneira quente, a chama do fogo matou aqueles
homens que carregaram a Sadraque, Mesaque, e Abednego. E
estes três homens, Sadraque, Mesaque e Abednego, caíram atados
dentro da fornalha de fogo ardente. Então o rei Nabucodonosor
se espantou, e se levantou depressa; falou, dizendo aos seus
conselheiros: Não lançamos nós, dentro do fogo, três homens atados?
Responderam e disseram ao rei: É verdade, ó rei. Respondeu,
dizendo: Eu, porém, vejo quatro homens soltos, que andam passeando
dentro do fogo, sem sofrer nenhum dano; e o aspecto do quarto é
semelhante ao Filho de Deus. Então chegando-se Nabucodonosor
à porta da fornalha de fogo ardente, falou, dizendo: Sadraque,
Mesaque e Abednego, servos do Deus Altíssimo, saí e vinde! Então
Sadraque, Mesaque e Abednego saíram do meio do fogo. E
reuniram-se os príncipes, os capitães, os governadores e os
conselheiros do rei e, contemplando estes homens, viram que o fogo
não tinha tido poder algum sobre os seus corpos; nem um só cabelo da
sua cabeça se tinha queimado, nem as suas capas se mudaram, nem
cheiro de fogo tinha passado sobre eles.

Falou Nabucodonosor, dizendo: Bendito seja o Deus de Sadraque,
Mesaque e Abednego, que enviou o seu anjo, e livrou os seus servos,
que confiaram nele, pois violaram a palavra do rei, preferindo
entregar os seus corpos, para que não servissem nem adorassem algum
outro deus, senão o seu Deus. Por mim, pois, é feito um
decreto, pelo qual todo o povo, e nação e língua que disser
blasfêmia contra o Deus de Sadraque, Mesaque e Abednego, seja
despedaçado, e as suas casas sejam feitas um monturo; porquanto não
há outro Deus que possa livrar como este. Então o rei fez
prosperar a Sadraque, Mesaque e Abednego, na província de Babilônia.

\medskip

\lettrine{4} Nabucodonosor rei, a todos os povos, nações e
línguas, que moram em toda a terra: Paz vos seja multiplicada.
Pareceu-me bem fazer conhecidos os sinais e maravilhas que Deus,
o Altíssimo, tem feito para comigo. Quão grandes são os seus
sinais, e quão poderosas as suas maravilhas! O seu reino é um reino
sempiterno\footnote{Que não teve princípio nem há de ter fim;
eterno. Que dura sempre; perpétuo, contínuo. Antiqüíssimo. Filos.:
Diz-se de acontecimento que se integra à eternidade, concebida esta
como a totalidade dos eventos passados, presentes e futuros.}, e o
seu domínio de geração em geração.

Eu, Nabucodonosor, estava sossegado em minha casa, e próspero no
meu palácio. Tive um sonho, que me espantou; e estando eu na
minha cama, as imaginações e as visões da minha cabeça me turbaram.
Por isso expedi um decreto, para que fossem introduzidos à minha
presença todos os sábios de Babilônia, para que me fizessem saber a
interpretação do sonho. Então entraram os magos, os astrólogos,
os caldeus e os adivinhadores, e eu contei o sonho diante deles; mas
não me fizeram saber a sua interpretação. Mas por fim entrou na
minha presença Daniel, cujo nome é Beltessazar, segundo o nome do
meu deus, e no qual há o espírito dos deuses santos; e eu lhe contei
o sonho, dizendo: Beltessazar, mestre dos magos, pois eu sei que
há em ti o espírito dos deuses santos, e nenhum mistério te é
difícil, dize-me as visões do meu sonho que tive e a sua
interpretação. Eis, pois, as visões da minha cabeça, estando
eu na minha cama: Eu estava assim olhando, e vi uma árvore no meio
da terra, cuja altura era grande; crescia esta árvore, e se
fazia forte, de maneira que a sua altura chegava até ao céu; e era
vista até aos confins da terra. A sua folhagem era formosa, e
o seu fruto abundante, e havia nela sustento para todos; debaixo
dela os animais do campo achavam sombra, e as aves do céu faziam
morada nos seus ramos, e toda a carne se mantinha dela.
Estava vendo isso nas visões da minha cabeça, estando eu na
minha cama; e eis que um vigia, um santo, descia do céu,
clamando fortemente, e dizendo assim: Derrubai a árvore, e
cortai-lhe os ramos, sacudi as suas folhas, espalhai o seu fruto;
afugentem-se os animais de debaixo dela, e as aves dos seus ramos.
Mas deixai na terra o tronco com as suas raízes, atada com
cadeias de ferro e de bronze, na erva do campo; e seja molhado do
orvalho do céu, e seja a sua porção com os animais na erva da terra;
seja mudado o seu coração, para que não seja mais coração de
homem, e lhe seja dado coração de animal; e passem sobre ele sete
tempos. Esta sentença é por decreto dos vigias, e esta ordem
por mandado dos santos, a fim de que conheçam os viventes que o
Altíssimo tem domínio sobre o reino dos homens, e o dá a quem quer,
e até ao mais humilde dos homens constitui sobre ele. Este
sonho eu, rei Nabucodonosor, vi. Tu, pois, Beltessazar, dize a
interpretação, porque todos os sábios do meu reino não puderam
fazer-me saber a sua interpretação, mas tu podes; pois há em ti o
espírito dos deuses santos.

Então Daniel, cujo nome era Beltessazar, esteve atônito por uma
hora, e os seus pensamentos o turbavam; falou, pois, o rei, dizendo:
Beltessazar, não te espante o sonho, nem a sua interpretação.
Respondeu Beltessazar, dizendo: Senhor meu, seja o sonho contra os
que te têm ódio, e a sua interpretação aos teus inimigos. A
árvore que viste, que cresceu, e se fez forte, cuja altura chegava
até ao céu, e que foi vista por toda a terra; cujas folhas
eram formosas, e o seu fruto abundante, e em que para todos havia
sustento, debaixo da qual moravam os animais do campo, e em cujos
ramos habitavam as aves do céu; és tu, ó rei, que cresceste,
e te fizeste forte; a tua grandeza cresceu, e chegou até ao céu, e o
teu domínio até à extremidade da terra. E quanto ao que viu o
rei, um vigia, um santo, que descia do céu, e dizia: Cortai a
árvore, e destruí-a, mas o tronco com as suas raízes deixai na
terra, e atada com cadeias de ferro e de bronze, na erva do campo; e
seja molhado do orvalho do céu, e a sua porção seja com os animais
do campo, até que passem sobre ele sete tempos; esta é a
interpretação, ó rei; e este é o decreto do Altíssimo, que virá
sobre o rei, meu senhor: serás tirado dentre os homens, e a
tua morada será com os animais do campo, e te farão comer erva como
os bois, e serás molhado do orvalho do céu; e passar-se-ão sete
tempos por cima de ti; até que conheças que o Altíssimo tem domínio
sobre o reino dos homens, e o dá a quem quer. E quanto ao que
foi falado, que deixassem o tronco com as raízes da árvore, o teu
reino voltará para ti, depois que tiveres conhecido que o céu reina.
Portanto, ó rei, aceita o meu conselho, e põe fim aos teus
pecados, praticando a justiça, e às tuas iniqüidades, usando de
misericórdia com os pobres, pois, talvez se prolongue a tua
tranqüilidade.

Todas estas coisas vieram sobre o rei Nabucodonosor. Ao
fim de doze meses, quando passeava no palácio real de Babilônia,
falou o rei, dizendo: Não é esta a grande Babilônia que eu
edifiquei para a casa real, com a força do meu poder, e para glória
da minha magnificência? Ainda estava a palavra na boca do
rei, quando caiu uma voz do céu: A ti se diz, ó rei Nabucodonosor:
Passou de ti o reino. E serás tirado dentre os homens, e a
tua morada será com os animais do campo; far-te-ão comer erva como
os bois, e passar-se-ão sete tempos sobre ti, até que conheças que o
Altíssimo domina sobre o reino dos homens, e o dá a quem quer.
Na mesma hora se cumpriu a palavra sobre Nabucodonosor, e foi
tirado dentre os homens, e comia erva como os bois, e o seu corpo
foi molhado do orvalho do céu, até que lhe cresceu pêlo, como as
penas da águia, e as suas unhas como as das aves.

Mas ao fim daqueles dias eu, Nabucodonosor, levantei os meus
olhos ao céu, e tornou-me a vir o entendimento, e eu bendisse o
Altíssimo, e louvei e glorifiquei ao que vive para sempre, cujo
domínio é um domínio sempiterno, e cujo reino é de geração em
geração. E todos os moradores da terra são reputados em nada,
e segundo a sua vontade ele opera com o exército do céu e os
moradores da terra; não há quem possa estorvar\footnote{Fazer
estorvo a; importunar, incomodar. Embaraçar, tolher, ou dificultar;
impedir. Impedir ou tolher a liberdade de movimentos a; embaraçar.}
a sua mão, e lhe diga: Que fazes? No mesmo tempo tornou a mim
o meu entendimento, e para a dignidade do meu reino tornou-me a vir
a minha majestade e o meu resplendor; e buscaram-me os meus
conselheiros e os meus senhores; e fui restabelecido no meu reino, e
a minha glória foi aumentada. Agora, pois, eu, Nabucodonosor,
louvo, exalço\footnote{Exalçar = Exaltar.} e glorifico ao Rei do
céu; porque todas as suas obras são verdade, e os seus caminhos
juízo, e pode humilhar aos que andam na soberba.

\medskip

\lettrine{5} O rei Belsazar deu um grande banquete a mil dos
seus senhores, e bebeu vinho na presença dos mil. Havendo
Belsazar provado o vinho, mandou trazer os vasos de ouro e de prata,
que Nabucodonosor, seu pai, tinha tirado do templo que estava em
Jerusalém, para que bebessem neles o rei, os seus príncipes, as suas
mulheres e concubinas. Então trouxeram os vasos de ouro, que
foram tirados do templo da casa de Deus, que estava em Jerusalém, e
beberam neles o rei, os seus príncipes, as suas mulheres e
concubinas. Beberam o vinho, e deram louvores aos deuses de
ouro, de prata, de bronze, de ferro, de madeira, e de pedra. Na
mesma hora apareceram uns dedos de mão de homem, e escreviam,
defronte do castiçal, na caiadura da parede do palácio real; e o rei
via a parte da mão que estava escrevendo. Mudou-se então o
semblante do rei, e os seus pensamentos o turbaram; as juntas dos
seus lombos se relaxaram, e os seus joelhos batiam um no outro.
E gritou o rei com força, que se introduzissem os astrólogos, os
caldeus e os adivinhadores; e falou o rei, dizendo aos sábios de
Babilônia: Qualquer que ler este escrito, e me declarar a sua
interpretação, será vestido de púrpura, e trará uma cadeia de ouro
ao pescoço e, no reino, será o terceiro governante. Então
entraram todos os sábios do rei; mas não puderam ler o escrito, nem
fazer saber ao rei a sua interpretação. Então o rei Belsazar
perturbou-se muito, e mudou-se-lhe o semblante; e os seus senhores
estavam sobressaltados.

A rainha, por causa das palavras do rei e dos seus senhores,
entrou na casa do banquete, e respondeu, dizendo: Ó rei, vive para
sempre! Não te perturbem os teus pensamentos, nem se mude o teu
semblante. Há no teu reino um homem, no qual há o espírito
dos deuses santos; e nos dias de teu pai se achou nele luz, e
inteligência, e sabedoria, como a sabedoria dos deuses; e teu pai, o
rei Nabucodonosor, sim, teu pai, o rei, o constituiu mestre dos
magos, dos astrólogos, dos caldeus e dos adivinhadores;
porquanto se achou neste Daniel um espírito excelente, e
conhecimento, e entendimento, interpretando sonhos e explicando
enigmas, e resolvendo dúvidas, ao qual o rei pôs o nome de
Beltessazar. Chame-se, pois, agora Daniel, e ele dará a
interpretação. Então Daniel foi introduzido à presença do
rei. Falou o rei, dizendo a Daniel: És tu aquele Daniel, um dos
filhos dos cativos de Judá, que o rei, meu pai, trouxe de Judá?
Tenho ouvido dizer a teu respeito que o espírito dos deuses
está em ti, e que em ti se acham a luz, e o entendimento e a
excelente sabedoria. Agora mesmo foram introduzidos à minha
presença os sábios e os astrólogos, para lerem este escrito, e me
fazerem saber a sua interpretação; mas não puderam dar a
interpretação destas palavras. Eu, porém, tenho ouvido dizer
de ti que podes dar interpretação e resolver dúvidas. Agora, se
puderes ler este escrito, e fazer-me saber a sua interpretação,
serás vestido de púrpura, e terás cadeia de ouro ao pescoço e no
reino serás o terceiro governante. Então respondeu Daniel, e
disse na presença do rei: As tuas dádivas fiquem contigo, e dá os
teus prêmios a outro; contudo lerei ao rei o escrito, e far-lhe-ei
saber a interpretação. Ó rei! Deus, o Altíssimo, deu a
Nabucodonosor, teu pai, o reino, e a grandeza, e a glória, e a
majestade. E por causa da grandeza, que lhe deu, todos os
povos, nações e línguas tremiam e temiam diante dele; a quem queria
matava, e a quem queria conservava em vida; e a quem queria
engrandecia, e a quem queria abatia. Mas quando o seu coração
se exaltou, e o seu espírito se endureceu em soberba, foi derrubado
do seu trono real, e passou dele a sua glória. E foi tirado
dentre os filhos dos homens, e o seu coração foi feito semelhante ao
dos animais, e a sua morada foi com os jumentos monteses; fizeram-no
comer a erva como os bois, e do orvalho do céu foi molhado o seu
corpo, até que conheceu que Deus, o Altíssimo, tem domínio sobre o
reino dos homens, e a quem quer constitui sobre ele. E tu,
Belsazar, que és seu filho, não humilhaste o teu coração, ainda que
soubeste tudo isto. E te levantaste contra o Senhor do céu,
pois foram trazidos à tua presença os vasos da casa dele, e tu, os
teus senhores, as tuas mulheres e as tuas concubinas, bebestes vinho
neles; além disso, deste louvores aos deuses de prata, de ouro, de
bronze, de ferro, de madeira e de pedra, que não vêem, não ouvem,
nem sabem; mas a Deus, em cuja mão está a tua vida, e de quem são
todos os teus caminhos, a ele não glorificaste. Então dele
foi enviada aquela parte da mão, que escreveu este escrito.
Este, pois, é o escrito que se escreveu: MENE, MENE, TEQUEL,
UFARSIM. Esta é a interpretação daquilo: MENE: Contou Deus o
teu reino, e o acabou. TEQUEL: Pesado foste na balança, e
foste achado em falta. PERES: Dividido foi o teu reino, e
dado aos medos e aos persas. Então mandou Belsazar que
vestissem a Daniel de púrpura, e que lhe pusessem uma cadeia de ouro
ao pescoço, e proclamassem a respeito dele que havia de ser o
terceiro no governo do seu reino.

Naquela noite foi morto Belsazar, rei dos caldeus. E
Dario, o medo, ocupou o reino, sendo da idade de sessenta e dois
anos.

\medskip

\lettrine{6} E pareceu bem a Dario constituir sobre o reino
cento e vinte príncipes, que estivessem sobre todo o reino; e
sobre eles três presidentes, dos quais Daniel era um, aos quais
estes príncipes dessem conta, para que o rei não sofresse dano.
Então o mesmo Daniel sobrepujou a estes presidentes e príncipes;
porque nele havia um espírito excelente; e o rei pensava
constituí-lo sobre todo o reino. Então os presidentes e os
príncipes procuravam achar ocasião contra Daniel a respeito do
reino; mas não podiam achar ocasião ou culpa alguma; porque ele era
fiel, e não se achava nele nenhum erro nem culpa. Então estes
homens disseram: Nunca acharemos ocasião alguma contra este Daniel,
se não a acharmos contra ele na lei do seu Deus.

Então estes presidentes e príncipes foram juntos ao rei, e
disseram-lhe assim: Ó rei Dario, vive para sempre! Todos os
presidentes do reino, os capitães e príncipes, conselheiros e
governadores, concordaram em promulgar um edito real e confirmar a
proibição que qualquer que, por espaço de trinta dias, fizer uma
petição a qualquer deus, ou a qualquer homem, e não a ti, ó rei,
seja lançado na cova dos leões. Agora, pois, ó rei, confirma a
proibição, e assina o edito, para que não seja mudado, conforme a
lei dos medos e dos persas, que não se pode revogar. Por esta
razão o rei Dario assinou o edito e a proibição. Daniel,
pois, quando soube que o edito estava assinado, entrou em sua casa
(ora havia no seu quarto janelas abertas do lado de Jerusalém), e
três vezes no dia se punha de joelhos, e orava, e dava graças diante
do seu Deus, como também antes costumava fazer.

Então aqueles homens foram juntos, e acharam a Daniel orando e
suplicando diante do seu Deus. Então se apresentaram ao rei
e, a respeito do edito real, disseram-lhe: Porventura não assinaste
o edito, pelo qual todo o homem que fizesse uma petição a qualquer
deus, ou a qualquer homem, por espaço de trinta dias, e não a ti, ó
rei, fosse lançado na cova dos leões? Respondeu o rei, dizendo: Esta
palavra é certa, conforme a lei dos medos e dos persas, que não se
pode revogar. Então responderam ao rei, dizendo-lhe: Daniel,
que é dos filhos dos cativos de Judá, não tem feito caso de ti, ó
rei, nem do edito que assinaste, antes três vezes por dia faz a sua
oração. Ouvindo então o rei essas palavras, ficou muito
penalizado, e a favor de Daniel propôs dentro do seu coração
livrá-lo; e até ao pôr do sol trabalhou para salvá-lo. Então
aqueles homens foram juntos ao rei, e disseram-lhe: Sabe, ó rei, que
é lei dos medos e dos persas que nenhum edito ou decreto, que o rei
estabeleça, se pode mudar. Então o rei ordenou que trouxessem
a Daniel, e lançaram-no na cova dos leões. E, falando o rei, disse a
Daniel: O teu Deus, a quem tu continuamente serves, ele te livrará.
E foi trazida uma pedra e posta sobre a boca da cova; e o rei
a selou com o seu anel e com o anel dos seus senhores, para que não
se mudasse a sentença acerca de Daniel.

Então o rei se dirigiu para o seu palácio, e passou a noite em
jejum, e não deixou trazer à sua presença instrumentos de música; e
fugiu dele o sono. Pela manhã, ao romper do dia, levantou-se
o rei, e foi com pressa à cova dos leões. E, chegando-se à
cova, chamou por Daniel com voz triste; e disse o rei a Daniel:
Daniel, servo do Deus vivo, dar-se-ia o caso que o teu Deus, a quem
tu continuamente serves, tenha podido livrar-te dos leões?
Então Daniel falou ao rei: Ó rei, vive para sempre! O
meu Deus enviou o seu anjo, e fechou a boca dos leões, para que não
me fizessem dano, porque foi achada em mim inocência diante dele; e
também contra ti, ó rei, não tenho cometido delito algum.
Então o rei muito se alegrou em si mesmo, e mandou tirar a
Daniel da cova. Assim foi tirado Daniel da cova, e nenhum dano se
achou nele, porque crera no seu Deus. E ordenou o rei, e
foram trazidos aqueles homens que tinham acusado a Daniel, e foram
lançados na cova dos leões, eles, seus filhos e suas mulheres; e
ainda não tinham chegado ao fundo da cova quando os leões se
apoderaram deles, e lhes esmigalharam todos os ossos.

Então o rei Dario escreveu a todos os povos, nações e línguas que
moram em toda a terra: A paz vos seja multiplicada. Da minha
parte é feito um decreto, pelo qual em todo o domínio do meu reino
os homens tremam e temam perante o Deus de Daniel; porque ele é o
Deus vivo e que permanece para sempre, e o seu reino não se pode
destruir, e o seu domínio durará até o fim. Ele salva, livra,
e opera sinais e maravilhas no céu e na terra; ele salvou e livrou
Daniel do poder dos leões. Este Daniel, pois, prosperou no
reinado de Dario, e no reinado de Ciro, o persa.

\medskip

\lettrine{7} No primeiro ano de Belsazar, rei de Babilônia,
teve Daniel um sonho e visões da sua cabeça quando estava na sua
cama; escreveu logo o sonho, e relatou a suma das coisas. Falou
Daniel, e disse: Eu estava olhando na minha visão da noite, e eis
que os quatro ventos do céu agitavam o mar grande. E quatro
animais grandes, diferentes uns dos outros, subiam do mar. O
primeiro era como leão, e tinha asas de águia; enquanto eu olhava,
foram-lhe arrancadas as asas, e foi levantado da terra, e posto em
pé como um homem, e foi-lhe dado um coração de homem. Continuei
olhando, e eis aqui o segundo animal, semelhante a um urso, o qual
se levantou de um lado, tendo na boca três costelas entre os seus
dentes; e foi-lhe dito assim: Levanta-te, devora muita carne.
Depois disto, eu continuei olhando, e eis aqui outro, semelhante
a um leopardo, e tinha quatro asas de ave nas suas costas; tinha
também este animal quatro cabeças, e foi-lhe dado domínio.
Depois disto eu continuei olhando nas visões da noite, e eis
aqui o quarto animal, terrível e espantoso, e muito forte, o qual
tinha dentes grandes de ferro; ele devorava e fazia em pedaços, e
pisava aos pés o que sobejava; era diferente de todos os animais que
apareceram antes dele, e tinha dez chifres. Estando eu a
considerar os chifres, eis que, entre eles subiu outro chifre
pequeno, diante do qual três dos primeiros chifres foram arrancados;
e eis que neste chifre havia olhos, como os de homem, e uma boca que
falava grandes coisas.

Eu continuei olhando, até que foram postos uns tronos, e um ancião
de dias se assentou; a sua veste era branca como a neve, e o cabelo
da sua cabeça como a pura lã; e seu trono era de chamas de fogo, e
as suas rodas de fogo ardente. Um rio de fogo manava e saía
de diante dele; milhares de milhares o serviam, e milhões de milhões
assistiam diante dele; assentou-se o juízo, e abriram-se os livros.
Então estive olhando, por causa da voz das grandes palavras
que o chifre proferia; estive olhando até que o animal foi morto, e
o seu corpo desfeito, e entregue para ser queimado pelo fogo;
e, quanto aos outros animais, foi-lhes tirado o domínio;
todavia foi-lhes prolongada a vida até certo espaço de tempo.
Eu estava olhando nas minhas visões da noite, e eis que vinha
nas nuvens do céu um como o filho do homem; e dirigiu-se ao ancião
de dias, e o fizeram chegar até ele. E foi-lhe dado o
domínio, e a honra, e o reino, para que todos os povos, nações e
línguas o servissem; o seu domínio é um domínio eterno, que não
passará, e o seu reino tal, que não será destruído.

Quanto a mim, Daniel, o meu espírito foi abatido dentro do corpo,
e as visões da minha cabeça me perturbaram. Cheguei-me a um
dos que estavam perto, e pedi-lhe a verdade acerca de tudo isto. E
ele me disse, e fez-me saber a interpretação das coisas.
Estes grandes animais, que são quatro, são quatro reis, que
se levantarão da terra. Mas os santos do Altíssimo receberão
o reino, e o possuirão para todo o sempre, e de eternidade em
eternidade. Então tive desejo de conhecer a verdade a
respeito do quarto animal, que era diferente de todos os outros,
muito terrível, cujos dentes eram de ferro e as suas unhas de
bronze; que devorava, fazia em pedaços e pisava aos pés o que
sobrava; e também a respeito dos dez chifres que tinha na
cabeça, e do outro que subiu, e diante do qual caíram três, isto é,
daquele que tinha olhos, e uma boca que falava grandes coisas, e
cujo parecer era mais robusto do que o dos seus companheiros.
Eu olhava, e eis que este chifre fazia guerra contra os
santos, e prevaleceu contra eles. Até que veio o ancião de
dias, e fez justiça aos santos do Altíssimo; e chegou o tempo em que
os santos possuíram o reino. Disse assim: O quarto animal
será o quarto reino na terra, o qual será diferente de todos os
reinos; e devorará toda a terra, e a pisará aos pés, e a fará em
pedaços. E, quanto aos dez chifres, daquele mesmo reino se
levantarão dez reis; e depois deles se levantará outro, o qual será
diferente dos primeiros, e abaterá a três reis. E proferirá
palavras contra o Altíssimo, e destruirá os santos do Altíssimo, e
cuidará em mudar os tempos e a lei; e eles serão entregues na sua
mão, por um tempo, e tempos, e a metade de um tempo. Mas o
juízo será estabelecido, e eles tirarão o seu domínio, para o
destruir e para o desfazer até ao fim. E o reino, e o
domínio, e a majestade dos reinos debaixo de todo o céu serão dados
ao povo dos santos do Altíssimo; o seu reino será um reino eterno, e
todos os domínios o servirão, e lhe obedecerão. Aqui terminou
o assunto. Quanto a mim, Daniel, os meus pensamentos muito me
perturbaram, e mudou-se em mim o meu semblante; mas guardei o
assunto no meu coração.

\medskip

\lettrine{8} No ano terceiro do reinado do rei Belsazar
apareceu-me uma visão, a mim, Daniel, depois daquela que me apareceu
no princípio. E vi na visão; e sucedeu que, quando vi, eu estava
na cidadela de Susã, na província de Elão; vi, pois, na visão, que
eu estava junto ao rio Ulai. E levantei os meus olhos, e vi, e
eis que um carneiro estava diante do rio, o qual tinha dois chifres;
e os dois chifres eram altos, mas um era mais alto do que o outro; e
o mais alto subiu por último. Vi que o carneiro dava marradas
para o ocidente, e para o norte e para o sul; e nenhum dos animais
lhe podia resistir; nem havia quem pudesse livrar-se da sua mão; e
ele fazia conforme a sua vontade, e se engrandecia. E, estando
eu considerando, eis que um bode vinha do ocidente sobre toda a
terra, mas sem tocar no chão; e aquele bode tinha um chifre
insigne\footnote{Muito distinto; notável, célebre, assinalado.}
entre os olhos. E dirigiu-se ao carneiro que tinha os dois
chifres, ao qual eu tinha visto em pé diante do rio, e correu contra
ele no ímpeto da sua força. E vi-o chegar perto do carneiro,
enfurecido contra ele, e ferindo-o quebrou-lhe os dois chifres, pois
não havia força no carneiro para lhe resistir, e o bode o lançou por
terra, e o pisou aos pés; não houve quem pudesse livrar o carneiro
da sua mão. E o bode se engrandeceu sobremaneira; mas, estando
na sua maior força, aquele grande chifre foi quebrado; e no seu
lugar subiram outros quatro também insignes, para os quatro ventos
do céu. E de um deles saiu um chifre muito pequeno, o qual
cresceu muito para o sul, e para o oriente, e para a terra formosa.
E se engrandeceu até contra o exército do céu; e a alguns do
exército, e das estrelas, lançou por terra, e os pisou. E se
engrandeceu até contra o príncipe do exército; e por ele foi tirado
o sacrifício contínuo, e o lugar do seu santuário foi lançado por
terra. E um exército foi dado contra o sacrifício contínuo,
por causa da transgressão; e lançou a verdade por terra, e o fez, e
prosperou. Depois ouvi um santo que falava; e disse outro
santo àquele que falava: Até quando durará a visão do sacrifício
contínuo, e da transgressão assoladora, para que sejam entregues o
santuário e o exército, a fim de serem pisados? E ele me
disse: Até duas mil e trezentas tardes e manhãs; e o santuário será
purificado.

E aconteceu que, havendo eu, Daniel, tido a visão, procurei o
significado, e eis que se apresentou diante de mim como que uma
semelhança de homem. E ouvi uma voz de homem entre as margens
do Ulai, a qual gritou, e disse: Gabriel, dá a entender a este a
visão. E veio perto de onde eu estava; e, vindo ele, me
amedrontei, e caí sobre o meu rosto; mas ele me disse: Entende,
filho do homem, porque esta visão acontecerá no fim do tempo.
E, estando ele falando comigo, caí adormecido com o rosto em
terra; ele, porém, me tocou, e me fez estar em pé. E disse:
Eis que te farei saber o que há de acontecer no último tempo da ira;
pois isso pertence ao tempo determinado do fim. Aquele
carneiro que viste com dois chifres são os reis da Média e da
Pérsia, mas o bode peludo é o rei da Grécia; e o grande
chifre que tinha entre os olhos é o primeiro rei; o ter sido
quebrado, levantando-se quatro em lugar dele, significa que quatro
reinos se levantarão da mesma nação, mas não com a força dele.
Mas, no fim do seu reinado, quando acabarem os
prevaricadores, se levantará um rei, feroz de semblante, e será
entendido em adivinhações. E se fortalecerá o seu poder, mas
não pela sua própria força; e destruirá maravilhosamente, e
prosperará, e fará o que lhe aprouver; e destruirá os poderosos e o
povo santo. E pelo seu entendimento também fará prosperar o
engano na sua mão; e no seu coração se engrandecerá, e destruirá a
muitos que vivem em segurança; e se levantará contra o Príncipe dos
príncipes, mas sem mão será quebrado. E a visão da tarde e da
manhã que foi falada, é verdadeira. Tu, porém, cerra a visão, porque
se refere a dias muito distantes. E eu, Daniel, enfraqueci, e
estive enfermo alguns dias; então levantei-me e tratei do negócio do
rei. E espantei-me acerca da visão, e não havia quem a entendesse.

\medskip

\lettrine{9} No ano primeiro de Dario, filho de Assuero, da
linhagem dos medos, o qual foi constituído rei sobre o reino dos
caldeus, no primeiro ano do seu reinado, eu, Daniel, entendi
pelos livros que o número dos anos, de que falara o Senhor ao
profeta Jeremias, em que haviam de cumprir-se as desolações de
Jerusalém, era de setenta anos. E eu dirigi o meu rosto ao
Senhor Deus, para o buscar com oração e súplicas, com jejum, e saco
e cinza.

E orei ao Senhor meu Deus, e confessei, e disse: Ah! Senhor! Deus
grande e tremendo, que guardas a aliança e a misericórdia para com
os que te amam e guardam os teus mandamentos; pecamos, e
cometemos iniqüidades, e procedemos impiamente, e fomos rebeldes,
apartando-nos dos teus mandamentos e dos teus juízos; e não
demos ouvidos aos teus servos, os profetas, que em teu nome falaram
aos nossos reis, aos nossos príncipes, e a nossos pais, como também
a todo o povo da terra. A ti, ó Senhor, pertence a justiça, mas
a nós a confusão de rosto, como hoje se vê; aos homens de Judá, e
aos moradores de Jerusalém, e a todo o Israel, aos de perto e aos de
longe, em todas as terras por onde os tens lançado, por causa das
suas rebeliões que cometeram contra ti. Ó Senhor, a nós pertence
a confusão de rosto, aos nossos reis, aos nossos príncipes, e a
nossos pais, porque pecamos contra ti. Ao Senhor, nosso Deus,
pertencem a misericórdia, e o perdão; pois nos rebelamos contra ele,
e não obedecemos à voz do Senhor, nosso Deus, para andarmos
nas suas leis, que nos deu por intermédio de seus servos, os
profetas. Sim, todo o Israel transgrediu a tua lei,
desviando-se para não obedecer à tua voz; por isso a maldição e o
juramento, que estão escritos na lei de Moisés, servo de Deus, se
derramaram sobre nós; porque pecamos contra ele. E ele
confirmou a sua palavra, que falou contra nós, e contra os nossos
juízes que nos julgavam, trazendo sobre nós um grande mal; porquanto
debaixo de todo o céu nunca se fez como se tem feito em Jerusalém.
Como está escrito na lei de Moisés, todo este mal nos
sobreveio; apesar disso, não suplicamos à face do Senhor nosso Deus,
para nos convertermos das nossas iniqüidades, e para nos aplicarmos
à tua verdade. Por isso o Senhor vigiou sobre o mal, e o
trouxe sobre nós; porque justo é o Senhor, nosso Deus, em todas as
suas obras, que fez, pois não obedecemos à sua voz. Agora,
pois, ó Senhor, nosso Deus, que tiraste o teu povo da terra do Egito
com mão poderosa, e ganhaste para ti nome, como hoje se vê; temos
pecado, temos procedido impiamente. Ó Senhor, segundo todas
as tuas justiças, aparte-se a tua ira e o teu furor da tua cidade de
Jerusalém, do teu santo monte; porque por causa dos nossos pecados,
e por causa das iniqüidades de nossos pais, tornou-se Jerusalém e o
teu povo um opróbrio para todos os que estão em redor de nós.
Agora, pois, ó Deus nosso, ouve a oração do teu servo, e as
suas súplicas, e sobre o teu santuário assolado faze resplandecer o
teu rosto, por amor do Senhor. Inclina, ó Deus meu, os teus
ouvidos, e ouve; abre os teus olhos, e olha para a nossa desolação,
e para a cidade que é chamada pelo teu nome, porque não lançamos as
nossas súplicas perante a tua face fiados em nossas justiças, mas em
tuas muitas misericórdias. Ó Senhor, ouve; ó Senhor, perdoa;
ó Senhor, atende-nos e age sem tardar; por amor de ti mesmo, ó Deus
meu; porque a tua cidade e o teu povo são chamados pelo teu nome.

Estando eu ainda falando e orando, e confessando o meu pecado, e
o pecado do meu povo Israel, e lançando a minha súplica perante a
face do Senhor, meu Deus, pelo monte santo do meu Deus,
estando eu, digo, ainda falando na oração, o homem Gabriel,
que eu tinha visto na minha visão ao princípio, veio, voando
rapidamente, e tocou-me, à hora do sacrifício da tarde. Ele
me instruiu, e falou comigo, dizendo: Daniel, agora saí para
fazer-te entender o sentido. No princípio das tuas súplicas,
saiu a ordem, e eu vim, para to declarar, porque és mui amado;
considera, pois, a palavra, e entende a visão. Setenta
semanas estão determinadas sobre o teu povo, e sobre a tua santa
cidade, para cessar a transgressão, e para dar fim aos pecados, e
para expiar a iniqüidade, e trazer a justiça eterna, e selar a visão
e a profecia, e para ungir o Santíssimo. Sabe e entende:
desde a saída da ordem para restaurar, e para edificar a Jerusalém,
até ao Messias, o Príncipe, haverá sete semanas, e sessenta e duas
semanas; as ruas e o muro se reedificarão, mas em tempos
angustiosos. E depois das sessenta e duas semanas será
cortado o Messias, mas não para si mesmo; e o povo do príncipe, que
há de vir, destruirá a cidade e o santuário, e o seu fim será com
uma inundação; e até ao fim haverá guerra; estão determinadas as
assolações. E ele firmará aliança com muitos por uma semana;
e na metade da semana fará cessar o sacrifício e a oblação; e sobre
a asa das abominações virá o assolador, e isso até à consumação; e o
que está determinado será derramado sobre o assolador.

\medskip

\lettrine{10} No terceiro ano de Ciro, rei da Pérsia, foi
revelada uma palavra a Daniel, cujo nome era Beltessazar; a palavra
era verdadeira e envolvia grande conflito; e ele entendeu esta
palavra, e tinha entendimento da visão. Naqueles dias eu,
Daniel, estive triste por três semanas. Alimento desejável não
comi, nem carne nem vinho entraram na minha boca, nem me ungi com
ungüento, até que se cumpriram as três semanas. E no dia vinte e
quatro do primeiro mês eu estava à borda do grande rio Hidequel;
e levantei os meus olhos, e olhei, e eis um homem vestido de
linho, e os seus lombos cingidos com ouro fino de Ufaz; e o seu
corpo era como berilo\footnote{Mineral hexagonal, silicato de
alumínio e glucínio, pedra semipreciosa.}, e o seu rosto parecia um
relâmpago, e os seus olhos como tochas de fogo, e os seus braços e
os seus pés brilhavam como bronze polido; e a voz das suas palavras
era como a voz de uma multidão. E só eu, Daniel, tive aquela
visão. Os homens que estavam comigo não a viram; contudo caiu sobre
eles um grande temor, e fugiram, escondendo-se. Fiquei, pois, eu
só, a contemplar esta grande visão, e não ficou força em mim;
transmudou-se o meu semblante em corrupção, e não tive força alguma.
Contudo ouvi a voz das suas palavras; e, ouvindo o som das suas
palavras, eu caí sobre o meu rosto num profundo sono, com o meu
rosto em terra.

E eis que certa mão me tocou, e fez com que me movesse sobre os
meus joelhos e sobre as palmas das minhas mãos. E me disse:
Daniel, homem muito amado, entende as palavras que vou te dizer, e
levanta-te sobre os teus pés, porque a ti sou enviado. E, falando
ele comigo esta palavra, levantei-me tremendo. Então me
disse: Não temas, Daniel, porque desde o primeiro dia em que
aplicaste o teu coração a compreender e a humilhar-te perante o teu
Deus, são ouvidas as tuas palavras; e eu vim por causa das tuas
palavras. Mas o príncipe do reino da Pérsia me resistiu vinte
e um dias, e eis que Miguel, um dos primeiros príncipes, veio para
ajudar-me, e eu fiquei ali com os reis da Pérsia. Agora vim,
para fazer-te entender o que há de acontecer ao teu povo nos
derradeiros dias; porque a visão é ainda para muitos dias. E,
falando ele comigo estas palavras, abaixei o meu rosto para a terra,
e emudeci. E eis que alguém, semelhante aos filhos dos
homens, tocou-me os lábios; então abri a minha boca, e falei,
dizendo àquele que estava em pé diante de mim: Senhor meu, por causa
da visão sobrevieram-me dores, e não me ficou força alguma.
Como, pois, pode o servo do meu senhor falar com o meu
senhor? Porque, quanto a mim, desde agora não resta força em mim, e
nem fôlego ficou em mim. E aquele, que tinha aparência de um
homem, tocou-me outra vez, e fortaleceu-me. E disse: Não
temas, homem muito amado, paz seja contigo; anima-te, sim, anima-te.
E, falando ele comigo, fiquei fortalecido, e disse: Fala, meu
senhor, porque me fortaleceste. E ele disse: Sabes por que eu
vim a ti? Agora, pois, tornarei a pelejar contra o príncipe dos
persas; e, saindo eu, eis que virá o príncipe da Grécia. Mas
eu te declararei o que está registrado na escritura da verdade; e
ninguém há que me anime contra aqueles, senão Miguel, vosso
príncipe.

\medskip

\lettrine{11} Eu, pois, no primeiro ano de Dario, o medo,
levantei-me para animá-lo e fortalecê-lo. E agora te declararei
a verdade: Eis que ainda três reis estarão na Pérsia, e o quarto
acumulará grandes riquezas, mais do que todos; e, tornando-se forte,
por suas riquezas, suscitará a todos contra o reino da Grécia.
Depois se levantará um rei valente, que reinará com grande
domínio, e fará o que lhe aprouver. Mas, estando ele em pé, o
seu reino será quebrado, e será repartido para os quatro ventos do
céu; mas não para a sua posteridade, nem tampouco segundo o seu
domínio com que reinou, porque o seu reino será arrancado, e passará
a outros que não eles.

E será forte o rei do sul; mas um dos seus príncipes será mais
forte do que ele, e reinará poderosamente; seu domínio será grande.
Mas, ao fim de alguns anos, eles se aliarão; e a filha do rei do
sul virá ao rei do norte para fazer um tratado; mas ela não reterá a
força do seu braço; nem ele persistirá, nem o seu braço, porque ela
será entregue, e os que a tiverem trazido, e seu pai, e o que a
fortalecia naqueles tempos. Mas de um renovo das raízes dela um
se levantará em seu lugar, e virá com o exército, e entrará na
fortaleza do rei do norte, e operará contra eles, e prevalecerá.
Também os seus deuses com as suas imagens de fundição, com os
seus objetos preciosos de prata e ouro, levará cativos para o Egito;
e por alguns anos ele persistirá contra o rei do norte. E
entrará no reino o rei do sul, e tornará para a sua terra.
Mas seus filhos intervirão e reunirão uma multidão de grandes
forças; e virá apressadamente e inundará, e passará adiante; e,
voltando levará a guerra até a sua fortaleza. Então o rei do
sul se exasperará, e sairá, e pelejará contra ele, contra o rei do
norte; este porá em campo grande multidão, e aquela multidão será
entregue na sua mão. A multidão será tirada e o seu coração
se elevará; mas ainda que derrubará muitos milhares, contudo não
prevalecerá. Porque o rei do norte tornará, e porá em campo
uma multidão maior do que a primeira, e ao fim dos tempos, isto é,
de anos, virá à pressa com grande exército e com muitas riquezas.
E, naqueles tempos, muitos se levantarão contra o rei do sul;
e os violentos dentre o teu povo se levantarão para cumprir a visão,
mas eles cairão. E o rei do norte virá, e levantará
baluartes, e tomará a cidade forte; e os braços do sul não poderão
resistir, nem o seu povo escolhido, pois não haverá força para
resistir. O que, pois, há de vir contra ele fará segundo a
sua vontade, e ninguém poderá resistir diante dele; e estará na
terra gloriosa, e por sua mão haverá destruição. E dirigirá o
seu rosto, para vir com a potência de todo o seu reino, e com ele os
retos, assim ele fará; e lhe dará uma filha das mulheres, para
corrompê-la; ela, porém, não subsistirá, nem será para ele.
Depois virará o seu rosto para as ilhas, e tomará muitas; mas
um príncipe fará cessar o seu opróbrio contra ele, e ainda fará
recair sobre ele o seu opróbrio. Virará então o seu rosto
para as fortalezas da sua própria terra, mas tropeçará, e cairá, e
não será achado. E em seu lugar se levantará quem fará passar
um arrecadador pela glória do reino; mas em poucos dias será
quebrantado, e isto sem ira e sem batalha.

Depois se levantará em seu lugar um homem vil, ao qual não tinham
dado a dignidade real; mas ele virá caladamente, e tomará o reino
com engano. E com os braços de uma inundação serão varridos
de diante dele; e serão quebrantados, como também o príncipe da
aliança. E, depois do concerto com ele, usará de engano; e
subirá, e se tornará forte com pouca gente. Virá também
caladamente aos lugares mais férteis da província, e fará o que
nunca fizeram seus pais, nem os pais de seus pais; repartirá entre
eles a presa e os despojos, e os bens, e formará os seus projetos
contra as fortalezas, mas por certo tempo. E suscitará a sua
força e a sua coragem contra o rei do sul com um grande exército; e
o rei do sul se envolverá na guerra com um grande e mui poderoso
exército; mas não subsistirá, porque maquinarão projetos contra ele.
E os que comerem os seus alimentos o destruirão; e o exército
dele será arrasado, e cairão muitos mortos. Também estes dois
reis terão o coração atento para fazerem o mal, e a uma mesma mesa
falarão a mentira; mas isso não prosperará, porque ainda verá o fim
no tempo determinado. Então tornará para a sua terra com
muitos bens, e o seu coração será contra a santa aliança; e fará o
que lhe aprouver, e tornará para a sua terra. No tempo
determinado tornará a vir em direção do sul; mas não será na última
vez como foi na primeira. Porque virão contra ele navios de
Quitim, que lhe causarão tristeza; e voltará, e se indignará contra
a santa aliança, e fará o que lhe aprouver; voltará e atenderá aos
que tiverem abandonado a santa aliança. E braços serão
colocados sobre ele, que profanarão o santuário e a fortaleza, e
tirarão o sacrifício contínuo, estabelecendo abominação desoladora.
E aos violadores da aliança ele com lisonjas perverterá, mas
o povo que conhece ao seu Deus se tornará forte e fará proezas.
E os entendidos entre o povo ensinarão a muitos; todavia
cairão pela espada, e pelo fogo, e pelo cativeiro, e pelo roubo, por
muitos dias. E, caindo eles, serão ajudados com pequeno
socorro; mas muitos se ajuntarão a eles com lisonjas. E
alguns dos entendidos cairão, para serem provados, purificados, e
embranquecidos, até ao fim do tempo, porque será ainda para o tempo
determinado. E este rei fará conforme a sua vontade, e
levantar-se-á, e engrandecer-se-á sobre todo deus; e contra o Deus
dos deuses falará coisas espantosas, e será próspero, até que a ira
se complete; porque aquilo que está determinado será feito. E
não terá respeito ao Deus de seus pais, nem terá respeito ao amor
das mulheres, nem a deus algum, porque sobre tudo se engrandecerá.
Mas em seu lugar honrará a um deus das forças; e a um deus a
quem seus pais não conheceram honrará com ouro, e com prata, e com
pedras preciosas, e com coisas agradáveis. Com o auxílio de
um deus estranho agirá contra as poderosas fortalezas; aos que o
reconhecerem multiplicará a honra, e os fará reinar sobre muitos, e
repartirá a terra por preço. E, no fim do tempo, o rei do sul
lutará com ele, e o rei do norte se levantará contra ele com carros,
e com cavaleiros, e com muitos navios; e entrará nas suas terras e
as inundará, e passará. E entrará na terra gloriosa, e muitos
países cairão, mas da sua mão escaparão estes: Edom e Moabe, e os
chefes dos filhos de Amom. E estenderá a sua mão contra os
países, e a terra do Egito não escapará. E apoderar-se-á dos
tesouros de ouro e de prata e de todas as coisas preciosas do Egito;
e os líbios e os etíopes o seguirão. Mas os rumores do
oriente e do norte o espantarão; e sairá com grande furor, para
destruir e extirpar a muitos. E armará as tendas do seu
palácio entre o mar grande e o monte santo e glorioso; mas chegará
ao seu fim, e não haverá quem o socorra.

\medskip

\lettrine{12} E naquele tempo se levantará Miguel, o grande
príncipe, que se levanta a favor dos filhos do teu povo, e haverá um
tempo de angústia, qual nunca houve, desde que houve nação até
àquele tempo; mas naquele tempo livrar-se-á o teu povo, todo aquele
que for achado escrito no livro. E muitos dos que dormem no pó
da terra ressuscitarão, uns para vida eterna, e outros para vergonha
e desprezo eterno. Os que forem sábios, pois, resplandecerão
como o fulgor do firmamento; e os que a muitos ensinam a justiça,
como as estrelas sempre e eternamente. E tu, Daniel, encerra
estas palavras e sela este livro, até ao fim do tempo; muitos
correrão de uma parte para outra, e o conhecimento se multiplicará.

Então eu, Daniel, olhei, e eis que estavam em pé outros dois, um
deste lado, à beira do rio, e o outro do outro lado, à beira do rio.
E ele disse ao homem vestido de linho, que estava sobre as águas
do rio: Quando será o fim destas maravilhas? E ouvi o homem
vestido de linho, que estava sobre as águas do rio, o qual levantou
ao céu a sua mão direita e a sua mão esquerda, e jurou por aquele
que vive eternamente que isso seria para um tempo, tempos e metade
do tempo, e quando tiverem acabado de espalhar o poder do povo
santo, todas estas coisas serão cumpridas. Eu, pois, ouvi, mas
não entendi; por isso eu disse: Senhor meu, qual será o fim destas
coisas? E ele disse: Vai, Daniel, porque estas palavras estão
fechadas e seladas até ao tempo do fim. Muitos serão
purificados, e embranquecidos, e provados; mas os ímpios procederão
impiamente, e nenhum dos ímpios entenderá, mas os sábios entenderão.
E desde o tempo em que o sacrifício contínuo for tirado, e
posta a abominação desoladora, haverá mil duzentos e noventa dias.
Bem-aventurado o que espera e chega até mil trezentos e
trinta e cinco dias. Tu, porém, vai até ao fim; porque
descansarás, e te levantarás na tua herança, no fim dos dias.

