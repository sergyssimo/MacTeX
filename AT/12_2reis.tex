\addchap{Segundo livro de Reis}

\lettrine{1} E depois da morte de Acabe, Moabe se rebelou
contra Israel. E caiu Acazias pelas grades de um quarto alto,
que tinha em Samaria, e adoeceu; e enviou mensageiros, e disse-lhes:
Ide, e perguntai a Baal-Zebube, deus de Ecrom, se sararei desta
doença. Mas o anjo do Senhor disse a Elias, o tisbita:
Levanta-te, sobe para te encontrares com os mensageiros do rei de
Samaria, e dize-lhes: Porventura não há Deus em Israel, para irdes
consultar a Baal-Zebube, deus de Ecrom? E por isso assim diz o
Senhor: Da cama, a que subiste, não descerás, mas sem falta
morrerás. Então Elias partiu. E os mensageiros voltaram para
ele; e ele lhes disse: Que há, que voltastes? E eles lhe
disseram: Um homem saiu ao nosso encontro, e nos disse: Ide, voltai
para o rei que vos mandou, e dizei-lhe: Assim diz o Senhor:
Porventura não há Deus em Israel, para que mandes consultar a
Baal-Zebube, deus de Ecrom? Portanto da cama, a que subiste, não
descerás, mas sem falta morrerás. E ele lhes disse: Qual era a
aparência do homem que veio ao vosso encontro e vos falou estas
palavras? E eles lhe disseram: Era um homem peludo, e com os
lombos cingidos de um cinto de couro. Então disse ele: É Elias, o
tisbita.

Então o rei lhe enviou um capitão de cinqüenta com seus cinqüenta;
e, subindo a ele (porque eis que estava assentado no cume do monte),
disse-lhe: Homem de Deus, o rei diz: Desce. Mas Elias
respondeu, e disse ao capitão de cinqüenta: Se eu, pois, sou homem
de Deus, desça fogo do céu, e te consuma a ti e aos teus cinqüenta.
Então fogo desceu do céu, e consumiu a ele e aos seus cinqüenta.
E tornou o rei a enviar-lhe outro capitão de cinqüenta, com
os seus cinqüenta; ele lhe respondeu, dizendo: Homem de Deus, assim
diz o rei: Desce depressa. E respondeu Elias: Se eu sou homem
de Deus, desça fogo do céu, e te consuma a ti e aos teus cinqüenta.
Então o fogo de Deus desceu do céu, e o consumiu a ele e aos seus
cinqüenta. E tornou a enviar um terceiro capitão de
cinqüenta, com os seus cinqüenta; então subiu o capitão de cinqüenta
e, chegando, pôs-se de joelhos diante de Elias, e suplicou-lhe,
dizendo: Homem de Deus, seja, peço-te, preciosa aos teus olhos a
minha vida, e a vida destes cinqüenta teus servos. Eis que
fogo desceu do céu, e consumiu aqueles dois primeiros capitães de
cinqüenta, com os seus cinqüenta; porém, agora seja preciosa aos
teus olhos a minha vida. Então o anjo do Senhor disse a
Elias: Desce com este, não temas. E levantou-se, e desceu com ele ao
rei. E disse-lhe: Assim diz o Senhor: Por que enviaste
mensageiros a consultar a Baal-Zebube, deus de Ecrom? Porventura é
porque não há Deus em Israel, para consultar a sua palavra? Portanto
desta cama, a que subiste, não descerás, mas certamente morrerás.
Assim, pois, morreu, conforme a palavra do Senhor, que Elias
falara; e Jorão começou a reinar no seu lugar no ano segundo de
Jeorão, filho de Jeosafá, rei de Judá; porquanto não tinha filho.
O mais dos atos de Acazias, tudo quanto fez, porventura não
está escrito no livro das crônicas dos reis de Israel?

\medskip

\lettrine{2} Sucedeu que, quando o Senhor estava para elevar a
Elias num redemoinho ao céu, Elias partiu de Gilgal com Eliseu.
E disse Elias a Eliseu: Fica-te aqui, porque o Senhor me enviou
a Betel. Porém Eliseu disse: Vive o Senhor, e vive a tua alma, que
não te deixarei. E assim foram a Betel. Então os filhos dos
profetas que estavam em Betel saíram ao encontro de Eliseu, e lhe
disseram: Sabes que o Senhor hoje tomará o teu senhor por sobre a
tua cabeça? E ele disse: Também eu bem o sei; calai-vos. E Elias
lhe disse: Eliseu, fica-te aqui, porque o Senhor me enviou a Jericó.
Porém ele disse: Vive o Senhor, e vive a tua alma, que não te
deixarei. E assim foram a Jericó. Então os filhos dos profetas
que estavam em Jericó se chegaram a Eliseu, e lhe disseram: Sabes
que o Senhor hoje tomará o teu Senhor por sobre a tua cabeça? E ele
disse: Também eu bem o sei; calai-vos. E Elias disse: Fica-te
aqui, porque o Senhor me enviou ao Jordão. Mas ele disse: Vive o
Senhor, e vive a tua alma, que não te deixarei. E assim ambos foram
juntos. E foram cinqüenta homens dos filhos dos profetas, e
pararam defronte deles, de longe: e assim ambos pararam junto ao
Jordão. Então Elias tomou a sua capa e a dobrou, e feriu as
águas, as quais se dividiram para os dois lados; e passaram ambos em
seco.

Sucedeu que, havendo eles passado, Elias disse a Eliseu: Pede-me o
que queres que te faça, antes que seja tomado de ti. E disse Eliseu:
Peço-te que haja porção dobrada de teu espírito sobre mim. E
disse: Coisa difícil pediste; se me vires quando for tomado de ti,
assim se te fará, porém, se não, não se fará. E sucedeu que,
indo eles andando e falando, eis que um carro de fogo, com cavalos
de fogo, os separou um do outro; e Elias subiu ao céu num
redemoinho. O que vendo Eliseu, clamou: Meu pai, meu pai,
carros de Israel, e seus cavaleiros! E nunca mais o viu; e, pegando
as suas vestes, rasgou-as em duas partes.

Também levantou a capa de Elias, que dele caíra; e, voltando-se,
parou à margem do Jordão. E tomou a capa de Elias, que dele
caíra, e feriu as águas, e disse: Onde está o Senhor Deus de Elias?
Quando feriu as águas elas se dividiram de um ao outro lado; e
Eliseu passou. Vendo-o, pois, os filhos dos profetas que
estavam defronte em Jericó, disseram: O espírito de Elias repousa
sobre Eliseu. E vieram-lhe ao encontro, e se prostraram diante dele
em terra. E disseram-lhe: Eis que agora entre os teus servos
há cinqüenta homens valentes; ora deixa-os ir para buscar a teu
senhor; pode ser que o elevasse o Espírito do Senhor e o lançasse em
algum dos montes, ou em algum dos vales. Porém ele disse: Não os
envieis. Mas eles insistiram com ele, até que, constrangido,
disse-lhes: Enviai. E enviaram cinqüenta homens, que o buscaram três
dias, porém não o acharam. Então voltaram para ele, pois
ficara em Jericó; e disse-lhes: Eu não vos disse que não fosseis?

E os homens da cidade disseram a Eliseu: Eis que é boa a situação
desta cidade, como o meu senhor vê; porém as águas são más, e a
terra é estéril. E ele disse: Trazei-me um prato novo, e
ponde nele sal. E lho trouxeram. Então saiu ele ao manancial
das águas, e deitou sal nele; e disse: Assim diz o Senhor: Sararei a
estas águas; e não haverá mais nelas morte nem esterilidade.
Ficaram, pois, sãs aquelas águas, até ao dia de hoje,
conforme a palavra que Eliseu tinha falado. Então subiu dali
a Betel; e, subindo ele pelo caminho, uns meninos saíram da cidade,
e zombavam dele, e diziam-lhe: Sobe, calvo; sobe, calvo! E,
virando-se ele para trás, os viu, e os amaldiçoou no nome do Senhor;
então duas ursas saíram do bosque, e despedaçaram quarenta e dois
daqueles meninos. E dali foi para o monte Carmelo de onde
voltou para Samaria.

\medskip

\lettrine{3} E Jorão, filho de Acabe, começou a reinar sobre
Israel, em Samaria, no décimo oitavo ano de Jeosafá, rei de Judá; e
reinou doze anos. E fez o que era mau aos olhos do Senhor; porém
não como seu pai, nem como sua mãe; porque tirou a estátua de Baal,
que seu pai fizera. Contudo aderiu aos pecados de Jeroboão,
filho de Nebate, com que fizera Israel pecar; não se apartou deles.
Então Mesa, rei dos moabitas, era criador de gado, e pagava de
tributo, ao rei de Israel, cem mil cordeiros, e cem mil carneiros
com a sua lã. Sucedeu, porém, que, morrendo Acabe, o rei dos
moabitas se rebelou contra o rei de Israel.

Por isso Jorão ao mesmo tempo saiu de Samaria, e fez revista de
todo o Israel. E foi, e mandou dizer a Jeosafá, rei de Judá: O
rei dos moabitas se rebelou contra mim; irás tu comigo à guerra
contra os moabitas? E disse ele: Subirei; e eu serei como tu, o meu
povo como o teu povo, e os meus cavalos como os teus cavalos. E
ele disse: Por que caminho subiremos? Então disse ele: Pelo caminho
do deserto de Edom. E partiram o rei de Israel, o rei de Judá e
o rei de Edom; e andaram rodeando com uma marcha de sete dias, e não
havia água para o exército e nem para o gado que os seguia.
Então disse o rei de Israel: Ah! o Senhor chamou a estes três
reis, para entregá-los nas mãos dos moabitas. E disse
Jeosafá: Não há aqui algum profeta do Senhor, para que consultemos
ao Senhor por ele? Então respondeu um dos servos do rei de Israel,
dizendo: Aqui está Eliseu, filho de Safate, que derramava água sobre
as mãos de Elias. E disse Jeosafá: Está com ele a palavra do
Senhor. Então o rei de Israel, Jeosafá, e o rei de Edom desceram a
ter com ele. Mas Eliseu disse ao rei de Israel: Que tenho eu
contigo? Vai aos profetas de teu pai e aos profetas de tua mãe.
Porém o rei de Israel lhe disse: Não, porque o Senhor chamou a estes
três reis para entregá-los nas mãos dos moabitas. E disse
Eliseu: Vive o Senhor dos Exércitos, em cuja presença estou, que se
eu não respeitasse a presença de Jeosafá, rei de Judá, não olharia
para ti nem te veria. Ora, pois, trazei-me um músico. E
sucedeu que, tocando o músico, veio sobre ele a mão do Senhor.
E disse: Assim diz o Senhor: Fazei neste vale muitas covas.
Porque assim diz o Senhor: Não vereis vento, e não vereis
chuva; todavia este vale se encherá de tanta água, que bebereis vós,
o vosso gado e os vossos animais. E ainda isto é pouco aos
olhos do Senhor; também entregará ele os moabitas nas vossas mãos.
E ferireis a todas as cidades fortes, e a todas as cidades
escolhidas, e todas as boas árvores cortareis, e entupireis todas as
fontes de água, e danificareis com pedras todos os bons campos.

E sucedeu que, pela manhã, oferecendo-se a oferta de alimentos,
eis que vinham as águas pelo caminho de Edom; e a terra se encheu de
água. Ouvindo, pois, todos os moabitas que os reis tinham
subido para pelejarem contra eles, convocaram a todos os que estavam
em idade de cingir cinto e daí para cima, e puseram-se às
fronteiras. E, levantando-se de madrugada, e saindo o sol
sobre as águas, viram os moabitas, defronte deles, as águas
vermelhas como sangue. E disseram: Isto é sangue; certamente
que os reis se destruíram à espada e se mataram um ao outro! Agora,
pois, à presa, moabitas! Porém, chegando eles ao arraial de
Israel, os israelitas se levantaram, e feriram os moabitas, os quais
fugiram diante deles e ainda entraram nas suas terras, ferindo ali
também os moabitas. E arrasaram as cidades, e cada um lançou
a sua pedra em todos os bons campos, e os entulharam, e entupiram
todas as fontes de água, e cortaram todas as boas árvores, até que
só em Quir-Haresete deixaram ficar as pedras, mas os fundeiros a
cercaram e a feriram. Mas, vendo o rei dos moabitas que a
peleja prevalecia contra ele, tomou consigo setecentos homens que
sacavam espada, para romperem contra o rei de Edom, porém não
puderam. Então tomou a seu filho primogênito, que havia de
reinar em seu lugar, e o ofereceu em holocausto sobre o muro; pelo
que houve grande indignação em Israel; por isso retiraram-se dele, e
voltaram para a sua terra.

\medskip

\lettrine{4} E uma mulher, das mulheres dos filhos dos
profetas, clamou a Eliseu, dizendo: Meu marido, teu servo, morreu; e
tu sabes que o teu servo temia ao Senhor; e veio o credor, para
levar os meus dois filhos para serem servos. E Eliseu lhe disse:
Que te hei de fazer? Dize-me que é o que tens em casa. E ela disse:
Tua serva não tem nada em casa, senão uma botija de azeite.
Então disse ele: Vai, pede emprestadas, de todos os teus
vizinhos, vasilhas vazias, não poucas. Então entra, e fecha a
porta sobre ti, e sobre teus filhos, e deita o azeite em todas
aquelas vasilhas, e põe à parte a que estiver cheia. Partiu,
pois, dele, e fechou a porta sobre si e sobre seus filhos; e eles
lhe traziam as vasilhas, e ela as enchia. E sucedeu que, cheias
que foram as vasilhas, disse a seu filho: Traze-me ainda uma
vasilha. Porém ele lhe disse: Não há mais vasilha alguma. Então o
azeite parou. Então veio ela, e o fez saber ao homem de Deus; e
disse ele: Vai, vende o azeite, e paga a tua dívida; e tu e teus
filhos vivei do resto.

Sucedeu também um dia que, indo Eliseu a Suném, havia ali uma
mulher importante, a qual o reteve para comer pão; e sucedeu que
todas as vezes que passava por ali entrava para comer pão. E ela
disse a seu marido: Eis que tenho observado que este que sempre
passa por nós é um santo homem de Deus. Façamos-lhe, pois, um
pequeno quarto junto ao muro, e ali lhe ponhamos uma cama, uma mesa,
uma cadeira e um candeeiro; e há de ser que, vindo ele a nós, para
ali se recolherá. E sucedeu que um dia ele chegou ali, e
recolheu-se àquele quarto, e se deitou. Então disse ao seu
servo Geazi: Chama esta sunamita. E chamando-a ele, ela se pôs
diante dele. Porque ele tinha falado a Geazi: Dize-lhe: Eis
que tu nos tens tratado com todo o desvelo; que se há de fazer por
ti? Haverá alguma coisa de que se fale por ti ao rei, ou ao capitão
do exército? E disse ela: Eu habito no meio do meu povo.
Então disse ele: Que se há de fazer por ela? E Geazi disse:
Ora ela não tem filho, e seu marido é velho. Por isso disse
ele: Chama-a. E, chamando-a ele, ela se pôs à porta. E ele
disse: A este tempo determinado, segundo o tempo da vida, abraçarás
um filho. E disse ela: Não, meu Senhor, homem de Deus, não mintas à
tua serva. E concebeu a mulher, e deu à luz um filho, no
tempo determinado, no ano seguinte, segundo Eliseu lhe dissera.

E, crescendo o filho, sucedeu que um dia saiu para ter com seu
pai, que estava com os segadores, e disse a seu pai: Ai, a
minha cabeça! Ai, a minha cabeça! Então disse a um moço: Leva-o à
sua mãe. E ele o tomou, e o levou à sua mãe; e esteve sobre
os seus joelhos até ao meio dia, e morreu. E subiu ela, e o
deitou sobre a cama do homem de Deus; e fechou a porta, e saiu.
E chamou a seu marido, e disse: Manda-me já um dos moços, e
uma das jumentas, para que eu corra ao homem de Deus, e volte.
E disse ele: Por que vais a ele hoje? Não é lua nova nem
sábado. E ela disse: Tudo vai bem. Então albardou a jumenta,
e disse ao seu servo: Guia e anda, e não te detenhas no caminhar,
senão quando eu to disser. Partiu ela, pois, e foi ao homem
de Deus, ao monte Carmelo; e sucedeu que, vendo-a o homem de Deus de
longe, disse a Geazi, seu servo: Eis aí a sunamita. Agora,
pois, corre-lhe ao encontro e dize-lhe: Vai bem contigo? Vai bem com
teu marido? Vai bem com teu filho? E ela disse: Vai bem.
Chegando ela, pois, ao homem de Deus, ao monte, pegou nos
seus pés; mas chegou Geazi para retirá-la; disse porém o homem de
Deus: Deixa-a, porque a sua alma está triste de amargura, e o Senhor
me encobriu, e não me manifestou. E disse ela: Pedi eu a meu
Senhor algum filho? Não disse eu: Não me enganes? E ele disse
a Geazi: Cinge os teus lombos, toma o meu bordão na tua mão, e vai;
se encontrares alguém não o saúdes, e se alguém te saudar, não lhe
respondas; e põe o meu bordão sobre o rosto do menino. Porém
disse a mãe do menino: Vive o Senhor, e vive a tua alma, que não te
hei de deixar. Então ele se levantou, e a seguiu. E Geazi
passou adiante deles, e pôs o bordão sobre o rosto do menino; porém
não havia nele voz nem sentido; e voltou a encontrar-se com ele, e
lhe trouxe aviso, dizendo: O menino não despertou. E,
chegando Eliseu àquela casa, eis que o menino jazia morto sobre a
sua cama. Então entrou ele, e fechou a porta sobre eles
ambos, e orou ao Senhor. E subiu à cama e deitou-se sobre o
menino, e, pondo a sua boca sobre a boca dele, e os seus olhos sobre
os olhos dele, e as suas mãos sobre as mãos dele, se estendeu sobre
ele; e a carne do menino aqueceu. Depois desceu, e andou
naquela casa de uma parte para a outra, e tornou a subir, e se
estendeu sobre ele, então o menino espirrou sete vezes, e abriu os
olhos. Então chamou a Geazi, e disse: Chama esta sunamita. E
chamou-a, e veio a ele. E disse ele: Toma o teu filho. E
entrou ela, e se prostrou a seus pés, e se inclinou à terra; e tomou
o seu filho e saiu.

E, voltando Eliseu a Gilgal, havia fome naquela terra, e os
filhos dos profetas estavam assentados na sua presença; e disse ao
seu servo: Põe a panela grande ao lume\footnote{Fogo.}, e faze um
caldo de ervas para os filhos dos profetas. Então um deles
saiu ao campo a apanhar ervas, e achou uma parra\footnote{Ramo de
videira; pâmpano.} brava\footnote{Diz-se da planta espontânea, em
oposição à cultivada.}, e colheu dela enchendo a sua capa de
colocíntidas\footnote{Ou coloquíntida: trepadeira ornamental, da
família das cucurbitáceas (Cucurbita pepo), de flores com corola
amarela e monopétala, e frutos com manchas amarelas e verde-escuras,
de várias formas.}; e veio, e as cortou na panela do caldo; porque
não as conheciam. Assim deram de comer para os homens. E
sucedeu que, comendo eles daquele caldo, clamaram e disseram: Homem
de Deus, há morte na panela. Não puderam comer. Porém ele
disse: Trazei farinha. E deitou-a na panela, e disse: Dai de comer
ao povo. E já não havia mal nenhum na panela. E um homem veio
de Baal-Salisa, e trouxe ao homem de Deus pães das primícias, vinte
pães de cevada, e espigas verdes na sua palha, e disse: Dá ao povo,
para que coma. Porém seu servo disse: Como hei de pôr isto
diante de cem homens? E disse ele: Dá ao povo, para que coma; porque
assim diz o Senhor: Comerão, e sobejará. Então lhos pôs
diante, e comeram e ainda sobrou, conforme a palavra do Senhor.

\medskip

\lettrine{5} E Naamã, capitão do exército do rei da Síria, era
um grande homem diante do seu senhor, e de muito respeito; porque
por ele o Senhor dera livramento aos sírios; e era este homem herói
valoroso, porém leproso. E saíram tropas da Síria, da terra de
Israel, e levaram presa uma menina que ficou ao serviço da mulher de
Naamã. E disse esta à sua senhora: Antes o meu senhor estivesse
diante do profeta que está em Samaria; ele o restauraria da sua
lepra. Então foi Naamã e notificou ao seu senhor, dizendo: Assim
e assim falou a menina que é da terra de Israel. Então disse o
rei da Síria: Vai, anda, e enviarei uma carta ao rei de Israel. E
foi, e tomou na sua mão dez talentos de prata, seis mil siclos de
ouro e dez mudas de roupas. E levou a carta ao rei de Israel,
dizendo: Logo, em chegando a ti esta carta, saibas que eu te enviei
Naamã, meu servo, para que o cures da sua lepra. E sucedeu que,
lendo o rei de Israel a carta, rasgou as suas vestes, e disse: Sou
eu Deus, para matar e para vivificar, para que este envie a mim um
homem, para que eu o cure da sua lepra? Pelo que deveras notai,
peço-vos, e vede que busca ocasião contra mim. Sucedeu, porém,
que, ouvindo Eliseu, homem de Deus, que o rei de Israel rasgara as
suas vestes, mandou dizer ao rei: Por que rasgaste as tuas vestes?
Deixa-o vir a mim, e saberá que há profeta em Israel.

Veio, pois, Naamã com os seus cavalos, e com o seu carro, e parou
à porta da casa de Eliseu. Então Eliseu lhe mandou um
mensageiro, dizendo: Vai, e lava-te sete vezes no Jordão, e a tua
carne será curada e ficarás purificado. Porém, Naamã muito se
indignou, e se foi, dizendo: Eis que eu dizia comigo: Certamente ele
sairá, pôr-se-á em pé, invocará o nome do Senhor seu Deus, e passará
a sua mão sobre o lugar, e restaurará o leproso. Não são
porventura Abana e Farpar, rios de Damasco, melhores do que todas as
águas de Israel? Não me poderia eu lavar neles, e ficar purificado?
E voltou-se, e se foi com indignação. Então chegaram-se a ele
os seus servos, e lhe falaram, e disseram: Meu pai, se o profeta te
dissesse alguma grande coisa, porventura não a farias? Quanto mais,
dizendo-te ele: Lava-te, e ficarás purificado. Então desceu,
e mergulhou no Jordão sete vezes, conforme a palavra do homem de
Deus; e a sua carne tornou-se como a carne de um menino, e ficou
purificado.

Então voltou ao homem de Deus, ele e toda a sua comitiva, e
chegando, pôs-se diante dele, e disse: Eis que agora sei que em toda
a terra não há Deus senão em Israel; agora, pois, peço-te que
aceites uma bênção do teu servo. Porém ele disse: Vive o
Senhor, em cuja presença estou, que não a aceitarei. E instou com
ele para que a aceitasse, mas ele recusou. E disse Naamã: Se
não queres, dê-se a este teu servo uma carga de terra que baste para
carregar duas mulas; porque nunca mais oferecerá este teu servo
holocausto nem sacrifício a outros deuses, senão ao Senhor.
Nisto perdoe o Senhor a teu servo; quando meu senhor entrar
na casa de Rimom para ali adorar, e ele se encostar na minha mão, e
eu também tenha de me encurvar na casa de Rimom; quando assim me
encurvar na casa de Rimom, nisto perdoe o Senhor a teu servo.
E ele lhe disse: Vai em paz. E foi dele a uma pequena
distância.

Então Geazi, servo de Eliseu, homem de Deus, disse: Eis que meu
senhor poupou a este sírio Naamã, não recebendo da sua mão alguma
coisa do que trazia; porém, vive o Senhor que hei de correr atrás
dele, e receber dele alguma coisa. E foi Geazi a alcançar
Naamã; e Naamã, vendo que corria atrás dele, desceu do carro a
encontrá-lo, e disse-lhe: Vai tudo bem? E ele disse: Tudo vai
bem; meu senhor me mandou dizer: Eis que agora mesmo vieram a mim
dois jovens dos filhos dos profetas da montanha de Efraim; dá-lhes,
pois, um talento de prata e duas mudas de roupas. E disse
Naamã: Sê servido tomar dois talentos. E instou com ele, e amarrou
dois talentos de prata em dois sacos, com duas mudas de roupas; e
pô-los sobre dois dos seus servos, os quais os levaram diante dele.
E, chegando ele a certa altura, tomou-os das suas mãos, e os
depositou na casa; e despediu aqueles homens, e foram-se.
Então ele entrou, e pôs-se diante de seu senhor. E disse-lhe
Eliseu: Donde vens, Geazi? E disse: Teu servo não foi nem a uma nem
a outra parte. Porém ele lhe disse: Porventura não foi
contigo o meu coração, quando aquele homem voltou do seu carro a
encontrar-te? Era a ocasião para receberes prata, e para tomares
roupas, olivais e vinhas, ovelhas e bois, servos e servas?
Portanto a lepra de Naamã se pegará a ti e à tua descendência
para sempre. Então saiu de diante dele leproso, branco como a neve.

\medskip

\lettrine{6} 1 E DISSERAM os filhos dos profetas a Eliseu: Eis
que o lugar em que habitamos diante da tua face, nos é estreito.
Vamos, pois, até ao Jordão e tomemos de lá, cada um de nós, uma
viga, e façamo-nos ali um lugar para habitar. E disse ele: Ide.
E disse um: Serve-te de ires com os teus servos. E disse: Eu
irei. E foi com eles; e, chegando eles ao Jordão, cortaram
madeira. E sucedeu que, derrubando um deles uma viga, o ferro
caiu na água; e clamou, e disse: Ai, meu senhor! ele era emprestado.
E disse o homem de Deus: Onde caiu? E mostrando-lhe ele o lugar,
cortou um pau, e o lançou ali, e fez flutuar o ferro. E disse:
Levanta-o. Então ele estendeu a sua mão e o tomou.

E o rei da Síria fazia guerra a Israel; e consultou com os seus
servos, dizendo: Em tal e tal lugar estará o meu acampamento.
Mas o homem de Deus enviou ao rei de Israel, dizendo: Guarda-te
de passares por tal lugar; porque os sírios desceram ali. Por
isso o rei de Israel enviou àquele lugar, de que o homem de Deus lhe
dissera, e de que o tinha avisado, e se guardou ali, não uma nem
duas vezes. Então se turbou com este incidente o coração do
rei da Síria, chamou os seus servos, e lhes disse: Não me fareis
saber quem dos nossos é pelo rei de Israel? E disse um dos
servos: Não, ó rei meu senhor; mas o profeta Eliseu, que está em
Israel, faz saber ao rei de Israel as palavras que tu falas no teu
quarto de dormir.

E ele disse: Vai, e vê onde ele está, para que envie, e mande
trazê-lo. E fizeram-lhe saber, dizendo: Eis que está em Dotã.
Então enviou para lá cavalos, e carros, e um grande exército,
os quais chegaram de noite, e cercaram a cidade. E o servo do
homem de Deus se levantou muito cedo e saiu, e eis que um exército
tinha cercado a cidade com cavalos e carros; então o seu servo lhe
disse: Ai, meu senhor! Que faremos? E ele disse: Não temas;
porque mais são os que estão conosco do que os que estão com eles.
E orou Eliseu, e disse: Senhor, peço-te que lhe abras os
olhos, para que veja. E o Senhor abriu os olhos do moço, e viu; e
eis que o monte estava cheio de cavalos e carros de fogo, em redor
de Eliseu. E, como desceram a ele, Eliseu orou ao Senhor e
disse: Fere, peço-te, esta gente de cegueira. E feriu-a de cegueira,
conforme a palavra de Eliseu. Então Eliseu lhes disse: Não é
este o caminho, nem é esta a cidade; segui-me, e guiar-vos-ei ao
homem que buscais. E os guiou a Samaria. E sucedeu que,
chegando eles a Samaria, disse Eliseu: Ó Senhor, abre a estes os
olhos para que vejam. O Senhor lhes abriu os olhos, para que vissem,
e eis que estavam no meio de Samaria. E, quando o rei de
Israel os viu, disse a Eliseu: Feri-los-ei, feri-los-ei, meu pai?
Mas ele disse: Não os ferirás; feririas tu os que tomasses
prisioneiros com a tua espada e com o teu arco? Põe-lhes diante pão
e água, para que comam e bebam, e se vão para seu senhor. E
apresentou-lhes um grande banquete, e comeram e beberam; e os
despediu e foram para seu senhor; e não entraram mais tropas de
sírios na terra de Israel.

E sucedeu, depois disto, que Ben-Hadade, rei da Síria, ajuntou
todo o seu exército; e subiu e cercou a Samaria. E houve
grande fome em Samaria, porque eis que a cercaram, até que se vendeu
uma cabeça de um jumento por oitenta peças de prata, e a quarta
parte de um cabo de esterco de pombas por cinco peças de
prata.\footnote{RA: (...) e um pouco de esterco de pombas por cinco
siclos de prata.} E sucedeu que, passando o rei pelo muro,
uma mulher lhe bradou, dizendo: Acode-me, ó rei meu senhor. E
ele lhe disse: Se o Senhor te não acode, donde te acudirei eu? Da
eira ou do lagar? Disse-lhe mais o rei: Que tens? E disse
ela: Esta mulher me disse: Dá cá o teu filho, para que hoje o
comamos, e amanhã comeremos o meu filho. Cozemos, pois, o meu
filho, e o comemos; mas dizendo-lhe eu ao outro dia: Dá cá o teu
filho, para que o comamos; escondeu o seu filho. E sucedeu
que, ouvindo o rei as palavras desta mulher, rasgou as suas vestes,
e ia passando pelo muro; e o povo viu que o rei trazia cilício por
dentro, sobre a sua carne, e disse: Assim me faça Deus, e
outro tanto, se a cabeça de Eliseu, filho de Safate, hoje ficar
sobre ele. Estava então Eliseu assentado em sua casa, e
também os anciãos estavam assentados com ele. E enviou o rei um
homem adiante de si; mas, antes que o mensageiro viesse a ele, disse
ele aos anciãos: Vistes como o filho do homicida mandou tirar-me a
cabeça? Olhai pois que, quando vier o mensageiro, fechai-lhe a
porta, e empurrai-o para fora com a porta; porventura não vem, após
ele, o ruído dos pés de seu senhor? E, estando ele ainda
falando com eles, eis que o mensageiro descia a ele; e disse: Eis
que este mal vem do Senhor, que mais, pois, esperaria do Senhor?

\medskip

\lettrine{7} Então disse Eliseu: Ouvi a palavra do Senhor;
assim diz o Senhor: Amanhã, quase a este tempo, haverá uma medida de
farinha por um siclo, e duas medidas de cevada por um siclo, à porta
de Samaria. Porém um senhor, em cuja mão o rei se encostava,
respondeu ao homem de Deus e disse: Eis que ainda que o Senhor
fizesse janelas no céu, poder-se-ia fazer isso? E ele disse: Eis que
o verás com os teus olhos, porém disso não comerás.

E quatro homens leprosos estavam à entrada da porta, os quais
disseram uns aos outros: Para que estaremos nós aqui até morrermos?
Se dissermos: Entremos na cidade, há fome na cidade, e
morreremos aí; e se ficarmos aqui, também morreremos. Vamos nós,
pois, agora, e passemos para o arraial dos sírios; se nos deixarem
viver, viveremos, e se nos matarem, tão-somente morreremos. E
levantaram-se ao crepúsculo, para irem ao arraial dos sírios; e,
chegando à entrada do arraial dos sírios, eis que não havia ali
ninguém. Porque o Senhor fizera ouvir no arraial dos sírios
ruído de carros e ruído de cavalos, como o ruído de um grande
exército; de maneira que disseram uns aos outros: Eis que o rei de
Israel alugou contra nós os reis dos heteus e os reis dos egípcios,
para virem contra nós. Por isso se levantaram, e fugiram no
crepúsculo, e deixaram as suas tendas, os seus cavalos, os seus
jumentos e o arraial como estava; e fugiram para salvarem a sua
vida. Chegando, pois, estes leprosos à entrada do arraial,
entraram numa tenda, e comeram, beberam e tomaram dali prata, ouro e
roupas, e foram e os esconderam; então voltaram, e entraram em outra
tenda, e dali também tomaram alguma coisa e a esconderam. Então
disseram uns para os outros: Não fazemos bem; este dia é dia de boas
novas, e nos calamos; se esperarmos até à luz da manhã, algum mal
nos sobrevirá; por isso agora vamos, e o anunciaremos à casa do rei.
Vieram, pois, e bradaram aos porteiros da cidade, e lhes
anunciaram, dizendo: Fomos ao arraial dos sírios e eis que lá não
havia ninguém, nem voz de homem, porém só cavalos atados, jumentos
atados, e as tendas como estavam. E chamaram os porteiros, e
\emph{estes} o anunciaram dentro da casa do rei\footnote{SBTB: E
chamaram os porteiros, e o anunciaram dentro da casa do rei. King
James: ``And he called the porters; and they told it to the king's
house within''. RA: Então, os porteiros gritaram e fizeram anunciar
a nova no interior da casa do rei. RC - 1969: E chamaram os
porteiros, e \emph{estes} o anunciaram dentro da casa do rei.}.

E o rei se levantou de noite, e disse a seus servos: Agora vos
farei saber o que é que os sírios nos fizeram; bem sabem eles que
esfaimados\footnote{Famintos.} estamos, pelo que saíram do arraial,
a esconder-se pelo campo, dizendo: Quando saírem da cidade, então os
tomaremos vivos, e entraremos na cidade. Então um dos seus
servos respondeu e disse: Tomem-se, pois, cinco dos cavalos que
restam aqui dentro (eis que são como toda a multidão dos israelitas
que ficaram aqui; e eis que são como toda a multidão dos israelitas
que já pereceram) e enviemo-los, e vejamos. Tomaram, pois,
dois cavalos de carro; e o rei os enviou com mensageiros após o
exército dos sírios, dizendo: Ide, e vede. E foram após eles
até ao Jordão, e eis que todo o caminho estava cheio de roupas e de
aviamentos que os sírios, apressando-se, lançaram fora; e voltaram
os mensageiros e o anunciaram ao rei. Então saiu o povo, e
saqueou o arraial dos sírios; e havia uma medida de farinha por um
siclo, e duas medidas de cevada por um siclo, conforme a palavra do
Senhor. E pusera o rei à porta o senhor em cuja mão se
encostava; e o povo o atropelou na porta, e morreu, como falara o
homem de Deus, o que falou quando o rei descera a ele. Porque
assim sucedeu como o homem de Deus falara ao rei dizendo: Amanhã,
quase a este tempo, haverá duas medidas de cevada por um siclo, e
uma medida de farinha por um siclo, à porta de Samaria. E
aquele senhor respondeu ao homem de Deus, e disse: Eis que ainda que
o Senhor fizesse janelas no céu poderia isso suceder? E ele disse:
Eis que o verás com os teus olhos, porém dali não comerás. E
assim lhe sucedeu, porque o povo o atropelou à porta, e morreu.

\medskip

\lettrine{8} E falou Eliseu àquela mulher cujo filho ele
ressuscitara, dizendo: Levanta-te e vai, tu e a tua família, e
peregrina onde puderes peregrinar; porque o Senhor chamou a fome, a
qual também virá à terra por sete anos. E levantou-se a mulher,
e fez conforme a palavra do homem de Deus; porque foi ela com a sua
família, e peregrinou na terra dos filisteus sete anos. E
sucedeu que, ao fim dos sete anos, a mulher voltou da terra dos
filisteus, e saiu a clamar ao rei pela sua casa e pelas suas terras.
Ora o rei falava a Geazi, servo do homem de Deus, dizendo:
Conta-me, peço-te, todas as grandes obras que Eliseu tem feito.
E sucedeu que, contando ele ao rei como ressuscitara a um morto,
eis que a mulher cujo filho ressuscitara clamou ao rei pela sua casa
e pelas suas terras. Então disse Geazi: Ó rei meu senhor, esta é a
mulher, e este o seu filho a quem Eliseu ressuscitou. E o rei
perguntou à mulher, e ela lho contou. Então o rei lhe deu um
oficial, dizendo: Faze-lhe restituir tudo quanto era seu, e todas as
rendas das terras desde o dia em que deixou a terra até agora.

Depois veio Eliseu a Damasco, estando Ben-Hadade, rei da Síria,
doente; e lho anunciaram, dizendo: O homem de Deus é chegado aqui.
Então o rei disse a Hazael: Toma um presente na tua mão, e vai a
encontrar-te com o homem de Deus; e pergunta por ele ao Senhor,
dizendo: Hei de sarar desta doença? Foi, pois, Hazael a
encontrar-se com ele, e tomou um presente na sua mão, a saber: de
tudo o que de bom havia em Damasco, quarenta camelos carregados; e
veio, e se pôs diante dele e disse: Teu filho Ben-Hadade, rei da
Síria, me enviou a ti, a dizer: Sararei eu desta doença? E
Eliseu lhe disse: Vai, e dize-lhe: Certamente viverás. Porém, o
Senhor me tem mostrado que certamente morrerá. E afirmou a
sua vista, e fitou os olhos nele até se envergonhar; e o homem de
Deus chorou. Então disse Hazael: Por que chora o meu senhor?
E ele disse: Porque sei o mal que hás de fazer aos filhos de Israel;
porás fogo às suas fortalezas, e os seus jovens matarás à espada, e
os seus meninos despedaçarás, e as suas mulheres grávidas fenderás.
E disse Hazael: Pois, que é teu servo, que não é mais do que
um cão, para fazer tão grande coisa? E disse Eliseu: O Senhor me tem
mostrado que tu hás de ser rei da Síria. Então partiu de
Eliseu, e foi a seu senhor, o qual lhe disse: Que te disse Eliseu? E
disse ele: Disse-me que certamente viverás. E sucedeu que no
outro dia tomou um cobertor e o molhou na água, e o estendeu sobre o
seu rosto, e morreu; e Hazael reinou em seu lugar.

E no ano quinto de Jorão, filho de Acabe, rei de Israel, reinando
ainda Jeosafá em Judá, começou a reinar Jeorão, filho de Jeosafá,
rei de Judá. Era ele da idade de trinta e dois anos quando
começou a reinar, e oito anos reinou em Jerusalém. E andou no
caminho dos reis de Israel, como também fizeram os da casa de Acabe,
porque tinha por mulher a filha de Acabe, e fez o que era mal aos
olhos do Senhor. Porém o Senhor não quis destruir a Judá por
amor de Davi, seu servo, como lhe tinha falado que lhe daria, para
sempre, uma lâmpada, a ele e a seus filhos. Nos seus dias se
rebelaram os edomitas, contra o mando de Judá, e puseram sobre si um
rei. Por isso Jeorão passou a Zair, e todos os carros com
ele; e ele se levantou de noite, e feriu os edomitas que estavam ao
redor dele, e os capitães dos carros; e o povo foi para as suas
tendas. Todavia os edomitas ficaram rebeldes, contra o mando
de Judá, até ao dia de hoje; então, no mesmo tempo, Libna também se
rebelou. O mais dos atos de Jeorão, e tudo quanto fez,
porventura não está escrito no livro das crônicas de Judá? E
Jeorão dormiu com seus pais, e foi sepultado com seus pais na cidade
de Davi; e Acazias, seu filho, reinou em seu lugar.

No ano doze de Jorão, filho de Acabe, rei de Israel, começou a
reinar Acazias, filho de Jeorão , rei de Judá. Era Acazias de
vinte e dois anos de idade quando começou a reinar, e reinou um ano
em Jerusalém; e era o nome de sua mãe Atalia, filha de Onri, rei de
Israel. E andou no caminho da casa de Acabe, e fez o que era
mal aos olhos do Senhor, como a casa de Acabe, porque era genro da
casa de Acabe. E foi com Jorão, filho de Acabe, a Ramote de
Gileade, à peleja contra Hazael, rei da Síria; e os sírios feriram a
Jorão. Então voltou o rei Jorão para se curar, em Jizreel,
das feridas que os sírios lhe fizeram em Ramá, quando pelejou contra
Hazael, rei da Síria; e desceu Acazias, filho de Jeorão, rei de
Judá, para ver a Jorão, filho de Acabe, em Jizreel, porquanto estava
doente.

\medskip

\lettrine{9} Então o profeta Eliseu chamou um dos filhos dos
profetas, e lhe disse: Cinge os teus lombos; e toma este vaso de
azeite na tua mão, e vai a Ramote de Gileade; e, chegando lá, vê
onde está Jeú, filho de Jeosafá, filho de Ninsi; entra, e faze que
ele se levante do meio de seus irmãos, e leva-o à câmara interior.
E toma o vaso de azeite, e derrama-o sobre a sua cabeça, e dize:
Assim diz o Senhor: Ungi-te rei sobre Israel. Então abre a porta,
foge, e não te detenhas. Foi, pois, o moço, o jovem profeta, a
Ramote de Gileade. E, entrando ele, eis que os capitães do
exército estavam assentados ali; e disse: Capitão, tenho uma palavra
que te dizer. E disse Jeú: A qual de todos nós? E disse: A ti,
capitão! Então se levantou, entrou na casa, e derramou o azeite
sobre a sua cabeça, e disse: Assim diz o Senhor Deus de Israel:
Ungi-te rei sobre o povo do Senhor, sobre Israel. E ferirás a
casa de Acabe, teu senhor, para que eu vingue o sangue de meus
servos, os profetas, e o sangue de todos os servos do Senhor, da mão
de Jezabel. E toda a casa de Acabe perecerá; destruirei de Acabe
todo o homem, tanto o encerrado como o absolvido em Israel.
Porque à casa de Acabe hei de fazer como à casa de Jeroboão,
filho de Nebate, e como à casa de Baasa, filho de Aías. E os
cães comerão a Jezabel no pedaço de campo de Jizreel; não haverá
quem a enterre. Então abriu a porta e fugiu.

E, saindo Jeú aos servos de seu senhor, disseram-lhe: Vai tudo
bem? Por que veio a ti este louco? E ele lhes disse: Bem conheceis o
homem e o seu falar. Mas eles disseram: É mentira; agora
faze-nos saber. E disse: Assim e assim me falou, a saber: Assim diz
o Senhor: Ungi-te rei sobre Israel. Então se apressaram,
tomando cada um a sua roupa puseram debaixo dele, no mais alto
degrau; e tocaram a buzina e disseram: Jeú reina! Assim Jeú,
filho de Jeosafá, filho de Ninsi, conspirou contra Jorão. Tinha,
porém, Jorão cercado a Ramote de Gileade, ele e todo o Israel, por
causa de Hazael, rei da Síria. Porém o rei Jorão voltou para
se curar em Jizreel das feridas que os sírios lhe fizeram, quando
pelejou contra Hazael, rei da Síria. E disse Jeú: Se é da vossa
vontade, ninguém saia da cidade, nem escape, para ir denunciar isto
em Jizreel.

Então Jeú subiu a um carro, e foi a Jizreel, porque Jorão estava
deitado ali; e também Acazias, rei de Judá, descera para ver a
Jorão. E o atalaia estava na torre de Jizreel, e viu a tropa
de Jeú, que vinha, e disse: Vejo uma tropa. Então disse Jorão: Toma
um cavaleiro, e envia-lho ao encontro; e diga: Há paz? E o
cavaleiro lhe foi ao encontro, e disse: Assim diz o rei: Há paz? E
disse Jeú: Que tens tu que fazer com a paz? Passa-te para trás de
mim. E o atalaia o fez saber, dizendo: Chegou a eles o mensageiro,
porém não volta. Então enviou outro cavaleiro; e, chegando
este a eles, disse: Assim diz o rei: Há paz? E disse Jeú: Que tens
tu que fazer com a paz? Passa-te para trás de mim. E o
atalaia o fez saber, dizendo: Também este chegou a eles, porém não
volta; e o andar parece como o andar de Jeú, filho de Ninsi, porque
anda furiosamente. Então disse Jorão: Aparelha o carro. E
aparelharam o seu carro. E saiu Jorão, rei de Israel, e Acazias, rei
de Judá, cada um em seu carro, e saíram ao encontro de Jeú, e o
acharam no pedaço de campo de Nabote, o jizreelita. E sucedeu
que, vendo Jorão a Jeú, disse: Há paz, Jeú? E disse ele: Que paz,
enquanto as prostituições da tua mãe Jezabel e as suas feitiçarias
são tantas? Então Jorão voltou as mãos e fugiu; e disse a
Acazias: Traição há, Acazias. Mas Jeú entesou o seu arco com
toda a força, e feriu a Jorão entre os braços, e a flecha lhe saiu
pelo coração; e ele caiu no seu carro. Então Jeú disse a
Bidcar, seu capitão: Toma-o, lança-o no pedaço do campo de Nabote, o
jizreelita; porque, lembra-te de que, indo eu e tu juntos a cavalo
após seu pai, Acabe, o Senhor pôs sobre ele esta sentença, dizendo:
Por certo vi ontem, à tarde, o sangue de Nabote e o sangue de
seus filhos, diz o Senhor; e neste mesmo campo te retribuirei, diz o
Senhor. Agora, pois, toma-o e lança-o neste campo, conforme a
palavra do Senhor. O que vendo Acazias, rei de Judá, fugiu
pelo caminho da casa do jardim; porém Jeú o perseguiu dizendo: Feri
também a este; e o feriram no carro à subida de Gur, que está junto
a Ibleão. E fugiu a Megido, e morreu ali. E seus servos o
levaram num carro a Jerusalém, e o sepultaram na sua sepultura junto
a seus pais, na cidade de Davi29(e no ano undécimo de Jorão, filho
de Acabe, começou Acazias a reinar sobre Judá).

Depois Jeú veio a Jizreel, o que ouvindo Jezabel, pintou-se em
volta dos olhos, enfeitou a sua cabeça, e olhou pela janela.
E, entrando Jeú pelas portas, disse ela: Teve paz Zinri, que
matou a seu senhor? E levantou ele o rosto para a janela e
disse: Quem é comigo? quem? E dois ou três eunucos olharam para ele.
Então disse ele: Lançai-a daí abaixo. E lançaram-na abaixo; e
foram salpicados com o seu sangue a parede e os cavalos, e Jeú a
atropelou. Entrando ele e havendo comido e bebido, disse:
Olhai por aquela maldita, e sepultai-a, porque é filha de rei.
E foram para a sepultar; porém não acharam dela senão somente
a caveira, os pés e as palmas das mãos. Então voltaram, e lho
fizeram saber; e ele disse: Esta é a palavra do Senhor, a qual falou
pelo ministério de Elias, o tisbita, seu servo, dizendo: No pedaço
do campo de Jizreel os cães comerão a carne de Jezabel. E o
cadáver de Jezabel será como esterco sobre o campo, na herdade de
Jizreel; de modo que não se possa dizer: Esta é Jezabel.

\medskip

\lettrine{10} E Acabe tinha setenta filhos em Samaria. Jeú
escreveu cartas, e as enviou a Samaria, aos chefes de Jizreel, aos
anciãos e aos aios dos filhos de Acabe, dizendo: Logo, em
chegando a vós esta carta, pois estão convosco os filhos de vosso
senhor, como também os carros, os cavalos, a cidade fortalecida e as
armas, olhai pelo melhor e mais reto dos filhos de vosso senhor,
o qual ponde sobre o trono de seu pai, e pelejai pela casa de vosso
senhor. Porém eles temeram muitíssimo, e disseram: Eis que dois
reis não puderam resistir a ele; como, pois, poderemos nós
resistir-lhe? Então o que tinha cargo da casa, e o que tinha
cargo da cidade, os anciãos e os aios mandaram dizer a Jeú: Teus
servos somos, e tudo quanto nos disseres faremos; a ninguém
constituiremos rei; faze o que parecer bom aos teus olhos. Então
segunda vez lhes escreveu outra carta, dizendo: Se fordes meus, e
ouvirdes a minha voz, tomai as cabeças dos homens, filhos de vosso
senhor, e vinde a mim amanhã, a este tempo, a Jizreel (os filhos do
rei, setenta homens, estavam com os grandes da cidade, que os
mantinham). Sucedeu que, chegada a eles a carta, tomaram os
filhos do rei, e os mataram, setenta homens e puseram as suas
cabeças nuns cestos, e lhas mandaram a Jizreel. E um mensageiro
veio, e lhe anunciou dizendo: Trouxeram as cabeças dos filhos do
rei. E ele disse: Ponde-as em dois montões à entrada da porta, até
amanhã. E sucedeu que, pela manhã, saindo ele, parou, e disse a
todo o povo: Vós sois justos; eis que eu conspirei contra o meu
senhor, e o matei; mas quem feriu a todos estes? Sabei, pois,
agora que, da palavra do Senhor que o Senhor falou contra a casa de
Acabe, nada cairá em terra, porque o Senhor tem feito o que falou
pelo ministério de seu servo Elias. Também Jeú feriu a todos
os restantes da casa de Acabe em Jizreel, como também a todos os
seus grandes, os seus conhecidos e seus sacerdotes, até não deixar
nenhum restante. Então se levantou e partiu, e foi a Samaria.
E, estando no caminho, em Bete-Equede dos pastores, Jeú achou
os irmãos de Acazias, rei de Judá, e disse: Quem sois vós? E eles
disseram: Os irmãos de Acazias somos; e descemos a saudar os filhos
do rei e os filhos da rainha. Então disse ele: Apanhai-os
vivos. E eles os apanharam vivos, e os mataram junto ao poço de
Bete-Equede, quarenta e dois homens; e a nenhum deles deixou ficar.

E, partindo dali, encontrou a Jonadabe, filho de Recabe, que lhe
vinha ao encontro, o qual saudou e lhe disse: Reto é o teu coração
para comigo, como o meu o é para contigo? E disse Jonadabe: É.
Então, se é, dá-me a mão. E deu-lhe a mão, e Jeú fê-lo subir consigo
ao carro. E disse: Vai comigo, e verás o meu zelo para com o
Senhor. E o puseram no seu carro. E, chegando a Samaria,
feriu a todos os que ficaram de Acabe em Samaria, até que os
destruiu, conforme a palavra que o Senhor dissera a Elias. E
ajuntou Jeú a todo o povo, e disse-lhe: Pouco serviu Acabe a Baal;
Jeú, porém, muito o servirá. Por isso chamai-me agora todos
os profetas de Baal, todos os seus servos e todos os seus
sacerdotes; não falte nenhum, porque tenho um grande sacrifício a
Baal; todo aquele que faltar não viverá. Porém Jeú fazia isto com
astúcia, para destruir os servos de Baal. Disse mais Jeú:
Consagrai a Baal uma assembléia solene. E a apregoaram.
Também Jeú enviou por todo o Israel; e vieram todos os servos
de Baal, e nenhum homem deles ficou que não viesse; e entraram na
casa de Baal, e encheu-se a casa de Baal, de um lado ao outro.
Então disse ao que tinha cargo das vestimentas: Tira as
vestimentas para todos os servos de Baal. E ele lhes tirou para fora
as vestimentas. E entrou Jeú com Jonadabe, filho de Recabe,
na casa de Baal, e disse aos servos de Baal: Examinai, e vede bem,
que porventura nenhum dos servos do Senhor aqui haja convosco, senão
somente os servos de Baal. E, entrando eles a fazerem
sacrifícios e holocaustos, Jeú preparou da parte de fora oitenta
homens, e disse-lhes: Se escapar algum dos homens que eu entregar em
vossas mãos, a vossa vida será pela vida dele. E sucedeu que,
acabando de fazer o holocausto, disse Jeú aos da sua guarda e aos
capitães: Entrai, feri-os, não escape nenhum. E os feriram ao fio da
espada; e os da guarda e os capitães os lançaram fora, e entraram no
mais interior da casa de Baal. E tiraram as estátuas da casa
de Baal, e as queimaram. Também quebraram a estátua de Baal;
e derrubaram a casa de Baal, e fizeram dela latrinas, até ao dia de
hoje. E assim Jeú destruiu a Baal de Israel.

Porém não se apartou Jeú de seguir os pecados de Jeroboão, filho
de Nebate, com que fez pecar a Israel, a saber: dos bezerros de
ouro, que estavam em Betel e em Dã. Por isso disse o Senhor a
Jeú: Porquanto bem agiste em fazer o que é reto aos meus olhos e,
conforme tudo quanto eu tinha no meu coração, fizeste à casa de
Acabe, teus filhos, até à quarta geração, se assentarão no trono de
Israel. Mas Jeú não teve cuidado de andar com todo o seu
coração na lei do Senhor Deus de Israel, nem se apartou dos pecados
de Jeroboão, com que fez pecar a Israel. Naqueles dias
começou o Senhor a diminuir os termos de Israel; porque Hazael os
feriu em todas as fronteiras de Israel. Desde o Jordão até ao
nascente do sol, a toda a terra de Gileade; os gaditas, os rubenitas
e os manassitas, desde Aroer, que está junto ao ribeiro de Arnom, a
saber, Gileade e Basã. Ora o mais dos atos de Jeú, tudo
quanto fez e todo o seu poder, porventura não está escrito no livro
das crônicas de Israel? E Jeú dormiu com seus pais, e o
sepultaram em Samaria; e Jeoacaz, seu filho, reinou em seu lugar.
E os dias que Jeú reinou sobre Israel, em Samaria, foram
vinte e oito anos.

\medskip

\lettrine{11} Vendo, pois, Atalia, mãe de Acazias, que seu
filho era morto, levantou-se, e destruiu toda a descendência real.
Mas Jeoseba, filha do rei Jorão, irmã de Acazias, tomou a Joás,
filho de Acazias, furtando-o dentre os filhos do rei, aos quais
matavam, e o pôs, a ele e à sua ama na recâmara\footnote{Alcova:
pequeno quarto de dormir situado no interior da casa, sem aberturas
para o exterior; recâmara. Quarto de mulher. Dormitório de casal.
Quarto de dormir.}, e o escondeu de Atalia, e assim não o mataram.
E esteve com ela escondido na casa do Senhor seis anos; e Atalia
reinava sobre o país.

E no sétimo ano enviou Joiada, e tomou os centuriões, com os
capitães, e com os da guarda, e os colocou consigo na casa do
Senhor; e fez com eles uma aliança e ajuramentou-os na casa do
Senhor; e mostrou-lhes o filho do rei. E deu-lhes ordem,
dizendo: Isto é o que haveis de fazer: Uma terça parte de vós, que
entrais no sábado, fará a guarda da casa do rei. E outra terça
parte estará à porta de Sur; e a outra terça parte à porta detrás
dos da guarda; assim fareis a guarda desta casa, afastando a todos.
E as duas partes de vós, a saber, todos os que saem no sábado,
farão a guarda da casa do Senhor junto ao rei. E rodeareis o
rei, cada um com as suas armas na mão, e aquele que entrar entre as
fileiras o matarão; e vós estareis com o rei quando sair e quando
entrar. Fizeram, pois, os centuriões conforme tudo quanto
ordenara o sacerdote Joiada, tomando cada um os seus homens, tanto
os que entravam no sábado como os que saíam no sábado; e foram ao
sacerdote Joiada. E o sacerdote deu aos centuriões as lanças
e os escudos que haviam sido do rei Davi, que estavam na casa do
Senhor. E os da guarda se puseram, cada um com as armas na
mão, desde o lado direito da casa até ao lado esquerdo da casa, do
lado do altar, e do lado da casa, em redor do rei. Então
Joiada fez sair o filho do rei, e lhe pôs a coroa, e lhe deu o
testemunho; e o fizeram rei, e o ungiram, e bateram as palmas, e
disseram: Viva o rei!

E Atalia, ouvindo a voz dos da guarda e do povo, foi ter com o
povo, na casa do Senhor. E olhou, e eis que o rei estava
junto à coluna, conforme o costume, e os príncipes e os trombeteiros
junto ao rei, e todo o povo da terra estava alegre e tocava as
trombetas; então Atalia rasgou as suas vestes, e clamou: Traição!
Traição! Porém o sacerdote Joiada deu ordem aos centuriões
que comandavam as tropas, dizendo-lhes: Tirai-a para fora das
fileiras, e a quem a seguir matai-o à espada. Porque o sacerdote
disse: Não a matem na casa do Senhor. E lançaram mão dela; e
ela foi, pelo caminho da entrada dos cavalos, à casa do rei, e ali a
mataram.

E Joiada fez uma aliança entre o Senhor e o rei e o povo, para
que fosse o povo do Senhor; como também entre o rei e o povo.
Então todo o povo da terra entrou na casa de Baal, e a
derrubaram, como também os seus altares, e as suas imagens,
totalmente quebraram, e a Matã, sacerdote de Baal, mataram diante
dos altares; então o sacerdote pôs oficiais sobre a casa do Senhor.
E tomou os centuriões, e os capitães, e os da guarda, e todo
o povo da terra; e conduziram da casa do Senhor o rei, e foram, pelo
caminho da porta dos da guarda, à casa do rei, e ele se assentou no
trono dos reis. E todo o povo da terra se alegrou, e a cidade
repousou, depois que mataram a Atalia, à espada, junto à casa do
rei. Era Joás da idade de sete anos quando o fizeram rei.

\medskip

\lettrine{12} No ano sétimo de Jeú começou a reinar Joás, e
quarenta anos reinou em Jerusalém; e era o nome de sua mãe Zíbia, de
Berseba. E fez Joás o que era reto aos olhos do Senhor todos os
dias em que o sacerdote Joiada o dirigia. Tão-somente os altos
não foram tirados; porque ainda o povo sacrificava e queimava
incenso nos altos.

E disse Joás aos sacerdotes: Todo o dinheiro das coisas santas que
se trouxer à casa do Senhor, a saber, o dinheiro daquele que passa o
arrolamento\footnote{Ato de arrolar; levantamento. Efeito de
arrolar; inventário, lista. Arrolar: Meter em rol ou lista. Fazer
relação de; inventariar. Pôr no rol de; classificar.}, o dinheiro de
cada uma das pessoas, segundo a sua avaliação, e todo o dinheiro que
trouxer cada um voluntariamente para a casa do Senhor, os
sacerdotes o recebam, cada um dos seus conhecidos; e eles mesmos
reparem as fendas da casa, toda a fenda que se achar nela.
Sucedeu, porém, que, no ano vinte e três do rei Joás, os
sacerdotes ainda não tinham reparado as fendas da casa. Então o
rei Joás chamou o sacerdote Joiada e os mais sacerdotes, e lhes
disse: Por que não reparais as fendas da casa? Agora, pois, não
tomeis mais dinheiro de vossos conhecidos, mas entregai-o para o
reparo das fendas da casa. E consentiram os sacerdotes em não
tomarem mais dinheiro do povo, e em não repararem as fendas da casa.
Porém o sacerdote Joiada tomou um cofre e fez um buraco na
tampa; e a pôs ao pé do altar, à mão direita dos que entravam na
casa do Senhor; e os sacerdotes que guardavam a entrada da porta
punham ali todo o dinheiro que se trazia à casa do Senhor.
Sucedeu que, vendo eles que já havia muito dinheiro no cofre,
o escrivão do rei subia com o sumo sacerdote, e contavam e ensacavam
o dinheiro que se achava na casa do Senhor. E o dinheiro,
depois de pesado, davam nas mãos dos que faziam a obra, que tinham a
seu cargo a casa do Senhor e eles o distribuíam aos carpinteiros e
aos edificadores que reparavam a casa do Senhor. Como também
aos pedreiros e aos cabouqueiros; e para se comprar madeira e pedras
de cantaria para repararem as fendas da casa do Senhor, e para tudo
quanto era necessário para reparar a casa. Todavia, do
dinheiro que se trazia à casa do Senhor não se faziam nem taças de
prata, nem garfos, nem bacias, nem trombetas, nem vaso algum de ouro
ou vaso de prata para a casa do Senhor. Porque o davam aos
que faziam a obra, e reparavam com ele a casa do Senhor.
Também não pediam contas aos homens em cujas mãos entregavam
aquele dinheiro, para o dar aos que faziam a obra, porque procediam
com fidelidade. Mas o dinheiro do sacrifício por delitos, e o
dinheiro por sacrifício de pecados, não se trazia à casa do Senhor;
porque era para os sacerdotes.

Então subiu Hazael, rei da Síria, e pelejou contra Gate, e a
tomou; depois Hazael resolveu marchar contra Jerusalém. Porém
Joás, rei de Judá, tomou todas as coisas santas que Jeosafá, Jorão e
Acazias, seus pais, reis de Judá, consagraram, como também todo o
ouro que se achou nos tesouros da casa do Senhor e na casa do rei e
o mandou a Hazael, rei da Síria; e então se desviou de Jerusalém.
Ora, o mais dos atos de Joás, e tudo quanto fez, porventura
não está escrito no livro das crônicas dos reis de Judá? E
levantaram-se os servos de Joás, e conspiraram contra ele ferindo-o
na casa de Milo, no caminho que desce para Sila. Porque
Jozacar, filho de Simeate, e Jozabade, filho de Somer, seus servos,
o feriram, e morreu, e o sepultaram com seus pais na cidade de Davi.
E Amazias, seu filho, reinou em seu lugar.

\medskip

\lettrine{13} No ano vinte e três de Joás, filho de Acazias,
rei de Judá, começou a reinar Jeoacaz, filho de Jeú, sobre Israel,
em Samaria, e reinou dezessete anos. E fez o que era mau aos
olhos do Senhor; porque seguiu os pecados de Jeroboão, filho de
Nebate, que fez pecar a Israel; não se apartou deles. Por isso a
ira do Senhor se acendeu contra Israel; e entregou-os na mão de
Hazael, rei da Síria, e na mão de Ben-Hadade, filho de Hazael, todos
aqueles dias. Porém Jeoacaz suplicou diante da face do Senhor; e
o Senhor ouviu; porque viu a opressão de Israel, pois o rei da Síria
os oprimia. E o Senhor deu um salvador a Israel, e saíram de sob
as mãos dos sírios; e os filhos de Israel habitaram nas suas tendas,
como no passado6(contudo não se apartaram dos pecados da casa de
Jeroboão, com que fez Israel pecar; porém ele andou neles e também o
bosque ficou em pé em Samaria). Porque não deixou a Jeoacaz, do
povo, senão só cinqüenta cavaleiros, dez carros e dez mil homens de
pé, porquanto o rei da Síria os tinha destruído e os tinha feito
como o pó, trilhando-os. Ora, o mais dos atos de Jeoacaz, e tudo
quanto fez, e o seu poder, porventura não está escrito no livro das
crônicas dos reis de Israel? E Jeoacaz dormiu com seus pais, e o
sepultaram em Samaria; e Jeoás, seu filho, reinou em seu lugar.

No ano trinta e sete de Joás, rei de Judá, começou a reinar
Jeoás, filho de Jeoacaz, sobre Israel, em Samaria, e reinou
dezesseis anos. E fez o que era mau aos olhos do Senhor; não
se apartou de nenhum dos pecados de Jeroboão, filho de Nebate, com
que fez Israel pecar, porém andou neles. Ora, o mais dos atos
de Jeoás, e tudo quanto fez, e o seu poder, com que pelejou contra
Amazias, rei de Judá, porventura não está escrito no livro das
crônicas dos reis de Israel? E Jeoás dormiu com seus pais, e
Jeroboão se assentou no seu trono; e Jeoás foi sepultado em Samaria,
junto aos reis de Israel. E Eliseu estava doente da
enfermidade de que morreu, e Jeoás, rei de Israel, desceu a ele, e
chorou sobre o seu rosto, e disse: Meu pai, meu pai, o carro de
Israel, e seus cavaleiros! E Eliseu lhe disse: Toma um arco e
flechas. E tomou um arco e flechas. Então disse ao rei de
Israel: Põe a tua mão sobre o arco. E pôs sobre ele a sua mão; e
Eliseu pôs as suas mãos sobre as do rei. E disse: Abre a
janela para o oriente. E abriu-a. Então disse Eliseu: Atira. E
atirou; e disse: A flecha do livramento do Senhor é a flecha do
livramento contra os sírios; porque ferirás os sírios; em Afeque,
até os consumir. Disse mais: Toma as flechas. E tomou-as.
Então disse ao rei de Israel: Fere a terra. E feriu-a três vezes, e
cessou. Então o homem de Deus se indignou muito contra ele, e
disse: Cinco ou seis vezes a deverias ter ferido; então feririas os
sírios até os consumir; porém agora só três vezes ferirás os sírios.

Depois morreu Eliseu, e o sepultaram. Ora, as tropas dos moabitas
invadiram a terra à entrada do ano. E sucedeu que, enterrando
eles um homem, eis que viram uma tropa, e lançaram o homem na
sepultura de Eliseu; e, caindo nela o homem, e tocando os ossos de
Eliseu, reviveu, e se levantou sobre os seus pés. E Hazael,
rei da Síria, oprimiu a Israel todos os dias de Jeoacaz.
Porém o Senhor teve misericórdia deles, e se compadeceu
deles, e tornou-se para eles por amor da sua aliança com Abraão,
Isaque e Jacó, e não os quis destruir, e não os lançou ainda da sua
presença. E morreu Hazael, rei da Síria e Ben-Hadade, seu
filho, reinou em seu lugar. E Jeoás, filho de Jeoacaz, tornou
a tomar as cidades das mãos de Ben-Hadade, que ele tinha tomado das
mãos de Jeoacaz, seu pai, na guerra; três vezes Jeoás o feriu, e
recuperou as cidades de Israel.

\medskip

\lettrine{14} No segundo ano de Jeoás, filho de Jeoacaz, rei
de Israel, começou a reinar Amazias, filho de Joás, rei de Judá.
Tinha vinte e cinco anos quando começou a reinar, e vinte e nove
anos reinou em Jerusalém. E era o nome de sua mãe Joadã, de
Jerusalém. E fez o que era reto aos olhos do Senhor, ainda que
não como seu pai Davi; fez, porém, conforme tudo o que fizera Joás
seu pai. Tão-somente os altos não foram tirados; porque o povo
ainda sacrificava e queimava incenso nos altos. Sucedeu que,
sendo já o reino confirmado na sua mão, matou os servos que tinham
matado o rei, seu pai. Porém os filhos dos assassinos não matou,
como está escrito no livro da lei de Moisés, no qual o Senhor deu
ordem, dizendo: Não matarão os pais por causa dos filhos, e os
filhos não matarão por causa dos pais; mas cada um será morto pelo
seu pecado. Este feriu a dez mil edomitas no vale do Sal, e
tomou a Sela na guerra; e chamou-a Jocteel, até ao dia de hoje.

Então Amazias enviou mensageiros a Jeoás, filho de Jeoacaz, filho
de Jeú, rei de Israel, dizendo: Vem, vejamo-nos face a face.
Porém Jeoás, rei de Israel, enviou a Amazias, rei de Judá,
dizendo: O cardo que estava no Líbano mandou dizer ao cedro que
estava no Líbano: Dá tua filha por mulher a meu filho; mas os
animais do campo, que estavam no Líbano, passaram e pisaram o cardo.
Na verdade feriste os moabitas, e o teu coração se
ensoberbeceu; gloria-te disso, e fica em tua casa; e por que te
entremeterias no mal, para caíres tu, e Judá contigo? Mas
Amazias não o ouviu. E subiu Jeoás, rei de Israel, e Amazias, rei de
Judá, e viram-se face a face, em Bete-Semes, que está em Judá.
E Judá foi ferido diante de Israel, e fugiu cada um para a
sua tenda. E Jeoás, rei de Israel, tomou a Amazias, rei de
Judá, filho de Joás, filho de Acazias, em Bete-Semes; e veio a
Jerusalém, e rompeu o muro de Jerusalém, desde a porta de Efraim até
a porta da esquina, quatrocentos côvados. E tomou todo o ouro
e a prata, e todos os vasos que se acharam na casa do Senhor e nos
tesouros da casa do rei, como também os reféns e voltou para
Samaria.

Ora, o mais dos atos de Jeoás, o que fez e o seu poder, e como
pelejou contra Amazias, rei de Judá, porventura não está escrito no
livro das crônicas dos reis de Israel? E dormiu Jeoás com
seus pais, e foi sepultado em Samaria, junto aos reis de Israel; e
Jeroboão, seu filho, reinou em seu lugar. E viveu Amazias,
filho de Joás, rei de Judá, depois da morte de Jeoás, filho de
Jeoacaz, rei de Israel, quinze anos. Ora, o mais dos atos de
Amazias, porventura não está escrito no livro das crônicas dos reis
de Judá? E conspiraram contra ele em Jerusalém, e fugiu para
Laquis; porém enviaram após ele até Laquis, e o mataram ali.
E o trouxeram em cima de cavalos; e o sepultaram em
Jerusalém, junto a seus pais, na cidade de Davi. E todo o
povo de Judá tomou a Azarias, que já era de dezesseis anos, e o
fizeram rei em lugar de Amazias, seu pai. Este edificou a
Elate, e a restituiu a Judá, depois que o rei dormiu com seus pais.

No décimo quinto ano de Amazias, filho de Joás, rei de Judá,
começou a reinar em Samaria, Jeroboão, filho de Jeoás, rei de
Israel, e reinou quarenta e um anos. E fez o que era mau aos
olhos do Senhor; nunca se apartou de nenhum dos pecados de Jeroboão,
filho de Nebate, com que fez pecar a Israel. Também este
restituiu os termos de Israel, desde a entrada de Hamate, até ao mar
da planície; conforme a palavra do Senhor Deus de Israel, a qual
falara pelo ministério de seu servo Jonas, filho do profeta Amitai,
o qual era de Gate-Hefer. Porque viu o Senhor que a miséria
de Israel era muito amarga, e que nem havia escravo, nem absolvido,
nem quem ajudasse a Israel. E ainda não falara o Senhor em
apagar o nome de Israel de debaixo do céu; porém os livrou por meio
de Jeroboão, filho de Jeoás. Ora, o mais dos atos de
Jeroboão, tudo quanto fez, e seu poder, como pelejou, e como
restituiu a Damasco e a Hamate, pertencentes a Judá, sendo rei em
Israel, porventura não está escrito no livro das crônicas de Israel?
E Jeroboão dormiu com seus pais, com os reis de Israel; e
Zacarias, seu filho, reinou em seu lugar.

\medskip

\lettrine{15} No ano vinte e sete de Jeroboão, rei de Israel,
começou a reinar Azarias, filho de Amazias, rei de Judá. Tinha
dezesseis anos quando começou a reinar, e cinqüenta e dois anos
reinou em Jerusalém; e era o nome de sua mãe Jecolias, de Jerusalém.
E fez o que era reto aos olhos do Senhor, conforme tudo o que
fizera Amazias, seu pai. Tão-somente os altos não foram tirados;
porque o povo ainda sacrificava e queimava incenso nos altos. E
o Senhor feriu o rei, e ficou leproso até ao dia da sua morte; e
habitou numa casa separada; porém Jotão, filho do rei, tinha o cargo
da casa, julgando o povo da terra. Ora, o mais dos atos de
Azarias, e tudo o que fez, porventura não está escrito no livro das
crônicas dos reis de Judá? E Azarias dormiu com seus pais e o
sepultaram junto a seus pais, na cidade de Davi; e Jotão, seu filho,
reinou em seu lugar.

No ano trinta e oito de Azarias, rei de Judá, reinou Zacarias,
filho de Jeroboão, sobre Israel, em Samaria, seis meses. E fez o
que era mau aos olhos do Senhor, como tinham feito seus pais; nunca
se apartou dos pecados de Jeroboão, filho de Nebate, com que fez
pecar a Israel. E Salum, filho de Jabes, conspirou contra ele
e feriu-o diante do povo, e matou-o; e reinou em seu lugar.
Ora, o mais dos atos de Zacarias, eis que está escrito no
livro das crônicas dos reis de Israel. Esta foi a palavra do
Senhor, que falou a Jeú: Teus filhos, até à quarta geração, se
assentarão sobre o trono de Israel. E assim foi. Salum, filho
de Jabes, começou a reinar no ano trinta e nove de Uzias, rei de
Judá, e reinou um mês inteiro em Samaria. Porque Menaém,
filho de Gadi, subiu de Tirza, e veio a Samaria; e feriu a Salum,
filho de Jabes, em Samaria, e o matou, e reinou em seu lugar.
Ora, o mais dos atos de Salum, e a conspiração que fez, eis
que está escrito no livro das crônicas dos reis de Israel.
Então Menaém feriu a Tifsa, e a todos os que nela havia, como
também a seus termos desde Tirza, porque não lha tinham aberto; e os
feriu, pois, e a todas as mulheres grávidas fendeu pelo meio.
Desde o ano trinta e nove de Azarias, rei de Judá, Menaém,
filho de Gadi, começou a reinar sobre Israel, e reinou dez anos em
Samaria. E fez o que era mau aos olhos do Senhor; todos os
seus dias não se apartou dos pecados de Jeroboão, filho de Nebate,
com que fez pecar a Israel. Então veio Pul, rei da Assíria,
contra a terra; e Menaém deu a Pul mil talentos de prata, para que
este o ajudasse a firmar o reino na sua mão. E Menaém tirou
este dinheiro de Israel, de todos os poderosos e ricos, para dá-lo
ao rei da Assíria, de cada homem cinqüenta siclos de prata; assim
voltou o rei da Assíria, e não ficou ali na terra. Ora, o
mais dos atos de Menaém, e tudo quanto fez, porventura não está
escrito no livro das crônicas dos reis de Israel? E Menaém
dormiu com seus pais; e Pecaías, seu filho, reinou em seu lugar.
No ano cinqüenta de Azarias, rei de Judá, começou a reinar
Pecaías, filho de Menaém, sobre Israel, em Samaria, e reinou dois
anos. E fez o que era mau aos olhos do Senhor; nunca se
apartou dos pecados de Jeroboão, filho de Nebate, com que fez pecar
a Israel. E Peca, filho de Remalias, seu capitão, conspirou
contra ele, e o feriu em Samaria, no paço\footnote{Palácio real ou
episcopal. Edifício suntuoso, nobre.} da casa do rei, juntamente com
Argobe e com Arié, e com ele cinqüenta homens dos filhos dos
gileaditas; e o matou, e reinou em seu lugar. Ora, o mais dos
atos de Pecaías, e tudo quanto fez, eis que está escrito no livro
das crônicas dos reis de Israel. No ano cinqüenta e dois de
Azarias, rei de Judá, começou a reinar Peca, filho de Remalias,
sobre Israel, em Samaria, e reinou vinte anos. E fez o que
era mau aos olhos do Senhor; nunca se apartou dos pecados de
Jeroboão, filho de Nebate, com que fez pecar a Israel. Nos
dias de Peca, rei de Israel, veio Tiglate-Pileser, rei da Assíria, e
tomou a Ijom, a Abel-Bete-Maaca, a Janoa, e a Quedes, a Hazor, a
Gileade, e a Galiléia, e a toda a terra de Naftali, e os levou à
Assíria. E Oséias, filho de Elá, conspirou contra Peca, filho
de Remalias, e o feriu, e o matou, e reinou em seu lugar, no
vigésimo ano de Jotão, filho de Uzias. Ora, o mais dos atos
de Peca, e tudo quanto fez, eis que está escrito no livro das
crônicas dos reis de Israel.

No ano segundo de Peca, filho de Remalias, rei de Israel, começou
a reinar Jotão, filho de Uzias, rei de Judá. Tinha vinte e
cinco anos de idade quando começou a reinar, e reinou dezesseis anos
em Jerusalém; e era o nome de sua mãe Jerusa, filha de Zadoque.
E fez o que era reto aos olhos do Senhor; fez conforme tudo
quanto fizera seu pai Uzias. Tão-somente os altos não foram
tirados; porque o povo ainda sacrificava e queimava incenso nos
altos. Este edificou a porta alta da casa do Senhor. Ora, o
mais dos atos de Jotão, e tudo quanto fez, porventura não está
escrito no livro das crônicas dos reis de Judá? Naqueles dias
começou o Senhor a enviar contra Judá a Rezim, rei da Síria, e a
Peca, filho de Remalias. E Jotão dormiu com seus pais, e foi
sepultado junto a seus pais, na cidade de Davi, seu pai; e Acaz, seu
filho, reinou em seu lugar.

\medskip

\lettrine{16} No ano dezessete de Peca, filho de Remalias,
começou a reinar Acaz, filho de Jotão, rei de Judá. Tinha Acaz
vinte anos de idade quando começou a reinar, e reinou dezesseis anos
em Jerusalém, e não fez o que era reto aos olhos do Senhor seu Deus,
como Davi, seu pai. Porque andou no caminho dos reis de Israel,
e até a seu filho fez passar pelo fogo, segundo as abominações dos
gentios que o Senhor lançara fora de diante dos filhos de Israel.
Também sacrificou, e queimou incenso nos altos e nos outeiros,
como também debaixo de todo o arvoredo.

Então subiu Rezim, rei da Síria, com Peca, filho de Remalias, rei
de Israel, a Jerusalém, para pelejar; e cercaram a Acaz, porém não o
puderam vencer. Naquele mesmo tempo Rezim, rei da Síria,
restituiu Elate à Síria, e lançou fora de Elate os judeus; e os
sírios vieram a Elate, e habitaram ali até ao dia de hoje. E
Acaz enviou mensageiros a Tiglate-Pileser, rei da Assíria, dizendo:
Eu sou teu servo e teu filho; sobe, e livra-me das mãos do rei da
Síria, e das mãos do rei de Israel, que se levantam contra mim.
E tomou Acaz a prata e o ouro que se achou na casa do Senhor, e
nos tesouros da casa do rei, e mandou um presente ao rei da Assíria.
E o rei da Assíria lhe deu ouvidos; pois o rei da Assíria subiu
contra Damasco, e tomou-a e levou cativo o povo para Quir, e matou a
Rezim.

Então o rei Acaz foi a Damasco, a encontrar-se com
Tiglate-Pileser, rei da Assíria; e, vendo um altar que estava em
Damasco, o rei Acaz enviou ao sacerdote Urias o desenho e o modelo
do altar, conforme toda a sua feitura. E Urias, o sacerdote,
edificou um altar conforme tudo o que o rei Acaz lhe tinha enviado
de Damasco; assim o fez o sacerdote Urias, antes que o rei Acaz
viesse de Damasco. Vindo, pois, o rei de Damasco, viu o
altar; e o rei se chegou ao altar, e sacrificou nele. E
queimou o seu holocausto, e a sua oferta de alimentos, e derramou a
sua libação, e espargiu o sangue dos seus sacrifícios pacíficos
sobre o altar. Porém o altar de cobre, que estava perante o
Senhor, ele tirou de diante da casa, de entre o seu altar e a casa
do Senhor, e pô-lo ao lado do altar, do lado do norte. E o
rei Acaz ordenou a Urias, o sacerdote, dizendo: Queima no grande
altar o holocausto da manhã, como também a oferta de alimentos da
noite, o holocausto do rei e a sua oferta de alimentos, e o
holocausto de todo o povo da terra, a sua oferta de alimentos, as
suas ofertas de bebidas e todo o sangue dos holocaustos, e todo o
sangue dos sacrifícios espargirás nele; porém o altar de cobre será
para mim, para nele inquirir. E fez Urias, o sacerdote,
conforme tudo quanto o rei Acaz lhe ordenara.

E o rei Acaz cortou as cintas das bases, e de cima delas tomou a
pia, e tirou o mar de sobre os bois de cobre, que estavam debaixo
dele, e pô-lo sobre um pavimento de pedra. Também a coberta
que, para o sábado, edificaram na casa, e a entrada real externa,
retirou da casa do Senhor, por causa do rei da Assíria. Ora,
o mais dos atos de Acaz e o que fez, porventura não está escrito no
livro das crônicas dos reis de Judá? E dormiu Acaz com seus
pais, e foi sepultado junto a seus pais, na cidade de Davi; e
Ezequias, seu filho, reinou em seu lugar.

\medskip

\lettrine{17} No ano duodécimo de Acaz, rei de Judá, começou a
reinar Oséias, filho de Elá, e reinou sobre Israel, em Samaria, nove
anos. E fez o que era mau aos olhos do Senhor, contudo não como
os reis de Israel que foram antes dele. Contra ele subiu
Salmaneser, rei da Assíria; e Oséias ficou sendo servo dele, e
pagava-lhe tributos. Porém o rei da Assíria achou em Oséias
conspiração; porque enviara mensageiros a Sô, rei do Egito, e não
pagava tributos ao rei da Assíria cada ano, como dantes; então o rei
da Assíria o encerrou e aprisionou na casa do cárcere. Porque o
rei da Assíria subiu por toda a terra, e veio até Samaria, e a
cercou três anos. No ano nono de Oséias, o rei da Assíria tomou
a Samaria, e levou Israel cativo para a Assíria; e fê-los habitar em
Hala e em Habor junto ao rio de Gozã, e nas cidades dos medos.

Porque sucedeu que os filhos de Israel pecaram contra o Senhor seu
Deus, que os fizera subir da terra do Egito, de debaixo da mão de
Faraó, rei do Egito; e temeram a outros deuses. E andaram nos
estatutos das nações que o Senhor lançara fora de diante dos filhos
de Israel, e nos dos reis de Israel, que eles fizeram. E os
filhos de Israel fizeram secretamente coisas que não eram retas,
contra o Senhor seu Deus; e edificaram altos em todas as suas
cidades, desde a torre dos atalaias até à cidade fortificada.
E levantaram, para si, estátuas e imagens do bosque, em todos
os altos outeiros, e debaixo de todas as árvores verdes. E
queimaram ali incenso em todos os altos, como as nações que o Senhor
expulsara de diante deles; e fizeram coisas ruins, para provocarem à
ira o Senhor. E serviram os ídolos, dos quais o Senhor lhes
dissera: Não fareis estas coisas. E o Senhor advertiu a
Israel e a Judá, pelo ministério de todos os profetas e de todos os
videntes, dizendo: Convertei-vos de vossos maus caminhos, e guardai
os meus mandamentos e os meus estatutos, conforme toda a lei que
ordenei a vossos pais e que eu vos enviei pelo ministério de meus
servos, os profetas. Porém não deram ouvidos; antes
endureceram a sua cerviz\footnote{A parte posterior do pescoço;
nuca. P. ext. Pescoço, cabeça. Dobrar a cerviz: submeter-se à
autoridade, à escravidão.}, como a cerviz de seus pais, que não
creram no Senhor seu Deus. E rejeitaram os seus estatutos, e
a sua aliança que fizera com seus pais, como também as suas
advertências, com que protestara contra eles; e seguiram a vaidade,
e tornaram-se vãos; como também seguiram as nações, que estavam ao
redor deles, das quais o Senhor lhes tinha ordenado que não as
imitassem. E deixaram todos os mandamentos do Senhor seu
Deus, e fizeram imagens de fundição, dois bezerros; e fizeram um
ídolo do bosque, e adoraram perante todo o exército do céu, e
serviram a Baal. Também fizeram passar pelo fogo a seus
filhos e suas filhas, e deram-se a adivinhações, e criam em
agouros\footnote{Agouro: Predição baseada no vôo ou no canto das
aves; augúrio. Profecia, prognóstico, vaticínio, predição, augúrio.
Presságio, pressentimento, previsão. Presságio de coisa má; mau
agouro.}; e venderam-se para fazer o que era mau aos olhos do
Senhor, para o provocarem à ira. Portanto o Senhor muito se
indignou contra Israel, e os tirou de diante da sua face; nada mais
ficou, senão somente a tribo de Judá. Até Judá não guardou os
mandamentos do Senhor seu Deus; antes andaram nos estatutos de
Israel, que eles fizeram. Por isso o Senhor rejeitou a toda a
descendência de Israel, e os oprimiu, e os deu nas mãos dos
despojadores, até que os expulsou da sua presença. Porque
rasgou a Israel da casa de Davi; e eles fizeram rei a Jeroboão,
filho de Nebate. E Jeroboão apartou a Israel de seguir ao Senhor, e
os fez cometer um grande pecado. Assim andaram os filhos de
Israel em todos os pecados que Jeroboão tinha feito; nunca se
apartaram deles; até que o Senhor tirou a Israel de diante da
sua presença, como falara pelo ministério de todos os seus servos,
os profetas; assim foi Israel expulso da sua terra à Assíria até ao
dia de hoje.

E o rei da Assíria trouxe gente de Babilônia, de Cuta, de Ava, de
Hamate e Sefarvaim, e a fez habitar nas cidades de Samaria, em lugar
dos filhos de Israel; e eles tomaram a Samaria em herança, e
habitaram nas suas cidades. E sucedeu que, no princípio da
sua habitação ali, não temeram ao Senhor; e o Senhor mandou entre
eles leões, que mataram a alguns deles. Por isso falaram ao
rei da Assíria, dizendo: A gente que transportaste e fizeste habitar
nas cidades de Samaria, não sabe o costume do Deus da terra; assim
mandou leões entre ela, e eis que a matam, porquanto não sabe o
culto do Deus da terra. Então o rei da Assíria mandou dizer:
Levai ali um dos sacerdotes que transportastes de lá; e vá e habite
lá, e ele lhes ensine o costume do Deus da terra. Veio, pois,
um dos sacerdotes que transportaram de Samaria, e habitou em Betel,
e lhes ensinou como deviam temer ao Senhor. Porém cada nação
fez os seus deuses, e os puseram nas casas dos altos que os
samaritanos fizeram, cada nação nas cidades, em que habitava.
E os de Babilônia fizeram Sucote-Benote; e os de Cuta fizeram
Nergal; e os de Hamate fizeram Asima. E os aveus fizeram
Nibaz e Tartaque; e os sefarvitas queimavam seus filhos no fogo a
Adrameleque, e a Anameleque, deuses de Sefarvaim. Também
temiam ao Senhor; e dos mais baixos do povo fizeram sacerdotes dos
lugares altos, os quais lhes faziam o ministério nas casas dos
lugares altos. Assim temiam ao Senhor, mas também serviam a
seus deuses, segundo o costume das nações dentre as quais tinham
sido transportados. Até ao dia de hoje fazem segundo os
primeiros costumes; não temem ao Senhor, nem fazem segundo os seus
estatutos, segundo as suas ordenanças, segundo a lei e segundo o
mandamento que o Senhor ordenou aos filhos de Jacó, a quem deu o
nome de Israel. Contudo o Senhor tinha feito uma aliança com
eles, e lhes ordenara, dizendo: Não temereis a outros deuses, nem
vos inclinareis diante deles, nem os servireis, nem lhes
sacrificareis. Mas o Senhor, que vos fez subir da terra do
Egito com grande força e com braço estendido, a este temereis, e a
ele vos inclinareis e a ele sacrificareis. E os estatutos, as
ordenanças, a lei e o mandamento, que vos escreveu, tereis cuidado
de fazer todos os dias; e não temereis a outros deuses. E da
aliança que fiz convosco não vos esquecereis; e não temereis a
outros deuses. Mas ao Senhor vosso Deus temereis, e ele vos
livrará das mãos de todos os vossos inimigos. Porém eles não
ouviram; antes fizeram segundo o seu primeiro costume. Assim
estas nações temiam ao Senhor e serviam as suas imagens de
escultura; também seus filhos, e os filhos de seus filhos, como
fizeram seus pais, assim fazem eles até ao dia de hoje.

\medskip

\lettrine{18} E sucedeu que, no terceiro ano de Oséias, filho
de Elá, rei de Israel, começou a reinar Ezequias, filho de Acaz, rei
de Judá. Tinha vinte e cinco anos de idade quando começou a
reinar, e vinte e nove anos reinou em Jerusalém; e era o nome de sua
mãe Abi, filha de Zacarias. E fez o que era reto aos olhos do
Senhor, conforme tudo o que fizera Davi, seu pai. Ele tirou os
altos, quebrou as estátuas, deitou abaixo os bosques, e fez em
pedaços a serpente de metal que Moisés fizera; porquanto até àquele
dia os filhos de Israel lhe queimavam incenso, e lhe chamaram
Neustã. No Senhor Deus de Israel confiou, de maneira que depois
dele não houve quem lhe fosse semelhante entre todos os reis de
Judá, nem entre os que foram antes dele. Porque se chegou ao
Senhor, não se apartou dele, e guardou os mandamentos que o Senhor
tinha dado a Moisés. Assim foi o Senhor com ele; para onde quer
que saía se conduzia com prudência; e se rebelou contra o rei da
Assíria, e não o serviu. Ele feriu os filisteus até Gaza, como
também os seus termos, desde a torre dos atalaias até à cidade
fortificada.

E sucedeu, no quarto ano do rei Ezequias (que era o sétimo ano de
Oséias, filho de Elá, rei de Israel), que Salmanasar, rei da
Assíria, subiu contra Samaria, e a cercou. E a tomaram ao fim
de três anos, no ano sexto de Ezequias, que era o ano nono de
Oséias, rei de Israel, quando tomaram Samaria. E o rei da
Assíria transportou a Israel para a Assíria; e os fez levar a Hala e
a Habor, junto ao rio de Gozã, e às cidades dos medos;
porquanto não obedeceram à voz do Senhor seu Deus, antes
transgrediram a sua aliança; e tudo quanto Moisés, servo do Senhor,
tinha ordenado, nem o ouviram nem o fizeram. Porém no ano
décimo quarto do rei Ezequias subiu Senaqueribe, rei da Assíria,
contra todas as cidades fortificadas de Judá, e as tomou.
Então Ezequias, rei de Judá, enviou ao rei da Assíria, a
Laquis, dizendo: Pequei; retira-te de mim; tudo o que me impuseres
suportarei. Então o rei da Assíria impôs a Ezequias, rei de Judá,
trezentos talentos de prata e trinta talentos de ouro. Assim
deu Ezequias toda a prata que se achou na casa do Senhor e nos
tesouros da casa do rei. Naquele tempo cortou Ezequias o ouro
das portas do templo do Senhor, e das ombreiras, de que ele, rei de
Judá, as cobrira, e o deu ao rei da Assíria.

Contudo enviou o rei da Assíria a Tartã, e a Rabe-Saris, e a
Rabsaqué, de Laquis, com grande exército ao rei Ezequias, a
Jerusalém; subiram, e vieram a Jerusalém. E, subindo e vindo eles,
pararam ao pé do aqueduto da piscina superior, que está junto ao
caminho do campo do lavandeiro. E chamaram o rei; e saíram a
eles Eliaquim, filho de Hilquias, o mordomo, e Sebna, o escrivão, e
Joá, filho de Asafe, o cronista. E Rabsaqué lhes disse: Ora,
dizei a Ezequias: Assim diz o grande rei, o rei da Assíria: Que
confiança é esta em que te estribas? Dizes tu (porém são
palavras só de lábios): Há conselho e poder para a guerra. Em quem,
pois, agora confias, que contra mim te rebelas? Eis que agora
tu confias naquele bordão de cana quebrada, no Egito, no qual, se
alguém se encostar, entrar-lhe-á pela mão e a furará; assim é Faraó,
rei do Egito, para com todos os que nele confiam. Se, porém,
me disserdes: No Senhor nosso Deus confiamos; porventura não é esse
aquele cujos altos e cujos altares Ezequias tirou, dizendo a Judá e
a Jerusalém: Perante este altar vos inclinareis em Jerusalém?
Ora, pois, dá agora reféns ao meu senhor, o rei da Assíria, e
dar-te-ei dois mil cavalos, se tu puderes dar cavaleiros para eles.
Como, pois, farias virar o rosto de um só capitão dos menores
servos de meu senhor, quando tu confias no Egito, por causa dos
carros e cavaleiros? Agora, pois, subi eu porventura sem o
Senhor contra este lugar, para o destruir? O Senhor me disse: Sobe
contra esta terra, e destrói-a. Então disse Eliaquim, filho
de Hilquias, e Sebna e Joá, a Rabsaqué: Rogamos-te que fales aos
teus servos em siríaco; porque bem o entendemos; e não nos fales em
judaico, aos ouvidos do povo que está em cima do muro. Porém
Rabsaqué lhes disse: Porventura mandou-me meu senhor somente a teu
senhor e a ti, para falar estas palavras e não antes aos homens, que
estão sentados em cima do muro, para que juntamente convosco comam o
seu excremento e bebam a sua urina? Rabsaqué, pois, se pôs em
pé, e clamou em alta voz em judaico, e respondeu, dizendo: Ouvi a
palavra do grande rei, do rei da Assíria. Assim diz o rei:
Não vos engane Ezequias; porque não vos poderá livrar da sua mão;
nem tampouco vos faça Ezequias confiar no Senhor, dizendo:
Certamente nos livrará o Senhor, e esta cidade não será entregue na
mão do rei da Assíria. Não deis ouvidos a Ezequias; porque
assim diz o rei da Assíria: Contratai comigo por presentes, e saí a
mim, e coma cada um da sua vide e da sua figueira, e beba cada um a
água da sua cisterna; até que eu venha, e vos leve para uma
terra como a vossa, terra de trigo e de mosto, terra de pão e de
vinhas, terra de oliveiras, de azeite e de mel; e assim vivereis, e
não morrereis; e não deis ouvidos a Ezequias; porque vos incita,
dizendo: O Senhor nos livrará. Porventura os deuses das
nações puderam livrar, cada um a sua terra, das mãos do rei da
Assíria? Que é feito dos deuses de Hamate e de Arpade? Que é
feito dos deuses de Sefarvaim, Hena e Iva? Porventura livraram a
Samaria da minha mão? Quais são eles, dentre todos os deuses
das terras, que livraram a sua terra da minha mão, para que o Senhor
livrasse a Jerusalém da minha mão?
 Porém calou-se o povo, e não lhe respondeu uma só palavra; porque
mandado do rei havia, dizendo: Não lhe respondereis. Então
Eliaquim, filho de Hilquias, o mordomo, e Sebna, o escrivão, e Joá,
filho de Asafe, o cronista, vieram a Ezequias com as vestes
rasgadas, e lhe fizeram saber as palavras de Rabsaqué.

\medskip

\lettrine{19} E aconteceu que, tendo Ezequias ouvido isto,
rasgou as suas vestes, e se cobriu de saco, e entrou na casa do
Senhor. Então enviou a Eliaquim, o mordomo, e a Sebna, o
escrivão, e os anciãos dos sacerdotes, cobertos de sacos, ao profeta
Isaías, filho de Amós. E disseram-lhe: Assim diz Ezequias: Este
dia é dia de angústia, de vituperação e de blasfêmia; porque os
filhos chegaram ao parto, e não há força para dá-los à luz. Bem
pode ser que o Senhor teu Deus ouça todas as palavras de Rabsaqué, a
quem enviou o seu Senhor, o rei da Assíria, para afrontar o Deus
vivo, e para vituperá-lo com as palavras que o Senhor teu Deus tem
ouvido; faze, pois, oração pelo restante que subsiste. E os
servos do rei Ezequias foram a Isaías. E Isaías lhes disse:
Assim direis a vosso senhor: Assim diz o Senhor: Não temas as
palavras que ouviste, com as quais os servos do rei da Assíria me
blasfemaram. Eis que porei nele um espírito, e ele ouvirá um
rumor, e voltará para a sua terra; à espada o farei cair na sua
terra.

Voltou, pois, Rabsaqué, e achou o rei da Assíria pelejando contra
Libna, porque tinha ouvido que o rei havia partido de Laquis. E,
ouvindo ele dizer de Tiraca, rei da Etiópia: Eis que saiu para te
fazer guerra; tornou a enviar mensageiros a Ezequias, dizendo:
Assim falareis a Ezequias, rei de Judá: Não te engane o teu
Deus, em quem confias, dizendo: Jerusalém não será entregue na mão
do rei da Assíria. Eis que já tens ouvido o que fizeram os
reis da Assíria a todas as terras, destruindo-as totalmente; e tu,
te livrarás? Porventura as livraram os deuses das nações, a
quem meus pais destruíram, como a Gozã, a Harã, a Resefe, e aos
filhos de Éden, que estavam em Telassar? Que é feito do rei
de Hamate, do rei de Arpade, e do rei da cidade de Sefarvaim, Hena e
Iva? Recebendo, pois, Ezequias as cartas das mãos dos
mensageiros e lendo-as, subiu à casa do Senhor; e Ezequias as
estendeu perante o Senhor. E orou Ezequias perante o Senhor e
disse: Ó Senhor Deus de Israel, que habitas entre os querubins, tu
mesmo, só tu és Deus de todos os reinos da terra; tu fizeste os céus
e a terra. Inclina, Senhor, o teu ouvido, e ouve; abre,
Senhor, os teus olhos, e olha; e ouve as palavras de Senaqueribe,
que enviou a este, para afrontar o Deus vivo. Verdade é, ó
Senhor, que os reis da Assíria assolaram as nações e as suas terras.
E lançaram os seus deuses no fogo; porquanto não eram deuses,
mas obra de mãos de homens, madeira e pedra; por isso os destruíram.
Agora, pois, ó Senhor nosso Deus, te suplico, livra-nos da
sua mão; e assim saberão todos os reinos da terra que só tu és o
Senhor Deus.

Então Isaías, filho de Amós, mandou dizer a Ezequias: Assim diz o
Senhor Deus de Israel: O que me pediste acerca de Senaqueribe, rei
da Assíria, ouvi. Esta é a palavra que o Senhor falou dele: A
virgem, a filha de Sião, te despreza, de ti zomba; a filha de
Jerusalém meneia a cabeça por detrás de ti. A quem afrontaste
e blasfemaste? E contra quem alçaste a voz e ergueste os teus olhos
ao alto? Contra o Santo de Israel? Por meio de teus
mensageiros afrontaste o Senhor, e disseste: Com a multidão de meus
carros subi ao alto dos montes, aos lados do Líbano, e cortarei os
seus altos cedros e as suas mais formosas faias, e entrarei nas suas
pousadas extremas, até no bosque do seu campo fértil. Eu
cavei, e bebi águas estranhas; e com as plantas de meus pés sequei
todos os rios do Egito. Porventura não ouviste que já dantes
fiz isto, e já desde os dias antigos o planejei? Agora, porém, o fiz
vir, para que fosses tu que reduzisses as cidades fortificadas a
montões desertos. Por isso os moradores delas, com pouca
força, ficaram pasmados e confundidos; eram como a erva do campo, e
a hortaliça verde, e o feno dos telhados, e o trigo queimado, antes
de amadurecer. Porém o teu assentar, e o teu sair e o teu
entrar, e o teu furor contra mim, eu o sei. Por causa do teu
furor contra mim, e porque a tua revolta subiu aos meus ouvidos,
portanto porei o meu anzol no teu nariz e o meu freio nos teus
lábios, e te farei voltar pelo caminho por onde vieste. E
isto te será por sinal: este ano se comerá o que nascer por si
mesmo, e no ano seguinte o que daí proceder; porém, no terceiro ano
semeai e segai, plantai vinhas, e comei os seus frutos.
Porque o que escapou da casa de Judá, e restou, tornará a
lançar raízes para baixo, e dará fruto para cima. Porque de
Jerusalém sairá o restante, e do monte Sião o que escapou; o zelo do
Senhor dos Exércitos fará isto. Portanto, assim diz o Senhor
acerca do rei da Assíria: Não entrará nesta cidade, nem lançará nela
flecha alguma; tampouco virá perante ela com escudo, nem levantará
contra ela trincheira alguma. Pelo caminho por onde vier, por
ele voltará; porém nesta cidade não entrará, diz o Senhor.
Porque eu ampararei a esta cidade, para a livrar, por amor de
mim e por amor do meu servo Davi.

Sucedeu, pois, que naquela mesma noite saiu o anjo do Senhor, e
feriu no arraial dos assírios a cento e oitenta e cinco mil deles;
e, levantando-se pela manhã cedo, eis que todos eram cadáveres.
Então Senaqueribe, rei da Assíria, partiu, e se foi, e voltou
e ficou em Nínive. E sucedeu que, estando ele prostrado na
casa de Nisroque, seu deus, Adrameleque e Sarezer, seus filhos, o
feriram à espada; porém eles escaparam para a terra de Ararate; e
Esar-Hadom, seu filho, reinou em seu lugar.

\medskip

\lettrine{20} Naqueles dias adoeceu Ezequias mortalmente; e o
profeta Isaías, filho de Amós, veio a ele e lhe disse: Assim diz o
Senhor: Põe em ordem a tua casa, porque morrerás, e não viverás.
Então virou o rosto para a parede, e orou ao Senhor, dizendo:
Ah, Senhor! Suplico-te lembrar de que andei diante de ti em
verdade, com o coração perfeito, e fiz o que era bom aos teus olhos.
E chorou Ezequias muitíssimo. Sucedeu, pois, que, não havendo
Isaías ainda saído do meio do pátio, veio a ele a palavra do Senhor
dizendo: Volta, e dize a Ezequias, capitão do meu povo: Assim
diz o Senhor, o Deus de Davi, teu pai: Ouvi a tua oração, e vi as
tuas lágrimas; eis que eu te sararei; ao terceiro dia subirás à casa
do Senhor. E acrescentarei aos teus dias quinze anos, e das mãos
do rei da Assíria te livrarei, a ti e a esta cidade; e ampararei
esta cidade por amor de mim, e por amor de Davi, meu servo.
Disse mais Isaías: Tomai uma pasta de figos. E a tomaram, e a
puseram sobre a chaga; e ele sarou. E Ezequias disse a Isaías:
Qual é o sinal de que o Senhor me sarará, e de que ao terceiro dia
subirei à casa do Senhor? Disse Isaías: Isto te será sinal, da
parte do Senhor, de que o Senhor cumprirá a palavra que disse:
Adiantar-se-á a sombra dez graus, ou voltará dez graus atrás?
Então disse Ezequias: É fácil que a sombra decline dez graus;
não seja assim, mas volte a sombra dez graus atrás. Então o
profeta Isaías clamou ao Senhor; e fez voltar a sombra dez graus
atrás, pelos graus que tinha declinado no relógio de sol de Acaz.

Naquele tempo enviou Berodaque-Baladã, filho de Baladã, rei de
Babilônia, cartas e um presente a Ezequias; porque ouvira que
Ezequias tinha estado doente. E Ezequias lhes deu ouvidos; e
lhes mostrou toda a casa de seu tesouro, a prata, o ouro, as
especiarias e os melhores ungüentos, e a sua casa de armas, e tudo
quanto se achou nos seus tesouros; coisa nenhuma houve que não lhes
mostrasse, nem em sua casa, nem em todo o seu domínio. Então
o profeta Isaías veio ao rei Ezequias, e lhe disse: Que disseram
aqueles homens, e de onde vieram a ti? Disse Ezequias: Vieram de um
país muito remoto, de Babilônia. E disse ele: Que viram em
tua casa? E disse Ezequias: Tudo quanto há em minha casa viram;
coisa nenhuma há nos meus tesouros que eu não lhes mostrasse.
Então disse Isaías a Ezequias: Ouve a palavra do Senhor.
Eis que vêm dias em que tudo quanto houver em tua casa, e o
que entesouraram teus pais até ao dia de hoje, será levado a
Babilônia; não ficará coisa alguma, disse o Senhor. E ainda
até de teus filhos, que procederem de ti, e que tu gerares, tomarão,
para que sejam eunucos no paço do rei da Babilônia. Então
disse Ezequias a Isaías: Boa é a palavra do Senhor que disseste.
Disse mais: E não haverá, pois, em meus dias paz e verdade?
Ora, o mais dos atos de Ezequias, e todo o seu poder, e como
fez a piscina e o aqueduto, e como fez vir a água à cidade,
porventura não está escrito no livro das crônicas dos reis de Judá?
E Ezequias dormiu com seus pais; e Manassés, seu filho,
reinou em seu lugar.

\medskip

\lettrine{21} Tinha Manassés doze anos de idade quando começou
a reinar, e cinqüenta e cinco anos reinou em Jerusalém; e era o nome
de sua mãe Hefzibá. E fez o que era mau aos olhos do Senhor,
conforme as abominações dos gentios que o Senhor expulsara de suas
possessões, de diante dos filhos de Israel. Porque tornou a
edificar os altos que Ezequias, seu pai, tinha destruído, e levantou
altares a Baal, e fez um bosque como o que fizera Acabe, rei de
Israel, e se inclinou diante de todo o exército dos céus, e os
serviu. E edificou altares na casa do Senhor, da qual o Senhor
tinha falado: Em Jerusalém porei o meu nome. Também edificou
altares a todo o exército dos céus em ambos os átrios da casa do
Senhor. E até fez passar a seu filho pelo fogo, adivinhava pelas
nuvens, era agoureiro e ordenou adivinhos e feiticeiros; e
prosseguiu em fazer o que era mau aos olhos do Senhor, para o
provocar à ira. Também pôs uma imagem de escultura, do bosque
que tinha feito, na casa de que o Senhor dissera a Davi e a Salomão,
seu filho: Nesta casa e em Jerusalém, que escolhi de todas as tribos
de Israel, porei o meu nome para sempre; e não mais farei mover
o pé de Israel desta terra que tenho dado a seus pais; contanto que
somente tenham cuidado de fazer conforme tudo o que lhes tenho
ordenado, e conforme toda a lei que Moisés, meu servo, lhes ordenou.
Porém não ouviram; porque Manassés de tal modo os fez errar, que
fizeram pior do que as nações, que o Senhor tinha destruído de
diante dos filhos de Israel.

Então o Senhor falou pelo ministério de seus servos, os profetas,
dizendo: Porquanto Manassés, rei de Judá, fez estas
abominações, fazendo pior do que tudo quanto fizeram os amorreus,
que foram antes dele, e até também a Judá fez pecar com os seus
ídolos; por isso, assim diz o Senhor Deus de Israel: Eis que
hei de trazer um mal sobre Jerusalém e Judá, que qualquer que ouvir,
lhe ficarão retinindo ambos os ouvidos. E estenderei sobre
Jerusalém o cordel de Samaria e o prumo da casa de Acabe; e limparei
a Jerusalém, como quem limpa o prato, limpa-o e vira-o para baixo.
E desampararei os restantes da minha herança, entregá-los-ei
na mão de seus inimigos; e servirão de presa e despojo para todos os
seus inimigos; porquanto fizeram o que era mau aos meus olhos
e me provocaram à ira, desde o dia em que seus pais saíram do Egito
até hoje. Além disso, também Manassés derramou muitíssimo
sangue inocente, até que encheu a Jerusalém de um ao outro extremo,
afora o seu pecado, com que fez Judá pecar, fazendo o que era mau
aos olhos do Senhor. Quanto ao mais dos feitos de Manassés, e
a tudo quanto fez, e ao seu pecado, que praticou, porventura não
está escrito no livro das crônicas dos reis de Judá? E
Manassés dormiu com seus pais, e foi sepultado no jardim da sua
casa, no jardim de Uzá; e Amom, seu filho, reinou em seu lugar.

Tinha Amom vinte e dois anos de idade quando começou a reinar, e
dois anos reinou em Jerusalém; e era o nome de sua mãe Mesulemete,
filha de Harus, de Jotbá. E fez o que era mau aos olhos do
Senhor, como fizera Manassés, seu pai. Porque andou em todo o
caminho em que andara seu pai; e serviu os ídolos, a que seu pai
tinha servido, e se inclinou diante deles. Assim deixou ao
Senhor Deus de seus pais, e não andou no caminho do Senhor. E
os servos de Amom conspiraram contra ele, e mataram o rei em sua
casa. Porém o povo da terra feriu a todos os que conspiraram
contra o rei Amom; e o povo da terra pôs Josias, seu filho, rei em
seu lugar. Quanto ao mais dos atos de Amom, que fez,
porventura não está escrito no livro das crônicas dos reis de Judá?
E o sepultaram na sua sepultura, no jardim de Uzá; e Josias,
seu filho, reinou em seu lugar.

\medskip

\lettrine{22} Tinha Josias oito anos de idade quando começou a
reinar, e reinou trinta e um anos em Jerusalém; e era o nome de sua
mãe, Jedida, filha de Adaías, de Bozcate. E fez o que era reto
aos olhos do Senhor; e andou em todo o caminho de Davi, seu pai, e
não se apartou dele nem para a direita nem para a esquerda.
Sucedeu que, no ano décimo oitavo do rei Josias, o rei mandou o
escrivão Safã, filho de Azalias, filho de Mesulão, à casa do Senhor,
dizendo: Sobe a Hilquias, o sumo sacerdote, para que tome o
dinheiro que se trouxe à casa do Senhor, o qual os guardas do umbral
da porta ajuntaram do povo, e que o dêem na mão dos que têm
cargo da obra, e estão encarregados da casa do Senhor; para que o
dêem àqueles que fazem a obra que há na casa do Senhor, para
repararem as fendas da casa; aos carpinteiros, aos edificadores
e aos pedreiros; e para comprar madeira e pedras lavradas, para
repararem a casa. Porém não se pediu conta do dinheiro que se
lhes entregara nas suas mãos, porquanto procediam com fidelidade.
Então disse o sumo sacerdote Hilquias ao escrivão Safã: Achei o
livro da lei na casa do Senhor. E Hilquias deu o livro a Safã, e ele
o leu. Então o escrivão Safã veio ter com o rei e, dando-lhe
conta, disse: Teus servos ajuntaram o dinheiro que se achou na casa,
e o entregaram na mão dos que têm cargo da obra, que estão
encarregados da casa do Senhor. Também Safã, o escrivão, fez
saber ao rei, dizendo: O sacerdote Hilquias me deu um livro. E Safã
o leu diante do rei.

Sucedeu, pois, que, ouvindo o rei as palavras do livro da lei,
rasgou as suas vestes. E o rei mandou a Hilquias, o
sacerdote, a Aicão, filho de Safã, a Acbor, filho de Micaías, a Safã
o escrivão e a Asaías, o servo do rei, dizendo: Ide, e
consultai o Senhor por mim, pelo povo e por todo o Judá, acerca das
palavras deste livro que se achou; porque grande é o furor do
Senhor, que se acendeu contra nós; porquanto nossos pais não deram
ouvidos às palavras deste livro, para fazerem conforme tudo quanto
acerca de nós está escrito. Então foi o sacerdote Hilquias, e
Aicão, Acbor, Safã e Asaías à profetiza Hulda, mulher de Salum,
filho de Ticvá, o filho de Harás, o guarda das vestiduras (e ela
habitava em Jerusalém, na segunda parte), e lhe falaram. E
ela lhes disse: Assim diz o Senhor Deus de Israel: Dizei ao homem
que vos enviou a mim: Assim diz o Senhor: Eis que trarei mal
sobre este lugar, e sobre os seus moradores, a saber: todas as
palavras do livro que leu o rei de Judá. Porquanto me
deixaram, e queimaram incenso a outros deuses, para me provocarem à
ira por todas as obras das suas mãos, o meu furor se acendeu contra
este lugar, e não se apagará. Porém ao rei de Judá, que vos
enviou a consultar o Senhor, assim lhe direis: Assim diz o Senhor
Deus de Israel, acerca das palavras, que ouviste: Porquanto o
teu coração se enterneceu, e te humilhaste perante o Senhor, quando
ouviste o que falei contra este lugar, e contra os seus moradores,
que seria para assolação e para maldição, e que rasgaste as tuas
vestes, e choraste perante mim, também eu te ouvi, diz o Senhor.
Por isso eis que eu te recolherei a teus pais, e tu serás
recolhido em paz à tua sepultura, e os teus olhos não verão todo o
mal que hei de trazer sobre este lugar. Então tornaram a trazer ao
rei a resposta.

\medskip

\lettrine{23} Então o rei ordenou, e todos os anciãos de Judá
e de Jerusalém se reuniram a ele. O rei subiu à casa do Senhor,
e com ele todos os homens de Judá, e todos os moradores de
Jerusalém, os sacerdotes, os profetas e todo o povo, desde o menor
até ao maior; e leu aos ouvidos deles todas as palavras do livro da
aliança, que se achou na casa do Senhor. E o rei se pôs em pé
junto à coluna, e fez a aliança perante o Senhor, para seguirem o
Senhor, e guardarem os seus mandamentos, os seus testemunhos e os
seus estatutos, com todo o coração e com toda a alma, confirmando as
palavras desta aliança, que estavam escritas naquele livro; e todo o
povo apoiou esta aliança.

E o rei mandou ao sumo sacerdote Hilquias, aos sacerdotes da
segunda ordem, e aos guardas do umbral da porta, que tirassem do
templo do Senhor todos os vasos que se tinham feito para Baal, para
o bosque e para todo o exército dos céus e os queimou fora de
Jerusalém, nos campos de Cedrom e levou as cinzas deles a Betel.
Também destituiu os sacerdotes que os reis de Judá estabeleceram
para incensarem sobre os altos nas cidades de Judá e ao redor de
Jerusalém, como também os que queimavam incenso a Baal, ao sol, à
lua, e aos planetas, e a todo o exército dos céus. Também tirou
da casa do Senhor o ídolo do bosque levando-o para fora de Jerusalém
até ao ribeiro de Cedrom, e o queimou junto ao ribeiro de Cedrom, e
o desfez em pó, e lançou o seu pó sobre as sepulturas dos filhos do
povo. Também derrubou as casas dos sodomitas que estavam na casa
do Senhor, em que as mulheres teciam casinhas para o ídolo do
bosque. E a todos os sacerdotes trouxe das cidades de Judá, e
profanou os altos em que os sacerdotes queimavam incenso, desde Geba
até Berseba; e derrubou os altos que estavam às portas, junto à
entrada da porta de Josué, o governador da cidade, que estava à
esquerda daquele que entrava pela porta da cidade. Mas os
sacerdotes dos altos não sacrificavam sobre o altar do Senhor em
Jerusalém; porém comiam pães ázimos no meio de seus irmãos.
Também profanou a Tofete, que está no vale dos filhos de
Hinom, para que ninguém fizesse passar a seu filho, ou sua filha,
pelo fogo a Moloque. Também tirou os cavalos que os reis de
Judá tinham dedicado ao sol, à entrada da casa do Senhor, perto da
câmara de Natã-Meleque, o camareiro, que estava no recinto; e os
carros do sol queimou a fogo. Também o rei derrubou os
altares que estavam sobre o terraço do cenáculo de Acaz, os quais os
reis de Judá tinham feito, como também o rei derrubou os altares que
fizera Manassés nos dois átrios da casa do Senhor; e esmiuçados os
tirou dali e lançou o pó deles no ribeiro de Cedrom. O rei
profanou também os altos que estavam defronte de Jerusalém, à mão
direita do monte de Masite, os quais edificara Salomão, rei de
Israel, a Astarote, a abominação dos sidônios, e a Quemós, a
abominação dos moabitas, e a Milcom, a abominação dos filhos de
Amom. Semelhantemente quebrou as estátuas, cortou os bosques
e encheu o seu lugar com ossos de homens.
 E também o altar que estava em Betel, e o alto que fez Jeroboão,
filho de Nebate, com que tinha feito Israel pecar, esse altar
derrubou juntamente com o alto; queimando o alto, em pó o esmiuçou,
e queimou o ídolo do bosque. E, virando-se Josias, viu as
sepulturas que estavam ali no monte; e mandou tirar os ossos das
sepulturas, e os queimou sobre aquele altar, e assim o profanou,
conforme a palavra do Senhor, que profetizara o homem de Deus,
quando anunciou estas palavras. Então disse: Que é este
monumento que vejo? E os homens da cidade lhe disseram: É a
sepultura do homem de Deus que veio de Judá, e anunciou estas coisas
que fizeste contra este altar de Betel. E disse: Deixai-o
estar; ninguém mexa nos seus ossos. Assim deixaram estar os seus
ossos com os ossos do profeta que viera de Samaria. Demais
disto também Josias tirou todas as casas dos altos que havia nas
cidades de Samaria, e que os reis de Israel tinham feito para
provocarem à ira o Senhor; e lhes fez conforme todos os atos que
tinha feito em Betel. E sacrificou todos os sacerdotes dos
altos, que havia ali, sobre os altares, e queimou ossos humanos
sobre eles; depois voltou a Jerusalém. O rei deu ordem a todo
o povo, dizendo: Celebrai a páscoa ao Senhor vosso Deus, como está
escrito no livro da aliança. Porque nunca se celebrou tal
páscoa como esta desde os dias dos juízes que julgaram a Israel, nem
em todos os dias dos reis de Israel, nem tampouco dos reis de Judá.
Porém no ano décimo oitavo do rei Josias esta páscoa se
celebrou ao Senhor em Jerusalém. E também os adivinhos, os
feiticeiros, os terafins, os ídolos, e todas as abominações que se
viam na terra de Judá e em Jerusalém, os extirpou Josias, para
confirmar as palavras da lei, que estavam escritas no livro que o
sacerdote Hilquias achara na casa do Senhor.

E antes dele não houve rei semelhante, que se convertesse ao
Senhor com todo o seu coração, com toda a sua alma e com todas as
suas forças, conforme toda a lei de Moisés; e depois dele nunca se
levantou outro tal. Todavia o Senhor não se demoveu do ardor
da sua grande ira, com que ardia contra Judá, por todas as
provocações com que Manassés o tinha provocado. E disse o
Senhor: Também a Judá hei de tirar de diante da minha face, como
tirei a Israel, e rejeitarei esta cidade de Jerusalém que escolhi,
como também a casa de que disse: Estará ali o meu nome. Ora,
o mais dos atos de Josias e tudo quanto fez, porventura não está
escrito no livro das crônicas dos reis de Judá? Nos seus dias
subiu Faraó Neco, rei do Egito, contra o rei da Assíria, ao rio
Eufrates; e o rei Josias lhe foi ao encontro; e, vendo-o ele, o
matou em Megido. E seus servos, num carro, o levaram morto,
de Megido, e o trouxeram a Jerusalém, e o sepultaram na sua
sepultura; e o povo da terra tomou a Joacaz, filho de Josias, e
ungiram-no, e fizeram-no rei em lugar de seu pai.

Tinha Joacaz vinte e três anos de idade quando começou a reinar,
e três meses reinou em Jerusalém; e era o nome de sua mãe Hamutal,
filha de Jeremias, de Libna. E fez o que era mau aos olhos do
Senhor, conforme tudo o que fizeram seus pais. Porém Faraó
Neco o mandou prender em Ribla, em terra de Hamate, para que não
reinasse em Jerusalém; e à terra impôs pena de cem talentos de prata
e um talento de ouro. Também Faraó Neco constituiu rei a
Eliaquim, filho de Josias, em lugar de seu pai Josias, e lhe mudou o
nome para Jeoiaquim; porém a Joacaz tomou consigo, e foi ao Egito, e
morreu ali. E Jeoiaquim deu aquela prata e aquele ouro a
Faraó; porém tributou a terra, para dar esse dinheiro conforme o
mandado de Faraó; a cada um segundo a sua avaliação exigiu a prata e
o ouro do povo da terra, para o dar a Faraó Neco. Tinha
Jeoiaquim vinte e cinco anos de idade quando começou a reinar, e
reinou onze anos em Jerusalém; e era o nome de sua mãe Zebida, filha
de Pedaías, de Ruma. E fez o que era mau aos olhos do Senhor,
conforme tudo quanto fizeram seus pais.

\medskip

\lettrine{24} Nos seus dias subiu Nabucodonosor, rei de
Babilônia, e Jeoiaquim ficou três anos seu servo; depois se virou, e
se rebelou contra ele. E o Senhor enviou contra ele as tropas
dos caldeus, as tropas dos sírios, as tropas dos moabitas e as
tropas dos filhos de Amom; e as enviou contra Judá, para o destruir,
conforme a palavra do Senhor, que falara pelo ministério de seus
servos, os profetas. E, na verdade, conforme o mandado do
Senhor, assim sucedeu a Judá, para o afastar da sua presença por
causa dos pecados de Manassés, conforme tudo quanto fizera. Como
também por causa do sangue inocente que derramou; pois encheu a
Jerusalém de sangue inocente; e por isso o Senhor não quis perdoar.
Ora, o mais dos atos de Jeoiaquim, e tudo quanto fez, porventura
não está escrito no livro das crônicas dos reis de Judá? E
Jeoiaquim dormiu com seus pais; e Joaquim, seu filho, reinou em seu
lugar. E o rei do Egito nunca mais saiu da sua terra; porque o
rei de Babilônia tomou tudo quanto era do rei do Egito, desde o rio
do Egito até ao rio Eufrates.

Tinha Joaquim dezoito anos de idade quando começou a reinar, e
reinou três meses em Jerusalém; e era o nome de sua mãe, Neusta,
filha de Elnatã, de Jerusalém. E fez o que era mau aos olhos do
Senhor, conforme tudo quanto fizera seu pai. Naquele tempo
subiram os servos de Nabucodonosor, rei de Babilônia, a Jerusalém; e
a cidade foi cercada. Também veio Nabucodonosor, rei de
Babilônia, contra a cidade, quando já os seus servos a estavam
sitiando. Então saiu Joaquim, rei de Judá, ao rei de
Babilônia, ele, sua mãe, seus servos, seus príncipes e seus
oficiais; e o rei de Babilônia o tomou preso, no ano oitavo do seu
reinado. E tirou dali todos os tesouros da casa do Senhor e
os tesouros da casa do rei; e partiu todos os vasos de ouro, que
fizera Salomão, rei de Israel, no templo do Senhor, como o Senhor
tinha falado. E transportou a toda a Jerusalém como também a
todos os príncipes, e a todos os homens valorosos, dez mil presos, e
a todos os artífices e ferreiros; ninguém ficou senão o povo pobre
da terra. Assim transportou Joaquim à Babilônia; como também
a mãe do rei, as mulheres do rei, os seus oficiais e os poderosos da
terra levou presos de Jerusalém à Babilônia. E todos os
homens valentes, até sete mil, e artífices e ferreiros até mil, e
todos os homens destros na guerra, a estes o rei de Babilônia levou
presos para Babilônia. E o rei de Babilônia estabeleceu a
Matanias, seu tio, rei em seu lugar; e lhe mudou o nome para
Zedequias. Tinha Zedequias vinte e um anos de idade quando
começou a reinar, e reinou onze anos em Jerusalém; e era o nome de
sua mãe Hamutal, filha de Jeremias, de Libna. E fez o que era
mau aos olhos do Senhor, conforme tudo quanto fizera Jeoiaquim.
Porque assim sucedeu por causa da ira do Senhor contra
Jerusalém, e contra Judá, até os rejeitar de diante da sua presença;
e Zedequias se rebelou contra o rei de Babilônia.

\medskip

\lettrine{25} E sucedeu que, no nono ano do seu reinado, no
mês décimo, aos dez do mês, Nabucodonosor, rei de Babilônia, veio
contra Jerusalém, ele e todo o seu exército, e se acampou contra
ela, e levantaram contra ela trincheiras em redor. E a cidade
foi sitiada até ao undécimo ano do rei Zedequias. Aos nove do
mês quarto, quando a cidade se via apertada pela fome, nem havia pão
para o povo da terra, então a cidade foi invadida, e todos os
homens de guerra fugiram de noite pelo caminho da porta, entre os
dois muros que estavam junto ao jardim do rei (porque os caldeus
estavam contra a cidade em redor), e o rei se foi pelo caminho da
campina. Porém o exército dos caldeus perseguiu o rei, e o
alcançou nas campinas de Jericó; e todo o seu exército se dispersou.
E tomaram o rei, e o fizeram subir ao rei de Babilônia, a Ribla;
e foi-lhe pronunciada a sentença. E aos filhos de Zedequias
mataram diante dos seus olhos; e vazaram os olhos de Zedequias, e o
ataram com duas cadeias de bronze, e o levaram a Babilônia.

E no quinto mês, no sétimo dia do mês (este era o ano décimo nono
de Nabucodonosor, rei de Babilônia), veio Nebuzaradã, capitão da
guarda, servo do rei de Babilônia, a Jerusalém. E queimou a casa
do Senhor e a casa do rei, como também todas as casas de Jerusalém,
e todas as casas dos grandes queimou. E todo o exército dos
caldeus, que estava com o capitão da guarda, derrubou os muros em
redor de Jerusalém. E o mais do povo que deixaram ficar na
cidade, os rebeldes que se renderam ao rei de Babilônia e o mais da
multidão, Nebuzaradã, o capitão da guarda, levou presos.
Porém dos mais pobres da terra deixou o capitão da guarda
ficar alguns para vinheiros e para lavradores. Quebraram
mais, os caldeus, as colunas de cobre que estavam na casa do Senhor,
como também as bases e o mar de cobre que estavam na casa do Senhor;
e levaram o seu bronze para Babilônia. Também tomaram as
caldeiras, as pás, os apagadores, as colheres e todos os vasos de
cobre, com que se ministrava. Também o capitão-da-guarda
tomou os braseiros, e as bacias, o que era de ouro puro, em ouro e o
que era de prata, em prata. As duas colunas, um mar, e as
bases, que Salomão fizera para a casa do Senhor; o cobre de todos
estes vasos não tinha peso. A altura de uma coluna era de
dezoito côvados, e sobre ela havia um capitel de cobre, e de altura
tinha o capitel três côvados; e a rede e as romãs em redor do
capitel, tudo era de cobre; e semelhante a esta era a outra coluna
com a rede. Também o capitão-da-guarda tomou a Seraías,
primeiro sacerdote, e a Sofonias, segundo sacerdote, e aos três
guardas do umbral da porta. E da cidade tomou a um oficial,
que tinha cargo dos homens de guerra, e a cinco homens dos que
estavam na presença do rei, e se achavam na cidade, como também ao
escrivão-mor do exército, que registrava o povo da terra para a
guerra, e a sessenta homens do povo da terra, que se achavam na
cidade. E tomando-os Nebuzaradã, o capitão da guarda, os
levou ao rei de Babilônia, a Ribla. E o rei de Babilônia os
feriu e os matou em Ribla, na terra de Hamate; e Judá foi levado
preso para fora da sua terra.

Porém, quanto ao povo que ficara na terra de Judá, que
Nabucodonosor, rei de Babilônia, deixou ficar, pôs sobre ele, por
governador a Gedalias, filho de Aicão, filho de Safã.
Ouvindo, pois, os capitães dos exércitos, eles e os seus
homens, que o rei de Babilônia pusera a Gedalias por governador,
vieram a Gedalias, a Mizpá, a saber: Ismael, filho de Netanias, e
Joanã, filho de Careá, e Seraías, filho de Tanumete, o netofatita, e
Jazanias, filho do maacatita, eles e os seus homens. E
Gedalias jurou a eles e aos seus homens, e lhes disse: Não temais
ser servos dos caldeus; ficai na terra, servi ao rei de Babilônia, e
bem vos irá. Sucedeu, porém, que, no sétimo mês, veio Ismael,
filho de Netanias, o filho de Elisama, da descendência real, e dez
homens com ele, e feriram a Gedalias, e ele morreu, como também aos
judeus, e aos caldeus que estavam com ele em Mizpá. Então
todo o povo se levantou, desde o menor até ao maior, como também os
capitães dos exércitos, e foram ao Egito, porque temiam os caldeus.
Depois disto sucedeu que, no ano trinta e sete do cativeiro
de Joaquim, rei de Judá, no mês duodécimo, aos vinte e sete do mês,
Evil-Merodaque, rei de Babilônia, no ano em que reinou, levantou a
cabeça de Joaquim, rei de Judá, tirando-o da casa da prisão.
E lhe falou benignamente; e pôs o seu trono acima do trono
dos reis que estavam com ele em Babilônia. E lhe mudou as
roupas de prisão, e de contínuo comeu pão na sua presença todos os
dias da sua vida. E, quanto à sua subsistência, pelo rei lhe
foi dada subsistência contínua, a porção de cada dia no seu dia,
todos os dias da sua vida.

