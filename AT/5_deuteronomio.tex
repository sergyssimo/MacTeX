\addchap{Deuteronômio}

\lettrine{1} Estas são as palavras que Moisés falou a todo o
Israel além do Jordão, no deserto, na planície defronte do Mar
Vermelho, entre Parã e Tôfel, e Labã, e Hazerote, e Di-Zaabe.
Onze jornadas há desde Horebe, caminho do monte Seir, até
Cades-Barnéia. E sucedeu que, no ano quadragésimo, no mês
undécimo, no primeiro dia do mês, Moisés falou aos filhos de Israel,
conforme a tudo o que o Senhor lhe mandara acerca deles. Depois
que feriu a Siom, rei dos amorreus, que habitava em Hesbom, e a
Ogue, rei de Basã, que habitava em Astarote, em Edrei. Além do
Jordão, na terra de Moabe, começou Moisés a declarar esta lei,
dizendo: O Senhor nosso Deus nos falou em Horebe, dizendo: Assaz
vos haveis demorado neste monte. Voltai-vos, e parti, e ide à
montanha dos amorreus, e a todos os seus vizinhos, à planície, e à
montanha, e ao vale, e ao sul, e à margem do mar; à terra dos
cananeus, e ao Líbano, até ao grande rio, o rio Eufrates. Eis
que tenho posto esta terra diante de vós; entrai e possuí a terra
que o Senhor jurou a vossos pais, Abraão, Isaque e Jacó, que a daria
a eles e à sua descendência depois deles.

E no mesmo tempo eu vos falei, dizendo: Eu sozinho não poderei
levar-vos. O Senhor vosso Deus já vos tem multiplicado; e eis
que em multidão sois hoje como as estrelas do céu. O Senhor
Deus de vossos pais vos aumente, ainda mil vezes mais do que sois; e
vos abençoe, como vos tem falado. Como suportaria eu sozinho
os vossos fardos, e as vossas cargas, e as vossas contendas?
Tomai-vos homens sábios e entendidos, experimentados entre as
vossas tribos, para que os ponha por chefes sobre vós. Então
vós me respondestes, e dissestes: Bom é fazer o que tens falado.
Tomei, pois, os chefes de vossas tribos, homens sábios e
experimentados, e os tenho posto por cabeças sobre vós, por capitães
de milhares, e por capitães de cem, e por capitães de cinqüenta, e
por capitães de dez, e por governadores das vossas tribos. E
no mesmo tempo mandei a vossos juízes, dizendo: Ouvi a causa entre
vossos irmãos, e julgai justamente entre o homem e seu irmão, e
entre o estrangeiro que está com ele. Não discriminareis as
pessoas em juízo; ouvireis assim o pequeno como o grande; não
temereis a face de ninguém, porque o juízo é de Deus; porém a causa
que vos for difícil fareis vir a mim, e eu a ouvirei. Assim
naquele tempo vos ordenei todas as coisas que havíeis de fazer.

Então partimos de Horebe, e caminhamos por todo aquele grande e
tremendo deserto que vistes, pelo caminho das montanhas dos
amorreus, como o Senhor nosso Deus nos ordenara; e chegamos a
Cades-Barnéia. Então eu vos disse: Chegados sois às montanhas
dos amorreus, que o Senhor nosso Deus nos dá. Eis aqui o
Senhor teu Deus tem posto esta terra diante de ti; sobe, toma posse
dela, como te falou o Senhor Deus de teus pais; não temas, e não te
assustes. Então todos vós chegastes a mim, e dissestes:
Mandemos homens adiante de nós, para que nos espiem a terra e, de
volta, nos ensinem o caminho pelo qual devemos subir, e as cidades a
que devemos ir. Isto me pareceu bem; de modo que de vós tomei
doze homens, de cada tribo um homem. E foram-se, e subiram à
montanha, e chegaram até ao vale de Escol, e o espiaram. E
tomaram do fruto da terra nas suas mãos, e no-lo trouxeram e nos
informaram, dizendo: Boa é a terra que nos dá o Senhor nosso Deus.
Porém vós não quisestes subir; mas fostes rebeldes ao mandado
do Senhor nosso Deus. E murmurastes nas vossas tendas, e
dissestes: Porquanto o Senhor nos odeia, nos tirou da terra do Egito
para nos entregar nas mãos dos amorreus, para destruir-nos.
Para onde subiremos? Nossos irmãos fizeram com que se
derretesse o nosso coração, dizendo: Maior e mais alto é este povo
do que nós, as cidades são grandes e fortificadas até aos céus; e
também vimos ali filhos dos gigantes. Então eu vos disse: Não
vos espanteis, nem os temais. O Senhor vosso Deus que vai
adiante de vós, ele pelejará por vós, conforme a tudo o que fez
convosco, diante de vossos olhos, no Egito; como também no
deserto, onde vistes que o Senhor vosso Deus nele vos levou, como um
homem leva seu filho, por todo o caminho que andastes, até chegardes
a este lugar. Mas nem por isso crestes no Senhor vosso Deus.
Que foi adiante de vós por todo o caminho, para vos achar o
lugar onde vós deveríeis acampar; de noite no fogo, para vos mostrar
o caminho por onde havíeis de andar, e de dia na nuvem.
Ouvindo, pois, o Senhor a voz das vossas palavras,
indignou-se, e jurou, dizendo: Nenhum dos homens desta
maligna geração verá esta boa terra que jurei dar a vossos pais.
Salvo Calebe, filho de Jefoné; ele a verá, e a terra que
pisou darei a ele e a seus filhos; porquanto perseverou em seguir ao
Senhor. Também o Senhor se indignou contra mim por causa de
vós, dizendo: Também tu lá não entrarás. Josué, filho de Num,
que está diante de ti, ele ali entrará; fortalece-o, porque ele a
fará herdar a Israel. E vossos meninos, de quem dissestes:
Por presa serão; e vossos filhos, que hoje não conhecem nem o bem
nem o mal, eles ali entrarão, e a eles a darei, e eles a possuirão.
Porém vós virai-vos, e parti para o deserto, pelo caminho do
Mar Vermelho. Então respondestes, e me dissestes: Pecamos
contra o Senhor; nós subiremos e pelejaremos, conforme a tudo o que
nos ordenou o Senhor nosso Deus. E armastes-vos, cada um de vós, dos
seus instrumentos de guerra, e estivestes prestes para subir à
montanha. E disse-me o Senhor: Dize-lhes: Não subais nem
pelejeis, pois não estou no meio de vós; para que não sejais feridos
diante de vossos inimigos. Porém, falando-vos eu, não
ouvistes; antes fostes rebeldes ao mandado do Senhor, e vos
ensoberbecestes, e subistes à montanha. E os amorreus, que
habitavam naquela montanha, vos saíram ao encontro; e
perseguiram-vos como fazem as abelhas e vos derrotaram desde Seir
até Horma. Tornando, pois, vós, e chorando perante o Senhor,
o Senhor não ouviu a vossa voz, nem vos escutou. Assim
permanecestes muitos dias em Cades, pois ali vos demorastes muito.

\medskip

\lettrine{2} Depois viramo-nos, e caminhamos ao deserto,
caminho do Mar Vermelho, como o Senhor me tinha dito, e muitos dias
rodeamos o monte Seir. Então o Senhor me falou, dizendo:
Tendes rodeado bastante esta montanha; virai-vos para o norte.
E dá ordem ao povo, dizendo: Passareis pelos termos de vossos
irmãos, os filhos de Esaú, que habitam em Seir; e eles terão medo de
vós; porém guardai-vos bem. Não vos envolvais com eles, porque
não vos darei da sua terra nem ainda a pisada da planta de um pé;
porquanto a Esaú tenho dado o monte Seir por herança. Comprareis
deles, por dinheiro, comida para comerdes; e também água para beber
deles comprareis por dinheiro. Pois o Senhor teu Deus te
abençoou em toda a obra das tuas mãos; ele sabe que andas por este
grande deserto; estes quarenta anos o Senhor teu Deus esteve
contigo, coisa nenhuma te faltou.

Passando, pois, por nossos irmãos, os filhos de Esaú, que
habitavam em Seir, desde o caminho da planície de Elate e de
Eziom-Geber, nos viramos e passamos o caminho do deserto de Moabe.
Então o Senhor me disse: Não molestes aos de Moabe, e não
contendas com eles em peleja, porque não te darei herança da sua
terra; porquanto tenho dado a Ar por herança aos filhos de Ló.
os emins dantes habitaram nela; um povo grande e numeroso, e
alto como os gigantes. Também estes foram considerados
gigantes como os anaquins; e os moabitas os chamavam emins.
Outrora os horeus também habitaram em Seir; porém os filhos
de Esaú os lançaram fora, e os destruíram de diante de si, e
habitaram no seu lugar, assim como Israel fez à terra da sua
herança, que o Senhor lhes tinha dado). Levantai-vos agora, e
passai o ribeiro de Zerede. Assim passamos o ribeiro de Zerede.
E os dias que caminhamos, desde Cades-Barnéia até que
passamos o ribeiro de Zerede, foram trinta e oito anos, até que toda
aquela geração dos homens de guerra se consumiu do meio do arraial,
como o Senhor lhes jurara. Assim também foi contra eles a mão
do Senhor, para os destruir do meio do arraial até os haver
consumido. E sucedeu que, sendo já consumidos todos os homens
de guerra, pela morte, do meio do povo, o Senhor me falou,
dizendo: Hoje passarás a Ar, pelos termos de Moabe; e
chegando até defronte dos filhos de Amom, não os molestes, e com
eles não contendas; porque da terra dos filhos de Amom não te darei
herança, porquanto aos filhos de Ló a tenho dado por herança.

essa foi considerada terra de gigantes; antes nela
habitavam gigantes, e os amonitas os chamavam zamzumins; um
povo grande, e numeroso, e alto, como os gigantes; e o Senhor os
destruiu de diante dos amonitas, e estes os lançaram fora, e
habitaram no seu lugar; assim como fez com os filhos de Esaú,
que habitavam em Seir, de diante dos quais destruiu os horeus, e
eles os lançaram fora, e habitaram no lugar deles até este dia;
também os caftorins, que saíram de Caftor, destruíram os
avins, que habitavam em Cazerim até Gaza, e habitaram no lugar
deles).

Levantai-vos, parti e passai o ribeiro de Arnom; eis aqui na tua
mão tenho dado a Siom, amorreu, rei de Hesbom, e a sua terra; começa
a possuí-la, e contende com eles em peleja. Neste dia
começarei a pôr um terror e um medo de ti diante dos povos que estão
debaixo de todo o céu; os que ouvirem a tua fama tremerão diante de
ti e se angustiarão. Então mandei mensageiros desde o deserto
de Quedemote a Siom, rei de Hesbom, com palavras de paz, dizendo:
Deixa-me passar pela tua terra; somente pela estrada irei;
não me desviarei para a direita nem para a esquerda. A
comida, para que eu coma, vender-me-ás por dinheiro, e dar-me-ás por
dinheiro a água para que eu beba; tão-somente deixa-me passar a pé;
como fizeram comigo os filhos de Esaú, que habitam em Seir, e
os moabitas que habitam em Ar; até que eu passe o Jordão, à terra
que o Senhor nosso Deus nos há de dar. Mas Siom, rei de
Hesbom, não nos quis deixar passar por sua terra, porquanto o Senhor
teu Deus endurecera o seu espírito, e fizera obstinado o seu coração
para to dar na tua mão, como hoje se vê. E o Senhor me disse:
Eis aqui, tenho começado a dar-te Siom, e a sua terra; começa, pois,
a possuí-la para que herdes a sua terra. E Siom saiu-nos ao
encontro, ele e todo o seu povo, à peleja, em Jaza; e o
Senhor nosso Deus no-lo entregou, e o ferimos a ele, e a seus
filhos, e a todo o seu povo. E naquele tempo tomamos todas as
suas cidades, e cada uma destruímos com os seus homens, mulheres e
crianças; não deixamos a ninguém. Somente tomamos por presa o
gado para nós, e o despojo das cidades que tínhamos tomado.
Desde Aroer, que está à margem do ribeiro de Arnom, e a
cidade que está junto ao ribeiro, até Gileade, nenhuma cidade houve
que de nós escapasse; tudo isto o Senhor nosso Deus nos entregou.
Somente à terra dos filhos de Amom não chegastes; nem a toda
a margem do ribeiro de Jaboque, nem às cidades da montanha, nem a
coisa alguma que nos proibira o Senhor nosso Deus.

\medskip

\lettrine{3} Depois nos viramos e subimos o caminho de Basã; e
Ogue, rei de Basã, nos saiu ao encontro, ele e todo o seu povo, à
peleja em Edrei. Então o Senhor me disse: Não o temas, porque a
ele e a todo o seu povo, e a sua terra, tenho dado na tua mão; e
far-lhe-ás como fizeste a Siom, rei dos amorreus, que habitava em
Hesbom. E também o Senhor nosso Deus nos deu na nossa mão a
Ogue, rei de Basã, e a todo o seu povo; de maneira que o ferimos até
que não lhe ficou sobrevivente algum. E naquele tempo tomamos
todas as suas cidades; nenhuma cidade houve que lhes não tomássemos;
sessenta cidades, toda a região de Argobe, o reino de Ogue em Basã.
Todas estas cidades eram fortificadas com altos muros, portas e
ferrolhos; e muitas outras cidades sem muros. E destruímo-las
como fizemos a Siom, rei de Hesbom, destruindo todas as cidades,
homens, mulheres e crianças. Porém todo o gado, e o despojo das
cidades, tomamos para nós por presa. Assim naquele tempo tomamos
a terra das mãos daqueles dois reis dos amorreus, que estavam além
do Jordão; desde o rio de Arnom, até ao monte de Hermom a Hermom
os sidônios chamam Siriom; porém os amorreus o chamam Senir);
todas as cidades do planalto, e todo o Gileade, e todo o
Basã, até Salcá e Edrei, cidades do reino de Ogue em Basã.
Porque só Ogue, o rei de Basã, restou dos gigantes; eis que o
seu leito, um leito de ferro, não está porventura em Rabá dos filhos
de Amom? De nove côvados, o seu comprimento, e de quatro côvados, a
sua largura, pelo côvado comum.

Tomamos, pois, esta terra em possessão naquele tempo: Desde
Aroer, que está junto ao ribeiro de Arnom, e a metade da montanha de
Gileade, com as suas cidades, tenho dado aos rubenitas e gaditas.
E o restante de Gileade, como também todo o Basã, o reino de
Ogue, dei à meia tribo de Manassés; toda aquela região de Argobe,
por todo o Basã, se chamava a terra dos gigantes. Jair, filho
de Manassés, alcançou toda a região de Argobe, até ao termo dos
gesuritas, e maacatitas, e a chamou de seu nome, Havote-Jair até
este dia. E a Maquir dei Gileade. Mas aos rubenitas e
gaditas dei desde Gileade até ao ribeiro de Arnom, cujo meio serve
de limite; e até ao ribeiro de Jaboque, o termo dos filhos de Amom.
Como também a campina, e o Jordão por termo; desde Quinerete
até ao mar da campina, o Mar Salgado, abaixo de Asdote-Pisga para o
oriente. E no mesmo tempo vos ordenei, dizendo: O Senhor
vosso Deus vos deu esta terra, para possuí-la; passai, pois, armados
vós, todos os homens valentes, diante de vossos irmãos, os filhos de
Israel. Tão-somente vossas mulheres, e vossas crianças, e
vosso gado (porque eu sei que tendes muito gado), ficarão nas vossas
cidades, que já vos tenho dado. Até que o Senhor dê descanso
a vossos irmãos como a vós; para que eles herdem também a terra que
o Senhor vosso Deus lhes há de dar além do Jordão; então voltareis
cada qual à sua herança que já vos tenho dado.

Também dei ordem a Josué no mesmo tempo, dizendo: Os teus olhos
têm visto tudo o que o Senhor vosso Deus tem feito a estes dois
reis; assim fará o Senhor a todos os reinos, a que tu passarás.
Não os temais, porque o Senhor vosso Deus é o que peleja por
vós. Também eu pedi graça ao Senhor no mesmo tempo, dizendo:
Senhor Deus! já começaste a mostrar ao teu servo a tua
grandeza e a tua forte mão; pois, que Deus há nos céus e na terra,
que possa fazer segundo as tuas obras, e segundo os teus grandes
feitos? Rogo-te que me deixes passar, para que veja esta boa
terra que está além do Jordão; esta boa montanha, e o Líbano!
Porém o Senhor indignou-se muito contra mim por causa de vós,
e não me ouviu; antes o Senhor me disse: Basta; não me fales mais
deste assunto; sobe ao cume de Pisga, e levanta os teus olhos
ao ocidente, e ao norte, e ao sul, e ao oriente, e vê com os teus
olhos; porque não passarás este Jordão. Manda, pois, a Josué,
e anima-o, e fortalece-o; porque ele passará adiante deste povo, e o
fará possuir a terra que verás. Assim ficamos neste vale,
defronte de Bete-Peor.

\medskip

\lettrine{4} Agora, pois, ó Israel, ouve os estatutos e os
juízos que eu vos ensino, para os cumprirdes; para que vivais, e
entreis, e possuais a terra que o Senhor Deus de vossos pais vos dá.
Não acrescentareis à palavra que vos mando, nem diminuireis
dela, para que guardeis os mandamentos do Senhor vosso Deus, que eu
vos mando. Os vossos olhos têm visto o que o Senhor fez por
causa de Baal-Peor; pois a todo o homem que seguiu a Baal-Peor o
Senhor teu Deus consumiu do meio de ti. Porém vós, que vos
achegastes ao Senhor vosso Deus, hoje todos estais vivos. Vedes
aqui vos tenho ensinado estatutos e juízos, como me mandou o Senhor
meu Deus; para que assim façais no meio da terra a qual ides a
herdar. Guardai-os pois, e cumpri-os, porque isso será a vossa
sabedoria e o vosso entendimento perante os olhos dos povos, que
ouvirão todos estes estatutos, e dirão: Este grande povo é nação
sábia e entendida. Pois, que nação há tão grande, que tenha
deuses tão chegados como o Senhor nosso Deus, todas as vezes que o
invocamos? E que nação há tão grande, que tenha estatutos e
juízos tão justos como toda esta lei que hoje ponho perante vós?
Tão-somente guarda-te a ti mesmo, e guarda bem a tua alma, que
não te esqueças daquelas coisas que os teus olhos têm visto, e não
se apartem do teu coração todos os dias da tua vida; e as farás
saber a teus filhos, e aos filhos de teus filhos. O dia em
que estiveste perante o Senhor teu Deus em Horebe, quando o Senhor
me disse: Ajunta-me este povo, e os farei ouvir as minhas palavras,
e aprendê-las-ão, para me temerem todos os dias que na terra
viverem, e as ensinarão a seus filhos; e vós vos chegastes, e
vos pusestes ao pé do monte; e o monte ardia em fogo até ao meio dos
céus, e havia trevas, e nuvens e escuridão; então o Senhor
vos falou do meio do fogo; a voz das palavras ouvistes; porém, além
da voz, não vistes figura alguma. Então vos anunciou ele a
sua aliança que vos ordenou cumprir, os dez mandamentos, e os
escreveu em duas tábuas de pedra. Também o Senhor me ordenou
ao mesmo tempo que vos ensinasse estatutos e juízos, para que os
cumprísseis na terra a qual passais a possuir. Guardai, pois,
com diligência as vossas almas, pois nenhuma figura vistes no dia em
que o Senhor, em Horebe, falou convosco do meio do fogo; para
que não vos corrompais, e vos façais alguma imagem esculpida na
forma de qualquer figura, semelhança de homem ou mulher;
figura de algum animal que haja na terra; figura de alguma
ave alada que voa pelos céus; figura de algum animal que se
arrasta sobre a terra; figura de algum peixe que esteja nas águas
debaixo da terra; que não levantes os teus olhos aos céus e
vejas o sol, e a lua, e as estrelas, todo o exército dos céus; e
sejas impelido a que te inclines perante eles, e sirvas àqueles que
o Senhor teu Deus repartiu a todos os povos debaixo de todos os
céus. Mas o Senhor vos tomou, e vos tirou da fornalha de
ferro do Egito, para que lhe sejais por povo hereditário, como neste
dia se vê. Também o Senhor se indignou contra mim por causa
das vossas palavras, e jurou que eu não passaria o Jordão, e que não
entraria na boa terra que o Senhor teu Deus te dará por herança.
Porque eu nesta terra morrerei, não passarei o Jordão; porém
vós o passareis, e possuireis aquela boa terra. Guardai-vos e
não vos esqueçais da aliança do Senhor vosso Deus, que tem feito
convosco, e não façais para vós escultura alguma, imagem de alguma
coisa que o Senhor vosso Deus vos proibiu. Porque o Senhor
teu Deus é um fogo que consome, um Deus zeloso. Quando, pois,
gerardes filhos, e filhos de filhos, e vos envelhecerdes na terra, e
vos corromperdes, e fizerdes alguma escultura, semelhança de alguma
coisa, e fizerdes o que é mau aos olhos do Senhor teu Deus, para o
provocar à ira; hoje tomo por testemunhas contra vós o céu e
a terra, que certamente logo perecereis da terra, a qual passais o
Jordão para a possuir; não prolongareis os vossos dias nela, antes
sereis de todo destruídos. E o Senhor vos espalhará entre os
povos, e ficareis poucos em número entre as nações às quais o Senhor
vos conduzirá. E ali servireis a deuses que são obra de mãos
de homens, madeira e pedra, que não vêem, nem ouvem, nem comem, nem
cheiram. Então dali buscarás ao Senhor teu Deus, e o acharás,
quando o buscares de todo o teu coração e de toda a tua alma.
Quando estiverdes em angústia, e todas estas coisas te
alcançarem, então nos últimos dias voltarás para o Senhor teu Deus,
e ouvirás a sua voz. Porquanto o Senhor teu Deus é Deus
misericordioso, e não te desamparará, nem te destruirá, nem se
esquecerá da aliança que jurou a teus pais. Agora, pois,
pergunta aos tempos passados, que te precederam desde o dia em que
Deus criou o homem sobre a terra, desde uma extremidade do céu até à
outra, se sucedeu jamais coisa tão grande como esta, ou se jamais se
ouviu coisa como esta? Ou se algum povo ouviu a voz de Deus
falando do meio do fogo, como tu a ouviste, e ficou vivo? Ou
se Deus intentou ir tomar para si um povo do meio de outro povo com
provas, com sinais, e com milagres, e com peleja, e com mão forte, e
com braço estendido, e com grandes espantos, conforme a tudo quanto
o Senhor vosso Deus vos fez no Egito aos vossos olhos? A ti
te foi mostrado para que soubesses que o Senhor é Deus; nenhum outro
há senão ele. Desde os céus te fez ouvir a sua voz, para te
ensinar, e sobre a terra te mostrou o seu grande fogo, e ouviste as
suas palavras do meio do fogo. E, porquanto amou teus pais, e
escolheu a sua descendência depois deles, te tirou do Egito diante
de si, com a sua grande força, para lançar fora de diante de
ti nações maiores e mais poderosas do que tu, para te introduzir e
te dar a sua terra por herança, como neste dia se vê. Por
isso hoje saberás, e refletirás no teu coração, que só o Senhor é
Deus, em cima no céu e em baixo na terra; nenhum outro há. E
guardarás os seus estatutos e os seus mandamentos, que te ordeno
hoje para que te vá bem a ti, e a teus filhos depois de ti, e para
que prolongues os dias na terra que o Senhor teu Deus te dá para
todo o sempre.

Então Moisés separou três cidades além do Jordão, do lado do
nascimento do sol; para que ali se acolhesse o homicida que
involuntariamente matasse o seu próximo a quem dantes não tivesse
ódio algum; e se acolhesse a uma destas cidades, e vivesse; a
Bezer, no deserto, no planalto, para os rubenitas; e a Ramote, em
Gileade, para os gaditas; e a Golã, em Basã, para os manassitas.
Esta é, pois, a lei que Moisés propôs aos filhos de Israel.
Estes são os testemunhos, e os estatutos, e os juízos, que
Moisés falou aos filhos de Israel, havendo saído do Egito;
além do Jordão, no vale defronte de Bete-Peor, na terra de
Siom, rei dos amorreus, que habitava em Hesbom, a quem feriu Moisés
e os filhos de Israel, havendo eles saído do Egito. E tomaram
a sua terra em possessão, como também a terra de Ogue, rei de Basã,
dois reis dos amorreus, que estavam além do Jordão, do lado do
nascimento do sol. Desde Aroer, que está à margem do ribeiro
de Arnom, até ao monte Sião, que é Hermom, e toda a campina
além do Jordão, do lado do oriente, até ao mar da campina, abaixo de
Asdote-Pisga.

\medskip

\lettrine{5} E chamou Moisés a todo o Israel, e disse-lhes:
Ouve, ó Israel, os estatutos e juízos que hoje vos falo aos ouvidos;
e aprendê-los-eis, e guardá-los-eis, para os cumprir. O Senhor
nosso Deus fez conosco aliança em Horebe. Não com nossos pais
fez o Senhor esta aliança, mas conosco, todos os que hoje aqui
estamos vivos. Face a face o Senhor falou conosco no monte, do
meio do fogo naquele tempo eu estava em pé entre o Senhor e vós,
para vos notificar a palavra do Senhor; porque temestes o fogo e não
subistes ao monte), dizendo:

Eu sou o Senhor teu Deus, que te tirei da terra do Egito, da casa
da servidão; não terás outros deuses diante de mim; não
farás para ti imagem de escultura, nem semelhança alguma do que há
em cima no céu, nem em baixo na terra, nem nas águas debaixo da
terra; não te encurvarás a elas, nem as servirás; porque eu, o
Senhor teu Deus, sou Deus zeloso, que visito a iniqüidade dos pais
nos filhos, até à terceira e quarta geração daqueles que me odeiam.
E faço misericórdia a milhares dos que me amam e guardam os
meus mandamentos. Não tomarás o nome do Senhor teu Deus em
vão; porque o Senhor não terá por inocente ao que tomar o seu nome
em vão. Guarda o dia de sábado, para o santificar, como te
ordenou o Senhor teu Deus. Seis dias trabalharás, e farás
todo o teu trabalho. Mas o sétimo dia é o sábado do Senhor
teu Deus; não farás nenhum trabalho nele, nem tu, nem teu filho, nem
tua filha, nem o teu servo, nem a tua serva, nem o teu boi, nem o
teu jumento, nem animal algum teu, nem o estrangeiro que está dentro
de tuas portas; para que o teu servo e a tua serva descansem como
tu; porque te lembrarás que foste servo na terra do Egito, e
que o Senhor teu Deus te tirou dali com mão forte e braço estendido;
por isso o Senhor teu Deus te ordenou que guardasses o dia de
sábado. Honra a teu pai e a tua mãe, como o Senhor teu Deus
te ordenou, para que se prolonguem os teus dias, e para que te vá
bem na terra que te dá o Senhor teu Deus. Não matarás.
Não adulterarás. Não furtarás. Não dirás falso
testemunho contra o teu próximo. Não cobiçarás a mulher do
teu próximo; e não desejarás a casa do teu próximo, nem o seu campo,
nem o seu servo, nem a sua serva, nem o seu boi, nem o seu jumento,
nem coisa alguma do teu próximo. Estas palavras falou o
Senhor a toda a vossa congregação no monte, do meio do fogo, da
nuvem e da escuridão, com grande voz, e nada acrescentou; e as
escreveu em duas tábuas de pedra, e a mim mas deu.

E sucedeu que, ouvindo a voz do meio das trevas, e vendo o monte
ardendo em fogo, vos achegastes a mim, todos os cabeças das vossas
tribos, e vossos anciãos; e dissestes: Eis aqui o Senhor
nosso Deus nos fez ver a sua glória e a sua grandeza, e ouvimos a
sua voz do meio do fogo; hoje vimos que Deus fala com o homem, e que
este permanece vivo. Agora, pois, por que morreríamos? Pois
este grande fogo nos consumiria; se ainda mais ouvíssemos a voz do
Senhor nosso Deus morreríamos. Porque, quem há de toda a
carne, que ouviu a voz do Deus vivente falando do meio do fogo, como
nós, e ficou vivo? Chega-te tu, e ouve tudo o que disser o
Senhor nosso Deus; e tu nos dirás tudo o que te disser o Senhor
nosso Deus, e o ouviremos, e o cumpriremos. Ouvindo, pois, o
Senhor as vossas palavras, quando me faláveis, o Senhor me disse: Eu
ouvi as palavras deste povo, que eles te disseram; em tudo falaram
bem. Quem dera que eles tivessem tal coração que me temessem,
e guardassem todos os meus mandamentos todos os dias, para que bem
lhes fosse a eles e a seus filhos para sempre. Vai,
dize-lhes: Tornai-vos às vossas tendas. Tu, porém, fica-te
aqui comigo, para que eu a ti te diga todos os mandamentos, e
estatutos, e juízos, que tu lhes hás de ensinar, para que cumpram na
terra que eu lhes darei para possuí-la. Olhai, pois, que
façais como vos mandou o Senhor vosso Deus; não vos desviareis, nem
para a direita nem para a esquerda. Andareis em todo o
caminho que vos manda o Senhor vosso Deus, para que vivais e bem vos
suceda, e prolongueis os dias na terra que haveis de possuir.

\medskip

\lettrine{6} Estes, pois, são os mandamentos, os estatutos e
os juízos que mandou o Senhor vosso Deus para ensinar-vos, para que
os cumprísseis na terra a que passais a possuir; para que temas
ao Senhor teu Deus, e guardes todos os seus estatutos e mandamentos,
que eu te ordeno, tu, e teu filho, e o filho de teu filho, todos os
dias da tua vida, e que teus dias sejam prolongados. Ouve, pois,
ó Israel, e atenta em os guardares, para que bem te suceda, e muito
te multipliques, como te disse o Senhor Deus de teus pais, na terra
que mana leite e mel.

Ouve, Israel, o Senhor nosso Deus é o único Senhor. Amarás,
pois, o Senhor teu Deus de todo o teu coração, e de toda a tua alma,
e de todas as tuas forças. E estas palavras, que hoje te ordeno,
estarão no teu coração; e as ensinarás a teus filhos e delas
falarás assentado em tua casa, e andando pelo caminho, e deitando-te
e levantando-te. Também as atarás por sinal na tua mão, e te
serão por frontais entre os teus olhos. E as escreverás nos
umbrais de tua casa, e nas tuas portas. Quando, pois, o
Senhor teu Deus te introduzir na terra que jurou a teus pais,
Abraão, Isaque e Jacó, que te daria, com grandes e boas cidades, que
tu não edificaste, e casas cheias de todo o bem, que tu não
encheste, e poços cavados, que tu não cavaste, vinhas e olivais, que
tu não plantaste, e comeres, e te fartares, guarda-te, que
não te esqueças do Senhor, que te tirou da terra do Egito, da casa
da servidão. O Senhor teu Deus temerás e a ele servirás, e
pelo seu nome jurarás. Não seguireis outros deuses, os deuses
dos povos que houver ao redor de vós; porque o Senhor teu
Deus é um Deus zeloso no meio de ti, para que a ira do Senhor teu
Deus se não acenda contra ti e te destrua de sobre a face da terra.
Não tentareis o Senhor vosso Deus, como o tentastes em Massá.

Diligentemente guardareis os mandamentos do Senhor vosso Deus,
como também os seus testemunhos, e seus estatutos, que te tem
mandado. E farás o que é reto e bom aos olhos do Senhor, para
que bem te suceda, e entres, e possuas a boa terra, a qual o Senhor
jurou dar a teus pais. Para que lance fora a todos os teus
inimigos de diante de ti, como o Senhor tem falado. Quando
teu filho te perguntar no futuro, dizendo: Que significam os
testemunhos, e estatutos e juízos que o Senhor nosso Deus vos
ordenou? Então dirás a teu filho: Éramos servos de Faraó no
Egito; porém o Senhor, com mão forte, nos tirou do Egito; e o
Senhor, aos nossos olhos, fez sinais e maravilhas, grandes e
terríveis, contra o Egito, contra Faraó e toda sua casa; e
dali nos tirou, para nos levar, e nos dar a terra que jurara a
nossos pais. E o Senhor nos ordenou que cumpríssemos todos
estes estatutos, que temêssemos ao Senhor nosso Deus, para o nosso
perpétuo bem, para nos guardar em vida, como no dia de hoje.
E será para nós justiça, quando tivermos cuidado de cumprir
todos estes mandamentos perante o Senhor nosso Deus, como nos tem
ordenado.

\medskip

\lettrine{7} Quando o Senhor teu Deus te houver introduzido na
terra, à qual vais para a possuir, e tiver lançado fora muitas
nações de diante de ti, os heteus, e os girgaseus, e os amorreus, e
os cananeus, e os perizeus, e os heveus, e os jebuseus, sete nações
mais numerosas e mais poderosas do que tu; e o Senhor teu Deus
as tiver dado diante de ti, para as ferir, totalmente as destruirás;
não farás com elas aliança, nem terás piedade delas; nem te
aparentarás com elas; não darás tuas filhas a seus filhos, e não
tomarás suas filhas para teus filhos; pois fariam desviar teus
filhos de mim, para que servissem a outros deuses; e a ira do Senhor
se acenderia contra vós, e depressa vos consumiria. Porém assim
lhes fareis: Derrubareis os seus altares, quebrareis as suas
estátuas; e cortareis os seus bosques, e queimareis a fogo as suas
imagens de escultura. Porque povo santo és ao Senhor teu Deus; o
Senhor teu Deus te escolheu, para que lhe fosses o seu povo
especial, de todos os povos que há sobre a terra. O Senhor não
tomou prazer em vós, nem vos escolheu, porque a vossa multidão era
mais do que a de todos os outros povos, pois vós éreis menos em
número do que todos os povos; mas, porque o Senhor vos amava, e
para guardar o juramento que fizera a vossos pais, o Senhor vos
tirou com mão forte e vos resgatou da casa da servidão, da mão de
Faraó, rei do Egito. Saberás, pois, que o Senhor teu Deus, ele é
Deus, o Deus fiel, que guarda a aliança e a misericórdia até mil
gerações aos que o amam e guardam os seus mandamentos. E
retribui no rosto qualquer dos que o odeiam, fazendo-o perecer; não
será tardio ao que o odeia; em seu rosto lho pagará. Guarda,
pois, os mandamentos e os estatutos e os juízos que hoje te mando
cumprir.

Será, pois, que, se ouvindo estes juízos, os guardardes e
cumprirdes, o Senhor teu Deus te guardará a aliança e a misericórdia
que jurou a teus pais; e amar-te-á, e abençoar-te-á, e te
fará multiplicar; abençoará o fruto do teu ventre, e o fruto da tua
terra, o teu grão, e o teu mosto, e o teu azeite, e a criação das
tuas vacas, e o rebanho do teu gado miúdo, na terra que jurou a teus
pais dar-te. Bendito serás mais do que todos os povos; não
haverá estéril entre ti, seja homem, seja mulher, nem entre os teus
animais. E o Senhor de ti desviará toda a enfermidade; sobre
ti não porá nenhuma das más doenças dos egípcios, que bem sabes,
antes as porá sobre todos os que te odeiam. Pois consumirás a
todos os povos que te der o Senhor teu Deus; os teus olhos não os
poupará; e não servirás a seus deuses, pois isto te seria por laço.
Se disseres no teu coração: Estas nações são mais numerosas
do que eu; como as poderei lançar fora? Delas não tenhas
temor; não deixes de te lembrar do que o Senhor teu Deus fez a Faraó
e a todos os egípcios; das grandes provas que viram os teus
olhos, e dos sinais, e maravilhas, e mão forte, e braço estendido,
com que o Senhor teu Deus te tirou; assim fará o Senhor teu Deus com
todos os povos, diante dos quais tu temes. E mais, o Senhor
teu Deus entre eles mandará vespões, até que pereçam os que ficarem
e se esconderem de diante de ti. Não te espantes diante
deles; porque o Senhor teu Deus está no meio de ti, Deus grande e
terrível. E o Senhor teu Deus lançará fora estas nações pouco
a pouco de diante de ti; não poderás destruí-las todas de pronto,
para que as feras do campo não se multipliquem contra ti. E o
Senhor teu Deus as entregará a ti, e lhes infligirá uma grande
confusão até que sejam consumidas. Também os seus reis te
entregará na mão, para que apagues os seus nomes de debaixo dos
céus; nenhum homem resistirá diante de ti, até que os destruas.
As imagens de escultura de seus deuses queimarás a fogo; a
prata e o ouro que estão sobre elas não cobiçarás, nem os tomarás
para ti, para que não te enlaces neles; pois abominação é ao Senhor
teu Deus. Não porás, pois, abominação em tua casa, para que
não sejas anátema, assim como ela; de todo a detestarás, e de todo a
abominarás, porque anátema é.

\medskip

\lettrine{8} Todos os mandamentos que hoje vos ordeno
guardareis para os cumprir; para que vivais, e vos multipliqueis, e
entreis, e possuais a terra que o Senhor jurou a vossos pais. E
te lembrarás de todo o caminho, pelo qual o Senhor teu Deus te guiou
no deserto estes quarenta anos, para te humilhar, e te provar, para
saber o que estava no teu coração, se guardarias os seus
mandamentos, ou não. E te humilhou, e te deixou ter fome, e te
sustentou com o maná, que tu não conheceste, nem teus pais o
conheceram; para te dar a entender que o homem não viverá só de pão,
mas de tudo o que sai da boca do Senhor viverá o homem. Nunca se
envelheceu a tua roupa sobre ti, nem se inchou o teu pé nestes
quarenta anos. Sabes, pois, no teu coração que, como um homem
castiga a seu filho, assim te castiga o Senhor teu Deus. E
guarda os mandamentos do Senhor teu Deus, para andares nos seus
caminhos e para o temeres. Porque o Senhor teu Deus te põe numa
boa terra, terra de ribeiros de águas, de fontes, e de mananciais,
que saem dos vales e das montanhas; terra de trigo e cevada, e
de vides e figueiras, e romeiras; terra de oliveiras, de azeite e
mel. Terra em que comerás o pão sem escassez, e nada te faltará
nela; terra cujas pedras são ferro, e de cujos montes tu cavarás o
cobre.

Quando, pois, tiveres comido, e fores farto, louvarás ao Senhor
teu Deus pela boa terra que te deu. Guarda-te que não te
esqueças do Senhor teu Deus, deixando de guardar os seus
mandamentos, e os seus juízos, e os seus estatutos que hoje te
ordeno; para não suceder que, havendo tu comido e fores
farto, e havendo edificado boas casas, e habitando-as, e se
tiverem aumentado os teus gados e os teus rebanhos, e se acrescentar
a prata e o ouro, e se multiplicar tudo quanto tens, se eleve
o teu coração e te esqueças do Senhor teu Deus, que te tirou da
terra do Egito, da casa da servidão; que te guiou por aquele
grande e terrível deserto de serpentes ardentes, e de escorpiões, e
de terra seca, em que não havia água; e tirou água para ti da rocha
pederneira\footnote{Pedra muito dura, que produz faíscas, quando
ferida com um fragmento de aço; sílex, pedernal, pedra-de-fogo.};
que no deserto te sustentou com maná, que teus pais não
conheceram; para te humilhar, e para te provar, para no fim te fazer
bem; e digas no teu coração: A minha força, e a fortaleza da
minha mão, me adquiriu este poder. Antes te lembrarás do
Senhor teu Deus, que ele é o que te dá força para adquirires poder;
para confirmar a sua aliança, que jurou a teus pais, como se vê
neste dia. Será, porém, que, se de qualquer modo te
esqueceres do Senhor teu Deus, e se ouvires outros deuses, e os
servires, e te inclinares perante eles, hoje eu testifico contra vós
que certamente perecereis. Como as nações que o Senhor
destruiu diante de vós, assim vós perecereis, porquanto não queríeis
obedecer à voz do Senhor vosso Deus.

\medskip

\lettrine{9} Ouve, ó Israel, hoje passarás o Jordão, para
entrares a possuir nações maiores e mais fortes do que tu; cidades
grandes, e muradas até aos céus; um povo grande e alto, filhos
de gigantes, que tu conheces, e de que já ouviste. Quem resistiria
diante dos filhos dos gigantes? Sabe, pois, hoje que o Senhor
teu Deus, que passa adiante de ti, é um fogo consumidor, que os
destruirá, e os derrubará de diante de ti; e tu os lançarás fora, e
cedo os desfarás, como o Senhor te tem falado. Quando, pois, o
Senhor teu Deus os lançar fora de diante de ti, não fales no teu
coração, dizendo: Por causa da minha justiça é que o Senhor me
trouxe a esta terra para a possuir; porque pela impiedade destas
nações é que o Senhor as lança fora de diante de ti. Não é por
causa da tua justiça, nem pela retidão do teu coração que entras a
possuir a sua terra, mas pela impiedade destas nações o Senhor teu
Deus as lança fora, de diante de ti, e para confirmar a palavra que
o Senhor jurou a teus pais, Abraão, Isaque e Jacó. Sabe, pois,
que não é por causa da tua justiça que o Senhor teu Deus te dá esta
boa terra para possuí-la, pois tu és povo obstinado.

Lembra-te, e não te esqueças, de que muito provocaste à ira ao
Senhor teu Deus no deserto; desde o dia em que saístes do Egito, até
que chegastes a esse lugar, rebeldes fostes contra o Senhor;
pois em Horebe provocastes à ira o Senhor, tanto que o Senhor se
indignou contra vós para os destruir. Subindo eu ao monte a
receber as tábuas de pedra, as tábuas da aliança que o Senhor fizera
convosco, então fiquei no monte quarenta dias e quarenta noites; pão
não comi, e água não bebi; e o Senhor me deu as duas tábuas
de pedra, escritas com o dedo de Deus; e nelas estava escrito
conforme a todas aquelas palavras que o Senhor tinha falado convosco
no monte, do meio do fogo, no dia da assembléia. Sucedeu,
pois, que ao fim dos quarenta dias e quarenta noites, o Senhor me
deu as duas tábuas de pedra, as tábuas da aliança. E o Senhor
me disse: Levanta-te, desce depressa daqui, porque o teu povo, que
tiraste do Egito, já se tem corrompido; cedo se desviaram do caminho
que eu lhes tinha ordenado; fizeram para si uma imagem de fundição.
Falou-me ainda o Senhor, dizendo: Atentei para este povo, e
eis que ele é povo obstinado; deixa-me que os destrua, e
apague o seu nome de debaixo dos céus; e te faça a ti nação mais
poderosa e mais numerosa do que esta. Então virei-me, e desci
do monte; o qual ardia em fogo e as duas tábuas da aliança estavam
em ambas as minhas mãos. E olhei, e eis que havíeis pecado
contra o Senhor vosso Deus; vós tínheis feito um bezerro de
fundição; cedo vos desviastes do caminho que o Senhor vos ordenara.
Então peguei das duas tábuas, e as arrojei das minhas mãos, e
as quebrei diante dos vossos olhos. E me lancei perante o
Senhor, como antes, quarenta dias, e quarenta noites; não comi pão e
não bebi água, por causa de todo o vosso pecado que havíeis
cometido, fazendo mal aos olhos do Senhor, para o provocar à ira.
Porque temi por causa da ira e do furor, com que o Senhor
tanto estava irado contra vós para vos destruir; porém ainda por
esta vez o Senhor me ouviu. Também o Senhor se irou muito
contra Arão para o destruir; mas também orei por Arão ao mesmo
tempo. Porém eu tomei o vosso pecado, o bezerro que tínheis
feito, e o queimei a fogo, e o pisei, moendo-o bem, até que se
desfez em pó; e o seu pó lancei no ribeiro que descia do monte.
Também em Taberá, e em Massá, e em Quibrote-Hataavá
provocastes muito a ira do Senhor. Quando também o Senhor vos
enviou de Cades-Barnéia, dizendo: Subi, e possuí a terra, que vos
tenho dado: rebeldes fostes ao mandado do Senhor vosso Deus, e não o
crestes, e não obedecestes à sua voz. Rebeldes fostes contra
o Senhor desde o dia em que vos conheci. E prostrei-me
perante o Senhor; aqueles quarenta dias e quarenta noites estive
prostrado, porquanto o Senhor dissera que vos queria destruir.
E orei ao Senhor, dizendo: Senhor Deus, não destruas o teu
povo e a tua herança, que resgataste com a tua grandeza, que tiraste
do Egito com mão forte. Lembra-te dos teus servos, Abraão,
Isaque, e Jacó. Não atentes para a dureza deste povo, nem para a sua
impiedade, nem para o seu pecado; para que o povo da terra
donde nos tiraste não diga: Porquanto o Senhor não os pode
introduzir na terra de que lhes tinha falado, e porque os odiava, os
tirou para matá-los no deserto; todavia são eles o teu povo e
a tua herança, que tiraste com a tua grande força e com o teu braço
estendido.

\medskip

\lettrine{10} Naquele mesmo tempo me disse o Senhor: Alisa
duas tábuas de pedra, como as primeiras, e sobe a mim ao monte, e
faze-te uma arca de madeira; e naquelas tábuas escreverei as
palavras que estavam nas primeiras tábuas, que quebraste, e as porás
na arca. Assim, fiz uma arca de madeira de acácia, e alisei duas
tábuas de pedra, como as primeiras; e subi ao monte com as duas
tábuas na minha mão. Então escreveu nas tábuas, conforme à
primeira escritura, os dez mandamentos, que o Senhor vos falara no
dia da assembléia, no monte, do meio do fogo; e o Senhor mas deu a
mim; e virei-me, e desci do monte, e pus as tábuas na arca que
fizera; e ali estão, como o Senhor me ordenou. E partiram os
filhos de Israel de Beerote-Bene-Jaacã a Moserá; ali faleceu Arão, e
ali foi sepultado, e Eleazar, seu filho, administrou o sacerdócio em
seu lugar. Dali partiram a Gudgodá, e de Gudgodá a Jotbatá,
terra de ribeiros de águas. No mesmo tempo o Senhor separou a
tribo de Levi, para levar a arca da aliança do Senhor, para estar
diante do Senhor, para o servir, e para abençoar em seu nome até ao
dia de hoje. Por isso Levi não tem parte nem herança com seus
irmãos; o Senhor é a sua herança, como o Senhor teu Deus lhe tem
falado. E eu estive no monte, como nos primeiros dias,
quarenta dias e quarenta noites; e o Senhor me ouviu ainda por esta
vez; não quis o Senhor destruir-te. Porém o Senhor me disse:
Levanta-te, põe-te a caminho adiante do povo, para que entrem, e
possuam a terra que jurei dar a seus pais.

Agora, pois, ó Israel, que é que o Senhor teu Deus pede de ti,
senão que temas o Senhor teu Deus, que andes em todos os seus
caminhos, e o ames, e sirvas ao Senhor teu Deus com todo o teu
coração e com toda a tua alma, que guardes os mandamentos do
Senhor, e os seus estatutos, que hoje te ordeno, para o teu bem?
Eis que os céus e os céus dos céus são do Senhor teu Deus, a
terra e tudo o que nela há. Tão-somente o Senhor se agradou
de teus pais para os amar; e a vós, descendência deles, escolheu,
depois deles, de todos os povos como neste dia se vê.
Circuncidai, pois, o prepúcio do vosso coração, e não mais
endureçais a vossa cerviz. Pois o Senhor vosso Deus é o Deus
dos deuses, e o Senhor dos senhores, o Deus grande, poderoso e
terrível, que não faz acepção de pessoas, nem aceita recompensas;
que faz justiça ao órfão e à viúva, e ama o estrangeiro,
dando-lhe pão e roupa. Por isso amareis o estrangeiro, pois
fostes estrangeiros na terra do Egito. Ao Senhor teu Deus
temerás; a ele servirás, e a ele te chegarás, e pelo seu nome
jurarás. Ele é o teu louvor e o teu Deus, que te fez estas
grandes e terríveis coisas que os teus olhos têm visto. Com
setenta almas teus pais desceram ao Egito; e agora o Senhor teu Deus
te pôs como as estrelas dos céus em multidão.

\medskip

\lettrine{11} Amarás, pois, ao Senhor teu Deus, e guardarás as
suas ordenanças, e os seus estatutos, e os seus juízos, e os seus
mandamentos, todos os dias. E hoje sabereis que falo, não com
vossos filhos, que o não sabem, e não viram a instrução do Senhor
vosso Deus, a sua grandeza, a sua mão forte, e o seu braço
estendido; nem tampouco os seus sinais, nem os seus feitos, que
fez no meio do Egito a Faraó, rei do Egito, e a toda a sua terra;
nem o que fez ao exército dos egípcios, aos seus cavalos e aos
seus carros, fazendo passar sobre eles as águas do Mar Vermelho
quando vos perseguiam, e como o Senhor os destruiu, até ao dia de
hoje; nem o que vos fez no deserto, até que chegastes a este
lugar; e o que fez a Datã e a Abirão, filhos de Eliabe, filho de
Rúben; como a terra abriu a sua boca e os tragou com as suas casas e
com as suas tendas, como também tudo o que subsistia, e lhes
pertencia, no meio de todo o Israel; porquanto os vossos olhos
são os que viram toda a grande obra que fez o Senhor.

Guardai, pois, todos os mandamentos que eu vos ordeno hoje, para
que sejais fortes, e entreis, e ocupeis a terra que passais a
possuir; e para que prolongueis os dias na terra que o Senhor
jurou dar a vossos pais e à sua descendência, terra que mana leite e
mel. Porque a terra que passas a possuir não é como a terra
do Egito, de onde saíste, em que semeavas a tua semente, e a regavas
com o teu pé, como a uma horta. Mas a terra que passais a
possuir é terra de montes e de vales; da chuva dos céus beberá as
águas; terra de que o Senhor teu Deus tem cuidado; os olhos
do Senhor teu Deus estão sobre ela continuamente, desde o princípio
até ao fim do ano. E será que, se diligentemente obedecerdes
a meus mandamentos que hoje vos ordeno, de amar ao Senhor vosso
Deus, e de o servir de todo o vosso coração e de toda a vossa alma,
então darei a chuva da vossa terra a seu tempo, a temporã e a
serôdia, para que recolhais o vosso grão, e o vosso mosto e o vosso
azeite. E darei erva no teu campo aos teus animais, e
comerás, e fartar-te-ás. Guardai-vos, que o vosso coração não
se engane, e vos desvieis, e sirvais a outros deuses, e vos
inclineis perante eles; e a ira do Senhor se acenda contra
vós, e feche ele os céus, e não haja água, e a terra não dê o seu
fruto, e cedo pereçais da boa terra que o Senhor vos dá.

Ponde, pois, estas minhas palavras no vosso coração e na vossa
alma, e atai-as por sinal na vossa mão, para que estejam por
frontais entre os vossos olhos. E ensinai-as a vossos filhos,
falando delas assentado em tua casa, e andando pelo caminho, e
deitando-te, e levantando-te; e escreve-as nos umbrais de tua
casa, e nas tuas portas; para que se multipliquem os vossos
dias e os dias de vossos filhos na terra que o Senhor jurou a vossos
pais dar-lhes, como os dias dos céus sobre a terra. Porque se
diligentemente guardardes todos estes mandamentos, que vos ordeno
para os guardardes, amando ao Senhor vosso Deus, andando em todos os
seus caminhos, e a ele vos achegardes, também o Senhor, de
diante de vós, lançará fora todas estas nações, e possuireis nações
maiores e mais poderosas do que vós. Todo o lugar que pisar a
planta do vosso pé será vosso; desde o deserto, e desde o Líbano,
desde o rio, o rio Eufrates, até ao mar ocidental, será o vosso
termo. Ninguém resistirá diante de vós; o Senhor vosso Deus
porá sobre toda a terra, que pisardes, o vosso terror e o temor de
vós, como já vos tem dito.

Eis que hoje eu ponho diante de vós a bênção e a maldição;
a bênção, quando cumprirdes os mandamentos do Senhor vosso
Deus, que hoje vos mando; porém a maldição, se não cumprirdes
os mandamentos do Senhor vosso Deus, e vos desviardes do caminho que
hoje vos ordeno, para seguirdes outros deuses que não conhecestes.
E será que, quando o Senhor teu Deus te introduzir na terra,
a que vais para possuí-la, então pronunciarás a bênção sobre o monte
Gerizim, e a maldição sobre o monte Ebal. Porventura não
estão eles além do Jordão, junto ao caminho do pôr do sol, na terra
dos cananeus, que habitam na campina defronte de Gilgal, junto aos
carvalhais de Moré? Porque passareis o Jordão para entrardes
a possuir a terra, que vos dá o Senhor vosso Deus; e a possuireis, e
nela habitareis. Tende, pois, cuidado em cumprir todos os
estatutos e os juízos, que eu hoje vos proponho.

\medskip

\lettrine{12} Estes são os estatutos e os juízos que tereis
cuidado em cumprir na terra que vos deu o Senhor Deus de vossos
pais, para a possuir todos os dias que viverdes sobre a terra.
Totalmente destruireis todos os lugares, onde as nações que
possuireis serviram os seus deuses, sobre as altas montanhas, e
sobre os outeiros, e debaixo de toda a árvore frondosa; e
derrubareis os seus altares, e quebrareis as suas estátuas, e os
seus bosques queimareis a fogo, e destruireis as imagens esculpidas
dos seus deuses, e apagareis o seu nome daquele lugar. Assim não
fareis ao Senhor vosso Deus.

Mas o lugar que o Senhor vosso Deus escolher de todas as vossas
tribos, para ali pôr o seu nome, buscareis, para sua habitação, e
ali vireis. E ali trareis os vossos holocaustos, e os vossos
sacrifícios, e os vossos dízimos, e a oferta alçada da vossa mão, e
os vossos votos, e as vossas ofertas voluntárias, e os primogênitos
das vossas vacas e das vossas ovelhas. E ali comereis perante o
Senhor vosso Deus, e vos alegrareis em tudo em que puserdes a vossa
mão, vós e as vossas casas, no que abençoar o Senhor vosso Deus.
Não fareis conforme a tudo o que hoje fazemos aqui, cada qual
tudo o que bem parece aos seus olhos. Porque até agora não
entrastes no descanso e na herança que vos dá o Senhor vosso Deus.
Mas passareis o Jordão, e habitareis na terra que vos fará
herdar o Senhor vosso Deus; e vos dará repouso de todos os vossos
inimigos em redor, e morareis seguros. Então haverá um lugar
que escolherá o Senhor vosso Deus para ali fazer habitar o seu nome;
ali trareis tudo o que vos ordeno; os vossos holocaustos, e os
vossos sacrifícios, e os vossos dízimos, e a oferta alçada da vossa
mão, e toda a escolha dos vossos votos que fizerdes ao Senhor.
E vos alegrareis perante o Senhor vosso Deus, vós, e vossos
filhos, e vossas filhas, e os vossos servos, e as vossas servas, e o
levita que está dentro das vossas portas; pois convosco não tem
parte nem herança. Guarda-te, que não ofereças os teus
holocaustos em todo o lugar que vires; mas no lugar que o
Senhor escolher numa das tuas tribos ali oferecerás os teus
holocaustos, e ali farás tudo o que te ordeno. Porém,
conforme a todo o desejo da tua alma, matarás e comerás carne,
dentro das tuas portas, segundo a bênção do Senhor teu Deus, que te
dá em todas as tuas portas; o imundo e o limpo dela comerá, como do
corço e do veado; tão-somente o sangue não comereis; sobre a
terra o derramareis como água. Dentro das tuas portas não
poderás comer o dízimo do teu grão, nem do teu mosto, nem do teu
azeite, nem os primogênitos das tuas vacas, nem das tuas ovelhas;
nem nenhum dos teus votos, que houveres prometido, nem as tuas
ofertas voluntárias, nem a oferta alçada da tua mão. Mas os
comerás perante o Senhor teu Deus, no lugar que escolher o Senhor
teu Deus, tu, e teu filho, e a tua filha, e o teu servo, e a tua
serva, e o levita que está dentro das tuas portas; e perante o
Senhor teu Deus te alegrarás em tudo em que puseres a tua mão.
Guarda-te, que não desampares ao levita todos os teus dias na
terra. Quando o Senhor teu Deus dilatar os teus termos, como
te disse, e disseres: Comerei carne; porquanto a tua alma tem desejo
de comer carne; conforme a todo o desejo da tua alma, comerás carne.
Se estiver longe de ti o lugar que o Senhor teu Deus
escolher, para ali pôr o seu nome, então matarás das tuas vacas e
das tuas ovelhas, que o Senhor te tiver dado, como te tenho
ordenado; e comerás dentro das tuas portas, conforme a todo o desejo
da tua alma. Porém, como se come o corço e o veado, assim
comerás; o imundo e o limpo também comerão deles. Somente
esforça-te para que não comas o sangue; pois o sangue é vida; pelo
que não comerás a vida com a carne; não o comerás; na terra o
derramarás como água. Não o comerás; para que bem te suceda a
ti, e a teus filhos, depois de ti, quando fizeres o que for reto aos
olhos do Senhor. Porém, as coisas santas que tiveres, e os
teus votos tomarás, e virás ao lugar que o Senhor escolher. E
oferecerás os teus holocaustos, a carne e o sangue sobre o altar do
Senhor teu Deus; e o sangue dos teus sacrifícios se derramará sobre
o altar do Senhor teu Deus; porém a carne comerás. Guarda e
ouve todas estas palavras que te ordeno, para que bem te suceda a ti
e a teus filhos depois de ti para sempre, quando fizeres o que for
bom e reto aos olhos do Senhor teu Deus. Quando o Senhor teu
Deus desarraigar de diante de ti as nações, aonde vais a possuí-las,
e as possuíres e habitares na sua terra, guarda-te, que não
te enlaces seguindo-as, depois que forem destruídas diante de ti; e
que não perguntes acerca dos seus deuses, dizendo: Assim como
serviram estas nações os seus deuses, do mesmo modo também farei eu.
Assim não farás ao Senhor teu Deus; porque tudo o que é
abominável ao Senhor, e que o aborrece, fizeram eles a seus deuses;
pois até seus filhos e suas filhas queimaram no fogo aos seus
deuses. Tudo o que eu te ordeno, observarás para fazer; nada
lhe acrescentarás nem diminuirás.

\medskip

\lettrine{13} Quando profeta ou sonhador de sonhos se levantar
no meio de ti, e te der um sinal ou prodígio, e suceder o tal
sinal ou prodígio, de que te houver falado, dizendo: Vamos após
outros deuses, que não conheceste, e sirvamo-los; não ouvirás as
palavras daquele profeta ou sonhador de sonhos; porquanto o Senhor
vosso Deus vos prova, para saber se amais o Senhor vosso Deus com
todo o vosso coração, e com toda a vossa alma. Após o Senhor
vosso Deus andareis, e a ele temereis, e os seus mandamentos
guardareis, e a sua voz ouvireis, e a ele servireis, e a ele vos
achegareis. E aquele profeta ou sonhador de sonhos morrerá, pois
falou rebeldia contra o Senhor vosso Deus, que vos tirou da terra do
Egito, e vos resgatou da casa da servidão, para te apartar do
caminho que te ordenou o Senhor teu Deus, para andares nele: assim
tirarás o mal do meio de ti.

Quando te incitar teu irmão, filho da tua mãe, ou teu filho, ou
tua filha, ou a mulher do teu seio, ou teu amigo, que te é como a
tua alma, dizendo-te em segredo: Vamos, e sirvamos a outros deuses
que não conheceste, nem tu nem teus pais; dentre os deuses dos
povos que estão em redor de vós, perto ou longe de ti, desde uma
extremidade da terra até à outra extremidade; não consentirás
com ele, nem o ouvirás; nem o teu olho o poupará, nem terás piedade
dele, nem o esconderás; mas certamente o matarás; a tua mão será
a primeira contra ele, para o matar; e depois a mão de todo o povo.
E o apedrejarás, até que morra, pois te procurou apartar do
Senhor teu Deus, que te tirou da terra do Egito, da casa da
servidão; para que todo o Israel o ouça e o tema, e não torne
a fazer semelhante maldade no meio de ti.

Quando ouvires dizer, de alguma das tuas cidades que o Senhor teu
Deus te dá para ali habitar: Uns homens, filhos de Belial,
que saíram do meio de ti, incitaram os moradores da sua cidade,
dizendo: Vamos, e sirvamos a outros deuses que não conhecestes;
então inquirirás e investigarás, e com diligência
perguntarás; e eis que, sendo verdade, e certo que se fez tal
abominação no meio de ti; certamente ferirás, ao fio da
espada, os moradores daquela cidade, destruindo a ela e a tudo o que
nela houver, até os animais. E ajuntarás todo o seu despojo
no meio da sua praça; e a cidade e todo o seu despojo queimarás
totalmente para o Senhor teu Deus, e será montão perpétuo, nunca
mais se edificará. Também não se pegará à tua mão nada do
anátema, para que o Senhor se aparte do ardor da sua ira, e te faça
misericórdia, e tenha piedade de ti, e te multiplique, como jurou a
teus pais; quando ouvires a voz do Senhor teu Deus, para
guardares todos os seus mandamentos que hoje te ordeno; para fazeres
o que for reto aos olhos do Senhor teu Deus.

\medskip

\lettrine{14} Filhos sois do Senhor vosso Deus; não vos dareis
golpes, nem fareis calva entre vossos olhos por causa de algum
morto. Porque és povo santo ao Senhor teu Deus; e o Senhor te
escolheu, de todos os povos que há sobre a face da terra, para lhe
seres o seu próprio povo. Nenhuma coisa abominável comereis. 4
Estes são os animais que comereis: o boi, a ovelha, e a cabra. O
veado e a corça, e o búfalo, e a cabra montês, e o texugo, e a
camurça\footnote{Mamífero caprino (Rupicapra rupicapra), facilmente
domesticável, de cornos verticais que terminam em báculos virados
para trás, corpo forte e pesado, pescoço curto e membros robustos.
Adaptado a grandes altitudes, habita a Europa e o Oriente Próximo.},
e o gamo. Todo o animal que tem unhas fendidas, divididas em
duas, que rumina, entre os animais, aquilo comereis. Porém estes
não comereis, dos que somente ruminam, ou que têm a unha fendida: o
camelo, e a lebre, e o coelho, porque ruminam mas não têm a unha
fendida; imundos vos serão. Nem o porco, porque tem unha
fendida, mas não rumina; imundo vos será; não comereis da carne
destes, e não tocareis nos seus cadáveres. Isto comereis de tudo
o que há nas águas; tudo o que tem barbatanas e escamas comereis.
Mas tudo o que não tiver barbatanas nem escamas não o
comereis; imundo vos será. Toda a ave limpa comereis.
Porém estas são as que não comereis: a águia, e o
quebrantosso, e o xofrango\footnote{A águia-pesqueira quando nova.},
e o abutre, e o falcão, e o milhafre\footnote{Ave de rapina
européia, falconídea (Milvus milvus).}, segundo a sua espécie.
E todo o corvo, segundo a sua espécie. E o avestruz, e
o mocho, e a gaivota, e o gavião, segundo a sua espécie. E o
bufo\footnote{Ave noturna, estrigídea; corujão.}, e a coruja, e a
gralha, e o cisne, e o pelicano, e o corvo marinho, e
a cegonha, e a garça, segundo a sua espécie, e a
poupa\footnote{Pássaro semelhante à pega.}, e o morcego.
Também todo o inseto que voa, vos será imundo; não se comerá.
Toda a ave limpa comereis. Não comereis nenhum animal
morto; ao estrangeiro, que está dentro das tuas portas, o darás a
comer, ou o venderás ao estranho, porquanto és povo santo ao Senhor
teu Deus. Não cozerás o cabrito com o leite da sua
mãe\footnote{SBTB: ``com leite''. KJ: Thou shalt not seethe a kid in
his mother's milk. RA: Não cozerás o cabrito no leite da sua própria
mãe. RC: Não cozerás o cabrito com o leite da sua mãe.}.

Certamente darás os dízimos de todo o fruto da tua semente, que
cada ano se recolher do campo. E, perante o Senhor teu Deus,
no lugar que escolher para ali fazer habitar o seu nome, comerás os
dízimos do teu grão, do teu mosto e do teu azeite, e os primogênitos
das tuas vacas e das tuas ovelhas; para que aprendas a temer ao
Senhor teu Deus todos os dias. E quando o caminho te for tão
comprido que os não possas levar, por estar longe de ti o lugar que
escolher o Senhor teu Deus para ali pôr o seu nome, quando o Senhor
teu Deus te tiver abençoado; então vende-os, e ata o dinheiro
na tua mão, e vai ao lugar que escolher o Senhor teu Deus; e
aquele dinheiro darás por tudo o que deseja a tua alma, por vacas, e
por ovelhas, e por vinho, e por bebida forte, e por tudo o que te
pedir a tua alma; come-o ali perante o Senhor teu Deus, e alegra-te,
tu e a tua casa; porém não desampararás o levita que está
dentro das tuas portas; pois não tem parte nem herança contigo.
Ao fim de três anos tirarás todos os dízimos da tua colheita
no mesmo ano, e os recolherás dentro das tuas portas; então
virá o levita (pois nem parte nem herança tem contigo), e o
estrangeiro, e o órfão, e a viúva, que estão dentro das tuas portas,
e comerão, e fartar-se-ão; para que o Senhor teu Deus te abençoe em
toda a obra que as tuas mãos fizerem.

\medskip

\lettrine{15} Ao fim dos sete anos farás remissão. Este,
pois, é o modo da remissão: todo o credor remitirá o que emprestou
ao seu próximo; não o exigirá do seu próximo ou do seu irmão, pois a
remissão do Senhor é apregoada. Do estrangeiro o exigirás; mas o
que tiveres em poder de teu irmão a tua mão o remitirá. Exceto
quando não houver entre ti pobre algum; pois o Senhor abundantemente
te abençoará na terra que o Senhor teu Deus te dará por herança,
para possuí-la. Se somente ouvires diligentemente a voz do
Senhor teu Deus para cuidares em cumprir todos estes mandamentos que
hoje te ordeno; porque o Senhor teu Deus te abençoará, como te
tem falado; assim, emprestarás a muitas nações, mas não tomarás
empréstimos; e dominarás sobre muitas nações, mas elas não dominarão
sobre ti. Quando entre ti houver algum pobre, de teus irmãos, em
alguma das tuas portas, na terra que o Senhor teu Deus te dá, não
endurecerás o teu coração, nem fecharás a tua mão a teu irmão que
for pobre; antes lhe abrirás de todo a tua mão, e livremente lhe
emprestarás o que lhe falta, quanto baste para a sua necessidade.
Guarda-te, que não haja palavra perversa no teu coração,
dizendo: Vai-se aproximando o sétimo ano, o ano da remissão; e que o
teu olho seja maligno para com teu irmão pobre, e não lhe dês nada;
e que ele clame contra ti ao Senhor, e que haja em ti pecado.
Livremente lhe darás, e que o teu coração não seja maligno,
quando lhe deres; pois por esta causa te abençoará o Senhor teu Deus
em toda a tua obra, e em tudo o que puseres a tua mão. Pois
nunca deixará de haver pobre na terra; pelo que te ordeno, dizendo:
Livremente abrirás a tua mão para o teu irmão, para o teu
necessitado, e para o teu pobre na tua terra.

Quando teu irmão hebreu ou irmã hebréia se vender a ti, seis anos
te servirá, mas no sétimo ano o deixarás ir livre. E, quando
o deixares ir livre, não o despedirás vazio. Liberalmente o
fornecerás do teu rebanho, e da tua eira, e do teu lagar; daquilo
com que o Senhor teu Deus te tiver abençoado lhe darás. E
lembrar-te-ás de que foste servo na terra do Egito, e de que o
Senhor teu Deus te resgatou; portanto hoje te ordeno isso.
Porém se ele te disser: Não sairei de ti; porquanto te amo a
ti, e a tua casa, por estar bem contigo; então tomarás uma
sovela\footnote{Instrumento de ferro ou de aço, em forma de haste
cortante e pontuda, que os sapateiros e correeiros usam para furar o
couro a fim de coser.}, e lhe furarás a orelha à porta, e teu servo
será para sempre; e também assim farás à tua serva. Não seja
duro aos teus olhos, quando despedi-lo liberto de ti; pois seis anos
te serviu em equivalência ao dobro do salário do diarista; assim o
Senhor teu Deus te abençoará em tudo o que fizeres.

Todo o primogênito que nascer das tuas vacas e das tuas ovelhas,
o macho santificarás ao Senhor teu Deus; com o primogênito do teu
boi não trabalharás, nem tosquiarás o primogênito das tuas ovelhas.
Perante o Senhor teu Deus os comerás de ano em ano, no lugar
que o Senhor escolher, tu e a tua casa. Porém, havendo nele
algum defeito, se for coxo, ou cego, ou tiver qualquer defeito, não
o sacrificarás ao Senhor teu Deus. Nas tuas portas o comerás;
o imundo e o limpo o comerão também, como da corça ou do veado.
Somente o seu sangue não comerás; sobre a terra o derramarás
como água.

\medskip

\lettrine{16} Guarda o mês de Abibe, e celebra a páscoa ao
Senhor teu Deus; porque no mês de Abibe o Senhor teu Deus te tirou
do Egito, de noite. Então sacrificarás a páscoa ao Senhor teu
Deus, das ovelhas e das vacas, no lugar que o Senhor escolher para
ali fazer habitar o seu nome. Nela não comerás levedado; sete
dias nela comerás pães ázimos, pão de aflição (porquanto
apressadamente saíste da terra do Egito), para que te lembres do dia
da tua saída da terra do Egito, todos os dias da tua vida.
Levedado não aparecerá contigo por sete dias em todos os teus
termos; também da carne que matares à tarde, no primeiro dia, nada
ficará até à manhã. Não poderás sacrificar a páscoa em nenhuma
das tuas portas que te dá o Senhor teu Deus; senão no lugar que
escolher o Senhor teu Deus, para fazer habitar o seu nome, ali
sacrificarás a páscoa à tarde, ao pôr do sol, ao tempo determinado
da tua saída do Egito. Então a cozerás, e comerás no lugar que
escolher o Senhor teu Deus; depois voltarás pela manhã, e irás às
tuas tendas. Seis dias comerás pães ázimos e no sétimo dia é
solenidade ao Senhor teu Deus; nenhum trabalho farás. Sete
semanas contarás; desde que a foice começar na seara iniciarás a
contar as sete semanas. Depois celebrarás a festa das semanas
ao Senhor teu Deus; o que deres será oferta voluntária da tua mão,
segundo o Senhor teu Deus te houver abençoado. E te alegrarás
perante o Senhor teu Deus, tu, e teu filho, e tua filha, e o teu
servo, e a tua serva, e o levita que está dentro das tuas portas, e
o estrangeiro, e o órfão, e a viúva, que estão no meio de ti, no
lugar que o Senhor teu Deus escolher para ali fazer habitar o seu
nome. E lembrar-te-ás de que foste servo no Egito; e
guardarás estes estatutos, e os cumprirás. A festa dos
tabernáculos celebrarás sete dias, quando tiveres colhido da tua
eira e do teu lagar. E, na tua festa, alegrar-te-ás, tu, e
teu filho, e tua filha, e o teu servo, e a tua serva, e o levita, e
o estrangeiro, e o órfão, e a viúva, que estão dentro das tuas
portas. Sete dias celebrarás a festa ao Senhor teu Deus, no
lugar que o Senhor escolher; porque o Senhor teu Deus te há de
abençoar em toda a tua colheita, e em todo o trabalho das tuas mãos;
por isso certamente te alegrarás. Três vezes no ano todo o
homem entre ti aparecerá perante o Senhor teu Deus, no lugar que
escolher, na festa dos pães ázimos, e na festa das semanas, e na
festa dos tabernáculos; porém não aparecerá vazio perante o Senhor;
cada um, conforme ao dom da sua mão, conforme a bênção do
Senhor teu Deus, que lhe tiver dado.

Juízes e oficiais porás em todas as tuas cidades que o Senhor teu
Deus te der entre as tuas tribos, para que julguem o povo com juízo
de justiça. Não torcerás o juízo, não farás acepção de
pessoas, nem receberás peitas; porquanto a peita cega os olhos dos
sábios, e perverte as palavras dos justos. A justiça, somente
a justiça seguirás; para que vivas, e possuas em herança a terra que
te dará o Senhor teu Deus. Não plantarás nenhuma árvore junto
ao altar do Senhor teu Deus, que fizeres para ti. Nem
levantarás imagem, a qual o Senhor teu Deus odeia.

\medskip

\lettrine{17} Não sacrificarás ao Senhor teu Deus, boi ou gado
miúdo em que haja defeito ou alguma coisa má; pois abominação é ao
Senhor teu Deus. Quando no meio de ti, em alguma das tuas portas
que te dá o Senhor teu Deus, se achar algum homem ou mulher que
fizer mal aos olhos do Senhor teu Deus, transgredindo a sua aliança.
Que se for, e servir a outros deuses, e se encurvar a eles ou ao
sol, ou à lua, ou a todo o exército do céu, o que eu não ordenei,
e te for denunciado, e o ouvires; então bem o inquirirás; e eis
que, sendo verdade, e certo que se fez tal abominação em Israel,
então tirarás o homem ou a mulher que fez este malefício, às
tuas portas, e apedrejarás o tal homem ou mulher, até que morra.
Por boca de duas testemunhas, ou três testemunhas, será morto o
que houver de morrer; por boca de uma só testemunha não morrerá.
As mãos das testemunhas serão primeiro contra ele, para matá-lo;
e depois as mãos de todo o povo; assim tirarás o mal do meio de ti.

Quando alguma coisa te for difícil demais em juízo, entre sangue e
sangue, entre demanda e demanda, entre ferida e ferida, em questões
de litígios nas tuas portas, então te levantarás, e subirás ao lugar
que escolher o Senhor teu Deus; e virás aos sacerdotes levitas,
e ao juiz que houver naqueles dias, e inquirirás, e te anunciarão a
sentença do juízo. E farás conforme ao mandado da palavra que
te anunciarem no lugar que escolher o Senhor; e terás cuidado de
fazer conforme a tudo o que te ensinarem. Conforme ao mandado
da lei que te ensinarem, e conforme ao juízo que te disserem, farás;
da palavra que te anunciarem te não desviarás, nem para a direita
nem para a esquerda. O homem, pois, que se houver
soberbamente, não dando ouvidos ao sacerdote, que está ali para
servir ao Senhor teu Deus, nem ao juiz, esse homem morrerá; e
tirarás o mal de Israel; para que todo o povo o ouça, e tema,
e nunca mais se ensoberbeça.

Quando entrares na terra que te dá o Senhor teu Deus, e a
possuíres, e nela habitares, e disseres: Porei sobre mim um rei,
assim como têm todas as nações que estão em redor de mim;
porás certamente sobre ti como rei aquele que escolher o
Senhor teu Deus; dentre teus irmãos porás rei sobre ti; não poderás
pôr homem estranho sobre ti, que não seja de teus irmãos.
Porém ele não multiplicará para si cavalos, nem fará voltar o
povo ao Egito para multiplicar cavalos; pois o Senhor vos tem dito:
Nunca mais voltareis por este caminho. Tampouco para si
multiplicará mulheres, para que o seu coração não se desvie; nem
prata nem ouro multiplicará muito para si. Será também que,
quando se assentar sobre o trono do seu reino, então escreverá para
si num livro, um traslado desta lei, do original que está diante dos
sacerdotes levitas. E o terá consigo, e nele lerá todos os
dias da sua vida, para que aprenda a temer ao Senhor seu Deus, para
guardar todas as palavras desta lei, e estes estatutos, para
cumpri-los; para que o seu coração não se levante sobre os
seus irmãos, e não se aparte do mandamento, nem para a direita nem
para a esquerda; para que prolongue os seus dias no seu reino, ele e
seus filhos no meio de Israel.

\medskip

\lettrine{18} Os sacerdotes levitas, toda a tribo de Levi, não
terão parte nem herança com Israel; das ofertas queimadas do Senhor
e da sua herança comerão. Por isso não terão herança no meio de
seus irmãos; o Senhor é a sua herança, como lhes tem dito. Este,
pois, será o direito dos sacerdotes, a receber do povo, dos que
oferecerem sacrifício, seja boi ou gado miúdo; que darão ao
sacerdote a espádua e as queixadas e o bucho. Dar-lhe-ás as
primícias do teu grão, do teu mosto e do teu azeite, e as primícias
da tosquia das tuas ovelhas. Porque o Senhor teu Deus o escolheu
de todas as tuas tribos, para que assista e sirva no nome do Senhor,
ele e seus filhos, todos os dias. E, quando chegar um levita de
alguma das tuas portas, de todo o Israel, onde habitar; e vier com
todo o desejo da sua alma ao lugar que o Senhor escolheu; e
servir no nome do Senhor seu Deus, como também todos os seus irmãos,
os levitas, que assistem ali perante o Senhor, igual porção
comerão, além das vendas do seu patrimônio.

Quando entrares na terra que o Senhor teu Deus te der, não
aprenderás a fazer conforme as abominações daquelas nações.
Entre ti não se achará quem faça passar pelo fogo a seu filho
ou a sua filha, nem adivinhador, nem prognosticador, nem agoureiro,
nem feiticeiro; nem encantador, nem quem consulte a um
espírito adivinhador, nem mágico, nem quem consulte os mortos;
pois todo aquele que faz tal coisa é abominação ao Senhor; e
por estas abominações o Senhor teu Deus os lança fora de diante de
ti. Perfeito serás, como o Senhor teu Deus. Porque
estas nações, que hás de possuir, ouvem os prognosticadores e os
adivinhadores; porém a ti o Senhor teu Deus não permitiu tal coisa.

O Senhor teu Deus te levantará um profeta do meio de ti, de teus
irmãos, como eu; a ele ouvireis; conforme a tudo o que
pediste ao Senhor teu Deus em Horebe, no dia da assembléia, dizendo:
Não ouvirei mais a voz do Senhor teu Deus, nem mais verei este
grande fogo, para que não morra. Então o Senhor me disse:
Falaram bem naquilo que disseram. Eis lhes suscitarei um
profeta do meio de seus irmãos, como tu, e porei as minhas palavras
na sua boca, e ele lhes falará tudo o que eu lhe ordenar. E
será que qualquer que não ouvir as minhas palavras, que ele falar em
meu nome, eu o requererei dele. Porém o profeta que tiver a
presunção de falar alguma palavra em meu nome, que eu não lhe tenha
mandado falar, ou o que falar em nome de outros deuses, esse profeta
morrerá. E, se disseres no teu coração: Como conhecerei a
palavra que o Senhor não falou? Quando o profeta falar em
nome do Senhor, e essa palavra não se cumprir, nem suceder assim;
esta é palavra que o Senhor não falou; com soberba a falou aquele
profeta; não tenhas temor dele.

\medskip

\lettrine{19} Quando o Senhor teu Deus desarraigar as nações
cuja terra te dará o Senhor teu Deus, e tu as possuíres, e morares
nas suas cidades e nas suas casas, três cidades separarás, no
meio da terra que te dará o Senhor teu Deus para a possuíres.
Preparar-te-ás o caminho; e os termos da tua terra, que te fará
possuir o Senhor teu Deus, dividirás em três; e isto será para que
todo o homicida se acolha ali. E este é o caso tocante ao
homicida, que se acolher ali, para que viva; aquele que por engano
ferir o seu próximo, a quem não odiava antes; como aquele que
entrar com o seu próximo no bosque, para cortar lenha, e, pondo
força na sua mão com o machado para cortar a árvore, o ferro saltar
do cabo e ferir o seu próximo e este morrer, aquele se acolherá a
uma destas cidades, e viverá; para que o vingador do sangue não
vá após o homicida, quando se enfurecer o seu coração, e o alcançar,
por ser comprido o caminho, e lhe tire a vida; porque não é culpado
de morte, pois o não odiava antes. Portanto te dou ordem,
dizendo: Três cidades separarás. E, se o Senhor teu Deus dilatar
os teus termos, como jurou a teus pais, e te der toda a terra que
disse daria a teus pais quando guardares todos estes
mandamentos, que hoje te ordeno, para cumpri-los, amando ao Senhor
teu Deus e andando nos seus caminhos todos os dias), então
acrescentarás outras três cidades além destas três. Para que
o sangue inocente não se derrame no meio da tua terra, que o Senhor
teu Deus te dá por herança, e haja sangue sobre ti. Mas,
havendo alguém que odeia a seu próximo, e lhe arma ciladas, e se
levanta contra ele, e o fere mortalmente, e se acolhe a alguma
destas cidades, então os anciãos da sua cidade mandarão
buscá-lo; e dali o tirarão, e o entregarão na mão do vingador do
sangue, para que morra. O teu olho não o perdoará; antes
tirarás o sangue inocente de Israel, para que bem te suceda.

Não mudes o limite do teu próximo, que estabeleceram os antigos
na tua herança, que receberás na terra que te dá o Senhor teu Deus
para a possuíres. Uma só testemunha contra alguém não se
levantará por qualquer iniqüidade, ou por qualquer pecado, seja qual
for o pecado que cometeu; pela boca de duas testemunhas, ou pela
boca de três testemunhas, se estabelecerá o fato. Quando se
levantar testemunha falsa contra alguém, para testificar contra ele
acerca de transgressão, então aqueles dois homens, que
tiverem a demanda, se apresentarão perante o Senhor, diante dos
sacerdotes e dos juízes que houver naqueles dias. E os juízes
inquirirão bem; e eis que, sendo a testemunha falsa, que testificou
falsamente contra seu irmão, far-lhe-eis como cuidou fazer a
seu irmão; e assim tirarás o mal do meio de ti. Para que os
que ficarem o ouçam e temam, e nunca mais tornem a fazer tal mal no
meio de ti. O teu olho não perdoará; vida por vida, olho por
olho, dente por dente, mão por mão, pé por pé.

\medskip

\lettrine{20} Quando saíres à peleja contra teus inimigos, e
vires cavalos, e carros, e povo maior em número do que tu, deles não
terás temor; pois o Senhor teu Deus, que te tirou da terra do Egito,
está contigo. E será que, quando vos achegardes à peleja, o
sacerdote se adiantará, e falará ao povo, e dir-lhe-á: Ouvi, ó
Israel, hoje vos achegais à peleja contra os vossos inimigos; não se
amoleça o vosso coração: não temais nem tremais, nem vos
aterrorizeis diante deles, pois o Senhor vosso Deus é o que vai
convosco, a pelejar contra os vossos inimigos, para salvar-vos.
Então os oficiais falarão ao povo, dizendo: Qual é o homem que
edificou casa nova e ainda não a consagrou? Vá, e torne-se à sua
casa para que porventura não morra na peleja e algum outro a
consagre. E qual é o homem que plantou uma vinha e ainda não a
desfrutou? Vá, e torne-se à sua casa, para que porventura não morra
na peleja e algum outro a desfrute. E qual é o homem que está
desposado com alguma mulher e ainda não a recebeu? Vá, e torne-se à
sua casa, para que porventura não morra na peleja e algum outro
homem a receba. E continuarão os oficiais a falar ao povo,
dizendo: Qual é o homem medroso e de coração tímido? Vá, e torne-se
à sua casa, para que o coração de seus irmãos não se derreta como o
seu coração. E será que, quando os oficiais acabarem de falar ao
povo, então designarão os capitães dos exércitos para a dianteira do
povo.

Quando te achegares a alguma cidade para combatê-la,
apregoar-lhe-ás a paz. E será que, se te responder em paz, e
te abrir as portas, todo o povo que se achar nela te será tributário
e te servirá. Porém, se ela não fizer paz contigo, mas antes
te fizer guerra, então a sitiarás. E o Senhor teu Deus a dará
na tua mão; e todo o homem que houver nela passarás ao fio da
espada. Porém, as mulheres, e as crianças, e os animais; e
tudo o que houver na cidade, todo o seu despojo, tomarás para ti; e
comerás o despojo dos teus inimigos, que te deu o Senhor teu Deus.
Assim farás a todas as cidades que estiverem mui longe de ti,
que não forem das cidades destas nações. Porém, das cidades
destas nações, que o Senhor teu Deus te dá em herança, nenhuma coisa
que tem fôlego deixarás com vida. Antes destruí-las-ás
totalmente: aos heteus, e aos amorreus, e aos cananeus, e aos
perizeus, e aos heveus, e aos jebuseus, como te ordenou o Senhor teu
Deus. Para que não vos ensinem a fazer conforme a todas as
suas abominações, que fizeram a seus deuses, e pequeis contra o
Senhor vosso Deus. Quando sitiares uma cidade por muitos
dias, pelejando contra ela para a tomar, não destruirás o seu
arvoredo, colocando nele o machado, porque dele comerás; pois que
não o cortarás (pois o arvoredo do campo é mantimento para o homem),
para empregar no cerco. Mas as árvores que souberes que não
são árvores de alimento, destruí-las-ás e cortá-las-ás; e contra a
cidade que guerrear contra ti edificarás baluartes, até que esta
seja vencida.

\medskip

\lettrine{21} Quando na terra que te der o Senhor teu Deus,
para possuí-la, se achar um morto, caído no campo, sem que se saiba
quem o matou, então sairão os teus anciãos e os teus juízes, e
medirão a distância até as cidades que estiverem em redor do morto;
e, na cidade mais próxima ao morto, os anciãos da mesma cidade
tomarão uma novilha da manada, que não tenha trabalhado nem tenha
puxado com o jugo; e os anciãos daquela cidade trarão a novilha
a um vale áspero, que nunca foi lavrado nem semeado; e ali, naquele
vale, degolarão a novilha; então se achegarão os sacerdotes,
filhos de Levi; pois o Senhor teu Deus os escolheu para o servirem,
e para abençoarem em nome do Senhor; e pela sua palavra se decidirá
toda a demanda e todo o ferimento; e todos os anciãos da mesma
cidade, mais próxima ao morto, lavarão as suas mãos sobre a novilha
degolada no vale; e protestarão, e dirão: As nossas mãos não
derramaram este sangue, e os nossos olhos o não viram. Sê
propício ao teu povo Israel, que tu, ó Senhor, resgataste, e não
ponhas o sangue inocente no meio do teu povo Israel. E aquele sangue
lhes será expiado. Assim tirarás o sangue inocente do meio de
ti; pois farás o que é reto aos olhos do Senhor.

Quando saíres à peleja contra os teus inimigos, e o Senhor teu
Deus os entregar nas tuas mãos, e tu deles levares prisioneiros,
e tu entre os presos vires uma mulher formosa à vista, e a
cobiçares, e a tomares por mulher, então a trarás para a tua
casa; e ela rapará a cabeça e cortará as suas unhas. E
despirá o vestido do seu cativeiro, e se assentará na tua casa, e
chorará a seu pai e a sua mãe um mês inteiro; e depois chegarás a
ela, e tu serás seu marido e ela tua mulher. E será que, se
te não contentares dela, a deixarás ir à sua vontade; mas de modo
algum a venderás por dinheiro, nem a tratarás como escrava, pois a
tens humilhado.

Quando um homem tiver duas mulheres, uma a quem ama e outra a
quem despreza, e a amada e a desprezada lhe derem filhos, e o filho
primogênito for da desprezada, será que, no dia em que fizer
herdar a seus filhos o que tiver, não poderá dar a primogenitura ao
filho da amada, preferindo-o ao filho da desprezada, que é o
primogênito. Mas ao filho da desprezada reconhecerá por
primogênito, dando-lhe dobrada porção de tudo quanto tiver;
porquanto aquele é o princípio da sua força, o direito da
primogenitura é dele.

Quando alguém tiver um filho contumaz e rebelde, que não obedecer
à voz de seu pai e à voz de sua mãe, e, castigando-o eles, lhes não
der ouvidos, Então seu pai e sua mãe pegarão nele, e o
levarão aos anciãos da sua cidade, e à porta do seu lugar; e
dirão aos anciãos da cidade: Este nosso filho é rebelde e contumaz,
não dá ouvidos à nossa voz; é um comilão e um beberrão. Então
todos os homens da sua cidade o apedrejarão, até que morra; e
tirarás o mal do meio de ti, e todo o Israel ouvirá e temerá.
Quando também em alguém houver pecado, digno do juízo de
morte, e for morto, e o pendurares num madeiro, o seu cadáver
não permanecerá no madeiro, mas certamente o enterrarás no mesmo
dia; porquanto o pendurado é maldito de Deus; assim não contaminarás
a tua terra, que o Senhor teu Deus te dá em herança.

\medskip

\lettrine{22} Vendo extraviado o boi ou ovelha de teu irmão,
não te desviarás deles; restituí-los-ás sem falta a teu irmão. E
se teu irmão não estiver perto de ti, ou não o conheceres,
recolhê-los-ás na tua casa, para que fiquem contigo, até que teu
irmão os busque, e tu lhos restituirás. Assim também farás com o
seu jumento, e assim farás com as suas roupas; assim farás também
com toda a coisa perdida, que se perder de teu irmão, e tu a
achares; não te poderás omitir. Se vires o jumento que é de teu
irmão, ou o seu boi, caídos no caminho, não te desviarás deles; sem
falta o ajudarás a levantá-los.

Não haverá traje de homem na mulher, e nem vestirá o homem roupa
de mulher; porque, qualquer que faz isto, abominação é ao Senhor teu
Deus. Quando encontrares pelo caminho um ninho de ave numa
árvore, ou no chão, com passarinhos, ou ovos, e a mãe posta sobre os
passarinhos, ou sobre os ovos, não tomarás a mãe com os filhotes;
deixarás ir livremente a mãe, e os filhotes tomarás para ti;
para que te vá bem e para que prolongues os teus dias. Quando
edificares uma casa nova, farás um parapeito, no eirado, para que
não ponhas culpa de sangue na tua casa, se alguém de algum modo cair
dela. Não semearás a tua vinha com diferentes espécies de
semente, para que não se degenere o fruto da semente que semeares, e
a novidade da vinha. Com boi e com jumento não lavrarás
juntamente. Não te vestirás de diversos estofos de lã e linho
juntamente. Franjas porás nas quatro bordas da tua manta, com
que te cobrires.

Quando um homem tomar mulher e, depois de coabitar com ela, a
desprezar, e lhe imputar coisas escandalosas, e contra ela
divulgar má fama, dizendo: Tomei esta mulher, e me cheguei a ela,
porém não a achei virgem; então o pai da moça e sua mãe
tomarão os sinais da virgindade da moça, e levá-los-ão aos anciãos
da cidade, à porta; e o pai da moça dirá aos anciãos: Eu dei
minha filha por mulher a este homem, porém ele a despreza; e
eis que lhe imputou coisas escandalosas, dizendo: Não achei virgem a
tua filha; porém eis aqui os sinais da virgindade de minha filha. E
estenderão a roupa diante dos anciãos da cidade. Então os
anciãos da mesma cidade tomarão aquele homem, e o castigarão.
E o multarão em cem siclos de prata, e os darão ao pai da
moça; porquanto divulgou má fama sobre uma virgem de Israel. E lhe
será por mulher, em todos os seus dias não a poderá despedir.
Porém se isto for verdadeiro, isto é, que a virgindade não se
achou na moça, então levarão a moça à porta da casa de seu
pai, e os homens da sua cidade a apedrejarão, até que morra; pois
fez loucura em Israel, prostituindo-se na casa de seu pai; assim
tirarás o mal do meio de ti. Quando um homem for achado
deitado com mulher que tenha marido, então ambos morrerão, o homem
que se deitou com a mulher, e a mulher; assim tirarás o mal de
Israel. Quando houver moça virgem, desposada, e um homem a
achar na cidade, e se deitar com ela, então trareis ambos à
porta daquela cidade, e os apedrejareis, até que morram; a moça,
porquanto não gritou na cidade, e o homem, porquanto humilhou a
mulher do seu próximo; assim tirarás o mal do meio de ti. E
se algum homem no campo achar uma moça desposada, e o homem a
forçar, e se deitar com ela, então morrerá só o homem que se deitou
com ela; porém à moça não farás nada. A moça não tem culpa de
morte; porque, como o homem que se levanta contra o seu próximo, e
lhe tira a vida, assim é este caso. Pois a achou no campo; a
moça desposada gritou, e não houve quem a livrasse. Quando um
homem achar uma moça virgem, que não for desposada, e pegar nela, e
se deitar com ela, e forem apanhados, então o homem que se
deitou com ela dará ao pai da moça cinqüenta siclos de prata; e
porquanto a humilhou, lhe será por mulher; não a poderá despedir em
todos os seus dias. Nenhum homem tomará a mulher de seu pai,
nem descobrirá a nudez de seu pai.

\medskip

\lettrine{23} Aquele a quem forem trilhados os testículos, ou
cortado o membro viril, não entrará na congregação do Senhor.
Nenhum bastardo entrará na congregação do Senhor; nem ainda a
sua décima geração entrará na congregação do Senhor. Nenhum
amonita nem moabita entrará na congregação do Senhor; nem ainda a
sua décima geração entrará na congregação do Senhor eternamente.
Porquanto não saíram com pão e água, a receber-vos no caminho,
quando saíeis do Egito; e porquanto alugaram contra ti a Balaão,
filho de Beor, de Petor, de Mesopotâmia, para te amaldiçoar.
Porém o Senhor teu Deus não quis ouvir Balaão; antes o Senhor
teu Deus trocou em bênção a maldição; porquanto o Senhor teu Deus te
amava. Não lhes procurarás nem paz nem bem em todos os teus dias
para sempre. Não abominarás o edomeu, pois é teu irmão; nem
abominarás o egípcio, pois estrangeiro foste na sua terra. Os
filhos que lhes nascerem na terceira geração, cada um deles entrará
na congregação do Senhor.

Quando o exército sair contra os teus inimigos, então te guardarás
de toda a coisa má. Quando entre ti houver alguém que, por
algum acidente noturno, não estiver limpo, sairá do
arraial\footnote{SBTB: \emph{sairá fora do arraial}. Pleonasmo. KJ:
If there be among you any man, that is not clean by reason of
uncleanness that chanceth him by night, then shall he go abroad out
of the camp, he shall not come within the camp. RA: Se houver entre
vós alguém que, por motivo de polução noturna, não esteja limpo,
sairá do acampamento; não permanecerá nele. RC: Quando entre ti
houver alguém que por algum acidente de noite não estiver limpo,
sairá fora do exército; não entrará no meio do exército.}; não
entrará no meio dele. Porém será que, declinando a tarde, se
lavará em água; e, em se pondo o sol, entrará no meio do arraial.
Também terás um lugar fora do arraial, para onde sairás.
E entre as tuas armas terás uma pá; e será que, quando
estiveres assentado, fora, então com ela cavarás e, virando-te,
cobrirás o que defecaste. Porquanto o Senhor teu Deus anda no
meio de teu arraial, para te livrar, e entregar a ti os teus
inimigos; pelo que o teu arraial será santo, para que ele não veja
coisa feia em ti, e se aparte de ti.

Não entregarás a seu senhor o servo que, tendo fugido dele, se
acolher a ti; contigo ficará, no meio de ti, no lugar que
escolher em alguma das tuas portas, onde lhe agradar; não o
oprimirás. Não haverá prostituta dentre as filhas de Israel;
nem haverá sodomita dentre os filhos de Israel. Não trarás o
salário da prostituta nem preço de um sodomita à casa do Senhor teu
Deus por qualquer voto; porque ambos são igualmente abominação ao
Senhor teu Deus. A teu irmão não emprestarás com juros, nem
dinheiro, nem comida, nem qualquer coisa que se empreste com juros.
Ao estranho emprestarás com juros, porém a teu irmão não
emprestarás com juros; para que o Senhor teu Deus te abençoe em tudo
que puseres a tua mão, na terra a qual vais a possuir. Quando
fizeres algum voto ao Senhor teu Deus, não tardarás em cumpri-lo;
porque o Senhor teu Deus certamente o requererá de ti, e em ti
haverá pecado. Porém, abstendo-te de votar, não haverá pecado
em ti. O que saiu dos teus lábios guardarás, e cumprirás, tal
como voluntariamente votaste ao Senhor teu Deus, declarando-o pela
tua boca. Quando entrares na vinha do teu próximo, comerás
uvas conforme ao teu desejo até te fartares, porém não as porás no
teu cesto. Quando entrares na seara do teu próximo, com a tua
mão arrancarás as espigas; porém não porás a foice na seara do teu
próximo.


\medskip

\lettrine{24} Quando um homem tomar uma mulher e se casar com
ela, então será que, se não achar graça em seus olhos, por nela
encontrar coisa indecente, far-lhe-á uma carta de repúdio, e lha
dará na sua mão, e a despedirá da sua casa. Se ela, pois, saindo
da sua casa, for e se casar com outro homem, e este também a
desprezar, e lhe fizer carta de repúdio, e lha der na sua mão, e a
despedir da sua casa, ou se este último homem, que a tomou para si
por mulher, vier a morrer, então seu primeiro marido, que a
despediu, não poderá tornar a tomá-la, para que seja sua mulher,
depois que foi contaminada; pois é abominação perante o Senhor;
assim não farás pecar a terra que o Senhor teu Deus te dá por
herança.

Quando um homem for recém-casado não sairá à guerra, nem se lhe
imporá encargo algum; por um ano inteiro ficará livre na sua casa
para alegrar a mulher, que tomou. Não se tomará em penhor ambas
as mós, nem a mó de cima nem a de baixo; pois se penhoraria assim a
vida. Quando se achar alguém que tiver furtado um dentre os seus
irmãos, dos filhos de Israel, e escravizá-lo, ou vendê-lo, esse
ladrão morrerá, e tirarás o mal do meio de ti. Guarda-te da
praga da lepra, e tenhas grande cuidado de fazer conforme a tudo o
que te ensinarem os sacerdotes levitas; como lhes tenho ordenado,
terás cuidado de o fazer. Lembra-te do que o Senhor teu Deus fez
a Miriã no caminho, quando saíste do Egito. Quando
emprestares alguma coisa ao teu próximo, não entrarás em sua casa,
para lhe tirar o penhor. Fora ficarás; e o homem, a quem
emprestaste, te trará fora o penhor. Porém, se for homem
pobre, não te deitarás com o seu penhor. Em se pondo o sol,
sem falta lhe restituirás o penhor; para que durma na sua roupa, e
te abençoe; e isto te será justiça diante do Senhor teu Deus.

Não oprimirás o diarista pobre e necessitado de teus irmãos, ou
de teus estrangeiros, que está na tua terra e nas tuas portas.
No seu dia lhe pagarás a sua diária, e o sol não se porá
sobre isso; porquanto pobre é, e sua vida depende disso; para que
não clame contra ti ao Senhor, e haja em ti pecado. Os pais
não morrerão pelos filhos, nem os filhos pelos pais; cada um morrerá
pelo seu pecado. Não perverterás o direito do estrangeiro e
do órfão; nem tomarás em penhor a roupa da viúva. Mas
lembrar-te-ás de que foste servo no Egito, e de que o Senhor teu
Deus te livrou dali; pelo que te ordeno que faças isso.
Quando no teu campo colheres a tua colheita, e esqueceres um
molho no campo, não tornarás a tomá-lo; para o estrangeiro, para o
órfão, e para a viúva será; para que o Senhor teu Deus te abençoe em
toda a obra das tuas mãos, quando sacudires a tua oliveira,
não voltarás para colher o fruto dos ramos; para o estrangeiro, para
o órfão, e para a viúva será. Quando vindimares a tua vinha,
não voltarás para a rebuscá-la; para o estrangeiro, para o órfão, e
para a viúva será. E lembrar-te-ás de que foste servo na
terra do Egito; portanto te ordeno que faças isso.

\medskip

\lettrine{25} Quando houver contenda entre alguns, e vierem a
juízo, para que os julguem, ao justo justificarão, e ao injusto
condenarão. E será que, se o injusto merecer, o juíz o fará
deitar-se, para que seja açoitado diante de si; segundo a sua culpa,
será o número de açoites. Quarenta açoites lhe fará dar, não
mais; para que, porventura, se lhe fizer dar mais açoites do que
estes, teu irmão não fique envilecido\footnote{Tornado vil;
aviltado, desonrado. Vendido por preço vil.} aos teus olhos. Não
atarás a boca ao boi, quando trilhar.

Quando irmãos morarem juntos, e um deles morrer, e não tiver
filho, então a mulher do falecido não se casará com homem estranho,
de fora; seu cunhado estará com ela, e a receberá por mulher, e fará
a obrigação de cunhado para com ela. E o primogênito que ela lhe
der será sucessor do nome do seu irmão falecido, para que o seu nome
não se apague em Israel. Porém, se o homem não quiser tomar sua
cunhada, esta subirá à porta dos anciãos, e dirá: Meu cunhado recusa
suscitar a seu irmão nome em Israel; não quer cumprir para comigo o
dever de cunhado. Então os anciãos da sua cidade o chamarão, e
com ele falarão; e, se ele persistir, e disser: Não quero tomá-la;
então sua cunhada se chegará a ele na presença dos anciãos, e
lhe descalçará o sapato do pé, e lhe cuspirá no rosto, e protestará,
e dirá: Assim se fará ao homem que não edificar a casa de seu irmão;
e o seu nome se chamará em Israel: A casa do descalçado.
Quando pelejarem dois homens, um contra o outro, e a mulher
de um chegar para livrar a seu marido da mão do que o fere, e ela
estender a sua mão, e lhe pegar pelas suas vergonhas, então
cortar-lhe-ás a mão; não a poupará o teu olho.

Na tua bolsa não terás pesos diversos, um grande e um pequeno.
Na tua casa não terás dois tipos de efa, um grande e um
pequeno. Peso inteiro e justo terás; efa inteiro e justo
terás; para que se prolonguem os teus dias na terra que te dará o
Senhor teu Deus. Porque abominação é ao Senhor teu Deus todo
aquele que faz isto, todo aquele que fizer injustiça.
Lembra-te do que te fez Amaleque no caminho, quando saías do
Egito; como te saiu ao encontro no caminho, e feriu na tua
retaguarda todos os fracos que iam atrás de ti, estando tu cansado e
afadigado; e não temeu a Deus. Será, pois, que, quando o
Senhor teu Deus te tiver dado repouso de todos os teus inimigos em
redor, na terra que o Senhor teu Deus te dá por herança, para
possuí-la, então apagarás a memória de Amaleque de debaixo do céu;
não te esqueças.

\medskip

\lettrine{26} E será que, quando entrares na terra que o
Senhor teu Deus te der por herança, e a possuíres, e nela habitares,
então tomarás das primícias de todos os frutos do solo, que
recolheres da terra, que te dá o Senhor teu Deus, e as porás num
cesto, e irás ao lugar que escolher o Senhor teu Deus, para ali
fazer habitar o seu nome. E irás ao sacerdote, que houver
naqueles dias, e dir-lhe-ás: Hoje declaro perante o Senhor teu Deus
que entrei na terra que o Senhor jurou a nossos pais dar-nos. E
o sacerdote tomará o cesto da tua mão, e o porá diante do altar do
Senhor teu Deus. Então testificarás perante o Senhor teu Deus, e
dirás: Arameu, prestes a perecer, foi meu pai, e desceu ao Egito, e
ali peregrinou com pouca gente, porém ali cresceu até vir a ser
nação grande, poderosa, e numerosa. Mas os egípcios nos
maltrataram e nos afligiram, e sobre nós impuseram uma dura
servidão. Então clamamos ao Senhor Deus de nossos pais; e o
Senhor ouviu a nossa voz, e atentou para a nossa miséria, e para o
nosso trabalho, e para a nossa opressão. E o Senhor nos tirou do
Egito com mão forte, e com braço estendido, e com grande espanto, e
com sinais, e com milagres; e nos trouxe a este lugar, e nos deu
esta terra, terra que mana leite e mel. E eis que agora eu
trouxe as primícias dos frutos da terra que tu, ó Senhor, me deste.
Então as porás perante o Senhor teu Deus, e te inclinarás perante o
Senhor teu Deus, e te alegrarás por todo o bem que o Senhor
teu Deus te tem dado a ti e à tua casa, tu e o levita, e o
estrangeiro que está no meio de ti.

Quando acabares de separar todos os dízimos da tua colheita no
ano terceiro, que é o ano dos dízimos, então os darás ao levita, ao
estrangeiro, ao órfão e à viúva, para que comam dentro das tuas
portas, e se fartem; e dirás perante o Senhor teu Deus: Tirei
da minha casa as coisas consagradas e as dei também ao levita, e ao
estrangeiro, e ao órfão e à viúva, conforme a todos os teus
mandamentos que me tens ordenado; não transgredi os teus
mandamentos, nem deles me esqueci; delas não comi no meu
luto, nem delas nada tirei quando imundo, nem delas dei para os
mortos; obedeci à voz do Senhor meu Deus; conforme a tudo o que me
ordenaste, tenho feito. Olha desde a tua santa habitação,
desde o céu, e abençoa o teu povo, a Israel, e a terra que nos
deste, como juraste a nossos pais, terra que mana leite e mel.

Neste dia, o Senhor teu Deus te manda cumprir estes estatutos e
juízos; guarda-os pois, e cumpre-os com todo o teu coração e com
toda a tua alma. Hoje declaraste ao Senhor que ele te será
por Deus, e que andarás nos seus caminhos, e guardarás os seus
estatutos, e os seus mandamentos, e os seus juízos, e darás ouvidos
à sua voz. E o Senhor hoje te declarou que tu lhe serás por
seu próprio povo, como te tem dito, e que guardarás todos os seus
mandamentos. Para assim te exaltar sobre todas as nações que
criou, para louvor, e para fama, e para glória, e para que sejas um
povo santo ao Senhor teu Deus, como tem falado.

\medskip

\lettrine{27} E deram ordem, Moisés e os anciãos, ao povo de
Israel, dizendo: Guardai todos estes mandamentos que hoje vos
ordeno; será, pois, que, no dia em que passares o Jordão à terra
que te der o Senhor teu Deus, levantar-te-ás umas pedras grandes, e
as caiarás. E, havendo-o passado, escreverás nelas todas as
palavras desta lei, para entrares na terra que te der o Senhor teu
Deus, terra que mana leite e mel, como te falou o Senhor Deus de
teus pais. Será, pois, que, quando houveres passado o Jordão,
levantareis estas pedras, que hoje vos ordeno, no monte Ebal, e as
caiarás. E ali edificarás um altar ao Senhor teu Deus, um altar
de pedras; não alçarás instrumento de ferro sobre elas. De
pedras brutas edificarás o altar do Senhor teu Deus; e sobre ele
oferecerás holocaustos ao Senhor teu Deus. Também sacrificarás
ofertas pacíficas, e ali comerás perante o Senhor teu Deus, e te
alegrarás. E naquelas pedras escreverás todas as palavras desta
lei, exprimindo-as nitidamente. Falou mais Moisés, juntamente
com os sacerdotes levitas, a todo o Israel, dizendo: Guarda silêncio
e ouve, ó Israel! Hoje vieste a ser povo do Senhor teu Deus.
Portanto obedecerás à voz do Senhor teu Deus, e cumprirás os
seus mandamentos e os seus estatutos que hoje te ordeno.

E Moisés deu ordem naquele dia ao povo, dizendo: Quando
houverdes passado o Jordão, estes estarão sobre o monte Gerizim,
para abençoarem o povo: Simeão, e Levi, e Judá, e Issacar, e José, e
Benjamim; e estes estarão sobre o monte Ebal para amaldiçoar:
Rúben, Gade, e Aser, e Zebulom, Dã e Naftali. E os levitas
testificarão a todo o povo de Israel em alta voz, e dirão:
Maldito o homem que fizer imagem de escultura, ou de
fundição, abominação ao Senhor, obra da mão do artífice, e a puser
em um lugar escondido. E todo o povo, respondendo, dirá: Amém.
Maldito aquele que desprezar a seu pai ou a sua mãe. E todo o
povo dirá: Amém. Maldito aquele que remover os marcos do seu
próximo. E todo o povo dirá: Amém. Maldito aquele que fizer
que o cego erre de caminho. E todo o povo dirá: Amém. Maldito
aquele que perverter o direito do estrangeiro, do órfão e da viúva.
E todo o povo dirá: Amém. Maldito aquele que se deitar com a
mulher de seu pai, porquanto descobriu a nudez de seu pai. E todo o
povo dirá: Amém. Maldito aquele que se deitar com algum
animal. E todo o povo dirá: Amém. Maldito aquele que se
deitar com sua irmã, filha de seu pai, ou filha de sua mãe. E todo o
povo dirá: Amém. Maldito aquele que se deitar com sua sogra.
E todo o povo dirá: Amém. Maldito aquele que ferir ao seu
próximo em oculto. E todo o povo dirá: Amém. Maldito aquele
que aceitar suborno para ferir uma pessoa inocente. E todo o povo
dirá: Amém. Maldito aquele que não confirmar as palavras
desta lei, não as cumprindo. E todo o povo dirá: Amém.

\medskip

\lettrine{28} E será que, se ouvires a voz do Senhor teu Deus,
tendo cuidado de guardar todos os seus mandamentos que eu hoje te
ordeno, o Senhor teu Deus te exaltará sobre todas as nações da
terra. E todas estas bênçãos virão sobre ti e te alcançarão,
quando ouvires a voz do Senhor teu Deus; bendito serás na
cidade, e bendito serás no campo. Bendito o fruto do teu ventre,
e o fruto da tua terra, e o fruto dos teus animais; e as crias das
tuas vacas e das tuas ovelhas. Bendito o teu cesto e a tua
amassadeira. 6 Bendito serás ao entrares, e bendito serás ao saíres.
O Senhor entregará, feridos diante de ti, os teus inimigos, que
se levantarem contra ti; por um caminho sairão contra ti, mas por
sete caminhos fugirão da tua presença. O Senhor mandará que a
bênção esteja contigo nos teus celeiros, e em tudo o que puseres a
tua mão; e te abençoará na terra que te der o Senhor teu Deus. O
Senhor te confirmará para si como povo santo, como te tem jurado,
quando guardares os mandamentos do Senhor teu Deus, e andares nos
seus caminhos. E todos os povos da terra verão que é invocado
sobre ti o nome do Senhor, e terão temor de ti. E o Senhor te
dará abundância de bens no fruto do teu ventre, e no fruto dos teus
animais, e no fruto do teu solo, sobre a terra que o Senhor jurou a
teus pais te dar. O Senhor te abrirá o seu bom tesouro, o
céu, para dar chuva à tua terra no seu tempo, e para abençoar toda a
obra das tuas mãos; e emprestarás a muitas nações, porém tu não
tomarás emprestado. E o Senhor te porá por cabeça, e não por
cauda; e só estarás em cima, e não debaixo, se obedeceres aos
mandamentos do Senhor teu Deus, que hoje te ordeno, para os guardar
e cumprir. E não te desviarás de todas as palavras que hoje
te ordeno, nem para a direita nem para a esquerda, andando após
outros deuses, para os servires.

Será, porém, que, se não deres ouvidos à voz do Senhor teu Deus,
para não cuidares em cumprir todos os seus mandamentos e os seus
estatutos, que hoje te ordeno, então virão sobre ti todas estas
maldições, e te alcançarão: Maldito serás tu na cidade, e
maldito serás no campo. Maldito o teu cesto e a tua
amassadeira. Maldito o fruto do teu ventre, e o fruto da tua
terra, e as crias das tuas vacas, e das tuas ovelhas. Maldito
serás ao entrares, e maldito serás ao saíres. O Senhor
mandará sobre ti a maldição; a confusão e a derrota em tudo em que
puseres a mão para fazer; até que sejas destruído, e até que
repentinamente pereças, por causa da maldade das tuas obras, pelas
quais me deixaste. O Senhor fará pegar em ti a pestilência,
até que te consuma da terra a que passas a possuir. O Senhor
te ferirá com a tísica e com a febre, e com a inflamação, e com o
calor ardente, e com a secura, e com crestamento\footnote{Cresta
(crestar: queimar à superfície, de leve; tostar. Dar a cor de
queimado a; tornar trigueiro ou atrigueirado. Secar, queimar, por
efeito do frio intenso, ou do calor. Fazer vacilar; enfraquecer.
Secar, queimar, por efeito do frio ou do calor; estiolar-se). Efeito
produzido pelo calor do Sol.} e com ferrugem; e te perseguirão até
que pereças. E os teus céus, que estão sobre a cabeça, serão
de bronze; e a terra que está debaixo de ti, será de ferro. O
Senhor dará por chuva sobre a tua terra, pó e poeira; dos céus
descerá sobre ti, até que pereças. O Senhor te fará cair
diante dos teus inimigos; por um caminho sairás contra eles, e por
sete caminhos fugirás de diante deles, e serás espalhado por todos
os reinos da terra. E o teu cadáver servirá de comida a todas
as aves dos céus, e aos animais da terra; e ninguém os espantará.
O Senhor te ferirá com as úlceras do Egito, com tumores, e
com sarna, e com coceira, de que não possas curar-te; o
Senhor te ferirá com loucura, e com cegueira, e com pasmo de
coração; e apalparás ao meio dia, como o cego apalpa na
escuridão, e não prosperarás nos teus caminhos; porém somente serás
oprimido e roubado todos os dias, e não haverá quem te salve.
Desposar-te-ás com uma mulher, porém outro homem dormirá com
ela; edificarás uma casa, porém não morarás nela; plantarás uma
vinha, porém não aproveitarás o seu fruto. O teu boi será
morto aos teus olhos, porém dele não comerás; o teu jumento será
roubado diante de ti, e não voltará a ti; as tuas ovelhas serão
dadas aos teus inimigos, e não haverá quem te salve. Teus
filhos e tuas filhas serão dados a outro povo, os teus olhos o
verão, e por eles desfalecerão todo o dia; porém não haverá poder na
tua mão. O fruto da tua terra e todo o teu trabalho, comerá
um povo que nunca conheceste; e tu serás oprimido e quebrantado
todos os dias. E enlouquecerás com o que vires com os teus
olhos. O Senhor te ferirá com úlceras malignas nos joelhos e
nas pernas, de que não possas sarar, desde a planta do teu pé até ao
alto da cabeça. O Senhor te levará a ti e a teu rei, que
tiveres posto sobre ti, a uma nação que não conheceste, nem tu nem
teus pais; e ali servirás a outros deuses, ao pau e à pedra.
E serás por pasmo, por ditado, e por fábula, entre todos os
povos a que o Senhor te levará. Lançarás muita semente ao
campo; porém colherás pouco, porque o gafanhoto a consumirá.
Plantarás vinhas, e cultivarás; porém não beberás vinho, nem
colherás as uvas; porque o bicho as colherá. Em todos os
termos terás oliveiras; porém não te ungirás com azeite; porque a
azeitona cairá da tua oliveira. Filhos e filhas gerarás;
porém não serão para ti; porque irão em cativeiro. Todo o teu
arvoredo e o fruto da tua terra consumirá a lagarta. O
estrangeiro, que está no meio de ti, se elevará muito sobre ti, e tu
mais baixo descerás; ele te emprestará a ti, porém tu não
emprestarás a ele; ele será por cabeça, e tu serás por cauda.

E todas estas maldições virão sobre ti, e te perseguirão, e te
alcançarão, até que sejas destruído; porquanto não ouviste à voz do
Senhor teu Deus, para guardares os seus mandamentos, e os seus
estatutos, que te tem ordenado; e serão entre ti por sinal e
por maravilha, como também entre a tua descendência para sempre.
Porquanto não serviste ao Senhor teu Deus com alegria e
bondade de coração, pela abundância de tudo. Assim servirás
aos teus inimigos, que o Senhor enviará contra ti, com fome e com
sede, e com nudez, e com falta de tudo; e sobre o teu pescoço porá
um jugo de ferro, até que te tenha destruído. O Senhor
levantará contra ti uma nação de longe, da extremidade da terra, que
voa como a águia, nação cuja língua não entenderás; nação
feroz de rosto, que não respeitará o rosto do velho, nem se apiedará
do moço; e comerá o fruto dos teus animais, e o fruto da tua
terra, até que sejas destruído; e não te deixará grão, mosto, nem
azeite, nem crias das tuas vacas, nem das tuas ovelhas, até que te
haja consumido; e sitiar-te-á em todas as tuas portas, até
que venham a cair os teus altos e fortes muros, em que confiavas em
toda a tua terra; e sitiará em todas as tuas portas, em toda a tua
terra que te tem dado o Senhor teu Deus. E comerás o fruto do
teu ventre, a carne de teus filhos e de tuas filhas, que te der o
Senhor teu Deus, no cerco e no aperto com que os teus inimigos te
apertarão. Quanto ao homem mais mimoso e delicado no meio de
ti, o seu olho será maligno para com o seu irmão, e para com a
mulher do seu regaço, e para com os demais de seus filhos que ainda
lhe ficarem; de sorte que não dará a nenhum deles da carne de
seus filhos, que ele comer; porquanto nada lhe ficou de resto no
cerco e no aperto, com que o teu inimigo te apertará em todas as
tuas portas. E quanto à mulher mais mimosa e delicada no meio
de ti, que de mimo e delicadeza nunca tentou pôr a planta de seu pé
sobre a terra, será maligno o seu olho contra o homem de seu regaço,
e contra seu filho, e contra sua filha; e isto por causa de
suas páreas, que saírem dentre os seus pés, e para com os seus
filhos que tiver, porque os comerá às escondidas pela falta de tudo,
no cerco e no aperto, com que o teu inimigo te apertará nas tuas
portas. Se não tiveres cuidado de guardar todas as palavras
desta lei, que estão escritas neste livro, para temeres este nome
glorioso e temível, o Senhor teu deus, então o Senhor fará
espantosas as tuas pragas, e as pragas de tua descendência, grandes
e permanentes pragas, e enfermidades malignas e duradouras; e
fará tornar sobre ti todos os males do Egito, de que tu tiveste
temor, e se apegarão a ti. Também o Senhor fará vir sobre ti
toda a enfermidade e toda a praga, que não está escrita no livro
desta lei, até que sejas destruído. E ficareis poucos em
número, em lugar de haverem sido como as estrelas dos céus em
multidão; porquanto não destes ouvidos à voz do Senhor teu Deus.
E será que, assim como o Senhor se deleitava em vós, em
fazer-vos bem e multiplicar-vos, assim o Senhor se deleitará em
destruir-vos e consumir-vos; e desarraigados sereis da terra a qual
passais a possuir. E o Senhor vos espalhará entre todos os
povos, desde uma extremidade da terra até à outra; e ali servireis a
outros deuses que não conheceste, nem tu nem teus pais; ao pau e à
pedra. E nem ainda entre estas nações descansarás, nem a
planta de teu pé terá repouso; porquanto o Senhor ali te dará
coração agitado, e desfalecimento de olhos, e desmaio da alma.
E a tua vida, como em suspenso, estará diante de ti; e
estremecerás de noite e de dia, e não crerás na tua própria vida.
Pela manhã dirás: Ah! Quem me dera ver a noite! E à tarde
dirás: Ah! Quem me dera ver a manhã! Pelo pasmo de teu coração, que
sentirás, e pelo que verás com os teus olhos. E o Senhor te
fará voltar ao Egito em navios, pelo caminho de que te tenho dito;
nunca jamais o verás; e ali sereis vendidos como escravos e escravas
aos vossos inimigos; mas não haverá quem vos compre.

\medskip

\lettrine{29} Estas são as palavras da aliança que o Senhor
ordenou a Moisés que fizesse com os filhos de Israel, na terra de
Moabe, além da aliança que fizera com eles em Horebe. E chamou
Moisés a todo o Israel, e disse-lhes: Tendes visto tudo quanto o
Senhor fez perante vossos olhos, na terra do Egito, a Faraó, e a
todos os seus servos, e a toda a sua terra; as grandes provas
que os teus olhos têm visto, aqueles sinais e grandes maravilhas;
porém não vos tem dado o Senhor um coração para entender, nem
olhos para ver, nem ouvidos para ouvir, até ao dia de hoje. E
quarenta anos vos fiz andar pelo deserto; não se envelheceram sobre
vós as vossas vestes, e nem se envelheceu o vosso sapato no vosso
pé. Pão não comestes, e vinho e bebida forte não bebestes; para
que soubésseis que eu sou o Senhor vosso Deus. Vindo vós, pois,
a este lugar, Siom, rei de Hesbom, e Ogue, rei de Basã, nos saíram
ao encontro, à peleja, e nós os ferimos; e tomamos a sua terra e
a demos por herança aos rubenitas, e aos gaditas, e à meia tribo dos
manassitas. Guardai, pois, as palavras desta aliança, e
cumpri-as, para que prospereis em tudo quanto fizerdes.

Vós todos estais hoje perante o Senhor vosso Deus; os capitães de
vossas tribos, vossos anciãos, e os vossos oficiais, todos os homens
de Israel; os vossos meninos, as vossas mulheres, e o
estrangeiro que está no meio do vosso arraial; desde o rachador da
vossa lenha até ao tirador da vossa água; para entrardes na
aliança do Senhor teu Deus, e no seu juramento que o Senhor teu Deus
hoje faz convosco; para que hoje te confirme por seu povo, e
ele te seja por Deus, como te tem dito, e como jurou a teus pais,
Abraão, Isaque e Jacó. E não somente convosco faço esta
aliança e este juramento; mas com aquele que hoje está aqui
em pé conosco perante o Senhor nosso Deus, e com aquele que hoje não
está aqui conosco. Porque vós sabeis como habitamos na terra
do Egito, e como passamos pelo meio das nações pelas quais
passastes; e vistes as suas abominações, e os seus ídolos, o
pau e a pedra, a prata e o ouro que havia entre eles, para
que entre vós não haja homem, nem mulher, nem família, nem tribo,
cujo coração hoje se desvie do Senhor nosso Deus, para que vá servir
aos deuses destas nações; para que entre vós não haja raiz que dê
veneno e fel; e aconteça que, alguém ouvindo as palavras
desta maldição, se abençoe no seu coração, dizendo: Terei paz, ainda
que ande conforme o parecer do meu coração; para acrescentar à sede
a bebedeira. O Senhor não lhe quererá perdoar; mas fumegará a
ira do Senhor e o seu zelo contra esse homem, e toda a maldição
escrita neste livro pousará sobre ele; e o Senhor apagará o seu nome
de debaixo do céu. E o Senhor o separará para mal, de todas
as tribos de Israel, conforme a todas as maldições da aliança
escrita no livro desta lei. Então dirá à geração vindoura, os
vossos filhos, que se levantarem depois de vós, e o estrangeiro que
virá de terras remotas, vendo as pragas desta terra, e as suas
doenças, com que o Senhor a terá afligido; e toda a sua terra
abrasada com enxofre, e sal, de sorte que não será semeada, e nada
produzirá, nem nela crescerá erva alguma; assim como foi a
destruição de Sodoma e de Gomorra, de Admá e de Zeboim, que o Senhor
destruiu na sua ira e no seu furor. E todas as nações dirão:
Por que fez o Senhor assim com esta terra? Qual foi a causa do furor
desta tão grande ira? Então se dirá: Porquanto deixaram a
aliança do Senhor Deus de seus pais, que com eles tinha feito,
quando os tirou do Egito; e foram, e serviram a outros
deuses, e se inclinaram diante deles; deuses que eles não
conheceram, e nenhum dos quais lhes tinha sido dado. Por isso
a ira do Senhor se acendeu contra esta terra, para trazer sobre ela
toda a maldição que está escrita neste livro. E o Senhor os
arrancou da sua terra com ira, e com indignação, e com grande furor,
e os lançou em outra terra como neste dia se vê. As coisas
encobertas pertencem ao Senhor nosso Deus, porém as reveladas nos
pertencem a nós e a nossos filhos para sempre, para que cumpramos
todas as palavras desta lei.

\medskip

\lettrine{30} E será que, sobrevindo-te todas estas coisas, a
bênção ou a maldição, que tenho posto diante de ti, e te recordares
delas entre todas as nações, para onde te lançar o Senhor teu Deus,
e te converteres ao Senhor teu Deus, e deres ouvidos à sua voz,
conforme a tudo o que eu te ordeno hoje, tu e teus filhos, com todo
o teu coração, e com toda a tua alma, então o Senhor teu Deus te
fará voltar do teu cativeiro, e se compadecerá de ti, e tornará a
ajuntar-te dentre todas as nações entre as quais te espalhou o
Senhor teu Deus. Ainda que os teus desterrados estejam na
extremidade do céu, desde ali te ajuntará o Senhor teu Deus, e te
tomará dali; e o Senhor teu Deus te trará à terra que teus pais
possuíram, e a possuirás; e te fará bem, e te multiplicará mais do
que a teus pais. E o Senhor teu Deus circuncidará o teu coração,
e o coração de tua descendência, para amares ao Senhor teu Deus com
todo o coração, e com toda a tua alma, para que vivas. E o
Senhor teu Deus porá todas estas maldições sobre os teus inimigos, e
sobre os teus odiadores, que te perseguiram. Converter-te-ás,
pois, e darás ouvidos à voz do Senhor; cumprirás todos os seus
mandamentos que hoje te ordeno. E o Senhor teu Deus te fará
prosperar em toda a obra das tuas mãos, no fruto do teu ventre, e no
fruto dos teus animais, e no fruto da tua terra para o teu bem;
porquanto o Senhor tornará a alegrar-se em ti para te fazer bem,
como se alegrou em teus pais, quando deres ouvidos à voz do
Senhor teu Deus, guardando os seus mandamentos e os seus estatutos,
escritos neste livro da lei, quando te converteres ao Senhor teu
Deus com todo o teu coração, e com toda a tua alma.

Porque este mandamento, que hoje te ordeno, não te é encoberto, e
tampouco está longe de ti. Não está nos céus, para dizeres:
Quem subirá por nós aos céus, que no-lo traga, e no-lo faça ouvir,
para que o cumpramos? Nem tampouco está além do mar, para
dizeres: Quem passará por nós além do mar, para que no-lo traga, e
no-lo faça ouvir, para que o cumpramos? Porque esta palavra
está mui perto de ti, na tua boca, e no teu coração, para a
cumprires.

Vês aqui, hoje te tenho proposto a vida e o bem, e a morte e o
mal; porquanto te ordeno hoje que ames ao Senhor teu Deus,
que andes nos seus caminhos, e que guardes os seus mandamentos, e os
seus estatutos, e os seus juízos, para que vivas, e te multipliques,
e o Senhor teu Deus te abençoe na terra a qual entras a possuir.
Porém se o teu coração se desviar, e não quiseres dar
ouvidos, e fores seduzido para te inclinares a outros deuses, e os
servires, então eu vos declaro hoje que, certamente,
perecereis; não prolongareis os dias na terra a que vais, passando o
Jordão, para que, entrando nela, a possuas; os céus e a terra
tomo hoje por testemunhas contra vós, de que te tenho proposto a
vida e a morte, a bênção e a maldição; escolhe pois a vida, para que
vivas, tu e a tua descendência, amando ao Senhor teu Deus,
dando ouvidos à sua voz, e achegando-te a ele; pois ele é a tua
vida, e o prolongamento dos teus dias; para que fiques na terra que
o Senhor jurou a teus pais, a Abraão, a Isaque, e a Jacó, que lhes
havia de dar.

\medskip

\lettrine{31} Depois foi Moisés, e falou estas palavras a todo
o Israel, e disse-lhes: Da idade de cento e vinte anos sou eu
hoje; já não poderei mais sair e entrar; além disto o Senhor me
disse: Não passarás o Jordão. O Senhor teu Deus passará adiante
de ti; ele destruirá estas nações de diante de ti, para que as
possuas; Josué passará adiante de ti, como o Senhor tem falado.
E o Senhor lhes fará como fez a Siom e a Ogue, reis dos
amorreus, e à sua terra, os quais destruiu. Quando, pois, o
Senhor vo-los der diante de vós, então com eles fareis conforme a
todo o mandamento que vos tenho ordenado. Esforçai-vos, e
animai-vos; não temais, nem vos espanteis diante deles; porque o
Senhor teu Deus é o que vai contigo; não te deixará nem te
desamparará. E chamou Moisés a Josué, e lhe disse aos olhos de
todo o Israel: Esforça-te e anima-te; porque com este povo entrarás
na terra que o Senhor jurou a teus pais lhes dar; e tu os farás
herdá-la. O Senhor, pois, é aquele que vai adiante de ti; ele
será contigo, não te deixará, nem te desamparará; não temas, nem te
espantes.

E Moisés escreveu esta lei, e a deu aos sacerdotes, filhos de
Levi, que levavam a arca da aliança do Senhor, e a todos os anciãos
de Israel. E ordenou-lhes Moisés, dizendo: Ao fim de cada
sete anos, no tempo determinado do ano da remissão, na festa dos
tabernáculos, quando todo o Israel vier a comparecer perante
o Senhor teu Deus, no lugar que ele escolher, lerás esta lei diante
de todo o Israel aos seus ouvidos. Ajunta o povo, os homens e
as mulheres, os meninos e os estrangeiros que estão dentro das tuas
portas, para que ouçam e aprendam e temam ao Senhor vosso Deus, e
tenham cuidado de fazer todas as palavras desta lei; e que
seus filhos, que não a souberem, ouçam e aprendam a temer ao Senhor
vosso Deus, todos os dias que viverdes sobre a terra a qual ides,
passando o Jordão, para a possuir.

E disse o Senhor a Moisés: Eis que os teus dias são chegados,
para que morras; chama a Josué, e apresentai-vos na tenda da
congregação, para que eu lhe dê ordens. Assim foram Moisés e Josué,
e se apresentaram na tenda da congregação. Então o Senhor
apareceu na tenda, na coluna de nuvem; e a coluna de nuvem estava
sobre a porta da tenda. E disse o Senhor a Moisés: Eis que
dormirás com teus pais; e este povo se levantará, e prostituir-se-á
indo após os deuses estranhos na terra, para cujo meio vai, e me
deixará, e anulará a minha aliança que tenho feito com ele.
Assim se acenderá a minha ira naquele dia contra ele, e
desampará-lo-ei, e esconderei o meu rosto dele, para que seja
devorado; e tantos males e angústias o alcançarão, que dirá naquele
dia: Não me alcançaram estes males, porque o meu Deus não está no
meio de mim? Esconderei, pois, totalmente o meu rosto naquele
dia, por todo o mal que tiver feito, por se haverem tornado a outros
deuses. Agora, pois, escrevei-vos este cântico, e ensinai-o
aos filhos de Israel; ponde-o na sua boca, para que este cântico me
seja por testemunha contra os filhos de Israel. Porque
introduzirei o meu povo na terra que jurei a seus pais, que mana
leite e mel; e comerá, e se fartará, e se engordará; então se
tornará a outros deuses, e os servirá, e me irritarão, e anularão a
minha aliança. E será que, quando o alcançarem muitos males e
angústias, então este cântico responderá contra ele por testemunha,
pois não será esquecido da boca de sua descendência; porquanto
conheço a sua boa imaginação, o que ele faz hoje, antes que o
introduza na terra que tenho jurado.

Assim Moisés escreveu este cântico naquele dia, e o ensinou aos
filhos de Israel. E ordenou a Josué, filho de Num, e disse:
Esforça-te e anima-te; porque tu introduzirás os filhos de Israel na
terra que lhes jurei; e eu serei contigo. E aconteceu que,
acabando Moisés de escrever num livro, todas as palavras desta lei,
deu ordem aos levitas, que levavam a arca da aliança do
Senhor, dizendo: Tomai este livro da lei, e ponde-o ao lado
da arca da aliança do Senhor vosso Deus, para que ali esteja por
testemunha contra ti. Porque conheço a tua rebelião e a tua
dura cerviz; eis que, vivendo eu ainda hoje convosco, rebeldes
fostes contra o Senhor; e quanto mais depois da minha morte?
Ajuntai perante mim todos os anciãos das vossas tribos, e
vossos oficiais, e aos seus ouvidos falarei estas palavras, e contra
eles por testemunhas tomarei o céu e a terra. Porque eu sei
que depois da minha morte certamente vos corrompereis, e vos
desviareis do caminho que vos ordenei; então este mal vos alcançará
nos últimos dias, quando fizerdes mal aos olhos do Senhor, para o
provocar à ira com a obra das vossas mãos. Então Moisés falou
as palavras deste cântico aos ouvidos de toda a congregação de
Israel, até se acabarem.

\medskip

\lettrine{32} Inclinai os ouvidos, ó céus, e falarei; e ouça a
terra as palavras da minha boca. Goteje a minha doutrina como a
chuva, destile a minha palavra como o orvalho, como chuvisco sobre a
erva e como gotas de água sobre a relva. Porque apregoarei o
nome do Senhor; engrandecei a nosso Deus. Ele é a Rocha, cuja
obra é perfeita, porque todos os seus caminhos justos são; Deus é a
verdade, e não há nele injustiça; justo e reto é. Corromperam-se
contra ele; não são seus filhos, mas a sua mancha; geração perversa
e distorcida é. Recompensais assim ao Senhor, povo louco e
ignorante? Não é ele teu pai que te adquiriu, te fez e te
estabeleceu?

Lembra-te dos dias da antiguidade, atenta para os anos de muitas
gerações: pergunta a teu pai, e ele te informará; aos teus anciãos,
e eles te dirão. Quando o Altíssimo distribuía as heranças às
nações, quando dividia os filhos de Adão uns dos outros, estabeleceu
os termos dos povos, conforme o número dos filhos de Israel.
Porque a porção do Senhor é o seu povo; Jacó é a parte da sua
herança. Achou-o numa terra deserta, e num ermo solitário
cheio de uivos; cercou-o, instruiu-o, e guardou-o como a menina do
seu olho. Como a águia desperta a sua ninhada, move-se sobre
os seus filhos, estende as suas asas, toma-os, e os leva sobre as
suas asas, assim só o Senhor o guiou; e não havia com ele
deus estranho. Ele o fez cavalgar sobre as alturas da terra,
e comer os frutos do campo, e o fez chupar mel da rocha e azeite da
dura pederneira. Manteiga de vacas, e leite de ovelhas, com a
gordura dos cordeiros e dos carneiros que pastam em Basã, e dos
bodes, com o mais escolhido trigo; e bebeste o sangue das uvas, o
vinho puro.

E, engordando-se Jesurum, deu coices (engordaste-te,
engrossaste-te, e de gordura te cobriste) e deixou a Deus, que o
fez, e desprezou a Rocha da sua salvação. Com deuses
estranhos o provocaram a zelos; com abominações o irritaram.
Sacrifícios ofereceram aos demônios, não a Deus; aos deuses
que não conheceram, novos deuses que vieram há pouco, aos quais não
temeram vossos pais. Esqueceste-te da Rocha que te gerou; e
em esquecimento puseste o Deus que te formou.

O que vendo o Senhor, os desprezou, por ter sido provocado à ira
contra seus filhos e suas filhas; e disse: Esconderei o meu
rosto deles, verei qual será o seu fim; porque são geração perversa,
filhos em quem não há lealdade. A zelos me provocaram com
aquilo que não é Deus; com as suas vaidades me provocaram à ira:
portanto eu os provocarei a zelos com o que não é povo; com nação
louca os despertarei à ira. Porque um fogo se acendeu na
minha ira, e arderá até ao mais profundo do inferno, e consumirá a
terra com a sua colheita, e abrasará os fundamentos dos montes.
Males amontoarei sobre eles; as minhas setas esgotarei contra
eles. Consumidos serão de fome, comidos pela febre ardente e
de peste amarga; e contra eles enviarei dentes de feras, com ardente
veneno de serpentes do pó. Por fora devastará a espada, e por
dentro o pavor; ao jovem, juntamente com a virgem, assim à criança
de peito como ao homem encanecido\footnote{Velho, idoso.}.

Eu disse: Por todos os cantos os espalharei; farei cessar a sua
memória dentre os homens, se eu não receasse a ira do
inimigo, para que os seus adversários não se iludam, e para que não
digam: A nossa mão está exaltada; o Senhor não fez tudo isto.
Porque são gente falta de conselhos, e neles não há
entendimento. Quem dera eles fossem sábios! Que isto
entendessem, e atentassem para o seu fim! Como poderia ser
que um só perseguisse mil, e dois fizessem fugir dez mil, se a sua
Rocha os não vendera, e o Senhor os não entregara? Porque a
sua rocha não é como a nossa Rocha, sendo até os nossos inimigos
juízes disto. Porque a sua vinha é a vinha de Sodoma e dos
campos de Gomorra; as suas uvas são uvas venenosas, cachos amargos
têm. O seu vinho é ardente veneno de serpentes, e peçonha
cruel de víboras. Não está isto guardado comigo? Selado nos
meus tesouros? Minha é a vingança e a recompensa, ao tempo
que resvalar o seu pé; porque o dia da sua ruína está próximo, e as
coisas que lhes hão de suceder, se apressam a chegar. Porque
o Senhor fará justiça ao seu povo, e se compadecerá de seus servos;
quando vir que o poder deles se foi, e não há preso nem desamparado.
Então dirá: Onde estão os seus deuses? A rocha em quem
confiavam, de cujos sacrifícios comiam a gordura, e de cujas
libações bebiam o vinho? Levantem-se, e vos ajudem, para que haja
para vós esconderijo.

Vede agora que eu, eu o sou, e mais nenhum deus há além de mim;
eu mato, e eu faço viver; eu firo, e eu saro, e ninguém há que
escape da minha mão. Porque levantarei a minha mão aos céus,
e direi: Eu vivo para sempre. Se eu afiar a minha espada
reluzente, e se a minha mão travar o juízo, retribuirei a vingança
sobre os meus adversários, e recompensarei aos que me odeiam.
Embriagarei as minhas setas de sangue, e a minha espada
comerá carne; do sangue dos mortos e dos prisioneiros, desde a
cabeça, haverá vinganças do inimigo. Jubilai, ó nações, o seu
povo, porque ele vingará o sangue dos seus servos, e sobre os seus
adversários retribuirá a vingança, e terá misericórdia da sua terra
e do seu povo.

E veio Moisés, e falou todas as palavras deste cântico aos
ouvidos do povo, ele e Josué, filho do Num. E, acabando
Moisés de falar todas estas palavras a todo o Israel,
disse-lhes: Aplicai o vosso coração a todas as palavras que
hoje testifico entre vós, para que as recomendeis a vossos filhos,
para que tenham cuidado de cumprir todas as palavras desta lei.
Porque esta palavra não vos é vã, antes é a vossa vida; e por
esta mesma palavra prolongareis os dias na terra a qual, passando o
Jordão, ides a possuir. Depois falou o Senhor a Moisés,
naquele mesmo dia, dizendo: Sobe ao monte de Abarim, ao monte
Nebo, que está na terra de Moabe, defronte de Jericó, e vê a terra
de Canaã, que darei aos filhos de Israel por possessão. E
morre no monte ao qual subirás; e recolhe-te ao teu povo, como Arão
teu irmão morreu no monte Hor, e se recolheu ao seu povo.
Porquanto transgredistes contra mim no meio dos filhos de
Israel, às águas de Meribá de Cades, no deserto de Zim; pois não me
santificastes no meio dos filhos de Israel. Pelo que verás a
terra diante de ti, porém não entrarás nela, na terra que darei aos
filhos de Israel.

\medskip

\lettrine{33} Esta, porém, é a bênção com que Moisés, homem de
Deus, abençoou os filhos de Israel antes da sua morte. Disse
pois: O Senhor veio de Sinai, e lhes subiu de Seir; resplandeceu
desde o monte Parã, e veio com dez milhares de santos; à sua direita
havia para eles o fogo da lei. Na verdade ama os povos; todos os
seus santos estão na sua mão; postos serão no meio, entre os teus
pés, e cada um receberá das tuas palavras. Moisés nos deu a lei,
como herança da congregação de Jacó. E foi rei em Jesurum,
quando se congregaram os cabeças do povo com as tribos de Israel.

Viva Rúben, e não morra, e que os seus homens não sejam poucos.
E isto é o que disse de Judá: Ouve, ó Senhor, a voz de Judá, e
introduze-o no seu povo; as suas mãos lhe bastem, e tu lhe sejas em
ajuda contra os seus inimigos.

E de Levi disse: Teu Tumim e teu Urim são para o teu amado, que tu
provaste em Massá, com quem contendeste junto às águas de Meribá.
Aquele que disse a seu pai, e à sua mãe: Nunca os vi; e não
conheceu a seus irmãos, e não estimou a seus filhos; pois guardaram
a tua palavra e observaram a tua aliança. Ensinaram os teus
juízos a Jacó, e a tua lei a Israel; puseram incenso no teu nariz, e
o holocausto sobre o teu altar. Abençoa o seu poder, ó
Senhor, e aceita a obra das suas mãos; fere os lombos dos que se
levantam contra ele e o odeiam, para que nunca mais se levantem.

E de Benjamim disse: O amado do Senhor habitará seguro com ele;
todo o dia o cobrirá, e morará entre os seus ombros. E de
José disse: Bendita do Senhor seja a sua terra, com o mais excelente
dos céus, com o orvalho e com o abismo que jaz abaixo. E com
os mais excelentes frutos do sol, e com as mais excelentes produções
das luas, e com o mais excelente dos montes antigos, e com o
mais excelente dos outeiros eternos. E com o mais excelente
da terra, e da sua plenitude, e com a benevolência daquele que
habitava na sarça, venha sobre a cabeça de José, e sobre o alto da
cabeça daquele que foi separado de seus irmãos. Ele tem a
glória do primogênito do seu touro, e os seus chifres são chifres de
boi selvagem; com eles rechaçará todos os povos até às extremidades
da terra; estes pois são os dez milhares de Efraim, e estes são os
milhares de Manassés.

E de Zebulom disse: Zebulom, alegra-te nas tuas saídas; e tu,
Issacar, nas tuas tendas. Eles chamarão os povos ao monte;
ali apresentarão ofertas de justiça, porque chuparão a abundância
dos mares e os tesouros escondidos da areia. E de Gade disse:
Bendito aquele que faz dilatar a Gade; habita como a leoa, e
despedaça o braço e o alto da cabeça. E se proveu da melhor
parte, porquanto ali estava escondida a porção do legislador; por
isso veio com os chefes do povo, executou a justiça do Senhor e os
seus juízos para com Israel.

E de Dã disse: Dã é cria de leão; que salta de Basã. E de
Naftali disse: Farta-te, ó Naftali, da benevolência, e enche-te da
bênção do Senhor; possui o ocidente e o sul. E de Aser disse:
Bendito seja Aser com seus filhos; agrade a seus irmãos, e banhe em
azeite o seu pé. Seja de ferro e de metal o teu calçado; e a
tua força seja como os teus dias.

Não há outro, ó Jesurum, semelhante a Deus, que cavalga sobre os
céus para a tua ajuda, e com a sua majestade sobre as mais altas
nuvens. O Deus eterno é a tua habitação, e por baixo estão os
braços eternos; e ele lançará o inimigo de diante de ti, e dirá:
Destrói-o. Israel, pois, habitará só, seguro, na terra da
fonte de Jacó, na terra de grão e de mosto; e os seus céus gotejarão
orvalho. Bem-aventurado tu, ó Israel! Quem é como tu? Um povo
salvo pelo Senhor, o escudo do teu socorro, e a espada da tua
majestade; por isso os teus inimigos te serão sujeitos, e tu pisarás
sobre as suas alturas.

\medskip

\lettrine{34} Então subiu Moisés das campinas de Moabe ao
monte Nebo, ao cume de Pisga, que está em frente a Jericó e o Senhor
mostrou-lhe toda a terra desde Gileade até Dã; e todo Naftali, e
a terra de Efraim, e Manassés e toda a terra de Judá, até ao mar
ocidental; e o sul, e a campina do vale de Jericó, a cidade das
palmeiras, até Zoar. E disse-lhe o Senhor: Esta é a terra que
jurei a Abraão, Isaque, e Jacó, dizendo: À tua descendência a darei;
eu te faço vê-la com os teus olhos, porém lá não passarás.

Assim morreu ali Moisés, servo do Senhor, na terra de Moabe,
conforme a palavra do Senhor. E o sepultou num vale, na terra de
Moabe, em frente de Bete-Peor; e ninguém soube até hoje o lugar da
sua sepultura. Era Moisés da idade de cento e vinte anos quando
morreu; os seus olhos nunca se escureceram, nem perdeu o seu vigor.
E os filhos de Israel prantearam a Moisés trinta dias, nas
campinas de Moabe; e os dias do pranto no luto de Moisés se
cumpriram.

E Josué, filho de Num, foi cheio do espírito de sabedoria,
porquanto Moisés tinha posto sobre ele as suas mãos; assim os filhos
de Israel lhe deram ouvidos, e fizeram como o Senhor ordenara a
Moisés. E nunca mais se levantou em Israel profeta algum como
Moisés, a quem o Senhor conhecera face a face; nem semelhante
em todos os sinais e maravilhas, que o Senhor o enviou para fazer na
terra do Egito, a Faraó, e a todos os seus servos, e a toda a sua
terra. E em toda a mão forte, e em todo o grande espanto, que
praticou Moisés aos olhos de todo o Israel.

