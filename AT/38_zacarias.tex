\addchap{Zacarias}

\lettrine{1} No oitavo mês do segundo ano de Dario veio a
palavra do Senhor ao profeta Zacarias, filho de Baraquias, filho de
Ido, dizendo: O Senhor se irou fortemente contra vossos pais.
Portanto dize-lhes: Assim diz o Senhor dos Exércitos: Tornai-vos
para mim, diz o Senhor dos Exércitos, e eu me tornarei para vós, diz
o Senhor dos Exércitos. E não sejais como vossos pais, aos quais
clamavam os primeiros profetas, dizendo: Assim diz o Senhor dos
Exércitos: Convertei-vos agora dos vossos maus caminhos e das vossas
más obras; mas não ouviram, nem me escutaram, diz o Senhor.
Vossos pais, onde estão? E os profetas, viverão eles para
sempre? Contudo as minhas palavras e os meus estatutos, que eu
ordenei aos profetas, meus servos, não alcançaram a vossos pais? E
eles voltaram, e disseram: Assim como o Senhor dos Exércitos fez
tenção de nos tratar, segundo os nossos caminhos, e segundo as
nossas obras, assim ele nos tratou.

Aos vinte e quatro dias do mês undécimo (que é o mês de Sebate),
no segundo ano de Dario, veio o palavra do Senhor ao profeta
Zacarias, filho de Baraquias, filho de Ido, dizendo: Olhei de
noite, e vi um homem montado num cavalo vermelho; e ele estava
parado entre as murtas que estavam na baixada; e atrás dele estavam
cavalos vermelhos, malhados e brancos. E eu disse: Senhor meu,
quem são estes? E disse-me o anjo que falava comigo: Eu te mostrarei
quem são estes. Então respondeu o homem que estava entre as
murtas, e disse: Estes são os que o Senhor tem enviado para
percorrerem a terra. E eles responderam ao anjo do Senhor,
que estava entre as murtas, e disseram: Nós já percorremos a terra,
e eis que toda a terra está tranqüila e quieta. Então o anjo
do Senhor respondeu, e disse: O Senhor dos Exércitos, até quando não
terás compaixão de Jerusalém, e das cidades de Judá, contra as quais
estiveste irado estes setenta anos? E respondeu o Senhor ao
anjo, que falava comigo, com palavras boas, palavras consoladoras.
E o anjo que falava comigo disse-me: Clama, dizendo: Assim
diz o Senhor dos Exércitos: Com grande zelo estou zelando por
Jerusalém e por Sião. E com grande indignação estou irado
contra os gentios em descanso; porque eu estava pouco indignado, mas
eles agravaram o mal. Portanto, assim diz o Senhor: Voltei-me
para Jerusalém com misericórdia; nela será edificada a minha casa,
diz o Senhor dos Exércitos, e o cordel será estendido sobre
Jerusalém: Clama outra vez, dizendo: Assim diz o Senhor dos
Exércitos: As minhas cidades ainda aumentarão e prosperarão; porque
o Senhor ainda consolará a Sião e ainda escolherá a Jerusalém.

E levantei os meus olhos, e vi, e eis quatro chifres. E eu
disse ao anjo que falava comigo: Que são estes? E ele me disse:
Estes são os chifres que dispersaram a Judá, a Israel e a Jerusalém.
E o Senhor me mostrou quatro carpinteiros. Então eu
disse: Que vêm estes fazer? E ele falou, dizendo: Estes são os
chifres que dispersaram a Judá, de maneira que ninguém pôde levantar
a sua cabeça; estes, pois, vieram para os amedrontarem, para
derrubarem os chifres dos gentios que levantaram o seu poder contra
a terra de Judá, para a espalharem.

\medskip

\lettrine{2} Tornei a levantar os meus olhos, e vi, e eis um
homem que tinha na mão um cordel de medir. E eu disse: Para onde
vais tu? E ele me disse: Vou medir Jerusalém, para ver qual é a sua
largura e qual o seu comprimento. E eis que saiu o anjo que
falava comigo, e outro anjo lhe saiu ao encontro. E disse-lhe:
Corre, fala a este jovem, dizendo: Jerusalém será habitada como as
aldeias sem muros, por causa da multidão dos homens e dos animais
que haverá nela. Pois eu, diz o Senhor, serei para ela um muro
de fogo em redor, e para glória estarei no meio dela.

Ah, ah! Fugi agora da terra do norte, diz o Senhor, porque vos
espalhei pelos quatro ventos do céu, diz o Senhor. Ah! Sião!
Escapa, tu, que habitas com a filha de Babilônia. Porque assim
diz o Senhor dos Exércitos: Depois da glória ele me enviou às nações
que vos despojaram; porque aquele que tocar em vós toca na menina do
seu olho. Porque eis aí levantarei a minha mão sobre eles, e
eles virão a ser a presa daqueles que os serviram; assim sabereis
vós que o Senhor dos Exércitos me enviou.

Exulta, e alegra-te ó filha de Sião, porque eis que venho, e
habitarei no meio de ti, diz o Senhor. E naquele dia muitas
nações se ajuntarão ao Senhor, e serão o meu povo, e habitarei no
meio de ti e saberás que o Senhor dos Exércitos me enviou a ti.
Então o Senhor herdará a Judá como sua porção na terra santa,
e ainda escolherá a Jerusalém. Cala-te, toda a carne, diante
do Senhor, porque ele se levantou da sua santa morada.

\medskip

\lettrine{3} E ele mostrou-me o sumo sacerdote Josué, o qual
estava diante do anjo do Senhor, e Satanás estava à sua mão direita,
para se lhe opor. Mas o Senhor disse a Satanás: O Senhor te
repreenda, ó Satanás, sim, o Senhor, que escolheu Jerusalém, te
repreenda; não é este um tição tirado do fogo? Josué, vestido de
vestes sujas, estava diante do anjo. Então respondeu, aos que
estavam diante dele, dizendo: Tirai-lhe estas vestes sujas. E a
Josué disse: Eis que tenho feito com que passe de ti a tua
iniqüidade, e te vestirei de vestes finas. E disse eu:
Ponham-lhe uma mitra limpa sobre a sua cabeça. E puseram uma mitra
limpa sobre a sua cabeça, e vestiram-no das roupas; e o anjo do
Senhor estava em pé. E o anjo do Senhor protestou a Josué,
dizendo: Assim diz o Senhor dos Exércitos: Se andares nos meus
caminhos, e se observares a minha ordenança, também tu julgarás a
minha casa, e também guardarás os meus átrios, e te darei livre
acesso entre os que estão aqui.

Ouve, pois, Josué, sumo sacerdote, tu e os teus companheiros que
se assentam diante de ti, porque são homens portentosos; eis que eu
farei vir o meu servo, o RENOVO. Porque eis aqui a pedra que pus
diante de Josué; sobre esta pedra única estão sete olhos; eis que eu
esculpirei a sua escultura, diz o Senhor dos Exércitos, e tirarei a
iniqüidade desta terra num só dia. Naquele dia, diz o Senhor
dos Exércitos, cada um de vós convidará o seu próximo para debaixo
da videira e para debaixo da figueira.

\medskip

\lettrine{4} E o anjo que falava comigo voltou, e
despertou-me, como a um homem que é despertado do seu sono, e
disse-me: Que vês? E eu disse: Olho, e eis que vejo um castiçal todo
de ouro, e um vaso de azeite no seu topo, com as suas sete lâmpadas;
e sete canudos, um para cada uma das lâmpadas que estão no seu topo.
E, por cima dele, duas oliveiras, uma à direita do vaso de
azeite, e outra à sua esquerda. E respondi, dizendo ao anjo que
falava comigo: Senhor meu, que é isto? Então respondeu o anjo
que falava comigo, dizendo-me: Não sabes tu o que é isto? E eu
disse: Não, senhor meu. E respondeu-me, dizendo: Esta é a
palavra do Senhor a Zorobabel, dizendo: Não por força nem por
violência, mas sim pelo meu Espírito, diz o Senhor dos Exércitos.
Quem és tu, ó grande monte? Diante de Zorobabel tornar-te-ás uma
campina; porque ele trará a pedra angular com aclamações: Graça,
graça a ela. E a palavra do Senhor veio novamente a mim,
dizendo: As mãos de Zorobabel têm lançado os alicerces desta
casa; também as suas mãos a acabarão, para que saibais que o Senhor
dos Exércitos me enviou a vós. Porque, quem despreza o dia
das coisas pequenas? Pois esses sete se alegrarão, vendo o prumo na
mão de Zorobabel; esses são os sete olhos do Senhor, que percorrem
por toda a terra.

Respondi mais, dizendo-lhe: Que são as duas oliveiras à direita e
à esquerda do castiçal? E, respondendo-lhe outra vez, disse:
Que são aqueles dois ramos de oliveira, que estão junto aos dois
tubos de ouro, e que vertem de si azeite dourado? E ele me
falou, dizendo: Não sabes tu o que é isto? E eu disse: Não, senhor
meu. Então ele disse: Estes são os dois ungidos, que estão
diante do Senhor de toda a terra.

\medskip

\lettrine{5} E outra vez levantei os meus olhos, e vi, e eis
um rolo volante. E disse-me o anjo: Que vês? E eu disse: Vejo um
rolo volante, que tem vinte côvados de comprido e dez côvados de
largo. Então disse-me: Esta é a maldição que sairá pela face de
toda a terra; porque qualquer que furtar, será desarraigado,
conforme está estabelecido de um lado do rolo; como também qualquer
que jurar falsamente, será desarraigado, conforme está estabelecido
do outro lado do rolo. Eu a farei sair, disse o Senhor dos
Exércitos, e ela entrará na casa do ladrão, e na casa do que jurar
falsamente pelo meu nome; e permanecerá no meio da sua casa, e a
consumirá juntamente com a sua madeira e com as suas pedras.

E saiu o anjo, que falava comigo, e disse-me: Levanta agora os
teus olhos, e vê que é isto que sai. E eu disse: Que é isto? E
ele disse: Isto é um efa que sai. Disse ainda: Este é o aspecto
deles em toda a terra. E eis que foi levantado um talento de
chumbo, e uma mulher estava assentada no meio do efa. E ele
disse: Esta é a impiedade. E a lançou dentro do efa; e lançou sobre
a boca deste o peso de chumbo. E levantei os meus olhos, e vi, e
eis que saíram duas mulheres; e traziam vento nas suas asas, pois
tinham asas como as da cegonha; e levantaram o efa entre a terra e o
céu. Então eu disse ao anjo que falava comigo: Para onde
levam elas o efa? E ele me disse: Para lhe edificarem uma
casa na terra de Sinar; e, estando ela acabada, ele será posto ali
na sua base.

\medskip

\lettrine{6} E outra vez levantei os meus olhos, e vi, e eis
que quatro carros saiam dentre dois montes, e estes montes eram
montes de bronze. No primeiro carro eram cavalos vermelhos, e no
segundo carro, cavalos pretos, e no terceiro carro, cavalos
brancos, e no quarto carro, cavalos malhados, todos eram fortes.
E respondi, dizendo ao anjo que falava comigo: Que é isto,
senhor meu? E o anjo respondeu, dizendo-me: Estes são os quatro
espíritos do céu, saindo donde estavam perante o Senhor de toda a
terra. O carro em que estão os cavalos pretos, sai para a terra
do norte, e os brancos saem atrás deles, e os malhados saem para a
terra do sul. E os cavalos fortes saíam, e procuravam ir por
diante, para percorrerem a terra. E ele disse: Ide, percorrei a
terra. E percorreram a terra. E chamou-me, e falou-me, dizendo:
Eis que aqueles que saíram para a terra do norte fizeram repousar o
meu Espírito na terra do norte.

E a palavra do Senhor veio a mim, dizendo: Toma dos que
foram levados cativos, a saber, de Heldai, de Tobias e de Jedaías,
os quais vieram de Babilônia, e vem tu no mesmo dia, e entra na casa
de Josias, filho de Sofonias. Toma, digo, prata e ouro, e
faze coroas, e põe-nas na cabeça do sumo sacerdote Josué, filho de
Jozadaque. E fala-lhe, dizendo: Assim diz o Senhor dos
Exércitos: Eis aqui o homem cujo nome é RENOVO; ele brotará do seu
lugar, e edificará o templo do Senhor. Ele mesmo edificará o
templo do Senhor, e ele levará a glória; assentar-se-á no seu trono
e dominará, e será sacerdote no seu trono, e conselho de paz haverá
entre ambos os ofícios. E estas coroas serão para Helém, e
para Tobias, e para Jedaías, e para Hem, filho de Sofonias, como um
memorial no templo do Senhor. E aqueles que estão longe
virão, e edificarão no templo do Senhor, e vós sabereis que o Senhor
dos Exércitos me tem enviado a vós; e isto sucederá assim, se
diligentemente ouvirdes a voz do Senhor vosso Deus.

\medskip

\lettrine{7} Aconteceu, no quarto ano do rei Dario, que a
palavra do Senhor veio a Zacarias, no quarto dia do nono mês, que é
Quisleu. Quando o povo enviou Sarezer e Régen-Meleque, e os seus
homens, à casa de Deus, para suplicarem o favor do Senhor. E
para dizerem aos sacerdotes, que estavam na casa do Senhor dos
Exércitos, e aos profetas: Chorarei eu no quinto mês, fazendo
abstinência, como tenho feito por tantos anos? Então a palavra
do Senhor dos Exércitos veio a mim, dizendo: Fala a todo o povo
desta terra, e aos sacerdotes, dizendo: Quando jejuastes, e
pranteastes, no quinto e no sétimo mês, durante estes setenta anos,
porventura, foi mesmo para mim que jejuastes? Ou quando
comestes, e quando bebestes, não foi para vós mesmos que comestes e
bebestes? Não foram estas as palavras que o Senhor pregou pelo
ministério dos primeiros profetas, quando Jerusalém estava habitada
e em paz, com as suas cidades ao redor dela, e o sul e a campina
eram habitados?

E a palavra do Senhor veio a Zacarias, dizendo: Assim falou o
Senhor dos Exércitos, dizendo: Executai juízo verdadeiro, mostrai
piedade e misericórdia cada um para com seu irmão. E não
oprimais a viúva, nem o órfão, nem o estrangeiro, nem o pobre, nem
intente cada um, em seu coração, o mal contra o seu irmão.
Eles, porém, não quiseram escutar, e deram-me o ombro
rebelde, e ensurdeceram os seus ouvidos, para que não ouvissem.
Sim, fizeram os seus corações como pedra de diamante, para
que não ouvissem a lei, nem as palavras que o Senhor dos Exércitos
enviara pelo seu Espírito por intermédio dos primeiros profetas; daí
veio a grande ira do Senhor dos Exércitos. E aconteceu que,
assim como ele clamou e eles não ouviram, também eles clamaram, e eu
não ouvi, diz o Senhor dos Exércitos. Assim os espalhei com
um turbilhão por entre todas as nações, que eles não conheceram, e a
terra foi assolada atrás deles, de sorte que ninguém passava por
ela, nem se voltava; porque fizeram da terra desejada uma desolação.

\medskip

\lettrine{8} Depois veio a mim a palavra do Senhor dos
Exércitos, dizendo: Assim diz o Senhor dos Exércitos: Zelei por
Sião com grande zelo, e com grande indignação zelei por ela.
Assim diz o Senhor: Voltarei para Sião, e habitarei no meio de
Jerusalém; e Jerusalém chamar-se-á a cidade da verdade, e o monte do
Senhor dos Exércitos, o monte santo. Assim diz o Senhor dos
Exércitos: Ainda nas praças de Jerusalém habitarão velhos e velhas;
levando cada um, na mão, o seu bordão, por causa da sua muita idade.
E as ruas da cidade se encherão de meninos e meninas, que nelas
brincarão. Assim diz o Senhor dos Exércitos: Se isto for
maravilhoso aos olhos do restante deste povo naqueles dias, será
também maravilhoso aos meus olhos? diz o Senhor dos Exércitos.
Assim diz o Senhor dos Exércitos: Eis que salvarei o meu povo da
terra do oriente e da terra do ocidente; E trá-los-ei, e
habitarão no meio de Jerusalém; e eles serão o meu povo, e eu lhes
serei o seu Deus em verdade e em justiça.

Assim diz o Senhor dos Exércitos: Esforcem-se as vossas mãos, ó
vós que nestes dias ouvistes estas palavras da boca dos profetas,
que estiveram no dia em que foi posto o fundamento da casa do Senhor
dos Exércitos, para que o templo fosse edificado. Porque
antes destes dias não tem havido salário para os homens, nem lhes
davam ganhos os animais; nem havia paz para o que entrava nem para o
que saía, por causa do inimigo, porque eu incitei a todos os homens,
cada um contra o seu próximo. Mas agora não serei para com o
restante deste povo como nos primeiros dias, diz o Senhor dos
Exércitos. Porque haverá semente de prosperidade; a vide dará
o seu fruto, e a terra dará a sua novidade, e os céus darão o seu
orvalho; e farei que o restante deste povo herde tudo isto. E
há de suceder, ó casa de Judá, e casa de Israel, que, assim como
fostes uma maldição entre os gentios, assim vos salvarei, e sereis
uma bênção; não temais, esforcem-se as vossas mãos. Porque
assim diz o Senhor dos Exércitos: Como pensei fazer-vos mal, quando
vossos pais me provocaram à ira, diz o Senhor dos Exércitos, e não
me arrependi, assim tornei a pensar nestes dias fazer o bem a
Jerusalém e à casa de Judá; não temais. Estas são as coisas
que deveis fazer: Falai a verdade cada um com o seu próximo;
executai juízo de verdade e de paz nas vossas portas. E
nenhum de vós pense mal no seu coração contra o seu próximo, nem
ameis o juramento falso; porque todas estas são coisas que eu odeio,
diz o Senhor.

E a palavra do Senhor dos Exércitos veio a mim, dizendo:
Assim diz o Senhor dos Exércitos: O jejum do quarto, e o
jejum do quinto, e o jejum do sétimo, e o jejum do décimo mês será
para a casa de Judá gozo, alegria, e festividades solenes; amai,
pois, a verdade e a paz. Assim diz o Senhor dos Exércitos:
Ainda sucederá que virão os povos e os habitantes de muitas cidades.
E os habitantes de uma cidade irão à outra, dizendo: Vamos
depressa suplicar o favor do Senhor, e buscar o Senhor dos
Exércitos; eu também irei. Assim virão muitos povos e
poderosas nações, a buscar em Jerusalém ao Senhor dos Exércitos, e a
suplicar o favor do Senhor. Assim diz o Senhor dos Exércitos:
Naquele dia sucederá que pegarão dez homens, de todas as línguas das
nações, pegarão, sim, na orla das vestes de um judeu, dizendo:
Iremos convosco, porque temos ouvido que Deus está convosco.

\medskip

\lettrine{9} O peso da palavra do Senhor contra a terra de
Hadraque, e Damasco, o seu repouso; porque o olhar do homem, e de
todas as tribos de Israel, se volta para o Senhor. E também
Hamate que confina com ela, e Tiro e Sidom, ainda que sejam mais
sábias. E Tiro edificou para si fortalezas, e amontoou prata
como o pó, e ouro fino como a lama das ruas. Eis que o Senhor a
despojará e ferirá no mar a sua força, e ela será consumida pelo
fogo. Ascalom o verá e temerá; também Gaza, e terá grande dor;
igualmente Ecrom; porque a sua esperança será confundida; e o rei de
Gaza perecerá, e Ascalom não será habitada. E um bastardo
habitará em Asdode, e exterminarei a soberba dos filisteus. E da
sua boca tirarei o seu sangue, e dentre os seus dentes as suas
abominações; e ele também ficará como um remanescente para o nosso
Deus; e será como governador em Judá, e Ecrom como um jebuseu. E
acampar-me-ei ao redor da minha casa, contra o exército, para que
ninguém passe, nem volte; para que não passe mais sobre eles o
opressor; porque agora vi com os meus olhos.

Alegra-te muito, ó filha de Sião; exulta, ó filha de Jerusalém;
eis que o teu rei virá a ti, justo e salvo\footnote{``\ldots{}justo
e SALVO''\{3467 Yâsha\}; ``\ldots{}justo e TENDO SALVAÇÃO \{3467
Yâsha\}. Mesmo se o hebraico der lugar a ambas as traduções
``salvo'' e ``salvador'', toda a Bíblia aponta a segunda como
melhor. Como nas KJV (``having salvation'') e na ARC 1948
(``salvador''). RC-1969: Alegra-te muito, ó filha de Sião; exulta, ó
filha de Jerusalém: eis que o teu rei virá a ti, justo e Salvador,
pobre, e montado sobre um jumento, sobre um asninho, filho de
jumenta. RA: Alegra-te muito, ó filha de Sião; exulta, ó filha de
Jerusalém: eis aí te vem o teu Rei, justo e salvador, humilde,
montado em jumento, num jumentinho, cria de jumenta.}, pobre, e
montado sobre um jumento, e sobre um jumentinho, filho de jumenta.
E de Efraim destruirei os carros, e de Jerusalém os cavalos;
e o arco de guerra será destruído, e ele anunciará paz aos gentios;
e o seu domínio se estenderá de mar a mar, e desde o rio até às
extremidades da terra. Ainda quanto a ti, por causa do sangue
da tua aliança, libertei os teus presos da cova em que não havia
água.

Voltai à fortaleza, ó presos de esperança; também hoje vos
anuncio que vos restaurarei em dobro. Porque curvei Judá para
mim, enchi com Efraim o arco; suscitarei a teus filhos, ó Sião,
contra os teus filhos, ó Grécia! E pôr-te-ei, ó Sião, como a espada
de um poderoso. E o Senhor será visto sobre eles, e as suas
flechas sairão como o relâmpago; e o Senhor Deus fará soar a
trombeta, e irá com os redemoinhos do sul. O Senhor dos
Exércitos os amparará; eles devorarão, depois que os tiverem
sujeitado, as pedras da funda; também beberão e farão barulho como
excitados pelo vinho; e encher-se-ão como bacias de sacrifício, como
os cantos do altar. E o Senhor seu Deus naquele dia os
salvará, como ao rebanho do seu povo: porque como pedras de uma
coroa eles resplandecerão na sua terra. Porque, quão grande é
a sua bondade! E quão grande é a sua formosura! O trigo fará
florescer os jovens e o mosto as virgens.

\medskip

\lettrine{10} Pedi ao Senhor chuva no tempo da chuva serôdia,
sim, ao Senhor que faz relâmpagos; e lhes dará chuvas abundantes, e
a cada um erva no campo. Porque os ídolos têm falado vaidade, e
os adivinhos têm visto mentira, e contam sonhos falsos; com vaidade
consolam, por isso seguem o seu caminho como ovelhas; estão aflitos,
porque não há pastor. Contra os pastores se acendeu a minha ira,
e castigarei os bodes; mas o Senhor dos Exércitos visitará o seu
rebanho, a casa de Judá, e os fará como o seu majestoso cavalo na
peleja. Dele sairá a pedra de esquina, dele a estaca, dele o
arco de guerra, dele juntamente sairá todo o opressor.

E serão como poderosos que na batalha esmagam ao inimigo no lodo
das ruas; porque o Senhor estará com eles; e confundirão os que
andam montados em cavalos. E fortalecerei a casa de Judá, e
salvarei a casa de José, e fá-los-ei voltar, porque me compadeci
deles; e serão como se eu não os tivera rejeitado, porque eu sou o
Senhor seu Deus, e os ouvirei. E os de Efraim serão como um
poderoso, e o seu coração se alegrará como pelo vinho; e seus filhos
o verão, e se alegrarão; o seu coração se regozijará no Senhor.
Eu lhes assobiarei, e os ajuntarei, porque eu os tenho remido; e
multiplicar-se-ão como antes se tinham multiplicado. Ainda que
os espalhei por entre os povos, eles se lembrarão de mim em lugares
remotos; e viverão com seus filhos, e voltarão. Porque eu os
farei voltar da terra do Egito, e os congregarei da Assíria; e
trá-los-ei à terra de Gileade e do Líbano, e não se achará lugar
bastante para eles. E ele passará pelo mar da angústia, e
ferirá as ondas no mar, e todas as profundezas do Nilo se secarão;
então será derrubada a soberba da Assíria, e o cetro do Egito se
retirará. E eu os fortalecerei no Senhor, e andarão no seu
nome, diz o Senhor.

\medskip

\lettrine{11} Abre, ó Líbano, as tuas portas para que o fogo
consuma os teus cedros. Geme, ó cipreste, porque o cedro caiu,
porque os mais poderosos são destruídos; gemei, ó carvalhos de Basã,
porque o bosque forte é derrubado. Voz de uivo dos pastores!
porque a sua glória é destruída; voz de bramido dos filhos de leões,
porque foi destruída a soberba do Jordão.

Assim diz o Senhor meu Deus: Apascenta as ovelhas da matança,
cujos possuidores as matam, e não se têm por culpados; e cujos
vendedores dizem: Louvado seja o Senhor, porque tenho enriquecido; e
os seus pastores não têm piedade delas. Certamente não terei
mais piedade dos moradores desta terra, diz o Senhor; mas, eis que
entregarei os homens cada um na mão do seu próximo e na mão do seu
rei; eles ferirão a terra, e eu não os livrarei da sua mão. Eu,
pois, apascentei as ovelhas da matança, as pobres ovelhas do
rebanho. Tomei para mim duas varas: a uma chamei Graça, e à outra
chamei União; e apascentei as ovelhas. E destruí os três
pastores num mês; porque a minha alma se impacientou deles, e também
a alma deles se enfastiou de mim. E eu disse: Não vos
apascentarei mais; o que morrer, morra; e o que for destruído, seja
destruído; e as que restarem comam cada uma a carne da outra.
E tomei a minha vara Graça, e a quebrei, para desfazer a
minha aliança, que tinha estabelecido com todos estes povos.
E foi desfeito naquele dia; e assim conheceram os pobres do
rebanho, que me respeitavam, que isto era palavra do Senhor.
Porque eu lhes disse: Se parece bem aos vossos olhos, dai-me
o meu salário e, se não, deixai-o. E pesaram o meu salário, trinta
moedas de prata. O Senhor, pois, disse-me: Arroja isso ao
oleiro, esse belo preço em que fui avaliado por eles. E tomei as
trinta moedas de prata, e as arrojei ao oleiro, na casa do Senhor.
Então quebrei a minha segunda vara União, para romper a
irmandade entre Judá e Israel.

E o Senhor disse-me: Toma ainda para ti o instrumento de um
pastor insensato. Porque, eis que suscitarei um pastor na
terra, que não cuidará das que estão perecendo, não buscará a
pequena, e não curará a ferida, nem apascentará a sã; mas comerá a
carne da gorda, e lhe despedaçará as unhas. Ai do pastor
inútil, que abandona o rebanho! A espada cairá sobre o seu braço e
sobre o seu olho direito; e o seu braço completamente se secará, e o
seu olho direito completamente se escurecerá.

\medskip

\lettrine{12} Peso da palavra do Senhor sobre Israel: Fala o
Senhor, o que estende o céu, e que funda a terra, e que forma o
espírito do homem dentro dele. Eis que eu farei de Jerusalém um
copo de tremor para todos os povos em redor, e também para Judá,
durante o cerco contra Jerusalém. E acontecerá naquele dia que
farei de Jerusalém uma pedra pesada para todos os povos; todos os
que a carregarem certamente serão despedaçados; e ajuntar-se-ão
contra ela todo o povo da terra. Naquele dia, diz o Senhor,
ferirei de espanto a todos os cavalos, e de loucura os que montam
neles; mas sobre a casa de Judá abrirei os meus olhos, e ferirei de
cegueira a todos os cavalos dos povos. Então os governadores de
Judá dirão no seu coração: Os habitantes de Jerusalém são a minha
força no Senhor dos Exércitos, seu Deus. Naquele dia porei os
governadores de Judá como um braseiro ardente no meio da lenha, e
como um facho de fogo entre gavelas; e à direita e à esquerda
consumirão a todos os povos em redor, e Jerusalém será habitada
outra vez no seu lugar, em Jerusalém; e o Senhor salvará
primeiramente as tendas de Judá, para que a glória da casa de Davi e
a glória dos habitantes de Jerusalém não seja exaltada sobre Judá.
Naquele dia o Senhor protegerá os habitantes de Jerusalém; e o
mais fraco dentre eles naquele dia será como Davi, e a casa de Davi
será como Deus, como o anjo do Senhor diante deles.

E acontecerá naquele dia, que procurarei destruir todas as nações
que vierem contra Jerusalém; mas sobre a casa de Davi, e
sobre os habitantes de Jerusalém, derramarei o Espírito de graça e
de súplicas; e olharão para mim, a quem traspassaram; e
pranteá-lo-ão sobre ele, como quem pranteia pelo filho unigênito; e
chorarão amargamente por ele, como se chora amargamente pelo
primogênito. Naquele dia será grande o pranto em Jerusalém,
como o pranto de Hadade-Rimom no vale de Megido. E a terra
pranteará, cada família à parte: a família da casa de Davi à parte,
e suas mulheres à parte; e a família da casa de Natã à parte, e suas
mulheres à parte; a família da casa de Levi à parte, e suas
mulheres à parte; a família de Simei à parte, e suas mulheres à
parte. Todas as mais famílias remanescentes, cada família à
parte, e suas mulheres à parte.

\medskip

\lettrine{13} Naquele dia haverá uma fonte aberta para a casa
de Davi, e para os habitantes de Jerusalém, para purificação do
pecado e da imundícia. E acontecerá naquele dia, diz o Senhor
dos Exércitos, que tirarei da terra os nomes dos ídolos, e deles não
haverá mais memória; e também farei sair da terra os profetas e o
espírito da impureza. E acontecerá que, quando alguém ainda
profetizar, seu pai e sua mãe, que o geraram, lhe dirão: Não
viverás, porque falaste mentira em nome do Senhor; e seu pai e sua
mãe, que o geraram, o traspassarão quando profetizar. E
acontecerá naquele dia que os profetas se envergonharão, cada um da
sua visão, quando profetizarem; nem mais se vestirão de manto de
pelos, para mentirem. Mas dirão: Não sou profeta, sou lavrador
da terra; porque certo homem ensinou-me a guardar o gado desde a
minha mocidade. E se alguém lhe disser: Que feridas são estas
nas tuas mãos? Dirá ele: São feridas com que fui ferido em casa dos
meus amigos.

Ó espada, desperta-te contra o meu pastor, e contra o homem que é
o meu companheiro, diz o Senhor dos Exércitos. Fere ao pastor, e
espalhar-se-ão as ovelhas; mas volverei a minha mão sobre os
pequenos. E acontecerá em toda a terra, diz o Senhor, que as
duas partes dela serão extirpadas, e expirarão; mas a terceira parte
restará nela. E farei passar esta terceira parte pelo fogo, e a
purificarei, como se purifica a prata, e a provarei, como se prova o
ouro. Ela invocará o meu nome, e eu a ouvirei; direi: É meu povo; e
ela dirá: O Senhor é o meu Deus.

\medskip

\lettrine{14} Eis que vem o dia do Senhor, em que teus
despojos se repartirão no meio de ti. Porque eu ajuntarei todas
as nações para a peleja contra Jerusalém; e a cidade será tomada, e
as casas serão saqueadas, e as mulheres forçadas; e metade da cidade
sairá para o cativeiro, mas o restante do povo não será extirpado da
cidade. E o Senhor sairá, e pelejará contra estas nações, como
pelejou, sim, no dia da batalha. E naquele dia estarão os seus
pés sobre o monte das Oliveiras, que está defronte de Jerusalém para
o oriente; e o monte das Oliveiras será fendido pelo meio, para o
oriente e para o ocidente, e haverá um vale muito grande; e metade
do monte se apartará para o norte, e a outra metade dele para o sul.
E fugireis pelo vale dos meus montes, pois o vale dos montes
chegará até Azel; e fugireis assim como fugistes de diante do
terremoto nos dias de Uzias, rei de Judá. Então virá o Senhor meu
Deus, e todos os santos contigo. E acontecerá naquele dia, que
não haverá preciosa luz, nem espessa escuridão. Mas será um dia
conhecido do Senhor; nem dia nem noite será; mas acontecerá que ao
cair da tarde haverá luz.

Naquele dia também acontecerá que sairão de Jerusalém águas vivas,
metade delas para o mar oriental, e metade delas para o mar
ocidental; no verão e no inverno sucederá isto. E o Senhor será
rei sobre toda a terra; naquele dia um será o Senhor, e um será o
seu nome. Toda a terra em redor se tornará em planície, desde
Geba até Rimom, ao sul de Jerusalém, e ela será exaltada, e habitada
no seu lugar, desde a porta de Benjamim até ao lugar da primeira
porta, até à porta da esquina, e desde a torre de Hananeel até aos
lagares do rei. E habitarão nela, e não haverá mais
destruição, porque Jerusalém habitará segura. E esta será a
praga com que o Senhor ferirá a todos os povos que guerrearam contra
Jerusalém: a sua carne apodrecerá, estando eles em pé, e lhes
apodrecerão os olhos nas suas órbitas, e a língua lhes apodrecerá na
sua boca. Naquele dia também acontecerá que haverá da parte
do Senhor uma grande perturbação entre eles; porque cada um pegará
na mão do seu próximo, e cada um levantará a mão contra o seu
próximo. E também Judá pelejará em Jerusalém, e as riquezas
de todos os gentios serão ajuntadas ao redor, ouro e prata e roupas
em grande abundância. Assim será também a praga dos cavalos,
dos mulos, dos camelos e dos jumentos e de todos os animais que
estiverem naqueles arraiais, como foi esta praga.

E acontecerá que, todos os que restarem de todas as nações que
vieram contra Jerusalém, subirão de ano em ano para adorar o Rei, o
Senhor dos Exércitos, e para celebrarem a festa dos tabernáculos.
E acontecerá que, se alguma das famílias da terra não subir a
Jerusalém, para adorar o Rei, o Senhor dos Exércitos, não virá sobre
ela a chuva. E, se a família dos egípcios não subir, nem
vier, não virá sobre ela a chuva; virá sobre eles a praga com que o
Senhor ferirá os gentios que não subirem a celebrar a festa dos
tabernáculos. Este será o castigo do pecado dos egípcios e o
castigo do pecado de todas as nações que não subirem a celebrar a
festa dos tabernáculos. Naquele dia será gravado sobre as
campainhas dos cavalos: SANTIDADE AO SENHOR; e as panelas na casa do
Senhor serão como as bacias diante do altar. E todas as
panelas em Jerusalém e Judá serão consagradas ao Senhor dos
Exércitos, e todos os que sacrificarem virão, e delas tomarão, e
nelas cozerão. E, naquele dia não haverá mais cananeu na casa do
Senhor dos Exércitos.

