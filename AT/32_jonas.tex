\addchap{Jonas}

\lettrine{1} E veio a palavra do Senhor a Jonas, filho de
Amitai, dizendo: Levanta-te, vai à grande cidade de Nínive, e
clama contra ela, porque a sua malícia subiu até à minha presença.
Porém, Jonas se levantou para fugir da presença do Senhor para
Társis. E descendo a Jope, achou um navio que ia para Társis; pagou,
pois, a sua passagem, e desceu para dentro dele, para ir com eles
para Társis, para longe da presença do Senhor.

Mas o Senhor mandou ao mar um grande vento, e fez-se no mar uma
forte tempestade, e o navio estava a ponto de quebrar-se. Então
temeram os marinheiros, e clamavam cada um ao seu deus, e lançaram
ao mar as cargas, que estavam no navio, para o aliviarem do seu
peso; Jonas, porém, desceu ao porão do navio, e, tendo-se deitado,
dormia um profundo sono. E o mestre do navio chegou-se a ele, e
disse-lhe: Que tens, dorminhoco? Levanta-te, clama ao teu Deus;
talvez assim ele se lembre de nós para que não pereçamos. E
diziam cada um ao seu companheiro: Vinde, e lancemos sortes, para
que saibamos por que causa nos sobreveio este mal. E lançaram
sortes, e a sorte caiu sobre Jonas. Então lhe disseram:
Declara-nos tu agora, por causa de quem nos sobreveio este mal. Que
ocupação é a tua? Donde vens? Qual é a tua terra? E de que povo és
tu? E ele lhes disse: Eu sou hebreu, e temo ao Senhor, o Deus do
céu, que fez o mar e a terra seca. Então estes homens se
encheram de grande temor, e disseram-lhe: Por que fizeste tu isto?
Pois sabiam os homens que fugia da presença do Senhor, porque ele
lho tinha declarado.

E disseram-lhe: Que te faremos nós, para que o mar se nos acalme?
Porque o mar ia se tornando cada vez mais tempestuoso. E ele
lhes disse: Levantai-me, e lançai-me ao mar, e o mar se vos
aquietará; porque eu sei que por minha causa vos sobreveio esta
grande tempestade. Entretanto, os homens remavam, para fazer
voltar o navio à terra, mas não podiam, porquanto o mar se ia
embravecendo cada vez mais contra eles. Então clamaram ao
Senhor, e disseram: Ah, Senhor! Nós te rogamos, que não pereçamos
por causa da alma deste homem, e que não ponhas sobre nós o sangue
inocente; porque tu, Senhor, fizeste como te aprouve. E
levantaram a Jonas, e o lançaram ao mar, e cessou o mar da sua
fúria. Temeram, pois, estes homens ao Senhor com grande
temor; e ofereceram sacrifício ao Senhor, e fizeram votos.
Preparou, pois, o Senhor um grande peixe, para que tragasse a
Jonas; e esteve Jonas três dias e três noites nas entranhas do
peixe.

\medskip

\lettrine{2} E orou Jonas ao Senhor, seu Deus, das entranhas
do peixe. E disse: Na minha angústia clamei ao Senhor, e ele me
respondeu; do ventre do inferno gritei, e tu ouviste a minha voz.
Porque tu me lançaste no profundo, no coração dos mares, e a
corrente das águas me cercou; todas as tuas ondas e as tuas vagas
têm passado por cima de mim. E eu disse: Lançado estou de diante
dos teus olhos; todavia tornarei a ver o teu santo templo. As
águas me cercaram até à alma, o abismo me rodeou, e as algas se
enrolaram na minha cabeça. Eu desci até aos fundamentos dos
montes; a terra me encerrou para sempre com os seus ferrolhos; mas
tu fizeste subir a minha vida da perdição, ó Senhor meu Deus.
Quando desfalecia em mim a minha alma, lembrei-me do Senhor; e
entrou a ti a minha oração, no teu santo templo. Os que observam
as falsas vaidades deixam a sua misericórdia. Mas eu te
oferecerei sacrifício com a voz do agradecimento; o que votei
pagarei. Do Senhor vem a salvação.

Falou, pois, o Senhor ao peixe, e este vomitou a Jonas na terra
seca.

\medskip

\lettrine{3} E veio a palavra do Senhor segunda vez a Jonas,
dizendo: Levanta-te, e vai à grande cidade de Nínive, e prega
contra ela a mensagem que eu te digo. E levantou-se Jonas, e foi
a Nínive, segundo a palavra do Senhor. Ora, Nínive era uma cidade
muito grande, de três dias de caminho. E começou Jonas a entrar
pela cidade caminho de um dia, e pregava, dizendo: Ainda quarenta
dias, e Nínive será subvertida.

E os homens de Nínive creram em Deus; e proclamaram um jejum, e
vestiram-se de saco, desde o maior até ao menor. Esta palavra
chegou também ao rei de Nínive; e ele levantou-se do seu trono, e
tirou de si as suas vestes, e cobriu-se de saco, e sentou-se sobre a
cinza. E fez uma proclamação que se divulgou em Nínive, pelo
decreto do rei e dos seus grandes, dizendo: Nem homens, nem animais,
nem bois, nem ovelhas provem coisa alguma, nem se lhes dê alimentos,
nem bebam água; mas os homens e os animais sejam cobertos de
sacos, e clamem fortemente a Deus, e convertam-se, cada um do seu
mau caminho, e da violência que há nas suas mãos. Quem sabe se
se voltará Deus, e se arrependerá, e se apartará do furor da sua
ira, de sorte que não pereçamos? E Deus viu as obras deles,
como se converteram do seu mau caminho; e Deus se arrependeu do mal
que tinha anunciado lhes faria, e não o fez.

\medskip

\lettrine{4} Mas isso desagradou extremamente a Jonas, e ele
ficou irado. E orou ao Senhor, e disse: Ah! Senhor! Não foi esta
minha palavra, estando ainda na minha terra? Por isso é que me
preveni, fugindo para Társis, pois sabia que és Deus compassivo e
misericordioso, longânimo e grande em benignidade, e que te
arrependes do mal. Peço-te, pois, ó Senhor, tira-me a vida,
porque melhor me é morrer do que viver. E disse o Senhor: Fazes
bem que assim te ires?

Então Jonas saiu da cidade, e sentou-se ao oriente dela; e ali fez
uma cabana, e sentou-se debaixo dela, à sombra, até ver o que
aconteceria à cidade. E fez o Senhor Deus nascer uma aboboreira,
e ela subiu por cima de Jonas, para que fizesse sombra sobre a sua
cabeça, a fim de o livrar do seu enfado; e Jonas se alegrou em
extremo por causa da aboboreira. Mas Deus enviou um verme, no
dia seguinte ao subir da alva, o qual feriu a aboboreira, e esta se
secou. E aconteceu que, aparecendo o sol, Deus mandou um vento
calmoso oriental, e o sol feriu a cabeça de Jonas; e ele desmaiou, e
desejou com toda a sua alma morrer, dizendo: Melhor me é morrer do
que viver. Então disse Deus a Jonas: Fazes bem que assim te ires
por causa da aboboreira? E ele disse: Faço bem que me revolte até à
morte. E disse o Senhor: Tiveste tu compaixão da aboboreira,
na qual não trabalhaste, nem a fizeste crescer, que numa noite
nasceu, e numa noite pereceu; e não hei de eu ter compaixão
da grande cidade de Nínive em que estão mais de cento e vinte mil
homens que não sabem discernir entre a sua mão direita e a sua mão
esquerda, e também muitos animais?

