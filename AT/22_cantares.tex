\addchap{Cantares de Salomão}

\lettrine{1} Cântico dos cânticos, que é de Salomão.
\textparagraph 2 Beije-me ele com os beijos da sua boca; porque
melhor é o teu amor do que o vinho. Suave é o aroma dos teus
ungüentos; como o ungüento derramado é o teu nome; por isso as
virgens te amam. Leva-me tu; correremos após ti. O rei me
introduziu nas suas câmaras; em ti nos regozijaremos e nos
alegraremos; do teu amor nos lembraremos, mais do que do vinho; os
retos te amam. Eu sou morena, porém formosa, ó filhas de
Jerusalém, como as tendas de Quedar, como as cortinas de Salomão.
Não olheis para o eu ser morena; porque o sol resplandeceu sobre
mim; os filhos de minha mãe indignaram-se contra mim, puseram-me por
guarda das vinhas; a minha vinha, porém, não guardei.

Dize-me, ó tu, a quem ama a minha alma: Onde apascentas o teu
rebanho, onde o fazes descansar ao meio-dia; pois por que razão
seria eu como a que anda errante junto aos rebanhos de teus
companheiros? Se tu não o sabes, ó mais formosa entre as
mulheres, sai-te pelas pisadas do rebanho, e apascenta as tuas
cabras junto às moradas dos pastores. Às éguas dos carros de
Faraó te comparo, ó meu amor. Formosas são as tuas faces
entre os teus enfeites, o teu pescoço com os colares.
Enfeites de ouro te faremos, com incrustações de prata.

Enquanto o rei está assentado à sua mesa, o meu nardo exala o seu
perfume. O meu amado é para mim como um ramalhete de mirra,
posto entre os meus seios. Como um ramalhete de hena nas
vinhas de En-Gedi é para mim o meu amado. Eis que és formosa,
ó meu amor, eis que és formosa; os teus olhos são como os das
pombas. Eis que és formoso, ó amado meu, e também amável; o
nosso leito é verde. As traves da nossa casa são de cedro, as
nossas varandas de cipreste.

\medskip

\lettrine{2} Eu sou a rosa de Sarom, o lírio dos vales.
Qual o lírio entre os espinhos, tal é meu amor entre as filhas.

Qual a macieira entre as árvores do bosque, tal é o meu amado
entre os filhos; desejo muito a sua sombra, e debaixo dela me
assento; e o seu fruto é doce ao meu paladar. Levou-me à casa do
banquete, e o seu estandarte sobre mim era o amor. Sustentai-me
com passas, confortai-me com maçãs, porque desfaleço de amor. A
sua mão esquerda esteja debaixo da minha cabeça, e a sua mão direita
me abrace. Conjuro-vos, ó filhas de Jerusalém, pelas gazelas e
cervas do campo, que não acordeis nem desperteis o meu amor, até que
queira.

Esta é a voz do meu amado; ei-lo aí, que já vem saltando sobre os
montes, pulando sobre os outeiros. O meu amado é semelhante ao
gamo, ou ao filho do veado; eis que está detrás da nossa parede,
olhando pelas janelas, espreitando pelas grades. O meu amado
fala e me diz: Levanta-te, meu amor, formosa minha, e vem.
Porque eis que passou o inverno; a chuva cessou, e se foi;
aparecem as flores na terra, o tempo de cantar chega, e a voz
da rola ouve-se em nossa terra. A figueira já deu os seus
figos verdes, e as vides em flor exalam o seu aroma; levanta-te, meu
amor, formosa minha, e vem.

Pomba minha, que andas pelas fendas das penhas, no oculto das
ladeiras, mostra-me a tua face, faze-me ouvir a tua voz, porque a
tua voz é doce, e a tua face graciosa. Apanhai-nos as
raposas, as raposinhas, que fazem mal às vinhas, porque as nossas
vinhas estão em flor. O meu amado é meu, e eu sou dele; ele
apascenta o seu rebanho entre os lírios. Até que refresque o
dia, e fujam as sombras, volta, amado meu; faze-te semelhante ao
gamo ou ao filho dos veados sobre os montes de Beter.

\medskip

\lettrine{3} De noite, em minha cama, busquei aquele a quem
ama a minha alma; busquei-o, e não o achei. Levantar-me-ei,
pois, e rodearei a cidade; pelas ruas e pelas praças buscarei aquele
a quem ama a minha alma; busquei-o, e não o achei. Acharam-me os
guardas, que rondavam pela cidade; eu lhes perguntei: Vistes aquele
a quem ama a minha alma? Apartando-me eu um pouco deles, logo
achei aquele a quem ama a minha alma; agarrei-me a ele, e não o
larguei, até que o introduzi em casa de minha mãe, na câmara daquela
que me gerou. Conjuro-vos, ó filhas de Jerusalém, pelas gazelas
e cervas do campo, que não acordeis, nem desperteis o meu amor, até
que queira.

Quem é esta que sobe do deserto, como colunas de fumaça, perfumada
de mirra, de incenso, e de todos os pós dos mercadores?

Eis que é a liteira de Salomão; sessenta valentes estão ao redor
dela, dos valentes de Israel; todos armados de espadas, destros
na guerra; cada um com a sua espada à cinta por causa dos temores
noturnos. O rei Salomão fez para si uma carruagem de madeira do
Líbano. Fez-lhe as colunas de prata, o estrado de ouro, o
assento de púrpura, o interior revestido com amor, pelas filhas de
Jerusalém. Saí, ó filhas de Sião, e contemplai ao rei Salomão
com a coroa com que o coroou sua mãe no dia do seu desposório e no
dia do júbilo do seu coração.

\medskip

\lettrine{4} Eis que és formosa, meu amor, eis que és formosa;
os teus olhos são como os das pombas entre as tuas tranças; o teu
cabelo é como o rebanho de cabras que pastam no monte de Gileade.
Os teus dentes são como o rebanho das ovelhas tosquiadas, que
sobem do lavadouro, e das quais todas produzem gêmeos, e nenhuma há
estéril entre elas. Os teus lábios são como um fio de escarlate,
e o teu falar é agradável; a tua fronte é qual um pedaço de romã
entre os teus cabelos. O teu pescoço é como a torre de Davi,
edificada para pendurar armas; mil escudos pendem dela, todos
broquéis de poderosos. Os teus dois seios são como dois filhos
gêmeos da gazela, que se apascentam entre os lírios. Até que
refresque o dia, e fujam as sombras, irei ao monte da mirra, e ao
outeiro do incenso. Tu és toda formosa, meu amor, e em ti não há
mancha.

Vem comigo do Líbano, ó minha esposa, vem comigo do Líbano; olha
desde o cume de Amana, desde o cume de Senir e de Hermom, desde os
covis dos leões, desde os montes dos leopardos. Enlevaste-me o
coração, minha irmã, minha esposa; enlevaste-me o coração com um dos
teus olhares, com um colar do teu pescoço. Que belos são os
teus amores, minha irmã, esposa minha! Quanto melhor é o teu amor do
que o vinho! E o aroma dos teus ungüentos do que o de todas as
especiarias! Favos de mel manam dos teus lábios, minha
esposa! Mel e leite estão debaixo da tua língua, e o cheiro dos teus
vestidos é como o cheiro do Líbano. Jardim fechado és tu,
minha irmã, esposa minha, manancial fechado, fonte selada. Os
teus renovos são um pomar de romãs, com frutos excelentes, o
cipreste com o nardo. O nardo, e o açafrão, o cálamo, e a
canela, com toda a sorte de árvores de incenso, a mirra e aloés, com
todas as principais especiarias.

És a fonte dos jardins, poço das águas vivas, que correm do
Líbano! Levanta-te, vento norte, e vem tu, vento sul; assopra
no meu jardim, para que destilem os seus aromas. Ah! entre o meu
amado no jardim, e coma os seus frutos excelentes!

\medskip

\lettrine{5} Já entrei no meu jardim, minha irmã, minha
esposa; colhi a minha mirra com a minha especiaria, comi o meu favo
com o meu mel, bebi o meu vinho com o meu leite; comei, amigos,
bebei abundantemente, ó amados.

Eu dormia, mas o meu coração velava; e eis a voz do meu amado que
está batendo: abre-me, minha irmã, meu amor, pomba minha, imaculada
minha, porque a minha cabeça está cheia de orvalho, os meus cabelos
das gotas da noite. Já despi a minha roupa; como as tornarei a
vestir? Já lavei os meus pés; como os tornarei a sujar? O meu
amado pôs a sua mão pela fresta da porta, e as minhas entranhas
estremeceram por amor dele. Eu me levantei para abrir ao meu
amado, e as minhas mãos gotejavam mirra, e os meus dedos mirra com
doce aroma, sobre as aldravas da fechadura. Eu abri ao meu
amado, mas já o meu amado tinha se retirado, e tinha ido; a minha
alma desfaleceu quando ele falou; busquei-o e não o achei, chamei-o
e não me respondeu. Acharam-me os guardas que rondavam pela
cidade; espancaram-me, feriram-me, tiraram-me o manto os guardas dos
muros. Conjuro-vos, ó filhas de Jerusalém, que, se achardes o
meu amado, lhe digais que estou enferma de amor.

Que é o teu amado mais do que outro amado, ó tu, a mais formosa
entre as mulheres? Que é o teu amado mais do que outro amado, que
tanto nos conjuras? O meu amado é branco e rosado; ele é o
primeiro entre dez mil. A sua cabeça é como o ouro mais
apurado, os seus cabelos são crespos, pretos como o corvo. Os
seus olhos são como os das pombas junto às correntes das águas,
lavados em leite, postos em engaste. As suas faces são como
um canteiro de bálsamo, como flores perfumadas; os seus lábios são
como lírios gotejando mirra com doce aroma. As suas mãos são
como anéis de ouro engastados de berilo; o seu ventre como alvo
marfim, coberto de safiras. As suas pernas como colunas de
mármore colocadas sobre bases de ouro puro; o seu aspecto como o
Líbano, excelente como os cedros.
 A sua boca é muitíssimo suave, sim, ele é totalmente desejável.
Tal é o meu amado, e tal o meu amigo, ó filhas de Jerusalém.

\medskip

\lettrine{6} Para onde foi o teu amado, ó mais formosa entre
as mulheres? Para onde se retirou o teu amado, para que o busquemos
contigo? O meu amado desceu ao seu jardim, aos canteiros de
bálsamo, para apascentar nos jardins e para colher os lírios. Eu
sou do meu amado, e o meu amado é meu; ele apascenta entre os
lírios.

Formosa és, meu amor, como Tirza, aprazível como Jerusalém,
terrível como um exército com bandeiras. Desvia de mim os teus
olhos, porque eles me dominam. O teu cabelo é como o rebanho das
cabras que aparecem em Gileade. Os teus dentes são como o
rebanho de ovelhas que sobem do lavadouro, e das quais todas
produzem gêmeos, e não há estéril entre elas. Como um pedaço de
romã, assim são as tuas faces entre os teus cabelos. Sessenta
são as rainhas, e oitenta as concubinas, e as virgens sem número.
Porém uma é a minha pomba, a minha imaculada, a única de sua
mãe, e a mais querida daquela que a deu à luz; viram-na as filhas e
chamaram-na bem-aventurada, as rainhas e as concubinas louvaram-na.
Quem é esta que aparece como a alva do dia, formosa como a
lua, brilhante como o sol, terrível como um exército com bandeiras?

Desci ao jardim das nogueiras, para ver os frutos do vale, a ver
se floresciam as vides e brotavam as romãzeiras. Antes de eu
o sentir, me pôs a minha alma nos carros do meu nobre povo.
Volta, volta, ó Sulamita, volta, volta, para que nós te
vejamos. Por que olhas para a Sulamita como para as fileiras de dois
exércitos?

\medskip

\lettrine{7} Quão formosos são os teus pés nos sapatos, ó
filha do príncipe! Os contornos de tuas coxas são como jóias,
trabalhadas por mãos de artista. O teu umbigo como uma taça
redonda, a que não falta bebida; o teu ventre como montão de trigo,
cercado de lírios. Os teus dois seios como dois filhos gêmeos de
gazela. O teu pescoço como a torre de marfim; os teus olhos como
as piscinas de Hesbom, junto à porta de Bate-Rabim; o teu nariz como
torre do Líbano, que olha para Damasco. A tua cabeça sobre ti é
como o monte Carmelo, e os cabelos da tua cabeça como a púrpura; o
rei está preso nas galerias. Quão formosa, e quão aprazível és,
ó amor em delícias! A tua estatura é semelhante à palmeira; e os
teus seios são semelhantes aos cachos de uvas. Dizia eu: Subirei
à palmeira, pegarei em seus ramos; e então os teus seios serão como
os cachos na vide, e o cheiro da tua respiração como o das maçãs.
E a tua boca como o bom vinho para o meu amado, que se bebe
suavemente, e faz com que falem os lábios dos que dormem.

Eu sou do meu amado, e ele me tem afeição. Vem, ó amado
meu, saiamos ao campo, passemos as noites nas aldeias.
Levantemo-nos de manhã para ir às vinhas, vejamos se
florescem as vides, se já aparecem as tenras uvas, se já brotam as
romãzeiras; ali te darei os meus amores. As mandrágoras
exalam o seu perfume, e às nossas portas há todo o gênero de
excelentes frutos, novos e velhos; ó amado meu, eu os guardei para
ti.

\medskip

\lettrine{8} Ah! quem me dera que foras como meu irmão, que
mamou aos seios de minha mãe! Quando te encontrasse lá fora,
beijar-te-ia, e não me desprezariam! Levar-te-ia e te
introduziria na casa de minha mãe, e tu me ensinarias; eu te daria a
beber do vinho aromático e do mosto das minhas romãs. A sua mão
esquerda esteja debaixo da minha cabeça, e a sua direita me abrace.
Conjuro-vos, ó filhas de Jerusalém, que não acordeis nem
desperteis o meu amor, até que queira.

Quem é esta que sobe do deserto, e vem encostada ao seu amado?
Debaixo da macieira te despertei, ali esteve tua mãe com dores; ali
esteve com dores aquela que te deu à luz. Põe-me como selo sobre
o teu coração, como selo sobre o teu braço, porque o amor é forte
como a morte, e duro como a sepultura o ciúme; as suas brasas são
brasas de fogo, com veementes labaredas. As muitas águas não
podem apagar este amor, nem os rios afogá-lo; ainda que alguém desse
todos os bens de sua casa pelo amor, certamente o desprezariam.

Temos uma irmã pequena, que ainda não tem seios; que faremos a
esta nossa irmã, no dia em que dela se falar? Se ela for um
muro, edificaremos sobre ela um palácio de prata; e, se ela for uma
porta, cercá-la-emos com tábuas de cedro. Eu sou um muro, e
os meus seios são como as suas torres; então eu era aos seus olhos
como aquela que acha paz. Teve Salomão uma vinha em
Baal-Hamom; entregou-a a uns guardas; e cada um lhe trazia pelo seu
fruto mil peças de prata. A minha vinha, que me pertence,
está diante de mim; as mil peças de prata são para ti, ó Salomão, e
duzentas para os que guardam o seu fruto.

Ó tu, que habitas nos jardins, os companheiros estão atentos para
ouvir a tua voz; faze-me, pois, também ouvi-la. Vem depressa,
amado meu, e faze-te semelhante ao gamo ou ao filho dos veados sobre
os montes dos aromas.

