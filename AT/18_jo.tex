\addchap{Jó}

\lettrine{1} Havia um homem na terra de Uz, cujo nome era Jó;
e era este homem íntegro, reto e temente a Deus e desviava-se do
mal. E nasceram-lhe sete filhos e três filhas. E o seu gado
era de sete mil ovelhas, três mil camelos, quinhentas juntas de bois
e quinhentas jumentas; eram também muitíssimos os servos a seu
serviço, de maneira que este homem era maior do que todos os do
oriente.

E iam seus filhos à casa uns dos outros e faziam banquetes cada um
por sua vez; e mandavam convidar as suas três irmãs a comerem e
beberem com eles. Sucedia, pois, que, decorrido o turno de dias
de seus banquetes, enviava Jó, e os santificava, e se levantava de
madrugada, e oferecia holocaustos segundo o número de todos eles;
porque dizia Jó: Talvez pecaram meus filhos, e amaldiçoaram a Deus
no seu coração. Assim fazia Jó continuamente.

E num dia em que os filhos de Deus vieram apresentar-se perante o
Senhor, veio também Satanás entre eles. Então o Senhor disse a
Satanás: Donde vens? E Satanás respondeu ao Senhor, e disse: De
rodear a terra, e passear por ela. E disse o Senhor a Satanás:
Observaste tu a meu servo Jó? Porque ninguém há na terra semelhante
a ele, homem íntegro e reto, temente a Deus, e que se desvia do mal.
Então respondeu Satanás ao Senhor, e disse: Porventura teme Jó a
Deus debalde?\footnote{Inutilmente, em vão; baldadamente, embalde.}
Porventura tu não cercaste de sebe, a ele, e a sua casa, e a
tudo quanto tem? A obra de suas mãos abençoaste e o seu gado se tem
aumentado na terra. Mas estende a tua mão, e toca-lhe em tudo
quanto tem, e verás se não blasfema contra ti na tua face. E
disse o Senhor a Satanás: Eis que tudo quanto ele tem está na tua
mão; somente contra ele não estendas a tua mão. E Satanás saiu da
presença do Senhor.

E sucedeu um dia, em que seus filhos e suas filhas comiam, e
bebiam vinho, na casa de seu irmão primogênito, que veio um
mensageiro a Jó, e lhe disse: Os bois lavravam, e as jumentas
pastavam junto a eles; e deram sobre eles os sabeus, e os
tomaram, e aos servos feriram ao fio da espada; e só eu escapei para
trazer-te a nova. Estando este ainda falando, veio outro e
disse: Fogo de Deus caiu do céu, e queimou as ovelhas e os servos, e
os consumiu, e só eu escapei para trazer-te a nova. Estando
ainda este falando, veio outro, e disse: Ordenando os caldeus três
tropas, deram sobre os camelos, e os tomaram, e aos servos feriram
ao fio da espada; e só eu escapei para trazer-te a nova.
Estando ainda este falando, veio outro, e disse: Estando teus
filhos e tuas filhas comendo e bebendo vinho, em casa de seu irmão
primogênito, eis que um grande vento sobreveio dalém do
deserto, e deu nos quatro cantos da casa, que caiu sobre os jovens,
e morreram; e só eu escapei para trazer-te a nova.

Então Jó se levantou, e rasgou o seu manto, e rapou a sua cabeça,
e se lançou em terra, e adorou. E disse: Nu saí do ventre de
minha mãe e nu tornarei para lá; o Senhor o deu, e o Senhor o tomou:
bendito seja o nome do Senhor. Em tudo isto Jó não pecou, nem
atribuiu a Deus falta alguma.

\medskip

\lettrine{2} E, vindo outro dia, em que os filhos de Deus
vieram apresentar-se perante o Senhor, veio também Satanás entre
eles, apresentar-se perante o Senhor. Então o Senhor disse a
Satanás: Donde vens? E respondeu Satanás ao Senhor, e disse: De
rodear a terra, e passear por ela. E disse o Senhor a Satanás:
Observaste o meu servo Jó? Porque ninguém há na terra semelhante a
ele, homem íntegro e reto, temente a Deus e que se desvia do mal, e
que ainda retém a sua sinceridade, havendo-me tu incitado contra
ele, para o consumir sem causa. Então Satanás respondeu ao
Senhor, e disse: Pele por pele, e tudo quanto o homem tem dará pela
sua vida. Porém estende a tua mão, e toca-lhe nos ossos, e na
carne, e verás se não blasfema contra ti na tua face! E disse o
Senhor a Satanás: Eis que ele está na tua mão; porém guarda a sua
vida.

Então saiu Satanás da presença do Senhor, e feriu a Jó de úlceras
malignas, desde a planta do pé até ao alto da cabeça. E Jó tomou
um caco para se raspar com ele; e estava assentado no meio da cinza.
Então sua mulher lhe disse: Ainda reténs a tua sinceridade?
Amaldiçoa a Deus, e morre. Porém ele lhe disse: Como fala
qualquer doida, falas tu; receberemos o bem de Deus, e não
receberíamos o mal? Em tudo isto não pecou Jó com os seus lábios.

Ouvindo, pois, três amigos de Jó todo este mal que tinha vindo
sobre ele, vieram cada um do seu lugar: Elifaz o temanita, e Bildade
o suíta, e Zofar o naamatita; e combinaram condoer-se dele, para o
consolarem. E, levantando de longe os seus olhos, não o
conheceram; e levantaram a sua voz e choraram, e rasgaram cada um o
seu manto, e sobre as suas cabeças lançaram pó ao ar. E
assentaram-se com ele na terra, sete dias e sete noites; e nenhum
lhe dizia palavra alguma, porque viam que a dor era muito grande.

\medskip

\lettrine{3} Depois disto abriu Jó a sua boca, e amaldiçoou o
seu dia. E Jó, falando, disse: Pereça o dia em que nasci, e
a noite em que se disse: Foi concebido um homem! Converta-se
aquele dia em trevas; e Deus, lá de cima, não tenha cuidado dele,
nem resplandeça sobre ele a luz. Contaminem-no as trevas e a
sombra da morte; habitem sobre ele nuvens; a escuridão do dia o
espante! Quanto àquela noite, dela se apodere a escuridão; e não
se regozije ela entre os dias do ano; e não entre no número dos
meses! Ah! que solitária seja aquela noite, e nela não entre voz
de júbilo! Amaldiçoem-na aqueles que amaldiçoam o dia, que estão
prontos para suscitar o seu pranto. Escureçam-se as estrelas do
seu crepúsculo; que espere a luz, e não venha; e não veja as
pálpebras da alva; porque não fechou as portas do ventre; nem
escondeu dos meus olhos a canseira.

Por que não morri eu desde a madre? E em saindo do ventre, não
expirei? Por que me receberam os joelhos? E por que os
peitos, para que mamasse? Porque já agora jazeria e
repousaria; dormiria, e então haveria repouso para mim. Com
os reis e conselheiros da terra, que para si edificam casas nos
lugares assolados, ou com os príncipes que possuem ouro, que
enchem as suas casas de prata, ou como aborto oculto, não
existiria; como as crianças que não viram a luz. Ali os maus
cessam de perturbar; e ali repousam os cansados. Ali os
presos juntamente repousam, e não ouvem a voz do
exator\footnote{Cobrador ou arrecadador de impostos e contribuições;
coletor. RA: feitor. KJ: oppressor.}. Ali está o pequeno e o
grande, e o servo livre de seu senhor.

Por que se dá luz ao miserável, e vida aos amargurados de ânimo?
Que esperam a morte, e ela não vem; e cavam em procura dela
mais do que de tesouros ocultos; que de alegria saltam, e
exultam, achando a sepultura? Por que se dá luz ao homem,
cujo caminho é oculto, e a quem Deus o encobriu? Porque antes
do meu pão vem o meu suspiro; e os meus gemidos se derramam como
água. Porque aquilo que temia me sobreveio; e o que receava
me aconteceu. Nunca estive tranqüilo, nem sosseguei, nem
repousei, mas veio sobre mim a perturbação.

\medskip

\lettrine{4} Então respondeu Elifaz o temanita, e disse:
Se intentarmos falar-te, enfadar-te-ás? Mas quem poderia conter
as palavras? Eis que ensinaste a muitos, e tens fortalecido as
mãos fracas. As tuas palavras firmaram os que tropeçavam e os
joelhos desfalecentes tens fortalecido. Mas agora, que se trata
de ti, te enfadas; e tocando-te a ti, te perturbas. Porventura
não é o teu temor de Deus a tua confiança, e a tua esperança a
integridade dos teus caminhos?

Lembra-te agora qual é o inocente que jamais pereceu? E onde foram
os sinceros destruídos? Segundo eu tenho visto, os que lavram
iniqüidade, e semeiam mal, segam o mesmo. Com o hálito de Deus
perecem; e com o sopro da sua ira se consomem. O rugido do
leão, e a voz do leão feroz, e os dentes dos leõezinhos se quebram.
Perece o leão velho, porque não tem presa; e os filhos da
leoa andam dispersos.

Uma coisa me foi trazida em segredo; e os meus ouvidos perceberam
um sussurro dela. Entre pensamentos vindos de visões da
noite, quando cai sobre os homens o sono profundo,
sobrevieram-me o espanto e o tremor, e todos os meus ossos
estremeceram. Então um espírito passou por diante de mim;
fez-me arrepiar os cabelos da minha carne. Parou ele, porém
não conheci a sua feição; um vulto estava diante dos meus olhos;
houve silêncio, e ouvi uma voz que dizia: Seria porventura o
homem mais justo do que Deus? Seria porventura o homem mais puro do
que o seu Criador? Eis que ele não confia nos seus servos e
aos seus anjos atribui loucura; quanto menos àqueles que
habitam em casas de lodo, cujo fundamento está no pó, e são
esmagados como a traça! Desde a manhã até à tarde são
despedaçados; e eternamente perecem sem que disso se faça caso.
Porventura não passa com eles a sua excelência? Morrem, mas
sem sabedoria.

\medskip

\lettrine{5} Chama agora; há alguém que te responda? E para
qual dos santos te virarás? Porque a ira destrói o louco; e o
zelo mata o tolo. Bem vi eu o louco lançar raízes; porém logo
amaldiçoei a sua habitação. Seus filhos estão longe da salvação;
e são despedaçados às portas, e não há quem os livre. A sua
messe\footnote{Seara em bom estado de se ceifar. Ceifa, colheita.
Fig. Aquisição, conquista. Fig. Conversão de almas.}, o faminto a
devora, e até dentre os espinhos a tira; e o salteador traga a sua
fazenda.

Porque do pó não procede a aflição, nem da terra brota o trabalho.
Mas o homem nasce para a tribulação, como as faíscas se levantam
para voar. Porém eu buscaria a Deus; e a ele entregaria a minha
causa. Ele faz coisas grandes e inescrutáveis, e maravilhas sem
número. Ele dá a chuva sobre a terra, e envia águas sobre os
campos. Para pôr aos abatidos num lugar alto; e para que os
enlutados se exaltem na salvação. Ele aniquila as imaginações
dos astutos, para que as suas mãos não possam levar coisa alguma a
efeito. Ele apanha os sábios na sua própria astúcia; e o
conselho dos perversos se precipita. Eles de dia encontram as
trevas; e ao meio dia andam às apalpadelas como de noite.
Porém ao necessitado livra da espada, e da boca deles, e da
mão do forte. Assim há esperança para o pobre; e a iniqüidade
tapa a sua boca.

Eis que bem-aventurado é o homem a quem Deus repreende; não
desprezes, pois, a correção do Todo-Poderoso. Porque ele faz
a chaga, e ele mesmo a liga; ele fere, e as suas mãos curam.
Em seis angústias te livrará; e na sétima o mal não te
tocará. Na fome te livrará da morte; e na guerra, da
violência da espada. Do açoite da língua estarás encoberto; e
não temerás a assolação, quando vier. Da assolação e da fome
te rirás, e os animais da terra não temerás. Porque até com
as pedras do campo terás o teu acordo, e as feras do campo serão
pacíficas contigo. E saberás que a tua tenda está em paz; e
visitarás a tua habitação, e não pecarás. Também saberás que
se multiplicará a tua descendência e a tua posteridade como a erva
da terra. Na velhice irás à sepultura, como se recolhe o
feixe de trigo a seu tempo. Eis que isto já o havemos
inquirido, e assim é; ouve-o, e medita nisso para teu bem.

\medskip

\lettrine{6} Então Jó respondeu, dizendo: Oh! se a minha
mágoa retamente se pesasse, e a minha miséria juntamente se pusesse
numa balança! Porque, na verdade, mais pesada seria do que a
areia dos mares; por isso é que as minhas palavras têm sido
engolidas. Porque as flechas do Todo-Poderoso estão em mim, cujo
ardente veneno suga o meu espírito; os terrores de Deus se armam
contra mim. Porventura zurrará o jumento montês junto à relva?
Ou mugirá o boi junto ao seu pasto? Ou comer-se-á sem sal o que
é insípido? Ou haverá gosto na clara do ovo? A minha alma recusa
tocá-las, pois são para mim como comida repugnante.

Quem dera que se cumprisse o meu desejo, e que Deus me desse o que
espero! E que Deus quisesse quebrantar-me, e soltasse a sua mão,
e me acabasse! Isto ainda seria a minha consolação, e me
refrigeraria no meu tormento, não me poupando ele; porque não
ocultei as palavras do Santo. Qual é a minha força, para que
eu espere? Ou qual é o meu fim, para que tenha ainda paciência?
É porventura a minha força a força da pedra? Ou é de cobre a
minha carne? Está em mim a minha ajuda? Ou desamparou-me a
verdadeira sabedoria?

Ao que está aflito devia o amigo mostrar compaixão, ainda ao que
deixasse o temor do Todo-Poderoso. Meus irmãos aleivosamente
me trataram, como um ribeiro, como a torrente dos ribeiros que
passam. Que estão encobertos com a geada, e neles se esconde
a neve. No tempo em que se derretem com o calor, se desfazem,
e em se aquentando, desaparecem do seu lugar. Desviam-se as
veredas dos seus caminhos; sobem ao vácuo, e perecem. Os
caminhantes de Tema os vêem; os passageiros de Sabá esperam por
eles. Ficam envergonhados, por terem confiado e, chegando
ali, se confundem. Agora sois semelhantes a eles; vistes o
terror, e temestes.

Acaso disse eu: Dai-me ou oferecei-me presentes de vossos bens?
Ou livrai-me das mãos do opressor? Ou redimi-me das mãos dos
tiranos? Ensinai-me, e eu me calarei; e fazei-me entender em
que errei. Oh! quão fortes são as palavras da boa razão! Mas
que é o que censura a vossa argüição? Porventura buscareis
palavras para me repreenderdes, visto que as razões do desesperado
são como vento? Mas antes lançais sortes sobre o órfão; e
cavais uma cova para o amigo. Agora, pois, se sois servidos,
olhai para mim; e vede se minto em vossa presença. Voltai,
pois, não haja iniqüidade; tornai-vos, digo, que ainda a minha
justiça aparecerá nisso. Há porventura iniqüidade na minha
língua? Ou não poderia o meu paladar distinguir coisas iníquas?

\medskip

\lettrine{7} Porventura não tem o homem guerra sobre a terra?
E não são os seus dias como os dias do jornaleiro\footnote{Operário
a quem se paga jornal (paga de cada dia de trabalho; salário,
jorna); ganhador, ganha-dinheiro. KJ: hireling = mercenário.}?
Como o servo que suspira pela sombra, e como o jornaleiro que
espera pela sua paga, assim me deram por herança meses de
vaidade; e noites de trabalho me prepararam. Deitando-me a
dormir, então digo: Quando me levantarei? Mas comprida é a noite, e
farto-me de me revolver na cama até à alva. A minha carne se tem
vestido de vermes e de torrões de pó; a minha pele está gretada, e
se fez abominável. Os meus dias são mais velozes do que a
lançadeira\footnote{Peça de tear, que contém um cilindro ou canela
por onde passa o fio da tecelagem.} do tecelão, e acabam-se, sem
esperança.

Lembra-te de que a minha vida é como o vento; os meus olhos não
tornarão a ver o bem. Os olhos dos que agora me vêem não me
verão mais; os teus olhos estarão sobre mim, porém não serei mais.
Assim como a nuvem se desfaz e passa, assim aquele que desce à
sepultura nunca tornará a subir. Nunca mais tornará à sua
casa, nem o seu lugar jamais o conhecerá. Por isso não
reprimirei a minha boca; falarei na angústia do meu espírito;
queixar-me-ei na amargura da minha alma. Sou eu porventura o
mar, ou a baleia, para que me ponhas uma guarda? Dizendo eu:
Consolar-me-á a minha cama; meu leito aliviará a minha ânsia;
então me espantas com sonhos, e com visões me assombras;
assim a minha alma escolheria antes a estrangulação; e antes
a morte do que a vida. A minha vida abomino, pois não viveria
para sempre; retira-te de mim; pois vaidade são os meus dias.

Que é o homem, para que tanto o engrandeças, e ponhas nele o teu
coração, e cada manhã o visites, e cada momento o proves?
Até quando não apartarás de mim, nem me largarás, até que
engula a minha saliva? Se pequei, que te farei, ó Guarda dos
homens? Por que fizeste de mim um alvo para ti, para que a mim mesmo
me seja pesado? E por que não perdoas a minha transgressão, e
não tiras a minha iniqüidade? Porque agora me deitarei no pó, e de
madrugada me buscarás, e não existirei mais.

\medskip

\lettrine{8} Então respondendo Bildade o suíta, disse: Até
quando falarás tais coisas, e as palavras da tua boca serão como um
vento impetuoso? Porventura perverteria Deus o direito? E
perverteria o Todo-Poderoso a justiça? Se teus filhos pecaram
contra ele, também ele os lançou na mão da sua transgressão.
Mas, se tu de madrugada buscares a Deus, e ao Todo-Poderoso
pedires misericórdia; se fores puro e reto, certamente logo
despertará por ti, e restaurará a morada da tua justiça. O teu
princípio, na verdade, terá sido pequeno, porém o teu último estado
crescerá em extremo.

Pois, eu te peço, pergunta agora às gerações passadas; e
prepara-te para a inquirição de seus pais. Porque nós somos de
ontem, e nada sabemos; porquanto nossos dias sobre a terra são como
a sombra. Porventura não te ensinarão eles, e não te falarão,
e do seu coração não tirarão palavras? Porventura cresce o
junco sem lodo? Ou cresce a espadana\footnote{Coisa em forma de
espada. Jacto de líquido em forma de lâmina de espada. Cauda de
cometa. Língua de fogo; labareda. Barbatana de peixe. Bras. PA Bot.
Planta herbácea, ornamental, aquática ou palustre, da família das
alismatáceas (Sagittaria acutifolia), de flores pediceladas,
actinomorfas, dispostas em espigas, e cujo fruto, carpelo, é
inteiramente comprimido. KJ: flag.} sem água? Estando ainda
no seu verdor, ainda que não cortada, todavia antes de qualquer
outra erva se seca. Assim são as veredas de todos quantos se
esquecem de Deus; e a esperança do hipócrita perecerá. Cuja
esperança fica frustrada; e a sua confiança será como a teia de
aranha. Encostar-se-á à sua casa, mas ela não subsistirá;
apegar-se-á a ela, mas não ficará em pé. Ele é viçoso perante
o sol, e os seus renovos\footnote{Broto. Ramo novo que cresce do
toco de uma árvore recém-cortada, e do qual se origina uma nova
árvore. Rebento. Fig. Descendência, renova. Renovos: Produtos
agrícolas.} saem sobre o seu jardim; as suas raízes se
entrelaçam, junto à fonte; para o pedregal\footnote{Lugar onde há
muitas pedras.} atenta. Se Deus o consumir do seu lugar,
negá-lo-á este, dizendo: Nunca te vi! Eis que este é a
alegria do seu caminho, e outros brotarão do pó.

Eis que Deus não rejeitará ao reto; nem toma pela mão aos
malfeitores; até que de riso te encha a boca, e os teus
lábios de júbilo. Os que te odeiam se vestirão de confusão, e
a tenda dos ímpios não existirá mais.

\medskip

\lettrine{9} Então Jó respondeu, dizendo: Na verdade sei
que assim é; porque, como se justificaria o homem para com Deus?
Se quiser contender com ele, nem a uma de mil coisas lhe poderá
responder. Ele é sábio de coração, e forte em poder; quem se
endureceu contra ele, e teve paz? Ele é o que remove os montes,
sem que o saibam, e o que os transtorna no seu furor. O que
sacode a terra do seu lugar, e as suas colunas estremecem. O que
fala ao sol, e ele não nasce, e sela as estrelas. O que sozinho
estende os céus, e anda sobre os altos do mar. O que fez a Ursa,
o Órion, e o Sete-estrelo\footnote{Plêiades: grupo de sete estrelas
visíveis a olho desarmado, que fazem parte do aglomerado galáctico
situado na constelação do Touro. KJ: Plêiades.}, e as recâmaras do
sul. O que faz coisas grandes e inescrutáveis; e maravilhas
sem número. Eis que ele passa por diante de mim, e não o
vejo; e torna a passar perante mim, e não o sinto. Eis que
arrebata a presa; quem lha fará restituir? Quem lhe dirá: Que é o
que fazes? Deus não revogará a sua ira; debaixo dele se
encurvam os auxiliadores soberbos.

Quanto menos lhe responderia eu, ou escolheria diante dele as
minhas palavras! Porque, ainda que eu fosse justo, não lhe
responderia; antes ao meu Juiz pediria misericórdia. Ainda
que chamasse, e ele me respondesse, nem por isso creria que desse
ouvidos à minha voz. Porque me quebranta com uma tempestade,
e multiplica as minhas chagas sem causa. Não me permite
respirar, antes me farta de amarguras. Quanto às forças, eis
que ele é o forte; e, quanto ao juízo, quem me citará com ele?
Se eu me justificar, a minha boca me condenará; se for
perfeito, então ela me declarará perverso. Se for perfeito,
não estimo a minha alma; desprezo a minha vida.

A coisa é esta; por isso eu digo que ele consome ao perfeito e ao
ímpio. Quando o açoite mata de repente, então ele zomba da
prova dos inocentes. A terra é entregue nas mãos do ímpio;
ele cobre o rosto dos juízes; se não é ele, quem é, logo?

E os meus dias são mais velozes do que um correio; fugiram, e não
viram o bem. Passam como navios veleiros; como águia que se
lança à comida. Se eu disser: Eu me esquecerei da minha
queixa, e mudarei o meu aspecto e tomarei alento, receio
todas as minhas dores, porque bem sei que não me terás por inocente.
E, sendo eu ímpio, por que trabalharei em vão? Ainda
que me lave com água de neve, e purifique as minhas mãos com sabão,
ainda me submergirás no fosso, e as minhas próprias vestes me
abominarão. Porque ele não é homem, como eu, a quem eu
responda, vindo juntamente a juízo. Não há entre nós árbitro
que ponha a mão sobre nós ambos. Tire ele a sua vara de cima
de mim, e não me amedronte o seu terror. Então falarei, e não
o temerei; porque não sou assim em mim mesmo.

\medskip

\lettrine{10} A minha alma tem tédio da minha vida; darei
livre curso à minha queixa, falarei na amargura da minha alma.
Direi a Deus: Não me condenes; faze-me saber por que contendes
comigo. Parece-te bem que me oprimas, que rejeites o trabalho
das tuas mãos e resplandeças sobre o conselho dos ímpios? Tens
tu porventura olhos de carne? Vês tu como vê o homem? São os
teus dias como os dias do homem? Ou são os teus anos como os anos de
um homem, para te informares da minha iniqüidade, e averiguares
o meu pecado? Bem sabes tu que eu não sou iníquo; todavia
ninguém há que me livre da tua mão.

As tuas mãos me fizeram e me formaram completamente; contudo me
consomes. Peço-te que te lembres de que como barro me formaste e
me farás voltar ao pó. Porventura não me vazaste como leite,
e como queijo não me coalhaste? De pele e carne me vestiste,
e de ossos e nervos me teceste. Vida e misericórdia me
concedeste; e o teu cuidado guardou o meu espírito. Porém
estas coisas as ocultaste no teu coração; bem sei eu que isto esteve
contigo.

Se eu pecar, tu me observas; e da minha iniqüidade não me
escusarás. Se for ímpio, ai de mim! E se for justo, não
levantarei a minha cabeça; farto estou da minha ignomínia; e vê qual
é a minha aflição, porque se vai crescendo; tu me caças como
a um leão feroz; tornas a fazer maravilhas para comigo. Tu
renovas contra mim as tuas testemunhas, e multiplicas contra mim a
tua ira; reveses e combate estão comigo. Por que, pois, me
tiraste da madre? Ah! se então tivera expirado, e olho nenhum me
visse! Então eu teria sido como se nunca fora; e desde o
ventre seria levado à sepultura! Porventura não são poucos os
meus dias? Cessa, pois, e deixa-me, para que por um pouco eu tome
alento. Antes que eu vá para o lugar de que não voltarei, à
terra da escuridão e da sombra da morte; terra escuríssima,
como a própria escuridão, terra da sombra da morte e sem ordem
alguma, e onde a luz é como a escuridão.

\medskip

\lettrine{11} Então respondeu Zofar, o naamatita, e disse:
Porventura não se dará resposta à multidão de palavras? E o
homem falador será justificado? Às tuas mentiras se hão de calar
os homens? E zombarás tu sem que ninguém te envergonhe? Pois
dizes: A minha doutrina é pura, e limpo sou aos teus olhos. Mas
na verdade, quem dera que Deus falasse e abrisse os seus lábios
contra ti! E te fizesse saber os segredos da sabedoria, que é
multíplice em eficácia; sabe, pois, que Deus exige de ti menos do
que merece a tua iniqüidade.

Porventura alcançarás os caminhos de Deus, ou chegarás à perfeição
do Todo-Poderoso? Como as alturas dos céus é a sua sabedoria;
que poderás tu fazer? É mais profunda do que o inferno, que poderás
tu saber? Mais comprida é a sua medida do que a terra, e mais
larga do que o mar. Se ele passar, aprisionar, ou chamar a
juízo, quem o impedirá? Porque ele conhece os homens vãos, e
vê o vício; e não o terá em consideração? Mas o homem vão é
falto de entendimento; sim, o homem nasce como a cria do jumento
montês.

Se tu preparares o teu coração, e estenderes as tuas mãos para
ele; se há iniqüidade na tua mão, lança-a para longe de ti e
não deixes habitar a injustiça nas tuas tendas. Porque então
o teu rosto levantarás sem mácula; e estarás firme, e não temerás.
Porque te esquecerás do cansaço, e lembrar-te-ás dele como
das águas que já passaram. E a tua vida mais clara se
levantará do que o meio dia; ainda que haja trevas, será como a
manhã. E terás confiança, porque haverá esperança; olharás em
volta e repousarás seguro. E deitar-te-ás, e ninguém te
espantará; muitos suplicarão o teu favor. Porém os olhos dos
ímpios desfalecerão, e perecerá o seu refúgio; e a sua esperança
será o expirar da alma.

\medskip

\lettrine{12} Então Jó respondeu, dizendo: Na verdade, vós
sois o povo, e convosco morrerá a sabedoria. Também eu tenho
entendimento como vós, e não vos sou inferior; e quem não sabe tais
coisas como essas? Eu sou motivo de riso para os meus amigos;
eu, que invoco a Deus, e ele me responde; o justo e perfeito serve
de zombaria. Tocha desprezível é, na opinião do que está
descansado, aquele que está pronto a vacilar com os pés.

As tendas dos assoladores têm descanso, e os que provocam a Deus
estão seguros; nas suas mãos Deus lhes põe tudo. Mas, pergunta
agora às alimárias\footnote{Alimária: animal irracional; animália.
Animal de carga; besta.}, e cada uma delas te ensinará; e às aves
dos céus, e elas te farão saber; ou fala com a terra, e ela te
ensinará; até os peixes do mar te contarão. Quem não entende,
por todas estas coisas, que a mão do Senhor fez isto? Na sua
mão está a alma de tudo quanto vive, e o espírito de toda a carne
humana. Porventura o ouvido não provará as palavras, como o
paladar prova as comidas?

Com os idosos está a sabedoria, e na longevidade o entendimento.
Com ele está a sabedoria e a força; conselho e entendimento
tem. Eis que ele derruba, e ninguém há que edifique; prende
um homem, e ninguém há que o solte. Eis que ele retém as
águas, e elas secam; e solta-as, e elas transtornam a terra.
Com ele está a força e a sabedoria; seu é o que erra e o que
o faz errar. Aos conselheiros leva despojados, e aos juízes
faz desvairar. Solta a autoridade dos reis, e ata o cinto aos
seus lombos. Aos sacerdotes leva despojados, aos poderosos
transtorna. Aos acreditados tira a fala, e tira o
entendimento aos anciãos. Derrama desprezo sobre os
príncipes, e afrouxa o cinto dos fortes. Das trevas descobre
coisas profundas, e traz à luz a sombra da morte. Multiplica
as nações e as faz perecer; dispersa as nações, e de novo as
reconduz. Tira o entendimento aos chefes dos povos da terra,
e os faz vaguear pelos desertos, sem caminho. Nas trevas
andam às apalpadelas, sem terem luz, e os faz desatinar como ébrios.

\medskip

\lettrine{13} Eis que tudo isto viram os meus olhos, e os meus
ouvidos o ouviram e entenderam. Como vós o sabeis, também eu o
sei; não vos sou inferior. Mas eu falarei ao Todo-Poderoso, e
quero defender-me perante Deus. Vós, porém, sois inventores de
mentiras, e vós todos médicos que não valem nada. Quem dera que
vos calásseis de todo, pois isso seria a vossa sabedoria. Ouvi
agora a minha defesa, e escutai os argumentos dos meus lábios.
Porventura por Deus falareis perversidade e por ele falareis
mentiras? Fareis acepção da sua pessoa? Contendereis por Deus?
Ser-vos-ia bom, se ele vos esquadrinhasse? Ou zombareis dele,
como se zomba de algum homem? Certamente vos repreenderá, se
em oculto fizerdes acepção de pessoas. Porventura não vos
espantará a sua alteza, e não cairá sobre vós o seu terror?
As vossas memórias são como provérbios de cinza; as vossas
defesas como defesas de lodo.

Calai-vos perante mim, e falarei eu, e venha sobre mim o que
vier. Por que razão tomarei eu a minha carne com os meus
dentes, e porei a minha vida na minha mão? Ainda que ele me
mate, nele esperarei; contudo os meus caminhos defenderei diante
dele. Também ele será a minha salvação; porém o hipócrita não
virá perante ele. Ouvi com atenção as minhas palavras, e com
os vossos ouvidos a minha declaração. Eis que já tenho
ordenado a minha causa, e sei que serei achado justo. Quem é
o que contenderá comigo? Se eu agora me calasse, renderia o
espírito. Duas coisas somente não faças para comigo; então
não me esconderei do teu rosto: desvia a tua mão para longe
de mim, e não me espante o teu terror. Chama, pois, e eu
responderei; ou eu falarei, e tu me responderás.

Quantas culpas e pecados tenho eu? Notifica-me a minha
transgressão e o meu pecado. Por que escondes o teu rosto, e
me tens por teu inimigo? Porventura acossarás uma folha
arrebatada pelo vento? E perseguirás o restolho\footnote{A parte
inferior das gramíneas que fica enraizada após a ceifa. Resíduos,
restos, sobras.} seco? Por que escreves contra mim coisas
amargas e me fazes herdar as culpas da minha mocidade? Também
pões os meus pés no tronco, e observas todos os meus caminhos, e
marcas os sinais dos meus pés. E ele me consome como a
podridão, e como a roupa, à qual rói a traça.

\medskip

\lettrine{14} O homem, nascido da mulher, é de poucos dias e
farto de inquietação. Sai como a flor, e murcha; foge também
como a sombra, e não permanece. E sobre este tal abres os teus
olhos, e a mim me fazes entrar no juízo contigo. Quem do imundo
tirará o puro? Ninguém. Visto que os seus dias estão
determinados, contigo está o número dos seus meses; e tu lhe puseste
limites, e não passará além deles. Desvia-te dele, para que
tenha repouso, até que, como o jornaleiro, tenha contentamento no
seu dia.

Porque há esperança para a árvore que, se for cortada, ainda se
renovará, e não cessarão os seus renovos. Se envelhecer na terra
a sua raiz, e o seu tronco morrer no pó, ao cheiro das águas
brotará, e dará ramos como uma planta. Porém, morto o homem,
é consumido; sim, rendendo o homem o espírito, então onde está ele?
Como as águas se retiram do mar, e o rio se esgota, e fica
seco, assim o homem se deita, e não se levanta; até que não
haja mais céus, não acordará nem despertará de seu sono. Quem
dera que me escondesses na sepultura, e me ocultasses até que a tua
ira se fosse; e me pusesses um limite, e te lembrasses de mim!
Morrendo o homem, porventura tornará a viver? Todos os dias
de meu combate esperaria, até que viesse a minha mudança.
Chamar-me-ias, e eu te responderia, e terias afeto à obra de
tuas mãos.

Mas agora contas os meus passos; porventura não vigias sobre o
meu pecado? A minha transgressão está selada num saco, e
amontoas as minhas iniqüidades. E, na verdade, caindo a
montanha, desfaz-se; e a rocha se remove do seu lugar. As
águas gastam as pedras, as cheias afogam o pó da terra; e tu fazes
perecer a esperança do homem; tu para sempre prevaleces
contra ele, e ele passa; mudas o seu rosto, e o despedes. Os
seus filhos recebem honra, sem que ele o saiba; são humilhados; sem
que ele o perceba; mas a sua carne nele tem dores; e a sua
alma nele lamenta.

\medskip

\lettrine{15} Então respondeu Elifaz o temanita, e disse:
Porventura proferirá o sábio vã sabedoria? E encherá do vento
oriental o seu ventre, argüindo com palavras que de nada servem,
e com razões, de que nada aproveita? E tu tens feito vão o
temor, e diminuis os rogos diante de Deus. Porque a tua boca
declara a tua iniqüidade; e tu escolhes a língua dos astutos. A
tua boca te condena, e não eu, e os teus lábios testificam contra
ti. És tu porventura o primeiro homem que nasceu? Ou foste
formado antes dos outeiros? Ou ouviste o secreto conselho de
Deus e a ti só limitaste a sabedoria? Que sabes tu, que nós não
saibamos? Que entendes, que não haja em nós? Também há entre
nós encanecidos e idosos, muito mais idosos do que teu pai.
Porventura fazes pouco caso das consolações de Deus, e da
suave palavra que te dirigimos? Por que te arrebata o teu
coração, e por que piscam os teus olhos? Para virares contra
Deus o teu espírito, e deixares sair tais palavras da tua boca?
Que é o homem, para que seja puro? E o que nasce da mulher,
para ser justo? Eis que ele não confia nos seus santos, e nem
os céus são puros aos seus olhos. Quanto mais abominável e
corrupto é o homem que bebe a iniqüidade como a água?

Escuta-me, mostrar-te-ei; e o que tenho visto te contarei18(o que
os sábios anunciaram, ouvindo-o de seus pais, e o não ocultaram;
aos quais somente se dera a terra, e nenhum estranho passou
por entre eles): Todos os dias o ímpio é atormentado, e se
reserva, para o tirano, um certo número de anos. O sonido dos
horrores está nos seus ouvidos; até na paz lhe sobrevém o assolador.
Não crê que tornará das trevas, mas que o espera a espada.
Anda vagueando por pão, dizendo: Onde está? Bem sabe que já o
dia das trevas lhe está preparado, à mão. Assombram-no a
angústia e a tribulação; prevalecem contra ele, como o rei preparado
para a peleja; porque estendeu a sua mão contra Deus, e
contra o Todo-Poderoso se embraveceu\footnote{Embravecer: Tornar
bravo, cruel, feroz. Enfurecer-se; embrabecer-se. Encapelar-se,
encrespar-se (o mar). Irritar-se, enfurecer-se.}. Arremete
contra ele com a dura cerviz, e contra os pontos grossos dos seus
escudos. Porquanto cobriu o seu rosto com a sua gordura, e
criou gordura nas ilhargas\footnote{Ilharga: cada uma das partes
laterais e inferiores do baixo-ventre.}. E habitou em cidades
assoladas, em casas em que ninguém morava, que estavam a ponto de
fazer-se montões de ruínas. Não se enriquecerá, nem
subsistirá a sua fazenda, nem se estenderão pela terra as suas
possessões. Não escapará das trevas; a chama do fogo secará
os seus renovos, e ao sopro da sua boca desaparecerá. Não
confie, pois, na vaidade, enganando-se a si mesmo, porque a vaidade
será a sua recompensa. Antes do seu dia ela se consumará; e o
seu ramo não reverdecerá. Sacudirá as suas uvas verdes, como
as da vide, e deixará cair a sua flor como a oliveira, porque
a congregação dos hipócritas se fará estéril, e o fogo consumirá as
tendas do suborno. Concebem a malícia, e dão à luz a
iniqüidade, e o seu ventre prepara enganos.

\medskip

\lettrine{16} Então respondeu Jó, dizendo: Tenho ouvido
muitas coisas como estas; todos vós sois consoladores molestos.
Porventura não terão fim essas palavras de vento? Ou o que te
irrita, para assim responderes? Falaria eu também como vós
falais, se a vossa alma estivesse em lugar da minha alma, ou
amontoaria palavras contra vós, e menearia contra vós a minha
cabeça? Antes vos fortaleceria com a minha boca, e a consolação
dos meus lábios abrandaria a vossa dor.

Se eu falar, a minha dor não cessa, e, calando-me eu, qual é o meu
alívio? Na verdade, agora tu me tens fatigado; tu assolaste toda
a minha companhia, testemunha disto é que já me fizeste
enrugado, e a minha magreza já se levanta contra mim, e no meu rosto
testifica contra mim. Na sua ira me despedaçou, e ele me
perseguiu; rangeu os seus dentes contra mim; aguça o meu adversário
os seus olhos contra mim. Abrem a sua boca contra mim; com
desprezo me feriram nos queixos, e contra mim se ajuntam todos.
Entrega-me Deus ao perverso, e nas mãos dos ímpios me faz
cair. Descansado estava eu, porém ele me quebrantou; e
pegou-me pela cerviz, e me despedaçou; também me pôs por seu alvo.
Cercam-me os seus flecheiros; atravessa-me os rins, e não me
poupa, e o meu fel derrama sobre a terra, fere-me com
ferimento sobre ferimento; arremete contra mim como um valente.
Cosi sobre a minha pele o cilício, e revolvi a minha cabeça
no pó. O meu rosto está todo avermelhado de chorar, e sobre
as minhas pálpebras está a sombra da morte, apesar de não
haver violência nas minhas mãos, e de ser pura a minha oração.

Ah! terra, não cubras o meu sangue e não haja lugar para ocultar
o meu clamor! Eis que também agora a minha testemunha está no
céu, e nas alturas o meu testemunho está. Os meus amigos são
os que zombam de mim; os meus olhos se desfazem em lágrimas diante
de Deus. Ah! se alguém pudesse contender com Deus pelo homem,
como o homem pelo seu próximo! Porque decorridos poucos anos,
eu seguirei o caminho por onde não tornarei.

\medskip

\lettrine{17} O meu espírito se vai consumindo, os meus dias
se vão apagando, e só tenho perante mim a sepultura. Deveras
estou cercado de zombadores, e os meus olhos contemplam as suas
provocações. Promete agora, e dá-me um fiador para contigo; quem
há que me dê a mão? Porque aos seus corações encobriste o
entendimento, por isso não os exaltarás. O que denuncia os seus
amigos, a fim de serem despojados, também os olhos de seus filhos
desfalecerão. Porém a mim me pôs por um provérbio dos povos, de
modo que me tornei uma abominação para eles. Pelo que já se
escureceram de mágoa os meus olhos, e já todos os meus membros são
como a sombra. Os retos pasmarão disto, e o inocente se
levantará contra o hipócrita. E o justo seguirá o seu caminho
firmemente, e o puro de mãos irá crescendo em força.

Mas, na verdade, tornai todos vós e vinde; porque sábio nenhum
acharei entre vós. Os meus dias passaram, e malograram-se os
meus propósitos, as aspirações do meu coração. Trocaram a
noite em dia; a luz está perto do fim, por causa das trevas.
Se eu esperar, a sepultura será a minha casa; nas trevas
estenderei a minha cama. À corrupção clamo: Tu és meu pai; e
aos vermes: Vós sois minha mãe e minha irmã. Onde, pois,
estaria agora a minha esperança? Sim, a minha esperança, quem a
poderá ver? As barras da sepultura descerão quando juntamente
no pó teremos descanso.

\medskip

\lettrine{18} Então respondeu Bildade, o suíta, e disse:
Até quando poreis fim às palavras? Considerai bem, e então
falaremos. Por que somos tratados como animais, e como imundos
aos vossos olhos? Oh tu, que despedaças a tua alma na tua ira,
será a terra deixada por tua causa? Remover-se-ão as rochas do seu
lugar?

Na verdade, a luz dos ímpios se apagará, e a chama do seu fogo não
resplandecerá. A luz se escurecerá nas suas tendas, e a sua
lâmpada sobre ele se apagará. Os seus passos firmes se
estreitarão, e o seu próprio conselho o derrubará. Porque por
seus próprios pés é lançado na rede, e andará nos fios enredados.
O laço o apanhará pelo calcanhar, e a armadilha o prenderá.
Está escondida debaixo da terra uma corda, e uma armadilha na
vereda.

Os assombros o espantarão de todos os lados, e o perseguirão a
cada passo. Será faminto o seu vigor, e a destruição está
pronta ao seu lado. Serão devorados os membros do seu corpo;
sim, o primogênito da morte devorará os seus membros. A sua
confiança será arrancada da sua tenda, onde está confiado, e isto o
fará caminhar para o rei dos terrores. Morará na sua mesma
tenda, o que não lhe pertence; espalhar-se-á enxofre sobre a sua
habitação. Por baixo se secarão as suas raízes e por cima
serão cortados os seus ramos. A sua memória perecerá da
terra, e pelas praças não terá nome. Da luz o lançarão nas
trevas, e afugentá-lo-ão do mundo. Não terá filho nem neto
entre o seu povo, e nem quem lhe suceda nas suas moradas. Do
seu dia se espantarão os do ocidente, assim como se espantam os do
oriente. Tais são, na verdade, as moradas do perverso, e este
é o lugar do que não conhece a Deus.

\medskip

\lettrine{19} Respondeu, porém, Jó, dizendo: Até quando
afligireis a minha alma, e me quebrantareis com palavras? Já dez
vezes me vituperastes; não tendes vergonha de injuriar-me.
Embora haja eu, na verdade, errado, comigo ficará o meu erro.
Se deveras vos quereis engrandecer contra mim, e argüir-me pelo
meu opróbrio, sabei agora que Deus é o que me transtornou, e com
a sua rede me cercou. Eis que clamo: Violência! Porém não sou
ouvido. Grito: Socorro! Porém não há justiça.

O meu caminho ele entrincheirou, e já não posso passar, e nas
minhas veredas pôs trevas. Da minha honra me despojou; e
tirou-me a coroa da minha cabeça. Quebrou-me de todos os
lados, e eu me vou; e arrancou a minha esperança, como a uma árvore.
E fez inflamar contra mim a sua ira, e me reputou para
consigo, como a seus inimigos. Juntas vieram as suas tropas,
e prepararam contra mim o seu caminho, e se acamparam ao redor da
minha tenda. Pôs longe de mim a meus irmãos, e os que me
conhecem, como estranhos se apartaram de mim. Os meus
parentes me deixaram, e os meus conhecidos se esqueceram de mim.
Os meus domésticos e as minhas servas me reputaram como um
estranho, e vim a ser um estrangeiro aos seus olhos. Chamei a
meu criado, e ele não me respondeu; cheguei a suplicar-lhe com a
minha própria boca. O meu hálito se fez estranho à minha
mulher; tanto que supliquei o interesse dos filhos do meu corpo.
Até os pequeninos me desprezam, e, levantando-me eu, falam
contra mim. Todos os homens da minha confidência me abominam,
e até os que eu amava se tornaram contra mim. Os meus ossos
se apegaram à minha pele e à minha carne, e escapei só com a pele
dos meus dentes. Compadecei-vos de mim, amigos meus,
compadecei-vos de mim, porque a mão de Deus me tocou. Por que
me perseguis assim como Deus, e da minha carne não vos fartais?

Quem me dera agora, que as minhas palavras fossem escritas! Quem
me dera, fossem gravadas num livro! E que, com pena de ferro,
e com chumbo, para sempre fossem esculpidas na rocha. Porque
eu sei que o meu Redentor vive, e que por fim se levantará sobre a
terra. E depois de consumida a minha pele, contudo ainda em
minha carne verei a Deus, vê-lo-ei, por mim mesmo, e os meus
olhos, e não outros o contemplarão; e por isso os meus rins se
consomem no meu interior. Na verdade, que devíeis dizer: Por
que o perseguimos? Pois a raiz da acusação se acha em mim.
Temei vós mesmos a espada; porque o furor traz os castigos da
espada, para saberdes que há um juízo.

\medskip

\lettrine{20} Então respondeu Zofar, o naamatita, e disse:
Visto que os meus pensamentos me fazem responder, eu me apresso.
Eu ouvi a repreensão, que me envergonha, mas o espírito do meu
entendimento responderá por mim. Porventura não sabes tu que
desde a antiguidade, desde que o homem foi posto sobre a terra,
o júbilo dos ímpios é breve, e a alegria dos hipócritas
momentânea? Ainda que a sua altivez suba até ao céu, e a sua
cabeça chegue até às nuvens, contudo, como o seu próprio
esterco, perecerá para sempre; e os que o viam dirão: Onde está?
Como um sonho voará, e não será achado, e será afugentado como
uma visão da noite. O olho, que já o viu, jamais o verá, nem o
seu lugar o verá mais.

Os seus filhos procurarão agradar aos pobres, e as suas mãos
restituirão os seus bens. Os seus ossos estão cheios do vigor
da sua mocidade, mas este se deitará com ele no pó. Ainda que
o mal lhe seja doce na boca, e ele o esconda debaixo da sua língua,
e o guarde, e não o deixe, antes o retenha no seu paladar,
contudo a sua comida se mudará nas suas entranhas; fel de
áspides será interiormente. Engoliu riquezas, porém
vomitá-las-á; do seu ventre Deus as lançará. Veneno de
áspides sorverá; língua de víbora o matará. Não verá as
correntes, os rios e os ribeiros de mel e manteiga.
Restituirá o seu trabalho, e não o engolirá; conforme ao
poder de sua mudança, e não saltará de gozo. Porquanto
oprimiu e desamparou os pobres, e roubou a casa que não edificou.
Porquanto não sentiu sossego no seu ventre; nada salvará das
coisas por ele desejadas. Nada lhe sobejará do que coma; por
isso as suas riquezas não durarão. Sendo plena a sua
abastança, estará angustiado; toda a força da miséria virá sobre
ele.

Mesmo estando ele a encher a sua barriga, Deus mandará sobre ele
o ardor da sua ira, e a fará chover sobre ele quando for comer.
Ainda que fuja das armas de ferro, o arco de bronze o
atravessará. Desembainhará a espada que sairá do seu corpo, e
resplandecendo virá do seu fel; e haverá sobre ele assombros.
Toda a escuridão se ocultará nos seus esconderijos; um fogo
não assoprado o consumirá, irá mal com o que ficar na sua tenda.
Os céus manifestarão a sua iniqüidade; e a terra se levantará
contra ele. As riquezas de sua casa serão transportadas; no
dia da sua ira todas se derramarão. Esta, da parte de Deus, é
a porção do homem ímpio; esta é a herança que Deus lhe decretou.

\medskip

\lettrine{21} Respondeu, porém, Jó, dizendo: Ouvi
atentamente as minhas razões; e isto vos sirva de consolação.
Sofrei-me, e eu falarei; e havendo eu falado, zombai.
Porventura eu me queixo de algum homem? Porém, ainda que assim
fosse, por que não se angustiaria o meu espírito? Olhai para
mim, e pasmai; e ponde a mão sobre a boca. Porque, quando me
lembro disto me perturbo, e a minha carne é sobressaltada de horror.

Por que razão vivem os ímpios, envelhecem, e ainda se robustecem
em poder? A sua descendência se estabelece com eles perante a
sua face; e os seus renovos perante os seus olhos. As suas casas
têm paz, sem temor; e a vara de Deus não está sobre eles. O
seu touro gera, e não falha; pare a sua vaca, e não aborta.
Fazem sair as suas crianças, como a um rebanho, e seus filhos
andam saltando. Levantam a voz, ao som do tamboril e da
harpa, e alegram-se ao som do órgão. Na prosperidade gastam
os seus dias, e num momento descem à sepultura. E, todavia,
dizem a Deus: Retira-te de nós; porque não desejamos ter
conhecimento dos teus caminhos. Quem é o Todo-Poderoso, para
que nós o sirvamos? E que nos aproveitará que lhe façamos orações?
Vede, porém, que a prosperidade não está nas mãos deles;
esteja longe de mim o conselho dos ímpios!

Quantas vezes sucede que se apaga a lâmpada dos ímpios, e lhes
sobrevém a sua destruição? E Deus na sua ira lhes reparte dores!
Porque são como a palha diante do vento, e como a
pragana\footnote{Barba de espiga de cereais; aresta.}, que arrebata
o redemoinho. Deus guarda a sua violência para seus filhos, e
dá-lhe o pago, para que o conheça. Seus olhos verão a sua
ruína, e ele beberá do furor do Todo-Poderoso. Por que, que
prazer teria na sua casa, depois de morto, cortando-se-lhe o número
dos seus meses? Porventura a Deus se ensinaria ciência, a ele
que julga os excelsos? Um morre na força da sua plenitude,
estando inteiramente sossegado e tranqüilo. Com seus baldes
cheios de leite, e a medula dos seus ossos umedecida. E
outro, ao contrário, morre na amargura do seu coração, não havendo
provado do bem. Juntamente jazem no pó, e os vermes os
cobrem.

Eis que conheço bem os vossos pensamentos; e os maus intentos com
que injustamente me fazeis violência. Porque direis: Onde
está a casa do príncipe, e onde a tenda em que moravam os ímpios?
Porventura não perguntastes aos que passam pelo caminho, e
não conheceis os seus sinais, que o mau é preservado para o
dia da destruição; e arrebatado no dia do furor? Quem acusará
diante dele o seu caminho, e quem lhe dará o pago do que faz?
Finalmente é levado à sepultura, e vigiam-lhe o túmulo.
Os torrões do vale lhe são doces, e o seguirão todos os
homens; e adiante dele foram inumeráveis. Como, pois, me
consolais com vaidade? Pois nas vossas respostas ainda resta a
transgressão.

\medskip

\lettrine{22} Então respondeu Elifaz, o temanita, dizendo:
Porventura será o homem de algum proveito a Deus? Antes a si
mesmo o prudente será proveitoso. Ou tem o Todo-Poderoso prazer
em que tu sejas justo, ou algum lucro em que tu faças perfeitos os
teus caminhos? Ou te repreende, pelo temor que tem de ti, ou
entra contigo em juízo?

Porventura não é grande a tua malícia, e sem termo as tuas
iniqüidades? Porque sem causa penhoraste a teus irmãos, e aos
nus despojaste as vestes. Não deste ao cansado água a beber, e
ao faminto retiveste o pão. Mas para o poderoso era a terra, e o
homem tido em respeito habitava nela. As viúvas despediste
vazias, e os braços dos órfãos foram quebrados. Por isso é
que estás cercado de laços, e te perturba um pavor repentino,
ou trevas em que nada vês, e a abundância de águas que te
cobre. Porventura Deus não está na altura dos céus? Olha para
a altura das estrelas; quão elevadas estão. E dizes: que sabe
Deus? Porventura julgará ele através da escuridão? As nuvens
são esconderijo para ele, para que não veja; e passeia pelo circuito
dos céus.

Porventura queres guardar a vereda antiga, que pisaram os homens
iníquos? Eles foram arrebatados antes do seu tempo; sobre o
seu fundamento um dilúvio se derramou. Diziam a Deus:
Retira-te de nós. E: Que foi que o Todo-Poderoso nos fez?
Contudo ele encheu de bens as suas casas; mas o conselho dos
ímpios esteja longe de mim. Os justos o vêem, e se alegram, e
o inocente escarnece deles. Porquanto o nosso adversário não
foi destruído, mas o fogo consumiu o que restou deles.

Apega-te, pois, a ele, e tem paz, e assim te sobrevirá o bem.
Aceita, peço-te, a lei da sua boca, e põe as suas palavras no
teu coração. Se te voltares ao Todo-Poderoso, serás
edificado; se afastares a iniqüidade da tua tenda, e deitares
o teu tesouro no pó, e o ouro de Ofir nas pedras dos ribeiros,
então o Todo-Poderoso será o teu tesouro, e a tua prata
acumulada. Porque então te deleitarás no Todo-Poderoso, e
levantarás o teu rosto para Deus. Orarás a ele, e ele te
ouvirá, e pagarás os teus votos. Determinarás tu algum
negócio, e ser-te-á firme, e a luz brilhará em teus caminhos.
Quando te abaterem, então tu dirás: Haja exaltação! E Deus
salvará ao humilde. E livrará até ao que não é inocente;
porque será libertado pela pureza de tuas mãos.

\medskip

\lettrine{23} Respondeu, porém, Jó, dizendo: Ainda hoje a
minha queixa está em amargura; a minha mão pesa sobre meu gemido.
Ah, se eu soubesse onde o poderia achar! Então me chegaria ao
seu tribunal. Exporia ante ele a minha causa, e a minha boca
encheria de argumentos. Saberia as palavras com que ele me
responderia, e entenderia o que me dissesse. Porventura segundo
a grandeza de seu poder contenderia comigo? Não: ele antes me
atenderia. Ali o reto pleitearia com ele, e eu me livraria para
sempre do meu Juiz.

Eis que se me adianto, ali não está; se torno para trás, não o
percebo. Se opera à esquerda, não o vejo; se se encobre à
direita, não o diviso. Porém ele sabe o meu caminho;
provando-me ele, sairei como o ouro. Nas suas pisadas os meus
pés se afirmaram; guardei o seu caminho, e não me desviei dele.
Do preceito de seus lábios nunca me apartei, e as palavras da
sua boca guardei mais do que a minha porção.

Mas, se ele resolveu alguma coisa, quem então o desviará? O que a
sua alma quiser, isso fará. Porque cumprirá o que está
ordenado a meu respeito, e muitas coisas como estas ainda tem
consigo. Por isso me perturbo perante ele, e quando isto
considero, temo-me dele. Porque Deus macerou o meu coração, e
o Todo-Poderoso me perturbou. Porquanto não fui desarraigado
por causa das trevas, e nem encobriu o meu rosto com a escuridão.

\medskip

\lettrine{24} Visto que do Todo-Poderoso não se encobriram os
tempos, por que, os que o conhecem, não vêem os seus dias? Até
os limites removem; roubam os rebanhos, e os apascentam. Do
órfão levam o jumento; tomam em penhor o boi da viúva. Desviam
do caminho os necessitados; e os pobres da terra juntos se escondem.
Eis que, como jumentos monteses no deserto, saem à sua obra,
madrugando para a presa; a campina dá mantimento a eles e aos seus
filhos. No campo segam o seu pasto, e vindimam a vinha do ímpio.
Ao nu fazem passar a noite sem roupa, não tendo ele coberta
contra o frio. Pelas chuvas das montanhas são molhados e, não
tendo refúgio, abraçam-se com as rochas. Ao orfãozinho arrancam
dos peitos, e tomam o penhor do pobre. Fazem com que os nus
vão sem roupa e aos famintos tiram as espigas. Dentro das
suas paredes espremem o azeite; pisam os lagares, e ainda têm sede.
Desde as cidades gemem os homens, e a alma dos feridos
exclama, e contudo Deus lho não imputa como loucura.

Eles estão entre os que se opõem à luz; não conhecem os seus
caminhos, e não permanecem nas suas veredas. De madrugada se
levanta o homicida, mata o pobre e necessitado, e de noite é como o
ladrão. Assim como o olho do adúltero aguarda o crepúsculo,
dizendo: Não me verá olho nenhum; e oculta o rosto, nas
trevas minam as casas, que de dia se marcaram; não conhecem a luz.
Porque a manhã para todos eles é como sombra de morte; pois,
sendo conhecidos, sentem os pavores da sombra da morte.

É ligeiro sobre a superfície das águas; maldita é a sua parte
sobre a terra; não volta pelo caminho das vinhas. A secura e
o calor desfazem as águas da neve; assim desfará a sepultura aos que
pecaram. A madre se esquecerá dele, os vermes o comerão
gostosamente; nunca mais haverá lembrança dele; e a iniqüidade se
quebrará como uma árvore. Aflige à estéril que não dá à luz,
e à viúva não faz bem. Até aos poderosos arrasta com a sua
força; se ele se levanta, não há vida segura. Se Deus lhes dá
descanso, estribam-se nisso; seus olhos porém estão nos caminhos
deles. Por um pouco se exaltam, e logo desaparecem; são
abatidos, encerrados como todos os demais; e cortados como as
cabeças das espigas. Se agora não é assim, quem me desmentirá
e desfará as minhas razões?

\medskip

\lettrine{25} Então respondeu Bildade, o suíta, e disse:
Com ele estão domínio e temor; ele faz paz nas suas alturas.
Porventura têm número as suas tropas? E sobre quem não se
levanta a sua luz?4 Como, pois, seria justo o homem para com Deus, e
como seria puro aquele que nasce de mulher? Eis que até a lua
não resplandece, e as estrelas não são puras aos seus olhos. E
quanto menos o homem, que é um verme, e o filho do homem, que é um
vermezinho!

\medskip

\lettrine{26} Jó, porém, respondeu, dizendo: Como ajudaste
aquele que não tinha força, e sustentaste o braço que não tinha
vigor? Como aconselhaste aquele que não tinha sabedoria, e
plenamente fizeste saber a causa, assim como era? A quem
proferiste palavras, e de quem é o espírito que saiu de ti?

Os mortos tremem debaixo das águas, com os seus moradores. O
inferno está nu perante ele, e não há coberta para a perdição. O
norte estende sobre o vazio; e suspende a terra sobre o nada.
Prende as águas nas suas nuvens, todavia a nuvem não se rasga
debaixo delas. Encobre a face do seu trono, e sobre ele estende
a sua nuvem. Marcou um limite sobre a superfície das águas em
redor, até aos confins da luz e das trevas. As colunas do céu
tremem, e se espantam da sua ameaça. Com a sua força fende o
mar, e com o seu entendimento abate a soberba. Pelo seu
Espírito ornou os céus; a sua mão formou a serpente enroscadiça.
Eis que isto são apenas as orlas dos seus caminhos; e quão
pouco é o que temos ouvido dele! Quem, pois, entenderia o trovão do
seu poder?

\medskip

\lettrine{27} E prosseguindo Jó em seu discurso, disse:
Vive Deus, que desviou a minha causa, e o Todo-Poderoso, que
amargurou a minha alma. Que, enquanto em mim houver alento, e o
sopro de Deus nas minhas narinas, não falarão os meus lábios
iniqüidade, nem a minha língua pronunciará engano. Longe de mim
que eu vos justifique; até que eu expire, nunca apartarei de mim a
minha integridade. À minha justiça me apegarei e não a largarei;
não me reprovará o meu coração em toda a minha vida.

Seja como o ímpio o meu inimigo, e como o perverso o que se
levantar contra mim. Porque qual será a esperança do hipócrita,
havendo sido avaro, quando Deus lhe arrancar a sua alma?
Porventura Deus ouvirá o seu clamor, sobrevindo-lhe a
tribulação? Deleitar-se-á no Todo-Poderoso, ou invocará a
Deus em todo o tempo?

Ensinar-vos-ei acerca da mão de Deus, e não vos encobrirei o que
está com o Todo-Poderoso. Eis que todos vós já o vistes; por
que, pois, vos desvaneceis na vossa vaidade? Esta, pois, é a
porção do homem ímpio da parte de Deus, e a herança, que os tiranos
receberão do Todo-Poderoso. Se os seus filhos se
multiplicarem, será para a espada, e a sua prole não se fartará de
pão. Os que ficarem dele na morte serão enterrados, e as suas
viúvas não chorarão. Se amontoar prata como pó, e aparelhar
roupas como lodo, ele as aparelhará, porém o justo as
vestirá, e o inocente repartirá a prata. E edificará a sua
casa como a traça, e como o guarda que faz a cabana. Rico se
deita, e não será recolhido; abre os seus olhos, e nada terá.
Pavores se apoderam dele como águas; de noite o arrebata a
tempestade. O vento oriental leva-o, e ele se vai, e varre-o
com ímpeto do seu lugar. E Deus lançará isto sobre ele, e não
lhe poupará; irá fugindo da sua mão. Cada um baterá palmas
contra ele e assobiará tirando-o do seu lugar.

\medskip

\lettrine{28} Na verdade, há veios de onde se extrai a prata,
e lugar onde se refina o ouro. O ferro tira-se da terra, e da
pedra se funde o cobre. Ele põe fim às trevas, e toda a
extremidade ele esquadrinha, a pedra da escuridão e a da sombra da
morte. Abre um poço de mina longe dos homens, em lugares
esquecidos do pé; ficando pendentes longe dos homens, oscilam de um
lado para outro. Da terra procede o pão, mas por baixo é
revolvida como por fogo. As suas pedras são o lugar da safira, e
tem pó de ouro. Essa vereda a ave de rapina a ignora, e não a
viram os olhos da gralha. Nunca a pisaram filhos de animais
altivos, nem o feroz leão passou por ela. Ele estende a sua mão
contra o rochedo, e revolve os montes desde as suas raízes.
Dos rochedos faz sair rios, e o seu olho vê tudo o que há de
precioso. Os rios tapa, e nem uma gota sai deles, e tira à
luz o que estava escondido. Porém onde se achará a sabedoria,
e onde está o lugar da inteligência? O homem não conhece o
seu valor, e nem ela se acha na terra dos viventes.

O abismo diz: Não está em mim; e o mar diz: Ela não está comigo.
Não se dará por ela ouro fino, nem se pesará prata em troca
dela. Nem se pode comprar por ouro fino de Ofir, nem pelo
precioso ônix\footnote{Variedade de ágata entre cujas camadas se
observa sensível destaque de cor. Mármore com camadas
policrômicas.}, nem pela safira. Com ela não se pode comparar
o ouro nem o cristal; nem se trocará por jóia de ouro fino.
Não se fará menção de coral nem de pérolas; porque o valor da
sabedoria é melhor que o dos rubis. Não se lhe igualará o
topázio da Etiópia, nem se pode avaliar por ouro puro.

Donde, pois, vem a sabedoria, e onde está o lugar da
inteligência? Pois está encoberta aos olhos de todo o
vivente, e oculta às aves do céu. A perdição e a morte dizem:
Ouvimos com os nossos ouvidos a sua fama. Deus entende o seu
caminho, e ele sabe o seu lugar. Porque ele vê as
extremidades da terra; e vê tudo o que há debaixo dos céus.
Quando deu peso ao vento, e tomou a medida das águas;
quando prescreveu leis para a chuva e caminho para o
relâmpago dos trovões; então a viu e relatou; estabeleceu-a,
e também a esquadrinhou. E disse ao homem: Eis que o temor do
Senhor é a sabedoria, e apartar-se do mal é a inteligência.

\medskip

\lettrine{29} E prosseguiu Jó no seu discurso, dizendo:
Ah! quem me dera ser como eu fui nos meses passados, como nos
dias em que Deus me guardava! Quando fazia resplandecer a sua
lâmpada sobre a minha cabeça e quando eu pela sua luz caminhava
pelas trevas. Como fui nos dias da minha mocidade, quando o
segredo de Deus estava sobre a minha tenda; quando o
Todo-Poderoso ainda estava comigo, e os meus filhos em redor de mim.
Quando lavava os meus passos na manteiga, e da rocha me corriam
ribeiros de azeite.

Quando eu saía para a porta da cidade, e na rua fazia preparar a
minha cadeira, os moços me viam, e se escondiam, e até os idosos
se levantavam e se punham em pé; os príncipes continham as suas
palavras, e punham a mão sobre a sua boca; a voz dos nobres
se calava, e a sua língua apegava-se ao seu paladar.
Ouvindo-me algum ouvido, me tinha por bem-aventurado;
vendo-me algum olho, dava testemunho de mim; porque eu
livrava o miserável, que clamava, como também o órfão que não tinha
quem o socorresse. A bênção do que ia perecendo vinha sobre
mim, e eu fazia que rejubilasse o coração da viúva. Vestia-me
da justiça, e ela me servia de vestimenta; como manto e diadema era
a minha justiça. Eu me fazia de olhos para o cego, e de pés
para o coxo. Dos necessitados era pai, e as causas de que eu
não tinha conhecimento inquiria com diligência. E quebrava os
queixos do perverso, e dos seus dentes tirava a presa.

E dizia: No meu ninho expirarei, e multiplicarei os meus dias
como a areia. A minha raiz se estendia junto às águas, e o
orvalho permanecia sobre os meus ramos; a minha honra se
renovava em mim, e o meu arco se reforçava na minha mão.
Ouviam-me e esperavam, e em silêncio atendiam ao meu
conselho. Havendo eu falado, não replicavam, e minhas razões
destilavam sobre eles; porque me esperavam, como à chuva; e
abriam a sua boca, como à chuva tardia. Se eu ria para eles,
não o criam, e a luz do meu rosto não faziam abater; eu
escolhia o seu caminho, assentava-me como chefe, e habitava como rei
entre as suas tropas; como aquele que consola os que pranteiam.

\medskip

\lettrine{30} Agora, porém, se riem de mim os de menos idade
do que eu, cujos pais eu teria desdenhado de pôr com os cães do meu
rebanho. De que também me serviria a força das mãos daqueles,
cujo vigor se tinha esgotado? De míngua e fome se debilitaram; e
recolhiam-se para os lugares secos, tenebrosos, assolados e
desertos. Apanhavam malvas\footnote{Malva: Gênero de plantas
herbáceas, nativas do Velho Mundo, dotadas de folhas palmadas e
flores tribacteadas. Qualquer espécie desse gênero, como, p. ex., a
Malva silvestris, cujas folhas encerram mucilagem, razão por que é
us. em medicina. Qualquer espécime desse gênero, ou a sua flor. A
cor rosa-arroxeada da flor da malva. Diz-se dessa cor.} junto aos
arbustos, e o seu mantimento eram as raízes dos
zimbros\footnote{Planta da família das pináceas (Juniperus
communis), cujos frutos se utilizam na preparação do gim ou da
genebra e na aromatização de conservas ou carnes defumadas;
junípero.}. Do meio dos homens eram expulsos, e gritavam contra
eles, como contra o ladrão; para habitarem nos barrancos dos
vales, e nas cavernas da terra e das rochas. Bramavam entre os
arbustos, e ajuntavam-se debaixo das urtigas. Eram filhos de
doidos, e filhos de gente sem nome, e da terra foram expulsos.
Agora, porém, sou a sua canção, e lhes sirvo de provérbio.
Abominam-me, e fogem para longe de mim, e no meu rosto não se
privam de cuspir. Porque Deus desatou a sua corda, e me
oprimiu, por isso sacudiram de si o freio perante o meu rosto.
À direita se levantam os moços; empurram os meus pés, e
preparam contra mim os seus caminhos de destruição.
Desbaratam-me o caminho; promovem a minha miséria; contra
eles não há ajudador. Vêm contra mim como por uma grande
brecha, e revolvem-se entre a assolação.

Sobrevieram-me pavores; como vento perseguem a minha honra, e
como nuvem passou a minha felicidade. E agora derrama-se em
mim a minha alma; os dias da aflição se apoderaram de mim. De
noite se me traspassam os meus ossos, e os meus nervos não
descansam. Pela grandeza do meu mal está desfigurada a minha
veste, que, como a gola da minha túnica, me cinge. Lançou-me
na lama, e fiquei semelhante ao pó e à cinza. Clamo a ti,
porém, tu não me respondes; estou em pé, porém, para mim não
atentas. Tornaste-te cruel contra mim; com a força da tua mão
resistes violentamente. Levantas-me sobre o vento, fazes-me
cavalgar sobre ele, e derretes-me o ser. Porque eu sei que me
levarás à morte e à casa do ajuntamento determinada a todos os
viventes. Porém não estenderá a mão para o túmulo, ainda que
eles clamem na sua destruição. Porventura não chorei sobre
aquele que estava aflito, ou não se angustiou a minha alma pelo
necessitado? Todavia aguardando eu o bem, então me veio o
mal; esperando eu a luz, veio a escuridão. As minhas
entranhas fervem e não estão quietas; os dias da aflição me
surpreendem. Denegrido ando, porém não do sol; levantando-me
na congregação, clamo por socorro. Irmão me fiz dos chacais,
e companheiro dos avestruzes. Enegreceu-se a minha pele sobre
mim, e os meus ossos estão queimados do calor. A minha harpa
se tornou em luto, e o meu órgão em voz dos que choram.

\medskip

\lettrine{31} Fiz aliança com os meus olhos; como, pois, os
fixaria numa virgem? Que porção teria eu do Deus lá de cima, ou
que herança do Todo-Poderoso desde as alturas? Porventura não é
a perdição para o perverso, o desastre para os que praticam
iniqüidade? Ou não vê ele os meus caminhos, e não conta todos os
meus passos? Se andei com falsidade, e se o meu pé se apressou
para o engano6(pese-me em balanças fiéis, e saberá Deus a minha
sinceridade), se os meus passos se desviaram do caminho, e se o
meu coração segue os meus olhos, e se às minhas mãos se apegou
qualquer coisa, então semeie eu e outro coma, e seja a minha
descendência arrancada até à raiz.

Se o meu coração se deixou seduzir por uma mulher, ou se eu armei
traições à porta do meu próximo, então moa minha mulher para
outro, e outros se encurvem sobre ela, porque é uma infâmia,
e é delito pertencente aos juízes. Porque é fogo que consome
até à perdição, e desarraigaria toda a minha renda. Se
desprezei o direito do meu servo ou da minha serva, quando eles
contendiam comigo; então que faria eu quando Deus se
levantasse? E, inquirindo a causa, que lhe responderia?
Aquele que me formou no ventre não o fez também a ele? Ou não
nos formou do mesmo modo na madre?

Se retive o que os pobres desejavam, ou fiz desfalecer os olhos
da viúva, ou se sozinho comi o meu bocado, e o órfão não
comeu dele18(porque desde a minha mocidade cresceu comigo como com
seu pai, e fui o guia da viúva desde o ventre de minha mãe),
se alguém vi perecer por falta de roupa, e ao necessitado por
não ter coberta, se os seus lombos não me abençoaram, se ele
não se aquentava com as peles dos meus cordeiros, se eu
levantei a minha mão contra o órfão, porquanto na porta via a minha
ajuda, então caia do ombro a minha espádua, e separe-se o meu
braço do osso. Porque o castigo de Deus era para mim um
assombro, e eu não podia suportar a sua grandeza.

Se no ouro pus a minha esperança, ou disse ao ouro fino: Tu és a
minha confiança; se me alegrei de que era muita a minha
riqueza, e de que a minha mão tinha alcançado muito; se olhei
para o sol, quando resplandecia, ou para a lua, caminhando gloriosa,
e o meu coração se deixou enganar em oculto, e a minha boca
beijou a minha mão, também isto seria delito à punição de
juízes; pois assim negaria a Deus que está lá em cima. Se me
alegrei da desgraça do que me tem ódio, e se exultei quando o mal o
atingiu30(também não deixei pecar a minha boca, desejando a sua
morte com maldição); se a gente da minha tenda não disse: Ah!
quem nos dará da sua carne? Nunca nos fartaríamos dela. O
estrangeiro não passava a noite na rua; as minhas portas abria ao
viandante.

Se, como Adão, encobri as minhas transgressões, ocultando o meu
delito no meu seio; porque eu temia a grande multidão, e o
desprezo das famílias me apavorava, e eu me calei, e não saí da
porta; ah! quem me dera um que me ouvisse! Eis que o meu
desejo é que o Todo-Poderoso me responda, e que o meu adversário
escreva um livro. Por certo que o levaria sobre o meu ombro,
sobre mim o ataria por coroa. O número dos meus passos lhe
mostraria; como príncipe me chegaria a ele. Se a minha terra
clamar contra mim, e se os seus sulcos juntamente chorarem,
se comi os seus frutos sem dinheiro, e sufoquei a alma dos
seus donos, por trigo me produza cardos, e por cevada joio.
Acabaram-se as palavras de Jó.

\medskip

\lettrine{32} Então aqueles três homens cessaram de responder
a Jó; porque era justo aos seus próprios olhos. E acendeu-se a
ira de Eliú, filho de Baraquel, o buzita, da família de Rão; contra
Jó se acendeu a sua ira, porque se justificava a si mesmo, mais do
que a Deus. Também a sua ira se acendeu contra os seus três
amigos, porque, não achando que responder, todavia condenavam a Jó.
Eliú, porém, esperou para falar a Jó, porquanto tinham mais
idade do que ele. Vendo, pois, Eliú que já não havia resposta na
boca daqueles três homens, a sua ira se acendeu.

E respondeu Eliú, filho de Baraquel, o buzita, dizendo: Eu sou de
menos idade, e vós sois idosos; receei-me e temi de vos declarar a
minha opinião. Dizia eu: Falem os dias, e a multidão dos anos
ensine a sabedoria. Na verdade, há um espírito no homem, e a
inspiração do Todo-Poderoso o faz entendido. Os grandes não são
os sábios, nem os velhos entendem o que é direito. Assim
digo: Dai-me ouvidos, e também eu declararei a minha opinião.
Eis que aguardei as vossas palavras, e dei ouvidos às vossas
considerações, até que buscásseis razões. Atentando, pois,
para vós, eis que nenhum de vós há que possa convencer a Jó, nem que
responda às suas razões; para que não digais: Achamos a
sabedoria; Deus o derrubou, e não homem algum. Ora ele não
dirigiu contra mim palavra alguma, nem lhe responderei com as vossas
palavras.

Estão pasmados, não respondem mais, faltam-lhes as palavras.
Esperei, pois, mas não falam; porque já pararam, e não
respondem mais. Também eu responderei pela minha parte;
também eu declararei a minha opinião. Porque estou cheio de
palavras; o meu espírito me constrange. Eis que dentro de mim
sou como o mosto, sem respiradouro, prestes a arrebentar, como odres
novos. Falarei, para que ache alívio; abrirei os meus lábios,
e responderei. Que não faça eu acepção de pessoas, nem use de
palavras lisonjeiras com o homem! Porque não sei usar de
lisonjas; em breve me levaria o meu Criador.

\medskip

\lettrine{33} Assim, na verdade, ó Jó, ouve as minhas razões,
e dá ouvidos a todas as minhas palavras. Eis que já abri a minha
boca; já falou a minha língua debaixo do meu paladar. As minhas
razões provam a sinceridade do meu coração, e os meus lábios
proferem o puro saber. O Espírito de Deus me fez; e a inspiração
do Todo-Poderoso me deu vida. Se podes, responde-me, põe em
ordem as tuas razões diante de mim, e apresenta-te. Eis que vim
de Deus, como tu; do barro também eu fui formado. Eis que não te
perturbará o meu terror, nem será pesada sobre ti a minha mão.

Na verdade tu falaste aos meus ouvidos; e eu ouvi a voz das tuas
palavras. Dizias: Limpo estou, sem transgressão; puro sou, e não
tenho iniqüidade. Eis que procura pretexto contra mim, e me
considera como seu inimigo. Põe no tronco os meus pés, e
observa todas as minhas veredas. Eis que nisso não tens
razão; eu te respondo; porque maior é Deus do que o homem.
Por que razão contendes com ele, sendo que não responde
acerca de todos os seus feitos?

Antes Deus fala uma e duas vezes; porém ninguém atenta para isso.
Em sonho ou em visão noturna, quando cai sono profundo sobre
os homens, e adormecem na cama. Então o revela ao ouvido dos
homens, e lhes sela a sua instrução, para apartar o homem
daquilo que faz, e esconder do homem a soberba. Para desviar
a sua alma da cova, e a sua vida de passar pela espada.

Também na sua cama é castigado com dores; e com incessante
contenda nos seus ossos; de modo que a sua vida abomina até o
pão, e a sua alma a comida apetecível. Desaparece a sua carne
a olhos vistos, e os seus ossos, que não se viam, agora aparecem.
E a sua alma se vai chegando à cova, e a sua vida aos que
trazem a morte. Se com ele, pois, houver um mensageiro, um
intérprete, um entre milhares, para declarar ao homem a sua retidão,
então terá misericórdia dele, e lhe dirá: Livra-o, para que
não desça à cova; já achei resgate. Sua carne se reverdecerá
mais do que era na mocidade, e tornará aos dias da sua juventude.
Deveras orará a Deus, o qual se agradará dele, e verá a sua
face com júbilo, e restituirá ao homem a sua justiça. Olhará
para os homens, e dirá: Pequei, e perverti o direito, o que de nada
me aproveitou. Porém Deus livrou a minha alma de ir para a
cova, e a minha vida verá a luz.

Eis que tudo isto é obra de Deus, duas e três vezes para com o
homem, para desviar a sua alma da perdição, e o iluminar com
a luz dos viventes. Escuta, pois, ó Jó, ouve-me; cala-te, e
eu falarei. Se tens alguma coisa que dizer, responde-me;
fala, porque desejo justificar-te. Se não, escuta-me tu;
cala-te, e ensinar-te-ei a sabedoria.

\medskip

\lettrine{34} Respondeu mais Eliú, dizendo: Ouvi, vós,
sábios, as minhas razões; e vós, entendidos, inclinai os ouvidos
para mim. Porque o ouvido prova as palavras, como o paladar
experimenta a comida. O que é direito escolhamos para nós; e
conheçamos entre nós o que é bom. Porque Jó disse: Sou justo, e
Deus tirou o meu direito. Apesar do meu direito sou considerado
mentiroso; a minha ferida é incurável, embora eu esteja sem
transgressão. Que homem há como Jó, que bebe a zombaria como
água? E caminha em companhia dos que praticam a iniqüidade, e
anda com homens ímpios? Porque disse: De nada aproveita ao homem
o comprazer-se em Deus.

Portanto vós, homens de entendimento, escutai-me: Longe de Deus
esteja o praticar a maldade e do Todo-Poderoso o cometer a
perversidade! Porque, segundo a obra do homem, ele lhe paga;
e faz a cada um segundo o seu caminho. Também, na verdade,
Deus não procede impiamente; nem o Todo-Poderoso perverte o juízo.
Quem lhe entregou o governo da terra? E quem fez todo o
mundo? Se ele pusesse o seu coração contra o homem, e
recolhesse para si o seu espírito e o seu fôlego, toda a
carne juntamente expiraria, e o homem voltaria para o pó.

Se, pois, há em ti entendimento, ouve isto; inclina os ouvidos ao
som da minha palavra. Porventura o que odiasse o direito se
firmaria? E tu condenarias aquele que é justo e poderoso? Ou
dir-se-á a um rei: Oh! vil? Ou aos príncipes: Oh! ímpios?
Quanto menos àquele, que não faz acepção das pessoas de
príncipes, nem estima o rico mais do que o pobre; porque todos são
obras de suas mãos. Eles num momento morrem; e até à meia
noite os povos são perturbados, e passam, e os poderosos serão
tomados não por mão humana. Porque os seus olhos estão sobre
os caminhos de cada um, e ele vê todos os seus passos. Não há
trevas nem sombra de morte, onde se escondam os que praticam a
iniqüidade. Porque Deus não sobrecarrega o homem mais do que
é justo, para o fazer ir a juízo diante dele. Quebranta aos
fortes, sem que se possa inquirir, e põe outros em seu lugar.
Ele conhece, pois, as suas obras; de noite os transtorna, e
ficam moídos. Ele os bate como ímpios que são, à vista dos
espectadores; porquanto se desviaram dele, e não
compreenderam nenhum de seus caminhos, de sorte que o clamor
do pobre subisse até ele, e que ouvisse o clamor dos aflitos.
Se ele aquietar, quem então inquietará? Se encobrir o rosto,
quem então o poderá contemplar? Seja isto para com um povo, seja
para com um homem só, para que o homem hipócrita nunca mais
reine, e não haja laços no povo.

Na verdade, quem a Deus disse: Suportei castigo, não ofenderei
mais. O que não vejo, ensina-me tu; se fiz alguma maldade,
nunca mais a hei de fazer? Virá de ti como há de ser a
recompensa, para que tu a rejeites? Faze tu, pois, e não eu, a
escolha; fala logo o que sabes. Os homens de entendimento
dirão comigo, e o homem sábio que me ouvir: Jó falou sem
conhecimento; e às suas palavras falta prudência. Pai meu!
Provado seja Jó até ao fim, pelas suas respostas próprias de homens
malignos. Porque ao seu pecado acrescenta a transgressão;
entre nós bate palmas, e multiplica contra Deus as suas palavras.

\medskip

\lettrine{35} Respondeu mais Eliú, dizendo: Tens por
direito dizeres: Maior é a minha justiça do que a de Deus?
Porque disseste: De que me serviria? Que proveito tiraria mais
do que do meu pecado? Eu te darei resposta, a ti e aos teus
amigos contigo. Atenta para os céus, e vê; e contempla as mais
altas nuvens, que são mais altas do que tu. Se pecares, que
efetuarás contra ele? Se as tuas transgressões se multiplicarem, que
lhe farás? Se fores justo, que lhe darás, ou que receberá ele da
tua mão? A tua impiedade faria mal a outro tal como tu; e a tua
justiça aproveitaria ao filho do homem.

Por causa das muitas opressões os homens clamam por causa do braço
dos grandes. Porém ninguém diz: Onde está Deus que me criou,
que dá salmos durante a noite; que nos ensina mais do que aos
animais da terra e nos faz mais sábios do que as aves dos céus?
Clamam, porém ele não responde, por causa da arrogância dos
maus. Certo é que Deus não ouvirá a vaidade, nem atentará
para ela o Todo-Poderoso.

E quanto ao que disseste, que o não verás, juízo há perante ele;
por isso espera nele. Mas agora, porque a sua ira ainda não
se exerce, nem grandemente considera a arrogância, logo Jó em
vão abre a sua boca, e sem ciência multiplica palavras.

\medskip

\lettrine{36} Prosseguiu ainda Eliú, e disse: Espera-me um
pouco, e mostrar-te-ei que ainda há razões a favor de Deus. De
longe trarei o meu conhecimento; e ao meu Criador atribuirei a
justiça. Porque, na verdade, as minhas palavras não serão
falsas; contigo está um que tem perfeito conhecimento.

Eis que Deus é mui grande, contudo a ninguém despreza; grande é em
força e sabedoria. Ele não preserva a vida do ímpio, e faz
justiça aos aflitos. Do justo não tira os seus olhos; antes
estão com os reis no trono; ali os assenta para sempre, e assim são
exaltados. E se estão presos em grilhões, amarrados com cordas
de aflição, então lhes faz saber a obra deles, e as suas
transgressões, porquanto prevaleceram nelas. Abre-lhes também
os seus ouvidos, para sua disciplina, e ordena-lhes que se convertam
da maldade. Se o ouvirem, e o servirem, acabarão seus dias em
bem, e os seus anos em delícias. Porém se não o ouvirem, à
espada serão passados, e expirarão sem conhecimento. E os
hipócritas de coração amontoam para si a ira; e amarrando-os ele,
não clamam por socorro. A sua alma morre na mocidade, e a sua
vida perece entre os impuros.

Ao aflito livra da sua aflição, e na opressão se revela aos seus
ouvidos. Assim também te desviará da boca da angústia para um
lugar espaçoso, em que não há aperto, e as iguarias da tua mesa
serão cheias de gordura. Mas tu estás cheio do juízo do
ímpio; o juízo e a justiça te sustentam. Porquanto há furor,
guarda-te de que não sejas atingido pelo castigo violento, pois nem
com resgate algum te livrarias dele. Estimaria ele tanto tuas
riquezas? Não, nem ouro, nem todas as forças do poder. Não
suspires pela noite, em que os povos sejam tomados do seu lugar.
Guarda-te, e não declines para a iniqüidade; porquanto isso
escolheste antes que a aflição. Eis que Deus é
excelso\footnote{Alto, elevado; sublime. Excelente, admirável.} em
seu poder; quem ensina como ele? Quem lhe prescreveu o seu
caminho? Ou, quem lhe dirá: Tu cometeste maldade?

Lembra-te de engrandecer a sua obra, que os homens contemplam.
Todos os homens a vêem, e o homem a enxerga de longe.
Eis que Deus é grande, e nós não o compreendemos, e o número
dos seus anos não se pode esquadrinhar. Porque faz miúdas as
gotas das águas que, do seu vapor, derramam a chuva, a qual
as nuvens destilam e gotejam sobre o homem abundantemente.
Porventura pode alguém entender as extensões das nuvens, e os
estalos da sua tenda? Eis que estende sobre elas a sua luz, e
encobre as profundezas do mar. Porque por estas coisas julga
os povos e lhes dá mantimento em abundância. Com as nuvens
encobre a luz, e ordena não brilhar, interpondo a nuvem.

O que nos dá a entender o seu pensamento, como também ao gado,
acerca do temporal que sobe.

\medskip

\lettrine{37} Sobre isto também treme o meu coração, e salta
do seu lugar. Atentamente ouvi a indignação da sua voz, e o
sonido que sai da sua boca. Ele o envia por debaixo de todos os
céus, e a sua luz até aos confins da terra. Depois disto ruge
uma voz; ele troveja com a sua voz majestosa; e ele não os detém
quando a sua voz é ouvida. Com a sua voz troveja Deus
maravilhosamente; faz grandes coisas, que nós não podemos
compreender.

Porque à neve diz: Cai sobre a terra; como também à garoa e à sua
forte chuva. Ele sela as mãos de todo o homem, para que conheçam
todos os homens a sua obra. E as feras entram nos seus
esconderijos e ficam nas suas cavernas. Da recâmara do sul sai o
tufão, e do norte o frio. Pelo sopro de Deus se dá a geada, e
as largas águas se congelam. Também de umidade carrega as
grossas nuvens, e esparge as nuvens com a sua luz. Então
elas, segundo o seu prudente conselho, se espalham em redor, para
que façam tudo quanto lhes ordena sobre a superfície do mundo na
terra. Seja que por vara, ou para a sua terra, ou por
misericórdia as faz vir.

A isto, ó Jó, inclina os teus ouvidos; pára e considera as
maravilhas de Deus. Porventura sabes tu como Deus as opera, e
faz resplandecer a luz da sua nuvem? Tens tu notícia do
equilíbrio das grossas nuvens e das maravilhas daquele que é
perfeito nos conhecimentos? Ou de como as tuas roupas
aquecem, quando do sul há calma sobre a terra? Ou estendeste
com ele os céus, que estão firmes como espelho fundido?
Ensina-nos o que lhe diremos: porque nós nada poderemos pôr
em boa ordem, por causa das trevas. Contar-lhe-ia alguém o
que tenho falado? Ou desejaria um homem que ele fosse devorado?

E agora não se pode olhar para o sol, que resplandece nas nuvens,
quando o vento, tendo passado, o deixa limpo. O esplendor de
ouro vem do norte; pois, em Deus há uma tremenda majestade.
Ao Todo-Poderoso não podemos alcançar; grande é em poder;
porém a ninguém oprime em juízo e grandeza de justiça. Por
isso o temem os homens; ele não respeita os que se julgam sábios de
coração.

\medskip

\lettrine{38} Depois disto o Senhor respondeu a Jó de um
redemoinho, dizendo: Quem é este que escurece o conselho com
palavras sem conhecimento? Agora cinge os teus lombos, como
homem; e perguntar-te-ei, e tu me ensinarás.

Onde estavas tu, quando eu fundava a terra? Faze-mo saber, se tens
inteligência. Quem lhe pôs as medidas, se é que o sabes? Ou quem
estendeu sobre ela o cordel? Sobre que estão fundadas as suas
bases, ou quem assentou a sua pedra de esquina, quando as
estrelas da alva juntas alegremente cantavam, e todos os filhos de
Deus jubilavam? Ou quem encerrou o mar com portas, quando este
rompeu e saiu da madre; quando eu pus as nuvens por sua
vestidura, e a escuridão por faixa? Quando eu lhe tracei
limites, e lhe pus portas e ferrolhos, e disse: Até aqui
virás, e não mais adiante, e aqui se parará o orgulho das tuas
ondas?

Ou desde os teus dias deste ordem à madrugada, ou mostraste à
alva o seu lugar; para que pegasse nas extremidades da terra,
e os ímpios fossem sacudidos dela; e se transformasse como o
barro sob o selo, e se pusessem como vestidos; e dos ímpios
se desvie a sua luz, e o braço altivo se quebrante; ou
entraste tu até às origens do mar, ou passeaste no mais profundo do
abismo? Ou descobriram-se-te as portas da morte, ou viste as
portas da sombra da morte? Ou com o teu entendimento chegaste
às larguras da terra? Faze-mo saber, se sabes tudo isto. Onde
está o caminho onde mora a luz? E, quanto às trevas, onde está o seu
lugar; para que as tragas aos seus limites, e para que saibas
as veredas da sua casa? De certo tu o sabes, porque já então
eras nascido, e por ser grande o número dos teus dias! Ou
entraste tu até aos tesouros da neve, e viste os tesouros da
saraiva, que eu retenho até ao tempo da angústia, até ao dia
da peleja e da guerra? Onde está o caminho em que se reparte
a luz, e se espalha o vento oriental sobre a terra?

Quem abriu para a inundação um leito, e um caminho para os
relâmpagos dos trovões, para chover sobre a terra, onde não
há ninguém, e no deserto, em que não há homem; para fartar a
terra deserta e assolada, e para fazer crescer os renovos da erva?
A chuva porventura tem pai? Ou quem gerou as gotas do
orvalho? De que ventre procedeu o gelo? E quem gerou a geada
do céu? Como debaixo de pedra as águas se endurecem, e a
superfície do abismo se congela. Ou poderás tu ajuntar as
delícias do Sete-estrelo ou soltar os cordéis do
Órion\footnote{Constelação equatorial, a O. do Unicórnio, a E. do
Touro e do Erídano, ao S. do Touro e ao N. da Lebre, formada de
estrelas brilhantes, três das quais são popularmente chamadas Três
Marias.}? Ou produzir as constelações a seu tempo, e guiar a
Ursa com seus filhos? Sabes tu as ordenanças dos céus, ou
podes estabelecer o domínio deles sobre a terra? Ou podes
levantar a tua voz até às nuvens, para que a abundância das águas te
cubra? Ou mandarás aos raios para que saiam, e te digam:
Eis-nos aqui? Quem pôs a sabedoria no íntimo, ou quem deu à
mente o entendimento? Quem numerará as nuvens com sabedoria?
Ou os odres dos céus, quem os esvaziará, quando se funde o pó
numa massa, e se apegam os torrões uns aos outros? Porventura
caçarás tu presa para a leoa, ou saciarás a fome dos filhos dos
leões, quando se agacham nos covis, e estão à espreita nas
covas? Quem prepara aos corvos o seu alimento, quando os seus
filhotes gritam a Deus e andam vagueando, por não terem o que comer?

\medskip

\lettrine{39} Sabes tu o tempo em que as cabras montesas têm
filhos, ou observastes as cervas quando dão suas crias? Contarás
os meses que cumprem, ou sabes o tempo do seu parto? Quando se
encurvam, produzem seus filhos, e lançam de si as suas dores.
Seus filhos enrijam, crescem com o trigo; saem, e nunca mais
tornam para elas. Quem despediu livre o jumento montês, e quem
soltou as prisões ao jumento bravo, ao qual dei o ermo por casa,
e a terra salgada por morada? Ri-se do ruído da cidade; não ouve
os muitos gritos do condutor. A região montanhosa é o seu pasto,
e anda buscando tudo que está verde. Ou, querer-te-á servir o
boi selvagem? Ou ficará no teu curral? Ou com corda
amarrarás, no arado, ao boi selvagem? Ou escavará ele os vales após
ti? Ou confiarás nele, por ser grande a sua força, ou
deixarás a seu cargo o teu trabalho? Ou fiarás\footnote{Ser o
fiador de; abonar, afiançar. Esperar, acreditar, confiar.} dele que
te torne o que semeaste e o recolha na tua eira?

A avestruz bate alegremente as suas asas, porém, são benignas as
suas asas e penas? Ela deixa os seus ovos na terra, e os
aquenta no pó, e se esquece de que algum pé os pode pisar, ou
que os animais do campo os podem calcar. Endurece-se para com
seus filhos, como se não fossem seus; debalde é seu trabalho, mas
ela está sem temor, porque Deus a privou de sabedoria, e não
lhe deu entendimento. A seu tempo se levanta ao alto; ri-se
do cavalo, e do que vai montado nele.

Ou darás tu força ao cavalo, ou revestirás o seu pescoço com
crinas? Ou espantá-lo-ás, como ao gafanhoto? Terrível é o
fogoso respirar das suas ventas. Escarva\footnote{Escarvar:
Abrir escarva [lesão produzida por enxoadas (=tumor que se forma nos
cascos das bestas) nos cascos dos eqüídeos] em. Cavar
superficialmente.} a terra, e folga na sua força, e sai ao encontro
dos armados. Ri-se do temor, e não se espanta, e não torna
atrás por causa da espada. Contra ele rangem a aljava, o
ferro flamante da lança e do dardo. Agitando-se e
indignando-se, serve a terra, e não faz caso do som da buzina.
Ao soar das buzinas diz: Eia! E cheira de longe a guerra, e o
trovão dos capitães, e o alarido.

Ou voa o gavião pela tua inteligência, e estende as suas asas
para o sul? Ou se remonta a águia ao teu mandado, e põe no
alto o seu ninho? Nas penhas mora e habita; no cume das
penhas, e nos lugares seguros. Dali descobre a presa; seus
olhos a avistam de longe. E seus filhos chupam o sangue; e
onde há mortos, ali está ela.

\medskip

\lettrine{40} Respondeu mais o Senhor a Jó, dizendo:
Porventura o contender contra o Todo-Poderoso é sabedoria? Quem
argüi assim a Deus, responda por isso. Então Jó respondeu ao
Senhor, dizendo: Eis que sou vil; que te responderia eu? A minha
mão ponho à boca. Uma vez tenho falado, e não replicarei; ou
ainda duas vezes, porém não prosseguirei.

Então o Senhor respondeu a Jó de um redemoinho, dizendo: Cinge
agora os teus lombos como homem; eu te perguntarei, e tu me
explicarás. Porventura também tornarás tu vão o meu juízo, ou tu
me condenarás, para te justificares? Ou tens braço como Deus, ou
podes trovejar com voz como ele o faz? Orna-te, pois, de
excelência e alteza; e veste-te de majestade e de glória.
Derrama os furores da tua ira, e atenta para todo o soberbo,
e abate-o. Olha para todo o soberbo, e humilha-o, e atropela
os ímpios no seu lugar. Esconde-os juntamente no pó; ata-lhes
os rostos em oculto. Então também eu a ti confessarei que a
tua mão direita te poderá salvar.

Contemplas agora o beemote, que eu fiz contigo, que come a erva
como o boi. Eis que a sua força está nos seus lombos, e o seu
poder nos músculos do seu ventre. Quando quer, move a sua
cauda como cedro; os nervos das suas coxas estão entretecidos.
Os seus ossos são como tubos de bronze; a sua ossada é como
barras de ferro. Ele é obra-prima dos caminhos de Deus; o que
o fez o proveu da sua espada. Em verdade os montes lhe
produzem pastos, onde todos os animais do campo folgam.
Deita-se debaixo das árvores sombrias, no esconderijo das
canas e da lama. As árvores sombrias o cobrem, com sua
sombra; os salgueiros do ribeiro o cercam. Eis que um rio
transborda, e ele não se apressa, confiando ainda que o Jordão se
levante até à sua boca. Podê-lo-iam porventura caçar à vista
de seus olhos, ou com laços lhe furar o nariz?

\medskip

\lettrine{41} Poderás tirar com anzol o
leviatã\footnote{Animal parecido com um crocodilo (Jó 41.1), ou um
monstro do mar (Sl 104.26), ou uma serpente (Is 27.1). Representa as
forças da desordem e do mal (Sl 74.14).}, ou ligarás a sua língua
com uma corda? Podes pôr um anzol no seu nariz, ou com um gancho
furar a sua queixada? Porventura multiplicará as súplicas para
contigo, ou brandamente falará? Fará ele aliança contigo, ou o
tomarás tu por servo para sempre? Brincarás com ele, como se
fora um passarinho, ou o prenderás para tuas meninas? Os teus
companheiros farão dele um banquete, ou o repartirão entre os
negociantes? Encherás a sua pele de ganchos, ou a sua cabeça com
arpões de pescadores? Põe a tua mão sobre ele, lembra-te da
peleja, e nunca mais tal intentarás. Eis que é vã a esperança de
apanhá-lo; pois não será o homem derrubado só ao vê-lo?
Ninguém há tão atrevido, que a despertá-lo se atreva; quem,
pois, é aquele que ousa erguer-se diante de mim?

Quem primeiro me deu, para que eu haja de retribuir-lhe? Pois o
que está debaixo de todos os céus é meu. Não me calarei a
respeito dos seus membros, nem da sua grande força, nem a graça da
sua compostura. Quem descobrirá a face da sua roupa? Quem
entrará na sua couraça dobrada? Quem abrirá as portas do seu
rosto? Pois ao redor dos seus dentes está o terror. As suas
fortes escamas são o seu orgulho, cada uma fechada como com selo
apertado. Uma à outra se chega tão perto, que nem o ar passa
por entre elas. Umas às outras se ligam; tanto aderem entre
si, que não se podem separar. Cada um dos seus espirros faz
resplandecer a luz, e os seus olhos são como as pálpebras da alva.
Da sua boca saem tochas; faíscas de fogo saltam dela.
Das suas narinas procede fumaça, como de uma panela fervente,
ou de uma grande caldeira. O seu hálito faz
incender\footnote{Fazer arder ou como que fazer arder; pôr fogo a;
inflamar. Fazer corar; afoguear, avermelhar, ruborizar. Fig.
Estimular, entusiasmar; excitar, incitar. Excitar, despertar,
provocar, suscitar.} os carvões; e da sua boca sai chama. No
seu pescoço reside a força; diante dele até a tristeza salta de
prazer. Os músculos da sua carne estão pegados entre si; cada
um está firme nele, e nenhum se move. O seu coração é firme
como uma pedra e firme como a mó de baixo. Levantando-se ele,
tremem os valentes; em razão dos seus abalos se purificam. Se
alguém lhe tocar com a espada, essa não poderá penetrar, nem lança,
dardo ou flecha. Ele considera o ferro como palha, e o cobre
como pau podre. A seta o não fará fugir; as pedras das fundas
se lhe tornam em restolho. As pedras atiradas são para ele
como arestas, e ri-se do brandir da lança; debaixo de si tem
conchas pontiagudas; estende-se sobre coisas pontiagudas como na
lama. As profundezas faz ferver, como uma panela; torna o mar
como uma vasilha de ungüento. Após si deixa uma vereda
luminosa; parece o abismo tornado em brancura de cãs. Na
terra não há coisa que se lhe possa comparar, pois foi feito para
estar sem pavor. Ele vê tudo que é alto; é rei sobre todos os
filhos da soberba.

\medskip

\lettrine{42} Então respondeu Jó ao Senhor, dizendo: Bem
sei eu que tudo podes, e que nenhum dos teus propósitos pode ser
impedido. Quem é este, que sem conhecimento encobre o conselho?
Por isso relatei o que não entendia; coisas que para mim eram
inescrutáveis, e que eu não entendia. Escuta-me, pois, e eu
falarei; eu te perguntarei, e tu me ensinarás. Com o ouvir dos
meus ouvidos ouvi, mas agora te vêem os meus olhos. Por isso me
abomino e me arrependo no pó e na cinza.

Sucedeu que, acabando o Senhor de falar a Jó aquelas palavras, o
Senhor disse a Elifaz, o temanita: A minha ira se acendeu contra ti,
e contra os teus dois amigos, porque não falastes de mim o que era
reto, como o meu servo Jó. Tomai, pois, sete bezerros e sete
carneiros, e ide ao meu servo Jó, e oferecei holocaustos por vós, e
o meu servo Jó orará por vós; porque deveras a ele aceitarei, para
que eu não vos trate conforme a vossa loucura; porque vós não
falastes de mim o que era reto como o meu servo Jó. Então foram
Elifaz, o temanita, e Bildade, o suíta, e Zofar, o naamatita, e
fizeram como o Senhor lhes dissera; e o Senhor aceitou a face de Jó.

E o Senhor virou o cativeiro de Jó, quando orava pelos seus
amigos; e o Senhor acrescentou, em dobro, a tudo quanto Jó antes
possuía. Então vieram a ele todos os seus irmãos, e todas as
suas irmãs, e todos quantos dantes o conheceram, e comeram com ele
pão em sua casa, e se condoeram dele, e o consolaram acerca de todo
o mal que o Senhor lhe havia enviado; e cada um deles lhe deu uma
peça de dinheiro, e um pendente de ouro. E assim abençoou o
Senhor o último estado de Jó, mais do que o primeiro; pois teve
catorze mil ovelhas, e seis mil camelos, e mil juntas de bois, e mil
jumentas. Também teve sete filhos e três filhas. E
chamou o nome da primeira Jemima, e o nome da segunda Quezia, e o
nome da terceira Quéren-Hapuque. E em toda a terra não se
acharam mulheres tão formosas como as filhas de Jó; e seu pai lhes
deu herança entre seus irmãos. E depois disto viveu Jó cento
e quarenta anos; e viu a seus filhos, e aos filhos de seus filhos,
até à quarta geração. Então morreu Jó, velho e farto de dias.

