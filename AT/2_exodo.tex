\addchap{Êxodo}

\lettrine{1} Estes pois são os nomes dos filhos de Israel, que
entraram no Egito com Jacó; cada um entrou com sua casa: Rúben,
Simeão, Levi, e Judá; Issacar, Zebulom, e Benjamim; Dã e
Naftali, Gade e Aser. Todas as almas, pois, que procederam dos
lombos de Jacó, foram \textbf{setenta almas}; José, porém, estava no
Egito. Faleceu José, e todos os seus irmãos, e toda aquela
geração. E os filhos de Israel frutificaram, aumentaram muito, e
multiplicaram-se, e foram fortalecidos grandemente; de maneira que a
terra se encheu deles.

E levantou-se um novo rei sobre o Egito, que não conhecera a José;
o qual disse ao seu povo: Eis que o povo dos filhos de Israel é
muito, e mais poderoso do que nós. Eia, usemos de sabedoria
para com eles, para que não se multipliquem, e aconteça que, vindo
guerra, eles também se ajuntem com os nossos inimigos, e pelejem
contra nós, e subam da terra. E puseram sobre eles maiorais
de tributos, para os afligirem com suas cargas. Porque edificaram a
Faraó cidades-armazéns, Pitom e Ramessés. Mas quanto mais os
afligiam, tanto mais se multiplicavam, e tanto mais cresciam; de
maneira que se enfadavam por causa dos filhos de Israel. E os
egípcios faziam servir os filhos de Israel com dureza; assim
que lhes fizeram amargar a vida com dura servidão, em barro e em
tijolos, e com todo o trabalho no campo; com todo o seu serviço, em
que os obrigavam com dureza.

E o rei do Egito falou às parteiras das hebréias (das quais o
nome de uma era Sifrá, e o da outra Puá), e disse: Quando
ajudardes a dar à luz às hebréias, e as virdes sobre os assentos, se
for filho, matai-o; mas se for filha, então viva. As
parteiras, porém, temeram a Deus e não fizeram como o rei do Egito
lhes dissera, antes conservavam os meninos com vida. Então o
rei do Egito chamou as parteiras e disse-lhes: Por que fizestes
isto, deixando os meninos com vida? E as parteiras disseram a
Faraó: É que as mulheres hebréias não são como as egípcias; porque
são vivas, e já têm dado à luz antes que a parteira venha a elas.
Portanto Deus fez bem às parteiras. E o povo se aumentou, e
se fortaleceu muito. E aconteceu que, como as parteiras
temeram a Deus, ele estabeleceu-lhes casas. Então ordenou
Faraó a todo o seu povo, dizendo: A todos os filhos que nascerem
lançareis no rio, mas a todas as filhas guardareis com vida.

\medskip

\lettrine{2} E foi um homem da casa de Levi e casou com uma
filha de Levi. E a mulher concebeu e deu à luz um filho; e,
vendo que ele era formoso, escondeu-o três meses. Não podendo,
porém, mais escondê-lo, tomou uma arca de juncos, e a revestiu com
barro e betume; e, pondo nela o menino, a pôs nos juncos à margem do
rio. E sua irmã postou-se de longe, para saber o que lhe havia
de acontecer.

E a filha de Faraó desceu a lavar-se no rio, e as suas donzelas
passeavam, pela margem do rio; e ela viu a arca no meio dos juncos,
e enviou a sua criada, que a tomou. E abrindo-a, viu ao menino e
eis que o menino chorava; e moveu-se de compaixão dele, e disse: Dos
meninos dos hebreus é este. Então disse sua irmã à filha de
Faraó: Irei chamar uma ama das hebréias, que crie este menino para
ti? E a filha de Faraó disse-lhe: Vai. Foi, pois, a moça, e
chamou a mãe do menino. Então lhe disse a filha de Faraó: Leva
este menino, e cria-mo; eu te darei teu salário. E a mulher tomou o
menino, e criou-o. E, quando o menino já era grande, ela o
trouxe à filha de Faraó, a qual o adotou; e chamou-lhe
\textbf{Moisés}, e disse: Porque das águas o tenho tirado.

E aconteceu naqueles dias que, sendo Moisés já homem, saiu a seus
irmãos, e atentou para as suas cargas; e viu que um egípcio feria a
um hebreu, homem de seus irmãos. E olhou a um e a outro lado
e, vendo que não havia ninguém ali, matou ao egípcio, e escondeu-o
na areia. E tornou a sair no dia seguinte, e eis que dois
homens hebreus contendiam; e disse ao injusto: Por que feres a teu
próximo? O qual disse: Quem te tem posto a ti por maioral e
juiz sobre nós? Pensas matar-me, como mataste o egípcio? Então temeu
Moisés, e disse: Certamente este negócio foi descoberto.
Ouvindo, pois, Faraó este caso, procurou matar a Moisés; mas
Moisés fugiu de diante da face de Faraó, e habitou na terra de
Midiã, e assentou-se junto a um poço.

E o sacerdote de Midiã tinha sete filhas, as quais vieram tirar
água, e encheram os bebedouros, para dar de beber ao rebanho de seu
pai. Então vieram os pastores, e expulsaram-nas dali; Moisés,
porém, levantou-se e defendeu-as, e deu de beber ao rebanho.
E voltando elas a Reuel seu pai, ele disse: Por que hoje
tornastes tão depressa? E elas disseram: Um homem egípcio nos
livrou da mão dos pastores; e também nos tirou água em abundância, e
deu de beber ao rebanho. E disse a suas filhas: E onde está
ele? Por que deixastes o homem? Chamai-o para que coma pão. E
Moisés consentiu em morar com aquele homem; e ele deu a Moisés sua
filha Zípora, a qual deu à luz um filho, a quem ele chamou
\textbf{Gérson}, porque disse: Peregrino fui em terra estranha.

E aconteceu, depois de muitos dias, que morrendo o rei do Egito,
os filhos de Israel suspiraram por causa da servidão, e clamaram; e
o seu clamor subiu a Deus por causa de sua servidão. E ouviu
Deus o seu gemido, e lembrou-se Deus da sua aliança com Abraão, com
Isaque, e com Jacó; e viu Deus os filhos de Israel, e atentou
Deus para a sua condição.

\medskip

\lettrine{3} E apascentava Moisés o rebanho de Jetro, seu
sogro, sacerdote em Midiã; e levou o rebanho atrás do deserto, e
chegou ao monte de Deus, a Horebe. E \textbf{apareceu-lhe o anjo
do Senhor} em uma chama de fogo do meio duma sarça; e olhou, e eis
que a sarça ardia no fogo, e a sarça não se consumia. E Moisés
disse: Agora me virarei para lá, e verei esta grande visão, porque a
sarça não se queima. E vendo o Senhor que se virava para ver,
bradou Deus a ele do meio da sarça, e disse: Moisés, Moisés.
Respondeu ele: Eis-me aqui. E disse: Não te chegues para cá;
tira os sapatos de teus pés; porque o lugar em que tu estás é terra
santa. Disse mais: Eu sou o Deus de teu pai, o Deus de Abraão, o
Deus de Isaque, e o Deus de Jacó. E Moisés encobriu o seu rosto,
porque temeu olhar para Deus.

E disse o Senhor: Tenho visto atentamente a aflição do meu povo,
que está no Egito, e tenho ouvido o seu clamor por causa dos seus
exatores\footnote{Exator: Que ou o que exige o que lhe é devido. Que
ou o que cobra tributos, impostos; que ou aquele que faz exações.
Relativo a ou executor de justiça.}, porque conheci as suas dores.
Portanto desci para livrá-lo da mão dos egípcios, e para fazê-lo
subir daquela terra, a uma terra boa e larga, a uma terra que mana
leite e mel; ao lugar do cananeu, e do heteu, e do amorreu, e do
perizeu, e do heveu, e do jebuseu. E agora, eis que o clamor dos
filhos de Israel é vindo a mim, e também tenho visto a opressão com
que os egípcios os oprimem. Vem agora, pois, e eu te enviarei
a Faraó para que tires o meu povo (os filhos de Israel) do Egito.
Então Moisés disse a Deus: Quem sou eu, que vá a Faraó e tire
do Egito os filhos de Israel? E disse: \textbf{Certamente eu
serei contigo}; e isto te será por sinal de que eu te enviei: Quando
houveres tirado este povo do Egito, servireis a Deus neste monte.
Então disse Moisés a Deus: Eis que quando eu for aos filhos
de Israel, e lhes disser: O Deus de vossos pais me enviou a vós; e
eles me disserem: Qual é o seu nome? Que lhes direi? E disse
Deus a Moisés: \textbf{EU SOU O QUE SOU}. Disse mais: Assim dirás
aos filhos de Israel: EU SOU me enviou a vós. E Deus disse
mais a Moisés: Assim dirás aos filhos de Israel: O Senhor Deus de
vossos pais, o Deus de Abraão, o Deus de Isaque, e o Deus de Jacó,
me enviou a vós; este é meu nome eternamente, e este é meu memorial
de geração em geração.

Vai, e ajunta os anciãos de Israel e dize-lhes: O Senhor Deus de
vossos pais, o Deus de Abraão, de Isaque e de Jacó, me apareceu,
dizendo: Certamente vos tenho visitado e visto o que vos é feito no
Egito. Portanto eu disse: Far-vos-ei subir da aflição do
Egito à terra do cananeu, do heteu, do amorreu, do perizeu, do heveu
e do jebuseu, a uma terra que mana leite e mel. E ouvirão a
tua voz; e irás, tu com os anciãos de Israel, ao rei do Egito, e
dir-lhe-eis: O Senhor Deus dos hebreus nos encontrou. Agora, pois,
deixa-nos ir caminho de três dias para o deserto, para que
sacrifiquemos ao Senhor nosso Deus. Eu sei, porém, que o rei
do Egito não vos deixará ir, nem ainda por uma mão forte.
Porque eu estenderei a minha mão, e ferirei ao Egito com
todas as minhas maravilhas que farei no meio dele; depois vos
deixará ir. E eu darei graça a este povo aos olhos dos
egípcios; e acontecerá que, quando sairdes, não saireis vazios,
porque cada mulher pedirá à sua vizinha e à sua hóspeda jóias
de prata, e jóias de ouro, e vestes, as quais poreis sobre vossos
filhos e sobre vossas filhas; e despojareis os egípcios.

\medskip

\lettrine{4} Então respondeu Moisés, e disse: Mas eis que não
me crerão, nem ouvirão a minha voz, porque dirão: O Senhor não te
apareceu. E o Senhor disse-lhe: Que é isso na tua mão? E ele
disse: Uma vara. E ele disse: Lança-a na terra. Ele a lançou na
terra, e tornou-se em cobra; e Moisés fugia dela. Então disse o
Senhor a Moisés: Estende a tua mão e pega-lhe pela cauda. E estendeu
sua mão, e pegou-lhe pela cauda, e tornou-se em vara na sua mão;
para que creiam que te apareceu o Senhor Deus de seus pais, o
Deus de Abraão, o Deus de Isaque e o Deus de Jacó. E disse-lhe
mais o Senhor: Põe agora a tua mão no teu seio. E, tirando-a, eis
que a sua mão estava leprosa, branca como a neve. E disse: Torna
a por a tua mão no teu seio. E tornou a colocar sua mão no seu seio;
depois tirou-a do seu seio, e eis que se tornara como a sua carne.
E acontecerá que, se eles não te crerem, nem ouvirem a voz do
primeiro sinal, crerão à voz do derradeiro sinal; e se acontecer
que ainda não creiam a estes dois sinais, nem ouvirem a tua voz,
tomarás das águas do rio, e as derramarás na terra seca; e as águas,
que tomarás do rio, tornar-se-ão em sangue sobre a terra seca.

Então disse Moisés ao Senhor: Ah, meu Senhor! eu não sou homem
eloqüente, nem de ontem nem de anteontem, nem ainda desde que tens
falado ao teu servo; porque sou pesado de boca e pesado de língua.
E disse-lhe o Senhor: Quem fez a boca do homem? ou quem fez o
mudo, ou o surdo, ou o que vê, ou o cego? Não sou eu, o Senhor?
Vai, pois, agora, e eu serei com a tua boca e te ensinarei o
que hás de falar. Ele, porém, disse: Ah, meu Senhor! Envia
pela mão daquele a quem tu hás de enviar. Então se acendeu a
ira do Senhor contra Moisés, e disse: Não é Arão, o levita, teu
irmão? Eu sei que ele falará muito bem; e eis que ele também sai ao
teu encontro; e, vendo-te, se alegrará em seu coração. E tu
lhe falarás, e porás as palavras na sua boca; e eu serei com a tua
boca, e com a dele, ensinando-vos o que haveis de fazer. E
ele falará por ti ao povo; e acontecerá que ele te será por boca, e
tu lhe serás por Deus. Toma, pois, esta vara na tua mão, com
que farás os sinais.

Então foi Moisés, e voltou para Jetro, seu sogro, e disse-lhe: Eu
irei agora, e tornarei a meus irmãos, que estão no Egito, para ver
se ainda vivem. Disse, pois, Jetro a Moisés: Vai em paz.
Disse também o Senhor a Moisés em Midiã: Vai, volta para o
Egito; porque todos os que buscavam a tua alma morreram.
Tomou, pois, Moisés sua mulher e seus filhos, e os levou
sobre um jumento, e tornou à terra do Egito; e Moisés tomou a vara
de Deus na sua mão. E disse o Senhor a Moisés: Quando
voltares ao Egito, atenta que faças diante de Faraó todas as
maravilhas que tenho posto na tua mão; mas eu lhe endurecerei o
coração, para que não deixe ir o povo. Então dirás a Faraó:
Assim diz o Senhor: \textbf{Israel é meu filho, meu primogênito.}
E eu te tenho dito: Deixa ir o meu filho, para que me sirva;
mas tu recusaste deixá-lo ir; eis que eu matarei a teu filho, o teu
primogênito.

E aconteceu no caminho, numa estalagem, que o Senhor o encontrou,
e o quis matar. Então Zípora tomou uma pedra aguda, e
circuncidou o prepúcio de seu filho, e lançou-o a seus pés, e disse:
Certamente me és um esposo sanguinário. E desviou-se dele.
Então ela disse: Esposo sanguinário, por causa da circuncisão.
Disse o Senhor a Arão: Vai ao deserto, ao encontro de Moisés.
E ele foi, e encontrou-o no \textbf{monte de Deus}, e beijou-o.
E relatou Moisés a Arão todas as palavras do Senhor, com que
o enviara, e todos os sinais que lhe mandara. Então foram
Moisés e Arão, e ajuntaram todos os anciãos dos filhos de Israel.
E Arão falou todas as palavras que o Senhor falara a Moisés e
fez os sinais perante os olhos do povo. E o povo creu; e
quando ouviram que o Senhor visitava aos filhos de Israel, e que via
a sua aflição, inclinaram-se, e adoraram.

\medskip

\lettrine{5} E depois foram Moisés e Arão e disseram a Faraó:
Assim diz o Senhor Deus de Israel: Deixa ir o meu povo, para que me
celebre uma festa no deserto. Mas Faraó disse: Quem é o Senhor,
cuja voz eu ouvirei, para deixar ir Israel? Não conheço o Senhor,
nem tampouco deixarei ir Israel.

E eles disseram: O Deus dos hebreus nos encontrou; portanto
deixa-nos agora ir caminho de três dias ao deserto, para que
ofereçamos sacrifícios ao Senhor nosso Deus, e ele não venha sobre
nós com pestilência ou com espada. Então disse-lhes o rei do
Egito: Moisés e Arão, por que fazeis cessar o povo das suas obras?
Ide às vossas cargas. E disse também Faraó: Eis que o povo da
terra já é muito, e vós os fazeis abandonar as suas cargas.
Portanto deu ordem Faraó, naquele mesmo dia, aos exatores do
povo, e aos seus oficiais, dizendo: Daqui em diante não torneis
a dar palha ao povo, para fazer tijolos, como fizestes antes: vão
eles mesmos, e colham palha para si. E lhes imporeis a conta dos
tijolos que fizeram antes; nada diminuireis dela, porque eles estão
ociosos; por isso clamam, dizendo: Vamos, sacrifiquemos ao nosso
Deus. Agrave-se o serviço sobre estes homens, para que se ocupem
nele e não confiem em palavras mentirosas.

Então saíram os exatores do povo, e seus oficiais, e falaram ao
povo, dizendo: Assim diz Faraó: Eu não vos darei palha; ide
vós mesmos, e tomai vós palha onde a achardes; porque nada se
diminuirá de vosso serviço. Então o povo se espalhou por toda
a terra do Egito, a colher restolho em lugar de palha. E os
exatores os apertavam, dizendo: Acabai vossa obra, a tarefa de cada
dia, como quando havia palha. E foram açoitados os oficiais
dos filhos de Israel, que os exatores de Faraó tinham posto sobre
eles, dizendo estes: Por que não acabastes vossa tarefa, fazendo
tijolos como antes, assim também ontem e hoje?

Por isso, os oficiais dos filhos de Israel, foram e clamaram a
Faraó, dizendo: Por que fazes assim a teus servos? Palha não
se dá a teus servos, e nos dizem: Fazei tijolos; e eis que teus
servos são açoitados; porém o teu povo tem a culpa. Mas ele
disse: Vós sois ociosos; vós sois ociosos; por isso dizeis: Vamos,
sacrifiquemos ao Senhor. Ide, pois, agora, trabalhai; palha
porém não se vos dará; contudo, dareis a conta dos tijolos.
Então os oficiais dos filhos de Israel viram-se em aflição,
porquanto se dizia: Nada diminuireis de vossos tijolos, da tarefa do
dia no seu dia. E encontraram a Moisés e a Arão, que estavam
defronte deles, quando saíram de Faraó. E disseram-lhes: O
Senhor atente sobre vós, e julgue isso, porquanto fizestes o nosso
caso repelente diante de Faraó, e diante de seus servos, dando-lhes
a espada nas mãos, para nos matar. Então, tornando-se Moisés
ao Senhor, disse: Senhor! por que fizeste mal a este povo? por que
me enviaste? Porque desde que me apresentei a Faraó para
falar em teu nome, ele maltratou a este povo; e de nenhuma sorte
livraste o teu povo.

\medskip

\lettrine{6} Então disse o Senhor a Moisés: Agora verás o que
hei de fazer a Faraó; porque por uma mão poderosa os deixará ir,
sim, por uma mão poderosa os lançará de sua terra. Falou mais
Deus a Moisés, e disse: \textbf{Eu sou o Senhor.} E eu apareci a
Abraão, a Isaque, e a Jacó, como o Deus Todo-Poderoso; mas pelo meu
nome, o Senhor, não lhes fui perfeitamente conhecido. E também
\textbf{estabeleci a minha aliança com eles, para dar-lhes a terra
de Canaã}, a terra de suas peregrinações, na qual foram peregrinos.
E também tenho ouvido o gemido dos filhos de Israel, aos quais
os egípcios fazem servir, e lembrei-me da minha aliança.
Portanto dize aos filhos de Israel: Eu sou o Senhor, e vos
tirarei de debaixo das cargas dos egípcios, e vos livrarei da
servidão, e vos resgatarei com braço estendido e com grandes juízos.
E eu vos tomarei por meu povo, e serei vosso Deus; e sabereis
que eu sou o Senhor vosso Deus, que vos tiro de debaixo das cargas
dos egípcios; e \textbf{eu vos levarei à terra}, acerca da qual
levantei minha mão, jurando que a daria a Abraão, a Isaque e a Jacó,
\textbf{e vo-la darei por herança, eu o Senhor}. Deste modo
falou Moisés aos filhos de Israel, mas eles não ouviram a Moisés,
por causa da angústia de espírito e da dura servidão.

Falou mais o Senhor a Moisés, dizendo: Entra, e fala a
Faraó rei do Egito, que deixe sair os filhos de Israel da sua terra.
Moisés, porém, falou perante o Senhor, dizendo: Eis que os
filhos de Israel não me têm ouvido; como, pois, Faraó me ouvirá?
Também eu sou incircunciso de lábios. Todavia o Senhor falou
a Moisés e a Arão, e deu-lhes mandamento para os filhos de Israel, e
para Faraó rei do Egito, para que tirassem os filhos de Israel da
terra do Egito.

Estas são as cabeças das casas de seus pais: Os filhos de Rúben,
o primogênito de Israel: Enoque e Palu, Hezrom e Carmi; estas são as
famílias de Rúben. E os filhos de Simeão: Jemuel, Jamin,
Oade, Jaquim, Zoar e Saul, filho de uma cananéia; estas são as
famílias de Simeão. E estes são os nomes dos filhos de Levi,
segundo as suas gerações: Gérson, Coate e Merari; e os anos da vida
de Levi foram cento e trinta e sete anos. Os filhos de
Gérson: Libni e Simei, segundo as suas famílias; e os filhos
de Coate: Anrão, Izar, Hebrom e Uziel; e os anos da vida de Coate
foram cento e trinta e três anos. E os filhos de Merari: Mali
e Musi; estas são as famílias de Levi, segundo as suas gerações.
E Anrão tomou por mulher a Joquebede, sua tia, e ela deu-lhe
Arão e Moisés: e os anos da vida de Anrão foram cento e trinta e
sete anos. E os filhos de Izar: Corá, Nefegue e Zicri.
E os filhos de Uziel: Misael, Elzafã e Sitri. E Arão
tomou por mulher a Eliseba, filha de Aminadabe, irmã de Naasson; e
ela deu-lhe Nadabe, Abiú, Eleazar e Itamar. E os filhos de
Corá: Assir, Elcana e Abiasafe; estas são as famílias dos coraítas.
E Eleazar, filho de Arão, tomou por mulher uma das filhas de
Putiel, e ela deu-lhe a Finéias; estes são os cabeças dos pais dos
levitas, segundo as suas famílias. Estes são Arão e Moisés,
aos quais o Senhor disse: Tirai os filhos de Israel da terra do
Egito, segundo os seus exércitos. Estes são os que falaram a
Faraó, rei do Egito, para que tirasse do Egito os filhos de Israel;
estes são Moisés e Arão. E aconteceu que naquele dia, quando
o Senhor falou a Moisés na terra do Egito, falou o Senhor a
Moisés, dizendo: Eu sou o Senhor; fala a Faraó, rei do Egito, tudo
quanto eu te digo. Então disse Moisés perante o Senhor: Eis
que eu sou incircunciso de lábios; como, pois, Faraó me ouvirá?

\medskip

\lettrine{7} Então disse o Senhor a Moisés: Eis que te tenho
posto por deus sobre Faraó, e Arão, teu irmão, será o teu profeta.
Tu falarás tudo o que eu te mandar; e Arão, teu irmão, falará a
Faraó, que deixe ir os filhos de Israel da sua terra. Eu, porém,
endurecerei o coração de Faraó, e multiplicarei na terra do Egito os
meus sinais e as minhas maravilhas. Faraó, pois, não vos ouvirá;
e eu porei minha mão sobre o Egito, e tirarei meus exércitos, meu
povo, os filhos de Israel, da terra do Egito, com grandes juízos.
Então os egípcios saberão que eu sou o Senhor, quando estender a
minha mão sobre o Egito, e tirar os filhos de Israel do meio deles.
Assim fizeram Moisés e Arão; como o Senhor lhes ordenara, assim
fizeram. \textbf{E Moisés era da idade de oitenta anos, e Arão
da idade de oitenta e três anos quando falaram a Faraó}.

E o Senhor falou a Moisés e a Arão, dizendo: Quando Faraó vos
falar, dizendo: Fazei vós um milagre, dirás a Arão: Toma a tua vara,
e lança-a diante de Faraó; e se tornará em serpente. Então
Moisés e Arão foram a Faraó, e fizeram assim como o Senhor ordenara;
e lançou Arão a sua vara diante de Faraó, e diante dos seus servos,
e tornou-se em serpente. E Faraó também chamou os sábios e
encantadores; e os magos do Egito fizeram também o mesmo com os seus
encantamentos. Porque cada um lançou sua vara, e tornaram-se
em serpentes; mas a vara de Arão tragou as varas deles. Porém
o coração de Faraó se endureceu, e não os ouviu, como o Senhor tinha
falado.

Então disse o Senhor a Moisés: O coração de Faraó está
endurecido, recusa deixar ir o povo. Vai pela manhã a Faraó;
eis que ele sairá às águas; põe-te em frente dele na beira do rio, e
tomarás em tua mão a vara que se tornou em cobra. E lhe
dirás: O Senhor Deus dos hebreus me tem enviado a ti, dizendo: Deixa
ir o meu povo, para que me sirva no deserto; porém eis que até agora
não tens ouvido. Assim diz o Senhor: Nisto saberás que eu sou
o Senhor: Eis que eu com esta vara, que tenho em minha mão, ferirei
as águas que estão no rio, e tornar-se-ão em sangue. E os
peixes, que estão no rio, morrerão, e o rio cheirará mal; e os
egípcios terão nojo de beber da água do rio. Disse mais o
Senhor a Moisés: Dize a Arão: Toma tua vara, e estende a tua mão
sobre as águas do Egito, sobre as suas correntes, sobre os seus
rios, e sobre os seus tanques, e sobre todo o ajuntamento das suas
águas, para que se tornem em sangue; e haja sangue em toda a terra
do Egito, assim nos vasos de madeira como nos de pedra. E
Moisés e Arão fizeram assim como o Senhor tinha mandado; e Arão
levantou a vara, e feriu as águas que estavam no rio, diante dos
olhos de Faraó, e diante dos olhos de seus servos; e todas as águas
do rio se tornaram em sangue, e os peixes, que estavam no
rio, morreram, e o rio cheirou mal, e os egípcios não podiam beber a
água do rio; e houve sangue por toda a terra do Egito. Porém
os magos do Egito também fizeram o mesmo com os seus encantamentos;
de modo que o coração de Faraó se endureceu, e não os ouviu, como o
Senhor tinha dito. E virou-se Faraó, e foi para sua casa; nem
ainda nisto pôs seu coração. E todos os egípcios cavaram
poços junto ao rio, para beberem água; porquanto não podiam beber da
água do rio. Assim se cumpriram sete dias, depois que o
Senhor ferira o rio.

\medskip

\lettrine{8} Depois disse o Senhor a Moisés: Vai a Faraó e
dize-lhe: Assim diz o Senhor: Deixa ir o meu povo, para que me
sirva. E se recusares deixá-lo ir, eis que ferirei com rãs todos
os teus termos. E o rio criará rãs, que subirão e virão à tua
casa, e ao teu dormitório, e sobre a tua cama, e as casas dos teus
servos, e sobre o teu povo, e aos teus fornos, e às tuas
amassadeiras. E as rãs subirão sobre ti, e sobre o teu povo, e
sobre todos os teus servos. Disse mais o Senhor a Moisés: Dize a
Arão: Estende a tua mão com tua vara sobre as correntes, e sobre os
rios, e sobre os tanques, e faze subir rãs sobre a terra do Egito.
E Arão estendeu a sua mão sobre as águas do Egito, e subiram
rãs, e cobriram a terra do Egito. Então os magos fizeram o mesmo
com os seus encantamentos, e fizeram subir rãs sobre a terra do
Egito. E Faraó chamou a Moisés e a Arão, e disse: Rogai ao
Senhor que tire as rãs de mim e do meu povo; depois deixarei ir o
povo, para que sacrifiquem ao Senhor. E disse Moisés a Faraó:
Digna-te dizer-me quando é que hei de rogar por ti, e pelos teus
servos, e por teu povo, para tirar as rãs de ti, e das tuas casas, e
fiquem somente no rio? E ele disse: Amanhã. E Moisés disse:
Seja conforme à tua palavra, para que saibas que ninguém há como o
Senhor nosso Deus. E as rãs apartar-se-ão de ti, das tuas
casas, dos teus servos, e do teu povo; somente ficarão no rio.
Então saíram Moisés e Arão da presença de Faraó; e Moisés
clamou ao Senhor por causa das rãs que tinha posto sobre Faraó.
E o Senhor fez conforme a palavra de Moisés; e as rãs
morreram nas casas, nos pátios, e nos campos. E ajuntaram-se
em montões, e a terra cheirou mal. Vendo, pois, Faraó que
havia descanso, endureceu o seu coração, e não os ouviu, como o
Senhor tinha dito.

Disse mais o Senhor a Moisés: Dize a Arão: Estende a tua vara, e
fere o pó da terra, para que se torne em piolhos por toda a terra do
Egito. E fizeram assim; e Arão estendeu a sua mão com a sua
vara, e feriu o pó da terra, e havia muitos piolhos nos homens e no
gado; todo o pó da terra se tornou em piolhos em toda a terra do
Egito. E os magos fizeram também assim com os seus
encantamentos para produzir piolhos, mas não puderam; e havia
piolhos nos homens e no gado. Então \textbf{disseram os magos
a Faraó: Isto é o dedo de Deus}. Porém o coração de Faraó se
endureceu, e não os ouvia, como o Senhor tinha dito.

Disse mais o Senhor a Moisés: Levanta-te pela manhã cedo e põe-te
diante de Faraó; eis que ele sairá às águas; e dize-lhe: Assim diz o
Senhor: Deixa ir o meu povo, para que me sirva. Porque se não
deixares ir o meu povo, eis que enviarei enxames de moscas sobre ti,
e sobre os teus servos, e sobre o teu povo, e às tuas casas; e as
casas dos egípcios se encherão destes enxames, e também a terra em
que eles estiverem. E naquele dia eu separarei a terra de
Gósen, em que meu povo habita, que nela não haja enxames de moscas,
para que saibas que eu sou o Senhor no meio desta terra. E
\textbf{porei separação entre o meu povo e o teu povo}; amanhã se
fará este sinal. E o Senhor fez assim; e vieram grandes
enxames de moscas à casa de Faraó e às casas dos seus servos, e
sobre toda a terra do Egito; a terra foi corrompida destes enxames.
Então chamou Faraó a Moisés e a Arão, e disse: Ide, e
sacrificai ao vosso Deus nesta terra. E Moisés disse: Não
convém que façamos assim, porque sacrificaríamos ao Senhor nosso
Deus a abominação dos egípcios; eis que se sacrificássemos a
abominação dos egípcios perante os seus olhos, não nos apedrejariam
eles? Deixa-nos ir caminho de três dias ao deserto, para que
sacrifiquemos ao Senhor nosso Deus, como ele nos disser.
Então disse Faraó: Deixar-vos-ei ir, para que sacrifiqueis ao
Senhor vosso Deus no deserto; somente que, indo, não vades longe;
orai também por mim. E Moisés disse: Eis que saio de ti, e
orarei ao Senhor, que estes enxames de moscas se retirem amanhã de
Faraó, dos seus servos, e do seu povo; somente que Faraó não mais me
engane, não deixando ir a este povo para sacrificar ao Senhor.
Então saiu Moisés da presença de Faraó, e orou ao Senhor.
E fez o Senhor conforme a palavra de Moisés, e os enxames de
moscas se retiraram de Faraó, dos seus servos, e do seu povo; não
ficou uma só. Mas endureceu Faraó ainda esta vez seu coração,
e não deixou ir o povo.

\medskip

\lettrine{9} Depois o Senhor disse a Moisés: Vai a Faraó, e
dize-lhe: Assim diz o Senhor Deus dos hebreus: Deixa ir o meu povo,
para que me sirva. Porque se recusares deixá-los ir, e ainda por
força os detiveres, eis que a mão do Senhor será sobre teu gado,
que está no campo, sobre os cavalos, sobre os jumentos, sobre os
camelos, sobre os bois, e sobre as ovelhas, com pestilência
gravíssima. E o Senhor fará separação entre o gado dos
israelitas e o gado dos egípcios, para que nada morra de tudo o que
for dos filhos de Israel. E o Senhor assinalou certo tempo,
dizendo: Amanhã fará o Senhor esta coisa na terra. E o Senhor
fez isso no dia seguinte, e todo o gado dos egípcios morreu; porém
do gado dos filhos de Israel não morreu nenhum. E Faraó enviou a
ver, e eis que do gado de Israel não morrera nenhum; porém o coração
de Faraó se agravou, e não deixou ir o povo.

Então disse o Senhor a Moisés e a Arão: Tomai vossas mãos cheias
de cinza do forno, e Moisés a espalhe para o céu diante dos olhos de
Faraó; e tornar-se-á em pó miúdo sobre toda a terra do Egito, e
se tornará em sarna, que arrebente em úlceras, nos homens e no gado,
por toda a terra do Egito. E eles tomaram a cinza do forno, e
puseram-se diante de Faraó, e Moisés a espalhou para o céu; e
tornou-se em sarna, que arrebentava em úlceras nos homens e no gado;
de maneira que os magos não podiam parar diante de Moisés,
por causa da sarna; porque havia sarna nos magos, e em todos os
egípcios. Porém o Senhor endureceu o coração de Faraó, e não
os ouviu, como o Senhor tinha dito a Moisés.

Então disse o Senhor a Moisés: Levanta-te pela manhã cedo, e
põe-te diante de Faraó, e dize-lhe: Assim diz o Senhor Deus dos
hebreus: Deixa ir o meu povo, para que me sirva; porque esta
vez enviarei todas as minhas pragas sobre o teu coração, e sobre os
teus servos, e sobre o teu povo, para que saibas que não há outro
como eu em toda a terra. Porque agora tenho estendido minha
mão, para te ferir a ti e ao teu povo com pestilência, e para que
sejas destruído da terra; mas, deveras, para isto te mantive,
para mostrar meu poder em ti, e para que o meu nome seja anunciado
em toda a terra. Tu ainda te exaltas contra o meu povo, para
não o deixar ir? Eis que amanhã por este tempo farei chover
saraiva mui grave, qual nunca houve no Egito, desde o dia em que foi
fundado até agora. Agora, pois, envia, recolhe o teu gado, e
tudo o que tens no campo; todo o homem e animal, que for achado no
campo, e não for recolhido à casa, a saraiva cairá sobre eles, e
morrerão. Quem dos servos de Faraó temia a palavra do Senhor,
fez fugir os seus servos e o seu gado para as casas; mas
aquele que não tinha considerado a palavra do Senhor deixou os seus
servos e o seu gado no campo.

Então disse o Senhor a Moisés: Estende a tua mão para o céu, e
haverá saraiva em toda a terra do Egito, sobre os homens e sobre o
gado, e sobre toda a erva do campo, na terra do Egito. E
Moisés estendeu a sua vara para o céu, e o Senhor deu trovões e
saraiva, e fogo corria pela terra; e o Senhor fez chover saraiva
sobre a terra do Egito. E havia saraiva, e fogo misturado
entre a saraiva, tão grave, qual nunca houve em toda a terra do
Egito desde que veio a ser uma nação. E a saraiva feriu, em
toda a terra do Egito, tudo quanto havia no campo, desde os homens
até aos animais; também a saraiva feriu toda a erva do campo, e
quebrou todas as árvores do campo. Somente na terra de Gósen,
onde estavam os filhos de Israel, não havia saraiva. Então
Faraó mandou chamar a Moisés e a Arão, e disse-lhes: Esta vez
pequei; o Senhor é justo, mas eu e o meu povo ímpios. Orai ao
Senhor (pois que basta) para que não haja mais trovões de Deus nem
saraiva; e eu vos deixarei ir, e não ficareis mais aqui.
Então lhe disse Moisés: Em saindo da cidade estenderei minhas
mãos ao Senhor; os trovões cessarão, e não haverá mais saraiva; para
que saibas que a terra é do Senhor. Todavia, quanto a ti e
aos teus servos, eu sei que ainda não temereis diante do Senhor
Deus. E o linho e a cevada foram feridos, porque a cevada já
estava na espiga, e o linho na haste. Mas o trigo e o centeio
não foram feridos, porque estavam cobertos. Saiu, pois,
Moisés da presença de Faraó, da cidade, e estendeu as suas mãos ao
Senhor; e cessaram os trovões e a saraiva, e a chuva não caiu mais
sobre a terra. Vendo Faraó que cessou a chuva, e a saraiva, e
os trovões, pecou ainda mais; e endureceu o seu coração, ele e os
seus servos. Assim o coração de Faraó se endureceu, e não
deixou ir os filhos de Israel, como o Senhor tinha dito por Moisés.

\medskip

\lettrine{10} Depois disse o Senhor a Moisés: Vai a Faraó,
porque tenho endurecido o seu coração, e o coração de seus servos,
para fazer estes meus sinais no meio deles, e para que contes
aos ouvidos de teus filhos, e dos filhos de teus filhos, as coisas
que fiz no Egito, e os meus sinais, que tenho feito entre eles; para
que saibais que eu sou o Senhor. Assim foram Moisés e Arão a
Faraó, e disseram-lhe: Assim diz o Senhor Deus dos hebreus: Até
quando recusarás humilhar-te diante de mim? Deixa ir o meu povo para
que me sirva; porque se ainda recusares deixar ir o meu povo,
eis que trarei amanhã gafanhotos aos teus termos. E cobrirão a
face da terra, de modo que não se poderá ver a terra; e eles comerão
o restante que escapou, o que vos ficou da saraiva; também comerão
toda a árvore que vos cresce no campo; e encherão as tuas casas,
e as casas de todos os teus servos e as casas de todos os egípcios,
quais nunca viram teus pais, nem os pais de teus pais, desde o dia
em que se acharam na terra até o dia de hoje. E virou-se, e saiu da
presença de Faraó. E os servos de Faraó disseram-lhe: Até quando
este homem nos há de ser por laço? Deixa ir os homens, para que
sirvam ao Senhor seu Deus; ainda não sabes que o Egito está
destruído? Então Moisés e Arão foram levados outra vez a Faraó,
e ele disse-lhes: Ide, servi ao Senhor vosso Deus. Quais são os que
hão de ir? E Moisés disse: Havemos de ir com os nossos jovens, e
com os nossos velhos; com os nossos filhos, e com as nossas filhas,
com as nossas ovelhas, e com os nossos bois havemos de ir; porque
temos de celebrar uma festa ao Senhor. Então ele lhes disse:
Seja o Senhor assim convosco, como eu vos deixarei ir a vós e a
vossos filhos; olhai que há mal diante da vossa face. Não
será assim; agora ide vós, homens, e servi ao Senhor; pois isso é o
que pedistes. E os expulsaram da presença de Faraó.

Então disse o Senhor a Moisés: Estende a tua mão sobre a terra do
Egito para que os gafanhotos venham sobre a terra do Egito, e comam
toda a erva da terra, tudo o que deixou a saraiva. Então
estendeu Moisés sua vara sobre a terra do Egito, e o Senhor trouxe
sobre a terra um vento oriental todo aquele dia e toda aquela noite;
e aconteceu que pela manhã o vento oriental trouxe os gafanhotos.
E vieram os gafanhotos sobre toda a terra do Egito, e
assentaram-se sobre todos os termos do Egito; tão numerosos foram
que, antes destes nunca houve tantos, nem depois deles haverá.
Porque cobriram a face de toda a terra, de modo que a terra
se escureceu; e comeram toda a erva da terra, e todo o fruto das
árvores, que deixara a saraiva; e não ficou verde algum nas árvores,
nem na erva do campo, em toda a terra do Egito. Então Faraó
se apressou a chamar a Moisés e a Arão, e disse: Pequei contra o
Senhor vosso Deus, e contra vós. Agora, pois, peço-vos que
perdoeis o meu pecado somente desta vez, e que oreis ao Senhor vosso
Deus que tire de mim somente esta morte. E saiu da presença
de Faraó, e orou ao Senhor. Então o Senhor trouxe um vento
ocidental fortíssimo, o qual levantou os gafanhotos e os lançou no
\textbf{Mar Vermelho}; não ficou um só gafanhoto em todos os termos
do Egito. O Senhor, porém, endureceu o coração de Faraó, e
este não deixou ir os filhos de Israel.

Então disse o Senhor a Moisés: Estende a tua mão para o céu, e
virão trevas sobre a terra do Egito, trevas que se apalpem. E
Moisés estendeu a sua mão para o céu, e houve trevas espessas em
toda a terra do Egito por três dias. Não viu um ao outro, e
ninguém se levantou do seu lugar por três dias; mas todos os filhos
de Israel tinham luz em suas habitações. Então Faraó chamou a
Moisés, e disse: Ide, servi ao Senhor; somente fiquem vossas ovelhas
e vossas vacas; vão também convosco as vossas crianças.
Moisés, porém, disse: Tu também darás em nossas mãos
sacrifícios e holocaustos, que ofereçamos ao Senhor nosso Deus.
E também o nosso gado há de ir conosco, nem uma unha ficará;
porque daquele havemos de tomar, para servir ao Senhor nosso Deus;
porque não sabemos com que havemos de servir ao Senhor, até que
cheguemos lá. O Senhor, porém, endureceu o coração de Faraó,
e este não os quis deixar ir. E disse-lhe Faraó: Vai-te de
mim, guarda-te que não mais vejas o meu rosto; porque no dia em que
vires o meu rosto, morrerás. E disse Moisés: Bem disseste; eu
nunca mais verei o teu rosto.

\medskip

\lettrine{11} E o Senhor disse a Moisés: Ainda uma praga
trarei sobre Faraó, e sobre o Egito; depois vos deixará ir daqui; e,
quando vos deixar ir totalmente, a toda a pressa vos lançará daqui.
Fala agora aos ouvidos do povo, que cada homem peça ao seu
vizinho, e cada mulher à sua vizinha, jóias de prata e jóias de
ouro. E o Senhor deu ao povo graça aos olhos dos egípcios;
também o homem Moisés era mui grande na terra do Egito, aos olhos
dos servos de Faraó e aos olhos do povo.

Disse mais Moisés: Assim o Senhor tem dito: À meia noite eu sairei
pelo meio do Egito; e todo o primogênito na terra do Egito
morrerá, desde o primogênito de Faraó, que haveria de assentar-se
sobre o seu trono, até ao primogênito da serva que está detrás da
mó, e todo o primogênito dos animais. E haverá grande clamor em
toda a terra do Egito, como nunca houve semelhante e nunca haverá;
mas entre todos os filhos de Israel nem mesmo um cão moverá a
sua língua, desde os homens até aos animais, para que saibais que o
Senhor fez diferença entre os egípcios e os israelitas. Então
todos estes teus servos descerão a mim, e se inclinarão diante de
mim, dizendo: Sai tu, e todo o povo que te segue as pisadas; e
depois eu sairei. E saiu da presença de Faraó ardendo em ira. O
Senhor dissera a Moisés: Faraó não vos ouvirá, para que as minhas
maravilhas se multipliquem na terra do Egito. E Moisés e Arão
fizeram todas estas maravilhas diante de Faraó; mas o Senhor
endureceu o coração de Faraó, que não deixou ir os filhos de Israel
da sua terra.

\medskip

\lettrine{12} E falou o Senhor a Moisés e a Arão na terra do
Egito, dizendo: Este mesmo mês vos será o princípio dos meses;
\textbf{este vos será o primeiro dos meses do ano}. Falai a toda
a congregação de Israel, dizendo: Aos dez deste mês tome cada um
para si um cordeiro, segundo as casas dos pais, um cordeiro para
cada família. Mas se a família for pequena para um cordeiro,
então tome um só com seu vizinho perto de sua casa, conforme o
número das almas; cada um conforme ao seu comer, fareis a conta
conforme ao cordeiro. O cordeiro, ou cabrito, será sem mácula,
um macho de um ano, o qual tomareis das ovelhas ou das cabras. E
o guardareis até ao décimo quarto dia deste mês, e todo o
ajuntamento da congregação de Israel o sacrificará à tarde. E
tomarão do sangue, e pô-lo-ão em ambas as ombreiras, e na verga da
porta, nas casas em que o comerem. E naquela noite comerão a
carne assada no fogo, com pães ázimos; com ervas amargosas a
comerão. Não comereis dele cru, nem cozido em água, senão assado
no fogo, a sua cabeça com os seus pés e com a sua fressura. E
nada dele deixareis até a manhã\footnote{SBTB: até amanhã. King
James: ``And ye shall let nothing of it remain until the morning;
and that which remaineth of it until the morning ye shall burn with
fire.'' Edição Contemporânea: ``Nada deixareis dele até pela manhã;
se algo ficar dele até pela manhã, queimareis ao fogo.''}; mas o que
dele ficar até a manhã, queimareis no fogo. Assim pois o
comereis: Os vossos lombos cingidos, os vossos sapatos nos pés, e o
vosso cajado na mão; e o comereis apressadamente; \textbf{esta é a
páscoa do Senhor}\footnote{KJ: \ldots{}it is the Lord's passover.}.
E eu passarei pela terra do Egito esta noite, e ferirei todo
o primogênito na terra do Egito, desde os homens até aos animais; e
em todos os deuses do Egito farei juízos. Eu sou o Senhor. E
aquele sangue vos será por sinal nas casas em que estiverdes; vendo
eu sangue, passarei por cima de vós, e não haverá entre vós praga de
mortandade, quando eu ferir a terra do Egito. E este dia vos
será por memória, e celebrá-lo-eis por festa ao Senhor; nas vossas
gerações o celebrareis por estatuto perpétuo. Sete dias
comereis pães ázimos; ao primeiro dia tirareis o fermento das vossas
casas; porque qualquer que comer pão levedado, desde o primeiro até
ao sétimo dia, aquela alma será cortada de Israel. E ao
primeiro dia haverá santa convocação; também ao sétimo dia tereis
santa convocação; nenhuma obra se fará neles, senão o que cada alma
houver de comer; isso somente aprontareis para vós. Guardai
pois a \textbf{festa dos pães ázimos}, porque naquele mesmo dia
tirei vossos exércitos da terra do Egito; pelo que guardareis a este
dia nas vossas gerações por estatuto perpétuo. No primeiro
mês, aos catorze dias do mês, à tarde, comereis pães ázimos até
vinte e um do mês à tarde. Por sete dias não se ache nenhum
fermento nas vossas casas; porque qualquer que comer pão levedado,
aquela alma será cortada da congregação de Israel, assim o
estrangeiro como o natural da terra. Nenhuma coisa levedada
comereis; em todas as vossas habitações comereis pães ázimos.

Chamou pois Moisés a todos os anciãos de Israel, e disse-lhes:
Escolhei e tomai vós cordeiros para vossas famílias, e sacrificai a
páscoa. Então tomai um molho de hissopo, e molhai-o no sangue
que estiver na bacia, e passai-o na verga da porta, e em ambas as
ombreiras, do sangue que estiver na bacia; porém nenhum de vós saia
da porta da sua casa até à manhã. Porque o Senhor passará
para ferir aos egípcios, porém quando vir o sangue na verga da
porta, e em ambas as ombreiras, o Senhor passará aquela porta, e não
deixará o destruidor entrar em vossas casas, para vos ferir.
Portanto \textbf{guardai isto por estatuto para vós, e para
vossos filhos para sempre}. E acontecerá que, quando
entrardes na terra que o Senhor vos dará, como tem dito, guardareis
este culto. E acontecerá que, quando vossos filhos vos
disserem: Que culto é este? Então direis: Este é o sacrifício
da páscoa ao Senhor, que passou as casas dos filhos de Israel no
Egito, quando feriu aos egípcios, e livrou as nossas casas. Então o
povo inclinou-se, e adorou. E foram os filhos de Israel, e
fizeram isso como o Senhor ordenara a Moisés e a Arão, assim
fizeram.

E aconteceu, à meia noite, que o Senhor feriu a todos os
primogênitos na terra do Egito, desde o primogênito de Faraó, que se
sentava em seu trono, até ao primogênito do cativo que estava no
cárcere, e todos os primogênitos dos animais. E Faraó
levantou-se de noite, ele e todos os seus servos, e todos os
egípcios; e havia grande clamor no Egito, porque não havia casa em
que não houvesse um morto. Então chamou a Moisés e a Arão de
noite, e disse: Levantai-vos, saí do meio do meu povo, tanto vós
como os filhos de Israel; e ide, servi ao Senhor, como tendes dito.
Levai também convosco vossas ovelhas e vossas vacas, como
tendes dito; e ide, e abençoai-me também a mim. E os egípcios
apertavam ao povo, apressando-se para lançá-los da terra; porque
diziam: Todos seremos mortos. E o povo tomou a sua massa,
antes que levedasse, e as suas amassadeiras atadas em suas roupas
sobre seus ombros. Fizeram, pois, os filhos de Israel
conforme à palavra de Moisés, e pediram aos egípcios jóias de prata,
e jóias de ouro, e roupas. E o Senhor deu ao povo graça aos
olhos dos egípcios, e estes lhe davam o que pediam; e despojaram aos
egípcios.

Assim partiram os filhos de Israel de Ramessés para Sucote, cerca
de seiscentos mil a pé, somente de homens, sem contar os meninos.
E subiu também com eles muita mistura de gente, e ovelhas, e
bois, uma grande quantidade de gado. E cozeram bolos ázimos
da massa que levaram do Egito, porque não se tinha levedado,
porquanto foram lançados do Egito; e não se puderam deter, nem
prepararam comida. \textbf{O tempo que os filhos de Israel
habitaram no Egito foi de quatrocentos e trinta anos}. E
aconteceu que, passados os quatrocentos e trinta anos, naquele mesmo
dia, todos os exércitos do Senhor saíram da terra do Egito.
Esta noite se guardará ao Senhor, porque nela os tirou da
terra do Egito; esta é a noite do Senhor, que devem guardar todos os
filhos de Israel nas suas gerações.

Disse mais o Senhor a Moisés e a Arão: Esta é a ordenança da
páscoa: nenhum filho do estrangeiro comerá dela. Porém todo o
servo comprado por dinheiro, depois que o houveres circuncidado,
então comerá dela. O estrangeiro e o assalariado não comerão
dela. Numa casa se comerá; não levarás daquela carne fora da
casa, nem dela quebrareis osso. Toda a congregação de Israel
o fará. Porém se algum estrangeiro se hospedar contigo e
quiser celebrar a páscoa ao Senhor, seja-lhe circuncidado todo o
homem, e então chegará a celebrá-la, e será como o natural da terra;
mas nenhum incircunciso comerá dela. Uma mesma lei haja para
o natural e para o estrangeiro que peregrinar entre vós. E
todos os filhos de Israel o fizeram; como o Senhor ordenara a Moisés
e a Arão, assim fizeram. E aconteceu naquele mesmo dia que o
Senhor tirou os filhos de Israel da terra do Egito, segundo os seus
exércitos.

\medskip

\lettrine{13} Então falou o Senhor a Moisés, dizendo:
\textbf{Santifica-me todo o primogênito}, o que abrir toda a
madre entre os filhos de Israel, de homens e de animais; porque meu
é. E Moisés disse ao povo: Lembrai-vos deste mesmo dia, em que
saístes do Egito, da casa da servidão; pois com mão forte o Senhor
vos tirou daqui; portanto não comereis pão levedado. Hoje, no
mês de \textbf{Abibe}, vós saís. E acontecerá que, quando o
Senhor te houver introduzido na terra dos cananeus, e dos heteus, e
dos amorreus, e dos heveus, e dos jebuseus, a qual jurou a teus pais
que te daria, terra que mana leite e mel, \textbf{guardarás este
culto neste mês}. Sete dias comerás pães ázimos, e ao sétimo dia
haverá festa ao Senhor. Sete dias se comerá pães ázimos, e o
levedado não se verá contigo, nem ainda fermento será visto em todos
os teus termos. E naquele mesmo dia farás saber a teu filho,
dizendo: Isto é pelo que o Senhor me tem feito, quando eu saí do
Egito. E te será por sinal sobre tua mão e por lembrança entre
teus olhos, para que a lei do Senhor esteja em tua boca; porquanto
com mão forte o Senhor te tirou do Egito. Portanto tu
guardarás este estatuto a seu tempo, de ano em ano.

Também acontecerá que, quando o Senhor te houver introduzido na
terra dos cananeus, como jurou a ti e a teus pais, quando ta houver
dado, separarás para o Senhor tudo o que abrir a madre e todo
o primogênito dos animais que tiveres; os machos serão do Senhor.
Porém, \textbf{todo o primogênito da jumenta resgatarás com
um cordeiro}; e se o não resgatares, cortar-lhe-ás a cabeça; mas
\textbf{todo o primogênito do homem, entre teus filhos, resgatarás}.
E quando teu filho te perguntar no futuro, dizendo: Que é
isto? Dir-lhe-ás: O Senhor nos tirou com mão forte do Egito, da casa
da servidão. Porque sucedeu que, endurecendo-se Faraó, para
não nos deixar ir, o Senhor matou todos os primogênitos na terra do
Egito, desde o primogênito do homem até o primogênito dos animais;
por isso eu sacrifico ao Senhor todos os primogênitos, sendo machos;
porém a todo o primogênito de meus filhos eu resgato. E será
isso por sinal sobre tua mão, e por frontais entre os teus olhos;
porque o Senhor, com mão forte, nos tirou do Egito.

E aconteceu que, quando Faraó deixou ir o povo, Deus não os levou
pelo caminho da terra dos filisteus, que estava mais perto; porque
Deus disse: Para que porventura o povo não se arrependa, vendo a
guerra, e volte ao Egito. Mas Deus fez o povo rodear pelo
caminho do deserto do Mar Vermelho; e armados, os filhos de Israel
subiram da terra do Egito. E Moisés levou consigo os ossos de
José, porquanto havia este solenemente ajuramentado os filhos de
Israel, dizendo: Certamente Deus vos visitará; fazei, pois, subir
daqui os meus ossos convosco. Assim partiram de Sucote, e
acamparam-se em Etã, à entrada do deserto. E o Senhor ia
adiante deles, de dia numa coluna de nuvem para os guiar pelo
caminho, e de noite numa coluna de fogo para os iluminar, para que
caminhassem de dia e de noite. Nunca tirou de diante do povo
a coluna de nuvem, de dia, nem a coluna de fogo, de noite.

\medskip

\lettrine{14} Então falou o Senhor a Moisés, dizendo: Fala
aos filhos de Israel que voltem, e que se acampem diante de
Pi-Hairote, entre Migdol e o mar, diante de Baal-Zefom; em frente
dele assentareis o campo\footnote{\emph{Campo} ou
\emph{acampamento}? Creio que aqui cabe uma atualização para
``acampamento''. Outras versões traduzem por ``acampamento'' e a
King James utiliza o verbo ``encamp'' (acampar).} junto ao mar.
Então Faraó dirá dos filhos de Israel: Estão embaraçados na
terra, o deserto os encerrou. E eu endurecerei o coração de
Faraó, para que os persiga, e serei glorificado em Faraó e em todo o
seu exército, e saberão os egípcios que eu sou o Senhor. E eles
fizeram assim. Sendo, pois, anunciado ao rei do Egito que o povo
fugia, mudou-se o coração de Faraó e dos seus servos contra o povo,
e disseram: Por que fizemos isso, havendo deixado ir a Israel, para
que não nos sirva? E aprontou o seu carro, e tomou consigo o seu
povo; e tomou seiscentos carros escolhidos, e todos os carros do
Egito, e os capitães sobre eles todos. Porque o Senhor endureceu
o coração de Faraó, rei do Egito, para que perseguisse aos filhos de
Israel; porém os filhos de Israel saíram com alta mão. E os
egípcios perseguiram-nos, todos os cavalos e carros de Faraó, e os
seus cavaleiros e o seu exército, e alcançaram-nos acampados junto
ao mar, perto de Pi-Hairote, diante de Baal-Zefom.

E aproximando Faraó, os filhos de Israel levantaram seus olhos, e
eis que os egípcios vinham atrás deles, e temeram muito; então os
filhos de Israel clamaram ao Senhor. E disseram a Moisés: Não
havia sepulcros no Egito, para nos tirar de lá, para que morramos
neste deserto? Por que nos fizeste isto, fazendo-nos sair do Egito?
Não é esta a palavra que te falamos no Egito, dizendo:
Deixa-nos, que sirvamos aos egípcios? Pois que melhor nos fora
servir aos egípcios, do que morrermos no deserto. Moisés,
porém, disse ao povo: Não temais; estai quietos, e vede o livramento
do Senhor, que hoje vos fará; porque aos egípcios, que hoje vistes,
nunca mais os tornareis a ver. O Senhor pelejará por vós, e
vós vos calareis.

Então disse o Senhor a Moisés: Por que clamas a mim? Dize aos
filhos de Israel que marchem. E tu, levanta a tua vara, e
estende a tua mão sobre o mar, e fende-o, para que os filhos de
Israel passem pelo meio do mar em seco. E eis que endurecerei
o coração dos egípcios, e estes entrarão atrás deles; e eu serei
glorificado em Faraó e em todo o seu exército, nos seus carros e nos
seus cavaleiros, e os egípcios saberão que eu sou o Senhor,
quando for glorificado em Faraó, nos seus carros e nos seus
cavaleiros. E o \textbf{anjo de Deus}, que ia diante do
exército de Israel, se retirou, e ia atrás deles; também a coluna de
nuvem se retirou de diante deles, e se pôs atrás deles. E ia
entre o campo dos egípcios e o campo de Israel; e a nuvem era trevas
para aqueles, e para estes clareava a noite; de maneira que em toda
a noite não se aproximou um do outro.

Então Moisés estendeu a sua mão sobre o mar, e o Senhor fez
retirar o mar por um forte vento oriental toda aquela noite; e o mar
tornou-se em seco, e as águas foram partidas. E os filhos de
Israel entraram pelo meio do mar em seco; e as águas foram-lhes como
muro à sua direita e à sua esquerda. E os egípcios os
seguiram, e entraram atrás deles todos os cavalos de Faraó, os seus
carros e os seus cavaleiros, até ao meio do mar. E aconteceu
que, na vigília daquela manhã, o Senhor, na coluna do fogo e da
nuvem, viu o campo\footnote{Ed. Contemp.: acampamento. KJ: the host
of the Egyptians.} dos egípcios; e alvoroçou o campo dos egípcios.
E tirou-lhes as rodas dos seus carros, e dificultosamente os
governavam. Então disseram os egípcios: Fujamos da face de Israel,
porque o Senhor por eles peleja contra os egípcios. E disse o
Senhor a Moisés: Estende a tua mão sobre o mar, para que as águas
tornem sobre os egípcios, sobre os seus carros e sobre os seus
cavaleiros. Então Moisés estendeu a sua mão sobre o mar, e o
mar retornou a sua força ao amanhecer, e os egípcios, ao fugirem,
foram de encontro a ele, e o Senhor derrubou os egípcios no meio do
mar, porque as águas, tornando, cobriram os carros e os
cavaleiros de todo o exército de Faraó, que os haviam seguido no
mar; nenhum deles ficou. Mas os filhos de Israel foram pelo
meio do mar seco; e as águas foram-lhes como muro à sua mão direita
e à sua esquerda. Assim o Senhor salvou Israel naquele dia da
mão dos egípcios; e Israel viu os egípcios mortos na praia do mar.
E viu Israel a grande mão que o Senhor mostrara aos egípcios;
e temeu o povo ao Senhor, e creu no Senhor e em Moisés, seu servo.

\medskip

\lettrine{15} Então cantou Moisés e os filhos de Israel este
cântico ao Senhor, e falaram, dizendo: Cantarei ao Senhor, porque
gloriosamente triunfou; lançou no mar o cavalo e o seu cavaleiro.
O Senhor é a minha força, e o meu cântico; ele me foi por
salvação; este é o meu Deus, portanto lhe farei uma habitação; ele é
o Deus de meu pai, por isso o exaltarei. O Senhor é homem de
guerra; o Senhor é o seu nome. Lançou no mar os carros de Faraó
e o seu exército; e os seus escolhidos príncipes afogaram-se no Mar
Vermelho. Os abismos os cobriram; desceram às profundezas como
pedra. A tua destra, ó Senhor, se tem glorificado em poder, a
tua destra, ó Senhor, tem despedaçado o inimigo; e com a
grandeza da tua excelência derrubaste aos que se levantaram contra
ti; enviaste o teu furor, que os consumiu como o restolho. E com
o sopro de tuas narinas amontoaram-se as águas, as correntes pararam
como montão; os abismos coalharam-se no coração do mar. O
inimigo dizia: Perseguirei, alcançarei, repartirei os despojos;
fartar-se-á a minha alma deles, arrancarei a minha espada, a minha
mão os destruirá. Sopraste com o teu vento, o mar os cobriu;
afundaram-se como chumbo em veementes águas. Ó Senhor, quem é
como tu entre os deuses? Quem é como tu glorificado em santidade,
admirável em louvores, realizando maravilhas? Estendeste a
tua mão direita; a terra os tragou. Tu, com a tua
beneficência, guiaste a este povo, que salvaste; com a tua força o
levaste à habitação da tua santidade. Os povos o ouviram,
eles estremeceram, uma dor apoderou-se dos habitantes da Filístia.
Então os príncipes de Edom se pasmaram; dos poderosos dos
moabitas apoderou-se um tremor; derreteram-se todos os habitantes de
Canaã. Espanto e pavor caiu sobre eles; pela grandeza do teu
braço emudeceram como pedra; até que o teu povo houvesse passado, ó
Senhor, até que passasse este povo que adquiriste. Tu os
introduzirás, e os plantarás no monte da tua herança, no lugar que
tu, ó Senhor, aparelhaste para a tua habitação, no santuário, ó
Senhor, que as tuas mãos estabeleceram. \textbf{O Senhor
reinará eterna e perpetuamente}; porque os cavalos de Faraó,
com os seus carros e com os seus cavaleiros, entraram no mar, e o
Senhor fez tornar as águas do mar sobre eles; mas os filhos de
Israel passaram em seco pelo meio do mar. Então Miriã, a
profetiza, a irmã de Arão, tomou o tamboril na sua mão, e todas as
mulheres saíram atrás dela com tamboris e com danças. E Miriã
lhes respondia: Cantai ao Senhor, porque gloriosamente triunfou; e
lançou no mar o cavalo com o seu cavaleiro.

Depois fez Moisés partir os israelitas do Mar Vermelho, e saíram
ao deserto de Sur; e andaram três dias no deserto, e não acharam
água. Então chegaram a Mara; mas não puderam beber das águas
de Mara, porque eram amargas; por isso chamou-se o lugar Mara.
E o povo murmurou contra Moisés, dizendo: Que havemos de
beber? E ele clamou ao Senhor, e o Senhor mostrou-lhe uma
árvore, que lançou nas águas, e as águas se tornaram doces.
\textbf{Ali lhes deu estatutos e uma ordenança, e ali os provou}.
E disse: Se ouvires atento a voz do Senhor teu Deus, e
fizeres o que é reto diante de seus olhos, e inclinares os teus
ouvidos aos seus mandamentos, e guardares todos os seus estatutos,
nenhuma das enfermidades porei sobre ti, que pus sobre o Egito;
porque eu sou o Senhor que te sara. Então vieram a Elim, e
havia ali doze fontes de água e setenta palmeiras; e ali se
acamparam junto das águas.

\medskip

\lettrine{16} E partindo de Elim, toda a congregação dos
filhos de Israel veio ao deserto de Sim, que está entre Elim e
Sinai, aos quinze dias do mês segundo, depois de sua saída da terra
do Egito. E toda a congregação dos filhos de Israel murmurou
contra Moisés e contra Arão no deserto. E os filhos de Israel
disseram-lhes: Quem dera tivéssemos morrido por mão do Senhor na
terra do Egito, quando estávamos sentados junto às panelas de carne,
quando comíamos pão até fartar! Porque nos tendes trazido a este
deserto, para matardes de fome a toda esta multidão. Então disse
o Senhor a Moisés: Eis que vos farei chover pão dos céus, e o povo
sairá, e colherá diariamente a porção para cada dia, para que eu o
prove se anda em minha lei ou não. E acontecerá, no sexto dia,
que prepararão o que colherem; e será o dobro do que colhem cada
dia. Então disseram Moisés e Arão a todos os filhos de Israel: À
tarde sabereis que o Senhor vos tirou da terra do Egito, e
amanhã vereis a glória do Senhor, porquanto ouviu as vossas
murmurações contra o Senhor. E quem somos nós, para que murmureis
contra nós? Disse mais Moisés: Isso será quando o Senhor à tarde
vos der carne para comer, e pela manhã pão a fartar, porquanto o
Senhor ouviu as vossas murmurações, com que murmurais contra ele. E
quem somos nós? As vossas murmurações não são contra nós, mas sim
contra o Senhor. Depois disse Moisés a Arão: Dize a toda a
congregação dos filhos de Israel: Chegai-vos à presença do Senhor,
porque ouviu as vossas murmurações. E aconteceu que, quando
falou Arão a toda a congregação dos filhos de Israel, e eles se
viraram para o deserto, eis que a glória do Senhor apareceu na
nuvem. E o Senhor falou a Moisés, dizendo: Tenho
ouvido as murmurações dos filhos de Israel. Fala-lhes, dizendo:
Entre as duas tardes comereis carne, e pela manhã vos fartareis de
pão; e sabereis que eu sou o Senhor vosso Deus.

E aconteceu que à tarde subiram codornizes, e cobriram o arraial;
e pela manhã jazia o orvalho ao redor do arraial. E quando o
orvalho se levantou, eis que sobre a face do deserto estava uma
coisa miúda, redonda, miúda como a geada sobre a terra. E,
vendo-a os filhos de Israel, disseram uns aos outros: Que é isto?
Porque não sabiam o que era. Disse-lhes pois Moisés: Este é o pão
que o Senhor vos deu para comer. Esta é a palavra que o
Senhor tem mandado: Colhei dele cada um conforme ao que pode comer,
um ômer por cabeça, segundo o número das vossas almas; cada um
tomará para os que se acharem na sua tenda. E os filhos de
Israel fizeram assim; e colheram, uns mais e outros menos.
Porém, medindo-o com o ômer, não sobejava ao que colhera
muito, nem faltava ao que colhera pouco; cada um colheu tanto quanto
podia comer. E disse-lhes Moisés: Ninguém deixe dele para
amanhã. Eles, porém, não deram ouvidos a Moisés, antes alguns
deles deixaram dele para o dia seguinte; e criou bichos, e cheirava
mal; por isso indignou-se Moisés contra eles. Eles, pois, o
colhiam cada manhã, cada um conforme ao que podia comer; porque,
aquecendo o sol, derretia-se.

E aconteceu que ao sexto dia colheram pão em dobro, dois ômeres
para cada um; e todos os príncipes da congregação vieram, e
contaram-no a Moisés. E ele disse-lhes: Isto é o que o Senhor
tem dito: Amanhã é repouso, o santo sábado do Senhor; o que
quiserdes cozer no forno, cozei-o, e o que quiserdes cozer em água,
cozei-o em água; e tudo o que sobejar, guardai para vós até amanhã.
E guardaram-no até o dia seguinte, como Moisés tinha
ordenado; e não cheirou mal nem nele houve algum bicho. Então
disse Moisés: Comei-o hoje, porquanto hoje é o \textbf{sábado do
Senhor}; hoje não o achareis no campo. Seis dias o colhereis,
mas o sétimo dia é o sábado; nele não haverá. E aconteceu ao
sétimo dia, que alguns do povo saíram para colher, mas não o
acharam. Então disse o Senhor a Moisés: Até quando recusareis
guardar os meus mandamentos e as minhas leis? Vede, porquanto
o Senhor vos deu o sábado, portanto ele no sexto dia vos dá pão para
dois dias; cada um fique no seu lugar, ninguém saia do seu lugar no
sétimo dia. Assim repousou o povo no sétimo dia. E
chamou a casa de Israel o seu nome \textbf{maná}; e era como semente
de coentro branco, e o seu sabor como bolos de mel. E disse
Moisés: Esta é a palavra que o Senhor tem mandado: Encherás um ômer
dele e guardá-lo-ás para as vossas gerações, para que vejam o pão
que vos tenho dado a comer neste deserto, quando eu vos tirei da
terra do Egito. Disse também Moisés a Arão: Toma um vaso, e
põe nele um ômer cheio de maná, e coloca-o diante do Senhor, para
guardá-lo para as vossas gerações. Como o Senhor tinha
ordenado a Moisés, assim Arão o pôs diante do \textbf{Testemunho},
para ser guardado. E comeram os filhos de Israel maná
quarenta anos, até que entraram em terra habitada; comeram maná até
que chegaram aos termos da terra de Canaã. E um ômer é a
décima parte do efa.

\medskip

\lettrine{17} Depois toda a congregação dos filhos de Israel
partiu do deserto de Sim pelas suas jornadas, segundo o mandamento
do Senhor, e acampou em Refidim; e não havia ali água para o povo
beber. Então contendeu o povo com Moisés, e disse: Dá-nos água
para beber. E Moisés lhes disse: Por que contendeis comigo? Por que
tentais ao Senhor? Tendo pois ali o povo sede de água, o povo
murmurou contra Moisés, e disse: Por que nos fizeste subir do Egito,
para nos matares de sede, a nós e aos nossos filhos, e ao nosso
gado? E clamou Moisés ao Senhor, dizendo: Que farei a este povo?
Daqui a pouco me apedrejará. Então disse o Senhor a Moisés:
Passa diante do povo, e toma contigo alguns dos anciãos de Israel; e
toma na tua mão a tua vara, com que feriste o rio, e vai. Eis
que eu estarei ali diante de ti sobre a rocha, em Horebe, e tu
ferirás a rocha, e dela sairão águas e o povo beberá. E Moisés assim
o fez, diante dos olhos dos anciãos de Israel. E chamou aquele
lugar Massá e Meribá, por causa da contenda dos filhos de Israel, e
porque tentaram ao Senhor, dizendo: Está o Senhor no meio de nós, ou
não?

Então veio \textbf{Amaleque}, e pelejou contra Israel em Refidim.
Por isso disse Moisés a \textbf{Josué}: Escolhe-nos homens, e
sai, peleja contra Amaleque; amanhã eu estarei sobre o cume do
outeiro, e a vara de Deus estará na minha mão. E fez Josué
como Moisés lhe dissera, pelejando contra Amaleque; mas Moisés,
Arão, e Hur subiram ao cume do outeiro. E acontecia que,
quando Moisés levantava a sua mão, Israel prevalecia; mas quando ele
abaixava a sua mão, Amaleque prevalecia. Porém as mãos de
Moisés eram pesadas, por isso tomaram uma pedra, e a puseram debaixo
dele, para assentar-se sobre ela; e Arão e Hur sustentaram as suas
mãos, um de um lado e o outro do outro; assim ficaram as suas mãos
firmes até que o sol se pôs. E assim Josué desfez a Amaleque
e a seu povo, ao fio da espada. Então disse o Senhor a
Moisés: \textbf{Escreve isto para memória num livro}, e relata-o aos
ouvidos de Josué; que eu totalmente hei de riscar a memória de
Amaleque de debaixo dos céus. E Moisés edificou um altar, ao
qual chamou: O SENHOR É MINHA BANDEIRA. E disse: Porquanto
jurou o Senhor, haverá guerra do Senhor contra Amaleque de geração
em geração.

\medskip

\lettrine{18} Ora Jetro, sacerdote de Midiã, sogro de Moisés,
ouviu todas as coisas que Deus tinha feito a Moisés e a Israel seu
povo, como o Senhor tinha tirado a Israel do Egito. E Jetro,
sogro de Moisés, tomou a Zípora, a mulher de Moisés, depois que ele
lha enviara, com seus dois filhos, dos quais um se chamava
Gérson; porque disse: Eu fui peregrino em terra estranha; e o
outro se chamava Eliézer; porque disse: O Deus de meu pai foi por
minha ajuda, e me livrou da espada de Faraó. Vindo, pois, Jetro,
o sogro de Moisés, com seus filhos e com sua mulher, a Moisés no
deserto, ao monte de Deus, onde se tinha acampado, disse a
Moisés: Eu, teu sogro Jetro, venho a ti, com tua mulher e seus dois
filhos com ela.

Então saiu Moisés ao encontro de seu sogro, e inclinou-se, e
beijou-o, e perguntaram um ao outro como estavam, e entraram na
tenda. E Moisés contou a seu sogro todas as coisas que o Senhor
tinha feito a Faraó e aos egípcios por amor de Israel, e todo o
trabalho que passaram no caminho, e como o Senhor os livrara. E
alegrou-se Jetro de todo o bem que o Senhor tinha feito a Israel,
livrando-o da mão dos egípcios. E Jetro disse: Bendito seja o
Senhor, que vos livrou das mãos dos egípcios e da mão de Faraó; que
livrou a este povo de debaixo da mão dos egípcios. Agora sei
que o Senhor é maior que todos os deuses; porque na coisa em que se
ensoberbeceram, os sobrepujou. Então Jetro, o sogro de
Moisés, tomou holocausto e sacrifícios para Deus; e veio Arão, e
todos os anciãos de Israel, para comerem pão com o sogro de Moisés
diante de Deus.

E aconteceu que, no outro dia, Moisés assentou-se para julgar o
povo; e o povo estava em pé diante de Moisés desde a manhã até à
tarde. Vendo, pois, o sogro de Moisés tudo o que ele fazia ao
povo, disse: Que é isto, que tu fazes ao povo? Por que te assentas
só, e todo o povo está em pé diante de ti, desde a manhã até à
tarde? Então disse Moisés a seu sogro: É porque este povo vem
a mim, para consultar a Deus; quando tem algum negócio vem a
mim, para que eu julgue entre um e outro e lhes declare os estatutos
de Deus e as suas leis. O sogro de Moisés, porém, lhe disse:
Não é bom o que fazes. Totalmente desfalecerás, assim tu como
este povo que está contigo; porque este negócio é mui difícil para
ti; tu só não o podes fazer. Ouve agora minha voz, eu te
aconselharei, e Deus será contigo. Sê tu pelo povo diante de Deus, e
leva tu as causas a Deus; e declara-lhes os estatutos e as
leis, e faze-lhes saber o caminho em que devem andar, e a obra que
devem fazer. E tu dentre todo o povo procura homens capazes,
tementes a Deus, homens de verdade, que odeiem a avareza; e põe-nos
sobre eles por maiorais de mil, maiorais de cem, maiorais de
cinqüenta, e maiorais de dez; para que julguem este povo em
todo o tempo; e seja que todo o negócio grave tragam a ti, mas todo
o negócio pequeno eles o julguem; assim a ti mesmo te aliviarás da
carga, e eles a levarão contigo. Se isto fizeres, e Deus to
mandar, poderás então subsistir; assim também todo este povo em paz
irá ao seu lugar. E Moisés deu ouvidos à voz de seu sogro, e
fez tudo quanto tinha dito; e escolheu Moisés homens capazes,
de todo o Israel, e os pôs por cabeças sobre o povo; maiorais de
mil, maiorais de cem, maiorais de cinqüenta e maiorais de dez.
E eles julgaram o povo em todo o tempo; o negócio árduo
trouxeram a Moisés, e todo o negócio pequeno julgaram eles.
Então despediu Moisés o seu sogro, o qual se foi à sua terra.

\medskip

\lettrine{19} Ao terceiro mês da saída dos filhos de Israel da
terra do Egito, no mesmo dia chegaram ao deserto de Sinai,
porque partiram de Refidim e entraram no deserto de Sinai, onde
se acamparam. Israel, pois, ali se acampou em frente ao monte. E
subiu Moisés a Deus, e o Senhor o chamou do monte, dizendo: Assim
falarás à casa de Jacó, e anunciarás aos filhos de Israel: Vós
tendes visto o que fiz aos egípcios, como vos levei sobre asas de
águias, e vos trouxe a mim; agora, pois, se diligentemente
ouvirdes a minha voz e guardardes a minha aliança, então sereis a
minha propriedade peculiar dentre todos os povos, porque toda a
terra é minha. E vós me sereis um reino sacerdotal e o povo
santo. Estas são as palavras que falarás aos filhos de Israel. E
veio Moisés, e chamou os anciãos do povo, e expôs diante deles todas
estas palavras, que o Senhor lhe tinha ordenado. Então todo o
povo respondeu a uma voz, e disse: Tudo o que o Senhor tem falado,
faremos. E relatou Moisés ao Senhor as palavras do povo.

E disse o Senhor a Moisés: \textbf{Eis que eu virei a ti numa
nuvem espessa, para que o povo ouça, falando eu contigo, e para que
também te creiam eternamente}. Porque Moisés tinha anunciado as
palavras do seu povo ao Senhor. Disse também o Senhor a
Moisés: Vai ao povo, e santifica-os hoje e amanhã, e lavem eles as
suas roupas, e estejam prontos para o terceiro dia; porquanto
no terceiro dia o Senhor descerá diante dos olhos de todo o povo
sobre o monte Sinai. E marcarás limites ao povo em redor,
dizendo: Guardai-vos, não subais ao monte, nem toqueis o seu termo;
todo aquele que tocar o monte, certamente morrerá. Nenhuma
mão tocará nele; porque certamente será apedrejado ou
asseteado\footnote{Provocar ferimento em ou matar (ger. com seta ou
instrumento de incisão). Sentido figurado: provocar grande
sofrimento moral; mortificar, martirizar; desacreditar publicamente,
caluniar; atacar com palavras ultrajantes; insultar, injuriar,
afrontar.}; quer seja animal, quer seja homem, não viverá; soando a
buzina longamente, então subirão ao monte. Então Moisés
desceu do monte ao povo, e santificou o povo; e lavaram as suas
roupas. E disse ao povo: Estai prontos ao terceiro dia; e não
vos chegueis a mulher.

E aconteceu que, ao terceiro dia, ao amanhecer, houve trovões e
relâmpagos sobre o monte, e uma espessa nuvem, e um sonido de buzina
mui forte, de maneira que estremeceu todo o povo que estava no
arraial. E Moisés levou o povo fora do arraial ao encontro de
Deus; e puseram-se ao pé do monte. E todo o monte Sinai
fumegava, porque o Senhor descera sobre ele em fogo; e a sua fumaça
subiu como fumaça de uma fornalha, e todo o monte tremia
grandemente. E o sonido da buzina ia crescendo cada vez mais;
\textbf{Moisés falava, e Deus lhe respondia em voz alta}. E,
descendo o Senhor sobre o monte Sinai, sobre o cume do monte, chamou
o Senhor a Moisés ao cume do monte; e Moisés subiu. E disse o
Senhor a Moisés: Desce, adverte ao povo que não traspasse o termo
para ver o Senhor, para que muitos deles não pereçam. E
também os sacerdotes, que se chegam ao Senhor, se hão de santificar,
para que o Senhor não se lance sobre eles. Então disse Moisés
ao Senhor: O povo não poderá subir ao monte Sinai, porque tu nos
tens advertido, dizendo: Marca termos ao redor do monte, e
santifica-o. E disse-lhe o Senhor: Vai, desce; depois subirás
tu, e Arão contigo; os sacerdotes, porém, e o povo não traspassem o
termo para subir ao Senhor, para que não se lance sobre eles.
Então Moisés desceu ao povo, e disse-lhe isto.

\medskip

\lettrine{20} Então falou Deus todas estas palavras, dizendo:
\textbf{Eu sou o Senhor teu Deus}, que te tirei da terra do
Egito, da casa da servidão. Não terás outros deuses diante de
mim. Não farás para ti imagem de escultura, nem alguma
semelhança do que há em cima nos céus, nem em baixo na terra, nem
nas águas debaixo da terra. Não te encurvarás a elas nem as
servirás; porque eu, o Senhor teu Deus, sou Deus zeloso, que visito
a iniqüidade dos pais nos filhos, até a terceira e quarta geração
daqueles que me odeiam. E faço misericórdia a milhares dos que
me amam e aos que guardam os meus mandamentos. Não tomarás o
nome do Senhor teu Deus em vão; porque o Senhor não terá por
inocente o que tomar o seu nome em vão. Lembra-te do dia do
sábado, para o santificar. Seis dias trabalharás, e farás toda a
tua obra. Mas o sétimo dia é o sábado do Senhor teu Deus; não
farás nenhuma obra, nem tu, nem teu filho, nem tua filha, nem o teu
servo, nem a tua serva, nem o teu animal, nem o teu estrangeiro, que
está dentro das tuas portas. Porque em seis dias fez o Senhor
os céus e a terra, o mar e tudo que neles há, e ao sétimo dia
descansou; portanto abençoou o Senhor o dia do sábado, e o
santificou.

Honra a teu pai e a tua mãe, para que se prolonguem os teus dias
na terra que o Senhor teu Deus te dá. Não matarás. Não
adulterarás. Não furtarás. Não dirás falso testemunho
contra o teu próximo. Não cobiçarás a casa do teu próximo,
não cobiçarás a mulher do teu próximo, nem o seu servo, nem a sua
serva, nem o seu boi, nem o seu jumento, nem coisa alguma do teu
próximo.

E todo o povo viu os trovões e os relâmpagos, e o sonido da
buzina, e o monte fumegando; e o povo, vendo isso, retirou-se e
pôs-se de longe. E disseram a Moisés: Fala tu conosco, e
ouviremos: e não fale Deus conosco, para que não morramos. E
disse Moisés ao povo: Não temais, Deus veio para vos provar, e para
que o seu temor esteja diante de vós, a fim de que não pequeis.
E o povo estava em pé de longe. Moisés, porém, se chegou à
escuridão, onde Deus estava.

Então disse o Senhor a Moisés: Assim dirás aos filhos de Israel:
\textbf{Vós tendes visto que, dos céus, eu falei convosco}.
Não fareis outros deuses comigo; deuses de prata ou deuses de
ouro não fareis para vós. Um altar de terra me farás, e sobre
ele sacrificarás os teus holocaustos, e as tuas ofertas pacíficas,
as tuas ovelhas, e as tuas vacas; em todo o lugar, onde eu fizer
celebrar a memória do meu nome, virei a ti e te abençoarei. E
se me fizeres um altar de pedras, não o farás de pedras lavradas; se
sobre ele levantares o teu buril\footnote{Ferramenta de aço com
ponta oblíqua cortante, us. na gravação em metal ou madeira;
ferramenta similar us. para lavrar pedra.}, profaná-lo-ás.
Também não subirás ao meu altar por degraus, para que a tua
nudez não seja descoberta diante deles.

\medskip

\lettrine{21} Estes são os estatutos que lhes proporás. Se
comprares um servo hebreu, seis anos servirá; mas ao sétimo sairá
livre, de graça. Se entrou só com o seu corpo, só com o seu
corpo sairá; se ele era homem casado, sua mulher sairá com ele.
Se seu senhor lhe houver dado uma mulher e ela lhe houver dado
filhos ou filhas, a mulher e seus filhos serão de seu senhor, e ele
sairá sozinho. Mas se aquele servo expressamente disser: Eu amo
a meu senhor, e a minha mulher, e a meus filhos; não quero sair
livre, então seu senhor o levará aos juízes, e o fará chegar à
porta, ou ao umbral da porta, e seu senhor lhe furará a orelha com
uma sovela\footnote{Instrumento para polir pedras. Instrumento
formado por uma espécie de agulha reta ou curva, com cabo, com que
os sapateiros e correeiros furam o couro para o costurar.}; e ele o
servirá para sempre. E se um homem vender sua filha para ser
serva, ela não sairá como saem os servos. Se ela não agradar ao
seu senhor, e ele não se desposar com ela, fará que se resgate; não
poderá vendê-la a um povo estranho, agindo deslealmente com ela.
Mas se a desposar com seu filho, fará com ela conforme ao
direito das filhas. Se lhe tomar outra, não diminuirá o
mantimento desta, nem o seu vestido, nem a sua obrigação marital.
E se lhe não fizer estas três coisas, sairá de graça, sem dar
dinheiro.

Quem ferir alguém, de modo que este morra, certamente será morto.
Porém se lhe não armou cilada, mas Deus lho entregou nas
mãos, ordenar-te-ei um lugar para onde fugirá. Mas se alguém
agir premeditadamente contra o seu próximo, matando-o à traição,
tirá-lo-ás do meu altar, para que morra. O que ferir a seu
pai, ou a sua mãe, certamente será morto. E quem raptar um
homem, e o vender, ou for achado na sua mão, certamente será morto.
E quem amaldiçoar a seu pai ou a sua mãe, certamente será
morto. E se dois homens pelejarem, ferindo-se um ao outro com
pedra ou com o punho, e este não morrer, mas cair na cama, se
ele tornar a levantar-se e andar fora, sobre o seu bordão, então
aquele que o feriu será absolvido; somente lhe pagará o tempo que
perdera e o fará curar totalmente. Se alguém ferir a seu
servo, ou a sua serva, com pau, e morrer debaixo da sua mão,
certamente será castigado; porém se sobreviver por um ou dois
dias, não será castigado, porque é dinheiro seu.

Se alguns homens pelejarem, e um ferir uma mulher grávida, e for
causa de que aborte, porém não havendo outro dano, certamente será
multado, conforme o que lhe impuser o marido da mulher, e julgarem
os juízes. Mas se houver morte, então darás vida por vida,
olho por olho, dente por dente, mão por mão, pé por pé,
queimadura por queimadura, ferida por ferida, golpe por
golpe. E quando alguém ferir o olho do seu servo, ou o olho
da sua serva, e o danificar, o deixará ir livre pelo seu olho.
E se tirar o dente do seu servo, ou o dente da sua serva, o
deixará ir livre pelo seu dente. E se algum boi escornear
homem ou mulher, que morra, o boi será apedrejado certamente, e a
sua carne não se comerá; mas o dono do boi será absolvido.
Mas se o boi dantes era escorneador, e o seu dono foi
conhecedor disso, e não o guardou, matando homem ou mulher, o boi
será apedrejado, e também o seu dono morrerá. Se lhe for
imposto resgate, então dará por resgate da sua vida tudo quanto lhe
for imposto, quer tenha escorneado um filho, quer tenha
escorneado uma filha; conforme a este estatuto lhe será feito.
Se o boi escornear um servo, ou uma serva, dar-se-á trinta
siclos de prata ao seu senhor, e o boi será apedrejado. Se
alguém abrir uma cova, ou se alguém cavar uma cova, e não a cobrir,
e nela cair um boi ou um jumento, o dono da cova o pagará;
pagará em dinheiro ao seu dono, mas o animal morto será seu.
Se o boi de alguém ferir o boi do seu próximo, e morrer,
então se venderá o boi vivo, e o dinheiro dele se repartirá
igualmente, e também repartirão entre si o boi morto. Mas se
foi notório que aquele boi antes era escorneador, e seu dono não o
guardou, certamente pagará boi por boi; porém o morto será seu.

\medskip

\lettrine{22} Se alguém furtar boi ou ovelha, e o degolar ou
vender, por um boi pagará cinco bois, e pela ovelha quatro ovelhas.
Se o ladrão for achado roubando, e for ferido, e morrer, o que o
feriu não será culpado do sangue. Se o sol houver saído sobre
ele, o agressor será culpado do sangue; o ladrão fará restituição
total; e se não tiver com que pagar, será vendido por seu furto.
Se o furto for achado vivo na sua mão, seja boi, ou jumento, ou
ovelha, pagará o dobro. Se alguém fizer pastar o seu animal num
campo ou numa vinha, e largá-lo para comer no campo de outro, o
melhor do seu próprio campo e o melhor da sua própria vinha
restituirá. Se irromper um fogo, e pegar nos espinhos, e queimar
a meda\footnote{Amontoado de feixes de trigo, palha etc., arrumados
uns sobre os outros, em forma proximamente cônica, e apoiados em uma
vara vertical encimada por uma proteção de palha que desvia a chuva.
Por extensão de sentido: monte de coisas de mesma espécie;
amontoado.} de trigo, ou a seara, ou o campo, aquele que acendeu o
fogo totalmente pagará o queimado.

Se alguém der ao seu próximo dinheiro, ou bens, a guardar, e isso
for furtado da casa daquele homem, o ladrão, se for achado, pagará o
dobro. Se o ladrão não for achado, então o dono da casa será
levado diante dos juízes, a ver se não pôs a sua mão nos bens do seu
próximo. Sobre todo o negócio fraudulento, sobre boi, sobre
jumento, sobre gado miúdo, sobre roupa, sobre toda a coisa perdida,
de que alguém disser que é sua, a causa de ambos será levada perante
os juízes; aquele a quem condenarem os juízes pagará em dobro ao seu
próximo. Se alguém der a seu próximo a guardar um jumento, ou
boi, ou ovelha, ou outro animal, e este morrer, ou for dilacerado,
ou arrebatado, ninguém o vendo, então haverá juramento do
Senhor entre ambos, de que não pôs a sua mão nos bens do seu
próximo; e seu dono o aceitará, e o outro não o restituirá.
Mas, se de fato lhe tiver sido furtado, pagá-lo-á ao seu
dono. Porém se lhe for dilacerado, trá-lo-á em testemunho
disso, e não pagará o dilacerado. E se alguém pedir
emprestado a seu próximo algum animal, e for danificado ou morto,
não estando presente o seu dono, certamente o pagará. Se o
seu dono estava presente, não o pagará; se foi alugado, será pelo
seu aluguel.

Se alguém enganar alguma virgem, que não for desposada, e se
deitar com ela, certamente a dotará e tomará por sua mulher.
Se seu pai inteiramente recusar dar-lha, pagará ele em
dinheiro conforme ao dote das virgens. \textbf{A feiticeira
não deixarás viver}. Todo aquele que se deitar com animal,
certamente morrerá. O que sacrificar aos deuses, e não só ao
Senhor, será morto. O estrangeiro não afligirás, nem o
oprimirás; pois estrangeiros fostes na terra do Egito. A
nenhuma viúva nem órfão afligireis. Se de algum modo os
afligires, e eles clamarem a mim, eu certamente ouvirei o seu
clamor. E a minha ira se acenderá, e vos matarei à espada; e
vossas mulheres ficarão viúvas, e vossos filhos órfãos.

Se emprestares dinheiro ao meu povo, ao pobre que está contigo,
não te haverás com ele como um usurário\footnote{Em que há usura.
Que faz empréstimos com usura; agiota.}; não lhe imporeis
usura\footnote{Juro, renda ou rendimento de capital.}. Se
tomares em penhor a roupa do teu próximo, lho restituirás antes do
pôr do sol, porque aquela é a sua cobertura, e o vestido da
sua pele; em que se deitaria? Será pois que, quando clamar a mim, eu
o ouvirei, porque sou misericordioso. A Deus não
amaldiçoarás, e o príncipe dentre o teu povo não maldirás. As
tuas primícias, e os teus licores não retardarás; o primogênito de
teus filhos me darás. Assim farás dos teus bois e das tuas
ovelhas: sete dias estarão com sua mãe, e ao oitavo dia mos darás.
E \textbf{ser-me-eis homens santos}; portanto não comereis
carne despedaçada no campo; aos cães a lançareis.

\medskip

\lettrine{23} Não admitirás falso boato, e não porás a tua mão
com o ímpio, para seres testemunha falsa. Não seguirás a
multidão para fazeres o mal; nem numa demanda falarás, tomando parte
com a maioria para torcer o direito. Nem ao pobre favorecerás na
sua demanda. Se encontrares o boi do teu inimigo, ou o seu
jumento, desgarrado, sem falta lho reconduzirás. Se vires o
jumento, daquele que te odeia, caído debaixo da sua carga, deixarás
pois de ajudá-lo? Certamente o ajudarás a levantá-lo. Não
perverterás o direito do teu pobre na sua demanda. De palavras
de falsidade te afastarás, e não matarás o inocente e o justo;
porque não justificarei o ímpio. Também suborno não tomarás;
porque o suborno cega os que têm vista, e perverte as palavras dos
justos. Também não oprimirás o estrangeiro; pois vós conheceis o
coração do estrangeiro, pois fostes estrangeiros na terra do Egito.

Também \textbf{seis anos semearás tua terra, e recolherás os seus
frutos}; \textbf{mas ao sétimo a dispensarás e deixarás
descansar}, para que possam comer os pobres do teu povo, e da sobra
comam os animais do campo. Assim farás com a tua vinha e com o teu
olival. Seis dias farás os teus trabalhos mas ao sétimo dia
descansarás; para que descanse o teu boi, e o teu jumento; e para
que tome alento o filho da tua escrava, e o estrangeiro. E em
tudo o que vos tenho dito, guardai-vos; e do nome de outros deuses
nem vos lembreis, nem se ouça da vossa boca. \textbf{Três
vezes no ano me celebrareis festa}. A \textbf{festa dos pães
ázimos} guardarás; sete dias comerás pães ázimos, como te tenho
ordenado, ao tempo apontado no mês de Abibe; porque nele saíste do
Egito; e ninguém apareça vazio perante mim; e a \textbf{festa
da sega dos primeiros frutos} do teu trabalho, que houveres semeado
no campo, e a \textbf{festa da colheita}, à saída do ano, quando
tiveres colhido do campo o teu trabalho. \textbf{Três vezes
no ano todos os teus homens aparecerão diante do Senhor
\textsc{Deus}}. Não oferecerás o sangue do meu sacrifício com
pão levedado; nem ficará a gordura da minha festa de noite até pela
manhã. As primícias dos primeiros frutos da tua terra trarás
à casa do Senhor teu Deus; não cozerás o cabrito no leite de sua
mãe.

Eis que eu envio um anjo diante de ti, para que te guarde pelo
caminho, e te leve ao lugar que te tenho preparado. Guarda-te
diante dele, e ouve a sua voz, e não o provoques à ira; porque não
perdoará a vossa rebeldia; porque o meu nome está nele. Mas
se diligentemente ouvires a sua voz, e fizeres tudo o que eu disser,
então serei inimigo dos teus inimigos, e adversário dos teus
adversários. Porque o meu anjo irá adiante de ti, e te levará
aos amorreus, e aos heteus, e aos perizeus, e aos cananeus, heveus e
jebuseus; e eu os destruirei. Não te inclinarás diante dos
seus deuses, nem os servirás, nem farás conforme às suas obras;
antes os destruirás totalmente, e quebrarás de todo as suas
estátuas. E servireis ao Senhor vosso Deus, e ele abençoará o
vosso pão e a vossa água; e eu tirarei do meio de vós as
enfermidades. Não haverá mulher que aborte, nem estéril na
tua terra; \textbf{o número dos teus dias cumprirei}.
Enviarei o meu terror adiante de ti, destruindo a todo o povo
aonde entrares, e farei que todos os teus inimigos te voltem as
costas. Também enviarei vespões adiante de ti, que lancem
fora os heveus, os cananeus, e os heteus de diante de ti. Não
os lançarei fora de diante de ti num só ano, para que a terra não se
torne em deserto, e as feras do campo não se multipliquem contra ti.
Pouco a pouco os lançarei de diante de ti, até que sejas
multiplicado, e possuas a terra por herança. E porei os teus
termos desde o Mar Vermelho até ao mar dos filisteus, e desde o
deserto até ao rio; porque darei nas tuas mãos os moradores da
terra, para que os lances fora de diante de ti. Não farás
aliança alguma com eles, ou com os seus deuses. Na tua terra
não habitarão, para que não te façam pecar contra mim; se servires
aos seus deuses, certamente isso será um laço para ti.

\medskip

\lettrine{24} Depois disse a Moisés: Sobe ao Senhor, tu e
Arão, Nadabe e Abiú, e setenta dos anciãos de Israel; e adorai de
longe. E só Moisés se chegará ao Senhor; mas eles não se
cheguem, nem o povo suba com ele. Veio, pois, Moisés, e contou
ao povo todas as palavras do Senhor, e todos os estatutos; então o
povo respondeu a uma voz, e disse: Todas as palavras, que o Senhor
tem falado, faremos. Moisés escreveu todas as palavras do
Senhor, e levantou-se pela manhã de madrugada, e edificou um altar
ao pé do monte, e doze monumentos, segundo as doze tribos de Israel;
e enviou alguns jovens dos filhos de Israel, os quais ofereceram
holocaustos e sacrificaram ao Senhor sacrifícios pacíficos de
bezerros. E Moisés tomou a metade do sangue, e a pôs em bacias;
e a outra metade do sangue espargiu sobre o altar. E tomou o
livro da aliança e o leu aos ouvidos do povo, e eles disseram: Tudo
o que o Senhor tem falado faremos, e obedeceremos. Então tomou
Moisés aquele sangue, e espargiu-o sobre o povo, e disse:
\textbf{Eis aqui o sangue da aliança que o Senhor tem feito convosco
sobre todas estas palavras}.

E subiram Moisés e Arão, Nadabe e Abiú, e setenta dos anciãos de
Israel. E viram o Deus de Israel, e debaixo de seus pés havia
como que uma pavimentação de pedra de safira, que se parecia com o
céu na sua claridade. Porém não estendeu a sua mão sobre os
escolhidos dos filhos de Israel, mas viram a Deus, e comeram e
beberam.

Então disse o Senhor a Moisés: Sobe a mim ao monte, e fica lá; e
dar-te-ei as tábuas de pedra e a lei, e os mandamentos que tenho
escrito, para os ensinar. E levantou-se Moisés com Josué seu
servidor; e subiu Moisés ao monte de Deus. E disse aos
anciãos: Esperai-nos aqui, até que tornemos a vós; e eis que Arão e
Hur ficam convosco; quem tiver algum negócio, se chegará a eles.
E, subindo Moisés ao monte, a nuvem cobriu o monte. E
a glória do Senhor repousou sobre o monte Sinai, e a nuvem o cobriu
por seis dias; e ao sétimo dia chamou a Moisés do meio da nuvem.
E o parecer da glória do Senhor era como um fogo consumidor
no cume do monte, aos olhos dos filhos de Israel. E Moisés
entrou no meio da nuvem, depois que subiu ao monte; e Moisés esteve
no monte quarenta dias e quarenta noites.

\medskip

\lettrine{25} Então falou o Senhor a Moisés, dizendo: Fala
aos filhos de Israel, que me tragam uma oferta alçada\footnote{KJ:
Speak unto the children of Israel, that they bring me an offering:
of every man that giveth it willingly with his heart ye shall take
my offering. RA: somente \emph{oferta}. Entre outros significados,
alçar também significa: fazer louvação a; celebrar, exaltar,
enaltecer. Colocar (alguém ou algo) em posição de destaque, comando,
glória; nomear, eleger, entronizar Ex.: alçaram-no presidente.
Pronominal: alcançar posição de destaque em; elevar-se Ex.: alçou-se
à chefia da organização.}; de todo o homem cujo coração se mover
voluntariamente, dele tomareis a minha oferta alçada. E esta é a
oferta alçada que recebereis deles: ouro, e prata, e cobre, e
azul, e púrpura, e carmesim, e linho fino, e pêlos de cabras, e
peles de carneiros tintas de vermelho, e peles de texugos, e madeira
de acácia, azeite para a luz, especiarias para o óleo da unção,
e especiarias para o incenso, pedras de ônix\footnote{Variedade
de ágata entre cujas camadas se observa sensível destaque de cor.},
e pedras de engaste para o éfode e para o peitoral. E me farão
um santuário, e habitarei no meio deles. Conforme a tudo o que
eu te mostrar para modelo do tabernáculo, e para modelo de todos os
seus pertences, assim mesmo o fareis.

Também farão uma arca de madeira de acácia; o seu comprimento
será de dois côvados e meio, e a sua largura de um côvado e meio, e
de um côvado e meio a sua altura. E cobri-la-á de ouro puro;
por dentro e por fora a cobrirás; e farás sobre ela uma coroa de
ouro ao redor; e fundirás para ela quatro argolas de ouro, e
as porás nos quatro cantos dela, duas argolas num lado dela, e duas
argolas noutro lado. E farás varas de madeira de acácia, e as
cobrirás com ouro. E colocarás as varas nas argolas, aos
lados da arca, para se levar com elas a arca. As varas
estarão nas argolas da arca, não se tirarão dela. Depois
porás na arca o testemunho, que eu te darei. Também farás um
\textbf{propiciatório}\footnote{KJ: And thou shalt make a mercy seat
of pure gold.} de ouro puro; o seu comprimento será de dois côvados
e meio, e a sua largura de um côvado e meio. Farás também
dois querubins\footnote{Houaiss: anjos da primeira hierarquia, entre
os serafins e os tronos.} de ouro; de ouro batido os farás, nas duas
extremidades do propiciatório. Farás um querubim na
extremidade de uma parte, e o outro querubim na extremidade da outra
parte; de uma só peça com o propiciatório, fareis os querubins nas
duas extremidades dele. Os querubins estenderão as suas asas
por cima, cobrindo com elas o propiciatório; as faces deles uma
defronte da outra; as faces dos querubins estarão voltadas para o
propiciatório. E porás o propiciatório em cima da arca,
depois que houveres posto na arca o testemunho que eu te darei.
E ali virei a ti, e falarei contigo de cima do propiciatório,
do meio dos dois querubins (que estão sobre a arca do testemunho),
tudo o que eu te ordenar para os filhos de Israel.

Também farás uma mesa de madeira de acácia; o seu comprimento
será de dois côvados, e a sua largura de um côvado, e a sua altura
de um côvado e meio. E cobri-la-ás com ouro puro; também lhe
farás uma coroa de ouro ao redor. Também lhe farás uma
moldura ao redor, da largura de quatro dedos, e lhe farás uma coroa
de ouro ao redor da moldura. Também lhe farás quatro argolas
de ouro; e porás as argolas aos quatro cantos, que estão nos seus
quatro pés. Defronte da moldura estarão as argolas, como
lugares para os varais, para se levar a mesa. Farás, pois,
estes varais de madeira de acácia, e cobri-los-ás com ouro; e
levar-se-á com eles a mesa. Também farás os seus pratos, e as
suas colheres, e as suas cobertas, e as suas tigelas com que se hão
de oferecer libações; de ouro puro os farás. E sobre a mesa
porás o pão da proposição perante a minha face perpetuamente.

Também farás um candelabro de ouro puro; de ouro batido se fará
este candelabro; o seu pé, as suas hastes, os seus copos, os seus
botões, e as suas flores serão do mesmo. E dos seus lados
sairão seis hastes; três hastes do candelabro de um lado dele, e
três hastes do outro lado dele. Numa haste haverá três copos
a modo de amêndoas, um botão e uma flor; e três copos a modo de
amêndoas na outra haste, um botão e uma flor; assim serão as seis
hastes que saem do candelabro. Mas no candelabro mesmo haverá
quatro copos a modo de amêndoas, com seus botões e com suas flores;
e um botão debaixo de duas hastes que saem dele; e ainda um
botão debaixo de duas outras hastes que saem dele; e ainda um botão
debaixo de duas outras hastes que saem dele; assim se fará com as
seis hastes que saem do candelabro. Os seus botões e as suas
hastes serão do mesmo; tudo será de uma só peça, obra batida de ouro
puro. Também lhe farás sete lâmpadas, as quais se acenderão
para iluminar defronte dele. Os seus espevitadores e os seus
apagadores serão de ouro puro. De um talento de ouro puro os
farás, com todos estes vasos. Atenta, pois, que o faças
conforme ao seu modelo, que te foi mostrado no monte.

\medskip

\lettrine{26} E o tabernáculo farás de dez cortinas de linho
fino torcido, e azul, púrpura\footnote{Houaiss: cor vibrante
vermelho-escura, tendente para o roxo. Substância corante,
vermelho-escura, extraída de moluscos. Tecido vermelho, tingido com
essa substância --- era muito valorizado na Antiguidade e na Idade
Média, dava status e era símbolo do poder real e eclesiástico.}, e
carmesim\footnote{A cor vermelha do carmim. Que tem a cor do carmim.
Carmim: substância corante, em vermelho vivo, extraída da
cochonilha-do-carmim; magenta. A cor desse corante; carmesim,
magenta.}; com querubins as farás de obra esmerada. O
comprimento de uma cortina será de vinte e oito côvados, e a largura
de uma cortina de quatro côvados; todas estas cortinas serão de uma
medida. Cinco cortinas se enlaçarão uma à outra; e as outras
cinco cortinas se enlaçarão uma com a outra. E farás laçadas de
azul na orla de uma cortina, na extremidade, e na juntura; assim
também farás na orla da extremidade da outra cortina, na segunda
juntura. Cinqüenta laçadas farás numa cortina, e outras
cinqüenta laçadas farás na extremidade da cortina que está na
segunda juntura; as laçadas estarão presas uma com a outra.
Farás também cinqüenta colchetes de ouro, e ajuntarás com estes
colchetes as cortinas, uma com a outra, e será um tabernáculo.

Farás também cortinas de pêlos de cabras para servirem de tenda
sobre o tabernáculo; onze cortinas farás. O comprimento de uma
cortina será de trinta côvados, e a largura da mesma cortina de
quatro côvados; estas onze cortinas serão da mesma medida. E
juntarás cinco destas cortinas à parte, e as outras seis cortinas
também à parte; e dobrarás a sexta cortina à frente da tenda.
E farás cinqüenta laçadas na borda de uma cortina, na
extremidade, na juntura, e outras cinqüenta laçadas na borda da
outra cortina, na segunda juntura. Farás também cinqüenta
colchetes de cobre, e colocarás os colchetes nas laçadas, e assim
ajuntarás a tenda, para que seja uma. E a parte que sobejar
das cortinas da tenda, a saber, a metade da cortina que sobejar,
penderá de sobra às costas do tabernáculo. E um côvado de um
lado, e outro côvado do outro, que sobejará no comprimento das
cortinas da tenda, penderá de sobra aos lados do tabernáculo de um e
de outro lado, para cobri-lo. Farás também à tenda uma
coberta de peles de carneiro, tintas de vermelho, e outra coberta de
peles de texugo em cima.

Farás também as tábuas para o tabernáculo de madeira de acácia,
que serão postas verticalmente. O comprimento de uma tábua
será de dez côvados, e a largura de cada tábua será de um côvado e
meio. Dois encaixes terá cada tábua, travados um com o outro;
assim farás com todas as tábuas do tabernáculo. E farás as
tábuas para o tabernáculo assim: vinte tábuas para o lado
meridional. Farás também quarenta bases de prata debaixo das
vinte tábuas; duas bases debaixo de uma tábua para os seus dois
encaixes e duas bases debaixo de outra tábua para os seus dois
encaixes. Também haverá vinte tábuas ao outro lado do
tabernáculo, para o lado norte, com as suas quarenta bases de
prata; duas bases debaixo de uma tábua, e duas bases debaixo de
outra tábua, e ao lado do tabernáculo para o ocidente farás
seis tábuas. Farás também duas tábuas para os cantos do
tabernáculo, de ambos os lados. E por baixo se ajuntarão, e
também em cima dele se ajuntarão numa argola. Assim se fará com as
duas tábuas; ambas serão por tábuas para os dois cantos.
Assim serão as oito tábuas com as suas bases de prata,
dezesseis bases; duas bases debaixo de uma tábua, e duas bases
debaixo da outra tábua. Farás também cinco travessas de
madeira de acácia, para as tábuas de um lado do tabernáculo,
e cinco travessas para as tábuas do outro lado do
tabernáculo; como também cinco travessas para as tábuas do outro
lado do tabernáculo, de ambos os lados, para o ocidente. E a
travessa central estará no meio das tábuas, passando de uma
extremidade até à outra. E cobrirás de ouro as tábuas, e
farás de ouro as suas argolas, para passar por elas as travessas;
também as travessas as cobrirás de ouro. Então levantarás o
tabernáculo conforme ao modelo que te foi mostrado no monte.

Depois farás um véu de azul, e púrpura, e carmesim, e de linho
fino torcido; com querubins de obra prima se fará. E
colocá-lo-ás sobre quatro colunas de madeira de acácia, cobertas de
ouro; seus colchetes serão de ouro, sobre quatro bases de prata.
Pendurarás o véu debaixo dos colchetes, e porás a arca do
testemunho ali dentro do véu; e este véu vos fará separação entre o
santuário e o lugar santíssimo, e porás a coberta do
propiciatório sobre a arca do testemunho no lugar santíssimo,
e a mesa porás fora do véu, e o candelabro defronte da mesa,
ao lado do tabernáculo, para o sul; mas a mesa porás ao lado do
norte. Farás também para a porta da tenda, uma cortina de
azul, e púrpura, e carmesim, e de linho fino torcido, de obra de
bordador. E farás para esta cortina cinco colunas de madeira
de acácia, e as cobrirás de ouro; seus colchetes serão de ouro, e
far-lhe-ás de fundição cinco bases de cobre.

\medskip

\lettrine{27} Farás também o altar de madeira de acácia; cinco
côvados será o comprimento, e cinco côvados a largura (será quadrado
o altar), e três côvados a sua altura. E farás as suas pontas
nos seus quatro cantos; as suas pontas serão do mesmo, e o cobrirás
de cobre. Far-lhe-ás também os seus recipientes, para recolher a
sua cinza, e as suas pás, e as suas bacias, e os seus garfos e os
seus braseiros; todos os seus utensílios farás de cobre.
Far-lhe-ás também um crivo de cobre em forma de rede, e farás a
esta rede quatro argolas de metal nos seus quatro cantos. E as
porás dentro da borda do altar para baixo, de maneira que a rede
chegue até ao meio do altar. Farás também varais para o altar,
varais de madeira de acácia, e os cobrirás de cobre. E os varais
serão postos nas argolas, de maneira que os varais estejam de ambos
os lados do altar, quando for levado. Oco e de tábuas o farás;
como se te mostrou no monte, assim o farão.

Farás também o pátio do tabernáculo, ao lado meridional que dá
para o sul; o pátio terá cortinas de linho fino torcido; o
comprimento de cada lado será de cem côvados. Também as suas
vinte colunas e as suas vinte bases serão de cobre; os colchetes das
colunas e as suas faixas serão de prata. Assim também para o
lado norte as cortinas, no comprimento, serão de cem côvados; e as
suas vinte colunas e as suas vinte bases serão de cobre; os
colchetes das colunas e as suas faixas serão de prata, e na
largura do pátio para o lado do ocidente haverá cortinas de
cinqüenta côvados; as suas colunas dez, e as suas bases dez.
Semelhantemente a largura do pátio do lado oriental para o
levante será de cinqüenta côvados. De maneira que haja quinze
côvados de cortinas de um lado; suas colunas três, e as suas bases
três. E quinze côvados das cortinas do outro lado; as suas
colunas três, e as suas bases três. E à porta do pátio haverá
uma cortina de vinte côvados, de azul, e púrpura, e carmesim, e de
linho fino torcido, de obra de bordador; as suas colunas quatro, e
as suas bases quatro. Todas as colunas do pátio ao redor
serão cingidas de faixas de prata; os seus colchetes serão de prata,
mas as suas bases de cobre. O comprimento do pátio será de
cem côvados, e a largura de cada lado de cinqüenta, e a altura de
cinco côvados, as cortinas serão de linho fino torcido; mas as suas
bases serão de cobre. No tocante a todos os vasos do
tabernáculo em todo o seu serviço, até todos os seus pregos, e todos
os pregos do pátio, serão de cobre.

Tu pois ordenarás aos filhos de Israel que te tragam azeite puro
de oliveiras, batido, para o candeeiro, para fazer arder as lâmpadas
continuamente. Na tenda da congregação, fora do véu que está
diante do testemunho, Arão e seus filhos as porão em ordem, desde a
tarde até a manhã, perante o Senhor; isto será um estatuto perpétuo
para os filhos de Israel, pelas suas gerações.

\medskip

\lettrine{28} Depois tu farás chegar a ti teu irmão Arão, e
seus filhos com ele, do meio dos filhos de Israel, para me
administrarem o ofício sacerdotal; a saber: Arão, Nadabe, e Abiú,
Eleazar e Itamar, os filhos de Arão. E farás vestes sagradas a
Arão teu irmão, para glória e ornamento. Falarás também a todos
os que são sábios de coração, a quem eu tenho enchido do espírito da
sabedoria, que façam vestes a Arão para santificá-lo; para que me
administre o ofício sacerdotal. Estas pois são as vestes que
farão: um peitoral, e um éfode, e um manto, e uma túnica bordada,
uma mitra, e um cinto; farão, pois, santas vestes para Arão, teu
irmão, e para seus filhos, para me administrarem o ofício
sacerdotal. E tomarão o ouro, e o azul, e a púrpura, e o
carmesim, e o linho fino, e farão o éfode de ouro, e de azul, e
de púrpura, e de carmesim, e de linho fino torcido, de obra
esmerada. Terá duas ombreiras, que se unam às suas duas pontas,
e assim se unirá. E o cinto de obra esmerada do seu éfode, que
estará sobre ele, será da sua mesma obra, igualmente, de ouro, de
azul, e de púrpura, e de carmesim, e de linho fino torcido. E
tomarás duas pedras de ônix, e gravarás nelas os nomes dos filhos de
Israel, seis dos seus nomes numa pedra, e os outros seis
nomes na outra pedra, segundo as suas gerações; conforme à
obra do lapidário, como o lavor\footnote{Trabalho; qualquer trabalho
manual; lavor de caráter artístico ou artesanal, executado com
cuidado e habilidade; lavor ornado de desenhos, entalhes, etc.;
lavrado.} de selos lavrarás estas duas pedras, com os nomes dos
filhos de Israel; engastadas ao redor em ouro as farás. E
porás as duas pedras nas ombreiras do éfode, por pedras de memória
para os filhos de Israel; e Arão levará os seus nomes sobre ambos os
seus ombros, para memória diante do Senhor. Farás também
engastes de ouro, e duas cadeiazinhas de ouro puro; de igual
medida, de obra de fieira as farás; e as cadeiazinhas de fieira
porás nos engastes.

Farás também o peitoral do juízo de obra esmerada, conforme à
obra do éfode o farás; de ouro, de azul, e de púrpura, e de
carmesim, e de linho fino torcido o farás. Quadrado e duplo,
será de um palmo o seu comprimento, e de um palmo a sua largura.
E o encherás de pedras de engaste, com quatro ordens de
pedras; a ordem de um sárdio\footnote{Variedade castanha de
calcedônia.}, de um topázio, e de um carbúnculo\footnote{Antiga
designação da granada almandina, lapidada em cabucho.}; esta será a
primeira ordem; e a segunda ordem será de uma esmeralda, de
uma safira, e de um diamante; e a terceira ordem será de um
jacinto, de uma ágata, e de uma ametista; e a quarta ordem
será de um berilo\footnote{Mineral hexagonal, silicato de alumínio e
glucínio, pedra semipreciosa.}, e de um ônix, e de um
jaspe\footnote{Variedade semicristalina de quartzo opaco, de cores
diversas, sendo a cor mais comum a vermelha.}; engastadas em ouro
serão nos seus engastes. E serão aquelas pedras segundo os
nomes dos filhos de Israel, doze segundo os seus nomes; serão
esculpidas como selos, cada uma com o seu nome, para as doze tribos.
Também farás para o peitoral cadeiazinhas de igual medida,
obra trançada de ouro puro. Também farás para o peitoral dois
anéis de ouro, e porás os dois anéis nas extremidades do peitoral.
Então porás as duas cadeiazinhas de fieira de ouro nos dois
anéis, nas extremidades do peitoral; e as duas pontas das
duas cadeiazinhas de fieira colocarás nos dois engastes, e as porás
nas ombreiras do éfode, na frente dele. Farás também dois
anéis de ouro, e os porás nas duas extremidades do peitoral, na sua
borda que estiver junto ao éfode por dentro. Farás também
dois anéis de ouro, que porás nas duas ombreiras do éfode, abaixo,
na frente dele, perto da sua juntura, sobre o cinto de obra esmerada
do éfode. E ligarão o peitoral, com os seus anéis, aos anéis
do éfode por cima, com um cordão de azul, para que esteja sobre o
cinto de obra esmerada do éfode; e nunca se separará o peitoral do
éfode. Assim Arão levará os nomes dos filhos de Israel no
peitoral do juízo sobre o seu coração, quando entrar no santuário,
para memória diante do Senhor continuamente. Também porás no
peitoral do juízo Urim e Tumim, para que estejam sobre o coração de
Arão, quando entrar diante do Senhor: assim Arão levará o juízo dos
filhos de Israel sobre o seu coração diante do Senhor continuamente.

Também farás o manto do éfode, todo de azul. E a abertura
da cabeça estará no meio dele; esta abertura terá uma borda de obra
tecida ao redor; como abertura de cota de malha será, para que não
se rompa. E nas suas bordas farás romãs de azul, e de
púrpura, e de carmesim, ao redor das suas bordas; e campainhas de
ouro no meio delas ao redor. Uma campainha de ouro, e uma
romã, outra campainha de ouro, e outra romã, haverá nas bordas do
manto ao redor, e estará sobre Arão quando ministrar, para
que se ouça o seu sonido, quando entrar no santuário diante do
Senhor, e quando sair, para que não morra. Também farás uma
lâmina de ouro puro, e nela gravarás como as gravuras de selos:
SANTIDADE AO SENHOR. E atá-la-ás com um cordão de azul, de
modo que esteja na mitra, na frente da mitra estará; e estará
sobre a testa de Arão, para que Arão leve a iniqüidade das coisas
santas, que os filhos de Israel santificarem em todas as ofertas de
suas coisas santas; e estará continuamente na sua testa, para que
tenham aceitação perante o Senhor. Também farás túnica de
linho fino; também farás uma mitra de linho fino; mas o cinto farás
de obra de bordador.

Também farás túnicas aos filhos de Arão, e far-lhes-ás cintos;
também lhes farás tiaras, para glória e ornamento. E vestirás
com eles a Arão, teu irmão, e também seus filhos; e os ungirás e
consagrarás, e os santificarás, para que me administrem o
sacerdócio. Faze-lhes também calções de linho, para cobrirem
a carne nua; irão dos lombos até as coxas. E estarão sobre
Arão e sobre seus filhos, quando entrarem na tenda da congregação,
ou quando chegarem ao altar para ministrar no santuário, para que
não levem iniqüidade e morram; isto será estatuto perpétuo para ele
e para a sua descendência depois dele.

\medskip

\lettrine{29} Isto é o que lhes hás de fazer, para os
santificar, para que me administrem o sacerdócio: Toma um novilho e
dois carneiros sem mácula, e pão ázimo, e bolos ázimos,
amassados com azeite, e coscorões\footnote{Filhó (Bolinho ou
biscoito de farinha e de ovos, frito em azeite, e que se come
polvilhado com açúcar e canela ou passado por calda de açúcar) de
farinha e de ovos.} ázimos, untados com azeite; com flor de farinha
de trigo os farás, e os porás num cesto, e os trarás no cesto,
com o novilho e os dois carneiros. Então farás chegar a Arão e a
seus filhos à porta da tenda da congregação, e os lavarás com água;
depois tomarás as vestes, e vestirás a Arão da túnica e do manto
do éfode, e do éfode, e do peitoral; e o cingirás com o cinto de
obra de artífice do éfode. E a mitra porás sobre a sua cabeça; a
coroa da santidade porás sobre a mitra. E tomarás o azeite da
unção, e o derramarás sobre a sua cabeça; assim o ungirás.
Depois farás chegar seus filhos, e lhes farás vestir túnicas.
E os cingirás com o cinto, a Arão e a seus filhos, e lhes atarás
as tiaras, para que tenham o sacerdócio por estatuto perpétuo, e
consagrarás a Arão e a seus filhos; e farás chegar o novilho
diante da tenda da congregação, e Arão e seus filhos porão as suas
mãos sobre a cabeça do novilho; e imolarás o novilho perante
o Senhor, à porta da tenda da congregação. Depois tomarás do
sangue do novilho, e o porás com o teu dedo sobre as pontas do
altar, e todo o sangue restante derramarás à base do altar.
Também tomarás toda a gordura que cobre as entranhas, e o
redenho\footnote{Epíploo; prega peritonial que se estende entre dois
órgãos viscerais abdominais, como, p. ex., o epíploo gastrocólico,
que se insere no estômago e no cólon.} de sobre o fígado, e ambos os
rins, e a gordura que houver neles, e queimá-los-ás sobre o altar;
mas a carne do novilho, e a sua pele, e o seu esterco
queimarás com fogo fora do arraial; é sacrifício pelo pecado.
Depois tomarás um carneiro, e Arão e seus filhos porão as
suas mãos sobre a cabeça do carneiro, e imolarás o carneiro,
e tomarás o seu sangue, e o espalharás sobre o altar ao redor;
e partirás o carneiro por suas partes, e lavarás as suas
entranhas e as suas pernas, e as porás sobre as suas partes e sobre
a sua cabeça. Assim queimarás todo o carneiro sobre o altar;
é um holocausto para o Senhor, cheiro suave; uma oferta queimada ao
Senhor. Depois tomarás o outro carneiro, e Arão e seus filhos
porão as suas mãos sobre a sua cabeça; e imolarás o carneiro
e tomarás do seu sangue, e o porás sobre a ponta da orelha direita
de Arão, e sobre as pontas das orelhas direitas de seus filhos, como
também sobre os dedos polegares das suas mãos direitas, e sobre os
dedos polegares dos seus pés direitos; e o restante do sangue
espalharás sobre o altar ao redor; então tomarás do sangue,
que estará sobre o altar, e do azeite da unção, e o espargirás sobre
Arão e sobre as suas vestes, e sobre seus filhos, e sobre as vestes
de seus filhos com ele; para que ele seja santificado, e as suas
vestes, também seus filhos, e as vestes de seus filhos com ele.
Depois tomarás do carneiro a gordura, e a cauda, e a gordura
que cobre as entranhas, e o redenho do fígado, e ambos os rins com a
gordura que houver neles, e o ombro direito, porque é carneiro das
consagrações; e um pão, e um bolo de pão azeitado, e um
coscorão do cesto dos pães ázimos que estão diante do Senhor.
E tudo porás nas mãos de Arão, e nas mãos de seus filhos; e
com movimento oferecerás perante o Senhor. Depois o tomarás
das suas mãos e o queimarás no altar sobre o holocausto por cheiro
suave perante o Senhor; é oferta queimada ao Senhor. E
tomarás o peito do carneiro das consagrações, que é de Arão, e com
movimento oferecerás perante o Senhor; e isto será a tua porção.
E santificarás o peito da oferta de movimento e o ombro da
oferta alçada, que foi movido e alçado do carneiro das consagrações,
que for de Arão e de seus filhos. E será para Arão e para
seus filhos por estatuto perpétuo dos filhos de Israel, porque é
oferta alçada; e a oferta alçada será dos filhos de Israel, dos seus
sacrifícios pacíficos; a sua oferta alçada será para o Senhor.
E as vestes sagradas, que são de Arão, serão de seus filhos
depois dele, para serem ungidos com elas para serem consagrados com
elas. Sete dias as vestirá aquele que de seus filhos for
sacerdote em seu lugar, quando entrar na tenda da congregação para
ministrar no santuário. E tomarás o carneiro das consagrações
e cozerás a sua carne no lugar santo; e Arão e seus filhos
comerão a carne deste carneiro, e o pão que está no cesto, à porta
da tenda da congregação. E comerão as coisas com que for
feita expiação, para consagrá-los, e para santificá-los; mas o
estranho delas não comerá, porque são santas. E se sobejar
alguma coisa da carne das consagrações ou do pão até pela manhã, o
que sobejar queimarás com fogo; não se comerá, porque é santo.
Assim, pois, farás a Arão e a seus filhos conforme a tudo o
que eu te tenho ordenado; por sete dias os consagrarás.
Também cada dia prepararás um novilho por sacrifício pelo
pecado para as expiações, e purificarás o altar, fazendo expiação
sobre ele; e o ungirás para santificá-lo. Sete dias farás
expiação pelo altar, e o santificarás; e o altar será santíssimo;
tudo o que tocar o altar será santo.

Isto, pois, é o que oferecereis sobre o altar: dois cordeiros de
um ano, cada dia, continuamente. Um cordeiro oferecerás pela
manhã, e o outro cordeiro oferecerás à tarde. Com um cordeiro
a décima parte de flor de farinha, misturada com a quarta parte de
um him de azeite batido, e para libação a quarta parte de um him de
vinho, e o outro cordeiro oferecerás à tarde, e com ele farás
como com a oferta da manhã, e conforme à sua libação, por cheiro
suave; oferta queimada é ao Senhor. Este será o holocausto
contínuo por vossas gerações, à porta da tenda da congregação,
perante o Senhor, onde vos encontrarei, para falar contigo ali.
E ali virei aos filhos de Israel, para que por minha glória
sejam santificados. E santificarei a tenda da congregação e o
altar; também santificarei a Arão e seus filhos, para que me
administrem o sacerdócio. E habitarei no meio dos filhos de
Israel, e lhes serei o seu Deus, e saberão que eu sou o
Senhor seu Deus, que os tenho tirado da terra do Egito, para habitar
no meio deles. Eu sou o Senhor seu Deus.

\medskip

\lettrine{30} E farás um altar para queimar o incenso; de
madeira de acácia o farás. O seu comprimento será de um côvado,
e a sua largura de um côvado; será quadrado, e dois côvados a sua
altura; dele mesmo serão as suas pontas. E com ouro puro o
forrarás, o seu teto, e as suas paredes ao redor, e as suas pontas;
e lhe farás uma coroa de ouro ao redor. Também lhe farás duas
argolas de ouro debaixo da sua coroa; nos dois cantos as farás, de
ambos os lados; e serão para lugares dos varais, com que será
levado. E os varais farás de madeira de acácia, e os forrarás
com ouro. E o porás diante do véu que está diante da arca do
testemunho, diante do propiciatório, que está sobre o testemunho,
onde me ajuntarei contigo. E Arão sobre ele queimará o incenso
das especiarias; cada manhã, quando puser em ordem as lâmpadas, o
queimará. E, acendendo Arão as lâmpadas à tarde, o queimará;
este será incenso contínuo perante o Senhor pelas vossas gerações.
Não oferecereis sobre ele incenso estranho, nem holocausto, nem
oferta; nem tampouco derramareis sobre ele libações. E uma
vez no ano Arão fará expiação sobre as suas pontas com o sangue do
sacrifício das expiações; uma vez no ano fará expiação sobre ele
pelas vossas gerações; santíssimo é ao Senhor.

Falou mais o Senhor a Moisés dizendo: Quando fizeres a
contagem dos filhos de Israel, conforme a sua soma, cada um deles
dará ao Senhor o resgate da sua alma, quando os contares; para que
não haja entre eles praga alguma, quando os contares. Todo
aquele que passar pelo arrolamento dará isto: a metade de um siclo,
segundo o siclo do santuário (este siclo é de vinte geras); a metade
de um siclo é a oferta ao Senhor. Qualquer que passar pelo
arrolamento, de vinte anos para cima, dará a oferta alçada ao
Senhor. O rico não dará mais, e o pobre não dará menos da
metade do siclo, quando derem a oferta alçada ao Senhor, para fazer
expiação por vossas almas. E tomarás o dinheiro das expiações
dos filhos de Israel, e o darás ao serviço da tenda da congregação;
e será para memória aos filhos de Israel diante do Senhor, para
fazer expiação por vossas almas.

E falou o Senhor a Moisés, dizendo: Farás também uma pia
de cobre com a sua base de cobre, para lavar; e a porás entre a
tenda da congregação e o altar; e nela deitarás água. E Arão
e seus filhos nela lavarão as suas mãos e os seus pés. Quando
entrarem na tenda da congregação, lavar-se-ão com água, para que não
morram, ou quando se chegarem ao altar para ministrar, para acender
a oferta queimada ao Senhor. Lavarão, pois, as suas mãos e os
seus pés, para que não morram; e isto lhes será por estatuto
perpétuo a ele e à sua descendência nas suas gerações.

Falou mais o Senhor a Moisés, dizendo: Tu, pois, toma para
ti das principais especiarias, da mais pura mirra quinhentos siclos,
e de canela aromática a metade, a saber, duzentos e cinqüenta
siclos, e de cálamo aromático duzentos e cinqüenta siclos, e
de cássia quinhentos siclos, segundo o siclo do santuário, e de
azeite de oliveiras um him. E disto farás o azeite da santa
unção, o perfume composto segundo a obra do perfumista: este será o
azeite da santa unção. E com ele ungirás a tenda da
congregação, e a arca do testemunho, e a mesa com todos os
seus utensílios, e o candelabro com os seus utensílios, e o altar do
incenso. E o altar do holocausto com todos os seus
utensílios, e a pia com a sua base. Assim santificarás estas
coisas, para que sejam santíssimas; tudo o que tocar nelas será
santo. Também ungirás a Arão e seus filhos, e os santificarás
para me administrarem o sacerdócio. E falarás aos filhos de
Israel, dizendo: Este me será o azeite da santa unção nas vossas
gerações. Não se ungirá com ele a carne do homem, nem fareis
outro de semelhante composição; santo é, e será santo para vós.
O homem que compuser um perfume como este, ou dele puser
sobre um estranho, será extirpado do seu povo. Disse mais o
Senhor a Moisés: Toma especiarias aromáticas,
estoraque\footnote{Arbusto ornamental, de origem asiática, da
família das estiracáceas (Styrax benjoin), que produz o benjoim
(Bálsamo aromático, amarelo, extraído do estoraque, utilizado na
fabricação de perfumes e em medicina).}, e onicha\footnote{Ônica.},
e gálbano\footnote{SBTB: galbano, sem acento. Planta umbelífera,
sempre verde (Ferula galbaniflua); Resina medicinal que se extrai
dessa planta.}; estas especiarias aromáticas e o incenso puro, em
igual proporção; e disto farás incenso, um perfume segundo a
arte do perfumista, temperado, puro e santo; e uma parte dele
moerás, e porás diante do testemunho, na tenda da congregação, onde
eu virei a ti; coisa santíssima vos será. Porém o incenso que
fareis conforme essa composição, não o fareis para vós mesmos; santo
será para o Senhor. O homem que fizer tal como este para
cheirar, será extirpado do seu povo.

\medskip

\lettrine{31} Depois falou o Senhor a Moisés, dizendo: Eis
que eu tenho chamado por nome a Bezalel, o filho de Uri, filho de
Hur, da tribo de Judá, e \textbf{o enchi do Espírito de Deus},
de sabedoria, e de entendimento, e de ciência, em todo o lavor,
para elaborar projetos, e trabalhar em ouro, em prata, e em
cobre, e em lapidar pedras para engastar, e em entalhes de
madeira, para trabalhar em todo o lavor. E eis que eu tenho
posto com ele a Aoliabe, o filho de Aisamaque, da tribo de Dã, e
tenho dado sabedoria ao coração de todos aqueles que são hábeis,
para que façam tudo o que te tenho ordenado. A saber: a tenda da
congregação, e a arca do testemunho, e o propiciatório que estará
sobre ela, e todos os pertences da tenda; e a mesa com os seus
utensílios, e o candelabro de ouro puro com todos os seus pertences,
e o altar do incenso; e o altar do holocausto com todos os seus
utensílios, e a pia com a sua base; e as vestes do
ministério, e as vestes sagradas de Arão o sacerdote, e as vestes de
seus filhos, para administrarem o sacerdócio; e o azeite da
unção, e o incenso aromático para o santuário; farão conforme a tudo
que te tenho mandado.

Falou mais o Senhor a Moisés, dizendo: Tu, pois, fala aos
filhos de Israel, dizendo: \textbf{Certamente guardareis meus
sábados}; porquanto isso é um sinal entre mim e vós nas vossas
gerações; para que saibais que eu sou o Senhor, que vos santifica.
Portanto guardareis o sábado, porque santo é para vós; aquele
que o profanar certamente morrerá; porque qualquer que nele fizer
alguma obra, aquela alma será eliminada do meio do seu povo.
Seis dias se trabalhará, porém o sétimo dia é o sábado do
descanso, santo ao Senhor; qualquer que no dia do sábado fizer algum
trabalho, certamente morrerá. Guardarão, pois, o sábado os
filhos de Israel, celebrando-o nas suas gerações por aliança
perpétua. Entre mim e os filhos de Israel será um sinal para
sempre; porque \textbf{em seis dias fez o Senhor os céus e a terra,
e ao sétimo dia descansou, e restaurou-se}. E deu a Moisés
(quando acabou de falar com ele no monte Sinai) as duas tábuas do
testemunho, tábuas de pedra, \textbf{escritas pelo dedo de Deus}.

\medskip

\lettrine{32} Mas vendo o povo que Moisés tardava em descer do
monte, acercou-se de Arão, e disse-lhe: Levanta-te, faze-nos deuses,
que vão adiante de nós; porque quanto a este Moisés, o homem que nos
tirou da terra do Egito, não sabemos o que lhe sucedeu. E Arão
lhes disse: Arrancai os pendentes de ouro, que estão nas orelhas de
vossas mulheres, e de vossos filhos, e de vossas filhas, e
trazei-mos. Então todo o povo arrancou os pendentes de ouro, que
estavam nas suas orelhas, e os trouxeram a Arão. E ele os tomou
das suas mãos, e trabalhou o ouro com um buril, e fez dele um
bezerro de fundição. Então disseram: Este é teu deus, ó Israel, que
te tirou da terra do Egito. E Arão, vendo isto, edificou um
altar diante dele; e apregoou Arão, e disse: Amanhã será festa ao
Senhor. E no dia seguinte madrugaram, e ofereceram holocaustos,
e trouxeram ofertas pacíficas; e o povo assentou-se a comer e a
beber; depois levantou-se a folgar.

Então disse o Senhor a Moisés: Vai, desce; porque o teu povo, que
fizeste subir do Egito, se tem corrompido, e depressa se tem
desviado do caminho que eu lhe tinha ordenado; eles fizeram para si
um bezerro de fundição, e perante ele se inclinaram, e
ofereceram-lhe sacrifícios, e disseram: Este é o teu deus, ó Israel,
que te tirou da terra do Egito. Disse mais o Senhor a Moisés:
Tenho visto a este povo, e eis que é povo de dura cerviz\footnote{A
parte posterior do pescoço; nuca.}. Agora, pois, deixa-me,
para que o meu furor se acenda contra ele, e o consuma; e eu farei
de ti uma grande nação. Moisés, porém, suplicou ao Senhor seu
Deus e disse: Ó Senhor, por que se acende o teu furor contra o teu
povo, que tiraste da terra do Egito com grande força e com forte
mão? Por que hão de falar os egípcios, dizendo: Para mal os
tirou, para matá-los nos montes, e para destruí-los da face da
terra? Torna-te do furor da tua ira, e arrepende-te deste mal contra
o teu povo. Lembra-te de Abraão, de Isaque, e de Israel, os
teus servos, aos quais por ti mesmo tens jurado, e lhes disseste:
Multiplicarei a vossa descendência como as estrelas dos céus, e
darei à vossa descendência toda esta terra, de que tenho falado,
para que a possuam por herança eternamente. Então o Senhor
arrependeu-se do mal que dissera que havia de fazer ao seu povo.

E virou-se Moisés e desceu do monte com as duas tábuas do
testemunho na mão, tábuas escritas de ambos os lados; de um e de
outro lado estavam escritas. E aquelas tábuas eram obra de
Deus; também a escritura era a mesma escritura de Deus, esculpida
nas tábuas. E, ouvindo Josué a voz do povo que jubilava,
disse a Moisés: Alarido de guerra há no arraial. Porém ele
respondeu: Não é alarido dos vitoriosos, nem alarido dos vencidos,
mas o alarido dos que cantam, eu ouço. E aconteceu que,
chegando Moisés ao arraial, e vendo o bezerro e as danças,
acendeu-se-lhe o furor, e arremessou as tábuas das suas mãos, e
quebrou-as ao pé do monte; e tomou o bezerro que tinham
feito, e queimou-o no fogo, moendo-o até que se tornou em pó; e o
espargiu sobre as águas, e deu-o a beber aos filhos de Israel.

E Moisés perguntou a Arão: Que te tem feito este povo, que sobre
ele trouxeste tamanho pecado? Então respondeu Arão: Não se
acenda a ira do meu senhor; tu sabes que este povo é inclinado ao
mal; e eles me disseram: Faze-nos um deus que vá adiante de
nós; porque não sabemos o que sucedeu a este Moisés, a este homem
que nos tirou da terra do Egito. Então eu lhes disse: Quem
tem ouro, arranque-o; e deram-mo, e lancei-o no fogo, e saiu este
bezerro. E, vendo Moisés que o povo estava despido, porque
Arão o havia deixado despir-se para vergonha entre os seus inimigos,
pôs-se em pé Moisés na porta do arraial e disse: Quem é do
Senhor, venha a mim. Então se ajuntaram a ele todos os filhos de
Levi. E disse-lhes: Assim diz o Senhor Deus de Israel: Cada
um ponha a sua espada sobre a sua coxa; e passai e tornai pelo
arraial de porta em porta, e mate cada um a seu irmão, e cada um a
seu amigo, e cada um a seu vizinho. E os filhos de Levi
fizeram conforme à palavra de Moisés; e caíram do povo aquele dia
uns três mil homens. Porquanto Moisés tinha dito: Consagrai
hoje as vossas mãos ao Senhor; porquanto cada um será contra o seu
filho e contra o seu irmão; e isto, para que ele vos conceda hoje
uma bênção.

E aconteceu que no dia seguinte Moisés disse ao povo: Vós
cometestes grande pecado. Agora, porém, subirei ao Senhor;
porventura farei propiciação por vosso pecado. Assim
tornou-se Moisés ao Senhor, e disse: Ora, este povo cometeu grande
pecado fazendo para si deuses de ouro. Agora, pois, perdoa o
seu pecado, se não, risca-me, peço-te, do teu livro, que tens
escrito. Então disse o Senhor a Moisés: Aquele que pecar
contra mim, a este riscarei do meu livro. Vai, pois, agora,
conduze este povo para onde te tenho dito; eis que o meu anjo irá
adiante de ti; porém \textbf{no dia da minha visitação visitarei
neles o seu pecado}. Assim feriu o Senhor o povo, por ter
sido feito o bezerro que Arão tinha formado.

\medskip

\lettrine{33} Disse mais o Senhor a Moisés: Vai, sobe daqui,
tu e o povo que fizeste subir da terra do Egito, à terra que jurei a
Abraão, a Isaque, e a Jacó, dizendo: À tua descendência a darei.
E enviarei um anjo adiante de ti, e lançarei fora os cananeus, e
os amorreus, e os heteus, e os perizeus, e os heveus, e os jebuseus,
a uma terra que mana leite e mel; porque eu não subirei no meio
de ti, porquanto és povo de dura cerviz, para que te não consuma eu
no caminho. E, ouvindo o povo esta má notícia, pranteou-se e
ninguém pôs sobre si os seus atavios\footnote{Adorno ou enfeite
requintado ou vistoso.}. Porquanto o Senhor tinha dito a Moisés:
Dize aos filhos de Israel: És povo de dura cerviz; se por um momento
subir no meio de ti, te consumirei; porém agora tira os teus
atavios, para que eu saiba o que te hei de fazer. Então os
filhos de Israel se despojaram dos seus atavios, ao pé do monte
Horebe.

E tomou Moisés a tenda, e a estendeu para si fora do arraial,
desviada longe do arraial, e chamou-lhe a tenda da congregação. E
aconteceu que todo aquele que buscava o Senhor saía à tenda da
congregação, que estava fora do arraial. E acontecia que, saindo
Moisés à tenda, todo o povo se levantava, e cada um ficava em pé à
porta da sua tenda; e olhava para Moisés pelas costas, até ele
entrar na tenda. E sucedia que, entrando Moisés na tenda, descia
a coluna de nuvem, e punha-se à porta da tenda; e o Senhor falava
com Moisés. E, vendo todo o povo a coluna de nuvem que estava
à porta da tenda, todo o povo se levantava e cada um, à porta da sua
tenda, adorava. E \textbf{falava o Senhor a Moisés face a
face, como qualquer fala com o seu amigo}; depois tornava-se ao
arraial; mas o seu servidor, o jovem Josué, filho de Num, nunca se
apartava do meio da tenda.

E Moisés disse ao Senhor: Eis que tu me dizes: Faze subir a este
povo, porém não me fazes saber a quem hás de enviar comigo; e tu
disseste: Conheço-te por teu nome, também achaste graça aos meus
olhos. Agora, pois, se tenho achado graça aos teus olhos,
rogo-te que me faças saber o teu caminho, e conhecer-te-ei, para que
ache graça aos teus olhos; e considera que esta nação é o teu povo.
Disse pois: Irá a minha presença contigo para te fazer
descansar. Então lhe disse: Se tu mesmo não fores conosco,
não nos faças subir daqui. Como, pois, se saberá agora que
tenho achado graça aos teus olhos, eu e o teu povo? Acaso não é por
andares tu conosco, de modo a sermos separados, eu e o teu povo, de
todos os povos que há sobre a face da terra? Então disse o
Senhor a Moisés: Farei também isto, que tens dito; porquanto achaste
graça aos meus olhos, e te conheço por nome. Então ele disse:
Rogo-te que me mostres a tua glória. Porém ele disse: Eu
farei passar toda a minha bondade por diante de ti, e proclamarei o
nome do Senhor diante de ti; e \textbf{terei misericórdia de quem eu
tiver misericórdia, e me compadecerei de quem eu me compadecer}.
E disse mais: Não poderás ver a minha face, porquanto homem
nenhum verá a minha face, e viverá. Disse mais o Senhor: Eis
aqui um lugar junto a mim; aqui te porás sobre a penha. E
acontecerá que, quando a minha glória passar, pôr-te-ei numa fenda
da penha, e te cobrirei com a minha mão, até que eu haja passado.
E, havendo eu tirado a minha mão, me verás pelas costas; mas
a minha face não se verá.

\medskip

\lettrine{34} Então disse o Senhor a Moisés: Lavra duas tábuas
de pedra, como as primeiras; e eu escreverei nas tábuas as mesmas
palavras que estavam nas primeiras tábuas, que tu quebraste. E
prepara-te para amanhã, para que subas pela manhã ao monte Sinai, e
ali põe-te diante de mim no cume do monte. E ninguém suba
contigo, e também ninguém apareça em todo o monte; nem ovelhas nem
bois se apascentem defronte do monte. Então Moisés lavrou duas
tábuas de pedra, como as primeiras; e levantando-se pela manhã de
madrugada, subiu ao monte Sinai, como o Senhor lhe tinha ordenado; e
levou as duas tábuas de pedra nas suas mãos.

E o Senhor desceu numa nuvem e se pôs ali junto a ele; e ele
proclamou o nome do Senhor. Passando, pois, o Senhor perante
ele, clamou: \textbf{O Senhor, o Senhor Deus, misericordioso e
piedoso, tardio em irar-se e grande em beneficência e verdade};
\textbf{que guarda a beneficência em milhares; que perdoa a
iniqüidade, e a transgressão e o pecado; que ao culpado não tem por
inocente; que visita a iniqüidade dos pais sobre os filhos e sobre
os filhos dos filhos até à terceira e quarta geração}. E Moisés
apressou-se, e inclinou a cabeça à terra, adorou, e disse:
Senhor, se agora tenho achado graça aos teus olhos, vá agora o
Senhor no meio de nós; porque este é povo de dura cerviz; porém
perdoa a nossa iniqüidade e o nosso pecado, e toma-nos por tua
herança.

Então disse: \textbf{Eis que eu faço uma aliança}; farei diante
de todo o teu povo maravilhas que nunca foram feitas em toda a
terra, nem em nação alguma; de maneira que todo este povo, em cujo
meio tu estás, veja a obra do Senhor; porque \textbf{coisa terrível
é o que faço contigo}. Guarda o que eu te ordeno hoje; eis
que eu lançarei fora diante de ti os amorreus, e os cananeus, e os
heteus, e os perizeus, e os heveus e os jebuseus. Guarda-te
de fazeres aliança com os moradores da terra aonde hás de entrar;
para que não seja por laço no meio de ti. Mas os seus altares
derrubareis, e as suas estátuas quebrareis, e os seus bosques
cortareis. Porque não te inclinarás diante de outro deus;
pois \textbf{o nome do Senhor é Zeloso; é um Deus zeloso}.
Para que não faças aliança com os moradores da terra, e
quando eles se prostituírem após os seus deuses, ou sacrificarem aos
seus deuses, tu, como convidado deles, comas também dos seus
sacrifícios, e tomes mulheres das suas filhas para os teus
filhos, e suas filhas, prostituindo-se com os seus deuses, façam que
também teus filhos se prostituam com os seus deuses. Não te
farás deuses de fundição.

A festa dos pães ázimos guardarás; sete dias comerás pães ázimos,
como te tenho ordenado, ao tempo apontado do mês de Abibe; porque no
mês de Abibe saíste do Egito. Tudo o que abre a madre meu é,
até todo o teu gado, que seja macho, e que abre a madre de vacas e
de ovelhas; o burro\footnote{Burro ou jumento? RA: O jumento,
porém, que abrir a madre, resgatá-lo-ás com cordeiro; mas, se o não
resgatares, será desnucado. Remirás todos os primogênitos de teus
filhos. Ninguém aparecerá diante de mim de mãos vazias. KJ: But the
firstling of an ass thou shalt redeem with a lamb: and if thou
redeem him not, then shalt thou break his neck. All the firstborn of
thy sons thou shalt redeem. And none shall appear before me empty.
Em Ex 13.13 a SBTB refere-se a ``jumento'': ``Porém, todo o
primogênito da jumenta resgatarás com um cordeiro; e se o não
resgatares, cortar-lhe-ás a cabeça; mas todo o primogênito do homem,
entre teus filhos, resgatarás.'' RA: Porém todo primogênito da
jumenta resgatarás com cordeiro; se o não resgatares, será
desnucado; mas todo primogênito do homem entre teus filhos
resgatarás. RC: Porém tudo o que abrir a madre da jumenta resgatarás
com cordeiro; e, se o não resgatares, cortar-lhe-ás a cabeça; mas
todo primogênito do homem entre teus filhos resgatarás. KJ: And
every firstling of an ass thou shalt redeem with a lamb; and if thou
wilt not redeem it, then thou shalt break his neck: and all the
firstborn of man among thy children shalt thou redeem.}, porém, que
abrir a madre, resgatarás com um cordeiro; mas, se o não resgatares,
cortar-lhe-ás a cabeça; todo o primogênito de teus filhos
resgatarás. E ninguém aparecerá vazio diante de mim. Seis
dias trabalharás, mas ao sétimo dia descansarás: na aradura e na
sega descansarás. Também guardarás a festa das semanas, que é
a festa das primícias da sega do trigo, e a festa da colheita no fim
do ano. Três vezes ao ano todos os homens aparecerão perante
o Senhor \textsc{Deus}, o Deus de Israel; porque eu lançarei
fora as nações de diante de ti, e alargarei o teu território;
ninguém cobiçará a tua terra, quando subires para aparecer três
vezes no ano diante do Senhor teu Deus. Não sacrificarás o
sangue do meu sacrifício com pão levedado, nem o sacrifício da festa
da páscoa ficará da noite para a manhã. As primícias dos
primeiros frutos da tua terra trarás à casa do Senhor teu Deus; não
cozerás o cabrito no leite de sua mãe. Disse mais o Senhor a
Moisés: Escreve estas palavras; porque \textbf{conforme ao teor
destas palavras tenho feito aliança contigo e com Israel}. E
esteve ali com o Senhor quarenta dias e quarenta noites; não comeu
pão, nem bebeu água, e escreveu nas tábuas as palavras da aliança,
os \textbf{dez mandamentos}. E aconteceu que, descendo Moisés
do monte Sinai trazia as duas tábuas do testemunho em suas mãos,
sim, quando desceu do monte, Moisés não sabia que a pele do seu
rosto resplandecia, depois que falara com ele. Olhando, pois,
Arão e todos os filhos de Israel para Moisés, eis que a pele do seu
rosto resplandecia; por isso temeram chegar-se a ele. Então
Moisés os chamou, e Arão e todos os príncipes da congregação
tornaram-se a ele; e Moisés lhes falou. Depois chegaram
também todos os filhos de Israel; e ele lhes ordenou tudo o que o
Senhor falara com ele no monte Sinai. Assim que Moisés acabou
de falar com eles, pôs um véu sobre o seu rosto. Porém,
entrando Moisés perante o Senhor, para falar com ele, tirava o véu
até sair; e, saindo, falava com os filhos de Israel o que lhe era
ordenado. Assim, pois, viam os filhos de Israel o rosto de
Moisés, e que resplandecia a pele do seu rosto; e tornava Moisés a
pôr o véu sobre o seu rosto, até entrar para falar com ele.

\medskip

\lettrine{35} Então Moisés convocou toda a congregação dos
filhos de Israel, e disse-lhes: Estas são as palavras que o Senhor
ordenou que se cumprissem. Seis dias se trabalhará, mas o sétimo
dia vos será santo, o sábado do repouso ao Senhor; todo aquele que
nele fizer qualquer trabalho morrerá. Não acendereis fogo em
nenhuma das vossas moradas no dia do sábado. Falou mais Moisés a
toda a congregação dos filhos de Israel, dizendo: Esta é a palavra
que o Senhor ordenou, dizendo: Tomai do que tendes, uma oferta
para o Senhor; cada um, cujo coração é voluntariamente disposto, a
trará por oferta alçada ao Senhor: ouro, prata e cobre, como
também azul, púrpura, carmesim, linho fino, pêlos de cabras, e
peles de carneiros, tintas de vermelho, e peles de texugos, madeira
de acácia, e azeite para a luminária, e especiarias para o
azeite da unção, e para o incenso aromático. E pedras de ônix, e
pedras de engaste, para o éfode e para o peitoral. E venham
todos os sábios de coração entre vós, e façam tudo o que o Senhor
tem mandado; o tabernáculo, a sua tenda e a sua coberta, os
seus colchetes e as suas tábuas, as suas barras, as suas colunas, e
as suas bases; a arca e os seus varais, o propiciatório e o
véu de cobertura, a mesa e os seus varais, e todos os seus
pertences; e os pães da proposição, e o candelabro da
luminária, e os seus utensílios, e as suas lâmpadas, e o azeite para
a luminária, e o altar do incenso e os seus varais, e o
azeite da unção, e o incenso aromático, e a cortina da porta para a
entrada do tabernáculo, o altar do holocausto, e o crivo de
cobre, os seus varais, e todos os seus pertences, a pia e a sua
base, as cortinas do pátio, as suas colunas e as suas bases,
e o reposteiro da porta do pátio, as estacas do tabernáculo,
e as estacas do pátio, e as suas cordas, as vestes do
ministério para ministrar no santuário, as vestes santas de Arão o
sacerdote, e as vestes de seus filhos, para administrarem o
sacerdócio.

Então toda a congregação dos filhos de Israel saiu da presença de
Moisés, e veio todo o homem, a quem o seu coração moveu, e
todo aquele cujo espírito voluntariamente o excitou, e trouxeram a
oferta alçada ao Senhor para a obra da tenda da congregação, e para
todo o seu serviço, e para as vestes santas. Assim vieram
homens e mulheres, todos dispostos de coração; trouxeram fivelas, e
pendentes, e anéis, e braceletes, todos os objetos de ouro; e todo o
homem fazia oferta de ouro ao Senhor; e todo o homem que se
achou com azul, e púrpura, e carmesim, e linho fino, e pêlos de
cabras, e peles de carneiro tintas de vermelho, e peles de texugos,
os trazia; todo aquele que fazia oferta alçada de prata ou de
metal, a trazia por oferta alçada ao Senhor; e todo aquele que
possuía madeira de acácia, a trazia para toda a obra do serviço.
E todas as mulheres sábias de coração fiavam com as suas
mãos, e traziam o que tinham fiado, o azul e a púrpura, o carmesim e
o linho fino. E todas as mulheres, cujo coração as moveu em
habilidade fiavam os pêlos das cabras. E os príncipes traziam
pedras de ônix e pedras de engastes para o éfode e para o peitoral,
e especiarias, e azeite para a luminária, e para o azeite da
unção, e para o incenso aromático. Todo homem e mulher, cujo
coração voluntariamente se moveu a trazer alguma coisa para toda a
obra que o Senhor ordenara se fizesse pela mão de Moisés; assim os
filhos de Israel trouxeram por oferta voluntária ao Senhor.

Depois disse Moisés aos filhos de Israel: Eis que o Senhor tem
chamado por nome a Bezalel, filho de Uri, filho de Hur, da tribo de
Judá. E \textbf{o Espírito de Deus o encheu} de sabedoria,
entendimento, ciência e em todo o lavor, e para criar
invenções, para trabalhar em ouro, e em prata, e em cobre, e
em lapidar de pedras para engastar, e em entalhar madeira, e para
trabalhar em toda a obra esmerada. Também lhe dispôs o
coração para ensinar a outros; a ele e a Aoliabe, o filho de
Aisamaque, da tribo de Dã. \textbf{Encheu-os de sabedoria do
coração}, para fazer toda a obra de mestre, até a mais engenhosa, e
a do gravador, em azul, e em púrpura, em carmesim, e em linho fino,
e do tecelão; fazendo toda a obra, \textbf{e criando invenções}.

\medskip

\lettrine{36} Assim trabalharam Bezalel e Aoliabe, e todo o
homem sábio de coração, a quem o Senhor dera sabedoria e
inteligência, para saber como haviam de fazer toda a obra para o
serviço do santuário, conforme a tudo o que o Senhor tinha ordenado.
Então Moisés chamou a Bezalel e a Aoliabe, e a todo o homem
sábio de coração, em cujo coração o Senhor tinha dado sabedoria; a
todo aquele a quem o seu coração moveu a se chegar à obra para
fazê-la. Estes receberam de Moisés toda a oferta alçada, que
trouxeram os filhos de Israel para a obra do serviço do santuário,
para fazê-la, e ainda eles lhe traziam cada manhã ofertas
voluntárias. E vieram todos os sábios, que faziam toda a obra do
santuário, cada um da obra que fazia, e falaram a Moisés,
dizendo: O povo traz muito mais do que basta para o serviço da obra
que o Senhor ordenou se fizesse. Então mandou Moisés que
proclamassem por todo o arraial, dizendo: Nenhum homem, nem mulher,
faça mais obra alguma para a oferta alçada do santuário. Assim o
povo foi proibido de trazer mais, porque tinham material
bastante para toda a obra que havia de fazer-se, e ainda sobejava.

Assim todo o sábio de coração, entre os que faziam a obra, fez o
tabernáculo de dez cortinas de linho fino torcido, e de azul, e de
púrpura, e de carmesim, com querubins; da obra mais esmerada as fez.
O comprimento de cada cortina era de vinte e oito côvados, e a
largura de quatro côvados; todas as cortinas tinham uma mesma
medida. E ligou cinco cortinas uma com a outra; e outras
cinco cortinas também ligou uma com outra. Depois fez laçadas
de azul na borda de uma cortina, à extremidade, na juntura; assim
também fez na borda, à extremidade da juntura da segunda cortina.
Cinqüenta laçadas fez numa cortina, e cinqüenta laçadas fez
numa extremidade da cortina, que se ligava com a segunda; estas
laçadas eram contrapostas uma a outra. Também fez cinqüenta
colchetes de ouro, e com estes colchetes uniu as cortinas uma com a
outra; e assim foi feito um tabernáculo.

Fez também cortinas de pêlos de cabras para a tenda sobre o
tabernáculo; fez onze cortinas. O comprimento de uma cortina
era de trinta côvados, e a largura de quatro côvados; estas onze
cortinas tinham uma mesma medida. E uniu cinco cortinas à
parte, e outras seis à parte, e fez cinqüenta laçadas na
borda da última cortina, na juntura; também fez cinqüenta laçadas na
borda da cortina, na outra juntura. Fez também cinqüenta
colchetes de metal, para ajuntar a tenda, para que fosse um todo.
Fez também, para a tenda, uma coberta de peles de carneiros,
tintas de vermelho; e por cima uma coberta de peles de texugos.
Também fez, de madeira de acácia, tábuas levantadas para o
tabernáculo, que foram colocadas verticalmente. O comprimento
de cada tábua era de dez côvados, e a largura de um côvado e meio.
Cada tábua tinha duas cavilhas pregadas uma a outra; assim
fez com todas as tábuas do tabernáculo. Assim, pois, fez as
tábuas para o tabernáculo; vinte tábuas para o lado que dá para o
sul; e fez quarenta bases de prata debaixo das vinte tábuas;
duas bases debaixo de uma tábua, para as suas duas cavilhas, e duas
debaixo de outra, para as suas duas cavilhas. Também fez
vinte tábuas ao outro lado do tabernáculo, do lado norte, com
as suas quarenta bases de prata; duas bases debaixo de uma tábua, e
duas bases debaixo de outra tábua. E ao lado do tabernáculo
para o ocidente fez seis tábuas. Fez também duas tábuas para
os cantos do tabernáculo nos dois lados, as quais por baixo
estavam juntas, e também se ajuntavam por cima com uma argola; assim
fez com ambas nos dois cantos. Assim eram oito tábuas com as
suas bases de prata, a saber, dezesseis bases; duas bases debaixo de
cada tábua. Fez também travessas de madeira de acácia; cinco
para as tábuas de um lado do tabernáculo, e cinco travessas
para as tábuas do outro lado do tabernáculo; e outras cinco
travessas para as tábuas do tabernáculo do lado ocidental. E
fez que a travessa do meio passasse pelo meio das tábuas de uma
extremidade até a outra. E cobriu as tábuas de ouro, e as
suas argolas (os lugares das travessas) fez de ouro; as travessas
também cobriu de ouro.

Depois fez o véu de azul, e de púrpura, e de carmesim, e de linho
fino torcido; de obra esmerada o fez com querubins. E fez-lhe
quatro colunas de madeira de acácia, e as cobriu de ouro; e seus
colchetes fez de ouro, e fundiu-lhe quatro bases de prata.
Fez também para a porta da tenda o véu de azul, e de púrpura,
e de carmesim, e de linho fino torcido, da obra do bordador,
com as suas cinco colunas e os seus colchetes; e as suas
cabeças e as suas molduras cobriu de ouro; e as suas cinco bases
eram de cobre.

\medskip

\lettrine{37} Fez também Bezalel a arca de madeira de acácia;
o seu comprimento era de dois côvados e meio; e a sua largura de um
côvado e meio; e a sua altura de um côvado e meio. E cobriu-a de
ouro puro por dentro e por fora; e fez-lhe uma coroa de ouro ao
redor; e fundiu-lhe quatro argolas de ouro nos seus quatro
cantos; num lado duas, e no outro lado duas argolas; e fez
varais de madeira de acácia, e os cobriu de ouro; e pôs os
varais pelas argolas aos lados da arca, para se levar a arca.
Fez também o propiciatório de ouro puro; o seu comprimento era
de dois côvados e meio, e a sua largura de um côvado e meio. Fez
também dois querubins de ouro; de obra batida os fez, nas duas
extremidades do propiciatório. Um querubim na extremidade de um
lado, e o outro na outra extremidade do outro lado; de uma só peça
com o propiciatório fez os querubins nas duas extremidades dele.
E os querubins estendiam as asas por cima, cobrindo com elas o
propiciatório; e os seus rostos estavam defronte um do outro; os
rostos dos querubins estavam virados para o propiciatório.

Fez também a mesa de madeira de acácia; o seu comprimento era de
dois côvados, e a sua largura de um côvado, e a sua altura de um
côvado e meio. E cobriu-a de ouro puro, e fez-lhe uma coroa
de ouro ao redor. Fez-lhe também, ao redor, uma moldura da
largura da mão; e fez uma coroa de ouro ao redor da moldura.
Fundiu-lhe também quatro argolas de ouro; e pôs as argolas
nos quatro cantos que estavam em seus quatro pés. Defronte da
moldura estavam as argolas para os lugares dos varais, para se levar
a mesa. Fez também os varais de madeira de acácia, e os
cobriu de ouro, para se levar a mesa. E fez de ouro puro os
utensílios que haviam de estar sobre a mesa, os seus pratos e as
suas colheres, e as suas tigelas e as suas taças em que se haviam de
oferecer libações. Fez também o candelabro de ouro puro; de
obra batida fez este candelabro; o seu pedestal, e as suas hastes,
os seus copos, as suas maçãs\footnote{RA: Fez também o candelabro de
ouro puro; de ouro batido o fez; o seu pedestal, a sua hástea, os
seus cálices, as suas maçanetas e as suas flores formavam com ele
uma só peça. Edição Contemporânea: botões. King James: And he made
the candlestick of pure gold: of beaten work made he the
candlestick; his shaft, and his branch, his bowls, his knops, and
his flowers, were of the same.}, e as suas flores, formavam com ele
uma só peça. Seis hastes saíam dos seus lados; três hastes do
candelabro, de um lado dele, e três do outro lado. Numa haste
estavam três copos do feitio de amêndoas, um botão e uma flor; e na
outra haste três copos do feitio de amêndoas, um botão e uma flor;
assim eram as seis hastes que saíam do candelabro. Mas no
mesmo candelabro havia quatro copos do feitio de amêndoas com os
seus botões e com as suas flores. E havia um botão debaixo de
duas hastes da mesma peça; e outro botão debaixo de duas hastes da
mesma peça; e mais um botão debaixo de duas hastes da mesma peça;
assim se fez para as seis hastes, que saíam dele. Os seus
botões e as suas hastes eram da mesma peça; tudo era uma obra batida
de ouro puro. E fez-lhe, de ouro puro, sete lâmpadas com os
seus espevitadores e os seus apagadores; de um talento de
ouro puro fez o candelabro e todos os seus utensílios.

E fez o altar do incenso de madeira de acácia; de um côvado era o
seu comprimento, e de um côvado a sua largura, era quadrado; e de
dois côvados a sua altura; dele mesmo eram feitas as suas pontas.
E cobriu-o de ouro puro, a parte superior e as suas paredes
ao redor, e as suas pontas; e fez-lhe uma coroa de ouro ao redor.
Fez-lhe também duas argolas de ouro debaixo da sua coroa, e
os seus dois cantos, de ambos os seus lados, para neles se colocar
os varais, e com eles levá-lo. E os varais fez de madeira de
acácia, e os cobriu de ouro. Também fez o azeite santo da
unção, e o incenso aromático, puro, qual obra do perfumista.

\medskip

\lettrine{38} Fez também o altar do holocausto de madeira de
acácia; de cinco côvados era o seu comprimento, e de cinco côvados a
sua largura, era quadrado; e de três côvados a sua altura. E
fez-lhe as suas pontas nos seus quatro cantos; da mesma peça eram as
suas pontas; e cobriu-o de cobre. Fez também todos os utensílios
do altar; os cinzeiros, e as pás, e as bacias, e os garfos, e os
braseiros; todos esses pertences fez de cobre. Fez também, para
o altar, um crivo de cobre, em forma de rede, na sua cercadura em
baixo, até ao meio do altar. E fundiu quatro argolas para as
quatro extremidades do crivo de cobre, para os lugares dos varais.
E fez os varais de madeira de acácia, e os cobriu de cobre.
E pôs os varais pelas argolas aos lados do altar, para com eles
levar o altar; fê-lo oco e de tábuas. Fez também a pia de cobre
com a sua base de cobre, dos espelhos das mulheres que se reuniam,
para servir à porta da tenda da congregação.

Fez também o pátio do lado meridional; as cortinas do pátio eram
de linho fino torcido, de cem côvados. As suas vinte colunas
e as suas vinte bases eram de cobre; os colchetes destas colunas e
as suas molduras eram de prata; e do lado norte cortinas de
cem côvados; as suas vinte colunas e as suas vinte bases eram de
cobre, os colchetes das colunas e as suas molduras eram de prata.
E do lado do ocidente cortinas de cinqüenta côvados, as suas
colunas dez, e as suas bases dez; os colchetes das colunas e as suas
molduras eram de prata. E do lado leste, ao oriente, cortinas
de cinqüenta côvados. As cortinas de um lado da porta eram de
quinze côvados; as suas colunas três e as suas bases três. E
do outro lado da porta do pátio, de ambos os lados, eram cortinas de
quinze côvados; as suas colunas três e as suas bases três.
Todas as cortinas do pátio ao redor eram de linho fino
torcido. E as bases das colunas eram de cobre; os colchetes
das colunas e as suas molduras eram de prata; e o revestimento dos
seus capitéis era de prata; e todas as colunas do pátio eram
cingidas de prata. E a cobertura da porta do pátio era de
obra de bordador, de azul, e de púrpura, e de carmesim, e de linho
fino torcido; e o comprimento era de vinte côvados, e a altura, na
largura, de cinco côvados, conforme as cortinas do pátio. E
as suas quatro colunas e as suas quatro bases eram de cobre, os seus
colchetes de prata, e o revestimento dos seus capitéis, e as suas
molduras, também de prata. E todas as estacas do tabernáculo
e do pátio ao redor eram de cobre.

Esta é a enumeração das coisas usadas no tabernáculo do
testemunho, que por ordem de Moisés foram contadas para o ministério
dos levitas, por intermédio de Itamar, filho de Arão, o sacerdote.
Fez, pois, Bezalel, o filho de Uri, filho de Hur, da tribo de
Judá, tudo quanto o Senhor tinha ordenado a Moisés. E com ele
Aoliabe, filho de Aisamaque, da tribo de Dã, um mestre de obra, e
engenhoso artífice, e bordador em azul, e em púrpura e em carmesim e
em linho fino. Todo o ouro gasto na obra, em toda a obra do
santuário, a saber o ouro da oferta, foi vinte e nove talentos e
setecentos e trinta siclos, conforme ao siclo do santuário; e
a prata dos arrolados da congregação foi cem talentos e mil e
setecentos e setenta e cinco siclos, conforme o siclo do santuário;
um beca por cabeça, isto é, meio siclo, conforme o siclo do
santuário; de todo aquele que passava aos arrolados, da idade de
vinte anos para cima, que foram seiscentos e três mil e quinhentos e
cinqüenta. E houve cem talentos de prata para fundir as bases
do santuário e as bases do véu; para as cem bases cem talentos; um
talento para cada base. E dos mil e setecentos e setenta e
cinco siclos fez os colchetes das colunas, e cobriu os seus
capitéis, e os cingiu de molduras. E o cobre da oferta foi
setenta talentos e dois mil e quatrocentos siclos. E dele fez
as bases da porta da tenda da congregação e o altar de cobre, e o
crivo de cobre e todos os utensílios do altar. E as bases do
pátio ao redor, e as bases da porta do pátio, e todas as estacas do
tabernáculo e todas as estacas do pátio ao redor.

\medskip

\lettrine{39} Fizeram também as vestes do ministério, para
ministrar no santuário, de azul, e de púrpura e de carmesim; também
fizeram as vestes santas, para Arão, como o Senhor ordenara a
Moisés. Assim se fez o éfode de ouro, de azul, e de púrpura, e
de carmesim e de linho fino torcido. E estenderam as lâminas de
ouro, e as cortaram em fios, para tecê-los entre o azul, e entre a
púrpura, e entre o carmesim, e entre o linho fino com trabalho
esmerado. Fizeram-lhe ombreiras que se ajuntavam; e uniam-se em
suas duas pontas. E o cinto de obra esmerada do éfode, que
estava sobre ele, formava com ele uma só peça e era de obra
semelhante, de ouro, de azul, e de púrpura, e de carmesim, e de
linho fino torcido, como o Senhor ordenara a Moisés. Também
prepararam as pedras de ônix, engastadas em ouro, lavradas com
gravuras de um selo, com os nomes dos filhos de Israel. E as pôs
sobre as ombreiras do éfode por pedras de memória para os filhos de
Israel, como o Senhor ordenara a Moisés. Fez-se também o
peitoral de obra de artífice, como a obra do éfode, de ouro, de
azul, e de púrpura, e de carmesim, e de linho fino torcido.
Quadrado era; duplo fizeram o peitoral; o seu comprimento era de
um palmo, e a sua largura de um palmo dobrado. E engastaram
nele quatro ordens de pedras; uma ordem de um sárdio, de um topázio,
e de um carbúnculo; esta era a primeira ordem; e a segunda
ordem de uma esmeralda, de uma safira e de um diamante; e a
terceira ordem de um jacinto, de uma ágata, e de uma ametista;
e a quarta ordem de um berilo, e de um ônix, e de um jaspe,
engastadas em engastes de ouro. Estas pedras, pois, eram
segundo os nomes dos filhos de Israel, doze segundo os seus nomes;
como gravuras de selo, cada uma com o seu nome, segundo as doze
tribos. Também fizeram para o peitoral cadeiazinhas de igual
medida, obra de ouro puro trançado. E fizeram dois engastes
de ouro e duas argolas de ouro; e puseram as duas argolas nas duas
extremidades do peitoral. E puseram as duas cadeiazinhas de
trança de ouro nas duas argolas, nas duas extremidades do peitoral.
E as outras duas pontas das duas cadeiazinhas de trança
puseram nos dois engastes; e as puseram sobre as ombreiras do éfode
na frente dele. Fizeram também duas argolas de ouro, que
puseram nas duas extremidades do peitoral, na sua borda que estava
junto ao éfode por dentro. Fizeram mais duas argolas de ouro,
que puseram nas duas ombreiras do éfode, abaixo, na frente dele,
perto da sua juntura, sobre o cinto de obra esmerada do éfode.
E ligaram o peitoral com as suas argolas às argolas do éfode
com um cordão de azul, para que estivesse sobre o cinto de obra
esmerada do éfode, e o peitoral não se separasse do éfode, como o
Senhor ordenara a Moisés. E fez-se o manto do éfode de obra
tecida, todo de azul. E a abertura do manto estava no meio
dele, como abertura de cota de malha; esta abertura tinha uma borda
em volta, para que se não rompesse. E nas bordas do manto
fizeram romãs de azul, e de púrpura, e de carmesim, de fio torcido.
Fizeram também as campainhas de ouro puro, pondo as
campainhas no meio das romãs nas bordas do manto, ao redor, entre as
romãs; uma campainha e uma romã, outra campainha e outra
romã, nas bordas do manto ao redor; para ministrar, como o Senhor
ordenara a Moisés. Fizeram também as túnicas de linho fino,
de obra tecida, para Arão e para seus filhos. E a mitra de
linho fino, e o ornato das tiaras de linho fino, e os calções de
linho fino torcido, e o cinto de linho fino torcido, e de
azul, e de púrpura, e de carmesim, obra de bordador, como o Senhor
ordenara a Moisés. Fizeram também, de ouro puro, a lâmina da
coroa de santidade, e nela escreveram o escrito como de gravura de
selo: SANTIDADE AO SENHOR. E ataram-na com um cordão de azul,
para prendê-la à parte superior da mitra, como o Senhor ordenara a
Moisés.

Assim se acabou toda a obra do tabernáculo da tenda da
congregação; e os filhos de Israel fizeram conforme a tudo o que o
Senhor ordenara a Moisés; assim o fizeram. Depois trouxeram a
Moisés o tabernáculo, a tenda e todos os seus pertences; os seus
colchetes, as suas tábuas, os seus varais, e as suas colunas, e as
suas bases; e a cobertura de peles de carneiro tintas de
vermelho, e a cobertura de peles de texugos, e o véu de cobertura;
a arca do testemunho, e os seus varais, e o propiciatório;
a mesa com todos os seus pertences, e os pães da proposição;
o candelabro puro com suas lâmpadas, as lâmpadas em ordem, e
todos os seus pertences, e o azeite para a luminária; também
o altar de ouro, e o azeite da unção, e o incenso aromático, e a
cortina da porta da tenda; o altar de cobre, e o seu crivo de
cobre, os seus varais, e todos os seus pertences, a pia, e a sua
base; as cortinas do pátio, as suas colunas, e as suas bases,
e a cortina da porta do pátio, as suas cordas, e os seus pregos, e
todos os utensílios do serviço do tabernáculo, para a tenda da
congregação; as vestes do ministério para ministrar no
santuário; as santas vestes de Arão o sacerdote, e as vestes dos
seus filhos, para administrarem o sacerdócio. Conforme a tudo
o que o Senhor ordenara a Moisés, assim fizeram os filhos de Israel
toda a obra. Viu, pois, Moisés toda a obra, e eis que a
tinham feito; como o Senhor ordenara, assim a fizeram; então Moisés
os abençoou.

\medskip

\lettrine{40} Falou mais o Senhor a Moisés, dizendo: No
primeiro mês, no primeiro dia do mês, levantarás o tabernáculo da
tenda da congregação, e porás nele a arca do testemunho, e
cobrirás a arca com o véu. Depois colocarás nele a mesa, e porás
em ordem o que se deve pôr em ordem nela; também colocarás nele o
candelabro, e acenderás as suas lâmpadas. E porás o altar de
ouro para o incenso diante da arca do testemunho; então pendurarás a
cortina da porta do tabernáculo. Porás também o altar do
holocausto diante da porta do tabernáculo da tenda da congregação.
E porás a pia entre a tenda da congregação e o altar, e nela
porás água. Depois porás o pátio ao redor, e pendurarás a
cortina à porta do pátio. Então tomarás o azeite da unção, e
ungirás o tabernáculo, e tudo o que há nele; e o santificarás com
todos os seus pertences, e será santo. Ungirás também o altar
do holocausto, e todos os seus utensílios; e santificarás o altar; e
o altar será santíssimo. Então ungirás a pia e a sua base, e
a santificarás. Farás também chegar a Arão e a seus filhos à
porta da tenda da congregação; e os lavarás com água. E
vestirás a Arão as vestes santas, e o ungirás, e o santificarás,
para que me administre o sacerdócio. Também farás chegar a
seus filhos, e lhes vestirás as túnicas, e os ungirás como
ungiste a seu pai, para que me administrem o sacerdócio, e a sua
unção lhes será por sacerdócio perpétuo nas suas gerações. E
Moisés fez conforme a tudo o que o Senhor lhe ordenou, assim o fez.
Assim, no primeiro mês, no ano segundo, ao primeiro dia do
mês foi levantado o tabernáculo. Moisés levantou o
tabernáculo, e pôs as suas bases, e armou as suas tábuas, e colocou
nele os seus varais, e levantou as suas colunas; e estendeu a
tenda sobre o tabernáculo, e pôs a cobertura da tenda sobre ela, em
cima, como o Senhor ordenara a Moisés. Tomou o testemunho, e
pô-lo na arca, e colocou os varais na arca; e pôs o propiciatório em
cima da arca. E introduziu a arca no tabernáculo, e pendurou
o véu da cobertura, e cobriu a arca do testemunho, como o Senhor
ordenara a Moisés. Pôs também a mesa na tenda da congregação,
ao lado do tabernáculo, para o norte, fora do véu, e sobre
ela pôs em ordem o pão perante o Senhor, como o Senhor ordenara a
Moisés. Pôs também na tenda da congregação o candelabro na
frente da mesa, ao lado do tabernáculo, para o sul, e acendeu
as lâmpadas perante o Senhor, como o Senhor ordenara a Moisés.
E pôs o altar de ouro na tenda da congregação, diante do véu,
e acendeu sobre ele o incenso de especiarias aromáticas, como
o Senhor ordenara a Moisés. Pendurou também a cortina da
porta do tabernáculo, e pôs o altar do holocausto à porta do
tabernáculo da tenda da congregação, e sobre ele ofereceu holocausto
e oferta de alimentos, como o Senhor ordenara a Moisés. Pôs
também a pia entre a tenda da congregação e o altar, e nela pôs água
para lavar. E Moisés, e Arão e seus filhos nela lavaram as
suas mãos e os seus pés. Quando entravam na tenda da
congregação, e quando chegavam ao altar, lavavam-se, como o Senhor
ordenara a Moisés. Levantou também o pátio ao redor do
tabernáculo e do altar, e pendurou a cortina da porta do pátio.
Assim Moisés acabou a obra.

Então a nuvem cobriu a tenda da congregação, e a glória do Senhor
encheu o tabernáculo; de maneira que Moisés não podia entrar
na tenda da congregação, porquanto a nuvem permanecia sobre ela, e a
glória do Senhor enchia o tabernáculo. Quando, pois, a nuvem
se levantava de sobre o tabernáculo, então os filhos de Israel
caminhavam em todas as suas jornadas. Se a nuvem, porém, não
se levantava, não caminhavam, até ao dia em que ela se levantasse;
porquanto a nuvem do Senhor estava de dia sobre o
tabernáculo, e o fogo estava de noite sobre ele, perante os olhos de
toda a casa de Israel, em todas as suas jornadas.

