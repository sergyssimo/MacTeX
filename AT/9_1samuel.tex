\addchap{Primeiro livro de Samuel}

\lettrine{1} Houve um homem de Ramataim-Zofim, da montanha de
Efraim, cujo nome era Elcana, filho de Jeroão, filho de Eliú, filho
de Toú, filho de Zufe, efrateu. E este tinha duas mulheres: o
nome de uma era Ana, e o da outra Penina. E Penina tinha filhos,
porém Ana não os tinha. Subia, pois, este homem, da sua cidade,
de ano em ano, a adorar e a sacrificar ao Senhor dos Exércitos em
Siló; e estavam ali os sacerdotes do Senhor, Hofni e Finéias, os
dois filhos de Eli. E sucedeu que no dia em que Elcana
sacrificava, dava ele porções a Penina, sua mulher, e a todos os
seus filhos, e a todas as suas filhas. Porém a Ana dava uma
parte excelente; porque amava a Ana, embora o Senhor lhe tivesse
cerrado a madre. E a sua rival excessivamente a provocava, para
a irritar; porque o Senhor lhe tinha cerrado a madre. E assim
fazia ele de ano em ano. Sempre que Ana subia à casa do Senhor, a
outra a irritava; por isso chorava, e não comia. Então Elcana,
seu marido, lhe disse: Ana, por que choras? E por que não comes? E
por que está mal o teu coração? Não te sou eu melhor do que dez
filhos?

Então Ana se levantou, depois que comeram e beberam em Siló; e
Eli, sacerdote, estava assentado numa cadeira, junto a um pilar do
templo do Senhor. Ela, pois, com amargura de alma, orou ao
Senhor, e chorou abundantemente. E fez um voto, dizendo:
Senhor dos Exércitos! Se benignamente atentares para a aflição da
tua serva, e de mim te lembrares, e da tua serva não te esqueceres,
mas à tua serva deres um filho homem, ao Senhor o darei todos os
dias da sua vida, e sobre a sua cabeça não passará navalha. E
sucedeu que, perseverando ela em orar perante o Senhor, Eli observou
a sua boca. Porquanto Ana no seu coração falava; só se moviam
os seus lábios, porém não se ouvia a sua voz; pelo que Eli a teve
por embriagada. E disse-lhe Eli: Até quando estarás tu
embriagada? Aparta de ti o teu vinho. Porém Ana respondeu:
Não, senhor meu, eu sou uma mulher atribulada de espírito; nem vinho
nem bebida forte tenho bebido; porém tenho derramado a minha alma
perante o Senhor. Não tenhas, pois, a tua serva por filha de
Belial; porque da multidão dos meus cuidados e do meu desgosto tenho
falado até agora. Então respondeu Eli: Vai em paz; e o Deus
de Israel te conceda a petição que lhe fizeste. E disse ela:
Ache a tua serva graça aos teus olhos. Assim a mulher foi o seu
caminho, e comeu, e o seu semblante já não era triste.

E levantaram-se de madrugada, e adoraram perante o Senhor, e
voltaram, e chegaram à sua casa, em Ramá, e Elcana conheceu a Ana
sua mulher, e o Senhor se lembrou dela. E sucedeu que,
passado algum tempo, Ana concebeu, e deu à luz um filho, ao qual
chamou Samuel; porque, dizia ela, o tenho pedido ao Senhor. E
subiu aquele homem Elcana com toda a sua casa, a oferecer ao Senhor
o sacrifício anual e a cumprir o seu voto. Porém Ana não
subiu; mas disse a seu marido: Quando o menino for desmamado, então
o levarei, para que apareça perante o Senhor, e lá fique para
sempre. E Elcana, seu marido, lhe disse: Faze o que bem te
parecer aos teus olhos; fica até que o desmames; então somente
confirme o Senhor a sua palavra. Assim ficou a mulher, e deu leite a
seu filho, até que o desmamou. E, havendo-o desmamado,
tomou-o consigo, com três bezerros, e um efa de farinha, e um odre
de vinho, e levou-o à casa do Senhor, em Siló, e era o menino ainda
muito criança. E degolaram um bezerro, e trouxeram o menino a
Eli. E disse ela: Ah, meu senhor, viva a tua alma, meu
Senhor; eu sou aquela mulher que aqui esteve contigo, para orar ao
Senhor. Por este menino orava eu; e o Senhor atendeu à minha
petição, que eu lhe tinha feito. Por isso também ao Senhor eu
o entreguei, por todos os dias que viver, pois ao Senhor foi pedido.
E adorou ali ao Senhor.

\medskip

\lettrine{2} Então orou Ana, e disse: O meu coração exulta ao
Senhor, o meu poder está exaltado no Senhor; a minha boca se dilatou
sobre os meus inimigos, porquanto me alegro na tua salvação. Não
há santo como o Senhor; porque não há outro fora de ti; e rocha
nenhuma há como o nosso Deus. Não multipliqueis palavras de
altivez, nem saiam coisas arrogantes da vossa boca; porque o Senhor
é o Deus de conhecimento, e por ele são as obras pesadas na balança.
O arco dos fortes foi quebrado, e os que tropeçavam foram
cingidos de força. Os fartos se alugaram por pão, e cessaram os
famintos; até a estéril deu à luz sete filhos, e a que tinha muitos
filhos enfraqueceu. O Senhor é o que tira a vida e a dá; faz
descer à sepultura e faz tornar a subir dela. O Senhor empobrece
e enriquece; abaixa e também exalta. Levanta o pobre do pó, e
desde o monturo exalta o necessitado, para o fazer assentar entre os
príncipes, para o fazer herdar o trono de glória; porque do Senhor
são os alicerces da terra, e assentou sobre eles o mundo. Os pés
dos seus santos guardará, porém os ímpios ficarão mudos nas trevas;
porque o homem não prevalecerá pela força. Os que contendem
com o Senhor serão quebrantados, desde os céus trovejará sobre eles;
o Senhor julgará as extremidades da terra; e dará força ao seu rei,
e exaltará o poder do seu ungido.

Então Elcana foi a Ramá, à sua casa; porém o menino ficou
servindo ao Senhor, perante o sacerdote Eli. Eram, porém, os
filhos de Eli filhos de Belial; não conheciam ao Senhor.
Porquanto o costume daqueles sacerdotes com o povo era que,
oferecendo alguém algum sacrifício, estando-se cozendo a carne,
vinha o moço do sacerdote, com um garfo de três dentes em sua mão;
e enfiava-o na caldeira, ou na panela, ou no caldeirão, ou na
marmita; e tudo quanto o garfo tirava, o sacerdote tomava para si;
assim faziam a todo o Israel que ia ali a Siló. Também antes
de queimarem a gordura vinha o moço do sacerdote, e dizia ao homem
que sacrificava: Dá essa carne para assar ao sacerdote; porque não
receberá de ti carne cozida, mas crua. E, dizendo-lhe o
homem: Queime-se primeiro a gordura de hoje, e depois toma para ti
quanto desejar a tua alma, então ele lhe dizia: Não, agora a hás de
dar, e, se não, por força a tomarei. Era, pois, muito grande
o pecado destes moços perante o Senhor, porquanto os homens
desprezavam a oferta do Senhor. Porém Samuel ministrava
perante o Senhor, sendo ainda jovem, vestido com um éfode de linho.
E sua mãe lhe fazia uma túnica pequena, e de ano em ano lha
trazia, quando com seu marido subia para oferecer o sacrifício
anual. E Eli abençoava a Elcana e a sua mulher, e dizia: O
Senhor te dê descendência desta mulher, pela petição que fez ao
Senhor. E voltavam para o seu lugar. Visitou, pois, o Senhor
a Ana, que concebeu, e deu à luz três filhos e duas filhas; e o
jovem Samuel crescia diante do Senhor. Era, porém, Eli já
muito velho, e ouvia tudo quanto seus filhos faziam a todo o Israel,
e de como se deitavam com as mulheres que em bandos se ajuntavam à
porta da tenda da congregação. E disse-lhes: Por que fazeis
tais coisas? Pois ouço de todo este povo os vossos malefícios.
Não, filhos meus, porque não é boa esta fama que ouço; fazeis
transgredir o povo do Senhor. Pecando homem contra homem, os
juízes o julgarão; pecando, porém, o homem contra o Senhor, quem
rogará por ele? Mas não ouviram a voz de seu pai, porque o Senhor os
queria matar. E o jovem Samuel ia crescendo, e fazia-se
agradável, assim para com o Senhor, como também para com os homens.

E veio um homem de Deus a Eli, e disse-lhe: Assim diz o Senhor:
Não me manifestei, na verdade, à casa de teu pai, estando eles ainda
no Egito, na casa de Faraó? E eu o escolhi dentre todas as
tribos de Israel por sacerdote, para oferecer sobre o meu altar,
para acender o incenso, e para trazer o éfode perante mim; e dei à
casa de teu pai todas as ofertas queimadas dos filhos de Israel.
Por que pisastes o meu sacrifício e a minha oferta de
alimentos, que ordenei na minha morada, e honras a teus filhos mais
do que a mim, para vos engordardes do principal de todas as ofertas
do meu povo de Israel? Portanto, diz o Senhor Deus de Israel:
Na verdade tinha falado eu que a tua casa e a casa de teu pai
andariam diante de mim perpetuamente; porém agora diz o Senhor:
Longe de mim tal coisa, porque aos que me honram honrarei, porém os
que me desprezam serão desprezados. Eis que vêm dias em que
cortarei o teu braço e o braço da casa de teu pai, para que não haja
mais ancião algum em tua casa. E verás o aperto da morada de
Deus, em lugar de todo o bem que houvera de fazer a Israel; nem
haverá por todos os dias ancião algum em tua casa. O homem,
porém, a quem eu não desarraigar do meu altar será para te consumir
os olhos e para te entristecer a alma; e toda a multidão da tua casa
morrerá quando chegar à idade varonil. E isto te será por
sinal, a saber: o que acontecerá a teus dois filhos, a Hofni e a
Finéias; ambos morrerão no mesmo dia. E eu suscitarei para
mim um sacerdote fiel, que procederá segundo o meu coração e a minha
alma, e eu lhe edificarei uma casa firme, e andará sempre diante do
meu ungido. E será que todo aquele que restar da tua casa
virá a inclinar-se diante dele por uma moeda de prata e por um
bocado de pão, e dirá: Rogo-te que me admitas a algum ministério
sacerdotal, para que possa comer um pedaço de pão.

\medskip

\lettrine{3} E o jovem Samuel servia ao Senhor perante Eli; e
a palavra do Senhor era de muita valia naqueles dias; não havia
visão manifesta. E sucedeu, naquele dia, que, estando Eli
deitado no seu lugar (e os seus olhos começavam a escurecer, pois
não podia ver), e estando também Samuel já deitado, antes que a
lâmpada de Deus se apagasse no templo do Senhor, onde estava a arca
de Deus, o Senhor chamou a Samuel, e disse ele: Eis-me aqui.
E correu a Eli, e disse: Eis-me aqui, porque tu me chamaste. Mas
ele disse: Não te chamei eu, torna a deitar-te. E foi e se deitou.
E o Senhor tornou a chamar outra vez a Samuel, e Samuel se
levantou, e foi a Eli, e disse: Eis-me aqui, porque tu me chamaste.
Mas ele disse: Não te chamei eu, filho meu, torna a deitar-te.
Porém Samuel ainda não conhecia ao Senhor, e ainda não lhe tinha
sido manifestada a palavra do Senhor. O Senhor, pois, tornou a
chamar a Samuel terceira vez, e ele se levantou, e foi a Eli, e
disse: Eis-me aqui, porque tu me chamaste. Então entendeu Eli que o
Senhor chamava o jovem. Por isso Eli disse a Samuel: Vai
deitar-te e há de ser que, se te chamar, dirás: Fala, Senhor, porque
o teu servo ouve. Então Samuel foi e se deitou no seu lugar.
Então veio o Senhor, e pôs-se ali, e chamou como das outras
vezes: Samuel, Samuel. E disse Samuel: Fala, porque o teu servo
ouve.

E disse o Senhor a Samuel: Eis que vou fazer uma coisa em Israel,
a qual todo o que ouvir lhe tinirão ambos os ouvidos. Naquele
mesmo dia suscitarei contra Eli tudo quanto tenho falado contra a
sua casa, começarei e acabarei. Porque eu já lhe fiz saber
que julgarei a sua casa para sempre, pela iniqüidade que ele bem
conhecia, porque, fazendo-se os seus filhos execráveis, não os
repreendeu. Portanto, jurei à casa de Eli que nunca jamais
será expiada a sua iniqüidade, nem com sacrifício, nem com oferta de
alimentos. E Samuel ficou deitado até pela manhã, e então
abriu as portas da casa do Senhor; porém temia Samuel relatar esta
visão a Eli. Então chamou Eli a Samuel, e disse: Samuel, meu
filho. E disse ele: Eis-me aqui. E ele disse: Qual é a
palavra que te falou? Peço-te que não ma encubras; assim Deus te
faça, e outro tanto, se me encobrires alguma palavra de todas as que
te falou. Então Samuel lhe contou todas aquelas palavras, e
nada lhe encobriu. E disse ele: Ele é o Senhor; faça o que bem
parecer aos seus olhos.

E crescia Samuel, e o Senhor era com ele, e nenhuma de todas as
suas palavras deixou cair em terra. E todo o Israel, desde Dã
até Berseba, conheceu que Samuel estava confirmado por profeta do
Senhor. E continuou o Senhor a aparecer em Siló; porquanto o
Senhor se manifestava a Samuel em Siló pela palavra do Senhor.

\medskip

\lettrine{4} E veio a palavra de Samuel a todo o Israel; e
Israel saiu à peleja contra os filisteus e acampou-se junto a
Ebenézer; e os filisteus se acamparam junto a Afeque. E os
filisteus se dispuseram em ordem de batalha, para sair contra
Israel; e, estendendo-se a peleja, Israel foi ferido diante dos
filisteus, porque feriram na batalha, no campo, uns quatro mil
homens. E voltando o povo ao arraial, disseram os anciãos de
Israel: Por que nos feriu o Senhor hoje diante dos filisteus?
Tragamos de Siló a arca da aliança do Senhor, e venha no meio de
nós, para que nos livre da mão de nossos inimigos. Enviou, pois,
o povo a Siló, e trouxeram de lá a arca da aliança do Senhor dos
Exércitos, que habita entre os querubins; e os dois filhos de Eli,
Hofni e Finéias, estavam ali com a arca da aliança de Deus. E
sucedeu que, vindo a arca da aliança do Senhor ao arraial, todo o
Israel gritou com grande júbilo, até que a terra estremeceu. E
os filisteus, ouvindo a voz de júbilo, disseram: Que voz de grande
júbilo é esta no arraial dos hebreus? Então souberam que a arca do
Senhor era vinda ao arraial. Por isso os filisteus se
atemorizaram, porque diziam: Deus veio ao arraial. E diziam mais: Ai
de nós! Tal nunca jamais sucedeu antes. Ai de nós! Quem nos
livrará da mão desses grandiosos deuses? Estes são os deuses que
feriram aos egípcios com todas as pragas junto ao deserto.
Esforçai-vos, e sede homens, ó filisteus, para que porventura
não venhais a servir aos hebreus, como eles serviram a vós; sede,
pois, homens, e pelejai.

Então pelejaram os filisteus, e Israel foi ferido, fugindo cada
um para a sua tenda; e foi tão grande o estrago, que caíram de
Israel trinta mil homens de pé. E foi tomada a arca de Deus:
e os dois filhos de Eli, Hofni e Finéias, morreram.

Então correu, da batalha, um homem de Benjamim, e chegou no mesmo
dia a Siló; e trazia as vestes rotas, e terra sobre a cabeça.
E, chegando ele, eis que Eli estava assentado numa cadeira,
olhando para o caminho; porquanto o seu coração estava tremendo pela
arca de Deus. Entrando, pois, aquele homem a anunciar isto na
cidade, toda a cidade gritou. E Eli, ouvindo os gritos,
disse: Que alvoroço é esse? Então chegou aquele homem
apressadamente, e veio, e o anunciou a Eli. E era Eli da
idade de noventa e oito anos; e estavam os seus olhos tão
escurecidos, que já não podia ver. E disse aquele homem a
Eli: Eu sou o que venho da batalha; porque eu fugi hoje da batalha.
E disse ele: Que coisa sucedeu, filho meu? Então respondeu o
que trazia as notícias, e disse: Israel fugiu de diante dos
filisteus, e houve também grande matança entre o povo; e, além
disso, também teus dois filhos, Hofni e Finéias, morreram, e a arca
de Deus foi tomada. E sucedeu que, fazendo ele menção da arca
de Deus, Eli caiu da cadeira para trás, ao lado da porta, e
quebrou-se-lhe o pescoço e morreu; porquanto o homem era velho e
pesado; e tinha ele julgado Israel quarenta anos.

E, estando sua nora, a mulher de Finéias, grávida, e próxima ao
parto, e ouvindo estas notícias, de que a arca de Deus era tomada, e
de que seu sogro e seu marido morreram, encurvou-se e deu à luz;
porquanto as dores lhe sobrevieram. E, ao tempo em que ia
morrendo, disseram as mulheres que estavam com ela: Não temas, pois
deste à luz um filho. Ela porém não respondeu, nem fez caso disso.
E chamou ao menino Icabode, dizendo: De Israel se foi a
glória! Porque a arca de Deus foi tomada, e por causa de seu sogro e
de seu marido. E disse: De Israel a glória é levada presa;
pois é tomada a arca de Deus.

\medskip

\lettrine{5} Os filisteus, pois, tomaram a arca de Deus e a
trouxeram de Ebenézer a Asdode. Tomaram os filisteus a arca de
Deus, e a colocaram na casa de Dagom, e a puseram junto a Dagom.
Levantando-se, porém, de madrugada no dia seguinte, os de
Asdode, eis que Dagom estava caído com o rosto em terra, diante da
arca do Senhor; e tomaram a Dagom, e tornaram a pô-lo no seu lugar.
E, levantando-se de madrugada, no dia seguinte, pela manhã, eis
que Dagom jazia caído com o rosto em terra diante da arca do Senhor;
e a cabeça de Dagom e ambas as palmas das suas mãos estavam cortadas
sobre o limiar; somente o tronco ficou a Dagom. Por isso nem os
sacerdotes de Dagom, nem nenhum de todos os que entram na casa de
Dagom pisam o limiar de Dagom em Asdode, até ao dia de hoje.

Porém a mão do Senhor se agravou sobre os de Asdode, e os assolou;
e os feriu com hemorróidas, em Asdode e nos seus termos. Vendo
então os homens de Asdode que assim foi, disseram: Não fique conosco
a arca do Deus de Israel; pois a sua mão é dura sobre nós, e sobre
Dagom, nosso deus. Por isso enviaram mensageiros e congregaram a
si todos os príncipes dos filisteus, e disseram: Que faremos nós da
arca do Deus de Israel? E responderam: A arca do Deus de Israel será
levada até Gate. Assim levaram para lá a arca do Deus de Israel.
E sucedeu que, assim que a levaram, a mão do Senhor veio contra
aquela cidade, com mui grande vexame; pois feriu aos homens daquela
cidade, desde o pequeno até ao grande; e tinham hemorróidas nas
partes íntimas. Então enviaram a arca de Deus a Ecrom.
Sucedeu, porém, que, vindo a arca de Deus a Ecrom, os de Ecrom
exclamaram, dizendo: Transportaram para nós a arca do Deus de
Israel, para nos matarem, a nós e ao nosso povo. E enviaram,
e congregaram a todos os príncipes dos filisteus, e disseram: Enviai
a arca do Deus de Israel, e torne para o seu lugar, para que não
mate nem a nós nem ao nosso povo. Porque havia mortal vexame em toda
a cidade, e a mão de Deus muito se agravara ali. E os homens
que não morriam eram tão atacados com hemorróidas que o clamor da
cidade subia até o céu.

\medskip

\lettrine{6} Havendo, pois, estado a arca do Senhor na terra
dos filisteus sete meses, os filisteus chamaram os sacerdotes e
os adivinhadores, dizendo: Que faremos nós com a arca do Senhor?
Fazei-nos saber como a tornaremos a enviar ao seu lugar. Os
quais disseram: Se enviardes a arca do Deus de Israel, não a envieis
vazia, porém sem falta enviareis uma oferta para a expiação da
culpa; então sereis curados, e se vos fará saber porque a sua mão
não se retira de vós. Então disseram: Qual é a expiação da culpa
que lhe havemos de enviar? E disseram: Segundo o número dos
príncipes dos filisteus, cinco hemorróidas de ouro e cinco ratos de
ouro; porquanto a praga é uma mesma sobre todos vós e sobre todos os
vossos príncipes. Fazei, pois, umas imagens das vossas
hemorróidas e dos vossos ratos, que andam destruindo a terra, e dai
glória ao Deus de Israel; porventura aliviará a sua mão de cima de
vós, e de cima do vosso deus, e de cima da vossa terra. Por que,
pois, endureceríeis o vosso coração, como os egípcios e Faraó
endureceram os seus corações? Porventura depois de os haver tratado
tão mal, os não deixaram ir, e eles não se foram? Agora, pois,
tomai e fazei-vos um carro novo, e tomai duas vacas com crias, sobre
as quais não tenha subido o jugo, e atai as vacas ao carro, e tirai
delas os seus bezerros e levai-os para casa. Então tomai a arca
do Senhor, e ponde-a sobre o carro, e colocai, num cofre, ao seu
lado, as figuras de ouro que lhe haveis de oferecer em expiação da
culpa, e assim a enviareis, para que se vá. Vede então: Se ela
subir pelo caminho do seu termo a Bete-Semes, foi ele quem nos fez
este grande mal; e, se não, saberemos que não nos tocou a sua mão, e
que isto nos sucedeu por acaso.

E assim fizeram aqueles homens, e tomaram duas vacas que criavam,
e as ataram ao carro; e os seus bezerros encerraram em casa.
E puseram a arca do Senhor sobre o carro, como também o cofre
com os ratos de ouro e com as imagens das suas hemorróidas.
Então as vacas se encaminharam diretamente pelo caminho de
Bete-Semes, e seguiam um mesmo caminho, andando e berrando, sem se
desviarem, nem para a direita nem para a esquerda; e os príncipes
dos filisteus foram atrás delas, até ao termo de Bete-Semes.
E andavam os de Bete-Semes fazendo a sega do trigo no vale,
e, levantando os seus olhos, viram a arca, e, vendo-a, se alegraram.
E o carro veio ao campo de Josué, o bete-semita, e parou ali
onde havia uma grande pedra. E fenderam a madeira do carro, e
ofereceram as vacas ao Senhor em holocausto. E os levitas
desceram a arca do Senhor, como também o cofre que estava junto a
ela, em que estavam os objetos de ouro, e puseram-nos sobre aquela
grande pedra; e os homens de Bete-Semes ofereceram holocaustos e
sacrifícios ao Senhor no mesmo dia. E, vendo aquilo os cinco
príncipes dos filisteus, voltaram para Ecrom no mesmo dia.
Estas, pois, são as hemorróidas de ouro que enviaram os
filisteus ao Senhor em expiação da culpa: Por Asdode uma, por Gaza
outra, por Ascalom outra, por Gate outra, por Ecrom outra.
Como também os ratos de ouro, segundo o número de todas as
cidades dos filisteus, pertencentes aos cinco príncipes, desde as
cidades fortificadas até às aldeias, e até Abel. A grande pedra,
sobre a qual puseram a arca do Senhor, ainda está até ao dia de hoje
no campo de Josué, o bete-semita.

E o Senhor feriu os homens de Bete-Semes, porquanto olharam para
dentro da arca do Senhor; feriu do povo cinqüenta mil e setenta
homens; então o povo se entristeceu, porquanto o Senhor fizera tão
grande estrago entre o povo. Então disseram os homens de
Bete-Semes: Quem poderia subsistir perante este santo Senhor Deus? E
a quem subirá de nós? Enviaram, pois, mensageiros aos
habitantes de Quiriate-Jearim, dizendo: Os filisteus remeteram a
arca do Senhor; descei, pois, e fazei-a subir para vós.

\medskip

\lettrine{7} Então vieram os homens de Quiriate-Jearim, e
levaram a arca do Senhor, e a trouxeram à casa de Abinadabe, no
outeiro; e consagraram a Eleazar, seu filho, para que guardasse a
arca do Senhor. E sucedeu que, desde aquele dia, a arca ficou em
Quiriate-Jearim, e tantos dias se passaram que até chegaram vinte
anos, e lamentava toda a casa de Israel pelo Senhor.

Então falou Samuel a toda a casa de Israel, dizendo: Se com todo o
vosso coração vos converterdes ao Senhor, tirai dentre vós os deuses
estranhos e os astarotes, e preparai o vosso coração ao Senhor, e
servi a ele só, e vos livrará da mão dos filisteus. Então os
filhos de Israel tiraram dentre si aos baalins e aos astarotes, e
serviram só ao Senhor. Disse mais Samuel: Congregai a todo o
Israel em Mizpá; e orarei por vós ao Senhor. E congregaram-se em
Mizpá, e tiraram água, e a derramaram perante o Senhor, e jejuaram
aquele dia, e disseram ali: Pecamos contra o Senhor. E julgava
Samuel os filhos de Israel em Mizpá.

Ouvindo, pois, os filisteus que os filhos de Israel estavam
congregados em Mizpá, subiram os maiorais dos filisteus contra
Israel; o que ouvindo os filhos de Israel, temeram por causa dos
filisteus. Por isso disseram os filhos de Israel a Samuel: Não
cesses de clamar ao Senhor nosso Deus por nós, para que nos livre da
mão dos filisteus. Então tomou Samuel um cordeiro de mama, e
sacrificou-o inteiro em holocausto ao Senhor; e clamou Samuel ao
Senhor por Israel, e o Senhor lhe deu ouvidos. E sucedeu que,
estando Samuel sacrificando o holocausto, os filisteus chegaram à
peleja contra Israel; e trovejou o Senhor aquele dia com grande
estrondo sobre os filisteus, e os confundiu de tal modo que foram
derrotados diante dos filhos de Israel. E os homens de Israel
saíram de Mizpá; e perseguiram os filisteus, e os feriram até abaixo
de Bete-Car. Então tomou Samuel uma pedra, e a pôs entre
Mizpá e Sem, e chamou-lhe Ebenézer; e disse: Até aqui nos ajudou o
Senhor.

Assim os filisteus foram abatidos, e nunca mais vieram aos termos
de Israel, porquanto foi a mão do Senhor contra os filisteus todos
os dias de Samuel. E as cidades que os filisteus tinham
tomado a Israel foram-lhe restituídas, desde Ecrom até Gate, e até
os seus termos Israel arrebatou da mão dos filisteus; e houve paz
entre Israel e entre os amorreus. E Samuel julgou a Israel
todos os dias da sua vida. E ia de ano em ano, e rodeava a
Betel, e a Gilgal, e a Mizpá, e julgava a Israel em todos aqueles
lugares. Porém voltava a Ramá, porque estava ali a sua casa,
e ali julgava a Israel; e edificou ali um altar ao Senhor.

\medskip

\lettrine{8} E sucedeu que, tendo Samuel envelhecido,
constituiu a seus filhos por juízes sobre Israel. E o nome do
seu filho primogênito era Joel, e o nome do seu segundo, Abia; e
foram juízes em Berseba. Porém seus filhos não andaram pelos
caminhos dele, antes se inclinaram à avareza, e aceitaram suborno, e
perverteram o direito.

Então todos os anciãos de Israel se congregaram, e vieram a
Samuel, a Ramá, e disseram-lhe: Eis que já estás velho, e teus
filhos não andam pelos teus caminhos; constitui-nos, pois, agora um
rei sobre nós, para que ele nos julgue, como o têm todas as nações.
Porém esta palavra pareceu mal aos olhos de Samuel, quando
disseram: Dá-nos um rei, para que nos julgue. E Samuel orou ao
Senhor. E disse o Senhor a Samuel: Ouve a voz do povo em tudo
quanto te dizem, pois não te têm rejeitado a ti, antes a mim me têm
rejeitado, para eu não reinar sobre eles. Conforme a todas as
obras que fizeram desde o dia em que os tirei do Egito até ao dia de
hoje, a mim me deixaram, e a outros deuses serviram, assim também
fazem a ti. Agora, pois, ouve à sua voz, porém protesta-lhes
solenemente, e declara-lhes qual será o costume do rei que houver de
reinar sobre eles. E falou Samuel todas as palavras do Senhor
ao povo, que lhe pedia um rei. E disse: Este será o costume
do rei que houver de reinar sobre vós; ele tomará os vossos filhos,
e os empregará nos seus carros, e como seus cavaleiros, para que
corram adiante dos seus carros. E os porá por chefes de mil,
e de cinqüenta; e para que lavrem a sua lavoura, e façam a sua sega,
e fabriquem as suas armas de guerra e os petrechos de seus carros.
E tomará as vossas filhas para perfumistas, cozinheiras e
padeiras. E tomará o melhor das vossas terras, e das vossas
vinhas, e dos vossos olivais, e os dará aos seus servos. E as
vossas sementes, e as vossas vinhas dizimará, para dar aos seus
oficiais, e aos seus servos. Também os vossos servos, e as
vossas servas, e os vossos melhores moços, e os vossos jumentos
tomará, e os empregará no seu trabalho. Dizimará o vosso
rebanho, e vós lhe servireis de servos. Então naquele dia
clamareis por causa do vosso rei, que vós houverdes escolhido; mas o
Senhor não vos ouvirá naquele dia. Porém o povo não quis
ouvir a voz de Samuel; e disseram: Não, mas haverá sobre nós um rei.
E nós também seremos como todas as outras nações; e o nosso
rei nos julgará, e sairá adiante de nós, e fará as nossas guerras.
Ouvindo, pois, Samuel todas as palavras do povo, as repetiu
aos ouvidos do Senhor. Então o Senhor disse a Samuel: Dá
ouvidos à sua voz, e constitui-lhes rei. Então Samuel disse aos
homens de Israel: Volte cada um à sua cidade.

\medskip

\lettrine{9} E havia um homem de Benjamim, cujo nome era Quis,
filho de Abiel, filho de Zeror, filho de Becorate, filho de Afia,
filho de um homem de Benjamim; homem poderoso. Este tinha um
filho, cujo nome era Saul, moço, e tão belo que entre os filhos de
Israel não havia outro homem mais belo do que ele; desde os ombros
para cima sobressaía a todo o povo.

E perderam-se as jumentas de Quis, pai de Saul; por isso disse
Quis a Saul, seu filho: Toma agora contigo um dos moços, e
levanta-te e vai procurar as jumentas. Passaram, pois, pela
montanha de Efraim, e dali passaram à terra de Salisa, porém não as
acharam; depois passaram à terra de Saalim, porém tampouco estavam
ali; também passaram à terra de Benjamim, porém tampouco as acharam.
Vindo eles então à terra de Zufe, Saul disse para o seu moço,
com quem ele ia: Vem, e voltemos; para que porventura meu pai não
deixe de inquietar-se pelas jumentas e se aflija por causa de nós.
Porém ele lhe disse: Eis que há nesta cidade um homem de Deus, e
homem honrado é; tudo quanto diz, sucede assim infalivelmente;
vamo-nos agora lá; porventura nos mostrará o caminho que devemos
seguir. Então Saul disse ao seu moço: Eis, porém, se lá formos,
que levaremos então àquele homem? Porque o pão de nossos alforjes se
acabou, e presente nenhum temos para levar ao homem de Deus; que
temos? E o moço tornou a responder a Saul, e disse: Eis que
ainda se acha na minha mão um quarto de um siclo de prata, o qual
darei ao homem de Deus, para que nos mostre o caminho
antigamente em Israel, indo alguém consultar a Deus, dizia
assim: Vinde, e vamos ao vidente; porque ao profeta de hoje,
antigamente se chamava vidente). Então disse Saul ao moço:
Bem dizes; vem, pois, vamos. E foram-se à cidade onde estava o homem
de Deus.

E, subindo eles à cidade, acharam umas moças que saíam a tirar
água; e disseram-lhes: Está aqui o vidente? E elas lhes
responderam, e disseram: Sim, eis aí o tens diante de ti;
apressa-te, pois, porque hoje veio à cidade; porquanto o povo tem
hoje sacrifício no alto. Entrando vós na cidade, logo o
achareis, antes que suba ao alto para comer; porque o povo não
comerá, até que ele venha; porque ele é o que abençoa o sacrifício,
e depois comem os convidados; subi, pois, agora, que hoje o
achareis. Subiram, pois, à cidade; e, vindo eles no meio da
cidade, eis que Samuel lhes saiu ao encontro, para subir ao alto.
Porque o Senhor revelara isto aos ouvidos de Samuel, um dia
antes que Saul viesse, dizendo: Amanhã a estas horas te
enviarei um homem da terra de Benjamim, o qual ungirás por capitão
sobre o meu povo de Israel, e ele livrará o meu povo da mão dos
filisteus; porque tenho olhado para o meu povo; porque o seu clamor
chegou a mim. E quando Samuel viu a Saul, o Senhor lhe
respondeu: Eis aqui o homem de quem eu te falei. Este dominará sobre
o meu povo.

E Saul se chegou a Samuel no meio da porta, e disse: Mostra-me,
peço-te, onde está a casa do vidente. E Samuel respondeu a
Saul, e disse: Eu sou o vidente; sobe diante de mim ao alto, e comei
hoje comigo; e pela manhã te despedirei, e tudo quanto está no teu
coração, to declararei. E quanto às jumentas que há três dias
se te perderam, não ocupes o teu coração com elas, porque já se
acharam. E para quem é todo o desejo de Israel? Porventura não é
para ti, e para toda a casa de teu pai? Então respondeu Saul,
e disse: Porventura não sou eu filho de Benjamim, da menor das
tribos de Israel? E a minha família a menor de todas as famílias da
tribo de Benjamim? Por que, pois, me falas com semelhantes palavras?
Porém Samuel tomou a Saul e ao seu moço, e os levou à câmara;
e deu-lhes lugar acima de todos os convidados, que eram uns trinta
homens. Então disse Samuel ao cozinheiro: Dá aqui a porção
que te dei, de que te disse: Põe-na à parte contigo.
Levantou, pois, o cozinheiro a espádua, com o que havia nela,
e pô-la diante de Saul; e disse Samuel: Eis que o que foi reservado
está diante de ti. Come; porque se guardou para ti para esta
ocasião, dizendo eu: Tenho convidado o povo. Assim comeu Saul aquele
dia com Samuel. Então desceram do alto para a cidade; e falou
com Saul sobre o eirado. E se levantaram de madrugada; e
sucedeu que, quase ao subir da alva, chamou Samuel a Saul ao eirado,
dizendo: Levanta-te, e despedir-te-ei. Levantou-se Saul, e saíram
ambos\footnote{SBTB: Pleonasmo: ``saíram ambos para fora''. AV: And
they arose early: and it came to pass about the spring of the day,
that Samuel called Saul to the top of the house, saying, Up, that I
may send thee away. And Saul arose, and they went out both of them,
he and Samuel, abroad. RA: Levantaram-se de madrugada; e, quase ao
subir da alva, chamou Samuel a Saul ao eirado, dizendo: Levanta-te;
eu irei contigo para te encaminhar. Levantou-se Saul, e saíram
ambos, ele e Samuel.}, ele e Samuel. E, descendo eles para a
extremidade da cidade, Samuel disse a Saul: Dize ao moço que passe
adiante de nós (e passou); porém tu espera agora, e te farei ouvir a
palavra de Deus.

\medskip

\lettrine{10} Então tomou Samuel um vaso de azeite, e lho
derramou sobre a cabeça, e beijou-o, e disse: Porventura não te
ungiu o Senhor por capitão sobre a sua herança? Apartando-te
hoje de mim, acharás dois homens junto ao sepulcro de Raquel, no
termo de Benjamim, em Zelza, os quais te dirão: Acharam-se as
jumentas que foste buscar, e eis que já o teu pai deixou o negócio
das jumentas, e anda aflito por causa de vós, dizendo: Que farei eu
por meu filho? E quando dali passares mais adiante, e chegares
ao carvalho de Tabor, ali te encontrarão três homens, que vão
subindo a Deus a Betel; um levando três cabritos, o outro três bolos
de pão e o outro um odre de vinho. E te perguntarão como estás,
e te darão dois pães, que tomarás das suas mãos. Então chegarás
ao outeiro de Deus, onde está a guarnição dos filisteus; e há de ser
que, entrando ali na cidade, encontrarás um grupo de profetas que
descem do alto, e trazem diante de si saltérios, e tambores, e
flautas, e harpas; e eles estarão profetizando. E o Espírito do
Senhor se apoderará de ti, e profetizarás com eles, e tornar-te-ás
um outro homem. E há de ser que, quando estes sinais te vierem,
faze o que achar a tua mão, porque Deus é contigo. Tu, porém,
descerás antes de mim a Gilgal, e eis que eu descerei a ti, para
sacrificar holocaustos, e para oferecer ofertas pacíficas; ali sete
dias esperarás, até que eu venha a ti, e te declare o que hás de
fazer.

Sucedeu, pois, que, virando ele as costas para partir de Samuel,
Deus lhe mudou o coração em outro; e todos aqueles sinais
aconteceram naquele mesmo dia. E, chegando eles ao outeiro,
eis que um grupo de profetas lhes saiu ao encontro; e o Espírito de
Deus se apoderou dele, e profetizou no meio deles. E
aconteceu que, como todos os que antes o conheciam viram que ele
profetizava com os profetas, então disse o povo, cada um ao seu
companheiro: Que é o que sucedeu ao filho de Quis? Está também Saul
entre os profetas? Então um homem dali respondeu, e disse:
Pois quem é o pai deles? Pelo que se tornou em provérbio: Está Saul
também entre os profetas? E, acabando de profetizar, foi ao
alto. E disse-lhe o tio de Saul, a ele e ao seu moço: Aonde
fostes? E disse ele: A buscar as jumentas, e, vendo que não
apareciam, fomos a Samuel. Então disse o tio de Saul:
Declara-me, peço-te, o que vos disse Samuel? E disse Saul a
seu tio: Declarou-nos, na verdade, que as jumentas foram
encontradas. Porém o negócio do reino, de que Samuel falara, não lhe
declarou.

Convocou, pois, Samuel o povo ao Senhor, em Mizpá. E disse
aos filhos de Israel: Assim disse o Senhor Deus de Israel: Eu fiz
subir a Israel do Egito, e livrei-vos da mão dos egípcios e da mão
de todos os reinos que vos oprimiam. Mas vós tendes rejeitado
hoje a vosso Deus, que vos livrou de todos os vossos males e
trabalhos, e lhe tendes falado: Põe um rei sobre nós. Agora, pois,
ponde-vos perante o Senhor, pelas vossas tribos e segundo os vossos
milhares. Tendo, pois, Samuel feito chegar todas as tribos,
tomou-se a tribo de Benjamim. E, fazendo chegar a tribo de
Benjamim pelas suas famílias, tomou-se a família de Matri; e dela se
tomou Saul, filho de Quis; e o buscaram, porém não se achou.
Então tornaram a perguntar ao Senhor se aquele homem ainda
viria ali. E disse o Senhor: Eis que se escondeu entre a bagagem.
E correram, e o tomaram dali, e pôs-se no meio do povo; e era
mais alto do que todo o povo desde o ombro para cima. Então
disse Samuel a todo o povo: Vedes já a quem o Senhor escolheu? Pois
em todo o povo não há nenhum semelhante a ele. Então jubilou todo o
povo, e disse: Viva o rei! E declarou Samuel ao povo o
direito do reino, e escreveu-o num livro, e pô-lo perante o Senhor;
então despediu Samuel a todo o povo, cada um para sua casa. E
foi também Saul à sua casa, em Gibeá; e foram com ele do exército
aqueles cujos corações Deus tocara. Mas os filhos de Belial
disseram: É este o que nos há de livrar? E o desprezaram, e não lhe
trouxeram presentes; porém ele se fez como surdo.

\medskip

\lettrine{11} Então subiu Naás, amonita, e sitiou a
Jabes-Gileade; e disseram todos os homens de Jabes a Naás: Faze
aliança conosco, e te serviremos. Porém Naás, amonita, lhes
disse: Com esta condição farei aliança convosco: que a todos vos
arranque o olho direito, e assim ponha esta afronta sobre todo o
Israel. Então os anciãos de Jabes lhe disseram: Deixa-nos por
sete dias, para que enviemos mensageiros por todos os termos de
Israel, e, não havendo ninguém que nos livre, então viremos a ti.
E, vindo os mensageiros a Gibeá de Saul, falaram estas palavras
aos ouvidos do povo. Então todo o povo levantou a sua voz, e chorou.

E eis que Saul vinha do campo, atrás dos bois; e disse Saul: Que
tem o povo, que chora? E contaram-lhe as palavras dos homens de
Jabes. Então o Espírito de Deus se apoderou de Saul, ouvindo
estas palavras; e acendeu-se em grande maneira a sua ira. E
tomou uma junta de bois, e cortou-os em pedaços, e os enviou a todos
os termos de Israel pelas mãos dos mensageiros, dizendo: Qualquer
que não seguir a Saul e a Samuel, assim se fará aos seus bois. Então
caiu o temor do Senhor sobre o povo, e saíram como um só homem.
E contou-os em Bezeque; e houve dos filhos de Israel trezentos
mil, e dos homens de Judá trinta mil. Então disseram aos
mensageiros que vieram: Assim direis aos homens de Jabes-Gileade:
Amanhã, em aquecendo o sol, vos virá livramento. Vindo, pois, os
mensageiros, e anunciando-o aos homens de Jabes, se alegraram.
E os homens de Jabes disseram aos amonitas: Amanhã sairemos a
vós; então nos fareis conforme a tudo o que parecer bem aos vossos
olhos. E sucedeu que ao outro dia Saul pôs o povo em três
companhias, e vieram ao meio do arraial pela vigília da manhã, e
feriram aos amonitas até que o dia aqueceu; e sucedeu que os
restantes se espalharam, de modo que não ficaram dois deles juntos.

Então disse o povo a Samuel: Quem é aquele que dizia que Saul não
reinaria sobre nós? Dai-nos aqueles homens, e os mataremos.
Porém Saul disse: Hoje não morrerá nenhum, pois hoje tem
feito o Senhor um livramento em Israel. E disse Samuel ao
povo: Vinde, vamos nós a Gilgal, e renovemos ali o reino. E
todo o povo partiu para Gilgal, onde proclamaram a Saul por rei
perante o Senhor, e ofereceram ali ofertas pacíficas perante o
Senhor; e Saul se alegrou muito ali com todos os homens de Israel.

\medskip

\lettrine{12} Então disse Samuel a todo o Israel: Eis que ouvi
a vossa voz em tudo quanto me dissestes, e constituí sobre vós um
rei. Agora, pois, eis que o rei vai adiante de vós. Eu já
envelheci e encaneci\footnote{Tornar branco a pouco e pouco (o
cabelo, a barba). Criar cãs; envelhecer.}, e eis que meus filhos
estão convosco, e tenho andado diante de vós desde a minha mocidade
até ao dia de hoje. Eis-me aqui; testificai contra mim perante o
Senhor, e perante o seu ungido, a quem o boi tomei, a quem o jumento
tomei, e a quem defraudei, a quem tenho oprimido, e de cuja mão
tenho recebido suborno e com ele encobri os meus olhos, e vo-lo
restituirei. Então disseram: Em nada nos defraudaste, nem nos
oprimiste, nem recebeste coisa alguma da mão de ninguém. E ele
lhes disse: O Senhor seja testemunha contra vós, e o seu ungido seja
hoje testemunha, que nada tendes achado na minha mão. E disse o
povo: Ele é testemunha.

Então disse Samuel ao povo: O Senhor é o que escolheu a Moisés e a
Arão, e tirou a vossos pais da terra do Egito. Agora, pois,
ponde-vos aqui em pé, e pleitearei convosco perante o Senhor, sobre
todos os atos de justiça do Senhor, que fez a vós e a vossos pais.
Havendo entrado Jacó no Egito, vossos pais clamaram ao Senhor, e
o Senhor enviou a Moisés e a Arão que tiraram a vossos pais do
Egito, e os fizeram habitar neste lugar. Porém esqueceram-se do
Senhor seu Deus; então os vendeu à mão de Sísera, capitão do
exército de Hazor, e na mão dos filisteus, e na mão do rei dos
moabitas, que pelejaram contra eles. E clamaram ao Senhor, e
disseram: Pecamos, pois deixamos ao Senhor, e servimos aos baalins e
astarotes; agora, pois, livra-nos da mão de nossos inimigos, e te
serviremos. E o Senhor enviou a Jerubaal, e a Baraque, e a
Jefté, e a Samuel; e livrou-vos da mão de vossos inimigos em redor,
e habitastes seguros. E vendo vós que Naás, rei dos filhos de
Amom, vinha contra vós, me dissestes: Não, mas reinará sobre nós um
rei; sendo, porém, o Senhor vosso Deus, o vosso rei. Agora,
pois, vedes aí o rei que elegestes e que pedistes; e eis que o
Senhor tem posto sobre vós um rei. Se temerdes ao Senhor, e o
servirdes, e derdes ouvidos à sua voz, e não fordes rebeldes ao
mandado do Senhor, assim vós, como o rei que reina sobre vós,
seguireis o Senhor vosso Deus. Mas se não derdes ouvidos à
voz do Senhor, e antes fordes rebeldes ao mandado do Senhor, a mão
do Senhor será contra vós, como o era contra vossos pais.

Ponde-vos também agora aqui, e vede esta grande coisa que o
Senhor vai fazer diante dos vossos olhos. Não é hoje a sega
do trigo? Clamarei, pois, ao Senhor, e dará trovões e chuva; e
sabereis e vereis que é grande a vossa maldade, que tendes feito
perante o Senhor, pedindo para vós um rei. Então invocou
Samuel ao Senhor, e o Senhor deu trovões e chuva naquele dia; por
isso todo o povo temeu sobremaneira ao Senhor e a Samuel. E
todo o povo disse a Samuel: Roga pelos teus servos ao Senhor teu
Deus, para que não venhamos a morrer; porque a todos os nossos
pecados temos acrescentado este mal, de pedirmos para nós um rei.
Então disse Samuel ao povo: Não temais; vós tendes cometido
todo este mal; porém não vos desvieis de seguir ao Senhor, mas servi
ao Senhor com todo o vosso coração. E não vos desvieis; pois
seguiríeis as vaidades, que nada aproveitam, e tampouco vos
livrarão, porque vaidades são. Pois o Senhor, por causa do
seu grande nome, não desamparará o seu povo; porque aprouve ao
Senhor fazer-vos o seu povo. E quanto a mim, longe de mim que
eu peque contra o Senhor, deixando de orar por vós; antes vos
ensinarei o caminho bom e direito. Tão-somente temei ao
Senhor, e servi-o fielmente com todo o vosso coração; porque vede
quão grandiosas coisas vos fez. Porém, se perseverardes em
fazer mal, perecereis, assim vós como o vosso rei.

\medskip

\lettrine{13} Saul reinou um ano; e no segundo ano do seu
reinado sobre Israel, Saul escolheu para si três mil homens de
Israel; e estavam com Saul dois mil em Micmás e na montanha de
Betel, e mil estavam com Jônatas em Gibeá de Benjamim; e o resto do
povo despediu, cada um para sua casa. E Jônatas feriu a
guarnição dos filisteus, que estava em Gibeá, o que os filisteus
ouviram; pelo que Saul tocou a trombeta por toda a terra, dizendo:
Ouçam os hebreus. Então todo o Israel ouviu dizer: Saul feriu a
guarnição dos filisteus, e também Israel se fez abominável aos
filisteus. Então o povo foi convocado para junto de Saul em Gilgal.
E os filisteus se ajuntaram para pelejar contra Israel, trinta
mil carros, e seis mil cavaleiros, e povo em multidão como a areia
que está à beira do mar; e subiram, e se acamparam em Micmás, ao
oriente de Bete-Áven. Vendo, pois, os homens de Israel que
estavam em apuros (porque o povo estava angustiado), o povo se
escondeu pelas cavernas, e pelos espinhais, e pelos penhascos, e
pelas fortificações, e pelas covas. E alguns dos hebreus
passaram o Jordão para a terra de Gade e Gileade; e, estando Saul
ainda em Gilgal, todo o povo ia atrás dele tremendo.

E esperou Saul sete dias, até ao tempo que Samuel determinara; não
vindo, porém, Samuel a Gilgal, o povo se dispersava dele. Então
disse Saul: Trazei-me aqui um holocausto, e ofertas pacíficas. E
ofereceu o holocausto. E sucedeu que, acabando ele de
oferecer o holocausto, eis que Samuel chegou; e Saul lhe saiu ao
encontro, para o saudar. Então disse Samuel: Que fizeste?
Disse Saul: Porquanto via que o povo se espalhava de mim, e tu não
vinhas nos dias aprazados, e os filisteus já se tinham ajuntado em
Micmás, eu disse: Agora descerão os filisteus sobre mim a
Gilgal, e ainda à face do Senhor não orei; e constrangi-me, e
ofereci holocausto. Então disse Samuel a Saul: Procedeste
nesciamente, e não guardaste o mandamento que o Senhor teu Deus te
ordenou; porque agora o Senhor teria confirmado o teu reino sobre
Israel para sempre; porém agora não subsistirá o teu reino;
já tem buscado o Senhor para si um homem segundo o seu coração, e já
lhe tem ordenado o Senhor, que seja capitão sobre o seu povo,
porquanto não guardaste o que o Senhor te ordenou.

Então se levantou Samuel, e subiu de Gilgal a Gibeá de Benjamim;
e Saul contou o povo que se achava com ele, uns seiscentos homens.
E Saul e Jônatas, seu filho, e o povo que se achou com eles,
ficaram em Gibeá de Benjamim; porém os filisteus se acamparam em
Micmás. E os saqueadores saíram do campo dos filisteus em
três companhias; uma das companhias foi pelo caminho de Ofra à terra
de Sual. Outra companhia seguiu pelo caminho de Bete-Horom, e
a outra companhia foi pelo caminho do termo que dá para o vale
Zeboim na direção do deserto. E em toda a terra de Israel nem
um ferreiro se achava, porque os filisteus tinham dito: Para que os
hebreus não façam espada nem lança. Por isso todo o Israel
tinha que descer aos filisteus para amolar cada um a sua relha, e a
sua enxada, e o seu machado, e o seu sacho\footnote{Pequena enxada,
estreita e longa, em geral com uma orelha pontiaguda ou bifurcada na
parte superior, acima do olho.}. Tinham porém limas para os
seus sachos, e para as suas enxadas, e para as forquilhas de três
dentes, e para os machados, e para consertar as
aguilhadas\footnote{Vara comprida com ferrão na ponta, usada para
tanger os bois.}. E sucedeu que, no dia da peleja, não se
achou nem espada nem lança na mão de todo o povo que estava com Saul
e com Jônatas; porém acharam-se com Saul e com Jônatas seu filho.
E saiu a guarnição dos filisteus ao desfiladeiro de Micmás.

\medskip

\lettrine{14} Sucedeu, pois, que um dia disse Jônatas, filho
de Saul, ao moço que lhe levava as armas: Vem, passemos à guarnição
dos filisteus, que está lá daquele lado. Porém não o fez saber a seu
pai. E estava Saul à extremidade de Gibeá, debaixo da romeira
que havia em Migrom; e o povo que estava com ele era uns seiscentos
homens. E Aías, filho de Aitube, irmão de Icabode, o filho de
Finéias, filho de Eli, sacerdote do Senhor em Siló, trazia o éfode;
porém o povo não sabia que Jônatas tinha ido. E entre os
desfiladeiros pelos quais Jônatas procurava passar à guarnição dos
filisteus, deste lado havia uma penha aguda, e do outro lado uma
penha aguda; e era o nome de uma Bozez, e o nome da outra Sené.
Uma penha para o norte estava defronte de Micmás, e a outra para
o sul, defronte de Gibeá. Disse, pois, Jônatas ao moço que lhe
levava as armas: Vem, passemos à guarnição destes incircuncisos;
porventura operará o Senhor por nós, porque para com o Senhor nenhum
impedimento há de livrar com muitos ou com poucos. Então o seu
pajem de armas lhe disse: Faze tudo o que tens no coração; segue,
eis-me aqui contigo, conforme o que quiseres. Disse, pois,
Jônatas: Eis que passaremos àqueles homens, e nos revelaremos a
eles. Se nos disserem assim: Parai até que cheguemos a vós;
então ficaremos no nosso lugar, e não subiremos a eles.
Porém, se disserem: Subi a nós; então subiremos, pois o
Senhor os tem entregado nas nossas mãos, e isto nos será por sinal.
Revelando-se eles à guarnição dos filisteus, disseram os
filisteus: Eis que já os hebreus saíram das cavernas em que se
tinham escondido. E os homens da guarnição responderam a
Jônatas e ao seu pajem de armas, e disseram: Subi a nós, e nós vos
ensinaremos uma lição. E disse Jônatas ao seu pajem de armas: Sobe
atrás de mim, porque o Senhor os tem entregado na mão de Israel.
Então subiu Jônatas com os pés e com as mãos, e o seu pajem
de armas atrás dele; e os filisteus caíam diante de Jônatas, e o seu
pajem de armas os matava atrás dele. E sucedeu esta primeira
derrota, em que Jônatas e o seu pajem de armas feriram uns vinte
homens, em cerca de meia jeira\footnote{Antiga unidade de medida de
área de superfície agrária, equivalente a 400 braças quadradas, ou
seja, 0,2 hectare.} de terra que uma junta de bois podia lavrar.
E houve tremor no arraial, no campo e em todo o povo; também
a mesma guarnição e os saqueadores tremeram, até a terra se
estremeceu porquanto era tremor de Deus.

Olharam, pois, as sentinelas de Saul em Gibeá de Benjamim, e eis
que a multidão se dissolvia, e fugia para cá e para lá. Disse
então Saul ao povo que estava com ele: Ora contai, e vede quem é que
saiu dentre nós. E contaram, e eis que nem Jônatas nem o seu pajem
de armas estavam ali. Então Saul disse a Aías: Traze aqui a
arca de Deus (porque naquele dia estava a arca de Deus com os filhos
de Israel). E sucedeu que, estando Saul ainda falando com o
sacerdote, o alvoroço que havia no arraial dos filisteus ia
crescendo muito, e se multiplicava, pelo que disse Saul ao
sacerdote: Retira a tua mão. Então Saul e todo o povo que
havia com ele se reuniram, e foram à peleja; e eis que a espada de
um era contra o outro, e houve mui grande tumulto. Também com
os filisteus havia hebreus, como dantes, que subiram com eles ao
arraial em redor; e também estes se ajuntaram com os israelitas que
estavam com Saul e Jônatas. Ouvindo, pois, todos os homens de
Israel que se esconderam pela montanha de Efraim que os filisteus
fugiam, eles também os perseguiram de perto na peleja. Assim
livrou o Senhor a Israel naquele dia; e o arraial passou a
Bete-Áven.

E estavam os homens de Israel já exaustos naquele dia, porquanto
Saul conjurou o povo, dizendo: Maldito o homem que comer pão até à
tarde, antes que me vingue de meus inimigos. Por isso todo o povo se
absteve de provar pão. E todo o povo chegou a um bosque; e
havia mel na superfície do campo. E, chegando o povo ao
bosque, eis que havia um manancial de mel; porém ninguém chegou a
mão à boca, porque o povo temia a conjuração. Porém Jônatas
não tinha ouvido quando seu pai conjurara o povo, e estendeu a ponta
da vara que tinha na mão, e a molhou no favo de mel; e, tornando a
mão à boca, aclararam-se os seus olhos. Então respondeu um do
povo, e disse: Solenemente conjurou teu pai o povo, dizendo: Maldito
o homem que comer hoje pão. Por isso o povo desfalecia. Então
disse Jônatas: Meu pai tem turbado a terra; ora vede como se me
aclararam os olhos por ter provado um pouco deste mel, quanto
mais se o povo hoje livremente tivesse comido do despojo que achou
de seus inimigos. Porém agora não foi tão grande o estrago dos
filisteus. Feriram, porém, aquele dia aos filisteus, desde
Micmás até Aijalom, e o povo desfaleceu em extremo. Então o
povo se lançou ao despojo, e tomaram ovelhas, e vacas, e bezerros, e
os degolaram no chão; e o povo os comeu com sangue. E o
anunciaram a Saul, dizendo: Eis que o povo peca contra o Senhor,
comendo com sangue. E disse: Aleivosamente procedestes; trazei-me
aqui já uma grande pedra. Disse mais Saul: Dispersai-vos
entre o povo, e dizei-lhes: Trazei-me cada um o seu boi, e cada um a
sua ovelha, e degolai-os aqui, e comei, e não pequeis contra o
Senhor, comendo com sangue. Então todo o povo trouxe de noite, cada
um pela sua mão, o seu boi, e os degolaram ali. Então
edificou Saul um altar ao Senhor; este foi o primeiro altar que
edificou ao Senhor.

Depois disse Saul: Desçamos de noite atrás dos filisteus, e
despojemo-los, até que amanheça o dia, e não deixemos deles um só
homem. E disseram: Tudo o que parecer bem aos teus olhos faze.
Disse, porém, o sacerdote: Cheguemo-nos aqui a Deus. Então
consultou Saul a Deus, dizendo: Descerei atrás dos filisteus?
Entregá-los-ás na mão de Israel? Porém aquele dia não lhe respondeu.
Então disse Saul: Chegai-vos para cá, todos os chefes do
povo, e informai-vos, e vede em que se cometeu hoje este pecado.
Porque vive o Senhor que salva a Israel, que, ainda que seja
em meu filho Jônatas, certamente morrerá. E nenhum de todo o povo
lhe respondeu. Disse mais a todo o Israel: Vós estareis de um
lado, e eu e meu filho Jônatas estaremos do outro lado. Então disse
o povo a Saul: Faze o que parecer bem aos teus olhos. Falou,
pois, Saul ao Senhor Deus de Israel: Mostra o inocente. Então
Jônatas e Saul foram tomados por sorte, e o povo saiu livre.
Então disse Saul: Lançai a sorte entre mim e Jônatas, meu
filho. E foi tomado Jônatas. Disse então Saul a Jônatas:
Declara-me o que tens feito. E Jônatas lho declarou, e disse:
Tão-somente provei um pouco de mel com a ponta da vara que tinha na
mão; eis que devo morrer? Então disse Saul: Assim me faça
Deus, e outro tanto, que com certeza morrerás, Jônatas. Porém
o povo disse a Saul: Morrerá Jônatas, que efetuou tão grande
salvação em Israel? Nunca tal suceda; vive o Senhor, que não lhe há
de cair no chão um só cabelo da sua cabeça! pois com Deus fez isso
hoje. Assim o povo livrou a Jônatas, para que não morresse. E
Saul deixou de seguir os filisteus; e os filisteus se foram ao seu
lugar.

Então tomou Saul o reino sobre Israel; e pelejou contra todos os
seus inimigos em redor; contra Moabe, e contra os filhos de Amom, e
contra Edom, e contra os reis de Zobá, e contra os filisteus, e para
onde quer que se tornava executava castigo. E houve-se
valorosamente, e feriu aos amalequitas, e liberou a Israel da mão
dos que o saqueavam. E os filhos de Saul eram Jônatas, e
Isvi, e Malquisua; e os nomes de suas duas filhas eram estes: o da
mais velha Merabe, e o da mais nova, Mical. E o nome da
mulher de Saul, Ainoã, filha de Aimaás; e o nome do capitão do
exército, Abner, filho de Ner, tio de Saul. E Quis, pai de
Saul, e Ner, pai de Abner, eram filhos de Abiel. E houve uma
forte guerra contra os filisteus, todos os dias de Saul; por isso
Saul a todos os homens valentes e valorosos que via, os agregava a
si.

\medskip

\lettrine{15} Então disse Samuel a Saul: Enviou-me o Senhor a
ungir-te rei sobre o seu povo, sobre Israel; ouve, pois, agora a voz
das palavras do Senhor. Assim diz o Senhor dos Exércitos: Eu me
recordei do que fez Amaleque a Israel; como se lhe opôs no caminho,
quando subia do Egito. Vai, pois, agora e fere a Amaleque; e
destrói totalmente a tudo o que tiver, e não lhe perdoes; porém
matarás desde o homem até à mulher, desde os meninos até aos de
peito, desde os bois até às ovelhas, e desde os camelos até aos
jumentos. O que Saul convocou ao povo, e os contou em Telaim,
duzentos mil homens de pé, e dez mil homens de Judá. Chegando,
pois, Saul à cidade de Amaleque, pôs emboscada no vale. E disse
Saul aos queneus: Ide-vos, retirai-vos e saí do meio dos
amalequitas, para que não vos destrua juntamente com eles, porque
vós usastes de misericórdia com todos os filhos de Israel, quando
subiram do Egito. Assim os queneus se retiraram do meio dos
amalequitas. Então feriu Saul aos amalequitas desde Havilá até
chegar a Sur, que está defronte do Egito. E tomou vivo a Agague,
rei dos amalequitas; porém a todo o povo destruiu ao fio da espada.
E Saul e o povo pouparam a Agague, e ao melhor das ovelhas e das
vacas, e as da segunda ordem, e aos cordeiros e ao melhor que havia,
e não os quiseram destruir totalmente; porém a toda a coisa vil e
desprezível destruíram totalmente.

Então veio a palavra do Senhor a Samuel, dizendo:
Arrependo-me de haver posto a Saul como rei; porquanto deixou
de me seguir, e não cumpriu as minhas palavras. Então Samuel se
contristou, e toda a noite clamou ao Senhor. E madrugou
Samuel para encontrar a Saul pela manhã; e anunciou-se a Samuel,
dizendo: Já chegou Saul ao Carmelo, e eis que levantou para si uma
coluna. Então voltando, passou e desceu a Gilgal. Veio, pois,
Samuel a Saul; e Saul lhe disse: Bendito sejas tu do Senhor; cumpri
a palavra do Senhor. Então disse Samuel: Que
balido\footnote{Grito de ovelha ou de cordeiro. Fig. Queixa dos
paroquianos contra o seu pároco.}, pois, de ovelhas é este aos meus
ouvidos, e o mugido de vacas que ouço? E disse Saul: De
Amaleque as trouxeram; porque o povo poupou ao melhor das ovelhas, e
das vacas, para as oferecer ao Senhor teu Deus; o resto, porém,
temos destruído totalmente. Então disse Samuel a Saul:
Espera, e te declararei o que o Senhor me disse esta noite. E ele
disse-lhe: Fala. E disse Samuel: Porventura, sendo tu pequeno
aos teus olhos, não foste por cabeça das tribos de Israel? E o
Senhor te ungiu rei sobre Israel. E enviou-te o Senhor a este
caminho, e disse: Vai, e destrói totalmente a estes pecadores, os
amalequitas, e peleja contra eles, até que os aniquiles. Por
que, pois, não deste ouvidos à voz do Senhor, antes te lançaste ao
despojo, e fizeste o que parecia mau aos olhos do Senhor?
Então disse Saul a Samuel: Antes dei ouvidos à voz do Senhor,
e caminhei no caminho pelo qual o Senhor me enviou; e trouxe a
Agague, rei de Amaleque, e os amalequitas destruí totalmente;
mas o povo tomou do despojo ovelhas e vacas, o melhor do
interdito, para oferecer ao Senhor teu Deus em Gilgal. Porém
Samuel disse: Tem porventura o Senhor tanto prazer em holocaustos e
sacrifícios, como em que se obedeça à palavra do Senhor? Eis que o
obedecer é melhor do que o sacrificar; e o atender melhor é do que a
gordura de carneiros. Porque a rebelião é como o pecado de
feitiçaria, e o porfiar\footnote{Discutir acaloradamente; contender,
debater, altercar. Debater com ardor; discutir, altercar. Fazer
empenho; teimar, insistir, obstinar-se. Competir, rivalizar,
concorrer.} é como iniqüidade e idolatria. Porquanto tu rejeitaste a
palavra do Senhor, ele também te rejeitou a ti, para que não sejas
rei.

Então disse Saul a Samuel: Pequei, porquanto tenho transgredido a
ordem do Senhor e as tuas palavras; porque temi ao povo, e dei
ouvidos à sua voz. Agora, pois, rogo-te perdoa o meu pecado;
e volta comigo, para que adore ao Senhor. Porém Samuel disse
a Saul: Não voltarei contigo; porquanto rejeitaste a palavra do
Senhor, já te rejeitou o Senhor, para que não sejas rei sobre
Israel. E virando-se Samuel para se ir, ele lhe pegou pela
orla da capa, e a rasgou. Então Samuel lhe disse: O Senhor
tem rasgado de ti hoje o reino de Israel, e o tem dado ao teu
próximo, melhor do que tu. E também aquele que é a Força de
Israel não mente nem se arrepende; porquanto não é um homem para que
se arrependa. Disse ele então: Pequei; honra-me, porém, agora
diante dos anciãos do meu povo, e diante de Israel; e volta comigo,
para que adore ao Senhor teu Deus. Então, voltando Samuel,
seguiu a Saul; e Saul adorou ao Senhor.

Então disse Samuel: Trazei-me aqui a Agague, rei dos amalequitas.
E Agague veio a ele animosamente; e disse Agague: Na verdade já
passou a amargura da morte. Disse, porém, Samuel: Assim como
a tua espada desfilhou as mulheres, assim ficará desfilhada a tua
mãe entre as mulheres. Então Samuel despedaçou a Agague perante o
Senhor em Gilgal. Então Samuel se foi a Ramá; e Saul subiu à
sua casa, a Gibeá de Saul. E nunca mais viu Samuel a Saul até
ao dia da sua morte; porque Samuel teve dó de Saul. E o Senhor se
arrependeu de haver posto a Saul rei sobre Israel.

\medskip

\lettrine{16} Então disse o Senhor a Samuel: Até quando terás
dó de Saul, havendo-o eu rejeitado, para que não reine sobre Israel?
Enche um chifre de azeite, e vem, enviar-te-ei a Jessé o belemita;
porque dentre os seus filhos me tenho provido de um rei. Porém
disse Samuel: Como irei eu? pois, ouvindo-o Saul, me matará. Então
disse o Senhor: Toma uma bezerra das vacas em tuas mãos, e dize: Vim
para sacrificar ao Senhor. E convidarás a Jessé ao sacrifício; e
eu te farei saber o que hás de fazer, e ungir-me-ás a quem eu te
disser. Fez, pois, Samuel o que dissera o Senhor, e veio a
Belém; então os anciãos da cidade saíram ao encontro, tremendo, e
disseram: De paz é a tua vinda? E disse ele: É de paz, vim
sacrificar ao Senhor; santificai-vos, e vinde comigo ao sacrifício.
E santificou ele a Jessé e a seus filhos, e os convidou ao
sacrifício.

E sucedeu que, entrando eles, viu a Eliabe, e disse: Certamente
está perante o Senhor o seu ungido. Porém o Senhor disse a
Samuel: Não atentes para a sua aparência, nem para a grandeza da sua
estatura, porque o tenho rejeitado; porque o Senhor não vê como vê o
homem, pois o homem vê o que está diante dos olhos, porém o Senhor
olha para o coração. Então chamou Jessé a Abinadabe, e o fez
passar diante de Samuel, o qual disse: Nem a este tem escolhido o
Senhor. Então Jessé fez passar a Sama; porém disse: Tampouco a
este tem escolhido o Senhor. Assim fez passar Jessé a seus
sete filhos diante de Samuel; porém Samuel disse a Jessé: O Senhor
não tem escolhido a estes. Disse mais Samuel a Jessé:
Acabaram-se os moços? E disse: Ainda falta o menor, que está
apascentando as ovelhas. Disse, pois, Samuel a Jessé: Manda
chamá-lo, porquanto não nos assentaremos até que ele venha aqui.
Então mandou chamá-lo e fê-lo entrar (e era ruivo e formoso
de semblante e de boa presença); e disse o Senhor: Levanta-te, e
unge-o, porque é este mesmo. Então Samuel tomou o chifre do
azeite, e ungiu-o no meio de seus irmãos; e desde aquele dia em
diante o Espírito do Senhor se apoderou de Davi; então Samuel se
levantou, e voltou a Ramá.

E o Espírito do Senhor se retirou de Saul, e atormentava-o um
espírito mau da parte do Senhor. Então os criados de Saul lhe
disseram: Eis que agora o espírito mau da parte de Deus te
atormenta; diga, pois, nosso senhor a seus servos, que estão
na tua presença, que busquem um homem que saiba tocar harpa, e será
que, quando o espírito mau da parte de Deus vier sobre ti, então ele
tocará com a sua mão, e te acharás melhor. Então disse Saul
aos seus servos: Buscai-me, pois, um homem que toque bem, e
trazei-mo. Então respondeu um dos moços, e disse: Eis que
tenho visto a um filho de Jessé, o belemita, que sabe tocar e é
valente e vigoroso, e homem de guerra, e prudente em palavras, e de
gentil presença; o Senhor é com ele. E Saul enviou
mensageiros a Jessé, dizendo: Envia-me Davi, teu filho, o que está
com as ovelhas. Então tomou Jessé um jumento carregado de
pão, e um odre de vinho, e um cabrito, e enviou-os a Saul pela mão
de Davi, seu filho. Assim Davi veio a Saul, e esteve perante
ele, e o amou muito, e foi seu pajem de armas. Então Saul
mandou dizer a Jessé: Deixa estar a Davi perante mim, pois achou
graça em meus olhos. E sucedia que, quando o espírito mau da
parte de Deus vinha sobre Saul, Davi tomava a harpa, e a tocava com
a sua mão; então Saul sentia alívio, e se achava melhor, e o
espírito mau se retirava dele.

\medskip

\lettrine{17} E os filisteus ajuntaram as suas forças para a
guerra e congregaram-se em Socó, que está em Judá, e acamparam-se
entre Socó e Azeca, em Efes-Damim. Porém Saul e os homens de
Israel se ajuntaram e acamparam no vale do carvalho, e ordenaram a
batalha contra os filisteus. E os filisteus estavam num monte de
um lado, e os israelitas estavam num monte do outro lado; e o vale
estava entre eles. Então saiu do arraial dos filisteus um homem
guerreiro, cujo nome era Golias, de Gate, que tinha de altura seis
côvados\footnote{Antiga unidade de medida de comprimento equivalente
a três palmos, ou seja, 0,66m.} e um palmo. Trazia na cabeça um
capacete de bronze, e vestia uma couraça de escamas; e era o peso da
couraça de cinco mil siclos\footnote{Unidade de peso utilizada no
Oriente antigo. Moeda dos hebreus, de prata pura, e que pesava seis
gramas.} de bronze. E trazia grevas\footnote{Parte da armadura
que cobria a perna, do joelho ao pé.} de bronze por cima de seus
pés, e um escudo de bronze entre os seus ombros. E a haste da
sua lança era como o eixo do tecelão, e a ponta da sua lança de
seiscentos siclos de ferro, e diante dele ia o escudeiro. E
parou, e clamou às companhias de Israel, e disse-lhes: Para que
saireis a ordenar a batalha? Não sou eu filisteu e vós servos de
Saul? Escolhei dentre vós um homem que desça a mim. Se ele puder
pelejar comigo, e me ferir, a vós seremos por servos; porém, se eu o
vencer, e o ferir, então a nós sereis por servos, e nos servireis.
Disse mais o filisteu: Hoje desafio as companhias de Israel,
dizendo: Dai-me um homem, para que ambos pelejemos. Ouvindo
então Saul e todo o Israel estas palavras do filisteu,
espantaram-se, e temeram muito.

E Davi era filho de um homem efrateu, de Belém de Judá, cujo nome
era Jessé, que tinha oito filhos; e nos dias de Saul era este homem
já velho e adiantado em idade entre os homens. Foram-se os
três filhos mais velhos de Jessé, e seguiram a Saul à guerra; e eram
os nomes de seus três filhos, que se foram à guerra, Eliabe, o
primogênito, e o segundo Abinadabe, e o terceiro Sama. E Davi
era o menor; e os três maiores seguiram a Saul. Davi, porém,
ia e voltava de Saul, para apascentar as ovelhas de seu pai em
Belém. Chegava-se, pois, o filisteu pela manhã e à tarde; e
apresentou-se por quarenta dias. E disse Jessé a Davi, seu
filho: Toma, peço-te, para teus irmãos um efa deste grão tostado e
estes dez pães, e corre a levá-los ao arraial, a teus irmãos.
Porém estes dez queijos de leite leva ao capitão de mil; e
visitarás a teus irmãos, a ver se vão bem; e tomarás o seu penhor.
E estavam Saul, e eles, e todos os homens de Israel no vale
do carvalho, pelejando com os filisteus. Davi então se
levantou de madrugada, pela manhã, e deixou as ovelhas com um
guarda, e carregou-se, e partiu, como Jessé lhe ordenara; e chegou
ao lugar dos carros, quando já o exército saía em ordem de batalha,
e a gritos chamavam à peleja. E os israelitas e filisteus se
puseram em ordem, fileira contra fileira. E Davi deixou a
carga que trouxera na mão do guarda da bagagem, e correu à batalha;
e, chegando, perguntou a seus irmãos se estavam bem. E,
estando ele ainda falando com eles, eis que vinha subindo do
exército dos filisteus o homem guerreiro, cujo nome era Golias, o
filisteu de Gate; e falou conforme àquelas palavras, e Davi as
ouviu. Porém todos os homens em Israel, vendo aquele homem,
fugiram de diante dele, e temiam grandemente. E diziam os
homens de Israel: Vistes aquele homem que subiu? Pois subiu para
afrontar a Israel; há de ser, pois, que, o homem que o ferir, o rei
o enriquecerá de grandes riquezas, e lhe dará a sua filha, e fará
livre a casa de seu pai em Israel. Então falou Davi aos
homens que estavam com ele, dizendo: Que farão àquele homem, que
ferir a este filisteu, e tirar a afronta de sobre Israel? Quem é,
pois, este incircunciso filisteu, para afrontar os exércitos do Deus
vivo? E o povo lhe tornou a falar conforme àquela palavra
dizendo: Assim farão ao homem que o ferir. E, ouvindo Eliabe,
seu irmão mais velho, falar àqueles homens, acendeu-se a ira de
Eliabe contra Davi, e disse: Por que desceste aqui? Com quem
deixaste aquelas poucas ovelhas no deserto? Bem conheço a tua
presunção, e a maldade do teu coração, que desceste para ver a
peleja. Então disse Davi: Que fiz eu agora? Porventura não há
razão para isso? E desviou-se dele para outro, e falou
conforme àquela palavra; e o povo lhe tornou a responder conforme às
primeiras palavras.

E, ouvidas as palavras que Davi havia falado, as anunciaram a
Saul, que mandou chamá-lo. E Davi disse a Saul: Não desfaleça
o coração de ninguém por causa dele; teu servo irá, e pelejará
contra este filisteu. Porém Saul disse a Davi: Contra este
filisteu não poderás ir para pelejar com ele; pois tu ainda és moço,
e ele homem de guerra desde a sua mocidade. Então disse Davi
a Saul: Teu servo apascentava as ovelhas de seu pai; e quando vinha
um leão e um urso, e tomava uma ovelha do rebanho, eu saía
após ele e o feria, e livrava-a da sua boca; e, quando ele se
levantava contra mim, lançava-lhe mão da barba, e o feria e o
matava. Assim feria o teu servo o leão, como o urso; assim
será este incircunciso filisteu como um deles; porquanto afrontou os
exércitos do Deus vivo. Disse mais Davi: O Senhor me livrou
das garras do leão, e das do urso; ele me livrará da mão deste
filisteu. Então disse Saul a Davi: Vai, e o Senhor seja contigo.
E Saul vestiu a Davi de suas vestes, e pôs-lhe sobre a cabeça
um capacete de bronze; e o vestiu de uma couraça. E Davi
cingiu a espada sobre as suas vestes, e começou a andar; porém nunca
o havia experimentado; então disse Davi a Saul: Não posso andar com
isto, pois nunca o experimentei. E Davi tirou aquilo de sobre si.

E tomou o seu cajado na mão, e escolheu para si cinco seixos do
ribeiro, e pô-los no alforje de pastor, que trazia, a saber, no
surrão\footnote{Bolsa ou saco de couro usado sobretudo para farnel
de pastores; sarrão.}, e lançou mão da sua funda; e foi
aproximando-se do filisteu. O filisteu também vinha se
aproximando de Davi; e o que lhe levava o escudo ia adiante dele.
E, olhando o filisteu, e vendo a Davi, o desprezou, porquanto
era moço, ruivo, e de gentil aspecto. Disse, pois, o filisteu
a Davi: Sou eu algum cão, para tu vires a mim com paus? E o filisteu
pelos seus deuses amaldiçoou a Davi. Disse mais o filisteu a
Davi: Vem a mim, e darei a tua carne às aves do céu e às bestas do
campo. Davi, porém, disse ao filisteu: Tu vens a mim com
espada, e com lança, e com escudo; porém eu venho a ti em nome do
Senhor dos Exércitos, o Deus dos exércitos de Israel, a quem tens
afrontado. Hoje mesmo o Senhor te entregará na minha mão, e
ferir-te-ei, e tirar-te-ei a cabeça, e os corpos do arraial dos
filisteus darei hoje mesmo às aves do céu e às feras da terra; e
toda a terra saberá que há Deus em Israel; e saberá toda esta
congregação que o Senhor salva, não com espada, nem com lança;
porque do Senhor é a guerra, e ele vos entregará na nossa mão.

E sucedeu que, levantando-se o filisteu, e indo encontrar-se com
Davi, apressou-se Davi, e correu ao combate, a encontrar-se com o
filisteu. E Davi pôs a mão no alforje, e tomou dali uma pedra
e com a funda lha atirou, e feriu o filisteu na testa, e a pedra se
lhe encravou na testa, e caiu sobre o seu rosto em terra.
Assim Davi prevaleceu contra o filisteu, com uma funda e com
uma pedra, e feriu o filisteu, e o matou; sem que Davi tivesse uma
espada na mão. Por isso correu Davi, e pôs-se em pé sobre o
filisteu, e tomou a sua espada, e tirou-a da bainha, e o matou, e
lhe cortou com ela a cabeça; vendo então os filisteus, que o seu
herói era morto, fugiram. Então os homens de Israel e Judá se
levantaram, e jubilaram, e seguiram os filisteus, até chegar ao
vale, e até às portas de Ecrom; e caíram os feridos dos filisteus
pelo caminho de Saaraim até Gate e até Ecrom. Então voltaram
os filhos de Israel de perseguirem os filisteus, e despojaram os
seus arraiais. E Davi tomou a cabeça do filisteu, e a trouxe
a Jerusalém; porém pôs as armas dele na sua tenda. Vendo,
porém, Saul, sair Davi a encontrar-se com o filisteu, disse a Abner,
o capitão do exército: De quem é filho este moço, Abner? E disse
Abner: Vive a tua alma, ó rei, que o não sei. Disse então o
rei: Pergunta, pois, de quem é filho este moço. Voltando,
pois, Davi de ferir o filisteu, Abner o tomou consigo, e o trouxe à
presença de Saul, trazendo ele na mão a cabeça do filisteu. E
disse-lhe Saul: De quem és filho, jovem? E disse Davi: Filho de teu
servo Jessé, belemita.

\medskip

\lettrine{18} E sucedeu que, acabando ele de falar com Saul, a
alma de Jônatas se ligou com a alma de Davi; e Jônatas o amou, como
à sua própria alma. E Saul naquele dia o tomou, e não lhe
permitiu que voltasse para casa de seu pai. E Jônatas e Davi
fizeram aliança; porque Jônatas o amava como à sua própria alma.
E Jônatas se despojou da capa que trazia sobre si, e a deu a
Davi, como também as suas vestes, até a sua espada, e o seu arco, e
o seu cinto. E saía Davi aonde quer que Saul o enviasse e
conduzia-se com prudência, e Saul o pôs sobre os homens de guerra; e
era aceito aos olhos de todo o povo, e até aos olhos dos servos de
Saul. Sucedeu, porém, que, vindo eles, quando Davi voltava de
ferir os filisteus, as mulheres de todas as cidades de Israel saíram
ao encontro do rei Saul, cantando e dançando, com adufes, com
alegria, e com instrumentos de música. E as mulheres dançando e
cantando se respondiam umas às outras, dizendo: Saul feriu os seus
milhares, porém, Davi os seus dez milhares.

Então Saul se indignou muito, e aquela palavra pareceu mal aos
seus olhos, e disse: Dez milhares deram a Davi, e a mim somente
milhares; na verdade, que lhe falta, senão só o reino? E, desde
aquele dia em diante, Saul tinha Davi em suspeita. E
aconteceu no outro dia, que o mau espírito da parte de Deus se
apoderou de Saul, e profetizava no meio da casa; e Davi tocava a
harpa com a sua mão, como nos outros dias; Saul, porém, tinha na mão
uma lança. E Saul atirou com a lança, dizendo: Encravarei a
Davi na parede. Porém Davi se desviou dele por duas vezes.

E temia Saul a Davi, porque o Senhor era com ele e se tinha
retirado de Saul. Por isso Saul o desviou de si, e o pôs por
capitão de mil; e saía e entrava diante do povo. E Davi se
conduzia com prudência em todos os seus caminhos, e o Senhor era com
ele. Vendo então Saul que tão prudentemente se conduzia,
tinha receio dele. Porém todo o Israel e Judá amava a Davi,
porquanto saía e entrava diante deles. Por isso Saul disse a
Davi: Eis que Merabe, minha filha mais velha, te darei por mulher;
sê-me somente filho valoroso, e guerreia as guerras do Senhor
(porque Saul dizia consigo: Não seja contra ele a minha mão, mas sim
a dos filisteus). Mas Davi disse a Saul: Quem sou eu, e qual
é a minha vida e a família de meu pai em Israel, para vir a ser
genro do rei? Sucedeu, porém, que ao tempo que Merabe, filha
de Saul, devia ser dada a Davi, ela foi dada por mulher a Adriel,
meolatita. Mas Mical, a outra filha de Saul amava a Davi; o
que, sendo anunciado a Saul, pareceu isto bom aos seus olhos.
E Saul disse: Eu lha darei, para que lhe sirva de laço, e
para que a mão dos filisteus venha a ser contra ele. Pelo que Saul
disse a Davi: Com a outra serás hoje meu genro. E Saul deu
ordem aos seus servos: Falai em segredo a Davi, dizendo: Eis que o
rei te está mui afeiçoado, e todos os seus servos te amam; agora,
pois, consente em ser genro do rei. E os servos de Saul
falaram todas estas palavras aos ouvidos de Davi. Então disse Davi:
Parece-vos pouco aos vossos olhos ser genro do rei, sendo eu homem
pobre e desprezível? E os servos de Saul lhe anunciaram isto,
dizendo: Foram tais as palavras que falou Davi. Então disse
Saul: Assim direis a Davi: O rei não tem necessidade de dote, senão
de cem prepúcios de filisteus, para se tomar vingança dos inimigos
do rei. Porquanto Saul tentava fazer cair a Davi pela mão dos
filisteus. E anunciaram os seus servos estas palavras a Davi,
e este negócio pareceu bem aos olhos de Davi, de que fosse genro do
rei; porém ainda os dias não se haviam cumprido. Então Davi
se levantou, e partiu com os seus homens, e feriu dentre os
filisteus duzentos homens, e Davi trouxe os seus prepúcios, e os
entregou todos ao rei, para que fosse genro do rei; então Saul lhe
deu por mulher a sua filha. E viu Saul, e notou que o Senhor
era com Davi; e Mical, filha de Saul, o amava. Então Saul
temeu muito mais a Davi; e Saul foi todos os seus dias inimigo de
Davi. E, saindo os príncipes dos filisteus à campanha,
sucedia que Davi se conduzia com mais êxito do que todos os servos
de Saul; portanto o seu nome era muito estimado.

\medskip

\lettrine{19} E falou Saul a Jônatas, seu filho, e a todos os
seus servos, para que matassem a Davi. Porém Jônatas, filho de Saul,
estava mui afeiçoado a Davi. E Jônatas o anunciou a Davi,
dizendo: Meu pai, Saul, procura matar-te, pelo que agora guarda-te
pela manhã, e fica-te em oculto, e esconde-te. E sairei eu, e
estarei à mão de meu pai no campo em que estiverdes, e eu falarei de
ti a meu pai, e verei o que há, e to anunciarei. Então Jônatas
falou bem de Davi a Saul, seu pai, e disse-lhe: Não peque o rei
contra seu servo Davi, porque ele não pecou contra ti, e porque os
seus feitos te são muito bons. Porque expôs a sua vida, e feriu
aos filisteus, e fez o Senhor um grande livramento a todo o Israel;
tu mesmo o viste, e te alegraste; porque, pois, pecarias contra o
sangue inocente, matando a Davi, sem causa? E Saul deu ouvidos à
voz de Jônatas, e jurou Saul: Vive o Senhor, que não morrerá. E
Jônatas chamou a Davi, e contou-lhe todas estas palavras; e Jônatas
levou Davi a Saul, e esteve perante ele como antes.

E tornou a haver guerra; e saiu Davi, e pelejou contra os
filisteus, e feriu-os com grande matança, e fugiram diante dele.
Porém o espírito mau da parte do Senhor se tornou sobre Saul,
estando ele assentado em sua casa, e tendo na mão a sua lança; e
tocava Davi com a mão, a harpa. E procurou Saul encravar a
Davi na parede, porém ele se desviou de diante de Saul, o qual feriu
com a lança a parede; então fugiu Davi, e escapou naquela mesma
noite.

Porém Saul mandou mensageiros à casa de Davi, que o guardassem, e
o matassem pela manhã; do que Mical, sua mulher, avisou a Davi,
dizendo: Se não salvares a tua vida esta noite, amanhã te matarão.
Então Mical desceu a Davi por uma janela; e ele se foi, e
fugiu, e escapou. E Mical tomou uma estátua e a deitou na
cama, e pôs-lhe à cabeceira uma pele de cabra, e a cobriu com uma
coberta. E, mandando Saul mensageiros que trouxessem a Davi,
ela disse: Está doente. Então Saul tornou a mandar
mensageiros que fossem a Davi, dizendo: Trazei-mo na cama, para que
o mate. Vindo, pois, os mensageiros, eis que a estátua estava
na cama, e a pele de cabra à sua cabeceira. Então disse Saul
a Mical: Por que assim me enganaste, e deixaste ir e escapar o meu
inimigo? E disse Mical a Saul: Porque ele me disse: Deixa-me ir, por
que hei de eu matar-te?

Assim Davi fugiu e escapou, e foi a Samuel, em Ramá, e lhe
participou tudo quanto Saul lhe fizera; e foram, ele e Samuel, e
ficaram em Naiote. E o anunciaram a Saul, dizendo: Eis que
Davi está em Naiote, em Ramá. Então enviou Saul mensageiros
para trazerem a Davi, os quais viram uma congregação de profetas
profetizando, onde estava Samuel que presidia sobre eles; e o
Espírito de Deus veio sobre os mensageiros de Saul, e também eles
profetizaram. E, avisado disto Saul, enviou outros
mensageiros, e também estes profetizaram; então enviou Saul ainda
uns terceiros mensageiros, os quais também profetizaram.
Então foi também ele mesmo a Ramá, e chegou ao poço grande
que estava em Secu; e, perguntando, disse: Onde estão Samuel e Davi?
E disseram-lhe: Eis que estão em Naiote, em Ramá. Então foi
para Naiote, em Ramá; e o mesmo Espírito de Deus veio sobre ele, e
ia profetizando, até chegar a Naiote, em Ramá. E ele também
despiu as suas vestes, e profetizou diante de Samuel, e esteve nu
por terra todo aquele dia e toda aquela noite; por isso se diz: Está
também Saul entre os profetas?

\medskip

\lettrine{20} Então fugiu Davi de Naiote, em Ramá; e veio, e
disse a Jônatas: Que fiz eu? Qual é o meu crime? E qual é o meu
pecado diante de teu pai, que procura tirar-me a vida? E ele lhe
disse: Tal não suceda; não morrerás; eis que meu pai não faz coisa
nenhuma grande, nem pequena, sem primeiro me informar; por que,
pois, meu pai me encobriria este negócio? Não será assim. Então
Davi tornou a jurar, e disse: Teu pai sabe muito bem que achei graça
em teus olhos; por isso disse: Não saiba isto Jônatas, para que não
se magoe. Mas, na verdade, como vive o Senhor, e como vive a tua
alma, há apenas um passo entre mim e a morte. E disse Jônatas a
Davi: O que disser a tua alma, eu te farei. Disse Davi a
Jônatas: Eis que amanhã é a lua nova, em que costumo assentar-me com
o rei para comer; porém deixa-me ir, e esconder-me-ei no campo, até
à tarde do terceiro dia. Se teu pai notar a minha ausência,
dirás: Davi me pediu muito que o deixasse ir correndo a Belém, sua
cidade; porquanto se faz lá o sacrifício anual para toda a linhagem.
Se disser assim: Está bem; então teu servo tem paz; porém se
muito se indignar, sabe que já está inteiramente determinado no mal.
Usa, pois, de misericórdia com o teu servo, porque o fizeste
entrar contigo em aliança do Senhor; se, porém, há em mim crime,
mata-me tu mesmo; por que me levarias a teu pai?

Então disse Jônatas: Longe de ti tal coisa; porém se de alguma
forma soubesse que já este mal está inteiramente determinado por meu
pai, para que viesse sobre ti, não to revelaria eu? E disse
Davi a Jônatas: Quem me fará saber, se por acaso teu pai te
responder asperamente? Então disse Jônatas a Davi: Vem e
saiamos ao campo. E saíram ambos ao campo. E disse Jônatas a
Davi: O Senhor Deus de Israel seja testemunha! Sondando eu a meu pai
amanhã a estas horas, ou depois de amanhã, e eis que se houver coisa
favorável para Davi, e eu então não enviar a ti, e não to fizer
saber; o Senhor faça assim com Jônatas outro tanto; que se
aprouver a meu pai fazer-te mal, também to farei saber, e te
deixarei partir, e irás em paz; e o Senhor seja contigo, assim como
foi com meu pai. E, se eu então ainda viver, porventura não
usarás comigo da beneficência do Senhor, para que não morra?
Nem tampouco cortarás da minha casa a tua beneficência
eternamente; nem ainda quando o Senhor desarraigar da terra a cada
um dos inimigos de Davi. Assim fez Jônatas aliança com a casa
de Davi, dizendo: O Senhor o requeira da mão dos inimigos de Davi.
E Jônatas fez jurar a Davi de novo, porquanto o amava; porque
o amava com todo o amor da sua alma. E disse-lhe Jônatas:
Amanhã é a lua nova, e não te acharão no teu lugar, pois o teu
assento se achará vazio. E, ausentando-te tu três dias, desce
apressadamente, e vai àquele lugar onde te escondeste no dia do
negócio; e fica-te junto à pedra de Ezel. E eu atirarei três
flechas para aquele lado, como se atirasse ao alvo. E eis que
mandarei o moço dizendo: Anda, busca as flechas. Se eu expressamente
disser ao moço: Olha que as flechas estão para cá de ti; toma-o
contigo, e vem, porque há paz para ti, e não há nada, vive o Senhor.
Porém se disser ao moço assim: Olha que as flechas estão para
lá de ti; vai-te embora, porque o Senhor te deixa ir. E
quanto ao negócio de que eu e tu falamos, eis que o Senhor está
entre mim e ti eternamente.

Escondeu-se, pois, Davi no campo; e, sendo a lua nova,
assentou-se o rei para comer pão. E, assentando-se o rei,
como das outras vezes, no seu assento, no lugar junto à parede,
Jônatas se levantou, e assentou-se Abner ao lado de Saul; e o lugar
de Davi apareceu vazio. Porém naquele dia não disse Saul
nada, porque dizia: Aconteceu-lhe alguma coisa, pela qual não está
limpo; certamente não está limpo. Sucedeu também no outro
dia, o segundo da lua nova, que o lugar de Davi apareceu vazio;
disse, pois, Saul a Jônatas, seu filho: Por que não veio o filho de
Jessé nem ontem nem hoje a comer pão? E respondeu Jônatas a
Saul: Davi me pediu encarecidamente que o deixasse ir a Belém.
Dizendo: Peço-te que me deixes ir, porquanto a nossa linhagem
tem um sacrifício na cidade, e meu irmão mesmo me mandou ir; se,
pois, agora tenho achado graça em teus olhos, peço-te que me deixes
partir, para que veja a meus irmãos; por isso não veio à mesa do
rei. Então se acendeu a ira de Saul contra Jônatas, e
disse-lhe: Filho da mulher perversa e rebelde; não sei eu que tens
escolhido o filho de Jessé, para vergonha tua e para vergonha da
nudez de tua mãe? Porque todos os dias que o filho de Jessé
viver sobre a terra nem tu estarás seguro, nem o teu reino; pelo que
envia, e traze-mo nesta hora; porque é digno de morte. Então
respondeu Jônatas a Saul, seu pai, e lhe disse: Por que há de
morrer? Que tem feito? Então Saul atirou-lhe com a lança,
para o ferir; assim entendeu Jônatas que já seu pai tinha
determinado matar a Davi. Por isso Jônatas, todo
encolerizado, se levantou da mesa; e no segundo dia da lua nova não
comeu pão; porque se magoava por causa de Davi, porque seu pai o
tinha humilhado.

E aconteceu, pela manhã, que Jônatas saiu ao campo, ao tempo que
tinha ajustado com Davi, e um moço pequeno com ele. Então
disse ao seu moço: Corre a buscar as flechas que eu atirar. Correu,
pois, o moço, e ele atirou uma flecha, que fez passar além dele.
E, chegando o moço ao lugar da flecha que Jônatas tinha
atirado, gritou Jônatas atrás do moço, e disse: Não está porventura
a flecha mais para lá de ti? E tornou Jônatas a gritar atrás
do moço: Apressa-te, corre, não te demores. E o moço de Jônatas
apanhou as flechas, e veio a seu senhor. E o moço não
entendeu coisa alguma; só Jônatas e Davi sabiam deste negócio.
Então Jônatas deu as suas armas ao moço que trazia, e
disse-lhe: Anda, e leva-as à cidade. E, indo-se o moço,
levantou-se Davi do lado do sul, e lançou-se sobre o seu rosto em
terra, e inclinou-se três vezes; e beijaram-se um ao outro, e
choraram juntos, mas Davi chorou muito mais. E disse Jônatas
a Davi: Vai-te em paz; o que nós temos jurado ambos em nome do
Senhor, dizendo: O Senhor seja entre mim e ti, e entre a minha
descendência e a tua descendência, seja perpetuamente. Então
se levantou Davi, e partiu; e Jônatas entrou na cidade.

\medskip

\lettrine{21} Então veio Davi a Nobe, ao sacerdote Aimeleque;
e Aimeleque, tremendo, saiu ao encontro de Davi, e disse-lhe: Por
que vens só, e ninguém contigo? E disse Davi ao sacerdote
Aimeleque: O rei me encomendou um negócio, e me disse: Ninguém saiba
deste negócio, pelo qual eu te enviei, e o qual te ordenei; quanto
aos moços, apontei-lhes tal e tal lugar. Agora, pois, que tens à
mão? Dá-me cinco pães na minha mão, ou o que se achar. E,
respondendo o sacerdote a Davi, disse: Não tenho pão comum à mão;
há, porém, pão sagrado, se ao menos os moços se abstiveram das
mulheres. E respondeu Davi ao sacerdote, e lhe disse: As
mulheres, na verdade, se nos vedaram desde ontem e anteontem; quando
eu saí, os vasos dos moços eram santos; e de algum modo é pão comum,
sendo que hoje santifica-se outro no vaso. Então o sacerdote lhe
deu o pão sagrado, porquanto não havia ali outro pão senão os pães
da proposição, que se tiraram de diante do Senhor, para se pôr ali
pão quente no dia em que aquele se tirasse. Estava, porém, ali
naquele dia um dos criados de Saul, detido perante o Senhor, e era
seu nome Doegue, edomeu, o mais poderoso dos pastores de Saul. E
disse Davi a Aimeleque: Não tens aqui à mão lança ou espada alguma?
Porque não trouxe à mão nem a minha espada nem as minhas armas,
porque o negócio do rei era apressado. E disse o sacerdote: A
espada de Golias, o filisteu, a quem tu feriste no vale do carvalho,
eis que está aqui envolta num pano detrás do éfode. Se tu a queres
tomar, toma-a, porque nenhuma outra há aqui, senão aquela. E disse
Davi: Não há outra semelhante; dá-ma.

E Davi levantou-se, e fugiu aquele dia de diante de Saul, e foi a
Aquis, rei de Gate. Porém os criados de Aquis lhe disseram:
Não é este Davi, o rei da terra? Não se cantava deste nas danças,
dizendo: Saul feriu os seus milhares, porém Davi os seus dez
milhares? E Davi considerou estas palavras no seu ânimo, e
temeu muito diante de Aquis, rei de Gate. Por isso se
contrafez\footnote{Reproduzir, imitando; imitar, arremedar. Imitar
por falsificação. Violentar a vontade de; constranger. Apresentar-se
de forma diferente, de modo que não possa ser reconhecido;
disfarçar-se. Violentar a própria vontade. Ficar constrangido, ou
contrafeito.} diante dos olhos deles, e fez-se como doido entre as
suas mãos, e esgravatava\footnote{Esgravatar (ou esgaravatar ou
esgravatear ou esgaravatear): Tirar ou limpar com o esgaravatador.
Remexer ou escarafunchar com as unhas. Coçar ou limpar (o nariz, as
orelhas, etc.). Fig. Inquirir; esmiuçar; pesquisar.} nas portas de
entrada, e deixava correr a saliva pela barba. Então disse
Aquis aos seus criados: Eis que bem vedes que este homem está louco;
por que mo trouxestes a mim? Faltam-me a mim doidos, para que
trouxésseis a este para que fizesse doidices diante de mim? Há de
entrar este na minha casa?

\medskip

\lettrine{22} Então Davi se retirou dali, e escapou para a
caverna de Adulão; e ouviram-no seus irmãos e toda a casa de seu
pai, e desceram ali para ter com ele. E ajuntou-se a ele todo o
homem que se achava em aperto, e todo o homem endividado, e todo o
homem de espírito desgostoso, e ele se fez capitão deles; e eram com
ele uns quatrocentos homens. E foi Davi dali a Mizpá dos
moabitas, e disse ao rei dos moabitas: Deixa estar meu pai e minha
mãe convosco, até que saiba o que Deus há de fazer de mim. E
trouxe-os perante o rei dos moabitas, e ficaram com ele todos os
dias que Davi esteve no lugar forte. Porém o profeta Gade disse
a Davi: Não fiques naquele lugar forte; vai, e entra na terra de
Judá. Então Davi saiu, e foi para o bosque de Herete.

E ouviu Saul que já se sabia de Davi e dos homens que estavam com
ele; e estava Saul em Gibeá, debaixo de um arvoredo, em Ramá, e
tinha na mão a sua lança, e todos os seus criados estavam com ele.
Então disse Saul a todos os seus criados que estavam com ele:
Ouvi, peço-vos, filhos de Benjamim, dar-vos-á também o filho de
Jessé, a todos vós, terras e vinhas, e far-vos-á a todos capitães de
milhares e capitães de centenas, para que todos vós tenhais
conspirado contra mim, e ninguém há que me dê aviso de que meu filho
tem feito aliança com o filho de Jessé, e nenhum dentre vós há que
se doa de mim, e mo participe, pois meu filho tem contra mim
sublevado a meu servo, para me armar ciladas, como se vê neste dia?
Então respondeu Doegue, o edomeu, que também estava com os
criados de Saul, e disse: Vi o filho de Jessé chegar a Nobe, a
Aimeleque, filho de Aitube, o qual consultou por ele ao
Senhor, e lhe deu mantimento, e lhe deu também a espada de Golias, o
filisteu. Então o rei mandou chamar a Aimeleque, sacerdote,
filho de Aitube, e a toda a casa de seu pai, os sacerdotes que
estavam em Nobe; e todos eles vieram ao rei. E disse Saul:
Ouve, peço-te, filho de Aitube. E ele disse: Eis-me aqui, senhor
meu. Então lhe disse Saul: Por que conspirastes contra mim,
tu e o filho de Jessé? Pois deste-lhe pão e espada, e consultaste
por ele a Deus, para que se levantasse contra mim a armar-me
ciladas, como se vê neste dia? E respondeu Aimeleque ao rei e
disse: E quem, entre todos os teus criados, há tão fiel como Davi, o
genro do rei, pronto na sua obediência, e honrado na tua casa?
Comecei, porventura, hoje a consultar por ele a Deus? Longe
de mim tal! Não impute o rei coisa nenhuma a seu servo, nem a toda a
casa de meu pai, pois o teu servo não soube nada de tudo isso, nem
muito nem pouco. Porém o rei disse: Aimeleque, morrerás
certamente, tu e toda a casa de teu pai. E disse o rei aos da
sua guarda que estavam com ele: Virai-vos, e matai os sacerdotes do
Senhor, porque também a sua mão é com Davi, e porque souberam que
fugiu e não mo fizeram saber. Porém os criados do rei não quiseram
estender as suas mãos para arremeter contra os sacerdotes do Senhor.
Então disse o rei a Doegue: Vira-te, e arremete contra os
sacerdotes. Então se virou Doegue, o edomeu, e arremeteu contra os
sacerdotes, e matou naquele dia oitenta e cinco homens que vestiam
éfode de linho. Também a Nobe, cidade destes sacerdotes,
passou a fio de espada, desde o homem até à mulher, desde os meninos
até aos de peito, e até os bois, jumentos e ovelhas passou a fio de
espada.

Porém escapou um dos filhos de Aimeleque, filho de Aitube, cujo
nome era Abiatar, o qual fugiu para Davi. E Abiatar anunciou
a Davi que Saul tinha matado os sacerdotes do Senhor. Então
Davi disse a Abiatar: Bem sabia eu naquele dia que, estando ali
Doegue, o edomeu, não deixaria de o denunciar a Saul; eu dei ocasião
contra todas as almas da casa de teu pai. Fica comigo, não
temas, porque quem procurar a minha morte também procurará a tua,
pois estarás salvo comigo.

\medskip

\lettrine{23} E foi anunciado a Davi, dizendo: Eis que os
filisteus pelejam contra Queila, e saqueiam as eiras\footnote{Área
de terra batida, lajeada ou cimentada, onde se malham, trilham,
secam e limpam cereais e legumes.}. E consultou Davi ao Senhor,
dizendo: Irei eu, e ferirei a estes filisteus? E disse o Senhor a
Davi: Vai, e ferirás aos filisteus, e livrarás a Queila. Porém
os homens de Davi lhe disseram: Eis que tememos aqui em Judá, quanto
mais indo a Queila contra os esquadrões dos filisteus. Então
Davi tornou a consultar ao Senhor, e o Senhor lhe respondeu, e
disse: Levanta-te, desce a Queila, porque te dou os filisteus na tua
mão. Então Davi partiu com os seus homens a Queila, e pelejou
contra os filisteus, e levou os gados, e fez grande estrago entre
eles; e Davi livrou os moradores de Queila. E sucedeu que,
quando Abiatar, filho de Aimeleque, fugiu para Davi, a Queila,
desceu com o éfode na mão.

E foi anunciado a Saul que Davi tinha ido a Queila, e disse Saul:
Deus o entregou nas minhas mãos, pois está encerrado, entrando numa
cidade de portas e ferrolhos. Então Saul mandou chamar a todo o
povo à peleja, para que descessem a Queila, para cercar a Davi e os
seus homens. Sabendo, pois, Davi, que Saul maquinava este mal
contra ele, disse a Abiatar, sacerdote: Traze aqui o éfode. E
disse Davi: Ó Senhor, Deus de Israel, teu servo tem ouvido que Saul
procura vir a Queila, para destruir a cidade por causa de mim.
Entregar-me-ão os cidadãos de Queila na sua mão? Descerá
Saul, como o teu servo tem ouvido? Ah! Senhor Deus de Israel! Faze-o
saber ao teu servo. E disse o Senhor: Descerá. Disse mais
Davi: Entregar-me-ão os cidadãos de Queila, a mim e aos meus homens,
nas mãos de Saul? E disse o Senhor: Entregarão. Então Davi se
levantou com os seus homens, uns seiscentos, e saíram de Queila, e
foram-se aonde puderam; e sendo anunciado a Saul, que Davi escapara
de Queila, cessou de sair contra ele.

E Davi permaneceu no deserto, nos lugares fortes, e ficou em um
monte no deserto de Zife; e Saul o buscava todos os dias, porém Deus
não o entregou na sua mão. Vendo, pois, Davi, que Saul saíra
à busca da sua vida, permaneceu no deserto de Zife, num bosque.
Então se levantou Jônatas, filho de Saul, e foi para Davi no
bosque, e confortou a sua mão em Deus; e disse-lhe: Não
temas, que não te achará a mão de Saul, meu pai; porém tu reinarás
sobre Israel, e eu serei contigo o segundo; o que também Saul, meu
pai, bem sabe. E ambos fizeram aliança perante o Senhor; Davi
ficou no bosque, e Jônatas voltou para a sua casa.

Então subiram os zifeus a Saul, a Gibeá, dizendo: Não se escondeu
Davi entre nós, nos lugares fortes no bosque, no outeiro de Haquilá,
que está à mão direita de Jesimom? Agora, pois, ó rei,
apressadamente desce conforme a todo o desejo da tua alma; a nós
cumpre entregá-lo nas mãos do rei. Então disse Saul: Bendito
sejais vós do Senhor, porque vos compadecestes de mim. Ide,
pois, e diligenciai ainda mais, e sabei e notai o lugar que
freqüenta, e quem o tenha visto ali; porque me foi dito que é
astutíssimo. Por isso atentai bem, e informai-vos acerca de
todos os esconderijos, em que ele se esconde; e então voltai para
mim com toda a certeza, e ir-me-ei convosco; e há de ser que, se
estiver naquela terra, o buscarei entre todos os milhares de Judá.
Então se levantaram eles e se foram a Zife, adiante de Saul;
Davi, porém, e os seus homens estavam no deserto de Maom, na
campina, à direita de Jesimom. E Saul e os seus homens se
foram em busca dele; o que anunciaram a Davi, que desceu para aquela
penha, e ficou no deserto de Maom; o que ouvindo Saul, seguiu a Davi
para o deserto de Maom. E Saul ia deste lado do monte, e Davi
e os seus homens do outro lado do monte; e, temeroso, Davi se
apressou a escapar de Saul; Saul, porém, e os seus homens cercaram a
Davi e aos seus homens, para lançar mão deles. Então veio um
mensageiro a Saul, dizendo: Apressa-te, e vem, porque os filisteus
com ímpeto entraram na terra. Por isso Saul voltou de
perseguir a Davi, e foi ao encontro dos filisteus; por esta razão
aquele lugar se chamou Rochedo das Divisões. E subiu Davi
dali, e ficou nos lugares fortes de En-Gedi.

\medskip

\lettrine{24} E sucedeu que, voltando Saul de perseguir os
filisteus, anunciaram-lhe, dizendo: Eis que Davi está no deserto de
En-Gedi. Então tomou Saul três mil homens, escolhidos dentre
todo o Israel, e foi em busca de Davi e dos seus homens, até sobre
os cumes das penhas das cabras montesas. E chegou a uns currais
de ovelhas no caminho, onde estava uma caverna; e entrou nela Saul,
a cobrir seus pés; e Davi e os seus homens estavam nos fundos da
caverna. Então os homens de Davi lhe disseram: Eis aqui o dia,
do qual o Senhor te diz: Eis que te dou o teu inimigo nas tuas mãos,
e far-lhe-ás como te parecer bem aos teus olhos. E levantou-se Davi,
e mansamente cortou a orla do manto de Saul. Sucedeu, porém, que
depois o coração doeu a Davi, por ter cortado a orla do manto de
Saul. E disse aos seus homens: O Senhor me guarde de que eu faça
tal coisa ao meu senhor, ao ungido do Senhor, estendendo eu a minha
mão contra ele; pois é o ungido do Senhor. E com estas palavras
Davi conteve os seus homens, e não lhes permitiu que se levantassem
contra Saul; e Saul se levantou da caverna, e prosseguiu o seu
caminho. Depois também Davi se levantou, e saiu da caverna, e
gritou por detrás de Saul, dizendo: Rei, meu senhor! E, olhando Saul
para trás, Davi se inclinou com o rosto em terra, e se prostrou.

E disse Davi a Saul: Por que dás tu ouvidos às palavras dos homens
que dizem: Eis que Davi procura o teu mal? Eis que este dia
os teus olhos viram, que o Senhor hoje te pôs em minhas mãos nesta
caverna, e alguns disseram que te matasse; porém a minha mão te
poupou; porque disse: Não estenderei a minha mão contra o meu
senhor, pois é o ungido do Senhor. Olha, pois, meu pai, vê
aqui a orla do teu manto na minha mão; porque cortando-te eu a orla
do manto, não te matei. Sabe, pois, e vê que não há na minha mão nem
mal nem rebeldia alguma, e não pequei contra ti; porém tu andas à
caça da minha vida, para ma tirares. Julgue o Senhor entre
mim e ti, e vingue-me o Senhor de ti; porém a minha mão não será
contra ti. Como diz o provérbio dos antigos: Dos ímpios
procede a impiedade; porém a minha mão não será contra ti.
Após quem saiu o rei de Israel? A quem persegues? A um cão
morto? A uma pulga? O Senhor, porém, será juiz, e julgará
entre mim e ti, e verá, e advogará a minha causa, e me defenderá da
tua mão.

E sucedeu que, acabando Davi de falar a Saul todas estas
palavras, disse Saul: É esta a tua voz, meu filho Davi? Então Saul
levantou a sua voz e chorou. E disse a Davi: Mais justo és do
que eu; pois tu me recompensaste com bem, e eu te recompensei com
mal. E tu mostraste hoje que procedeste bem para comigo, pois
o Senhor me tinha posto em tuas mãos, e tu não me mataste.
Porque, quem há que, encontrando o seu inimigo, o deixaria ir
por bom caminho? O Senhor, pois, te pague com bem, por isso que hoje
me fizeste. Agora, pois, eis que bem sei que certamente hás
de reinar, e que o reino de Israel há de ser firme na tua mão.
Portanto agora jura-me pelo Senhor que não desarraigarás a
minha descendência depois de mim, nem desfarás o meu nome da casa de
meu pai. Então jurou Davi a Saul. E foi Saul para a sua casa;
porém Davi e os seus homens subiram ao lugar forte.

\medskip

\lettrine{25} E faleceu Samuel, e todo o Israel se ajuntou, e
o prantearam, e o sepultaram na sua casa, em Ramá. E Davi se
levantou e desceu ao deserto de Parã.

E havia um homem em Maom, que tinha as suas possessões no Carmelo;
e era este homem muito poderoso, e tinha três mil ovelhas e mil
cabras; e estava tosquiando as suas ovelhas no Carmelo. E era o
nome deste homem Nabal, e o nome de sua mulher Abigail; e era a
mulher de bom entendimento e formosa; porém o homem era duro, e
maligno nas obras, e era da casa de Calebe. E ouviu Davi no
deserto que Nabal tosquiava as suas ovelhas, e enviou Davi dez
moços, e disse aos moços: Subi ao Carmelo, e, indo a Nabal,
perguntai-lhe, em meu nome, como está. E assim direis àquele
próspero: Paz tenhas, e que a tua casa tenha paz, e tudo o que tens
tenha paz! Agora, pois, tenho ouvido que tens tosquiadores. Ora,
os pastores que tens estiveram conosco; agravo nenhum lhes fizemos,
nem coisa alguma lhes faltou todos os dias que estiveram no Carmelo.
Pergunta-o aos teus moços, e eles to dirão. Estes moços, pois,
achem graça em teus olhos, porque viemos em boa ocasião. Dá, pois, a
teus servos e a Davi, teu filho, o que achares à mão. Chegando,
pois, os moços de Davi, e falando a Nabal todas aquelas palavras em
nome de Davi, se calaram. E Nabal respondeu aos criados de
Davi, e disse: Quem é Davi, e quem é o filho de Jessé? Muitos servos
há hoje, que fogem ao seu senhor. Tomaria eu, pois, o meu
pão, e a minha água, e a carne das minhas reses que degolei para os
meus tosquiadores, e o daria a homens que eu não sei donde vêm?

Então os moços de Davi puseram-se a caminho e voltaram, e
chegando, lhe anunciaram tudo conforme a todas estas palavras.
Por isso disse Davi aos seus homens: Cada um cinja a sua
espada. E cada um cingiu a sua espada, e cingiu também Davi a sua; e
subiram após Davi uns quatrocentos homens, e duzentos ficaram com a
bagagem. Porém um dentre os moços o anunciou a Abigail,
mulher de Nabal, dizendo: Eis que Davi enviou mensageiros desde o
deserto a saudar o nosso amo; porém ele os destratou.
Todavia, aqueles homens têm-nos sido muito bons, e nunca
fomos agravados por eles, e nada nos faltou em todos os dias que
convivemos com eles quando estavam no campo. De muro em redor
nos serviram, assim de dia como de noite, todos os dias que andamos
com eles apascentando as ovelhas. Considera, pois, agora, e
vê o que hás de fazer, porque o mal já está de todo determinado
contra o nosso amo e contra toda a sua casa, e ele é um homem vil,
que não há quem lhe possa falar.

Então Abigail se apressou, e tomou duzentos pães, e dois odres de
vinho, e cinco ovelhas guisadas, e cinco medidas de trigo tostado, e
cem cachos de passas, e duzentas pastas de figos passados, e os pôs
sobre jumentos. E disse aos seus moços: Ide adiante de mim,
eis que vos seguirei de perto. O que, porém, não declarou a seu
marido Nabal. E sucedeu que, andando ela montada num jumento,
desceu pelo encoberto do monte, e eis que Davi e os seus homens lhe
vinham ao encontro, e ela encontrou-se com eles. E disse
Davi: Na verdade que em vão tenho guardado tudo quanto este tem no
deserto, e nada lhe faltou de tudo quanto tem, e ele me pagou mal
por bem. Assim faça Deus aos inimigos de Davi, e outro tanto,
se eu deixar até amanhã de tudo o que tem, até mesmo um menino.
Vendo, pois, Abigail a Davi, apressou-se, e desceu do
jumento, e prostrou-se sobre o seu rosto diante de Davi, e se
inclinou à terra. E lançou-se a seus pés, e disse: Ah, senhor
meu, minha seja a transgressão; deixa, pois, falar a tua serva aos
teus ouvidos, e ouve as palavras da tua serva. Meu senhor,
agora não faça este homem vil, a saber, Nabal, impressão no seu
coração, porque tal é ele qual é o seu nome. Nabal é o seu nome, e a
loucura está com ele, e eu, tua serva, não vi os moços de meu
senhor, que enviaste. Agora, pois, meu senhor, vive o Senhor,
e vive a tua alma, que o Senhor te impediu de vires com sangue, e de
que a tua mão te salvasse; e, agora, tais quais Nabal sejam os teus
inimigos e os que procuram mal contra o meu senhor. E agora
este é o presente que trouxe a tua serva a meu senhor; seja dado aos
moços que seguem ao meu senhor. Perdoa, pois, à tua serva
esta transgressão, porque certamente fará o Senhor casa firme a meu
senhor, porque meu senhor guerreia as guerras do Senhor, e não se
tem achado mal em ti por todos os teus dias, e, levantando-se
algum homem para te perseguir, e para procurar a tua morte, contudo
a vida de meu senhor será atada no feixe dos que vivem com o Senhor
teu Deus; porém a vida de teus inimigos ele arrojará ao longe, como
do meio do côncavo de uma funda. E há de ser que, usando o
Senhor com o meu senhor conforme a todo o bem que já tem falado de
ti, e te houver estabelecido príncipe sobre Israel, então,
meu senhor, não te será por tropeço, nem por pesar no coração, o
sangue que sem causa derramaste, nem tampouco por ter se vingado o
meu senhor a si mesmo; e quando o Senhor fizer bem a meu senhor,
lembra-te então da tua serva.

Então Davi disse a Abigail: Bendito o Senhor Deus de Israel, que
hoje te enviou ao meu encontro. E bendito o teu conselho, e
bendita tu, que hoje me impediste de derramar sangue, e de vingar-me
pela minha própria mão. Porque, na verdade, vive o Senhor
Deus de Israel, que me impediu de que te fizesse mal, que se tu não
te apressaras, e não me vieras ao encontro, não ficaria a Nabal até
a luz da manhã nem mesmo um menino. Então Davi tomou da sua
mão o que tinha trazido, e lhe disse: Sobe em paz à tua casa; vês
aqui que tenho dado ouvidos à tua voz, e tenho aceitado a tua face.

E, vindo Abigail a Nabal, eis que tinha em sua casa um banquete,
como banquete de rei; e o coração de Nabal estava alegre nele, e ele
já muito embriagado, pelo que ela não lhe deu a entender coisa
alguma, pequena nem grande, até à luz da manhã. Sucedeu,
pois, que pela manhã, estando Nabal já livre do vinho, sua mulher
lhe deu a entender aquelas coisas; e se amorteceu o seu coração, e
ficou ele como pedra. E aconteceu que, passados quase dez
dias, feriu o Senhor a Nabal, e este morreu. E, ouvindo Davi
que Nabal morrera, disse: Bendito seja o Senhor, que julgou a causa
de minha afronta recebida da mão de Nabal, e deteve a seu servo do
mal, fazendo o Senhor tornar o mal de Nabal sobre a sua cabeça. E
mandou Davi falar a Abigail, para tomá-la por sua mulher.
Vindo, pois, os criados de Davi a Abigail, no Carmelo, lhe
falaram, dizendo: Davi nos tem mandado a ti, para te tomar por sua
mulher. Então ela se levantou, e se inclinou com o rosto em
terra, e disse: Eis que a tua serva servirá de criada para lavar os
pés dos criados de meu senhor. E Abigail se apressou, e se
levantou, e montou num jumento com as suas cinco moças que seguiam
as suas pisadas; e ela seguiu os mensageiros de Davi, e foi sua
mulher. Também tomou Davi a Ainoã de Jizreel; e ambas foram
suas mulheres. Porque Saul tinha dado sua filha Mical, mulher
de Davi, a Palti, filho de Laís, o qual era de Galim.

\medskip

\lettrine{26} E vieram os zifeus a Saul, a Gibeá, dizendo: Não
está Davi escondido no outeiro de Haquilá, defronte de Jesimom?
Então Saul se levantou e desceu ao deserto de Zife, e com ele
três mil homens escolhidos de Israel, a buscar a Davi no deserto de
Zife. E acampou-se Saul no outeiro de Haquilá, que está defronte
de Jesimom, junto ao caminho; porém Davi ficou no deserto, e viu que
Saul vinha seguindo-o no deserto. Pois Davi enviou espias, e
soube que Saul tinha vindo. E Davi se levantou, e foi ao lugar
onde Saul se tinha acampado; viu Davi o lugar onde se tinha deitado
Saul, e Abner, filho de Ner, capitão do seu exército; e Saul estava
deitado dentro do lugar dos carros, e o povo estava acampado ao
redor dele.

E dirigindo-se Davi a Aimeleque, o heteu, e a Abisai, filho de
Zeruia, irmão de Joabe, disse: Quem descerá comigo a Saul ao
arraial? E respondeu Abisai: Eu descerei contigo. Foram, pois,
Davi e Abisai de noite ao povo, e eis que Saul estava deitado
dormindo dentro do lugar dos carros, e a sua lança estava fincada na
terra à sua cabeceira; e Abner e o povo deitavam-se ao redor dele.
Então disse Abisai a Davi: Deus te entregou hoje nas mãos o teu
inimigo; deixa-me, pois, agora encravá-lo com a lança de uma vez na
terra, e não o ferirei segunda vez. E disse Davi a Abisai:
Nenhum dano lhe faças; porque quem estendeu a sua mão contra o
ungido do Senhor, e ficou inocente? Disse mais Davi: Vive o
Senhor que o Senhor o ferirá, ou o seu dia chegará em que morra, ou
descerá para a batalha e perecerá. O Senhor me guarde, de que
eu estenda a mão contra o ungido do Senhor; agora, porém, toma a
lança que está à sua cabeceira e a bilha de água, e vamo-nos.
Tomou, pois, Davi a lança e a bilha de água, da cabeceira de
Saul, e foram-se; e ninguém houve que o visse, nem que o advertisse,
nem que acordasse; porque todos estavam dormindo, porque da parte do
Senhor havia caído sobre eles um profundo sono.

E Davi, passando ao outro lado, pôs-se no cume do monte ao longe,
de maneira que entre eles havia grande distância. E Davi
bradou ao povo, e a Abner, filho de Ner, dizendo: Não responderás,
Abner? Então Abner respondeu e disse: Quem és tu, que bradas ao rei?
Então disse Davi a Abner: Porventura não és homem? E quem há
em Israel como tu? Por que, pois, não guardaste o rei, teu senhor?
Porque um do povo veio para destruir o rei, teu senhor. Não é
bom isso, que fizeste; vive o Senhor, que sois dignos de morte, vós
que não guardastes a vosso senhor, o ungido do Senhor; vede, pois,
agora onde está a lança do rei, e a bilha de água, que tinha à sua
cabeceira. Então conheceu Saul a voz de Davi, e disse: Não é
esta a tua voz, meu filho Davi? E disse Davi: É minha voz, ó rei meu
Senhor. Disse mais: Por que persegue o meu Senhor tanto o seu
servo? Que fiz eu? E que maldade se acha nas minhas mãos?
Ouve, pois, agora, te rogo, rei meu Senhor, as palavras de
teu servo: Se o Senhor te incita contra mim, receba ele a oferta de
alimentos; se, porém, são os filhos dos homens, malditos sejam
perante o Senhor; pois eles me têm expulsado hoje para que eu não
tenha parte na herança do Senhor, dizendo: Vai, serve a outros
deuses. Agora, pois, não se derrame o meu sangue na terra
diante do Senhor; pois saiu o rei de Israel em busca de uma pulga,
como quem persegue uma perdiz nos montes.

Então disse Saul: Pequei; volta, meu filho Davi, porque não
tornarei a fazer-te mal; porque foi hoje preciosa a minha vida aos
teus olhos; eis que procedi loucamente, e errei grandissimamente.
Davi então respondeu, e disse: Eis aqui a lança do rei; venha
cá um dos moços, e leve-a. O Senhor, porém, pague a cada um a
sua justiça e a sua lealdade; pois o Senhor te entregou hoje na
minha mão, porém não quis estender a minha mão contra o ungido do
Senhor. E eis que, assim como foi a tua vida hoje de tanta
estima aos meus olhos, assim seja a minha vida de muita estima aos
olhos do Senhor, e ele me livre de toda a tribulação. Então
Saul disse a Davi: Bendito sejas tu, meu filho Davi; pois grandes
coisas farás e também prevalecerás. Então Davi se foi pelo seu
caminho e Saul voltou para o seu lugar.

\medskip

\lettrine{27} Disse, porém, Davi no seu coração: Ora, algum
dia ainda perecerei pela mão de Saul; não há coisa melhor para mim
do que escapar apressadamente para a terra dos filisteus, para que
Saul perca a esperança de mim, e cesse de me buscar por todos os
termos de Israel; e assim escaparei da sua mão. Então Davi se
levantou, e passou, com os seiscentos homens que com ele estavam, a
Aquis, filho de Maoque, rei de Gate. E Davi ficou com Aquis em
Gate, ele e os seus homens, cada um com a sua casa; Davi com ambas
as suas mulheres, Ainoã, a jizreelita, e Abigail, a mulher de Nabal,
o carmelita. E, sendo Saul avisado que Davi tinha fugido para
Gate, não cuidou mais de buscá-lo. E disse Davi a Aquis: Se eu
tenho achado graça em teus olhos, dá-me lugar numa das cidades da
terra, para que ali habite; pois por que razão habitaria o teu servo
contigo na cidade real? Então lhe deu Aquis, naquele dia, a
cidade de Ziclague (por isso Ziclague pertence aos reis de Judá, até
ao dia de hoje). E foi o número dos dias, que Davi habitou na
terra dos filisteus, um ano e quatro meses.

E subia Davi com os seus homens, e davam sobre os gesuritas, e os
gersitas, e os amalequitas; porque antigamente foram estes os
moradores da terra que se estende na direção de Sur, até à terra do
Egito. E Davi feria aquela terra, e não dava vida nem a homem
nem a mulher, e tomava ovelhas, e vacas, e jumentos, e camelos, e
vestes; e voltava, e vinha a Aquis. E dizendo Aquis: Onde
atacastes hoje? Davi dizia: Sobre o sul de Judá, e sobre o sul dos
jerameelitas, e sobre o sul dos queneus. E Davi não deixava
com vida nem a homem nem a mulher, para trazê-los a Gate, dizendo:
Para que porventura não nos denunciem, dizendo: Assim Davi o fazia.
E este era o seu costume por todos os dias que habitou na terra dos
filisteus. E Aquis confiava em Davi, dizendo: Fez-se ele por
certo aborrecível para com o seu povo em Israel; por isso me será
por servo para sempre.

\medskip

\lettrine{28} E sucedeu naqueles dias que, juntando os
filisteus os seus exércitos à peleja, para fazer guerra contra
Israel, disse Aquis a Davi: Sabe de certo que comigo sairás ao
arraial, tu e os teus homens. Então disse Davi a Aquis: Assim
saberás o que fará o teu servo. E disse Aquis a Davi: Por isso te
terei por guarda da minha pessoa para sempre. E Samuel já estava
morto, e todo o Israel o tinha chorado, e o tinha sepultado em Ramá,
que era a sua cidade; e Saul tinha desterrado os adivinhos e os
encantadores. E ajuntaram-se os filisteus, e vieram, e
acamparam-se em Suném; e ajuntou Saul a todo o Israel, e se
acamparam em Gilboa. E, vendo Saul o arraial dos filisteus,
temeu, e estremeceu muito o seu coração. E perguntou Saul ao
Senhor, porém o Senhor não lhe respondeu, nem por sonhos, nem por
Urim, nem por profetas.

Então disse Saul aos seus criados: Buscai-me uma mulher que tenha
o espírito de feiticeira, para que vá a ela, e consulte por ela. E
os seus criados lhe disseram: Eis que em En-Dor há uma mulher que
tem o espírito de adivinhar. E Saul se disfarçou, e vestiu
outras roupas, e foi ele com dois homens, e de noite chegaram à
mulher; e disse: Peço-te que me adivinhes pelo espírito de
feiticeira, e me faças subir a quem eu te disser. Então a mulher
lhe disse: Eis aqui tu sabes o que Saul fez, como tem destruído da
terra os adivinhos e os encantadores; por que, pois, me armas um
laço à minha vida, para me fazeres morrer? Então Saul lhe
jurou pelo Senhor, dizendo: Vive o Senhor, que nenhum mal te
sobrevirá por isso. A mulher então lhe disse: A quem te farei
subir? E disse ele: Faze-me subir a Samuel. Vendo, pois, a
mulher a Samuel, gritou com alta voz, e falou a Saul, dizendo: Por
que me tens enganado? Pois tu mesmo és Saul. E o rei lhe
disse: Não temas; que é que vês? Então a mulher disse a Saul: Vejo
deuses que sobem da terra. E lhe disse: Como é a sua figura?
E disse ela: Vem subindo um homem ancião, e está envolto numa capa.
Entendendo Saul que era Samuel, inclinou-se com o rosto em terra, e
se prostrou.

Samuel disse a Saul: Por que me inquietaste, fazendo-me subir?
Então disse Saul: Mui angustiado estou, porque os filisteus
guerreiam contra mim, e Deus se tem desviado de mim, e não me
responde mais, nem pelo ministério dos profetas, nem por sonhos; por
isso te chamei a ti, para que me faças saber o que hei de fazer.
Então disse Samuel: Por que, pois, me perguntas a mim, visto
que o Senhor te tem desamparado, e se tem feito teu inimigo?
Porque o Senhor tem feito para contigo como pela minha boca
te disse, e o Senhor tem rasgado o reino da tua mão, e o tem dado ao
teu próximo, a Davi. Como tu não deste ouvidos à voz do
Senhor, e não executaste o fervor da sua ira contra Amaleque, por
isso o Senhor te fez hoje isto. E o Senhor entregará também a
Israel contigo na mão dos filisteus, e amanhã tu e teus filhos
estareis comigo; e o arraial de Israel o Senhor entregará na mão dos
filisteus.

E imediatamente Saul caiu estendido por terra, e grandemente
temeu por causa daquelas palavras de Samuel; e não houve força nele;
porque não tinha comido pão todo aquele dia e toda aquela noite.
Então veio a mulher a Saul e, vendo que estava tão
perturbado, disse-lhe: Eis que a tua criada deu ouvidos à tua voz, e
pus a minha vida na minha mão, e ouvi as palavras que disseste.
Agora, pois, ouve também tu as palavras da tua serva, e porei
um bocado de pão diante de ti, e come, para que tenhas forças para
te pores a caminho. Porém ele o recusou, e disse: Não
comerei. Porém os seus criados e a mulher o constrangeram; e deu
ouvidos à sua voz; e levantou-se do chão, e se assentou sobre uma
cama. E tinha a mulher em casa um bezerro cevado, e se
apressou, e o matou, e tomou farinha, e a amassou, e a cozeu em
bolos ázimos. E os trouxe diante de Saul e de seus criados, e
comeram; depois levantaram-se e partiram naquela mesma noite.

\medskip

\lettrine{29} E ajuntaram os filisteus todos os seus exércitos
em Afeque; e acamparam-se os israelitas junto à fonte que está em
Jizreel. E os príncipes dos filisteus se foram para lá com
centenas e com milhares; porém Davi e os seus homens iam com Aquis
na retaguarda. Disseram então os príncipes dos filisteus: Que
fazem aqui estes hebreus? E disse Aquis aos príncipes dos filisteus:
Não é este Davi, o servo de Saul, rei de Israel, que esteve comigo
há alguns dias ou anos? Coisa nenhuma achei nele desde o dia em que
se revoltou, até ao dia de hoje. Porém os príncipes dos
filisteus muito se indignaram contra ele; e disseram-lhe os
príncipes dos filisteus: Faze voltar este homem, para que torne ao
lugar em que tu o puseste, e não desça conosco à batalha, para que
não se torne nosso adversário na batalha; pois, com que poderia este
agradar a seu senhor? Porventura não seria com as cabeças destes
homens? Não é este aquele Davi, de quem uns aos outros cantaram
nas danças, dizendo: Saul feriu os seus milhares, porém Davi os seus
dez milhares?

Então Aquis chamou a Davi e disse-lhe: Vive o Senhor, que tu és
reto, e que a tua entrada e a tua saída comigo no arraial é boa aos
meus olhos; porque nenhum mal em ti achei, desde o dia em que a mim
vieste, até ao dia de hoje; porém aos olhos dos príncipes não
agradas. Volta, pois, agora, e vai em paz; para que não faças
mal aos olhos dos príncipes dos filisteus. Então Davi disse a
Aquis: Por quê? Que fiz? Ou que achaste no teu servo, desde o dia em
que estive diante de ti, até ao dia de hoje, para que não vá e
peleje contra os inimigos do rei, meu senhor? Respondeu, porém,
Aquis, e disse a Davi: Bem o sei; e que na verdade aos meus olhos és
bom como um anjo de Deus; porém disseram os príncipes dos filisteus:
Não suba este conosco à batalha. Agora, pois, amanhã de
madrugada levanta-te com os servos de teu senhor, que têm vindo
contigo; e, levantando-vos pela manhã, de madrugada, e havendo luz,
parti. Então Davi de madrugada se levantou, ele e os seus
homens, para partirem pela manhã, e voltarem à terra dos filisteus;
e os filisteus subiram a Jizreel.

\medskip

\lettrine{30} Sucedeu, pois, que, chegando Davi e os seus
homens ao terceiro dia a Ziclague, já os amalequitas tinham invadido
o sul, e Ziclague, e tinham ferido a Ziclague e a tinham queimado a
fogo. E tinham levado cativas as mulheres, e todos os que
estavam nela, tanto pequenos como grandes; a ninguém, porém,
mataram, tão-somente os levaram consigo, e foram o seu caminho.
E Davi e os seus homens chegaram à cidade e eis que estava
queimada a fogo, e suas mulheres, seus filhos e suas filhas tinham
sido levados cativos. Então Davi e o povo que se achava com ele
alçaram a sua voz, e choraram, até que neles não houve mais forças
para chorar. Também as duas mulheres de Davi foram levadas
cativas; Ainoã, a jizreelita, e Abigail, a mulher de Nabal, o
carmelita. E Davi muito se angustiou, porque o povo falava de
apedrejá-lo, porque a alma de todo o povo estava em amargura, cada
um por causa dos seus filhos e das suas filhas; todavia Davi se
fortaleceu no Senhor seu Deus.

E disse Davi a Abiatar, o sacerdote, filho de Aimeleque: Traze-me,
peço-te, aqui o éfode. E Abiatar trouxe o éfode a Davi. Então
consultou Davi ao Senhor, dizendo: Perseguirei eu a esta tropa?
Alcançá-la-ei? E lhe disse: Persegue-a, porque decerto a alcançarás
e tudo libertarás. Partiu, pois, Davi, ele e os seiscentos
homens que com ele se achavam, e chegaram ao ribeiro de Besor, onde
pararam os que ficaram atrás. E perseguiu-os Davi, ele e os
quatrocentos homens, pois que duzentos homens ficaram, por não
poderem, de cansados que estavam, passar o ribeiro de Besor.
E acharam no campo um homem egípcio, e o trouxeram a Davi;
deram-lhe pão, e comeu, e deram-lhe a beber água. Deram-lhe
também um pedaço de massa de figos secos e dois cachos de passas, e
comeu, e voltou-lhe o seu espírito, porque havia três dias e três
noites que não tinha comido pão nem bebido água. Então Davi
lhe disse: De quem és tu, e de onde és? E disse o moço egípcio: Sou
servo de um homem amalequita, e meu senhor me deixou, porque adoeci
há três dias. Nós invadimos o lado do sul dos queretitas, e o
lado de Judá, e o lado do sul de Calebe, e pusemos fogo a Ziclague.
E disse-lhe Davi: Poderias, descendo, guiar-me a essa tropa?
E disse-lhe: Por Deus jura-me que não me matarás, nem me entregarás
na mão de meu senhor, e, descendo, te guiarei a essa tropa.
E, descendo, o guiou e eis que estavam espalhados sobre a
face de toda a terra, comendo, e bebendo, e dançando, por todo
aquele grande despojo que tomaram da terra dos filisteus e da terra
de Judá. E feriu-os Davi, desde o crepúsculo até à tarde do
dia seguinte; nenhum deles escapou, senão só quatrocentos moços que,
montados sobre camelos, fugiram. Assim salvou Davi tudo
quanto tomaram os amalequitas; também as suas duas mulheres salvou
Davi. E ninguém lhes faltou, desde o menor até ao maior, e
até os filhos e as filhas; e também desde o despojo até tudo quanto
lhes tinham tomado, tudo Davi tornou a trazer. Também tomou
Davi todas as ovelhas e vacas, e levavam-nas adiante do outro gado,
e diziam: Este é o despojo de Davi.

E, chegando Davi aos duzentos homens que, de cansados que
estavam, não puderam seguir a Davi, e que deixaram ficar no ribeiro
de Besor, estes saíram ao encontro de Davi e do povo que com ele
vinha; e, chegando-se Davi com o povo, os saudou em paz.
Então todos os maus e perversos, dentre os homens que tinham
ido com Davi, responderam, e disseram: Visto que não foram conosco,
não lhes daremos do despojo que libertamos; mas que leve cada um sua
mulher e seus filhos, e se vá. Porém Davi disse: Não fareis
assim, irmãos meus, com o que nos deu o Senhor, que nos guardou, e
entregou a tropa que contra nós vinha, nas nossas mãos. E
quem vos daria ouvidos nisso? Porque qual é a parte dos que desceram
à peleja, tal também será a parte dos que ficaram com a bagagem;
igualmente repartirão. O que assim foi desde aquele dia em
diante, porquanto o pôs por estatuto e direito em Israel até ao dia
de hoje. E, chegando Davi a Ziclague, enviou do despojo aos
anciãos de Judá, seus amigos, dizendo: Eis aí para vós uma bênção do
despojo dos inimigos do Senhor; aos de Betel, e aos de Ramote
do sul, e aos de Jater, e aos de Aroer, e aos de Sifmote, e
aos de Estemoa, e aos de Racal, e aos que estavam nas cidades
jerameelitas e nas cidades dos queneus, e aos de Hormá, e aos
de Corasã, e aos de Ataca, e aos de Hebrom, e a todos os
lugares em que andara Davi, ele e os seus homens.

\medskip

\lettrine{31} Os filisteus, pois, pelejaram contra Israel; e
os homens de Israel fugiram de diante dos filisteus, e caíram mortos
na montanha de Gilboa. E os filisteus perseguiram a Saul e a
seus filhos; e mataram a Jônatas, e a Abinadabe, e a Malquisua,
filhos de Saul. E a peleja se agravou contra Saul, e os
flecheiros o alcançaram; e muito temeu por causa dos flecheiros.
Então disse Saul ao seu pajem de armas: Arranca a tua espada, e
atravessa-me com ela, para que porventura não venham estes
incircuncisos, e me atravessem e escarneçam de mim. Porém o seu
pajem de armas não quis, porque temia muito; então Saul tomou a
espada, e se lançou sobre ela. Vendo, pois, o seu pajem de armas
que Saul já era morto, também ele se lançou sobre a sua espada, e
morreu com ele. Assim faleceu Saul, e seus três filhos, e o seu
pajem de armas, e também todos os seus homens morreram juntamente
naquele dia. E, vendo os homens de Israel, que estavam deste
lado do vale e deste lado do Jordão, que os homens de Israel
fugiram, e que Saul e seus filhos estavam mortos, abandonaram as
cidades, e fugiram; e vieram os filisteus, e habitaram nelas.

Sucedeu, pois, que, vindo os filisteus no outro dia para despojar
os mortos, acharam a Saul e a seus três filhos estirados na montanha
de Gilboa. E cortaram-lhe a cabeça, e o despojaram das suas
armas, e enviaram pela terra dos filisteus, em redor, a anunciá-lo
no templo dos seus ídolos e entre o povo. E puseram as suas
armas no templo de Astarote, e o seu corpo o afixaram no muro de
Bete-Seã. Ouvindo então os moradores de Jabes-Gileade, o que
os filisteus fizeram a Saul, todo o homem valoroso se
levantou, e caminharam toda a noite, e tiraram o corpo de Saul e os
corpos de seus filhos do muro, de Bete-Seã, e, vindo a Jabes, os
queimaram. E tomaram os seus ossos, e os sepultaram debaixo
de um arvoredo, em Jabes, e jejuaram sete dias.

