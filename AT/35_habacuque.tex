\addchap{Habacuque}

\lettrine{1} O peso que viu o profeta Habacuque. Até
quando, Senhor, clamarei eu, e tu não me escutarás? Gritar-te-ei:
Violência! e não salvarás? Por que razão me mostras a
iniqüidade, e me fazes ver a opressão? Pois que a destruição e a
violência estão diante de mim, havendo também quem suscite a
contenda e o litígio. Por esta causa a lei se afrouxa, e a
justiça nunca se manifesta; porque o ímpio cerca o justo, e a
justiça se manifesta distorcida.

Vede entre os gentios e olhai, e maravilhai-vos, e admirai-vos;
porque realizarei em vossos dias uma obra que vós não crereis,
quando for contada. Porque eis que suscito os caldeus, nação
amarga e impetuosa, que marcha sobre a largura da terra, para
apoderar-se de moradas que não são suas. Horrível e terrível é;
dela mesma sairá o seu juízo e a sua dignidade. E os seus
cavalos são mais ligeiros do que os leopardos, e mais espertos do
que os lobos à tarde; os seus cavaleiros espalham-se por toda parte;
os seus cavaleiros virão de longe; voarão como águias que se
apressam a devorar. Eles todos virão para fazer violência; os
seus rostos buscarão o vento oriental, e reunirão os cativos como
areia. E escarnecerão dos reis, e dos príncipes farão
zombaria; eles se rirão de todas as fortalezas, porque amontoarão
terra, e as tomarão. Então muda a sua mente, e seguirá, e se
fará culpado, atribuindo este seu poder ao seu deus.

Não és tu desde a eternidade, ó Senhor meu Deus, meu Santo? Nós
não morreremos. Ó Senhor, para juízo o puseste, e tu, ó Rocha, o
fundaste para castigar. Tu és tão puro de olhos, que não
podes ver o mal, e a opressão não podes contemplar. Por que olhas
para os que procedem aleivosamente, e te calas quando o ímpio devora
aquele que é mais justo do que ele? E por que farias os
homens como os peixes do mar, como os répteis, que não têm quem os
governe? Ele a todos levantará com o anzol, apanhá-los-á com
a sua rede, e os ajuntará na sua rede varredoura; por isso ele se
alegrará e se regozijará. Por isso sacrificará à sua rede, e
queimará incenso à sua varredoura; porque com elas engordou a sua
porção, e engrossou a sua comida. Porventura por isso
esvaziará a sua rede e não terá piedade de matar as nações
continuamente?

\medskip

\lettrine{2} Sobre a minha guarda estarei, e sobre a fortaleza
me apresentarei e vigiarei, para ver o que falará a mim, e o que eu
responderei quando eu for argüido. Então o Senhor me respondeu,
e disse: Escreve a visão e torna bem legível sobre tábuas, para que
a possa ler quem passa correndo. Porque a visão é ainda para o
tempo determinado, mas se apressa para o fim, e não enganará; se
tardar, espera-o, porque certamente virá, não tardará. Eis que a
sua alma está orgulhosa, não é reta nele; mas o justo pela sua fé
viverá.

Tanto mais que, por ser dado ao vinho é desleal; homem soberbo que
não permanecerá; que alarga como o inferno a sua alma; e é como a
morte que não se farta, e ajunta a si todas as nações, e congrega a
si todos os povos. Não levantarão, pois, todos estes contra ele
uma parábola e um provérbio sarcástico contra ele? E se dirá: Ai
daquele que multiplica o que não é seu! (até quando?) e daquele que
carrega sobre si dívidas! Porventura não se levantarão de
repente os teus extorquiadores, e não despertarão os que te farão
tremer, e não lhes servirás tu de despojo? Porquanto despojaste
a muitas nações, todos os demais povos te despojarão a ti, por causa
do sangue dos homens, e da violência feita à terra, à cidade, e a
todos os que nela habitam. Ai daquele que, para a sua casa,
ajunta cobiçosamente bens mal adquiridos, para pôr o seu ninho no
alto, a fim de se livrar do poder do mal! Vergonha maquinaste
para a tua casa; destruindo tu a muitos povos, pecaste contra a tua
alma. Porque a pedra clamará da parede, e a trave lhe
responderá do madeiramento. Ai daquele que edifica a cidade
com sangue, e que funda a cidade com iniqüidade! Porventura
não vem do Senhor dos Exércitos que os povos trabalhem pelo fogo e
os homens se cansem em vão? Porque a terra se encherá do
conhecimento da glória do Senhor, como as águas cobrem o mar.

Ai daquele que dá de beber ao seu companheiro! Ai de ti, que
adiciona à bebida o teu furor, e o embebedas para ver a sua nudez!
Serás farto de ignomínia em lugar de honra; bebe tu também, e
sê como um incircunciso; o cálice da mão direita do Senhor voltará a
ti, e ignomínia cairá sobre a tua glória. Porque a violência
cometida contra o Líbano te cobrirá, e a destruição das feras te
amedrontará, por causa do sangue dos homens, e da violência feita à
terra, à cidade, e a todos os que nela habitam. Que aproveita
a imagem de escultura, depois que a esculpiu o seu artífice? Ela é
máscara e ensina mentira, para que quem a formou confie na sua obra,
fazendo ídolos mudos? Ai daquele que diz ao pau: Acorda! e à
pedra muda: Desperta! Pode isso ensinar? Eis que está coberta de
ouro e de prata, mas dentro dela não há espírito algum. Mas o
Senhor está no seu santo templo; cale-se diante dele toda a terra.

\medskip

\lettrine{3} Oração do profeta Habacuque sobre Sigionote.
Ouvi, Senhor, a tua palavra, e temi; aviva, ó Senhor, a tua obra
no meio dos anos, no meio dos anos faze-a conhecida; na tua ira
lembra-te da misericórdia.

Deus veio de Temã, e do monte de Parã o Santo (Selá). A sua glória
cobriu os céus, e a terra encheu-se do seu louvor. E o
resplendor se fez como a luz, raios brilhantes saíam da sua mão, e
ali estava o esconderijo da sua força. Adiante dele ia a peste,
e brasas ardentes saíam dos seus passos. Parou, e mediu a terra;
olhou, e separou as nações; e os montes perpétuos foram esmiuçados;
ou outeiros eternos se abateram, porque os caminhos eternos lhe
pertencem. Vi as tendas de Cusã em aflição; tremiam as cortinas
da terra de Midiã. Acaso é contra os rios, Senhor, que estás
irado? É contra os ribeiros a tua ira, ou contra o mar o teu furor,
visto que andas montado sobre os teus cavalos, e nos teus carros de
salvação? Descoberto se movimentou o teu arco; os juramentos
feitos às tribos foram uma palavra segura. (Selá.) Tu fendeste a
terra com rios. Os montes te viram, e tremeram; a inundação
das águas passou; o abismo deu a sua voz, levantou ao alto as suas
mãos. O sol e a lua pararam nas suas moradas; andaram à luz
das tuas flechas, ao resplendor do relâmpago da tua lança.
Com indignação marchaste pela terra, com ira trilhaste os
gentios. Tu saíste para salvação do teu povo, para salvação
do teu ungido; tu feriste a cabeça da casa do ímpio, descobrindo o
alicerce até ao pescoço. (Selá.) Tu traspassaste com as suas
próprias lanças a cabeça das suas vilas; eles me acometeram
tempestuosos para me espalharem; alegravam-se, como se estivessem
para devorar o pobre em segredo. Tu com os teus cavalos
marchaste pelo mar, pela massa de grandes águas.

Ouvindo-o eu, o meu ventre se comoveu, à sua voz tremeram os meus
lábios; entrou a podridão nos meus ossos, e estremeci dentro de mim;
no dia da angústia descansarei, quando subir contra o povo que
invadirá com suas tropas. \textbf{Porque ainda que a figueira
não floresça, nem haja fruto na vide; ainda que decepcione o produto
da oliveira, e os campos não produzam mantimento; ainda que as
ovelhas da malhada sejam arrebatadas, e nos currais não haja gado};
\textbf{todavia eu me alegrarei no Senhor; exultarei no Deus
da minha salvação}. O Senhor Deus é a minha força, e fará os
meus pés como os das cervas, e me fará andar sobre as minhas
alturas. (Para o cantor-mor sobre os meus instrumentos de corda).

