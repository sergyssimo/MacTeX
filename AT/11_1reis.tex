\addchap{Primeiro livro de Reis}

\lettrine{1} Sendo, pois, o rei Davi já velho, e entrado em
dias, cobriam-no de roupas, porém não se aquecia. Então
disseram-lhe os seus servos: Busquem para o rei meu senhor uma moça
virgem, que esteja perante o rei, e tenha cuidado dele; e durma no
seu seio, para que o rei meu senhor se aqueça. E buscaram por
todos os termos de Israel uma moça formosa, e acharam a Abisague,
sunamita; e a trouxeram ao rei. E era a moça sobremaneira
formosa; e tinha cuidado do rei, e o servia; porém o rei não a
conheceu.

Então Adonias, filho de Hagite, se levantou, dizendo: Eu reinarei.
E preparou carros, e cavaleiros, e cinqüenta homens, que corressem
adiante dele. E nunca seu pai o tinha contrariado, dizendo: Por
que fizeste assim? E era ele também muito formoso de parecer; e
Hagite o tivera depois de Absalão. E tinha entendimento com
Joabe, filho de Zeruia, e com Abiatar o sacerdote; os quais o
ajudavam, seguindo a Adonias. Porém Zadoque, o sacerdote, e
Benaia, filho de Joiada, e Natã, o profeta, e Simei, e Rei, e os
poderosos que Davi tinha, não estavam com Adonias. E matou
Adonias ovelhas, e vacas, e animais cevados, junto à pedra de
Zoelete, que está perto da fonte de Rogel; e convidou a todos os
seus irmãos, os filhos do rei, e a todos os homens de Judá, servos
do rei. Porém a Natã, o profeta, e a Benaia, e aos poderosos,
e a Salomão, seu irmão, não convidou.

Então falou Natã a Bate-Seba, mãe de Salomão, dizendo: Não
ouviste que Adonias, filho de Hagite, reina? E que nosso senhor Davi
não o sabe? Vem, pois, agora, e deixa-me dar-te um conselho,
para que salves a tua vida, e a de Salomão teu filho. Vai, e
chega ao rei Davi, e dize-lhe: Não juraste tu, rei senhor meu, à tua
serva, dizendo: Certamente teu filho Salomão reinará depois de mim,
e ele se assentará no meu trono? Por que, pois, reina Adonias?
Eis que, estando tu ainda aí falando com o rei, eu também
entrarei depois de ti, e confirmarei as tuas palavras. E foi
Bate-Seba ao rei na sua câmara; e o rei era muito velho; e Abisague,
a sunamita, servia ao rei. E Bate-Seba inclinou a cabeça, e
se prostrou perante o rei; e disse o rei: Que tens? E ela lhe
disse: Senhor meu, tu juraste à tua serva pelo Senhor teu Deus,
dizendo: Salomão, teu filho, reinará depois de mim, e ele se
assentará no meu trono. E agora eis que Adonias reina; e tu,
ó rei meu senhor, não o sabes. E matou vacas, e animais
cevados, e ovelhas em abundância, e convidou a todos os filhos do
rei, e a Abiatar, o sacerdote, e a Joabe, capitão do exército, mas a
teu servo Salomão não convidou. Porém, ó rei meu senhor, os
olhos de todo o Israel estão sobre ti, para que lhe declares quem se
assentará sobre o trono do rei meu senhor, depois dele. De
outro modo sucederá que, quando o rei meu senhor dormir com seus
pais, eu e Salomão meu filho seremos os culpados. E, estando
ela ainda falando com o rei, eis que entra o profeta Natã. E
o fizeram saber ao rei, dizendo: Eis aí está o profeta Natã. E
entrou à presença do rei, e prostrou-se diante dele com o rosto em
terra. E disse Natã: Ó rei meu senhor, disseste tu: Adonias
reinará depois de mim, e ele se assentará sobre o meu trono?
Porque hoje desceu, e matou vacas, e animais cevados, e
ovelhas em abundância, e convidou a todos os filhos do rei e aos
capitães do exército, e a Abiatar, o sacerdote, e eis que estão
comendo e bebendo perante ele; e dizem: Viva o rei Adonias.
Porém a mim, sendo eu teu servo, e a Zadoque, o sacerdote, e
a Benaia, filho de Joiada, e a Salomão, teu servo, não convidou.
Foi feito isto da parte do rei meu senhor? E não fizeste
saber a teu servo quem se assentaria no trono do rei meu senhor
depois dele? E respondeu o rei Davi, e disse: Chamai-me a
Bate-Seba. E ela entrou à presença do rei; e ficou em pé diante do
rei. Então jurou o rei e disse: Vive o Senhor, o qual remiu a
minha alma de toda a angústia, que, como te jurei pelo Senhor
Deus de Israel, dizendo: Certamente teu filho Salomão reinará depois
de mim, e ele se assentará no meu trono, em meu lugar, assim o farei
no dia de hoje. Então Bate-Seba se inclinou com o rosto em
terra e se prostrou diante do rei, e disse: Viva o rei Davi meu
senhor para sempre.

E disse o rei Davi: Chamai-me a Zadoque, o sacerdote, e a Natã, o
profeta, e a Benaia, filho de Joiada. E eles entraram à presença do
rei. E o rei lhes disse: Tomai convosco os servos de vosso
senhor, e fazei subir a meu filho Salomão na mula que é minha; e
levai-o a Giom. E Zadoque, o sacerdote, com Natã, o profeta,
ali o ungirão rei sobre Israel; então tocareis a trombeta, e direis:
Viva o rei Salomão! Então subireis após ele, e virá e se
assentará no meu trono, e ele reinará em meu lugar; porque tenho
ordenado que ele seja guia sobre Israel e sobre Judá. Então
Benaia, filho de Joiada, respondeu ao rei, e disse: Amém; assim o
diga o Senhor Deus do rei meu senhor. Como o Senhor foi com o
rei meu senhor, assim o seja com Salomão, e faça que o seu trono
seja maior do que o trono do rei Davi meu senhor. Então
desceu Zadoque, o sacerdote, e Natã, o profeta, e Benaia, filho de
Joiada, e os quereteus, e os peleteus, e fizeram montar a Salomão na
mula do rei Davi, e o levaram a Giom. E Zadoque, o sacerdote,
tomou o chifre de azeite do tabernáculo, e ungiu a Salomão; e
tocaram a trombeta, e todo o povo disse: Viva o rei Salomão!
E todo o povo subiu após ele, e o povo tocava gaitas, e
alegrava-se com grande alegria; de maneira que com o seu clamor a
terra retiniu.

E o ouviu Adonias, e todos os convidados que estavam com ele, que
tinham acabado de comer; também Joabe ouviu o sonido das trombetas,
e disse: Por que há tal ruído de cidade alvoroçada? Estando
ele ainda falando, eis que vem Jônatas, filho de Abiatar, o
sacerdote, e disse Adonias: Entra, porque és homem valente, e trarás
boas novas. E respondeu Jônatas, e disse a Adonias:
Certamente nosso senhor, rei Davi, constituiu rei a Salomão;
e o rei enviou com ele a Zadoque, o sacerdote, e a Natã, o
profeta, e a Benaia, filho de Joiada, e aos quereteus e aos
peleteus; e o fizeram montar na mula do rei. E Zadoque, o
sacerdote, e Natã, o profeta, o ungiram rei em Giom, e dali subiram
alegres, e a cidade está alvoroçada; este é o clamor que ouviste.
E também Salomão está assentado no trono do reino. E
também os servos do rei vieram abençoar a nosso senhor, o rei Davi,
dizendo: Faça teu Deus que o nome de Salomão seja melhor do que o
teu nome; e faça que o seu trono seja maior do que o teu trono. E o
rei se inclinou no leito. E também disse o rei assim: Bendito
o Senhor Deus de Israel, que hoje tem dado quem se assente no meu
trono, e que os meus olhos o vissem. Então estremeceram e se
levantaram todos os convidados que estavam com Adonias; e cada um se
foi ao seu caminho. Porém Adonias temeu a Salomão; e
levantou-se, e foi, e apegou-se às pontas do altar. E fez-se
saber a Salomão, dizendo: Eis que Adonias teme ao rei Salomão;
porque eis que apegou-se às pontas do altar, dizendo: Jure-me hoje o
rei Salomão que não matará o seu servo à espada. E disse
Salomão: Se for homem de bem, nem um de seus cabelos cairá em terra;
se, porém, se achar nele maldade, morrerá. E mandou o rei
Salomão, e o fizeram descer do altar; e veio, e prostrou-se perante
o rei Salomão, e Salomão lhe disse: Vai para tua casa.

\medskip

\lettrine{2} E aproximaram-se os dias da morte de Davi; e deu
ele ordem a Salomão, seu filho, dizendo: Eu vou pelo caminho de
toda a terra; esforça-te, pois, e sê homem. E guarda a ordenança
do Senhor teu Deus, para andares nos seus caminhos, e para guardares
os seus estatutos, e os seus mandamentos, e os seus juízos, e os
seus testemunhos, como está escrito na lei de Moisés; para que
prosperes em tudo quanto fizeres, e para onde quer que fores.
Para que o Senhor confirme a palavra, que falou de mim, dizendo:
Se teus filhos guardarem o seu caminho, para andarem perante a minha
face fielmente, com todo o seu coração e com toda a sua alma, nunca,
disse, te faltará sucessor ao trono de Israel. E também tu sabes
o que me fez Joabe, filho de Zeruia, e o que fez aos dois capitães
do exército de Israel, a Abner filho de Ner, e a Amasa, filho de
Jeter, os quais matou, e em paz derramou o sangue de guerra, e pôs o
sangue de guerra no cinto que tinha nos lombos, e nos sapatos que
trazia nos pés. Faze, pois, segundo a tua sabedoria, e não
permitas que suas cãs desçam à sepultura em paz. Porém com os
filhos de Barzilai, o gileadita, usarás de beneficência, e estarão
entre os que comem à tua mesa, porque assim se chegaram eles a mim,
quando eu fugia por causa de teu irmão Absalão. E eis que também
contigo está Simei, filho de Gera, filho de Benjamim, de Baurim, que
me maldisse com maldição atroz, no dia em que ia a Maanaim; porém
ele saiu a encontrar-se comigo junto ao Jordão, e eu pelo Senhor lhe
jurei, dizendo que o não mataria à espada. Mas agora não o
tenhas por inculpável, pois és homem sábio, e bem saberás o que lhe
hás de fazer para que faças com que as suas cãs desçam à sepultura
com sangue. E Davi dormiu com seus pais, e foi sepultado na
cidade de Davi. E foram os dias que Davi reinou sobre Israel
quarenta anos: sete anos reinou em Hebrom, e em Jerusalém reinou
trinta e três anos.

E Salomão se assentou no trono de Davi, seu pai, e o seu reino se
fortificou sobremaneira. Então veio Adonias, filho de Hagite,
a Bate-Seba, mãe de Salomão; e disse ela: De paz é a tua vinda? E
ele disse: É de paz. Então disse ele: Uma palavra tenho que
dizer-te. E ela disse: Fala. Disse, pois, ele: Bem sabes que
o reino era meu, e todo o Israel tinha posto a vista em mim para que
eu viesse a reinar, contudo o reino foi transferido e veio a ser de
meu irmão, porque foi feito seu pelo Senhor. Assim que agora
uma só petição te faço; não ma rejeites. E ela lhe disse: Fala.
E ele disse: Peço-te que fales ao rei Salomão (porque ele não
te rejeitará) que me dê por mulher a Abisague, a sunamita. E
disse Bate-Seba: Bem, eu falarei por ti ao rei. Assim foi
Bate-Seba ao rei Salomão, a falar-lhe por Adonias; e o rei se
levantou a encontrar-se com ela, e se inclinou diante dela; então se
assentou no seu trono, e fez pôr uma cadeira para a sua mãe, e ela
se assentou à sua direita. Então disse ela: Só uma pequena
petição te faço; não ma rejeites. E o rei lhe disse: Pede, minha
mãe, porque não ta negarei. E ela disse: Dê-se Abisague, a
sunamita, a Adonias, teu irmão, por mulher. Então respondeu o
rei Salomão, e disse a sua mãe: E por que pedes a Abisague, a
sunamita, para Adonias? Pede também para ele o reino (porque é meu
irmão maior), para ele, digo, e também para Abiatar, sacerdote, e
para Joabe, filho de Zeruia. E jurou o rei Salomão pelo
Senhor, dizendo: Assim Deus me faça, e outro tanto, se não falou
Adonias esta palavra contra a sua vida. Agora, pois, vive o
Senhor, que me confirmou, e me fez assentar no trono de Davi, meu
pai, e que me tem feito casa, como tinha falado, que hoje morrerá
Adonias. E enviou o rei Salomão pela mão de Benaia, filho de
Joiada, o qual arremeteu contra ele de modo que morreu.

E a Abiatar, o sacerdote, disse o rei: Vai para Anatote, para os
teus campos, porque és homem digno de morte; porém hoje não te
matarei, porquanto levaste a arca do Senhor DEUS diante de Davi, meu
pai, e porquanto foste aflito em tudo quanto meu pai foi aflito.
Lançou, pois, Salomão fora a Abiatar, para que não fosse
sacerdote do Senhor, para cumprir a palavra do Senhor, que tinha
falado sobre a casa de Eli em Siló. E chegou a notícia até
Joabe (porque Joabe tinha se desviado seguindo a Adonias, ainda que
não tinha se desviado seguindo a Absalão), e Joabe fugiu para o
tabernáculo do Senhor, e apegou-se às pontas do altar. E
disseram ao rei Salomão que Joabe tinha fugido para o tabernáculo do
Senhor; e eis que está junto ao altar; então Salomão enviou Benaia,
filho de Joiada, dizendo: Vai, arremete sobre ele. E foi
Benaia ao tabernáculo do Senhor, e lhe disse: Assim diz o rei: Sai
daí. E disse ele: Não, porém aqui morrerei. E Benaia tornou com a
resposta ao rei, dizendo: Assim falou Joabe, e assim me respondeu.
E disse-lhe o rei: Faze como ele disse, e arremete contra
ele, e sepulta-o, para que tires de mim e da casa de meu pai o
sangue que Joabe sem causa derramou.

Assim o Senhor fará recair o sangue dele sobre a sua cabeça,
porque deu sobre dois homens mais justos e melhores do que ele, e os
matou à espada, sem que meu pai Davi o soubesse, a saber: a Abner,
filho de Ner, capitão do exército de Israel, e a Amasa, filho de
Jeter, capitão do exército de Judá. Assim recairá o sangue
destes sobre a cabeça de Joabe e sobre a cabeça da sua descendência
para sempre; mas a Davi, e à sua descendência, e à sua casa, e ao
seu trono, dará o Senhor paz para todo o sempre. E subiu
Benaia, filho de Joiada, e arremeteu contra ele, e o matou; e foi
sepultado em sua casa, no deserto. E o rei pôs a Benaia,
filho de Joiada, em seu lugar sobre o exército, e a Zadoque, o
sacerdote, pôs o rei em lugar de Abiatar. Depois mandou o
rei, e chamou a Simei, e disse-lhe: Edifica-te uma casa em
Jerusalém, e habita aí, e daí não saias, nem para uma nem para outra
parte. Porque há de ser que no dia em que saíres e passares o
ribeiro de Cedrom, de certo que sem dúvida morrerás; o teu sangue
será sobre a tua cabeça. E Simei disse ao rei: Boa é essa
palavra; como tem falado o rei meu Senhor, assim fará o teu servo. E
Simei habitou em Jerusalém muitos dias. Sucedeu, porém, que,
ao cabo de três anos, dois servos de Simei fugiram para Aquis, filho
de Maaca, rei de Gate; e deram parte a Simei, dizendo: Eis que teus
servos estão em Gate. Então Simei se levantou, e albardou o
seu jumento, e foi a Gate, ter com Aquis, em busca de seus servos;
assim foi Simei, e trouxe os seus servos de Gate.

E disseram a Salomão como Simei fora de Jerusalém a Gate, e já
tinha voltado. Então o rei mandou chamar a Simei, e
disse-lhe: Não te conjurei eu pelo Senhor, e protestei contra ti,
dizendo: No dia em que saíres para uma ou outra parte, sabe de certo
que, sem dúvida, morrerás? E tu me disseste: Boa é essa palavra que
ouvi. Por que, pois, não guardaste o juramento do Senhor, nem
a ordem que te dei? Disse mais o rei a Simei: Bem sabes tu
toda a maldade que o teu coração reconhece, que fizeste a Davi, meu
pai; pelo que o Senhor fez recair a tua maldade sobre a tua cabeça.
Mas o rei Salomão será abençoado, e o trono de Davi será
confirmado perante o Senhor para sempre. E o rei mandou a
Benaia, filho de Joiada, o qual saiu, e arremeteu contra ele, de
modo que morreu; assim foi confirmado o reino na mão de Salomão.

\medskip

\lettrine{3} E Salomão se aparentou com Faraó, rei do Egito; e
tomou a filha de Faraó, e a trouxe à cidade de Davi, até que
acabasse de edificar a sua casa, e a casa do Senhor, e a muralha de
Jerusalém em redor. Entretanto, o povo sacrificava sobre os
altos; porque até àqueles dias ainda não se havia edificado casa ao
nome do Senhor. E Salomão amava ao Senhor, andando nos estatutos
de Davi seu pai; somente que nos altos sacrificava, e queimava
incenso. E foi o rei a Gibeom para lá sacrificar, porque aquele
era o alto maior; mil holocaustos sacrificou Salomão naquele altar.

E em Gibeom apareceu o Senhor a Salomão de noite em sonhos; e
disse-lhe Deus: Pede o que queres que eu te dê. E disse Salomão:
De grande beneficência usaste tu com teu servo Davi, meu pai, como
também ele andou contigo em verdade, e em justiça, e em retidão de
coração, perante a tua face; e guardaste-lhe esta grande
beneficência, e lhe deste um filho que se assentasse no seu trono,
como se vê neste dia. Agora, pois, ó Senhor meu Deus, tu fizeste
reinar a teu servo em lugar de Davi meu pai; e sou apenas um menino
pequeno; não sei como sair, nem como entrar. E teu servo está no
meio do teu povo que elegeste; povo grande, que nem se pode contar,
nem numerar, pela sua multidão. A teu servo, pois, dá um coração
entendido para julgar a teu povo, para que prudentemente discirna
entre o bem e o mal; porque quem poderia julgar a este teu tão
grande povo? E esta palavra pareceu boa aos olhos do Senhor,
de que Salomão pedisse isso. E disse-lhe Deus: Porquanto
pediste isso, e não pediste para ti muitos dias, nem pediste para ti
riquezas, nem pediste a vida de teus inimigos; mas pediste para ti
entendimento, para discernires o que é justo; eis que fiz
segundo as tuas palavras; eis que te dei um coração tão sábio e
entendido, que antes de ti igual não houve, e depois de ti igual não
se levantará. E também até o que não pediste te dei, assim
riquezas como glória; de modo que não haverá um igual entre os reis,
por todos os teus dias. E, se andares nos meus caminhos,
guardando os meus estatutos, e os meus mandamentos, como andou Davi
teu pai, também prolongarei os teus dias. E acordou Salomão,
e eis que era sonho. E indo a Jerusalém, pôs-se perante a arca da
aliança do Senhor, e sacrificou holocausto, e preparou sacrifícios
pacíficos, e fez um banquete a todos os seus servos.

Então vieram duas mulheres prostitutas ao rei, e se puseram
perante ele. E disse-lhe uma das mulheres: Ah! senhor meu, eu
e esta mulher moramos numa casa; e tive um filho, estando com ela
naquela casa. E sucedeu que, ao terceiro dia, depois do meu
parto, teve um filho também esta mulher; estávamos juntas; nenhum
estranho estava conosco na casa; somente nós duas naquela casa.
E de noite morreu o filho desta mulher, porquanto se deitara
sobre ele. E levantou-se à meia noite, e tirou o meu filho do
meu lado, enquanto dormia a tua serva, e o deitou no seu seio; e a
seu filho morto deitou no meu seio. E, levantando-me eu pela
manhã, para dar de mamar a meu filho, eis que estava morto; mas,
atentando pela manhã para ele, eis que não era meu filho, que eu
havia tido. Então disse à outra mulher: Não, mas o vivo é meu
filho, e teu filho o morto. Porém esta disse: Não, por certo, o
morto é teu filho, e meu filho o vivo. Assim falaram perante o rei.
Então disse o rei: Esta diz: Este que vive é meu filho, e teu
filho o morto; e esta outra diz: Não, por certo, o morto é teu filho
e meu filho o vivo. Disse mais o rei: Trazei-me uma espada. E
trouxeram uma espada diante do rei. E disse o rei: Dividi em
duas partes o menino vivo; e dai metade a uma, e metade a outra.
Mas a mulher, cujo filho era o vivo, falou ao rei (porque as
suas entranhas se lhe enterneceram por seu filho), e disse: Ah!
senhor meu, dai-lhe o menino vivo, e de modo nenhum o mateis. Porém
a outra dizia: Nem teu nem meu seja; dividi-o. Então
respondeu o rei, e disse: Dai a esta o menino vivo, e de maneira
nenhuma o mateis, porque esta é sua mãe. E todo o Israel
ouviu o juízo que havia dado o rei, e temeu ao rei; porque viram que
havia nele a sabedoria de Deus, para fazer justiça.

\medskip

\lettrine{4} Assim foi Salomão rei sobre todo o Israel. E
estes eram os príncipes que tinha: Azarias, filho de Zadoque,
sacerdote; Eliorefe e Aías, filhos de Sisa, secretários;
Jeosafá, filho de Ailude, cronista; Benaia, filho de Joiada,
sobre o exército; e Zadoque e Abiatar eram sacerdotes; e
Azarias, filho de Natã, sobre os provedores; e Zabude, filho de
Natã, oficial-mor, amigo do rei; e Aisar, mordomo; Adonirão,
filho de Abda, sobre o tributo. E tinha Salomão doze oficiais
sobre todo o Israel, que proviam ao rei e à sua casa; e cada um
tinha que abastecê-lo por um mês no ano. E estes são os seus
nomes: Ben-Hur, nas montanhas de Efraim; Ben-Dequer em Macaz, e
em Saalbim, e em Bete-Semes, e em Elom, e em Bete-Hanã;
Ben-Hesede em Arubote; também este tinha a Socó e a toda a
terra de Hefer; Ben-Abinadabe em todo o termo de Dor; tinha
este a Tafate, filha de Salomão, por mulher; Baaná, filho de
Ailude, tinha a Taanaque, e a Megido, e a toda a Bete-Seã, que está
junto a Zaretã, abaixo de Jizreel, desde Bete-Seã até Abel-Meolá,
para além de Jocmeão; o filho de Geber, em Ramote de Gileade;
tinha este as aldeias de Jair, filho de Manassés, as quais estão em
Gileade; também tinha o termo de Argobe, o qual está em Basã,
sessenta grandes cidades, com muros e ferrolhos de cobre;
Ainadabe, filho de Ido, em Maanaim. Aimaás em Naftali;
também este tomou a Basemate, filha de Salomão, por mulher;
Baaná, filho de Husai, em Aser e em Alote; Jeosafá,
filho de Parua, em Issacar; Simei, filho de Elá, em Benjamim;
Geber, filho de Uri, na terra de Gileade, a terra de Siom,
rei dos amorreus, e de Ogue, rei de Basã; e só uma guarnição havia
naquela terra.

Eram, pois, os de Judá e Israel muitos, como a areia que está
junto ao mar em multidão, comendo, e bebendo, e alegrando-se.
E dominava Salomão sobre todos os reinos desde o rio até à
terra dos filisteus, e até ao termo do Egito; os quais traziam
presentes, e serviram a Salomão todos os dias da sua vida.
Era, pois, o provimento de Salomão cada dia, trinta
coros\footnote{Antiga medida hebraica: 10 batos (bato = efa: 37
litros) ou 370 litros.} de flor de farinha, e sessenta coros de
farinha; dez bois cevados, e vinte bois de pasto, e cem
carneiros; afora os veados e as cabras montesas, e os corços, e aves
cevadas. Porque dominava sobre tudo quanto havia do lado de
cá do rio, Tifsa até Gaza, sobre todos os reis do lado de cá do rio;
e tinha paz de todos os lados em redor dele. E Judá e Israel
habitavam seguros, cada um debaixo da sua videira, e debaixo da sua
figueira, desde Dã até Berseba, todos os dias de Salomão.
Tinha também Salomão quarenta mil estrebarias de cavalos para
os seus carros, e doze mil cavaleiros. Proviam, pois, estes
provedores, cada um no seu mês, ao rei Salomão e a todos quantos se
chegaram à mesa do rei Salomão; coisa nenhuma deixavam faltar.
E traziam a cevada e a palha para os cavalos e para os
ginetes\footnote{Cavalo de boa raça, fino e bem adestrado.}, para o
lugar onde estava, cada um segundo o seu cargo.

E deu Deus a Salomão sabedoria, e muitíssimo entendimento, e
largueza de coração, como a areia que está na praia do mar. E
era a sabedoria de Salomão maior do que a sabedoria de todos os do
oriente e do que toda a sabedoria dos egípcios. E era ele
ainda mais sábio do que todos os homens, e do que Etã, ezraíta, e
Hemã, e Calcol, e Darda, filhos de Maol; e correu o seu nome por
todas as nações em redor. E disse três mil provérbios, e
foram os seus cânticos mil e cinco. Também falou das árvores,
desde o cedro que está no Líbano até ao hissopo que nasce na parede;
também falou dos animais e das aves, e dos répteis e dos peixes.
E vinham de todos os povos a ouvir a sabedoria de Salomão, e
de todos os reis da terra que tinham ouvido da sua sabedoria.

\medskip

\lettrine{5} E enviou Hirão, rei de Tiro, os seus servos a
Salomão (porque ouvira que ungiram a Salomão rei em lugar de seu
pai), porquanto Hirão sempre tinha amado a Davi. Então Salomão
mandou dizer a Hirão: Bem sabes tu que Davi, meu pai, não pôde
edificar uma casa ao nome do Senhor seu Deus, por causa da guerra
com que o cercaram, até que o Senhor pôs seus inimigos debaixo das
plantas dos seus pés. Porém agora o Senhor meu Deus me tem dado
descanso de todos os lados; adversário não há, nem algum mau
encontro. E eis que eu intento edificar uma casa ao nome do
Senhor meu Deus, como falou o Senhor a Davi, meu pai, dizendo: Teu
filho, que porei em teu lugar no teu trono, ele edificará uma casa
ao meu nome. Dá ordem, pois, agora, que do Líbano me cortem
cedros, e os meus servos estarão com os teus servos, e eu te darei o
salário dos teus servos, conforme a tudo o que disseres; porque bem
sabes tu que entre nós ninguém há que saiba cortar a madeira como os
sidônios. E aconteceu que, ouvindo Hirão as palavras de Salomão,
muito se alegrou, e disse: Bendito seja hoje o Senhor, que deu a
Davi um filho sábio sobre este tão grande povo. E enviou Hirão a
Salomão, dizendo: Ouvi o que me mandaste dizer. Eu farei toda a tua
vontade acerca das madeiras de cedro e de cipreste. Os meus
servos as levarão desde o Líbano até ao mar, e eu as farei conduzir
em jangadas pelo mar até ao lugar que me designares, e ali as
desamarrarei; e tu as tomarás; tu também farás a minha vontade,
dando sustento à minha casa.

Assim deu Hirão a Salomão madeira de cedro e madeira de cipreste,
conforme a toda a sua vontade. E Salomão deu a Hirão vinte
mil coros de trigo, para sustento da sua casa, e vinte coros de
azeite batido; isto dava Salomão a Hirão anualmente. Deu,
pois, o Senhor a Salomão sabedoria, como lhe tinha falado; e houve
paz entre Hirão e Salomão, e ambos fizeram acordo. E o rei
Salomão fez subir uma leva de gente dentre todo o Israel, e foi a
leva de gente trinta mil homens; e os enviava ao Líbano, cada
mês, dez mil por turno; um mês estavam no Líbano, e dois meses cada
um em sua casa; e Adonirão estava sobre a leva de gente.
Tinha também Salomão setenta mil que levavam as cargas, e
oitenta mil que talhavam pedras nas montanhas, afora os
chefes dos oficiais de Salomão, que estavam sobre aquela obra, três
mil e trezentos, os quais davam as ordens ao povo que fazia aquela
obra. E mandou o rei que trouxessem pedras grandes, e pedras
valiosas, pedras lavradas, para fundarem a casa. E as
lavraram os edificadores de Hirão, e os giblitas; e preparavam a
madeira e as pedras para edificar a casa.

\medskip

\lettrine{6} E sucedeu que no ano de quatrocentos e oitenta,
depois de saírem os filhos de Israel do Egito, no ano quarto do
reinado de Salomão sobre Israel, no mês de Zive (este é o mês
segundo), começou a edificar a casa do Senhor. E a casa que o
rei Salomão edificou ao Senhor era de sessenta
côvados\footnote{Côvado: do cotovelo à ponta dos dedos = 45
centímetros.} de comprimento, e de vinte côvados de largura, e de
trinta côvados de altura. E o pórtico diante do templo da casa
era de vinte côvados de comprimento, segundo a largura da casa, e de
dez côvados de largura diante da casa. E fez para a casa janelas
de gelosias\footnote{Grade de fasquias de madeira cruzadas
intervaladamente, que ocupa o vão de uma janela; rótula. Janela de
rótula. SBTB grafa erradamente \emph{gelósias}, com acento. AV: And
for the house he made windows of narrow lights. RA: Para a casa, fez
janelas de fasquias fixas superpostas. RC: E fez à casa janelas de
vista estreita.} fixas. E edificou câmaras junto ao muro da
casa, contra as paredes da casa, em redor, tanto do templo como do
oráculo; e assim lhe fez câmaras laterais em redor. A câmara de
baixo era de cinco côvados de largura, e a do meio de seis côvados
de largura, e a terceira de sete côvados de largura; porque pela
parte de fora da casa, em redor, fizera encostos, para que as vigas
não se apoiassem nas paredes da casa. E edificava-se a casa com
pedras preparadas, como as traziam se edificava; de maneira que nem
martelo, nem machado, nem nenhum outro instrumento de ferro se ouviu
na casa quando a edificavam. A porta da câmara do meio estava ao
lado direito da casa, e por caracóis se subia à do meio, e da do
meio à terceira. Assim, pois, edificou a casa, e a acabou; e
cobriu a casa com pranchões e tabuados de cedro. Também
edificou as câmaras em volta de toda a casa, de cinco côvados de
altura, e as ligou à casa com madeira de cedro.

Então veio a palavra do Senhor a Salomão, dizendo: Quanto
a esta casa que tu edificas, se andares nos meus estatutos, e
fizeres os meus juízos, e guardares todos os meus mandamentos,
andando neles, confirmarei para contigo a minha palavra, a qual
falei a Davi, teu pai; e habitarei no meio dos filhos de
Israel, e não desampararei o meu povo de Israel. Assim
edificou Salomão aquela casa, e a acabou.

Também cobriu as paredes da casa por dentro com tábuas de cedro;
desde o soalho\footnote{Pavimento de madeira; soalhado, sobrado.} da
casa até ao teto tudo cobriu com madeira por dentro; e cobriu o
soalho da casa com tábuas de cipreste. Edificou mais vinte
côvados de tábuas de cedro nos lados da casa, desde o soalho até às
paredes; e por dentro lhas edificou para o oráculo, para o Santo dos
Santos. A casa, isto é, o templo anterior tinha quarenta
côvados. E o cedro da casa por dentro era lavrado de botões e
flores abertas; tudo era cedro, pedra nenhuma se via. E por
dentro da casa, na parte mais interior, preparou o oráculo, para pôr
ali a arca da aliança do Senhor. E o oráculo no interior era
de vinte côvados de comprimento, e de vinte côvados de largura, e de
vinte côvados de altura; e o revestiu de ouro puro; também revestiu
de cedro o altar. E revestiu Salomão a casa por dentro de
ouro puro; e com cadeias de ouro pôs uma cortina diante do oráculo,
e o revestiu com ouro. Assim cobriu de ouro toda a casa,
inteiramente; também cobriu de ouro todo o altar que estava diante
do oráculo. E no oráculo fez dois querubins de madeira de
oliveira, cada um da altura de dez côvados. E uma asa de um
querubim era de cinco côvados, e a outra asa do querubim de outros
cinco côvados; dez côvados havia desde a extremidade de uma das suas
asas até à extremidade da outra das suas asas. Assim era
também de dez côvados o outro querubim; ambos os querubins eram de
uma mesma medida e de um mesmo talhe. A altura de um querubim
era de dez côvados, e assim a do outro querubim. E pôs a
estes querubins no meio da casa de dentro; e os querubins estendiam
as asas, de maneira que a asa de um tocava na parede, e a asa do
outro querubim tocava na outra parede; e as suas asas no meio da
casa tocavam uma na outra. E revestiu de ouro os querubins.
E todas as paredes da casa, em redor, lavrou de esculturas e
entalhes de querubins, e de palmas, e de flores abertas, por dentro
e por fora. Também revestiu de ouro o soalho da casa, por
dentro e por fora. E à entrada do oráculo fez portas de
madeira de oliveira; o umbral de cima com as ombreiras faziam a
quinta parte da parede. Também as duas portas eram de madeira
de oliveira; e lavrou nelas entalhes de querubins, e de palmas, e de
flores abertas, os quais revestiu de ouro; também estendeu ouro
sobre os querubins e sobre as palmas. E assim fez à porta do
templo ombreiras de madeira de oliveira, da quarta parte da parede.
E eram as duas portas de madeira de cipreste; e as duas
folhas de uma porta eram dobradiças, assim como eram também
dobradiças as duas folhas entalhadas das outras portas. E as
lavrou de querubins e de palmas, e de flores abertas, e as revestiu
de ouro acomodado ao lavor. Também edificou o pátio interior
de três ordens de pedras lavradas e de uma ordem de vigas de cedro.
No ano quarto se pôs o fundamento da casa do Senhor, no mês
de Zive. E no ano undécimo, no mês de Bul, que é o mês
oitavo, se acabou esta casa com todas as suas coisas, e com tudo o
que lhe convinha; e a edificou em sete anos.

\medskip

\lettrine{7} Porém a sua casa edificou Salomão em treze anos;
e acabou toda a sua casa. Também edificou a casa do bosque do
Líbano de cem côvados de comprimento, e de cinqüenta côvados de
largura, e de trinta côvados de altura, sobre quatro ordens de
colunas de cedro, e vigas de cedro sobre as colunas. E por cima
estava coberta de cedro sobre as vigas, que estavam sobre quarenta e
cinco colunas, quinze em cada ordem. E havia três ordens de
janelas; e uma janela estava defronte da outra janela, em três
ordens. Também todas as portas e ombreiras quadradas eram de uma
mesma vista; e uma janela estava defronte da outra, em três ordens.
Depois fez um pórtico de colunas de cinqüenta côvados de
comprimento e de trinta côvados de largura; e o pórtico estava
defronte delas, e as colunas com as grossas vigas defronte delas.
Também fez o pórtico para o trono onde julgava, isto é, o
pórtico do juízo, que estava revestido de cedro de soalho a soalho.
E em sua casa, em que morava, havia outro pátio, por dentro do
pórtico, de obra semelhante à deste; também para a filha de Faraó,
que Salomão tomara por mulher, fez uma casa semelhante àquele
pórtico. Todas estas coisas eram de pedras de grande valor,
cortadas à medida, serradas à serra por dentro e por fora; e isto
desde o fundamento até às beiras do teto, e por fora até ao grande
pátio. Também estava fundado sobre pedras finas, pedras
grandes; sobre pedras de dez côvados e pedras de oito côvados.
E em cima delas pedras de grande valor, lavradas segundo as
medidas, e madeira de cedro. Havia três ordens de pedras
lavradas, com uma ordem de vigas de cedro; assim era também o pátio
interior da casa do Senhor e o pórtico daquela casa.

E enviou o rei Salomão um mensageiro e mandou trazer a Hirão de
Tiro. Era ele filho de uma mulher viúva, da tribo de Naftali,
e fora seu pai um homem de Tiro, que trabalhava em cobre; e era
cheio de sabedoria, e de entendimento, e de ciência para fazer toda
a obra de cobre; este veio ao rei Salomão, e fez toda a sua obra.
E formou duas colunas de cobre; a altura de cada coluna era
de dezoito côvados, e um fio de doze côvados cercava cada uma das
colunas. Também fez dois capitéis\footnote{Capitel:
Coroamento do fuste (A parte principal da coluna, entre o capitel e
a base. Haste, cabo.) de uma coluna. Arremate superior, em geral
esculturado, de pilastra, balaústre, etc.} de fundição de cobre para
pôr sobre as cabeças das colunas; de cinco côvados era a altura de
um capitel, e de cinco côvados a altura do outro capitel. As
redes eram de malhas, as ligas de obra de cadeia para os capitéis
que estavam sobre a cabeça das colunas, sete para um capitel e sete
para o outro capitel. Assim fez as colunas, juntamente com
duas fileiras em redor sobre uma rede, para cobrir os capitéis que
estavam sobre a cabeça das romãs, assim também fez com o outro
capitel. E os capitéis que estavam sobre a cabeça das colunas
eram de obra de lírios no pórtico, de quatro côvados. Os
capitéis, pois, sobre as duas colunas estavam também defronte, em
cima da parte globular que estava junto à rede; e duzentas romãs, em
fileiras em redor, estavam também sobre o outro capitel.
Depois levantou as colunas no pórtico do templo; e levantando
a coluna direita, pôs-lhe o nome de Jaquim; e levantando a coluna
esquerda, pôs-lhe o nome de Boaz. E sobre a cabeça das
colunas estava a obra de lírios; e assim se acabou a obra das
colunas. Fez mais o mar de fundição, de dez côvados de uma
borda até à outra borda, perfeitamente redondo, e de cinco côvados
de alto; e um cordão de trinta côvados o cingia em redor. E
por baixo da sua borda em redor havia botões que o cingiam; por dez
côvados cercavam aquele mar em redor; duas ordens destes botões
foram fundidas quando o mar foi fundido. E firmava-se sobre
doze bois, três que olhavam para o norte, e três que olhavam para o
ocidente, e três que olhavam para o sul, e três que olhavam para o
oriente; e o mar estava em cima deles, e todas as suas partes
posteriores para o lado de dentro. E a grossura era de um
palmo, e a sua borda era como a de um copo, como de flor de lírios;
ele levava dois mil batos\footnote{Bato = efa: 37 litros.}.
Fez também as dez bases de cobre; o comprimento de uma base
de quatro côvados, e de quatro côvados a sua largura, e três côvados
a sua altura.
 E esta era a obra das bases; tinham cintas, e as cintas estavam
entre as molduras. E sobre as cintas que estavam entre as
molduras havia leões, bois, e querubins, e sobre as molduras uma
base por cima; e debaixo dos leões e dos bois junturas de obra
estendida. E uma base tinha quatro rodas de metal, e lâminas
de cobre; e os seus quatro cantos tinham suportes; debaixo da pia
estavam estes suportes fundidos, do lado de cada uma das junturas.
E a boca da pia estava dentro da coroa, e de um côvado por
cima; e era a sua boca redonda segundo a obra da base, de côvado e
meio; e também sobre a sua boca havia entalhes, e as suas cintas
eram quadradas, não redondas. E as quatro rodas estavam
debaixo das cintas, e os eixos das rodas na base; e era a altura de
cada roda de côvado e meio. E era a obra das rodas como a
obra da roda de carro; seus eixos, e suas cambas\footnote{Cada uma
das peças curvas das rodas de um veículo, onde se prendem os raios.}
e seus cubos, e seus raios, todos eram fundidos. E havia
quatro suportes aos quatro cantos de cada base; seus suportes saíam
da base. E no alto de cada base havia uma peça redonda de
meio côvado de altura; também sobre o alto de cada base havia asas e
cintas, que saíam delas. E nas placas de seus esteios e nas
suas cintas lavrou querubins, leões, e palmas, segundo o espaço de
cada uma, e outros adornos em redor. Conforme a esta fez as
dez bases; todas tinham uma mesma fundição, uma mesma medida, e um
mesmo entalhe. Também fez dez pias de cobre; em cada pia
cabiam quarenta batos, e cada pia era de quatro côvados, e sobre
cada uma das dez bases estava uma pia. E pôs cinco bases à
direita da casa, e cinco à esquerda da casa; porém o mar pôs ao lado
direito da casa para o lado do oriente, da parte do sul.
Depois fez Hirão as pias, e as pás, e as bacias; e acabou
Hirão de fazer toda a obra que fez para o rei Salomão, para a casa
do Senhor. A saber: as duas colunas, e os globos dos capitéis
que estavam sobre a cabeça das duas colunas; e as duas redes, para
cobrir os dois globos dos capitéis que estavam sobre a cabeça das
colunas. E as quatrocentas romãs para as duas redes, a saber:
duas carreiras de romãs para cada rede, para cobrirem os dois globos
dos capitéis que estavam em cima das colunas. Juntamente com
as dez bases, e as dez pias sobre as bases; como também um
mar, e os doze bois debaixo daquele mar; e os caldeirões, e
as pás, e as bacias, e todos estes objetos que fez Hirão para o rei
Salomão, para a casa do Senhor, todos eram de cobre polido.
Na planície do Jordão, o rei os fundiu em terra barrenta;
entre Sucote e Zaretã. E deixou Salomão de pesar todos os
objetos, pelo seu excessivo número; nem se averiguou o peso do
cobre.

Também fez Salomão todos os objetos que convinham à casa do
Senhor; o altar de ouro, e a mesa de ouro, sobre a qual estavam os
pães da proposição. E os castiçais, cinco à direita e cinco à
esquerda, diante do oráculo, de ouro finíssimo; e as flores, e as
lâmpadas, e os espevitadores, também de ouro. Como também os
vasos, e os apagadores, e as bacias, e as colheres, e os
perfumadores, de ouro finíssimo; e as dobradiças para as portas da
casa interior para o lugar santíssimo, e as das portas da casa do
templo, também de ouro. Assim se acabou toda a obra que fez o
rei Salomão para a casa do Senhor; então trouxe Salomão as coisas
que seu pai Davi havia consagrado; a prata, e o ouro, e os objetos
pôs entre os tesouros da casa do Senhor.

\medskip

\lettrine{8} Então congregou Salomão os anciãos de Israel, e
todos os cabeças das tribos, os chefes dos pais dos filhos de
Israel, diante de si em Jerusalém; para fazerem subir a arca da
aliança do Senhor da cidade de Davi, que é Sião. E todos os
homens de Israel se congregaram ao rei Salomão, na ocasião da festa,
no mês de Etanim, que é o sétimo mês. E vieram todos os anciãos
de Israel; e os sacerdotes alçaram a arca. E trouxeram a arca do
Senhor para cima, e o tabernáculo da congregação, juntamente com
todos os objetos sagrados que havia no tabernáculo; assim os
trouxeram para cima os sacerdotes e os levitas. E o rei Salomão,
e toda a congregação de Israel que se congregara a ele, estava com
ele diante da arca, sacrificando ovelhas e vacas, que não se podiam
contar nem numerar pela sua quantidade. Assim trouxeram os
sacerdotes a arca da aliança do Senhor ao seu lugar, ao oráculo da
casa, ao lugar santíssimo, até debaixo das asas dos querubins.
Porque os querubins estendiam ambas as asas sobre o lugar da
arca; e os querubins cobriam, por cima, a arca e os seus varais.
E os varais sobressaíram tanto, que as pontas dos varais se viam
desde o santuário diante do oráculo, porém de fora não se viam; e
ficaram ali até ao dia de hoje. Na arca nada havia, senão só as
duas tábuas de pedra, que Moisés ali pusera junto a Horebe, quando o
Senhor fez a aliança com os filhos de Israel, saindo eles da terra
do Egito. E sucedeu que, saindo os sacerdotes do santuário,
uma nuvem encheu a casa do Senhor. E os sacerdotes não podiam
permanecer em pé para ministrar, por causa da nuvem, porque a glória
do Senhor enchera a casa do Senhor.

Então falou Salomão: O Senhor disse que ele habitaria nas trevas.
Certamente te edifiquei uma casa para morada, assento para a
tua eterna habitação. Então virou o rei o seu rosto, e
abençoou toda a congregação de Israel; e toda a congregação de
Israel estava em pé. E disse: Bendito seja o Senhor Deus de
Israel, que falou pela sua boca a Davi, meu pai, e pela sua mão o
cumpriu, dizendo: Desde o dia em que eu tirei o meu povo
Israel do Egito, não escolhi cidade alguma de todas as tribos de
Israel, para edificar alguma casa para ali estabelecer o meu nome;
porém escolhi a Davi, para que presidisse sobre o meu povo Israel.
Também Davi, meu pai, propusera em seu coração o edificar
casa ao nome do Senhor Deus de Israel. Porém o Senhor disse a
Davi, meu pai: Porquanto propuseste no teu coração o edificar casa
ao meu nome, bem fizeste em o propor no teu coração. Todavia
tu não edificarás esta casa; porém teu filho, que sair de teus
lombos, edificará esta casa ao meu nome. Assim confirmou o
Senhor a sua palavra que falou; porque me levantei em lugar de Davi,
meu pai, e me assentei no trono de Israel, como tem falado o Senhor;
e edifiquei uma casa ao nome do Senhor Deus de Israel. E
constituí ali lugar para a arca em que está a aliança do Senhor, a
qual fez com nossos pais, quando os tirou da terra do Egito.

E pôs-se Salomão diante do altar do Senhor, na presença de toda a
congregação de Israel; e estendeu as suas mãos para os céus,
e disse: Ó Senhor Deus de Israel, não há Deus como tu, em
cima nos céus nem em baixo na terra; que guardas a aliança e a
beneficência a teus servos que andam com todo o seu coração diante
de ti. Que guardaste a teu servo Davi, meu pai, o que lhe
disseras; porque com a tua boca o disseste, e com a tua mão o
cumpriste, como neste dia se vê. Agora, pois, ó Senhor Deus
de Israel, guarda a teu servo Davi, meu pai, o que lhe falaste,
dizendo: Não te faltará sucessor diante de mim, que se assente no
trono de Israel; somente que teus filhos guardem o seu caminho, para
andarem diante de mim como tu andaste diante de mim. Agora
também, ó Deus de Israel, cumpra-se a tua palavra que disseste a teu
servo Davi, meu pai. Mas, na verdade, habitaria Deus na
terra? Eis que os céus, e até o céu dos céus, não te poderiam
conter, quanto menos esta casa que eu tenho edificado.
Volve-te, pois, para a oração de teu servo, e para a sua
súplica, ó Senhor meu Deus, para ouvires o clamor e a oração que o
teu servo hoje faz diante de ti. Para que os teus olhos noite
e dia estejam abertos sobre esta casa, sobre este lugar, do qual
disseste: O meu nome estará ali; para ouvires a oração que o teu
servo fizer neste lugar. Ouve, pois, a súplica do teu servo,
e do teu povo Israel, quando orarem neste lugar; também ouve tu no
lugar da tua habitação nos céus; ouve também, e perdoa.
Quando alguém pecar contra o seu próximo, e puserem sobre ele
juramento de maldição, fazendo-o jurar, e vier juramento de maldição
diante do teu altar nesta casa, ouve tu, então, nos céus e
age e julga a teus servos, condenando ao injusto, fazendo recair o
seu proceder sobre a sua cabeça, e justificando ao justo,
rendendo-lhe segundo a sua justiça. Quando o teu povo Israel
for ferido diante do inimigo, por ter pecado contra ti, e se
converterem a ti, e confessarem o teu nome, e orarem e suplicarem a
ti nesta casa, ouve tu então nos céus, e perdoa o pecado do
teu povo Israel, e torna-o a levar à terra que tens dado a seus
pais. Quando os céus se fecharem\footnote{SBTB: ``Quando os
céus se \emph{fechar}''(?!).}, e não houver chuva, por terem pecado
contra ti, e orarem neste lugar, e confessarem o teu nome, e se
converterem dos seus pecados, havendo-os tu afligido, ouve tu
então nos céus, e perdoa o pecado de teus servos e do teu povo
Israel, ensinando-lhes o bom caminho em que andem, e dá chuva na tua
terra que deste ao teu povo em herança. Quando houver fome na
terra, quando houver peste, quando houver queima de searas,
ferrugem, gafanhotos ou pulgão, quando o seu inimigo o cercar na
terra das suas portas, ou houver alguma praga ou doença, toda
a oração, toda a súplica, que qualquer homem de todo o teu povo
Israel fizer, conhecendo cada um a chaga do seu coração, e
estendendo as suas mãos para esta casa, ouve tu então nos
céus, assento da tua habitação, e perdoa, e age, e dá a cada um
conforme a todos os seus caminhos, e segundo vires o seu coração,
porque só tu conheces o coração de todos os filhos dos homens.
Para que te temam todos os dias que viverem na terra que
deste a nossos pais. E também ouve ao estrangeiro, que não
for do teu povo Israel, quando vier de terras remotas, por amor do
teu nome42(porque ouvirão do teu grande nome, e da tua forte mão, e
do teu braço estendido), e vier orar voltado para esta casa,
ouve tu nos céus, assento da tua habitação, e faze conforme a
tudo o que o estrangeiro a ti clamar, a fim de que todos os povos da
terra conheçam o teu nome, para te temerem como o teu povo Israel, e
para saberem que o teu nome é invocado sobre esta casa que tenho
edificado. Quando o teu povo sair à guerra contra o seu
inimigo, pelo caminho por que os enviares, e orarem ao Senhor, para
o lado desta cidade, que tu elegeste, e desta casa, que edifiquei ao
teu nome, ouve, então, nos céus a sua oração e a sua súplica,
e faze-lhes justiça. Quando pecarem contra ti (pois não há
homem que não peque), e tu te indignares contra eles, e os
entregares às mãos do inimigo, de modo que os levem em cativeiro
para a terra inimiga, quer longe ou perto esteja, e na terra
aonde forem levados em cativeiro caírem em si, e se converterem, e
na terra do seu cativeiro te suplicarem, dizendo: Pecamos, e
perversamente procedemos, e cometemos iniqüidade, e se
converterem a ti com todo o seu coração e com toda a sua alma, na
terra de seus inimigos que os levarem em cativeiro, e orarem a ti
para o lado da sua terra que deste a seus pais, para esta cidade que
elegeste, e para esta casa que edifiquei ao teu nome; ouve
então nos céus, assento da tua habitação, a sua oração e a sua
súplica, e faze-lhes justiça. E perdoa ao teu povo que houver
pecado contra ti, todas as transgressões que houverem cometido
contra ti; e dá-lhes misericórdia perante aqueles que os têm
cativos, para que deles tenham compaixão. Porque são o teu
povo e a tua herança que tiraste da terra do Egito, do meio do forno
de ferro. Para que teus olhos estejam abertos à súplica do
teu servo e à súplica do teu povo Israel, a fim de os ouvires em
tudo quanto clamarem a ti. Pois tu para tua herança os
elegeste de todos os povos da terra, como tens falado pelo
ministério de Moisés, teu servo, quando tiraste a nossos pais do
Egito, Senhor DEUS.

Sucedeu, pois, que, acabando Salomão de fazer ao Senhor esta
oração e esta súplica, estando de joelhos e com as mãos estendidas
para os céus, se levantou de diante do altar do Senhor. E
pôs-se em pé, e abençoou a toda a congregação de Israel em alta voz,
dizendo: Bendito seja o Senhor, que deu repouso ao seu povo
Israel, segundo tudo o que disse; nem uma só palavra caiu de todas
as suas boas palavras que falou pelo ministério de Moisés, seu
servo. O Senhor nosso Deus seja conosco, como foi com nossos
pais; não nos desampare, e não nos deixe. Inclinando a si o
nosso coração, para andar em todos os seus caminhos, e para guardar
os seus mandamentos, e os seus estatutos, e os seus juízos que
ordenou a nossos pais. E que estas minhas palavras, com que
supliquei perante o Senhor, estejam perto, diante do Senhor nosso
Deus, de dia e de noite, para que execute o juízo do seu servo e o
juízo do seu povo Israel, a cada qual no seu dia. Para que
todos os povos da terra saibam que o Senhor é Deus, e que não há
outro. E seja o vosso coração inteiro para com o Senhor nosso
Deus, para andardes nos seus estatutos, e guardardes os seus
mandamentos como hoje.

E o rei e todo o Israel com ele ofereceram sacrifícios perante a
face do Senhor. E deu Salomão para o sacrifício pacífico que
ofereceu ao Senhor, vinte e duas mil vacas e cento e vinte mil
ovelhas; assim o rei e todos os filhos de Israel consagraram a casa
do Senhor. No mesmo dia santificou o rei o meio do átrio que
estava diante da casa do Senhor; porquanto ali preparara os
holocaustos e as ofertas com a gordura dos sacrifícios pacíficos;
porque o altar de cobre que estava diante da face do Senhor era
muito pequeno para nele caberem os holocaustos e as ofertas, e a
gordura dos sacrifícios pacíficos. No mesmo tempo celebrou
Salomão a festa, e todo o Israel com ele, uma grande congregação,
desde a entrada de Hamate até ao rio do Egito, perante a face do
Senhor nosso Deus; por sete dias, e mais sete dias; catorze dias.
E no oitavo dia despediu o povo, e eles abençoaram o rei;
então se foram às suas tendas, alegres e felizes de coração, por
causa de todo o bem que o Senhor fizera a Davi seu servo, e a Israel
seu povo.

\medskip

\lettrine{9} Sucedeu, pois, que, acabando Salomão de edificar
a casa do Senhor, e a casa do rei, e todo o desejo de Salomão, que
lhe veio à vontade fazer, o Senhor tornou a aparecer a Salomão;
como lhe tinha aparecido em Gibeom. E o Senhor lhe disse: Ouvi a
tua oração, e a súplica que fizeste perante mim; santifiquei a casa
que edificaste, a fim de pôr ali o meu nome para sempre; e os meus
olhos e o meu coração estarão ali todos os dias. E se tu andares
perante mim como andou Davi, teu pai, com inteireza de coração e com
sinceridade, para fazeres segundo tudo o que te mandei, e guardares
os meus estatutos e os meus juízos, então confirmarei o trono de
teu reino sobre Israel para sempre; como falei acerca de teu pai
Davi, dizendo: Não te faltará sucessor sobre o trono de Israel.
Porém, se vós e vossos filhos de qualquer maneira vos apartardes
de mim, e não guardardes os meus mandamentos, e os meus estatutos,
que vos tenho proposto, mas fordes, e servirdes a outros deuses, e
vos prostrardes perante eles, então destruirei a Israel da terra
que lhes dei; e a esta casa, que santifiquei a meu nome, lançarei
longe da minha presença; e Israel será por provérbio e
motejo\footnote{Zombaria. Dito picante; gracejo.}, entre todos os
povos. E desta casa, que é tão exaltada, todo aquele que por ela
passar pasmará, e assobiará, e dirá: Por que fez o Senhor assim a
esta terra e a esta casa? E dirão: Porque deixaram ao Senhor seu
Deus, que tirou da terra do Egito a seus pais, e se apegaram a
deuses alheios, e se encurvaram perante eles, e os serviram; por
isso trouxe o Senhor sobre eles todo este mal.

E sucedeu, ao fim de vinte anos, que Salomão edificara as duas
casas; a casa do Senhor e a casa do rei (para o que Hirão,
rei de Tiro, trouxera a Salomão madeira de cedro e de cipreste, e
ouro, segundo todo o seu desejo); então deu o rei Salomão a Hirão
vinte cidades na terra da Galiléia. E saiu Hirão de Tiro a
ver as cidades que Salomão lhe dera, porém não foram boas aos seus
olhos. Por isso disse: Que cidades são estas que me deste,
irmão meu? E chamaram-nas: Terra de Cabul, até hoje. E
enviara Hirão ao rei cento e vinte talentos\footnote{Talento: cerca
de 34 quilos.} de ouro.

E esta é a causa do tributo que impôs o rei Salomão, para
edificar a casa do Senhor e a sua casa, e Milo, e o muro de
Jerusalém, como também a Hasor, e a Megido, e a Gezer. Porque
Faraó, rei do Egito, subiu e tomou a Gezer, e a queimou a fogo, e
matou os cananeus que moravam na cidade, e a deu em dote à sua
filha, mulher de Salomão. Assim edificou Salomão a Gezer, e
Bete-Horom, a baixa, e a Baalate, e a Tadmor, no deserto
daquela terra, e a todas as cidades de provisões que Salomão
tinha, e as cidades dos carros, e as cidades dos cavaleiros, e tudo
o que Salomão quis edificar em Jerusalém, e no Líbano, e em toda a
terra do seu domínio. Quanto a todo o povo que restou dos
amorreus, heteus, perizeus, heveus, e jebuseus, e que não eram dos
filhos de Israel, a seus filhos, que restaram depois deles na
terra, os quais os filhos de Israel não puderam destruir totalmente,
Salomão os reduziu a tributo servil, até hoje. Porém dos
filhos de Israel não fez Salomão servo algum; porém eram homens de
guerra, e seus criados, e seus príncipes, e seus capitães, e chefes
dos seus carros e dos seus cavaleiros. Estes eram os chefes
dos oficiais que estavam sobre a obra de Salomão, quinhentos e
cinqüenta, que davam ordens ao povo que trabalhava na obra.
Subiu, porém, a filha de Faraó da cidade de Davi, à sua casa,
que Salomão lhe edificara; então edificou a Milo. E oferecia
Salomão três vezes cada ano holocaustos e sacrifícios pacíficos
sobre o altar que edificaram ao Senhor, e queimava incenso sobre o
que estava perante o Senhor; e assim acabou a casa. Também o
rei Salomão fez naus em Eziom-Geber, que está junto a Elate, à praia
do mar de Sufe, na terra de Edom. E mandou Hirão com aquelas
naus a seus servos, marinheiros, que sabiam do mar, com os servos de
Salomão. E vieram a Ofir, e tomaram de lá quatrocentos e
vinte talentos de ouro, e os trouxeram ao rei Salomão.

\medskip

\lettrine{10} E ouvindo a rainha de Sabá a fama de Salomão,
acerca do nome do Senhor, veio prová-lo com questões difíceis. E
chegou a Jerusalém com uma grande comitiva; com camelos carregados
de especiarias, e muitíssimo ouro, e pedras preciosas; e foi a
Salomão, e disse-lhe tudo quanto tinha no seu coração. E Salomão
lhe deu resposta a todas as suas perguntas, nada houve que não lhe
pudesse esclarecer. Vendo, pois, a rainha de Sabá toda a
sabedoria de Salomão, e a casa que edificara, e a comida da sua
mesa, e o assentar de seus servos, e o estar de seus criados, e as
vestes deles, e os seus copeiros, e os holocaustos que ele oferecia
na casa do Senhor, ficou fora de si. E disse ao rei: Era verdade
a palavra que ouvi na minha terra, dos teus feitos e da tua
sabedoria. E eu não cria naquelas palavras, até que vim e os
meus olhos o viram; eis que não me disseram metade; sobrepujaste em
sabedoria e bens a fama que ouvi. Bem-aventurados os teus
homens, bem-aventurados estes teus servos, que estão sempre diante
de ti, que ouvem a tua sabedoria! Bendito seja o Senhor teu
Deus, que teve agrado em ti, para te pôr no trono de Israel; porque
o Senhor ama a Israel para sempre, por isso te estabeleceu rei, para
fazeres juízo e justiça. E deu ao rei cento e vinte talentos
de ouro, e muitíssimas especiarias, e pedras preciosas; nunca veio
especiaria em tanta abundância, como a que a rainha de Sabá deu ao
rei Salomão. Também as naus de Hirão, que de Ofir levavam
ouro, traziam de Ofir muita madeira de almugue, e pedras preciosas.
E desta madeira de almugue fez o rei
balaústres\footnote{Colunelo de madeira, pedra ou metal, que
sustenta, junto com outros iguais, regularmente distribuídos, uma
travessa, corrimão ou peitoril. Parte lateral da voluta (ornato
espiralado de um capitel de coluna) do capitel jônico. Haste
vertical de madeira ou de metal, presa no chão e no teto junto às
portas ou outros pontos de acesso de veículos coletivos, para
auxiliar o passageiro no embarque e desembarque. Cada uma das peças
torneadas que formam o espaldar de cadeira ou a cabeceira de cama.}
para a casa do Senhor, e para a casa do rei, como também harpas e
alaúdes para os cantores; nunca veio tal madeira de almugue, nem se
viu até o dia de hoje. E o rei Salomão deu à rainha de Sabá
tudo o que ela desejou, tudo quanto pediu, além do que dera por sua
generosidade; então voltou e partiu para a sua terra, ela e os seus
servos.

E o peso do ouro que se trazia a Salomão cada ano era de
seiscentos e sessenta e seis talentos de ouro; além do que
entrava dos negociantes, e do contrato dos especieiros, e de todos
os reis da Arábia, e dos governadores da mesma terra. Também
o rei Salomão fez duzentos paveses\footnote{Pavês: escudo grande.}
de ouro batido; seiscentos siclos de ouro destinou para cada pavês;
fez também trezentos escudos de ouro batido; três
arráteis\footnote{Arrátel: Antiga unidade de medida de peso,
equivalente a 459g ou 16 onças; libra.} de ouro destinou para cada
escudo; e o rei os pôs na casa do bosque do Líbano. Fez mais
o rei um grande trono de marfim, e o revestiu de ouro puríssimo.
Tinha este trono seis degraus, e era o alto do trono por
detrás redondo, e de ambos os lados tinha encostos até ao assento; e
dois leões, em pé, juntos aos encostos. Também doze leões
estavam ali sobre os seis degraus de ambos os lados; nunca se tinha
feito obra semelhante em nenhum dos reinos. Também todas as
taças de beber do rei Salomão eram de ouro, e todos os vasos da casa
do bosque do Líbano eram de ouro puro; não havia neles prata, porque
nos dias de Salomão não tinha valor algum. Porque o rei tinha
no mar as naus de Társis, com as naus de Hirão; uma vez em três anos
tornavam as naus de Társis, e traziam ouro e prata, marfim, e
bugios\footnote{Macacos.}, e pavões. Assim o rei Salomão
excedeu a todos os reis da terra, tanto em riquezas como em
sabedoria. E toda a terra buscava a face de Salomão, para
ouvir a sabedoria que Deus tinha posto no seu coração. E cada
um trazia o seu presente, vasos de prata e vasos de ouro, e roupas,
e armaduras, e especiarias, cavalos e mulas; isso faziam de ano em
ano. Também ajuntou Salomão carros e cavaleiros, de sorte que
tinha mil e quatrocentos carros e doze mil cavaleiros; e os levou às
cidades dos carros, e junto ao rei em Jerusalém. E fez o rei
que em Jerusalém houvesse prata como pedras; e cedros em abundância
como sicômoros\footnote{Falso-plátano: Árvore grande, ornamental, da
família das aceráceas, originária da Europa, dotada de flores com
propriedades melíferas, dispostas em cachos compridos, pedunculados,
racemosos e vilosos, e cujo fruto é sâmara dupla, com várias
sementes revestidas de arilo.} que estão nas planícies. E
traziam do Egito, para Salomão, cavalos e fio de linho; e os
mercadores do rei recebiam o fio de linho, por um certo preço.
E subia e saía um carro do Egito por seiscentos siclos de
prata, e um cavalo por cento e cinqüenta; e assim, por meio deles,
eram exportados para todos os reis dos heteus e para os reis da
Síria.

\medskip

\lettrine{11} E o rei Salomão amou muitas mulheres
estrangeiras, além da filha de Faraó: moabitas, amonitas, edomitas,
sidônias e hetéias, das nações de que o Senhor tinha falado aos
filhos de Israel: Não chegareis a elas, e elas não chegarão a vós;
de outra maneira perverterão o vosso coração para seguirdes os seus
deuses. A estas se uniu Salomão com amor. E tinha setecentas
mulheres, princesas, e trezentas concubinas; e suas mulheres lhe
perverteram o coração. Porque sucedeu que, no tempo da velhice
de Salomão, suas mulheres lhe perverteram o coração para seguir
outros deuses; e o seu coração não era perfeito para com o Senhor
seu Deus, como o coração de Davi, seu pai, porque Salomão seguiu
a Astarote, deusa dos sidônios, e Milcom, a abominação dos amonitas.
Assim fez Salomão o que parecia mal aos olhos do Senhor; e não
perseverou em seguir ao Senhor, como Davi, seu pai. Então
edificou Salomão um alto a Quemós, a abominação dos moabitas, sobre
o monte que está diante de Jerusalém, e a Moloque, a abominação dos
filhos de Amom. E assim fez para com todas as suas mulheres
estrangeiras; as quais queimavam incenso e sacrificavam a seus
deuses.

Pelo que o Senhor se indignou contra Salomão; porquanto desviara o
seu coração do Senhor Deus de Israel, o qual duas vezes lhe
aparecera. E acerca deste assunto lhe tinha dado ordem que
não seguisse a outros deuses; porém não guardou o que o Senhor lhe
ordenara. Assim disse o Senhor a Salomão: Pois que houve isto
em ti, que não guardaste a minha aliança e os meus estatutos que te
mandei, certamente rasgarei de ti este reino, e o darei a teu servo.
Todavia nos teus dias não o farei, por amor de Davi, teu pai;
da mão de teu filho o rasgarei; porém todo o reino não
rasgarei; uma tribo darei a teu filho, por amor de meu servo Davi, e
por amor a Jerusalém, que tenho escolhido.

Levantou, pois, o Senhor contra Salomão um adversário, Hadade, o
edomeu; ele era da descendência do rei em Edom. Porque
sucedeu que, estando Davi em Edom, e subindo Joabe, o capitão do
exército, a enterrar os mortos, feriu a todo o homem em
Edom16(porque Joabe ficou ali seis meses com todo o Israel, até que
destruiu a todo o homem em Edom). Hadade, porém, fugiu, ele e
alguns homens edomeus, dos servos de seu pai, com ele, para ir ao
Egito; era, porém, Hadade muito jovem. E levantaram-se de
Midiã, e foram a Parã, e tomaram consigo homens de Parã, e foram ao
Egito ter com Faraó, rei do Egito, o qual lhe deu uma casa, e lhe
prometeu sustento, e lhe deu uma terra. E achou Hadade grande
graça diante de Faraó, de maneira que lhe deu por mulher a irmã de
sua mulher, a irmã de Tafnes, a rainha. E a irmã de Tafnes
deu-lhe um filho, Genubate, o qual Tafnes criou na casa de Faraó; e
Genubate estava na casa de Faraó, entre os filhos de Faraó.
Ouvindo, pois, Hadade, no Egito, que Davi adormecera com seus
pais, e que Joabe, capitão do exército, era morto, disse Hadade a
Faraó: Despede-me, para que vá à minha terra. Porém Faraó lhe
disse: Pois que te falta comigo, que procuras partir para a tua
terra? E disse ele: Nada, mas todavia despede-me. Também Deus
lhe levantou outro adversário, a Rezom, filho de Eliada, que tinha
fugido de seu senhor Hadadezer, rei de Zoba, contra quem
também ajuntou homens, e foi capitão de um esquadrão, quando Davi os
matou; e, indo-se para Damasco, habitaram ali, e reinaram em
Damasco. E foi adversário de Israel, por todos os dias de
Salomão, e isto além do mal que Hadade fazia; porque detestava a
Israel, e reinava sobre a Síria.

Até Jeroboão, filho de Nebate, efrateu, de Zereda, servo de
Salomão (cuja mãe era mulher viúva, por nome Zerua), também levantou
a mão contra o rei. E esta foi a causa por que levantou a mão
contra o rei: Salomão tinha edificado a Milo, e cerrou as aberturas
da cidade de Davi, seu pai. E o homem Jeroboão era forte e
valente; e vendo Salomão a este jovem, que era laborioso, ele o pôs
sobre todo o cargo da casa de José. Sucedeu, pois, naquele
tempo que, saindo Jeroboão de Jerusalém, o profeta Aías, o silonita,
o encontrou no caminho, e ele estava vestido com uma roupa nova, e
os dois estavam sós no campo. E Aías pegou na roupa nova que
tinha sobre si, e a rasgou em doze pedaços. E disse a
Jeroboão: Toma para ti os dez pedaços, porque assim diz o Senhor
Deus de Israel: Eis que rasgarei o reino da mão de Salomão, e a ti
darei as dez tribos. Porém ele terá uma tribo, por amor de
Davi, meu servo, e por amor de Jerusalém, a cidade que escolhi de
todas as tribos de Israel. Porque me deixaram, e se
encurvaram a Astarote, deusa dos sidônios, a Quemós, deus dos
moabitas, e a Milcom, deus dos filhos de Amom; e não andaram pelos
meus caminhos, para fazerem o que é reto aos meus olhos, a saber, os
meus estatutos e os meus juízos, como Davi, seu pai. Porém
não tomarei nada deste reino da sua mão; mas por príncipe o ponho
por todos os dias da sua vida, por amor de Davi, meu servo, a quem
escolhi, o qual guardou os meus mandamentos e os meus estatutos.
Mas da mão de seu filho tomarei o reino, e darei a ti, as dez
tribos dele. E a seu filho darei uma tribo; para que Davi,
meu servo, sempre tenha uma lâmpada diante de mim em Jerusalém, a
cidade que escolhi para pôr ali o meu nome. E te tomarei, e
reinarás sobre tudo o que desejar a tua alma; e serás rei sobre
Israel. E há de ser que, se ouvires tudo o que eu te mandar,
e andares pelos meus caminhos, e fizeres o que é reto aos meus
olhos, guardando os meus estatutos e os meus mandamentos, como fez
Davi, meu servo, eu serei contigo, e te edificarei uma casa firme,
como edifiquei a Davi, e te darei Israel. E por isso
afligirei a descendência de Davi; todavia não para sempre.
Assim Salomão procurou matar Jeroboão; porém Jeroboão se
levantou, e fugiu para o Egito, a ter com Sisaque, rei do Egito; e
esteve no Egito até que Salomão morreu.

Quanto ao mais dos atos de Salomão, e a tudo quanto fez, e à sua
sabedoria, porventura não está escrito no livro dos feitos de
Salomão? E o tempo que reinou Salomão, em Jerusalém, sobre
todo o Israel foi quarenta anos. E adormeceu Salomão com seus
pais, e foi sepultado na cidade de Davi, seu pai; e Roboão, seu
filho, reinou em seu lugar.

\medskip

\lettrine{12} E foi Roboão para Siquém; porque todo o Israel
se reuniu em Siquém, para o fazerem rei. Sucedeu que, Jeroboão,
filho de Nebate, achando-se ainda no Egito, para onde fugira de
diante do rei Salomão, voltou do Egito, porque mandaram
chamá-lo; veio, pois, Jeroboão e toda a congregação de Israel, e
falaram a Roboão, dizendo: Teu pai agravou o nosso jugo; agora,
pois, alivia tu a dura servidão de teu pai, e o pesado jugo que nos
impôs, e nós te serviremos. E ele lhes disse: Ide-vos até ao
terceiro dia, e então voltai a mim. E o povo se foi. E teve o
rei Roboão conselho com os anciãos que estiveram na presença de
Salomão, seu pai, quando este ainda vivia, dizendo: Como aconselhais
vós que se responda a este povo? E eles lhe falaram, dizendo: Se
hoje fores servo deste povo, e o servires, e respondendo-lhe, lhe
falares boas palavras, todos os dias serão teus servos. Porém
ele deixou o conselho que os anciãos lhe tinham dado, e teve
conselho com os jovens que haviam crescido com ele, que estavam
diante dele. E disse-lhes: Que aconselhais vós que respondamos a
este povo, que me falou, dizendo: Alivia o jugo que teu pai nos
impôs? E os jovens que haviam crescido com ele lhe falaram:
Assim dirás a este povo que te falou: Teu pai fez pesadíssimo o
nosso jugo, mas tu o alivia de sobre nós; assim lhe falarás: Meu
dedo mínimo é mais grosso do que os lombos de meu pai. Assim
que, se meu pai vos carregou de um jugo pesado, ainda eu aumentarei
o vosso jugo; meu pai vos castigou com açoites, porém eu vos
castigarei com escorpiões. Veio, pois, Jeroboão e todo o
povo, ao terceiro dia, a Roboão, como o rei havia ordenado, dizendo:
Voltai a mim ao terceiro dia. E o rei respondeu ao povo
duramente; porque deixara o conselho que os anciãos lhe haviam dado.
E lhe falou conforme ao conselho dos jovens, dizendo: Meu pai
agravou o vosso jugo, porém eu ainda aumentarei o vosso jugo; meu
pai vos castigou com açoites, porém eu vos castigarei com
escorpiões. O rei, pois, não deu ouvidos ao povo; porque esta
revolta vinha do Senhor, para confirmar a palavra que o Senhor tinha
falado pelo ministério de Aías, o silonita, a Jeroboão, filho de
Nebate.

Vendo, pois, todo o Israel que o rei não lhe dava ouvidos,
tornou-lhe o povo a responder, dizendo: Que parte temos nós com
Davi? Não há para nós herança no filho de Jessé. Às tuas tendas, ó
Israel! Provê agora a tua casa, ó Davi. Então Israel se foi às suas
tendas. No tocante, porém, aos filhos de Israel que habitavam
nas cidades de Judá, também sobre eles reinou Roboão. Então o
rei Roboão enviou a Adorão, que estava sobre os tributos; e todo o
Israel o apedrejou, e ele morreu; mas o rei Roboão se animou a subir
ao carro para fugir para Jerusalém. Assim se rebelaram os
israelitas contra a casa de Davi, até ao dia de hoje. E
sucedeu que, ouvindo todo o Israel que Jeroboão tinha voltado,
enviaram, e o chamaram para a congregação, e o fizeram rei sobre
todo o Israel; e ninguém seguiu a casa de Davi senão somente a tribo
de Judá. Vindo, pois, Roboão a Jerusalém, reuniu toda a casa
de Judá e a tribo de Benjamim, cento e oitenta mil escolhidos,
destros para a guerra, para pelejar contra a casa de Israel, para
restituir o reino a Roboão, filho de Salomão. Porém veio a
palavra de Deus a Semaías, homem de Deus, dizendo: Fala a
Roboão, filho de Salomão, rei de Judá, e a toda a casa de Judá, e a
Benjamim, e ao restante do povo, dizendo: Assim diz o Senhor:
Não subireis nem pelejareis contra vossos irmãos, os filhos de
Israel; volte cada um para a sua casa, porque eu é que fiz esta
obra. E ouviram a palavra do Senhor, e voltaram segundo a palavra do
Senhor.

E Jeroboão edificou a Siquém, no monte de Efraim, e habitou ali;
e saiu dali, e edificou a Penuel. E disse Jeroboão no seu
coração: Agora tornará o reino à casa de Davi. Se este povo
subir para fazer sacrifícios na casa do Senhor, em Jerusalém, o
coração deste povo se tornará a seu Senhor, a Roboão, rei de Judá; e
me matarão, e tornarão a Roboão, rei de Judá. Assim o rei
tomou conselho, e fez dois bezerros de ouro; e lhes disse: Muito
trabalho vos será o subir a Jerusalém; vês aqui teus deuses, ó
Israel, que te fizeram subir da terra do Egito. E pôs um em
Betel, e colocou o outro em Dã. E este feito se tornou em
pecado; pois que o povo ia até Dã para adorar o bezerro.
Também fez casa nos altos; e constituiu sacerdotes dos mais
baixos do povo, que não eram dos filhos de Levi. E fez
Jeroboão uma festa no oitavo mês, no dia décimo quinto do mês, como
a festa que se fazia em Judá, e sacrificou no altar; semelhantemente
fez em Betel, sacrificando aos bezerros que fizera; também em Betel
estabeleceu sacerdotes dos altos que fizera. E sacrificou no
altar que fizera em Betel, no dia décimo quinto do oitavo mês, que
ele tinha imaginado no seu coração; assim fez a festa aos filhos de
Israel, e sacrificou no altar, queimando incenso.

\medskip

\lettrine{13} E eis que, por ordem do Senhor, veio, de Judá a
Betel, um homem de Deus; e Jeroboão estava junto ao altar, para
queimar incenso. E ele clamou contra o altar por ordem do
Senhor, e disse: Altar, altar! Assim diz o Senhor: Eis que um filho
nascerá à casa de Davi, cujo nome será Josias, o qual sacrificará
sobre ti os sacerdotes dos altos que sobre ti queimam incenso, e
ossos de homens se queimarão sobre ti. E deu, naquele mesmo dia,
um sinal, dizendo: Este é o sinal de que o Senhor falou: Eis que o
altar se fenderá, e a cinza, que nele está, se derramará.
Sucedeu, pois, que, ouvindo o rei a palavra do homem de Deus,
que clamara contra o altar de Betel, Jeroboão estendeu a sua mão de
sobre o altar, dizendo: Pegai-o! Mas a sua mão, que estendera contra
ele, se secou, e não podia tornar a trazê-la a si. E o altar se
fendeu, e a cinza se derramou do altar, segundo o sinal que o homem
de Deus apontara por ordem do Senhor. Então respondeu o rei, e
disse ao homem de Deus: Suplica ao Senhor teu Deus, e roga por mim,
para que se me restitua a minha mão. Então o homem de Deus suplicou
ao Senhor, e a mão do rei se lhe restituiu, e ficou como dantes.
E o rei disse ao homem de Deus: Vem comigo para casa, e
conforta-te; e dar-te-ei um presente. Porém o homem de Deus
disse ao rei: Ainda que me desses metade da tua casa, não iria
contigo, nem comeria pão nem beberia água neste lugar. Porque
assim me ordenou o Senhor pela sua palavra, dizendo: Não comerás pão
nem beberás água; e não voltarás pelo caminho por onde vieste.
Assim foi por outro caminho; e não voltou pelo caminho, por
onde viera a Betel.

E morava em Betel um velho profeta; e vieram seus filhos, e
contaram-lhe tudo o que o homem de Deus fizera aquele dia em Betel,
e as palavras que dissera ao rei; e as contaram a seu pai. E
disse-lhes seu pai: Por que caminho se foi? E seus filhos lhe
mostraram o caminho por onde fora o homem de Deus que viera de Judá.
Então disse a seus filhos: Albardai-me\footnote{Albardar: Pôr
albarda (sela grosseira, enchumaçada de palha, para bestas de carga)
ou albardão em.} um jumento. E albardaram-lhe o jumento no qual ele
montou. E foi após o homem de Deus, e achou-o assentado
debaixo de um carvalho, e disse-lhe: És tu o homem de Deus que
vieste de Judá? E ele disse: Sou. Então lhe disse: Vem comigo
à casa, e come pão. Porém ele disse: Não posso voltar
contigo, nem entrarei contigo; nem tampouco comerei pão, nem beberei
contigo água neste lugar. Porque me foi mandado pela palavra
do Senhor: Ali não comerás pão, nem beberás água; nem voltarás pelo
caminho por onde vieste. E ele lhe disse: Também eu sou
profeta como tu, e um anjo me falou por ordem do Senhor, dizendo:
Faze-o voltar contigo à tua casa, para que coma pão e beba água
(porém mentiu-lhe). Assim voltou com ele, e comeu pão em sua
casa e bebeu água. E sucedeu que, estando eles à mesa, a
palavra do Senhor veio ao profeta que o tinha feito voltar. E
clamou ao homem de Deus, que viera de Judá, dizendo: Assim diz o
Senhor: Porquanto foste rebelde à ordem do Senhor, e não guardaste o
mandamento que o Senhor teu Deus te mandara, antes voltaste,
e comeste pão e bebeste água no lugar de que o Senhor te dissera:
Não comerás pão nem beberás água; o teu cadáver não entrará no
sepulcro de teus pais.

E sucedeu que, depois que comeu pão, e depois que bebeu, albardou
ele o jumento para o profeta que fizera voltar. Este, pois,
se foi, e um leão o encontrou no caminho, e o matou; e o seu cadáver
ficou estendido no caminho, e o jumento estava parado junto a ele, e
também o leão estava junto ao cadáver. E eis que alguns
homens passaram, e viram o corpo lançado no caminho, como também o
leão, que estava junto ao corpo; e foram, e o disseram na cidade
onde o velho profeta habitava. E, ouvindo-o o profeta que o
fizera voltar do caminho, disse: É o homem de Deus, que foi rebelde
à ordem do Senhor; por isso o Senhor o entregou ao leão, que o
despedaçou e matou, segundo a palavra que o Senhor lhe dissera.
Então disse a seus filhos: Albardai-me o jumento. Eles o
albardaram. Então foi, e achou o cadáver estendido no
caminho, e o jumento e o leão, que estavam parados junto ao cadáver;
e o leão não tinha devorado o corpo, nem tinha despedaçado o
jumento. Então o profeta levantou o cadáver do homem de Deus,
e pô-lo em cima do jumento levando-o consigo; assim veio o velho
profeta à cidade, para o chorar e enterrar. E colocou o
cadáver no seu próprio sepulcro; e prantearam-no, dizendo: Ah, irmão
meu! E sucedeu que, depois de o haver sepultado, disse a seus
filhos: Morrendo eu, sepultai-me no sepulcro em que o homem de Deus
está sepultado; ponde os meus ossos junto aos ossos dele.
Porque certamente se cumprirá o que pela palavra do Senhor
exclamou contra o altar que está em Betel, como também contra todas
as casas dos altos que estão nas cidades de Samaria. Nem
depois destas coisas deixou Jeroboão o seu mau caminho; antes, de
todo o povo, tornou a constituir sacerdotes dos lugares altos; e a
qualquer que queria consagrava sacerdote dos lugares altos. E
isso foi causa de pecado à casa de Jeroboão, para destruí-la e
extingui-la da terra.

\medskip

\lettrine{14} Naquele tempo adoeceu Abias, filho de Jeroboão.
E disse Jeroboão à sua mulher: Levanta-te agora, e disfarça-te,
para que não conheçam que és mulher de Jeroboão; e vai a Siló. Eis
que lá está o profeta Aías, o qual falou de mim, que eu seria rei
sobre este povo. E leva contigo dez pães, e bolos, e uma botija
de mel, e vai a ele; ele te declarará o que há de suceder a este
menino. E a mulher de Jeroboão assim fez, e se levantou, e foi a
Siló, e entrou na casa de Aías; e já Aías não podia ver, porque os
seus olhos estavam já escurecidos por causa da sua velhice.
Porém o Senhor disse a Aías: Eis que a mulher de Jeroboão vem
consultar-te sobre seu filho, porque está doente; assim e assim lhe
falarás; porque há de ser que, entrando ela, fingirá ser outra.
E sucedeu que, ouvindo Aías o ruído de seus pés, entrando ela
pela porta, disse-lhe ele: Entra, mulher de Jeroboão; por que te
disfarças assim? Pois eu sou enviado a ti com duras novas.

Vai, dize a Jeroboão: Assim diz o Senhor Deus de Israel: Porquanto
te levantei do meio do povo, e te pus por príncipe sobre o meu povo
de Israel, e rasguei o reino da casa de Davi, e o dei a ti, e tu
não foste como o meu servo Davi, que guardou os meus mandamentos e
que andou após mim com todo o seu coração para fazer somente o que
era reto aos meus olhos, antes tu fizeste o mal, pior do que
todos os que foram antes de ti; e foste, e fizeste outros deuses e
imagens de fundição, para provocar-me à ira, e me lançaste para trás
das tuas costas; portanto, eis que trarei mal sobre a casa de
Jeroboão; destruirei de Jeroboão todo o homem até ao menino, tanto o
escravo como o livre em Israel; e lançarei fora os descendentes da
casa de Jeroboão, como se lança fora o esterco, até que de todo se
acabe. Quem morrer dos de Jeroboão, na cidade, os cães o
comerão, e o que morrer no campo as aves do céu o comerão, porque o
Senhor o disse. Tu, pois, levanta-te, e vai para tua casa;
entrando os teus pés na cidade, o menino morrerá. E todo o
Israel o pranteará, e o sepultará; porque de Jeroboão só este
entrará em sepultura, porquanto se achou nele coisa boa para com o
Senhor Deus de Israel em casa de Jeroboão. O Senhor, porém,
levantará para si um rei sobre Israel, que destruirá a casa de
Jeroboão no mesmo dia. Que digo eu? Há de ser já. Também o
Senhor ferirá a Israel como se agita a cana nas águas; e arrancará a
Israel desta boa terra que tinha dado a seus pais, e o espalhará
para além do rio; porquanto fizeram os seus ídolos, provocando o
Senhor à ira. E entregará a Israel por causa dos pecados de
Jeroboão, o qual pecou, e fez pecar a Israel. Então a mulher
de Jeroboão se levantou, e foi, e chegou a Tirza; chegando ela ao
limiar da porta, morreu o menino. E o sepultaram, e todo o
Israel o pranteou, conforme a palavra do Senhor, a qual dissera pelo
ministério de seu servo Aías, o profeta. Quanto ao mais dos
atos de Jeroboão, como guerreou, e como reinou, eis que está escrito
no livro das crônicas dos reis de Israel. E foram os dias que
Jeroboão reinou vinte e dois anos; e dormiu com seus pais; e Nadabe,
seu filho, reinou em seu lugar.

E Roboão, filho de Salomão, reinava em Judá; de quarenta e um
anos de idade era Roboão quando começou a reinar, e dezessete anos
reinou em Jerusalém, na cidade que o Senhor escolhera de todas as
tribos de Israel para pôr ali o seu nome; e era o nome de sua mãe
Naamá, amonita. E fez Judá o que era mau aos olhos do Senhor;
e com os seus pecados que cometeram, provocaram-no a zelos, mais do
que todos os seus pais fizeram. Porque também eles edificaram
altos, e estátuas, e imagens de Asera sobre todo o alto outeiro e
debaixo de toda a árvore verde.
 Havia também sodomitas na terra; fizeram conforme a todas as
abominações dos povos que o Senhor tinha expulsado de diante dos
filhos de Israel. Ora, sucedeu que, no quinto ano do rei
Roboão, Sisaque, rei do Egito, subiu contra Jerusalém, e
tomou os tesouros da casa do Senhor e os tesouros da casa do rei; e
levou tudo. Também tomou todos os escudos de ouro que Salomão tinha
feito. E em lugar deles fez o rei Roboão escudos de cobre, e
os entregou nas mãos dos chefes da guarda que guardavam a porta da
casa do rei. E todas as vezes que o rei entrava na casa do
Senhor, os da guarda os levavam, e depois os tornavam à câmara da
guarda. Quanto ao mais dos atos de Roboão, e a tudo quanto
fez, porventura não está escrito no livro das crônicas dos reis de
Judá? E houve guerra entre Roboão e Jeroboão todos os seus
dias. E Roboão dormiu com seus pais, e foi sepultado com seus
pais na cidade de Davi; e era o nome de sua mãe Naamá, amonita; e
Abias, seu filho, reinou em seu lugar.

\medskip

\lettrine{15} E no décimo oitavo ano do rei Jeroboão, filho de
Nebate, Abias começou a reinar sobre Judá. E reinou três anos em
Jerusalém; e era o nome de sua mãe Maaca, filha de Absalão. E
andou em todos os pecados que seu pai tinha cometido antes dele; e
seu coração não foi perfeito para com o Senhor seu Deus como o
coração de Davi, seu pai. Mas por amor de Davi o Senhor seu Deus
lhe deu uma lâmpada em Jerusalém, levantando a seu filho depois
dele, e confirmando a Jerusalém. Porquanto Davi tinha feito o
que era reto aos olhos do Senhor, e não se tinha desviado de tudo
quanto lhe ordenara em todos os dias da sua vida, senão só no
negócio de Urias, o heteu. E houve guerra entre Roboão e
Jeroboão todos os dias da sua vida. Quanto ao mais dos atos de
Abias, e a tudo quanto fez, porventura não está escrito no livro das
crônicas dos reis de Judá? Também houve guerra entre Abias e
Jeroboão. E Abias dormiu com seus pais, e o sepultaram na cidade
de Davi; e Asa, seu filho, reinou em seu lugar.

E no vigésimo ano de Jeroboão, rei de Israel, começou Asa a reinar
em Judá. E quarenta e um anos reinou em Jerusalém; e era o
nome de sua mãe Maaca, filha de Absalão. E Asa fez o que era
reto aos olhos do Senhor, como Davi seu pai. Porque tirou da
terra os sodomitas, e removeu todos os ídolos que seus pais fizeram.
E até a Maaca, sua mãe, removeu para que não fosse rainha,
porquanto tinha feito um horrível ídolo a Asera; também Asa desfez o
seu ídolo horrível, e o queimou junto ao ribeiro de Cedrom.
Os altos, porém, não foram tirados; todavia foi o coração de
Asa reto para com o Senhor todos os seus dias. E à casa do
Senhor trouxe as coisas consagradas por seu pai, e as coisas que ele
mesmo consagrara; prata, ouro e vasos. E houve guerra entre
Asa e Baasa, rei de Israel, todos os seus dias. Porque Baasa,
rei de Israel, subiu contra Judá, e edificou a Ramá, para que a
ninguém fosse permitido sair, nem entrar a ter com Asa, rei de Judá.
Então Asa tomou toda a prata e ouro que ficaram nos tesouros
da casa do Senhor, e os tesouros da casa do rei, e os entregou nas
mãos de seus servos; e o rei Asa os enviou a Ben-Hadade, filho de
Tabrimom, filho de Heziom, rei da Síria, que habitava em Damasco,
dizendo: Haja acordo entre mim e ti, como houve entre meu pai
e teu pai; eis que te mando um presente, prata e ouro; vai, e anula
o teu acordo com Baasa, rei de Israel, para que se retire de sobre
mim. E Ben-Hadade deu ouvidos ao rei Asa, e enviou os
capitães dos seus exércitos contra as cidades de Israel; e feriu a
Ijom, e a Dã, e a Abel-Bete-Maaca, e a toda a Quinerete, com toda a
terra de Naftali. E sucedeu que, ouvindo-o Baasa, deixou de
edificar a Ramá; e ficou em Tirza. Então o rei Asa fez
apregoar por toda a Judá que todos, sem exceção, trouxessem as
pedras de Ramá, e a sua madeira com que Baasa edificara; e com elas
edificou o rei Asa a Geba de Benjamim e a Mizpá. Quanto ao
mais de todos os atos de Asa, e a todo o seu poder, e a tudo quanto
fez, e as cidades que edificou, porventura não está escrito no livro
das crônicas dos reis de Judá? Porém, no tempo da sua velhice,
padeceu dos pés. E Asa dormiu com seus pais, e foi sepultado
com seus pais na cidade de Davi, seu pai; e Jeosafá, seu filho,
reinou em seu lugar.

E Nadabe, filho de Jeroboão, começou a reinar sobre Israel no ano
segundo de Asa, rei de Judá; e reinou sobre Israel dois anos.
E fez o que era mau aos olhos do Senhor; e andou nos caminhos
de seu pai, e no seu pecado com que seu pai fizera pecar a Israel.
E conspirou contra ele Baasa, filho de Aías, da casa de
Issacar, e feriu-o Baasa em Gibetom, que era dos filisteus, quando
Nadabe e todo o Israel cercavam a Gibetom. E matou-o, pois,
Baasa no ano terceiro de Asa, rei de Judá, e reinou em seu lugar.
Sucedeu que, reinando ele, feriu a toda a casa de Jeroboão;
nada de Jeroboão deixou que tivesse fôlego, até o destruir, conforme
à palavra do Senhor que dissera pelo ministério de seu servo Aías, o
silonita. Por causa dos pecados que Jeroboão cometera, e fez
pecar a Israel, e por causa da provocação com que irritara ao Senhor
Deus de Israel. Quanto ao mais dos atos de Nadabe, e a tudo
quanto fez, porventura não está escrito no livro das crônicas dos
reis de Israel? E houve guerra entre Asa e Baasa, rei de
Israel, todos os seus dias. No ano terceiro de Asa, rei de
Judá, Baasa, filho de Aías, começou a reinar sobre todo o Israel em
Tirza, e reinou vinte e quatro anos. E fez o que era mau aos
olhos do Senhor; e andou no caminho de Jeroboão, e no pecado com que
ele tinha feito Israel pecar.

\medskip

\lettrine{16} Então veio a palavra do Senhor a Jeú, filho de
Hanani, contra Baasa, dizendo: Porquanto te levantei do pó, e te
pus por príncipe sobre o meu povo Israel, e tu tens andado no
caminho de Jeroboão, e tens feito pecar a meu povo Israel,
irritando-me com os seus pecados, eis que tirarei os
descendentes de Baasa, e os descendentes da sua casa, e farei a tua
casa como a casa de Jeroboão, filho de Nebate. Quem morrer dos
de Baasa, na cidade, os cães o comerão; e o que dele morrer no
campo, as aves do céu o comerão. Quanto ao mais dos atos de
Baasa, e ao que fez, e ao seu poder, porventura não está escrito no
livro das crônicas dos reis de Israel? E Baasa dormiu com seus
pais, e foi sepultado em Tirza; e Elá, seu filho, reinou em seu
lugar. Assim veio também a palavra do Senhor, pelo ministério do
profeta Jeú, filho de Hanani, contra Baasa e contra a sua casa; e
isso por todo o mal que fizera aos olhos do Senhor, irritando-o com
a obra de suas mãos, para ser como a casa de Jeroboão; e porque o
havia ferido. No ano vinte e seis de Asa, rei de Judá, Elá,
filho de Baasa, começou a reinar em Tirza sobre Israel; e reinou
dois anos. E Zinri, seu servo, capitão de metade dos carros,
conspirou contra ele, estando ele em Tirza, bebendo e embriagando-se
em casa de Arsa, mordomo em Tirza. Entrou, pois, Zinri, e o
feriu, e o matou, no ano vigésimo sétimo de Asa, rei de Judá; e
reinou em seu lugar. E sucedeu que, reinando ele, e estando
assentado no seu trono, feriu a toda a casa de Baasa; não lhe deixou
homem algum, nem a seus parentes, nem a seus amigos. Assim
destruiu Zinri toda a casa de Baasa, conforme à palavra do Senhor
que, contra Baasa, ele falara pelo ministério do profeta Jeú,
por todos os pecados de Baasa, e os pecados de Elá, seu
filho, que cometeram, e com que fizeram pecar a Israel, irritando ao
Senhor Deus de Israel com as suas vaidades. Quanto ao mais
dos atos de Elá, e a tudo quanto fez, não está escrito no livro das
crônicas dos reis de Israel?

No ano vigésimo sétimo de Asa, rei de Judá, reinou Zinri sete
dias em Tirza; e o povo estava acampado contra Gibetom, que era dos
filisteus. E o povo que estava acampado ouviu dizer: Zinri
tem conspirado, e até matou o rei. Todo o Israel pois, no mesmo dia,
no arraial, constituiu rei sobre Israel a Onri, capitão do exército.
E subiu Onri, e todo o Israel com ele, de Gibetom, e cercaram
a Tirza. E sucedeu que Zinri, vendo que a cidade era tomada,
foi ao paço da casa do rei e queimou-a sobre si; e morreu,
por causa dos pecados que cometera, fazendo o que era mau aos
olhos do Senhor, andando no caminho de Jeroboão, e no pecado que ele
cometera, fazendo Israel pecar. Quanto ao mais dos atos de
Zinri, e à conspiração que fez, porventura não está escrito no livro
das crônicas dos reis de Israel? Então o povo de Israel se
dividiu em dois partidos: metade do povo seguia a Tibni, filho de
Ginate, para o fazer rei, e a outra metade seguia a Onri. Mas
o povo que seguia a Onri foi mais forte do que o povo que seguia a
Tibni, filho de Ginate; e Tibni morreu, e Onri reinou. No ano
trinta e um de Asa, rei de Judá, Onri começou a reinar sobre Israel,
e reinou doze anos; e em Tirza reinou seis anos. E de Semer
comprou o monte de Samaria por dois talentos de prata, e edificou
nele; e chamou a cidade que edificou Samaria, do nome de Semer, dono
do monte. E fez Onri o que era mau aos olhos do Senhor; e fez
pior do que todos quantos foram antes dele. E andou em todos
os caminhos de Jeroboão, filho de Nebate, como também nos pecados
com que ele tinha feito pecar a Israel, irritando ao Senhor Deus de
Israel com as suas vaidades. Quanto ao mais dos atos de Onri,
ao que fez, e ao poder que manifestou, porventura não está escrito
no livro das crônicas dos reis de Israel? E Onri dormiu com
seus pais, e foi sepultado em Samaria; e Acabe, seu filho, reinou em
seu lugar.

E Acabe, filho de Onri, começou a reinar sobre Israel no ano
trigésimo oitavo de Asa, rei de Judá; e reinou Acabe, filho de Onri,
sobre Israel, em Samaria, vinte e dois anos. E fez Acabe,
filho de Onri, o que era mau aos olhos do Senhor, mais do que todos
os que foram antes dele.
 E sucedeu que (como se fora pouco andar nos pecados de Jeroboão,
filho de Nebate) ainda tomou por mulher a Jezabel, filha de Etbaal,
rei dos sidônios; e foi e serviu a Baal, e o adorou. E
levantou um altar a Baal, na casa de Baal que edificara em Samaria.
Também Acabe fez um ídolo; de modo que Acabe fez muito mais
para irritar ao Senhor Deus de Israel, do que todos os reis de
Israel que foram antes dele. Em seus dias Hiel, o betelita,
edificou a Jericó; em Abirão, seu primogênito, a fundou, e em
Segube, seu filho menor, pôs as suas portas; conforme a palavra do
Senhor, que falara pelo ministério de Josué, filho de Num.

\medskip

\lettrine{17} Então Elias, o tisbita, dos moradores de
Gileade, disse a Acabe: Vive o Senhor Deus de Israel, perante cuja
face estou, que nestes anos nem orvalho nem chuva haverá, senão
segundo a minha palavra. Depois veio a ele a palavra do Senhor,
dizendo: Retira-te daqui, e vai para o oriente, e esconde-te
junto ao ribeiro de Querite, que está diante do Jordão. E há de
ser que beberás do ribeiro; e eu tenho ordenado aos corvos que ali
te sustentem. Foi, pois, e fez conforme a palavra do Senhor;
porque foi, e habitou junto ao ribeiro de Querite, que está diante
do Jordão. E os corvos lhe traziam pão e carne pela manhã; como
também pão e carne à noite; e bebia do ribeiro. E sucedeu que,
passados dias, o ribeiro se secou, porque não tinha havido chuva na
terra. Então veio a ele a palavra do Senhor, dizendo:
Levanta-te, e vai para Sarepta, que é de Sidom, e habita ali;
eis que eu ordenei ali a uma mulher viúva que te sustente.
Então ele se levantou, e foi a Sarepta; e, chegando à porta
da cidade, eis que estava ali uma mulher viúva apanhando lenha; e
ele a chamou, e lhe disse: Traze-me, peço-te, num vaso um pouco de
água que beba. E, indo ela a trazê-la, ele a chamou e lhe
disse: Traze-me agora também um bocado de pão na tua mão.
Porém ela disse: Vive o Senhor teu Deus, que nem um bolo
tenho, senão somente um punhado de farinha numa panela, e um pouco
de azeite numa botija; e vês aqui apanhei dois
cavacos\footnote{Estilha ou lasca de madeira. Gravetos, em outras
versões.}, e vou prepará-lo para mim e para o meu filho, para que o
comamos, e morramos. E Elias lhe disse: Não temas; vai, faze
conforme à tua palavra; porém faze dele primeiro para mim um bolo
pequeno, e traze-mo aqui; depois farás para ti e para teu filho.
Porque assim diz o Senhor Deus de Israel: A farinha da panela
não se acabará, e o azeite da botija não faltará até ao dia em que o
Senhor dê chuva sobre a terra. E ela foi e fez conforme a
palavra de Elias; e assim comeu ela, e ele, e a sua casa muitos
dias. Da panela a farinha não se acabou, e da botija o azeite
não faltou; conforme a palavra do Senhor, que ele falara pelo
ministério de Elias.

E depois destas coisas sucedeu que adoeceu o filho desta mulher,
dona da casa; e a sua doença se agravou muito, até que nele nenhum
fôlego ficou. Então ela disse a Elias: Que tenho eu contigo,
homem de Deus? vieste tu a mim para trazeres à memória a minha
iniqüidade, e matares a meu filho? E ele disse: Dá-me o teu
filho. E ele o tomou do seu regaço, e o levou para cima, ao quarto,
onde ele mesmo habitava, e o deitou em sua cama, e clamou ao
Senhor, e disse: Ó Senhor meu Deus, também até a esta viúva, com
quem me hospedo, afligiste, matando-lhe o filho? Então se
estendeu sobre o menino três vezes, e clamou ao Senhor, e disse: Ó
Senhor meu Deus, rogo-te que a alma deste menino torne a entrar
nele. E o Senhor ouviu a voz de Elias; e a alma do menino
tornou a entrar nele, e reviveu. E Elias tomou o menino, e o
trouxe do quarto à casa, e o deu a sua mãe; e disse Elias: Vês aí,
teu filho vive. Então a mulher disse a Elias: Nisto conheço
agora que tu és homem de Deus, e que a palavra do Senhor na tua boca
é verdade.

\medskip

\lettrine{18} E sucedeu que, depois de muitos dias, a palavra
do Senhor veio a Elias, no terceiro ano, dizendo: Vai, apresenta-te
a Acabe; porque darei chuva sobre a terra. E foi Elias
apresentar-se a Acabe; e a fome era extrema em Samaria. E Acabe
chamou a Obadias, o mordomo; e Obadias temia muito ao Senhor,
porque sucedeu que, destruindo Jezabel os profetas do Senhor,
Obadias tomou cem profetas, e de cinqüenta em cinqüenta os escondeu
numa cova, e os sustentou com pão e água. E disse Acabe a
Obadias: Vai pela terra a todas as fontes de água, e a todos os
rios; pode ser que achemos erva, para que em vida conservemos os
cavalos e mulas, e não percamos todos os animais. E repartiram
entre si a terra, para a percorrerem: Acabe foi à parte por um
caminho, e Obadias também foi sozinho por outro caminho.
Estando, pois, Obadias já em caminho, eis que Elias o encontrou;
e Obadias, reconhecendo-o, prostrou-se sobre o seu rosto, e disse:
És tu o meu senhor Elias? E disse-lhe ele: Eu sou; vai, e dize a
teu senhor: Eis que Elias está aqui. Porém ele disse: Em que
pequei, para que entregues a teu servo na mão de Acabe, para que me
mate? Vive o Senhor teu Deus, que não houve nação nem reino
aonde o meu senhor não mandasse em busca de ti; e dizendo eles: Aqui
não está, então fazia jurar os reinos e nações, que não te haviam
achado. E agora dizes tu: Vai, dize a teu senhor: Eis que
aqui está Elias. E poderia ser que, apartando-me eu de ti, o
Espírito do Senhor te tomasse, não sei para onde, e, vindo eu a dar
as novas a Acabe, e não te achando ele, me mataria; porém eu, teu
servo, temo ao Senhor desde a minha mocidade. Porventura não
disseram a meu senhor o que fiz, quando Jezabel matava os profetas
do Senhor? Como escondi a cem homens dos profetas do Senhor, de
cinqüenta em cinqüenta, numa cova, e os sustentei com pão e água?
E agora dizes tu: Vai, dize a teu senhor: Eis que Elias está
aqui; ele me mataria. E disse Elias: Vive o Senhor dos
Exércitos, perante cuja face estou, que deveras hoje me apresentarei
a ele. Então foi Obadias encontrar-se com Acabe, e lho
anunciou; e foi Acabe encontrar-se com Elias.

E sucedeu que, vendo Acabe a Elias, disse-lhe: És tu o
perturbador de Israel? Então disse ele: Eu não tenho
perturbado a Israel, mas tu e a casa de teu pai, porque deixastes os
mandamentos do Senhor, e seguistes a Baalim. Agora, pois,
manda reunir-se a mim todo o Israel no monte Carmelo; como também os
quatrocentos e cinqüenta profetas de Baal, e os quatrocentos
profetas de Asera, que comem da mesa de Jezabel. Então Acabe
convocou todos os filhos de Israel; e reuniu os profetas no monte
Carmelo.

Então Elias se chegou a todo o povo, e disse: Até quando
coxeareis entre dois pensamentos? Se o Senhor é Deus, segui-o, e se
Baal, segui-o. Porém o povo nada lhe respondeu. Então disse
Elias ao povo: Só eu fiquei por profeta do Senhor, e os profetas de
Baal são quatrocentos e cinqüenta homens. Dêem-se-nos, pois,
dois bezerros, e eles escolham para si um dos bezerros, e o dividam
em pedaços, e o ponham sobre a lenha, porém não lhe coloquem fogo, e
eu prepararei o outro bezerro, e o porei sobre a lenha, e não lhe
colocarei fogo. Então invocai o nome do vosso deus, e eu
invocarei o nome do Senhor; e há de ser que o deus que responder por
meio de fogo esse será Deus. E todo o povo respondeu, dizendo: É boa
esta palavra. E disse Elias aos profetas de Baal: Escolhei
para vós um dos bezerros, e preparai-o primeiro, porque sois muitos,
e invocai o nome do vosso deus, e não lhe ponhais fogo. E
tomaram o bezerro que lhes dera, e o prepararam; e invocaram o nome
de Baal, desde a manhã até ao meio dia, dizendo: Ah! Baal,
responde-nos! Porém nem havia voz, nem quem respondesse; e saltavam
sobre o altar que tinham feito. E sucedeu que ao meio dia
Elias zombava deles e dizia: Clamai em altas vozes, porque ele é um
deus; pode ser que esteja falando, ou que tenha alguma coisa que
fazer, ou que intente alguma viagem; talvez esteja dormindo, e
despertará. E eles clamavam em altas vozes, e se retalhavam
com facas e com lancetas, conforme ao seu costume, até derramarem
sangue sobre si. E sucedeu que, passado o meio dia,
profetizaram eles, até a hora de se oferecer o sacrifício da tarde;
porém não houve voz, nem resposta, nem atenção alguma. Então
Elias disse a todo o povo: Chegai-vos a mim. E todo o povo se chegou
a ele; e restaurou o altar do Senhor, que estava quebrado. E
Elias tomou doze pedras, conforme ao número das tribos dos filhos de
Jacó, ao qual veio a palavra do Senhor, dizendo: Israel será o teu
nome. E com aquelas pedras edificou o altar em nome do
Senhor; depois fez um rego em redor do altar, segundo a largura de
duas medidas de semente. Então armou a lenha, e dividiu o
bezerro em pedaços, e o pôs sobre a lenha. E disse: Enchei de
água quatro cântaros, e derramai-a sobre o holocausto e sobre a
lenha. E disse: Fazei-o segunda vez; e o fizeram segunda vez. Disse
ainda: Fazei-o terceira vez; e o fizeram terceira vez; de
maneira que a água corria ao redor do altar; e até o rego ele encheu
de água. Sucedeu que, no momento de ser oferecido o
sacrifício da tarde, o profeta Elias se aproximou, e disse: Ó Senhor
Deus de Abraão, de Isaque e de Israel, manifeste-se hoje que tu és
Deus em Israel, e que eu sou teu servo, e que conforme à tua palavra
fiz todas estas coisas. Responde-me, Senhor, responde-me,
para que este povo conheça que tu és o Senhor Deus, e que tu fizeste
voltar o seu coração. Então caiu fogo do Senhor, e consumiu o
holocausto, e a lenha, e as pedras, e o pó, e ainda lambeu a água
que estava no rego. O que vendo todo o povo, caíram sobre os
seus rostos, e disseram: Só o Senhor é Deus! Só o Senhor é Deus!
E Elias lhes disse: Lançai mão dos profetas de Baal, que
nenhum deles escape. E lançaram mão deles; e Elias os fez descer ao
ribeiro de Quisom, e ali os matou.

Então disse Elias a Acabe: Sobe, come e bebe, porque há ruído de
uma abundante chuva. E Acabe subiu a comer e a beber; mas
Elias subiu ao cume do Carmelo, e se inclinou por terra, e pôs o seu
rosto entre os seus joelhos. E disse ao seu servo: Sobe
agora, e olha para o lado do mar. E subiu, e olhou, e disse: Não há
nada. Então disse ele: Volta lá sete vezes. E sucedeu que, à
sétima vez, disse: Eis aqui uma pequena nuvem, como a mão de um
homem, subindo do mar. Então disse ele: Sobe, e dize a Acabe:
Aparelha o teu carro, e desce, para que a chuva não te impeça.
E sucedeu que, entretanto, os céus se enegreceram com nuvens
e vento, e veio uma grande chuva; e Acabe subiu ao carro, e foi para
Jizreel. E a mão do Senhor estava sobre Elias, o qual cingiu
os lombos, e veio correndo perante Acabe, até à entrada de Jizreel.

\medskip

\lettrine{19} E Acabe fez saber a Jezabel tudo quanto Elias
havia feito, e como totalmente matara todos os profetas à espada.

Então Jezabel mandou um mensageiro a Elias, a dizer-lhe: Assim me
façam os deuses, e outro tanto, se de certo amanhã a estas horas não
puser a tua vida como a de um deles. O que vendo ele, se
levantou e, para escapar com vida, se foi, e chegando a Berseba, que
é de Judá, deixou ali o seu servo. Ele, porém, foi ao deserto,
caminho de um dia, e foi sentar-se debaixo de um
zimbro\footnote{Planta da família das pináceas (Juniperus communis),
cujos frutos se utilizam na preparação do gim ou da genebra e na
aromatização de conservas ou carnes defumadas; junípero.}; e pediu
para si a morte, e disse: Já basta, ó Senhor; toma agora a minha
vida, pois não sou melhor do que meus pais. E deitou-se, e
dormiu debaixo do zimbro; e eis que então um anjo o tocou, e lhe
disse: Levanta-te, come. E olhou, e eis que à sua cabeceira
estava um pão cozido sobre as brasas, e uma botija de água; e comeu,
e bebeu, e tornou a deitar-se. E o anjo do Senhor tornou segunda
vez, e o tocou, e disse: Levanta-te e come, porque te será muito
longo o caminho. Levantou-se, pois, e comeu e bebeu; e com a
força daquela comida caminhou quarenta dias e quarenta noites até
Horebe, o monte de Deus.

E ali entrou numa caverna e passou ali a noite; e eis que a
palavra do Senhor veio a ele, e lhe disse: Que fazes aqui Elias?
E ele disse: Tenho sido muito zeloso pelo Senhor Deus dos
Exércitos, porque os filhos de Israel deixaram a tua aliança,
derrubaram os teus altares, e mataram os teus profetas à espada, e
só eu fiquei, e buscam a minha vida para ma tirarem. E Deus
lhe disse: Sai\footnote{SBTB: Sai para fora --- pleonasmo. AV: And
he said, Go forth, and stand upon the mount before the LORD. RA:
Disse-lhe Deus: Sai e põe-te neste monte perante o SENHOR.} e põe-te
neste monte perante o Senhor. E eis que passava o Senhor, como
também um grande e forte vento que fendia os montes e quebrava as
penhas diante do Senhor; porém o Senhor não estava no vento; e
depois do vento um terremoto; também o Senhor não estava no
terremoto; e depois do terremoto um fogo; porém também o
Senhor não estava no fogo; e depois do fogo uma voz mansa e
delicada. E sucedeu que, ouvindo-a Elias, envolveu o seu
rosto na sua capa, e saiu\footnote{AV: And it was so, when Elijah
heard it, that he wrapped his face in his mantle, and went out, and
stood in the entering in of the cave. And, behold, there came a
voice unto him, and said, What doest thou here, Elijah? RA:
Ouvindo-o Elias, envolveu o rosto no seu manto e, saindo, pôs-se à
entrada da caverna. Eis que lhe veio uma voz e lhe disse: Que fazes
aqui, Elias?}, e pôs-se à entrada da caverna; e eis que veio a ele
uma voz, que dizia: Que fazes aqui, Elias? E ele disse: Eu
tenho sido em extremo zeloso pelo Senhor Deus dos Exércitos, porque
os filhos de Israel deixaram a tua aliança, derrubaram os teus
altares, e mataram os teus profetas à espada, e só eu fiquei; e
buscam a minha vida para ma tirarem. E o Senhor lhe disse:
Vai, volta pelo teu caminho para o deserto de Damasco; e, chegando
lá, unge a Hazael rei sobre a Síria. Também a Jeú, filho de
Ninsi, ungirás rei de Israel; e também a Eliseu, filho de Safate de
Abel-Meolá, ungirás profeta em teu lugar. E há de ser que o
que escapar da espada de Hazael, matá-lo-á Jeú; e o que escapar da
espada de Jeú, matá-lo-á Eliseu. Também deixei ficar em
Israel sete mil: todos os joelhos que não se dobraram a Baal, e toda
a boca que não o beijou.

Partiu, pois, Elias dali, e achou a Eliseu, filho de Safate, que
andava lavrando com doze juntas de bois adiante dele, e ele estava
com a duodécima; e Elias passou por ele, e lançou a sua capa sobre
ele. Então deixou ele os bois, e correu após Elias; e disse:
Deixa-me beijar a meu pai e a minha mãe, e então te seguirei. E ele
lhe disse: Vai, e volta; pois, que te fiz eu? Voltou, pois,
de o seguir, e tomou a junta de bois, e os matou, e com os aparelhos
dos bois cozeu as carnes, e as deu ao povo, e comeram; então se
levantou e seguiu a Elias, e o servia.

\medskip

\lettrine{20} E Ben-Hadade, rei da Síria, ajuntou todo o seu
exército; e havia com ele trinta e dois reis, e cavalos e carros; e
subiu, e cercou a Samaria, e pelejou contra ela. E enviou à
cidade mensageiros, a Acabe, rei de Israel, que lhe disseram:
Assim diz Ben-Hadade: A tua prata e o teu ouro são meus; e tuas
mulheres e os melhores de teus filhos são meus. E respondeu o
rei de Israel, e disse: Conforme a tua palavra, ó rei meu senhor, eu
sou teu, e tudo quanto tenho. E tornaram a vir os mensageiros, e
disseram: Assim diz Ben-Hadade: Enviei-te, na verdade, mensageiros
que dissessem: Tu me hás de dar a tua prata, e o teu ouro, e as tuas
mulheres e os teus filhos; todavia amanhã a estas horas enviarei
os meus servos a ti, e esquadrinharão a tua casa, e as casas dos
teus servos; e há de ser que tudo o que de precioso tiveres, eles
tomarão consigo, e o levarão. Então o rei de Israel chamou a
todos os anciãos da terra, e disse: Notai agora, e vede como este
homem procura o mal; pois mandou pedir-me as mulheres, os meus
filhos, a minha prata e o meu ouro, e não lhos neguei. E todos
os anciãos e todo o povo lhe disseram: Não lhe dês ouvidos, nem
consintas. Por isso disse aos mensageiros de Ben-Hadade: Dizei
ao rei, meu senhor: Tudo o que primeiro mandaste pedir a teu servo,
farei, porém isto não posso fazer. E voltaram os mensageiros, e lhe
levaram a resposta. E Ben-Hadade enviou a ele mensageiros
dizendo: Assim me façam os deuses, e outro tanto, que o pó de
Samaria não bastará para encher as mãos de todo o povo que me segue.
Porém o rei de Israel respondeu: Dizei-lhe: Não se gabe quem
se cinge das armas, como aquele que as descinge.

E sucedeu que, ouvindo ele esta palavra, estando a beber com os
reis nas tendas, disse aos seus servos: Ponde-vos em ordem contra a
cidade. E eis que um profeta se chegou a Acabe rei de Israel,
e lhe disse: Assim diz o Senhor: Viste toda esta grande multidão?
Eis que hoje ta entregarei nas tuas mãos, para que saibas que eu sou
o Senhor. E disse Acabe: Por quem? E ele disse: Assim diz o
Senhor: Pelos moços dos príncipes das províncias. E disse: Quem
começará a peleja? E disse: Tu. Então contou os moços dos
príncipes das províncias, e foram duzentos e trinta e dois; e depois
deles contou a todo o povo, todos os filhos de Israel, sete mil.
E saíram ao meio dia; e Ben-Hadade estava bebendo e
embriagando-se nas tendas, ele e os reis, os trinta e dois reis, que
o ajudavam. E os moços dos príncipes das províncias saíram
primeiro; e Ben-Hadade enviou espias, que lhe deram avisos, dizendo:
Saíram de Samaria uns homens. E ele disse: Ainda que para paz
saíssem, tomai-os vivos; e ainda que à peleja saíssem, tomai-os
vivos.
 Saíram, pois, da cidade os moços dos príncipes das províncias, e
o exército que os seguia. E eles feriram cada um o seu
adversário, e os sírios fugiram, e Israel os perseguiu; porém
Ben-Hadade, rei da Síria, escapou a cavalo, com alguns cavaleiros.
E saiu o rei de Israel, e feriu os cavalos e os carros; e
feriu os sírios com grande estrago.

Então o profeta chegou-se ao rei de Israel e lhe disse: Vai,
esforça-te, e atenta, e olha o que hás de fazer; porque no decurso
de um ano o rei da Síria subirá contra ti. Porque os servos
do rei da Síria lhe disseram: Seus deuses são deuses dos montes, por
isso foram mais fortes do que nós; mas pelejemos com eles em campo
raso, e por certo veremos, se não somos mais fortes do que eles!
Faze, pois, isto: tira os reis, cada um do seu lugar, e
substitui-os por capitães; e forma outro exército, igual ao
exército que perdeste, cavalo por cavalo, e carro por carro, e
pelejemos com eles em campo raso, e veremos se não somos mais fortes
do que eles! E deu ouvidos à sua voz, e assim fez. E sucedeu
que, passado um ano, Ben-Hadade passou revista aos sírios, e subiu a
Afeque, para pelejar contra Israel. Também aos filhos de
Israel se passou revista, e providos de víveres marcharam contra
eles; e os filhos de Israel acamparam-se defronte deles, como dois
pequenos rebanhos de cabras; mas os sírios enchiam a terra. E
chegou o homem de Deus, e falou ao rei de Israel, e disse: Assim diz
o Senhor: Porquanto os sírios disseram: O Senhor é Deus dos montes,
e não Deus dos vales; toda esta grande multidão entregarei nas tuas
mãos; para que saibas que eu sou o Senhor. E sete dias
estiveram acampados uns defronte dos outros; e sucedeu ao sétimo dia
que a peleja começou, e os filhos de Israel feriram dos sírios cem
mil homens de pé, num dia. E os restantes fugiram a Afeque, à
cidade; e caiu o muro sobre vinte e sete mil homens, que restaram;
Ben-Hadade, porém, fugiu, e veio à cidade, escondendo-se de câmara
em câmara.

Então lhe disseram os seus servos: Eis que já temos ouvido que os
reis da casa de Israel são reis clementes; ponhamos, pois, sacos aos
lombos, e cordas às cabeças, e saiamos ao rei de Israel; pode ser
que ele te poupe a vida. Então cingiram sacos aos lombos e
cordas às cabeças, e foram ao rei de Israel, e disseram: Diz o teu
servo Ben-Hadade: Deixa-me viver. E disse Acabe: Pois ainda vive? É
meu irmão. E aqueles homens tomaram isto por bom presságio, e
apressaram-se em apanhar a sua palavra, e disseram: Teu irmão
Ben-Hadade vive. E ele disse: Vinde, trazei-mo. Então Ben-Hadade foi
a ele, e ele o fez subir ao carro. E disse ele: As cidades
que meu pai tomou de teu pai tas restituirei, e faze para ti ruas em
Damasco, como meu pai as fez em Samaria. E eu, respondeu Acabe, te
deixarei ir com esta aliança. E fez com ele aliança e o deixou ir.
Então um dos homens dos filhos dos profetas disse ao seu
companheiro, pela palavra do Senhor: Ora fere-me. E o homem recusou
feri-lo. E ele lhe disse: Porque não obedeceste à voz do
Senhor, eis que, em te apartando de mim, um leão te ferirá. E como
dele se apartou, um leão o encontrou e o feriu. Depois
encontrou outro homem, e disse-lhe: Ora fere-me. E aquele homem
deu-lhe um golpe, ferindo-o. Então foi o profeta, e pôs-se
perante o rei no caminho; e disfarçou-se com cinza sobre os seus
olhos. E sucedeu que, passando o rei, clamou ele ao rei,
dizendo: Teu servo estava no meio da peleja, e eis que, desviando-se
um homem, trouxe-me outro homem, e disse: Guarda-me este homem; se
vier a faltar, será a tua vida em lugar da vida dele, ou pagarás um
talento de prata. Sucedeu, pois, que, estando o teu servo
ocupado de uma e de outra parte, eis que o homem desapareceu. Então
o rei de Israel lhe disse: Esta é a tua sentença; tu mesmo a
pronunciaste. Então ele se apressou, e tirou a cinza de sobre
os seus olhos; e o rei de Israel o reconheceu, que era um dos
profetas. E disse-lhe: Assim diz o Senhor: Porquanto soltaste
da mão o homem que eu havia posto para destruição, a tua vida será
em lugar da sua vida, e o teu povo em lugar do seu povo. E
foi o rei de Israel para a sua casa, desgostoso e indignado; e
chegou a Samaria.

\medskip

\lettrine{21} E sucedeu depois destas coisas que, Nabote, o
jizreelita, tinha uma vinha em Jizreel junto ao palácio de Acabe,
rei de Samaria. Então Acabe falou a Nabote, dizendo: Dá-me a tua
vinha, para que me sirva de horta, pois está vizinha ao lado da
minha casa; e te darei por ela outra vinha melhor; ou, se for do teu
agrado, dar-te-ei o seu valor em dinheiro. Porém Nabote disse a
Acabe: Guarde-me o Senhor de que eu te dê a herança de meus pais.
Então Acabe veio desgostoso e indignado à sua casa, por causa da
palavra que Nabote, o jizreelita, lhe falara, quando disse: Não te
darei a herança de meus pais. E deitou-se na sua cama, e voltou o
rosto, e não comeu pão.

Porém, vindo a ele Jezabel, sua mulher, lhe disse: Que há, que
está tão desgostoso o teu espírito, e não comes pão? E ele lhe
disse: Porque falei a Nabote, o jizreelita, e lhe disse: Dá-me a tua
vinha por dinheiro; ou, se te apraz, te darei outra vinha em seu
lugar. Porém ele disse: Não te darei a minha vinha. Então
Jezabel, sua mulher lhe disse: Governas tu agora no reino de Israel?
Levanta-te, come pão, e alegre-se o teu coração; eu te darei a vinha
de Nabote, o jizreelita. Então escreveu cartas em nome de Acabe,
e as selou com o seu sinete; e mandou as cartas aos anciãos e aos
nobres que havia na sua cidade e habitavam com Nabote. E
escreveu nas cartas, dizendo: Apregoai um jejum, e ponde Nabote
diante do povo. E ponde defronte dele dois filhos de Belial,
que testemunhem contra ele, dizendo: Blasfemaste contra Deus e
contra o rei; e trazei-o fora, e apedrejai-o para que morra.
E os homens da sua cidade, os anciãos e os nobres que
habitavam na sua cidade, fizeram como Jezabel lhes ordenara,
conforme estava escrito nas cartas que lhes mandara.
Apregoaram um jejum, e puseram a Nabote diante do povo.
Então vieram dois homens, filhos de Belial, e puseram-se
defronte dele; e os homens, filhos de Belial, testemunharam contra
ele, contra Nabote, perante o povo, dizendo: Nabote blasfemou contra
Deus e contra o rei. E o levaram para fora da cidade, e o
apedrejaram, e morreu. Então mandaram dizer a Jezabel: Nabote
foi apedrejado, e morreu. E sucedeu que, ouvindo Jezabel que
já fora apedrejado Nabote, e morrera, disse a Acabe: Levanta-te, e
possui a vinha de Nabote, o jizreelita, a qual te recusou dar por
dinheiro; porque Nabote não vive, mas é morto. E sucedeu que,
ouvindo Acabe, que Nabote já era morto, levantou-se para descer para
a vinha de Nabote, o jizreelita, para tomar posse dela.

Então veio a palavra do Senhor a Elias, o tisbita, dizendo:
Levanta-te, desce para encontrar-te com Acabe, rei de Israel,
que está em Samaria; eis que está na vinha de Nabote, aonde tem
descido para possuí-la. E falar-lhe-ás, dizendo: Assim diz o
Senhor: Porventura não mataste e tomaste a herança? Falar-lhe-ás
mais, dizendo: Assim diz o Senhor: No lugar em que os cães lamberam
o sangue de Nabote lamberão também o teu próprio sangue. E
disse Acabe a Elias: Já me achaste, inimigo meu? E ele disse:
Achei-te; porquanto já te vendeste para fazeres o que é mau aos
olhos do Senhor. Eis que trarei mal sobre ti, e arrancarei a
tua posteridade, e arrancarei de Acabe a todo o homem, tanto o
escravo como o livre em Israel; e farei a tua casa como a
casa de Jeroboão, filho de Nebate, e como a casa de Baasa, filho de
Aías; por causa da provocação, com que me provocaste e fizeste pecar
a Israel. E também acerca de Jezabel falou o Senhor, dizendo:
Os cães comerão a Jezabel junto ao antemuro de Jizreel.
Aquele que morrer dos de Acabe, na cidade, os cães o comerão;
e o que morrer no campo as aves do céu o comerão. Porém
ninguém fora como Acabe, que se vendera para fazer o que era mau aos
olhos do Senhor; porque Jezabel, sua mulher, o incitava. E
fez grandes abominações, seguindo os ídolos, conforme a tudo o que
fizeram os amorreus, os quais o Senhor lançou fora da sua possessão,
de diante dos filhos de Israel. Sucedeu, pois, que Acabe,
ouvindo estas palavras, rasgou as suas vestes, e cobriu a sua carne
de saco, e jejuou; e jazia em saco, e andava mansamente.
Então veio a palavra do Senhor a Elias tisbita, dizendo:
Não viste que Acabe se humilha perante mim? Por isso,
porquanto se humilha perante mim, não trarei este mal nos seus dias,
mas nos dias de seu filho o trarei sobre a sua casa.

\medskip

\lettrine{22} E estiveram quietos três anos, não havendo
guerra entre a Síria e Israel. Porém no terceiro ano sucedeu que
Jeosafá, rei de Judá, desceu para avistar-se com o rei de Israel.
E o rei de Israel disse aos seus servos: Não sabeis vós que
Ramote de Gileade é nossa, e nós estamos quietos, sem a tomar da mão
do rei da Síria? Então perguntou a Jeosafá: Irás tu comigo à
peleja a Ramote de Gileade? E disse Jeosafá ao rei de Israel: Serei
como tu, e o meu povo como o teu povo, e os meus cavalos como os
teus cavalos. Disse mais Jeosafá ao rei de Israel: Peço-te,
consulta hoje a palavra do Senhor. Então o rei de Israel reuniu
os profetas até quase quatrocentos homens, e disse-lhes: Irei à
peleja contra Ramote de Gileade, ou deixarei de ir? E eles disseram:
Sobe, porque o Senhor a entregará na mão do rei. Disse, porém,
Jeosafá: Não há aqui ainda algum profeta do Senhor, ao qual possamos
consultar? Então disse o rei de Israel a Jeosafá: Ainda há um
homem por quem podemos consultar ao Senhor; porém eu o odeio, porque
nunca profetiza de mim o que é bom, mas só o mal; este é Micaías,
filho de Inlá. E disse Jeosafá: Não fale o rei assim. Então o
rei de Israel chamou um oficial, e disse: Traze-me depressa a
Micaías, filho de Inlá. E o rei de Israel e Jeosafá, rei de
Judá, estavam assentados cada um no seu trono, vestidos de trajes
reais, na praça, à entrada da porta de Samaria; e todos os profetas
profetizavam na sua presença. E Zedequias, filho de Quenaaná,
fez para si uns chifres de ferro, e disse: Assim diz o Senhor: Com
estes ferirás aos sírios, até de todo os consumir. E todos os
profetas profetizaram assim, dizendo: Sobe a Ramote de Gileade, e
triunfarás, porque o Senhor a entregará na mão do rei. E o
mensageiro que foi chamar a Micaías falou-lhe, dizendo: Vês aqui que
as palavras dos profetas a uma voz predizem coisas boas para o rei;
seja, pois, a tua palavra como a palavra de um deles, e fala bem.
Porém Micaías disse: Vive o Senhor que o que o Senhor me
disser isso falarei.

E, vindo ele ao rei, o rei lhe disse: Micaías, iremos a Ramote de
Gileade à peleja, ou deixaremos de ir? E ele lhe disse: Sobe, e
serás bem sucedido; porque o Senhor a entregará na mão do rei.
E o rei lhe disse: Até quantas vezes te conjurarei, que não
me fales senão a verdade em nome do Senhor? Então disse ele:
Vi a todo o Israel disperso pelos montes, como ovelhas que não têm
pastor; e disse o Senhor: Estes não têm senhor; torne cada um em paz
para sua casa. Então o rei de Israel disse a Jeosafá: Não te
disse eu, que nunca profetizará de mim o que é bom, senão só o que é
mal? Então ele disse: Ouve, pois, a palavra do Senhor: Vi ao
Senhor assentado sobre o seu trono, e todo o exército do céu estava
junto a ele, à sua mão direita e à sua esquerda. E disse o
Senhor: Quem induzirá Acabe, para que suba, e caia em Ramote de
Gileade? E um dizia desta maneira e outro de outra. Então
saiu um espírito, e se apresentou diante do Senhor, e disse: Eu o
induzirei. E o Senhor lhe disse: Com quê? E disse ele: Eu
sairei, e serei um espírito de mentira na boca de todos os seus
profetas. E ele disse: Tu o induzirás, e ainda prevalecerás; sai e
faze assim. Agora, pois, eis que o Senhor pôs o espírito de
mentira na boca de todos estes teus profetas, e o Senhor falou o mal
contra ti. Então Zedequias, filho de Quenaaná, chegou, e
feriu a Micaías no queixo, e disse: Por onde saiu de mim o Espírito
do Senhor para falar a ti? E disse Micaías: Eis que o verás
naquele mesmo dia, quando entrares de câmara em câmara, para te
esconderes. Então disse o rei de Israel: Tomai a Micaías, e
tornai a levá-lo a Amom, o governador da cidade, e a Joás filho do
rei. E direis: Assim diz o rei: Colocai este homem na casa do
cárcere, e sustentai-o com o pão de angústia, e com água de
amargura, até que eu venha em paz. E disse Micaías: Se tu
voltares em paz, o Senhor não tem falado por mim. Disse mais: Ouvi,
povos todos!

Assim o rei de Israel e Jeosafá, rei de Judá, subiram a Ramote de
Gileade. E disse o rei de Israel a Jeosafá: Eu me
disfarçarei, e entrarei na peleja; tu porém veste as tuas roupas.
Disfarçou-se, pois, o rei de Israel, e entrou na peleja. E o
rei da Síria dera ordem aos capitães dos carros, que eram trinta e
dois, dizendo: Não pelejareis nem contra pequeno nem contra grande,
mas só contra o rei de Israel. Sucedeu que, vendo os capitães
dos carros a Jeosafá, disseram eles: Certamente este é o rei de
Israel. E chegaram-se a ele, para pelejar com ele; porém Jeosafá
gritou. E sucedeu que, vendo os capitães dos carros que não
era o rei de Israel, deixaram de segui-lo. Então um homem
armou o arco, e atirou a esmo, e feriu o rei de Israel por entre as
fivelas e as couraças; então ele disse ao seu carreteiro: Dá volta,
e tira-me do exército, porque estou gravemente ferido. E a
peleja foi crescendo naquele dia, e o rei foi sustentado no carro
defronte dos sírios; porém ele morreu à tarde; e o sangue da ferida
corria para o fundo do carro. E depois do sol posto passou um
pregão pelo exército, dizendo: Cada um para a sua cidade, e cada um
para a sua terra! E morreu o rei, e o levaram a Samaria; e
sepultaram o rei em Samaria. E, lavando-se o carro no tanque
de Samaria, os cães lamberam o seu sangue (ora as prostitutas se
lavavam ali), conforme à palavra que o Senhor tinha falado.
Quanto ao mais dos atos de Acabe, e a tudo quanto fez, e à
casa de marfim que edificou, e a todas as cidades que edificou,
porventura não está escrito no livro das crônicas dos reis de
Israel? Assim dormiu Acabe com seus pais; e Acazias, seu
filho, reinou em seu lugar.

E Jeosafá, filho de Asa, começou a reinar sobre Judá no quarto
ano de Acabe, rei de Israel. E era Jeosafá da idade de trinta
e cinco anos quando começou a reinar; e vinte e cinco anos reinou em
Jerusalém; e era o nome de sua mãe Azuba, filha de Sili. E
andou em todos os caminhos de seu pai Asa, não se desviou deles,
fazendo o que era reto aos olhos do Senhor. Todavia os altos
não se tiraram; ainda o povo sacrificava e queimava incenso nos
altos. E Jeosafá esteve em paz com o rei de Israel.
Quanto ao mais dos atos de Jeosafá, e ao poder que mostrou, e
como guerreou, porventura não está escrito no livro das crônicas dos
reis de Judá? Também expulsou da terra o restante dos
sodomitas, que ficaram nos dias de seu pai Asa. Então não
havia rei em Edom, porém um vice-rei. E fez Jeosafá navios de
Társis, para irem a Ofir por causa do ouro; porém não foram, porque
os navios se quebraram em Eziom-Geber. Então Acazias, filho
de Acabe, disse a Jeosafá: Vão os meus servos com os teus servos nos
navios. Porém Jeosafá não quis. E Jeosafá dormiu com seus
pais, e foi sepultado junto a eles, na cidade de Davi, seu pai; e
Jeorão, seu filho, reinou em seu lugar. E Acazias, filho de
Acabe, começou a reinar sobre Israel, em Samaria, no ano dezessete
de Jeosafá, rei de Judá; e reinou dois anos sobre Israel. E
fez o que era mau aos olhos do Senhor; porque andou no caminho de
seu pai, como também no caminho de sua mãe, e no caminho de
Jeroboão, filho de Nebate, que fez pecar a Israel. E serviu a
Baal, e adorou-o, e provocou a ira do Senhor Deus de Israel,
conforme a tudo quanto fizera seu pai.

