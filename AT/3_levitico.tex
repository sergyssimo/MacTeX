\addchap{Levítico}

\lettrine{1} E chamou o Senhor a Moisés, e falou com ele da
tenda da congregação, dizendo: Fala aos filhos de Israel, e
dize-lhes: Quando algum de vós oferecer oferta ao Senhor, oferecerá
a sua oferta de gado, isto é, de gado vacum\footnote{Diz-se do gado
constituído de vacas, bois e novilhos.} e de ovelha. Se a sua
oferta for holocausto de gado, oferecerá macho sem defeito; à porta
da tenda da congregação a oferecerá, de sua própria vontade, perante
o Senhor. E porá a sua mão sobre a cabeça do holocausto, para
que seja aceito a favor dele, para a sua expiação. Depois
degolará o bezerro perante o Senhor; e os filhos de Arão, os
sacerdotes, oferecerão o sangue, e espargirão o sangue em redor
sobre o altar que está diante da porta da tenda da congregação.
Então esfolará o holocausto, e o partirá nos seus pedaços. E
os filhos de Arão, o sacerdote, porão fogo sobre o altar, pondo em
ordem a lenha sobre o fogo. Também os filhos de Arão, os
sacerdotes, porão em ordem os pedaços, a cabeça e o redenho sobre a
lenha que está no fogo em cima do altar; porém a sua fressura e
as suas pernas lavar-se-ão com água; e o sacerdote tudo isso
queimará sobre o altar; holocausto é, oferta queimada, de cheiro
suave ao Senhor.

E se a sua oferta for de gado miúdo, de ovelhas ou de cabras,
para holocausto, oferecerá macho sem defeito. E o degolará ao
lado do altar que dá para o norte, perante o Senhor; e os filhos de
Arão, os sacerdotes, espargirão o seu sangue em redor sobre o altar.
Depois o partirá nos seus pedaços, como também a sua cabeça e
o seu redenho; e o sacerdote os porá em ordem sobre a lenha que está
no fogo sobre o altar; porém a fressura e as pernas
lavar-se-ão com água; e o sacerdote tudo oferecerá, e o queimará
sobre o altar; holocausto é, oferta queimada, de cheiro suave ao
Senhor. E se a sua oferta ao Senhor for holocausto de aves,
oferecerá a sua oferta de rolas ou de pombinhos; e o
sacerdote a oferecerá sobre o altar, e tirar-lhe-á a cabeça, e a
queimará sobre o altar; e o seu sangue será espremido na parede do
altar; e o seu papo com as suas penas tirará e o lançará
junto ao altar, para o lado do oriente, no lugar da cinza; e
fendê-la-á junto às suas asas, porém não a partirá; e o sacerdote a
queimará em cima do altar sobre a lenha que está no fogo; holocausto
é, oferta queimada de cheiro suave ao Senhor.

\medskip

\lettrine{2} E quando alguma pessoa oferecer oferta de
alimentos ao Senhor, a sua oferta será de flor de farinha, e nela
deitará azeite, e porá o incenso sobre ela; e a trará aos filhos
de Arão, os sacerdotes, um dos quais tomará dela um punhado da flor
de farinha, e do seu azeite com todo o seu incenso; e o sacerdote a
queimará como memorial sobre o altar; oferta queimada é, de cheiro
suave ao Senhor. E o que sobejar da oferta de alimentos, será de
Arão e de seus filhos; coisa santíssima é, das ofertas queimadas ao
Senhor. E, quando ofereceres oferta de alimentos, cozida no
forno, será de bolos ázimos de flor de farinha, amassados com
azeite, e coscorões ázimos untados com azeite. E, se a tua
oferta for oferta de alimentos cozida na caçoula, será da flor de
farinha sem fermento, amassada com azeite. Em pedaços a
partirás, e sobre ela deitarás azeite; oferta é de alimentos. E,
se a tua oferta for oferta de alimentos de frigideira, far-se-á da
flor de farinha com azeite. Então trarás a oferta de alimentos,
que se fará daquilo, ao Senhor; e se apresentará ao sacerdote, o
qual a levará ao altar. E o sacerdote tomará daquela oferta de
alimentos como memorial, e a queimará sobre o altar; oferta queimada
é de cheiro suave ao Senhor. E, o que sobejar da oferta de
alimentos, será de Arão e de seus filhos; coisa santíssima é, das
ofertas queimadas ao Senhor.

Nenhuma oferta de alimentos, que oferecerdes ao Senhor, se fará
com fermento; porque de nenhum fermento, nem de mel algum,
oferecereis oferta queimada ao Senhor. Deles oferecereis ao
Senhor por oferta das primícias; porém sobre o altar não subirão por
cheiro suave. E todas as tuas ofertas dos teus alimentos
temperarás com sal; e não deixarás faltar à tua oferta de alimentos
o sal da aliança do teu Deus; em todas as tuas ofertas oferecerás
sal. E, se fizeres ao Senhor oferta de alimentos das
primícias, oferecerás como oferta de alimentos das tuas primícias de
espigas verdes, tostadas ao fogo; isto é, do grão trilhado de
espigas verdes cheias. E sobre ela deitarás azeite, e porás
sobre ela incenso; oferta é de alimentos. Assim o sacerdote
queimará o seu memorial do seu grão trilhado, e do seu azeite, com
todo o seu incenso; oferta queimada é ao Senhor.

\medskip

\lettrine{3} E se a sua oferta for sacrifício pacífico; se a
oferecer de gado, macho ou fêmea, a oferecerá sem defeito diante do
Senhor. E porá a sua mão sobre a cabeça da sua oferta, e a
degolará diante da porta da tenda da congregação; e os filhos de
Arão, os sacerdotes, espargirão o sangue sobre o altar em redor.
Depois oferecerá, do sacrifício pacífico, a oferta queimada ao
Senhor; a gordura que cobre a fressura, e toda a gordura que está
sobre a fressura, e ambos os rins, e a gordura que está sobre
eles, e junto aos lombos, e o redenho que está sobre o fígado com os
rins, tirará. E os filhos de Arão queimarão isso sobre o altar,
em cima do holocausto, que estará sobre a lenha que está no fogo;
oferta queimada é de cheiro suave ao Senhor.

E se a sua oferta for de gado miúdo por sacrifício pacífico ao
Senhor, seja macho ou fêmea, sem defeito o oferecerá. Se
oferecer um cordeiro por sua oferta, oferecê-lo-á perante o Senhor;
e porá a sua mão sobre a cabeça da sua oferta, e a degolará
diante da tenda da congregação; e os filhos de Arão espargirão o seu
sangue sobre o altar em redor. Então, do sacrifício pacífico,
oferecerá ao Senhor, por oferta queimada, a sua gordura, a cauda
toda, a qual tirará do espinhaço, e a gordura que cobre a fressura,
e toda a gordura que está sobre a fressura; como também ambos
os rins, e a gordura que está sobre eles, e junto aos lombos, e o
redenho que está sobre o fígado com os rins, tirá-los-á. E o
sacerdote queimará isso sobre o altar; alimento é da oferta queimada
ao Senhor. Mas, se a sua oferta for uma cabra, perante o
Senhor a oferecerá, e porá a sua mão sobre a sua cabeça, e a
degolará diante da tenda da congregação; e os filhos de Arão
espargirão o seu sangue sobre o altar em redor. Depois
oferecerá dela a sua oferta por oferta queimada ao Senhor, a gordura
que cobre a fressura, e toda a gordura que está sobre a fressura;
como também ambos os rins, e a gordura que está sobre eles, e
junto aos lombos, e o redenho que está sobre o fígado com os rins,
tirá-los-á. E o sacerdote o queimará sobre o altar; alimento
é da oferta queimada de cheiro suave. Toda a gordura será do Senhor.
Estatuto perpétuo é pelas vossas gerações, em todas as vossas
habitações: nenhuma gordura nem sangue algum comereis.

\medskip

\lettrine{4} Falou mais o Senhor a Moisés, dizendo: 2 Fala aos
filhos de Israel, dizendo: Quando uma alma pecar, por ignorância,
contra alguns dos mandamentos do Senhor, acerca do que não se deve
fazer, e proceder contra algum deles; se o sacerdote ungido
pecar para escândalo do povo, oferecerá ao Senhor, pelo seu pecado,
que cometeu, um novilho sem defeito, por expiação do pecado. E
trará o novilho à porta da tenda da congregação, perante o Senhor, e
porá a sua mão sobre a cabeça do novilho, e degolará o novilho
perante o Senhor. Então o sacerdote ungido tomará do sangue do
novilho, e o trará à tenda da congregação; e o sacerdote molhará
o seu dedo no sangue, e daquele sangue espargirá sete vezes perante
o Senhor diante do véu do santuário. Também o sacerdote porá
daquele sangue sobre as pontas do altar do incenso aromático,
perante o Senhor que está na tenda da congregação; e todo o restante
do sangue do novilho derramará à base do altar do holocausto, que
está à porta da tenda da congregação. E tirará toda a gordura do
novilho da expiação; a gordura que cobre a fressura, e toda a
gordura que está sobre a fressura, e os dois rins, e a gordura
que está sobre eles, que está junto aos lombos, e o redenho de sobre
o fígado, com os rins, tirá-los-á, como se tira do boi do
sacrifício pacífico; e o sacerdote os queimará sobre o altar do
holocausto. Mas o couro do novilho, e toda a sua carne, com a
sua cabeça e as suas pernas, e as suas entranhas, e o seu esterco,
enfim, o novilho todo levará fora do arraial a um lugar
limpo, onde se lança a cinza, e o queimará com fogo sobre a lenha;
onde se lança a cinza se queimará.

Mas, se toda a congregação de Israel pecar por ignorância, e o
erro for oculto aos olhos do povo, e se fizerem contra alguns dos
mandamentos do Senhor, aquilo que não se deve fazer, e forem
culpados, e quando o pecado que cometeram for conhecido,
então a congregação oferecerá um novilho, por expiação do pecado, e
o trará diante da tenda da congregação, e os anciãos da
congregação porão as suas mãos sobre a cabeça do novilho perante o
Senhor; e degolar-se-á o novilho perante o Senhor. Então o
sacerdote ungido trará do sangue do novilho à tenda da congregação,
e o sacerdote molhará o seu dedo naquele sangue, e o
espargirá sete vezes perante o Senhor, diante do véu. E
daquele sangue porá sobre as pontas do altar, que está perante a
face do Senhor, na tenda da congregação; e todo o restante do sangue
derramará à base do altar do holocausto, que está diante da porta da
tenda da congregação. E tirará dele toda a sua gordura, e
queimá-la-á sobre o altar; e fará a este novilho, como fez ao
novilho da expiação; assim lhe fará, e o sacerdote por eles fará
propiciação, e lhes será perdoado o pecado. Depois levará o
novilho fora do arraial, e o queimará como queimou o primeiro
novilho; é expiação do pecado da congregação.

Quando um príncipe pecar, e por ignorância proceder contra algum
dos mandamentos do Senhor seu Deus, naquilo que não se deve fazer, e
assim for culpado; ou se o pecado que cometeu lhe for
notificado, então trará pela sua oferta um bode tirado das cabras,
macho sem defeito; e porá a sua mão sobre a cabeça do bode, e
o degolará no lugar onde se degola o holocausto, perante a face do
Senhor; expiação do pecado é. Depois o sacerdote com o seu
dedo tomará do sangue da expiação, e o porá sobre as pontas do altar
do holocausto; então o restante do seu sangue derramará à base do
altar do holocausto. Também queimará sobre o altar toda a sua
gordura como gordura do sacrifício pacífico; assim o sacerdote por
ele fará expiação do seu pecado, e lhe será perdoado.

E, se qualquer pessoa do povo da terra pecar por ignorância,
fazendo contra algum dos mandamentos do Senhor, aquilo que não se
deve fazer, e assim for culpada; ou se o pecado que cometeu
lhe for notificado, então trará pela sua oferta uma cabra sem
defeito, pelo seu pecado que cometeu, e porá a sua mão sobre
a cabeça da oferta da expiação do pecado, e a degolará no lugar do
holocausto. Depois o sacerdote com o seu dedo tomará do seu
sangue, e o porá sobre as pontas do altar do holocausto; e todo o
restante do seu sangue derramará à base do altar; e tirará
toda a gordura, como se tira a gordura do sacrifício pacífico; e o
sacerdote a queimará sobre o altar, por cheiro suave ao Senhor; e o
sacerdote fará expiação por ela, e ser-lhe-á perdoado o pecado.
Mas, se pela sua oferta trouxer uma cordeira para expiação do
pecado, sem defeito trará. E porá a sua mão sobre a cabeça da
oferta da expiação do pecado, e a degolará por oferta pelo pecado,
no lugar onde se degola o holocausto. Depois o sacerdote com
o seu dedo tomará do sangue da expiação do pecado, e o porá sobre as
pontas do altar do holocausto; então todo o restante do seu sangue
derramará na base do altar. E tirará toda a sua gordura, como
se tira a gordura do cordeiro do sacrifício pacífico; e o sacerdote
a queimará sobre o altar, em cima das ofertas queimadas do Senhor;
assim o sacerdote por ele fará expiação dos seus pecados que
cometeu, e ele será perdoado.

\medskip

\lettrine{5} E quando alguma pessoa pecar, ouvindo uma voz de
blasfêmia, de que for testemunha, seja porque viu, ou porque soube,
se o não denunciar, então levará a sua iniqüidade. Ou, quando
alguma pessoa tocar em alguma coisa imunda, seja corpo morto de fera
imunda, seja corpo morto de animal imundo, seja corpo morto de
réptil imundo, ainda que não soubesse, contudo será ele imundo e
culpado. Ou, quando tocar a imundícia de um homem, seja qualquer
que for a sua imundícia, com que se faça imundo, e lhe for oculto, e
o souber depois, será culpado. Ou, quando alguma pessoa jurar,
pronunciando temerariamente com os seus lábios, para fazer mal, ou
para fazer bem, em tudo o que o homem pronuncia temerariamente com
juramento, e lhe for oculto, e o souber depois, culpado será numa
destas coisas. Será, pois, que, culpado sendo numa destas
coisas, confessará aquilo em que pecou. E a sua expiação trará
ao Senhor, pelo seu pecado que cometeu: uma fêmea de gado miúdo, uma
cordeira, ou uma cabrinha pelo pecado; assim o sacerdote por ela
fará expiação do seu pecado.

Mas, se em sua mão não houver recurso para gado miúdo, então
trará, para expiação da culpa que cometeu, ao Senhor, duas rolas ou
dois pombinhos; um para expiação do pecado, e o outro para
holocausto; e os trará ao sacerdote, o qual primeiro oferecerá
aquele que é para expiação do pecado; e com a sua unha lhe fenderá a
cabeça junto ao pescoço, mas não o partirá; e do sangue da
expiação do pecado espargirá sobre a parede do altar, porém o que
sobejar daquele sangue espremer-se-á à base do altar; expiação do
pecado é. E do outro fará holocausto conforme ao costume;
assim o sacerdote por ela fará expiação do seu pecado que cometeu, e
ele será perdoado. Porém, se em sua mão não houver recurso
para duas rolas, ou dois pombinhos, então aquele que pecou trará
como oferta a décima parte de um efa de flor de farinha, para
expiação do pecado; não deitará sobre ela azeite nem lhe porá em
cima o incenso, porquanto é expiação do pecado; e a trará ao
sacerdote, e o sacerdote dela tomará a sua mão cheia pelo seu
memorial, e a queimará sobre o altar, em cima das ofertas queimadas
do Senhor; expiação de pecado é. Assim o sacerdote por ela
fará expiação do seu pecado, que cometeu em alguma destas coisas, e
lhe será perdoado; e o restante será do sacerdote, como a oferta de
alimentos.

E falou o Senhor a Moisés, dizendo: Quando alguma pessoa
cometer uma transgressão, e pecar por ignorância nas coisas sagradas
do Senhor, então trará ao Senhor pela expiação, um carneiro sem
defeito do rebanho, conforme à tua estimação em siclos de prata,
segundo o siclo do santuário, para expiação da culpa. Assim
restituirá o que pecar nas coisas sagradas, e ainda lhe acrescentará
a quinta parte, e a dará ao sacerdote; assim o sacerdote, com o
carneiro da expiação, fará expiação por ele, e ser-lhe-á perdoado o
pecado. E, se alguma pessoa pecar, e fizer, contra algum dos
mandamentos do Senhor, aquilo que não se deve fazer, ainda que o não
soubesse, contudo será ela culpada, e levará a sua iniqüidade;
e trará ao sacerdote um carneiro sem defeito do rebanho,
conforme à tua estimação, para expiação da culpa, e o sacerdote por
ela fará expiação do erro que cometeu sem saber; e ser-lhe-á
perdoado. Expiação de culpa é; certamente se fez culpado
diante do Senhor.


\medskip

\lettrine{6} Falou mais o Senhor a Moisés, dizendo: Quando
alguma pessoa pecar, e transgredir contra o Senhor, e negar ao seu
próximo o que lhe deu em guarda, ou o que deixou na sua mão, ou o
roubo, ou o que reteve violentamente ao seu próximo, ou que
achou o perdido, e o negar com falso juramento, ou fizer alguma
outra coisa de todas em que o homem costuma pecar; será pois
que, como pecou e tornou-se culpado, restituirá o que roubou, ou o
que reteve violentamente, ou o depósito que lhe foi dado em guarda,
ou o perdido que achou, ou tudo aquilo sobre que jurou
falsamente; e o restituirá no seu todo, e ainda sobre isso
acrescentará o quinto; àquele de quem é o dará no dia de sua
expiação. E a sua expiação trará ao Senhor: um carneiro sem
defeito do rebanho, conforme à tua estimação, para expiação da culpa
trará ao sacerdote; e o sacerdote fará expiação por ela diante
do Senhor, e será perdoada de qualquer das coisas que fez,
tornando-se culpada.

Falou mais o Senhor a Moisés, dizendo: 9 Dá ordem a Arão e a seus
filhos, dizendo: Esta é a lei do holocausto; o holocausto será
queimado sobre o altar toda a noite até pela manhã, e o fogo do
altar arderá nele. E o sacerdote vestirá a sua veste de
linho, e vestirá as calças de linho, sobre a sua carne, e levantará
a cinza, quando o fogo houver consumido o holocausto sobre o altar,
e a porá junto ao altar. Depois despirá as suas vestes, e
vestirá outras vestes; e levará a cinza fora do arraial para um
lugar limpo. O fogo que está sobre o altar arderá nele, não
se apagará; mas o sacerdote acenderá lenha nele cada manhã, e sobre
ele porá em ordem o holocausto e sobre ele queimará a gordura das
ofertas pacíficas. O fogo arderá continuamente sobre o altar;
não se apagará.

E esta é a lei da oferta de alimentos: os filhos de Arão a
oferecerão perante o Senhor diante do altar. E dela tomará um
punhado da flor de farinha, da oferta e do seu azeite, e todo o
incenso que estiver sobre a oferta de alimentos; então o acenderá
sobre o altar, cheiro suave é isso, por ser memorial ao Senhor.
E o restante dela comerão Arão e seus filhos; ázimo se comerá
no lugar santo, no pátio da tenda da congregação o comerão.
Levedado não se cozerá; sua porção é que lhes dei das minhas
ofertas queimadas; coisa santíssima é, como a expiação do pecado e
como a expiação da culpa. Todo o homem entre os filhos de
Arão comerá dela; estatuto perpétuo será para as vossas gerações das
ofertas queimadas do Senhor; todo o que as tocar será santo.
Falou mais o Senhor a Moisés, dizendo: Esta é a oferta
de Arão e de seus filhos, a qual oferecerão ao Senhor no dia em que
ele for ungido; a décima parte de um efa de flor de farinha pela
oferta de alimentos contínua; a metade dela pela manhã, e a outra
metade à tarde. Numa caçoula se fará com azeite; cozida a
trarás; e os pedaços cozidos da oferta oferecerás em cheiro suave ao
Senhor. Também o sacerdote, que de entre seus filhos for
ungido em seu lugar, fará o mesmo; por estatuto perpétuo será ela
toda queimada ao Senhor. Assim toda a oferta do sacerdote
será totalmente queimada; não se comerá.

Falou mais o Senhor a Moisés, dizendo: Fala a Arão e a
seus filhos, dizendo: Esta é a lei da expiação do pecado; no lugar
onde se degola o holocausto se degolará a expiação do pecado perante
o Senhor; coisa santíssima é. O sacerdote que a oferecer pelo
pecado a comerá; no lugar santo se comerá, no pátio da tenda da
congregação. Tudo o que tocar a carne da oferta será santo;
se o seu sangue for espargido sobre as vestes de alguém, lavarás em
lugar santo aquilo sobre o que caiu. E o vaso de barro em que
for cozida será quebrado; porém, se for cozida num vaso de cobre,
esfregar-se-á e lavar-se-á na água. Todo o homem entre os
sacerdotes a comerá; coisa santíssima é. Porém, não se comerá
nenhuma oferta pelo pecado, cujo sangue se traz à tenda da
congregação, para expiar no santuário; no fogo será queimada.

\medskip

\lettrine{7} E esta é a lei da expiação da culpa; coisa
santíssima é. No lugar onde degolam o holocausto, degolarão a
oferta pela expiação da culpa, e o seu sangue se espargirá sobre o
altar em redor. E dela se oferecerá toda a sua gordura; a cauda,
e a gordura que cobre a fressura. Também ambos os rins, e a
gordura que neles há, que está junto aos lombos, e o redenho sobre o
fígado, com os rins se tirará; e o sacerdote os queimará sobre o
altar em oferta queimada ao Senhor; expiação da culpa é. Todo o
varão entre os sacerdotes a comerá; no lugar santo se comerá; coisa
santíssima é. Como a expiação pelo pecado, assim será a expiação
da culpa; uma mesma lei haverá para elas; será do sacerdote que
houver feito propiciação com ela. Também o sacerdote, que
oferecer o holocausto de alguém, terá para si o couro do holocausto
que oferecer. Como também toda a oferta que se cozer no forno,
com tudo que se preparar na frigideira e na caçoula, será do
sacerdote que a oferecer. Também toda a oferta amassada com
azeite, ou seca, será de todos os filhos de Arão, assim de um como
de outro.

E esta é a lei do sacrifício pacífico que se oferecerá ao Senhor:
Se o oferecer por oferta de ação de graças, com o sacrifício
de ação de graças, oferecerá bolos ázimos amassados com azeite; e
coscorões ázimos amassados com azeite; e os bolos amassados com
azeite serão fritos, de flor de farinha. Com os bolos
oferecerá por sua oferta pão levedado, com o sacrifício de ação de
graças da sua oferta pacífica. E de toda a oferta oferecerá
uma parte por oferta alçada ao Senhor, que será do sacerdote que
espargir o sangue da oferta pacífica. Mas a carne do
sacrifício de ação de graças da sua oferta pacífica se comerá no dia
do seu oferecimento; nada se deixará dela até à manhã. E, se
o sacrifício da sua oferta for voto, ou oferta voluntária, no dia em
que oferecer o seu sacrifício se comerá; e o que dele ficar também
se comerá no dia seguinte; e o que ainda ficar da carne do
sacrifício ao terceiro dia será queimado no fogo. Porque, se
da carne do seu sacrifício pacífico se comer ao terceiro dia, aquele
que a ofereceu não será aceito, nem lhe será imputado; coisa
abominável será, e a pessoa que dela comer levará a sua iniqüidade.
E a carne que tocar alguma coisa imunda não se comerá; com
fogo será queimada; mas da outra carne, qualquer que estiver limpo,
comerá dela. Porém, se alguma pessoa comer a carne do
sacrifício pacífico, que é do Senhor, tendo ela sobre si a sua
imundícia, aquela pessoa será extirpada do seu povo. E, se
uma pessoa tocar alguma coisa imunda, como imundícia de homem, ou
gado imundo, ou qualquer abominação imunda, e comer da carne do
sacrifício pacífico, que é do Senhor, aquela pessoa será extirpada
do seu povo. Depois falou o Senhor a Moisés, dizendo: 23 Fala
aos filhos de Israel, dizendo: Nenhuma gordura de boi, nem de
carneiro, nem de cabra comereis; porém pode-se usar da
gordura de corpo morto, e da gordura do dilacerado por feras, para
toda a obra, mas de nenhuma maneira a comereis; porque
qualquer que comer a gordura do animal, do qual se oferecer ao
Senhor oferta queimada, a pessoa que a comer será extirpada do seu
povo. E nenhum sangue comereis em qualquer das vossas
habitações, quer de aves quer de gado. Toda a pessoa que
comer algum sangue, aquela pessoa será extirpada do seu povo.
Falou mais o Senhor a Moisés, dizendo: 29 Fala aos filhos de
Israel, dizendo: Quem oferecer ao Senhor o seu sacrifício pacífico,
trará a sua oferta ao Senhor do seu sacrifício pacífico. As
suas próprias mãos trarão as ofertas queimadas do Senhor; a gordura
do peito com o peito trará para movê-lo por oferta movida perante o
Senhor. E o sacerdote queimará a gordura sobre o altar, porém
o peito será de Arão e de seus filhos. Também a espádua
direita dareis ao sacerdote por oferta alçada dos vossos sacrifícios
pacíficos. Aquele dos filhos de Arão que oferecer o sangue do
sacrifício pacífico, e a gordura, esse terá a espádua direita para a
sua porção; porque o peito movido e a espádua alçada tomei
dos filhos de Israel dos seus sacrifícios pacíficos, e os dei a
Arão, o sacerdote, e a seus filhos, por estatuto perpétuo dos filhos
de Israel.

Esta é a porção de Arão e a porção de seus filhos das ofertas
queimadas do Senhor, desde o dia em que ele os apresentou para
administrar o sacerdócio ao Senhor. O que o Senhor ordenou
que se lhes desse dentre os filhos de Israel no dia em que os ungiu;
estatuto perpétuo é pelas suas gerações. Esta é a lei do
holocausto, da oferta de alimentos, e da expiação do pecado, e da
expiação da culpa, e da oferta das consagrações, e do sacrifício
pacífico, que o Senhor ordenou a Moisés no monte Sinai, no
dia em que ordenou aos filhos de Israel que oferecessem as suas
ofertas ao Senhor, no deserto de Sinai.

\medskip

\lettrine{8} Falou mais o Senhor a Moisés, dizendo: Toma a
Arão e a seus filhos com ele, e as vestes, e o azeite da unção, como
também o novilho da expiação do pecado, e os dois carneiros, e o
cesto dos pães ázimos, e reúne toda a congregação à porta da
tenda da congregação. Fez, pois, Moisés como o Senhor lhe
ordenara, e a congregação reuniu-se à porta da tenda da congregação.
Então disse Moisés à congregação: Isto é o que o Senhor ordenou
que se fizesse. E Moisés fez chegar a Arão e a seus filhos, e os
lavou com água. E vestiu-lhe a túnica, e cingiu-o com o cinto, e
pôs sobre ele o manto; também pôs sobre ele o éfode, e cingiu-o com
o cinto de obra esmerada do éfode e o apertou com ele. Depois
pôs-lhe o peitoral, pondo no peitoral o Urim e o Tumim; e pôs a
mitra sobre a sua cabeça; e sobre esta, na parte dianteira, pôs a
lâmina de ouro, a coroa da santidade, como o Senhor ordenara a
Moisés. Então Moisés tomou o azeite da unção, e ungiu o
tabernáculo, e tudo o que havia nele, e o santificou; e dele
espargiu sete vezes sobre o altar, e ungiu o altar e todos os seus
utensílios, como também a pia e a sua base, para santificá-las.
Depois derramou do azeite da unção sobre a cabeça de Arão, e
ungiu-o, para santificá-lo. Também Moisés fez chegar os
filhos de Arão, e vestiu-lhes as túnicas, e cingiu-os com o cinto, e
apertou-lhes as tiaras, como o Senhor ordenara a Moisés.

Então fez chegar o novilho da expiação do pecado; e Arão e seus
filhos puseram as suas mãos sobre a cabeça do novilho da expiação do
pecado; e o degolou; e Moisés tomou o sangue, e pôs dele com
o seu dedo sobre as pontas do altar em redor, e purificou o altar;
depois derramou o restante do sangue à base do altar, e o
santificou, para fazer expiação por ele. Depois tomou toda a
gordura que está na fressura, e o redenho do fígado, e os dois rins
e a sua gordura; e Moisés queimou-os sobre o altar. Mas o
novilho com o seu couro, e a sua carne, e o seu esterco, queimou com
fogo fora do arraial, como o Senhor ordenara a Moisés. Depois
fez chegar o carneiro do holocausto; e Arão e seus filhos puseram as
suas mãos sobre a cabeça do carneiro; e degolou-o; e Moisés
espargiu o sangue sobre o altar em redor. Partiu também o
carneiro nos seus pedaços; e Moisés queimou a cabeça, e os pedaços e
a gordura. Porém a fressura e as pernas lavou com água; e
Moisés queimou todo o carneiro sobre o altar; holocausto de cheiro
suave, uma oferta queimada ao Senhor, como o Senhor ordenou a
Moisés. Depois fez chegar o outro carneiro, o carneiro da
consagração; e Arão com seus filhos puseram as suas mãos sobre a
cabeça do carneiro. E degolou-o; e Moisés tomou do seu
sangue, e o pôs sobre a ponta da orelha direita de Arão, e sobre o
polegar da sua mão direita, e sobre o polegar do seu pé direito.
Moisés também fez chegar os filhos de Arão, e pôs daquele
sangue sobre a ponta da orelha direita deles, e sobre o polegar da
sua mão direita, e sobre o polegar do seu pé direito; e Moisés
espargiu o restante do sangue sobre o altar em redor. E tomou
a gordura, e a cauda, e toda a gordura que está na fressura, e o
redenho do fígado, e ambos os rins, e a sua gordura e a espádua
direita. Também do cesto dos pães ázimos, que estava diante
do Senhor, tomou um bolo ázimo, e um bolo de pão azeitado, e um
coscorão, e os pôs sobre a gordura e sobre a espádua direita.
E tudo isto pôs nas mãos de Arão e nas mãos de seus filhos; e
os ofereceu por oferta movida perante o Senhor. Depois Moisés
tomou-os das suas mãos, e os queimou no altar sobre o holocausto;
estes foram uma consagração, por cheiro suave, oferta queimada ao
Senhor. E tomou Moisés o peito, e ofereceu-o por oferta
movida perante o Senhor. Aquela foi a porção de Moisés do carneiro
da consagração, como o Senhor ordenara a Moisés. Tomou Moisés
também do azeite da unção, e do sangue que estava sobre o altar, e o
espargiu sobre Arão e sobre as suas vestes, e sobre os seus filhos,
e sobre as vestes de seus filhos com ele; e santificou a Arão e as
suas vestes, e seus filhos, e as vestes de seus filhos com ele.
E Moisés disse a Arão, e a seus filhos: Cozei a carne diante
da porta da tenda da congregação, e ali a comereis com o pão que
está no cesto da consagração, como tenho ordenado, dizendo: Arão e
seus filhos a comerão. Mas o que sobejar da carne e do pão,
queimareis com fogo. Também da porta da tenda da congregação
não saireis por sete dias, até ao dia em que se cumprirem os dias da
vossa consagração; porquanto por sete dias ele vos consagrará.
Como se fez neste dia, assim o Senhor ordenou se fizesse,
para fazer expiação por vós. Ficareis, pois, à porta da tenda
da congregação dia e noite por sete dias, e guardareis as ordenanças
do Senhor, para que não morrais; porque assim me foi ordenado.
E Arão e seus filhos fizeram todas as coisas que o Senhor
ordenara pela mão de Moisés.

\medskip

\lettrine{9} E aconteceu, ao dia oitavo, que Moisés chamou a
Arão e seus filhos, e os anciãos de Israel, e disse a Arão: Toma
um bezerro, para expiação do pecado, e um carneiro para holocausto,
sem defeito; e traze-os perante o Senhor. Depois falarás aos
filhos de Israel, dizendo: Tomai um bode para expiação do pecado, e
um bezerro, e um cordeiro de um ano, sem defeito, para holocausto;
também um boi e um carneiro por sacrifício pacífico, para
sacrificar perante o Senhor, e oferta de alimentos, amassada com
azeite; porquanto hoje o Senhor vos aparecerá. Então trouxeram o
que ordenara Moisés, diante da tenda da congregação, e chegou-se
toda a congregação e se pôs perante o Senhor. E disse Moisés:
Esta é a coisa que o Senhor ordenou que fizésseis; e a glória do
Senhor vos aparecerá. E disse Moisés a Arão: Chega-te ao altar,
e faze a tua expiação de pecado e o teu holocausto; e faze expiação
por ti e pelo povo; depois faze a oferta do povo, e faze expiação
por eles, como ordenou o Senhor. Então Arão se chegou ao altar,
e degolou o bezerro da expiação que era por si mesmo. E os
filhos de Arão trouxeram-lhe o sangue, e molhou o seu dedo no
sangue, e o pôs sobre as pontas do altar; e o restante do sangue
derramou à base do altar. Mas a gordura, e os rins, e o
redenho do fígado de expiação do pecado, queimou sobre o altar, como
o Senhor ordenara a Moisés. Porém a carne e o couro queimou
com fogo fora do arraial. Depois degolou o holocausto, e os
filhos de Arão lhe entregaram o sangue, e espargiu-o sobre o altar
em redor. Também lhe entregaram o holocausto nos seus
pedaços, com a cabeça; e queimou-o sobre o altar. E lavou a
fressura e as pernas, e as queimou sobre o holocausto no altar.
Depois fez chegar a oferta do povo, e tomou o bode da
expiação do pecado, que era pelo povo, e o degolou, e o preparou por
expiação do pecado, como o primeiro. Fez também chegar o
holocausto, e ofereceu-o segundo o rito. E fez chegar a
oferta de alimentos, e a sua mão encheu dela, e queimou-a sobre o
altar, além do holocausto da manhã. Depois degolou o boi e o
carneiro em sacrifício pacífico, que era pelo povo; e os filhos de
Arão entregaram-lhe o sangue, que espargiu sobre o altar em redor.
Como também a gordura do boi e do carneiro, a cauda, e o que
cobre a fressura, e os rins, e o redenho do fígado. E puseram
a gordura sobre os peitos, e queimou a gordura sobre o altar;
mas os peitos e a espádua direita Arão ofereceu por oferta
movida perante o Senhor, como Moisés tinha ordenado. Depois
Arão levantou as suas mãos ao povo e o abençoou; e desceu, havendo
feito a expiação do pecado, e o holocausto, e a oferta pacífica.

Então entraram Moisés e Arão na tenda da congregação; depois
saíram, e abençoaram ao povo; e a glória do Senhor apareceu a todo o
povo. Porque o fogo saiu de diante do Senhor, e consumiu o
holocausto e a gordura, sobre o altar; o que vendo todo o povo,
jubilaram e caíram sobre as suas faces.

\medskip

\lettrine{10} E os filhos de Arão, Nadabe e Abiú, tomaram cada
um o seu incensário e puseram neles fogo, e colocaram incenso sobre
ele, e ofereceram fogo estranho perante o Senhor, o que não lhes
ordenara. Então saiu fogo de diante do Senhor e os consumiu; e
morreram perante o Senhor.

E disse Moisés a Arão: Isto é o que o Senhor falou, dizendo: Serei
santificado naqueles que se chegarem a mim, e serei glorificado
diante de todo o povo. Porém Arão calou-se. E Moisés chamou a
Misael e a Elzafã, filhos de Uziel, tio de Arão, e disse-lhes:
Chegai, levai a vossos irmãos de diante do santuário, para fora do
arraial. Então chegaram, e os levaram nas suas túnicas para fora
do arraial, como Moisés lhes dissera. E Moisés disse a Arão, e a
seus filhos Eleazar e Itamar: Não descobrireis as vossas cabeças,
nem rasgareis vossas vestes, para que não morrais, nem venha grande
indignação sobre toda a congregação; mas vossos irmãos, toda a casa
de Israel, lamentem este incêndio que o Senhor acendeu. Nem
saireis da porta da tenda da congregação, para que não morrais;
porque está sobre vós o azeite da unção do Senhor. E fizeram
conforme à palavra de Moisés.

E falou o Senhor a Arão, dizendo: ão bebereis vinho nem
bebida forte, nem tu nem teus filhos contigo, quando entrardes na
tenda da congregação, para que não morrais; estatuto perpétuo será
isso entre as vossas gerações; e para fazer diferença entre o
santo e o profano e entre o imundo e o limpo, e para ensinar
aos filhos de Israel todos os estatutos que o Senhor lhes tem falado
por meio de Moisés.

E disse Moisés a Arão, e a Eleazar e a Itamar, seus filhos, que
lhe ficaram: Tomai a oferta de alimentos, restante das ofertas
queimadas do Senhor, e comei-a sem levedura junto ao altar,
porquanto é coisa santíssima. Portanto a comereis no lugar
santo; porque isto é a tua porção, e a porção de teus filhos, das
ofertas queimadas do Senhor; porque assim me foi ordenado.
Também o peito da oferta movida e a espádua da oferta alçada,
comereis em lugar limpo, tu, e teus filhos e tuas filhas contigo;
porque foram dados por tua porção, e por porção de teus filhos, dos
sacrifícios pacíficos dos filhos de Israel. A espádua da
oferta alçada e o peito da oferta movida trarão com as ofertas
queimadas de gordura, para oferecer por oferta movida perante o
Senhor; o que será por estatuto perpétuo, para ti e para teus filhos
contigo, como o Senhor tem ordenado. E Moisés diligentemente
buscou o bode da expiação, e eis que já fora queimado; portanto
indignou-se grandemente contra Eleazar e contra Itamar, os filhos de
Arão que ficaram, dizendo: Por que não comestes a expiação do
pecado no lugar santo, pois é coisa santíssima e Deus a deu a vós,
para que levásseis a iniqüidade da congregação, para fazer expiação
por eles diante do Senhor? Eis que não se trouxe o seu sangue
para dentro do santuário; certamente devíeis ter comido no
santuário, como tenho ordenado. Então disse Arão a Moisés:
Eis que hoje ofereceram a sua expiação pelo pecado e o seu
holocausto perante o Senhor, e tais coisas me sucederam; se hoje
tivesse comido da oferta da expiação pelo pecado, seria isso
porventura aceito aos olhos do Senhor? E Moisés, ouvindo
isto, deu-se por satisfeito.

\medskip

\lettrine{11} E falou o Senhor a Moisés e a Arão,
dizendo-lhes: Fala aos filhos de Israel, dizendo: Estes são os
animais, que comereis dentre todos os animais que há sobre a terra;
dentre os animais, todo o que tem unhas fendidas, e a fenda das
unhas se divide em duas, e rumina, deles comereis. Destes,
porém, não comereis; dos que ruminam ou dos que têm unhas fendidas;
o camelo, que rumina, mas não tem unhas fendidas; esse vos será
imundo; e o coelho, porque rumina, mas não tem as unhas
fendidas; esse vos será imundo; e a lebre, porque rumina, mas
não tem as unhas fendidas; essa vos será imunda. Também o porco,
porque tem unhas fendidas, e a fenda das unhas se divide em duas,
mas não rumina; este vos será imundo. Das suas carnes não
comereis, nem tocareis nos seus cadáveres; estes vos serão imundos.

De todos os animais que há nas águas, comereis os seguintes: todo
o que tem barbatanas e escamas, nas águas, nos mares e nos rios,
esses comereis. Mas todo o que não tem barbatanas, nem
escamas, nos mares e nos rios, todo o réptil das águas, e todo o ser
vivente que há nas águas, estes serão para vós abominação.
Ser-vos-ão, pois, por abominação; da sua carne não comereis,
e abominareis o seu cadáver. Todo o que não tem barbatanas ou
escamas, nas águas, será para vós abominação. Das aves, estas
abominareis; não se comerão, serão abominação: a águia, e o
quebrantosso, e o xofrango\footnote{A águia-pesqueira quando nova.},
e o milhano\footnote{Ou milhafre: Ave de rapina européia,
falconídea (Milvus milvus).}, e o abutre segundo a sua espécie.
Todo o corvo segundo a sua espécie, e o avestruz, e o
mocho\footnote{Designação vulgar das corujas ou caburés sem penacho
ou tufo de penas na cabeça.}, e a gaivota, e o gavião segundo a sua
espécie. E o bufo\footnote{Ave noturna, estrigídea;
corujão.}, e o corvo marinho, e a coruja, e a gralha, e o
cisne, e o pelicano, e a cegonha, a garça segundo a sua
espécie, e a poupa\footnote{Pássaro semelhante à pega (ave européia,
corvídea (Pica pica), de coloração preta, tendendo ao verde no
dorso, flancos, abdome e baixo dorso brancos, asas azuis, e
coberteiras primárias verdes).}, e o morcego.

Todo o inseto que voa, que anda sobre quatro pés, será para vós
uma abominação. Mas isto comereis de todo o inseto que voa,
que anda sobre quatro pés: o que tiver pernas sobre os seus pés,
para saltar com elas sobre a terra. Deles comereis estes: a
locusta\footnote{Gafanhoto.} segundo a sua espécie, o gafanhoto
devorador segundo a sua espécie, o grilo segundo a sua espécie, e o
gafanhoto segundo a sua espécie. E todos os outros insetos
que voam, que têm quatro pés, serão para vós uma abominação.
E por estes sereis imundos: qualquer que tocar os seus
cadáveres, imundo será até à tarde. Qualquer que levar os
seus cadáveres lavará as suas vestes, e será imundo até à tarde.
Todo o animal que tem unha fendida, mas a fenda não se divide
em duas, e todo o que não rumina, vos será por imundo; qualquer que
tocar neles será imundo. E todo o animal que anda sobre as
suas patas, todo o animal que anda a quatro pés, vos será por
imundo; qualquer que tocar nos seus cadáveres será imundo até à
tarde. E o que levar os seus cadáveres lavará as suas vestes,
e será imundo até à tarde; eles vos serão por imundos. Estes
também vos serão por imundos entre os répteis que se arrastam sobre
a terra; a doninha, e o rato, e a tartaruga segundo a sua espécie,
e o ouriço cacheiro\footnote{Mamífero roedor (que se cacha ou
esconde).}, e o lagarto, e a lagartixa, e a lesma e a toupeira.
Estes vos serão por imundos dentre todos os répteis; qualquer
que os tocar, estando eles mortos, será imundo até à tarde. E
tudo aquilo sobre o que cair alguma coisa deles estando eles mortos
será imundo; seja vaso de madeira, ou veste, ou pele, ou saco,
qualquer instrumento, com que se faz alguma obra, será posto na
água, e será imundo até à tarde; depois será limpo. E todo o
vaso de barro, em que cair alguma coisa deles, tudo o que houver
nele será imundo, e o vaso quebrareis. Todo o alimento que se
come, sobre o qual cair água de tais vasos, será imundo; e toda a
bebida que se bebe, depositada nesses vasos, será imunda. E
aquilo sobre o que cair alguma parte de seu corpo morto, será
imundo; o forno e o vaso de barro serão quebrados; imundos são:
portanto vos serão por imundos. Porém a fonte ou cisterna, em
que se recolhem águas, será limpa, mas quem tocar no seu cadáver
será imundo. E, se dos seus cadáveres cair alguma coisa sobre
alguma semente que se vai semear, será limpa; mas se for
deitada água sobre a semente, e se dos seus cadáveres cair alguma
coisa sobre ela, vos será por imunda. E se morrer algum dos
animais, que vos servem de mantimento, quem tocar no seu cadáver
será imundo até à tarde; e quem comer do seu cadáver lavará
as suas vestes, e será imundo até à tarde; e quem levar o seu corpo
morto lavará as suas vestes, e será imundo até à tarde.
Também todo o réptil, que se arrasta sobre a terra, será
abominação; não se comerá. Tudo o que anda sobre o ventre, e
tudo o que anda sobre quatro pés, ou que tem muitos pés, entre todo
o réptil que se arrasta sobre a terra, não comereis, porquanto são
uma abominação.

Não vos façais abomináveis, por nenhum réptil que se arrasta, nem
neles vos contamineis, para não serdes imundos por eles;
porque eu sou o Senhor vosso Deus; portanto vós vos
santificareis, e sereis santos, porque eu sou santo; e não vos
contaminareis com nenhum réptil que se arrasta sobre a terra;
porque eu sou o Senhor, que vos fiz subir da terra do Egito,
para que eu seja vosso Deus, e para que sejais santos; porque eu sou
santo. Esta é a lei dos animais, e das aves, e de toda
criatura vivente que se move nas águas, e de toda criatura que se
arrasta sobre a terra; para fazer diferença entre o imundo e
o limpo; e entre animais que se podem comer e os animais que não se
podem comer.

\medskip

\lettrine{12} Falou mais o Senhor a Moisés, dizendo: Fala
aos filhos de Israel, dizendo: Se uma mulher conceber e der à luz um
menino, será imunda sete dias, assim como nos dias da separação da
sua enfermidade, será imunda. E no dia oitavo se circuncidará ao
menino a carne do seu prepúcio. Depois ficará ela trinta e três
dias no sangue da sua purificação; nenhuma coisa santa tocará e não
entrará no santuário até que se cumpram os dias da sua purificação.
Mas, se der à luz uma menina será imunda duas semanas, como na
sua separação; depois ficará sessenta e seis dias no sangue da sua
purificação.

E, quando forem cumpridos os dias da sua purificação por filho ou
por filha, trará um cordeiro de um ano por holocausto, e um pombinho
ou uma rola para expiação do pecado, diante da porta da tenda da
congregação, ao sacerdote. O qual o oferecerá perante o Senhor,
e por ela fará propiciação; e será limpa do fluxo do seu sangue;
esta é a lei da que der à luz menino ou menina. Mas, se em sua
mão não houver recursos para um cordeiro, então tomará duas rolas,
ou dois pombinhos, um para o holocausto e outro para a propiciação
do pecado; assim o sacerdote por ela fará expiação, e será limpa.

\medskip

\lettrine{13} Falou mais o Senhor a Moisés e a Arão, dizendo:
Quando um homem tiver na pele da sua carne, inchação, ou
pústula, ou mancha lustrosa, na pele de sua carne como praga da
lepra, então será levado a Arão, o sacerdote, ou a um de seus
filhos, os sacerdotes. E o sacerdote examinará a praga na pele
da carne; se o pêlo na praga se tornou branco, e a praga parecer
mais profunda do que a pele da sua carne, é praga de lepra; o
sacerdote o examinará, e o declarará por imundo. Mas, se a
mancha na pele de sua carne for branca, e não parecer mais profunda
do que a pele, e o pêlo não se tornou branco, então o sacerdote
encerrará o que tem a praga por sete dias; e ao sétimo dia o
sacerdote o examinará; e eis que, se a praga, ao seu parecer parou,
e na pele não se estendeu, então o sacerdote o encerrará por outros
sete dias; e o sacerdote ao sétimo dia o examinará outra vez; e
eis que, se a praga se recolheu, e na pele não se estendeu, então o
sacerdote o declarará por limpo; é uma pústula; e lavará as suas
vestes, e será limpo. Mas, se a pústula na pele se estende
grandemente, depois que foi mostrado ao sacerdote para a sua
purificação, outra vez será mostrado ao sacerdote, e o sacerdote
o examinará, e eis que, se a pústula na pele se tem estendido, o
sacerdote o declarará por imundo; é lepra. Quando no homem
houver praga de lepra, será levado ao sacerdote, e o
sacerdote o examinará, e eis que, se há inchação branca na pele, a
qual tornou o pêlo em branco, e houver carne viva na inchação,
lepra inveterada é na pele da sua carne; portanto, o
sacerdote o declarará por imundo; não o encerrará, porque imundo é.
E, se a lepra se espalhar de todo na pele, e a lepra cobrir
toda a pele do que tem a praga, desde a sua cabeça até aos seus pés,
quanto podem ver os olhos do sacerdote, então o sacerdote
examinará, e eis que, se a lepra tem coberto toda a sua carne, então
declarará o que tem a praga por limpo; todo se tornou branco; limpo
está. Mas no dia em que aparecer nela carne viva será imundo.
Vendo, pois, o sacerdote a carne viva, declará-lo-á por
imundo; a carne é imunda; é lepra. Ou, tornando a carne viva,
e mudando-se em branca, então virá ao sacerdote, e este o
examinará, e eis que, se a praga se tornou branca, então o sacerdote
declarará limpo o que tem a praga; limpo está.

Se também a carne, em cuja pele houver alguma úlcera, sarar,
e, em lugar da pústula, vier inchação branca ou mancha
lustrosa, tirando a vermelho, mostrar-se-á então ao sacerdote.
E o sacerdote examinará, e eis que, se ela parece mais funda
do que a pele, e o seu pêlo se tornou branco, o sacerdote o
declarará por imundo; é praga da lepra que brotou da pústula.
E o sacerdote, vendo-a, e eis que se nela não houver pêlo
branco, nem estiver mais funda do que a pele, mas encolhida, então o
sacerdote o encerrará por sete dias. Se ela grandemente se
estender na pele, o sacerdote o declarará por imundo; praga é.
Mas se a mancha parar no seu lugar, não se estendendo,
inflamação da pústula é; o sacerdote, pois, o declarará por limpo.
Ou, quando na pele da carne houver queimadura de fogo, e no
que é sarado da queimadura houver mancha lustrosa, tirando a
vermelho ou branco, e o sacerdote vendo-a, e eis que se o
pêlo na mancha se tornou branco e ela parece mais funda do que a
pele, lepra é, que floresceu pela queimadura; portanto o sacerdote o
declarará por imundo; é praga de lepra. Mas, se o sacerdote,
vendo-a, e eis que, se na mancha não aparecer pêlo branco, nem
estiver mais funda do que a pele, mas recolhida, o sacerdote o
encerrará por sete dias. Depois o sacerdote o examinará ao
sétimo dia; se grandemente se houver estendido na pele, o sacerdote
o declarará por imundo; é praga de lepra. Mas se a mancha
parar no seu lugar, e na pele não se estender, mas se recolher,
inchação da queimadura é; portanto o sacerdote o declarará por
limpo, porque inflamação é da queimadura. E, quando homem ou
mulher tiver chaga na cabeça ou na barba, e o sacerdote,
examinando a chaga, e eis que, se ela parece mais funda do que a
pele, e pêlo amarelo fino há nela, o sacerdote o declarará por
imundo; é tinha, é lepra da cabeça ou da barba. Mas, se o
sacerdote, havendo examinado a praga da tinha, e eis que, se ela não
parece mais funda do que a pele, e se nela não houver pêlo preto,
então o sacerdote encerrará o que tem a praga da tinha por sete
dias. E o sacerdote examinará a praga ao sétimo dia; e eis
que, se a tinha não se tiver estendido, e nela não houver pêlo
amarelo, nem a tinha parecer mais funda do que a pele, então
se rapará; mas não rapará a tinha; e o sacerdote segunda vez
encerrará o que tem a tinha por sete dias. Depois o sacerdote
examinará a tinha ao sétimo dia; e eis que, se a tinha não se houver
estendido na pele, e ela não parecer mais funda do que a pele, o
sacerdote o declarará por limpo, e lavará as suas vestes, e será
limpo. Mas, se a tinha, depois da sua purificação, se houver
estendido grandemente na pele, então o sacerdote o examinará,
e eis que, se a tinha se tem estendido na pele, o sacerdote não
buscará pêlo amarelo; imundo está. Mas, se a tinha ao seu ver
parou, e pêlo preto nela cresceu, a tinha está sã, limpo está;
portanto o sacerdote o declarará por limpo.

E, quando homem ou mulher tiver manchas lustrosas brancas na pele
da sua carne, então o sacerdote olhará, e eis que, se na pele
da sua carne aparecem manchas lustrosas escurecidas, é
impigem\footnote{Designação imprecisa, comum a várias dermatoses.}
que floresceu na pele, limpo está. E, quando os cabelos do
homem caírem da cabeça, calvo é, mas limpo está. E, se lhe
caírem os cabelos na frente da cabeça, meio calvo é; mas limpo está.
Porém, se na calva, ou na meia calva, houver praga branca
avermelhada, é lepra, florescendo na sua calva ou na sua meia calva.
Havendo, pois, o sacerdote examinado, e eis que, se a
inchação da praga, na sua calva ou meia calva, está branca, tirando
a vermelho, como parece a lepra na pele da carne, leproso é
aquele homem, imundo está; o sacerdote o declarará totalmente por
imundo, na sua cabeça tem a praga. Também as vestes do
leproso, em quem está a praga, serão rasgadas, e a sua cabeça será
descoberta, e cobrirá o lábio superior, e clamará: Imundo, imundo.
Todos os dias em que a praga houver nele, será imundo; imundo
está, habitará só; a sua habitação será fora do arraial.

Quando também em alguma roupa houver praga de lepra, em roupa de
lã, ou em roupa de linho, ou no fio urdido, ou no fio tecido,
seja de linho, ou seja de lã, ou em pele, ou em qualquer obra de
peles, e a praga na roupa, ou na pele, ou no fio urdido, ou
no fio tecido, ou em qualquer coisa de peles aparecer verde ou
vermelha, praga de lepra é, por isso se mostrará ao sacerdote,
e o sacerdote examinará a praga, e encerrará aquilo que tem a
praga por sete dias. Então examinará a praga ao sétimo dia;
se a praga se houver estendido na roupa, ou no fio urdido, ou no fio
tecido ou na pele, para qualquer obra que for feita da pele, lepra
roedora é, imunda está; por isso se queimará aquela roupa, ou
fio urdido, ou fio tecido de lã, ou de linho, ou de qualquer obra de
peles, em que houver a praga, porque lepra roedora é; com fogo se
queimará. Mas, o sacerdote, vendo, e eis que, se a praga não
se estendeu na roupa, ou no fio urdido, ou no tecido, ou em qualquer
obra de peles, então o sacerdote ordenará que se lave aquilo
no qual havia a praga, e o encerrará segunda vez por sete dias;
e o sacerdote, examinando a praga, depois que for lavada, e
eis que se ela não mudou o seu aspecto, nem se estendeu, imundo
está, com fogo o queimarás; praga penetrante é, seja por dentro ou
por fora. Mas se o sacerdote verificar que a praga se tem
recolhido, depois de lavada, então a rasgará da roupa, ou da pele ou
do fio urdido ou tecido; e, se ainda aparecer na roupa, ou no
fio urdido ou tecido ou em qualquer coisa de peles, lepra brotante
é; com fogo queimarás aquilo em que há a praga; mas a roupa
ou fio urdido ou tecido ou qualquer coisa de peles, que lavares, e
de que a praga se retirar, se lavará segunda vez, e será limpa.

Esta é a lei da praga da lepra na roupa de lã, ou de linho, ou do
fio urdido, ou tecido, ou de qualquer coisa de peles, para
declará-la limpa, ou para declará-la imunda.

\medskip

\lettrine{14} Depois falou o Senhor a Moisés, dizendo:
Esta será a lei do leproso no dia da sua purificação: será
levado ao sacerdote, e o sacerdote sairá fora do arraial, e o
examinará, e eis que, se a praga da lepra do leproso for sarada,
então o sacerdote ordenará que por aquele que se houver de
purificar se tomem duas aves vivas e limpas, e pau de cedro, e
carmesim, e hissopo. Mandará também o sacerdote que se degole
uma ave num vaso de barro sobre águas vivas, e tomará a ave
viva, e o pau de cedro, e o carmesim, e o hissopo, e os molhará, com
a ave viva, no sangue da ave que foi degolada sobre as águas
correntes. E sobre aquele que há de purificar-se da lepra
espargirá sete vezes; então o declarará por limpo, e soltará a ave
viva sobre a face do campo. E aquele que tem de purificar-se
lavará as suas vestes, e rapará todo o seu pêlo, e se lavará com
água; assim será limpo; e depois entrará no arraial, porém, ficará
fora da sua tenda por sete dias; e será que ao sétimo dia rapará
todo o seu pêlo, a sua cabeça, e a sua barba, e as sobrancelhas;
sim, rapará todo o pêlo, e lavará as suas vestes, e lavará a sua
carne com água, e será limpo.

E ao oitavo dia tomará dois cordeiros sem defeito, e uma cordeira
sem defeito, de um ano, e três dízimas de flor de farinha para
oferta de alimentos, amassada com azeite, e um logue de azeite;
e o sacerdote que faz a purificação apresentará o homem que
houver de purificar-se, com aquelas coisas, perante o Senhor, à
porta da tenda da congregação. E o sacerdote tomará um dos
cordeiros, e o oferecerá por expiação da culpa, e o logue de azeite;
e os oferecerá por oferta movida perante o Senhor. Então
degolará o cordeiro no lugar em que se degola a oferta da expiação
do pecado e o holocausto, no lugar santo; porque quer a oferta da
expiação da culpa como a da expiação do pecado é para o sacerdote;
coisa santíssima é. E o sacerdote tomará do sangue da
expiação da culpa, e o porá sobre a ponta da orelha direita daquele
que tem de purificar-se e sobre o dedo polegar da sua mão direita, e
no dedo polegar do seu pé direito. Também o sacerdote tomará
do logue de azeite, e o derramará na palma da sua própria mão
esquerda. Então o sacerdote molhará o seu dedo direito no
azeite que está na sua mão esquerda, e daquele azeite com o seu dedo
espargirá sete vezes perante o Senhor; e o restante do
azeite, que está na sua mão, o sacerdote porá sobre a ponta da
orelha direita daquele que tem de purificar-se, e sobre o dedo
polegar da sua mão direita, e sobre o dedo polegar do seu pé
direito, em cima do sangue da expiação da culpa; e o restante
do azeite que está na mão do sacerdote, o porá sobre a cabeça
daquele que tem de purificar-se; assim o sacerdote fará expiação por
ele perante o Senhor. Também o sacerdote fará a expiação do
pecado, e fará expiação por aquele que tem de purificar-se da sua
imundícia; e depois degolará o holocausto; e o sacerdote
oferecerá o holocausto e a oferta de alimentos sobre o altar; assim
o sacerdote fará expiação por ele, e será limpo.

Porém se for pobre, e em sua mão não houver recursos para tanto,
tomará um cordeiro para expiação da culpa em oferta de movimento,
para fazer expiação por ele, e a dízima de flor de farinha, amassada
com azeite, para oferta de alimentos, e um logue de azeite, e
duas rolas, ou dois pombinhos, conforme as suas posses, dos quais um
será para expiação do pecado, e o outro para holocausto. E ao
oitavo dia da sua purificação os trará ao sacerdote, à porta da
tenda da congregação, perante o Senhor. E o sacerdote tomará
o cordeiro da expiação da culpa, e o logue de azeite, e os oferecerá
por oferta movida perante o Senhor. Então degolará o cordeiro
da expiação da culpa, e o sacerdote tomará do sangue da expiação da
culpa, e o porá sobre a ponta da orelha direita daquele que tem de
purificar-se, e sobre o dedo polegar da sua mão direita, e sobre o
dedo polegar do seu pé direito. Também o sacerdote derramará
do azeite na palma da sua própria mão esquerda. Depois o
sacerdote com o seu dedo direito espargirá do azeite que está na sua
mão esquerda, sete vezes perante o Senhor. E o sacerdote porá
do azeite que está na sua mão na ponta da orelha direita daquele que
tem de purificar-se, e no dedo polegar da sua mão direita, e no dedo
polegar do seu pé direito; no lugar do sangue da expiação da culpa.
E o que sobejar do azeite que está na mão do sacerdote porá
sobre a cabeça daquele que tem de purificar-se, para fazer expiação
por ele perante o Senhor. Depois oferecerá uma das rolas ou
um dos pombinhos, conforme suas posses, sim, conforme as suas
posses, será um para expiação do pecado e o outro para holocausto
com a oferta de alimentos; e assim o sacerdote fará expiação por
aquele que tem de purificar-se perante o Senhor. Esta é a lei
daquele em quem estiver a praga da lepra, cujas posses não lhe
permitirem o devido para purificação.

Falou mais o Senhor a Moisés e a Arão, dizendo: Quando
tiverdes entrado na terra de Canaã que vos hei de dar por possessão,
e eu enviar a praga da lepra em alguma casa da terra da vossa
possessão, então aquele, de quem for a casa, virá e informará
ao sacerdote, dizendo: Parece-me que há como que praga em minha
casa. E o sacerdote ordenará que desocupem a casa, antes que
entre para examinar a praga, para que tudo o que está na casa não
seja contaminado; e depois entrará o sacerdote, para examinar a
casa; e, vendo a praga, e eis que se ela estiver nas paredes
da casa em covinhas verdes ou vermelhas, e parecerem mais fundas do
que a parede, então o sacerdote sairá da casa para fora da
porta, e fechá-la-á por sete dias. Depois, ao sétimo dia o
sacerdote voltará, e examinará; e se vir que a praga nas paredes da
casa se tem estendido, então o sacerdote ordenará que
arranquem as pedras, em que estiver a praga, e que as lancem fora da
cidade, num lugar imundo; e fará raspar a casa por dentro ao
redor, e o pó que houverem raspado lançarão fora da cidade, num
lugar imundo; depois tomarão outras pedras, e as porão no
lugar das primeiras pedras; e outro barro se tomará, e a casa se
rebocará. Porém, se a praga tornar a brotar na casa, depois
de arrancadas as pedras e raspada a casa, e de novo rebocada,
então o sacerdote entrará e examinará, se a praga na casa se
tem estendido, lepra roedora há na casa; imunda está.
Portanto se derribará a casa, as suas pedras, e a sua
madeira, como também todo o barro da casa; e se levará para fora da
cidade a um lugar imundo. E o que entrar naquela casa, em
qualquer dia em que estiver fechada, será imundo até à tarde.
Também o que se deitar a dormir em tal casa, lavará as suas
roupas; e o que comer em tal casa lavará as suas roupas.
Porém, tornando o sacerdote a entrar na casa e examinando-a,
se a praga não se tem estendido, depois que a casa foi rebocada, o
sacerdote a declarará por limpa, porque a praga está curada.
Depois tomará, para expiar a casa, duas aves, e pau de cedro,
e carmesim e hissopo; e degolará uma ave num vaso de barro
sobre águas correntes; então tomará pau de cedro, e o
hissopo, e o carmesim, e a ave viva, e os molhará no sangue da ave
degolada e nas águas correntes, e espargirá a casa sete vezes;
assim expiará aquela casa com o sangue da ave, e com as águas
correntes, e com a ave viva, e com o pau de cedro, e com o hissopo,
e com o carmesim. Então soltará a ave viva para fora da
cidade, sobre a face do campo; assim fará expiação pela casa, e será
limpa.

Esta é a lei de toda a praga da lepra, e da tinha, e da
lepra das roupas, e das casas, e da inchação, e das pústulas,
e das manchas lustrosas; para ensinar quando alguma coisa
será imunda, e quando será limpa. Esta é a lei da lepra.

\medskip

\lettrine{15} Falou mais o Senhor a Moisés e a Arão dizendo:
Falai aos filhos de Israel, e dizei-lhes: Qualquer homem que
tiver fluxo da sua carne, será imundo por causa do seu fluxo.
Esta, pois, será a sua imundícia, por causa do seu fluxo; se a
sua carne vasa o seu fluxo ou se a sua carne estanca o seu fluxo,
esta é a sua imundícia. Toda a cama, em que se deitar o que
tiver fluxo, será imunda; e toda a coisa, sobre o que se assentar,
será imunda. E qualquer que tocar a sua cama, lavará as suas
roupas, e se banhará em água, e será imundo até à tarde. E
aquele que se assentar sobre aquilo em que se assentou o que tem o
fluxo, lavará as suas roupas, e se banhará em água, e será imundo
até à tarde. E aquele que tocar a carne do que tem o fluxo,
lavará as suas roupas, e se banhará em água, e será imundo até à
tarde. Quando também o que tem o fluxo cuspir sobre um limpo,
então lavará este as suas roupas, e se banhará em água, e será
imundo até à tarde. Também toda a sela, em que cavalgar o que
tem o fluxo, será imunda. E qualquer que tocar em alguma
coisa que esteve debaixo dele, será imundo até à tarde; e aquele que
a levar, lavará as suas roupas, e se banhará em água, e será imundo
até à tarde. Também todo aquele em quem tocar o que tem o
fluxo, sem haver lavado as suas mãos com água, lavará as suas
roupas, e se banhará em água, e será imundo até à tarde. E o
vaso de barro, que tocar o que tem o fluxo, será quebrado; porém,
todo o vaso de madeira será lavado com água. Quando, pois, o
que tem o fluxo, estiver limpo do seu fluxo, contar-se-ão sete dias
para a sua purificação, e lavará as suas roupas, e banhará a sua
carne em águas correntes; e será limpo. E ao oitavo dia
tomará duas rolas ou dois pombinhos, e virá perante o Senhor, à
porta da tenda da congregação e os dará ao sacerdote; e o
sacerdote oferecerá um para expiação do pecado, e o outro para
holocausto; e assim o sacerdote fará por ele expiação do seu fluxo
perante o Senhor. Também o homem, quando sair dele o sêmen da
cópula, toda a sua carne banhará com água, e será imundo até à
tarde. Também toda a roupa, e toda a pele em que houver sêmen
da cópula se lavará com água, e será imundo até à tarde. E
também se um homem se deitar com a mulher e tiver emissão de sêmen,
ambos se banharão com água, e serão imundos até à tarde.

Mas a mulher, quando tiver fluxo, e o seu fluxo de sangue estiver
na sua carne, estará sete dias na sua separação, e qualquer que a
tocar, será imundo até à tarde. E tudo aquilo sobre o que ela
se deitar durante a sua separação, será imundo; e tudo sobre o que
se assentar, será imundo. E qualquer que tocar na sua cama,
lavará as suas vestes, e se banhará com água, e será imundo até à
tarde. E qualquer que tocar alguma coisa, sobre o que ela se
tiver assentado, lavará as suas vestes, e se banhará com água, e
será imundo até à tarde. Se também tocar alguma coisa que
estiver sobre a cama ou sobre aquilo em que ela se assentou, será
imundo até à tarde. E se, com efeito, qualquer homem se
deitar com ela, e a sua imundícia estiver sobre ele, imundo será por
sete dias; também toda a cama, sobre que se deitar, será imunda.
Também a mulher, quando tiver o fluxo do seu sangue, por
muitos dias fora do tempo da sua separação, ou quando tiver fluxo de
sangue por mais tempo do que a sua separação, todos os dias do fluxo
da sua imundícia será imunda, como nos dias da sua separação.
Toda a cama, sobre que se deitar todos os dias do seu fluxo,
ser-lhe-á como a cama da sua separação; e toda a coisa, sobre que se
assentar, será imunda, conforme a imundícia da sua separação.
E qualquer que a tocar será imundo; portanto lavará as suas
vestes, e se banhará com água, e será imundo até à tarde.
Porém quando for limpa do seu fluxo, então se contarão sete
dias, e depois será limpa. E ao oitavo dia tomará duas rolas,
ou dois pombinhos, e os trará ao sacerdote, à porta da tenda da
congregação. Então o sacerdote oferecerá um para expiação do
pecado, e o outro para holocausto; e o sacerdote fará por ela
expiação do fluxo da sua imundícia perante o Senhor. Assim
separareis os filhos de Israel das suas imundícias, para que não
morram nas suas imundícias, contaminando o meu tabernáculo, que está
no meio deles. Esta é a lei daquele que tem o fluxo, e
daquele de quem sai o sêmen da cópula, e que fica por eles imundo;
como também da mulher enferma na sua separação, e daquele que
padece do seu fluxo, seja homem ou mulher, e do homem que se deita
com mulher imunda.

\medskip

\lettrine{16} E falou o Senhor a Moisés, depois da morte dos
dois filhos de Arão, que morreram quando se chegaram diante do
Senhor. Disse, pois, o Senhor a Moisés: Dize a Arão, teu irmão,
que não entre no santuário em todo o tempo, para dentro do véu,
diante do propiciatório que está sobre a arca, para que não morra;
porque eu aparecerei na nuvem sobre o propiciatório. Com isto
Arão entrará no santuário: com um novilho, para expiação do pecado,
e um carneiro para holocausto. Vestirá ele a túnica santa de
linho, e terá ceroulas de linho sobre a sua carne, e cingir-se-á com
um cinto de linho, e se cobrirá com uma mitra de linho; estas são
vestes santas; por isso banhará a sua carne na água, e as vestirá.

E da congregação dos filhos de Israel tomará dois bodes para
expiação do pecado e um carneiro para holocausto. Depois Arão
oferecerá o novilho da expiação, que será para ele; e fará expiação
por si e pela sua casa. Também tomará ambos os bodes, e os porá
perante o Senhor, à porta da tenda da congregação. E Arão
lançará sortes sobre os dois bodes; uma pelo Senhor, e a outra pelo
bode emissário. Então Arão fará chegar o bode, sobre o qual cair
a sorte pelo Senhor, e o oferecerá para expiação do pecado.
Mas o bode, sobre que cair a sorte para ser bode emissário,
apresentar-se-á vivo perante o Senhor, para fazer expiação com ele,
a fim de enviá-lo ao deserto como bode emissário. E Arão fará
chegar o novilho da expiação, que será por ele, e fará expiação por
si e pela sua casa; e degolará o novilho da sua expiação.
Tomará também o incensário cheio de brasas de fogo do altar,
de diante do Senhor, e os seus punhos cheios de incenso aromático
moído, e o levará para dentro do véu. E porá o incenso sobre
o fogo perante o Senhor, e a nuvem do incenso cobrirá o
propiciatório, que está sobre o testemunho, para que não morra.
E tomará do sangue do novilho, e com o seu dedo espargirá
sobre a face do propiciatório, para o lado oriental; e perante o
propiciatório espargirá sete vezes do sangue com o seu dedo.

Depois degolará o bode, da expiação, que será pelo povo, e trará
o seu sangue para dentro do véu; e fará com o seu sangue como fez
com o sangue do novilho, e o espargirá sobre o propiciatório, e
perante a face do propiciatório. Assim fará expiação pelo
santuário por causa das imundícias dos filhos de Israel e das suas
transgressões, e de todos os seus pecados; e assim fará para a tenda
da congregação que reside com eles no meio das suas imundícias.
E nenhum homem estará na tenda da congregação quando ele
entrar para fazer expiação no santuário, até que ele saia, depois de
feita expiação por si mesmo, e pela sua casa, e por toda a
congregação de Israel. Então sairá ao altar, que está perante
o Senhor, e fará expiação por ele; e tomará do sangue do novilho, e
do sangue do bode, e o porá sobre as pontas do altar ao redor.
E daquele sangue espargirá sobre o altar, com o seu dedo,
sete vezes, e o purificará das imundícias dos filhos de Israel, e o
santificará.

Havendo, pois, acabado de fazer expiação pelo santuário, e pela
tenda da congregação, e pelo altar, então fará chegar o bode vivo.
E Arão porá ambas as suas mãos sobre a cabeça do bode vivo, e
sobre ele confessará todas as iniqüidades dos filhos de Israel, e
todas as suas transgressões, e todos os seus pecados; e os porá
sobre a cabeça do bode, e enviá-lo-á ao deserto, pela mão de um
homem designado para isso. Assim aquele bode levará sobre si
todas as iniqüidades deles à terra solitária; e deixará o bode no
deserto. Depois Arão virá à tenda da congregação, e despirá
as vestes de linho, que havia vestido quando entrara no santuário, e
ali as deixará. E banhará a sua carne em água no lugar santo,
e vestirá as suas vestes; então sairá e preparará o seu holocausto,
e o holocausto do povo, e fará expiação por si e pelo povo.
Também queimará a gordura da expiação do pecado sobre o
altar. E aquele que tiver levado o bode emissário lavará as
suas vestes, e banhará a sua carne em água; e depois entrará no
arraial. Mas o novilho da expiação, e o bode da expiação do
pecado, cujo sangue foi trazido para fazer expiação no santuário,
serão levados fora do arraial; porém as suas peles, a sua carne, e o
seu esterco queimarão com fogo. E aquele que os queimar
lavará as suas vestes, e banhará a sua carne em água; e depois
entrará no arraial.

E isto vos será por estatuto perpétuo: no sétimo mês, aos dez do
mês, afligireis as vossas almas, e nenhum trabalho fareis nem o
natural nem o estrangeiro que peregrina entre vós. Porque
naquele dia se fará expiação por vós, para purificar-vos; e sereis
purificados de todos os vossos pecados perante o Senhor. É um
sábado de descanso para vós, e afligireis as vossas almas; isto é
estatuto perpétuo. E o sacerdote, que for ungido, e que for
sagrado, para administrar o sacerdócio, no lugar de seu pai, fará a
expiação, havendo vestido as vestes de linho, as vestes santas;
assim fará expiação pelo santo santuário; também fará
expiação pela tenda da congregação e pelo altar; semelhantemente
fará expiação pelos sacerdotes e por todo o povo da congregação.
E isto vos será por estatuto perpétuo, para fazer expiação
pelos filhos de Israel de todos os seus pecados, uma vez no ano. E
fez Arão como o Senhor ordenara a Moisés.

\medskip

\lettrine{17} Falou mais o Senhor a Moisés, dizendo: Fala
a Arão e aos seus filhos, e a todos os filhos de Israel, e
dize-lhes: Esta é a palavra que o Senhor ordenou, dizendo:
Qualquer homem da casa de Israel que degolar boi, ou cordeiro,
ou cabra, no arraial, ou quem os degolar fora do arraial, e não
os trouxer à porta da tenda da congregação, para oferecer oferta ao
Senhor diante do tabernáculo do Senhor, a esse homem será imputado o
sangue; derramou sangue; por isso será extirpado do seu povo;
para que os filhos de Israel, trazendo os seus sacrifícios, que
oferecem sobre a face do campo, os tragam ao Senhor, à porta da
tenda da congregação, ao sacerdote, e os ofereçam por sacrifícios
pacíficos ao Senhor. E o sacerdote espargirá o sangue sobre o
altar do Senhor, à porta da tenda da congregação, e queimará a
gordura por cheiro suave ao Senhor. E nunca mais oferecerão os
seus sacrifícios aos demônios, após os quais eles se prostituem;
isto ser-lhes-á por estatuto perpétuo nas suas gerações.
Dize-lhes pois: Qualquer homem da casa de Israel, ou dos
estrangeiros que peregrinam entre vós, que oferecer holocausto ou
sacrifício, e não o trouxer à porta da tenda da congregação,
para oferecê-lo ao Senhor, esse homem será extirpado do seu povo.

E qualquer homem da casa de Israel, ou dos estrangeiros que
peregrinam entre eles, que comer algum sangue, contra aquela alma
porei a minha face, e a extirparei do seu povo. Porque a vida
da carne está no sangue; pelo que vo-lo tenho dado sobre o altar,
para fazer expiação pelas vossas almas; porquanto é o sangue que
fará expiação pela alma. Portanto tenho dito aos filhos de
Israel: Nenhum dentre vós comerá sangue, nem o estrangeiro, que
peregrine entre vós, comerá sangue. Também qualquer homem dos
filhos de Israel, ou dos estrangeiros que peregrinam entre eles, que
caçar animal ou ave que se come, derramará o seu sangue, e o cobrirá
com pó; porquanto a vida de toda a carne é o seu sangue; por
isso tenho dito aos filhos de Israel: Não comereis o sangue de
nenhuma carne, porque a vida de toda a carne é o seu sangue;
qualquer que o comer será extirpado. E todo o homem entre os
naturais, ou entre os estrangeiros, que comer corpo morto ou
dilacerado, lavará as suas vestes, e se banhará com água, e será
imundo até à tarde; depois será limpo. Mas, se os não lavar,
nem banhar a sua carne, levará sobre si a sua iniqüidade.

\medskip

\lettrine{18} Falou mais o Senhor a Moisés, dizendo: Fala
aos filhos de Israel, e dize-lhes: Eu sou o Senhor vosso Deus.
Não fareis segundo as obras da terra do Egito, em que
habitastes, nem fareis segundo as obras da terra de Canaã, para a
qual vos levo, nem andareis nos seus estatutos. Fareis conforme
os meus juízos, e os meus estatutos guardareis, para andardes neles.
Eu sou o Senhor vosso Deus. Portanto, os meus estatutos e os
meus juízos guardareis; os quais, observando-os o homem, viverá por
eles. Eu sou o Senhor.

Nenhum homem se chegará a qualquer parenta da sua carne, para
descobrir a sua nudez. Eu sou o Senhor. Não descobrirás a nudez
de teu pai e de tua mãe: ela é tua mãe; não descobrirás a sua nudez.
Não descobrirás a nudez da mulher de teu pai; é nudez de teu
pai. A nudez da tua irmã, filha de teu pai, ou filha de tua mãe,
nascida em casa, ou fora de casa, a sua nudez não descobrirás.
A nudez da filha do teu filho, ou da filha de tua filha, a
sua nudez não descobrirás; porque é tua nudez. A nudez da
filha da mulher de teu pai, gerada de teu pai (ela é tua irmã), a
sua nudez não descobrirás. A nudez da irmã de teu pai não
descobrirás; ela é parenta de teu pai. A nudez da irmã de tua
mãe não descobrirás; pois ela é parenta de tua mãe. A nudez
do irmão de teu pai não descobrirás; não te chegarás à sua mulher;
ela é tua tia. A nudez de tua nora não descobrirás: ela é
mulher de teu filho; não descobrirás a sua nudez. A nudez da
mulher de teu irmão não descobrirás; é a nudez de teu irmão.
A nudez de uma mulher e de sua filha não descobrirás; não
tomarás a filha de seu filho, nem a filha de sua filha, para
descobrir a sua nudez; parentas são; maldade é. E não tomarás
uma mulher juntamente com sua irmã, para fazê-la sua rival,
descobrindo a sua nudez diante dela em sua vida.

E não chegarás à mulher durante a separação da sua imundícia,
para descobrir a sua nudez, nem te deitarás com a mulher de
teu próximo para cópula, para te contaminares com ela. E da
tua descendência não darás nenhum para fazer passar pelo fogo
perante Moloque; e não profanarás o nome de teu Deus. Eu sou o
Senhor. Com homem não te deitarás, como se fosse mulher;
abominação é; nem te deitarás com um animal, para te
contaminares com ele; nem a mulher se porá perante um animal, para
ajuntar-se com ele; confusão é. Com nenhuma destas coisas vos
contamineis; porque com todas estas coisas se contaminaram as nações
que eu expulso de diante de vós. Por isso a terra está
contaminada; e eu visito a sua iniqüidade, e a terra vomita os seus
moradores. Porém vós guardareis os meus estatutos e os meus
juízos, e nenhuma destas abominações fareis, nem o natural, nem o
estrangeiro que peregrina entre vós; porque todas estas
abominações fizeram os homens desta terra, que nela estavam antes de
vós; e a terra foi contaminada. Para que a terra não vos
vomite, havendo-a contaminado, como vomitou a nação que nela estava
antes de vós. Porém, qualquer que fizer alguma destas
abominações, sim, aqueles que as fizerem serão extirpados do seu
povo. Portanto guardareis o meu mandamento, não fazendo
nenhuma das práticas abomináveis que se fizeram antes de vós, e não
vos contamineis com elas. Eu sou o Senhor vosso Deus.

\medskip

\lettrine{19} Falou mais o Senhor a Moisés, dizendo: Fala
a toda a congregação dos filhos de Israel, e dize-lhes: Santos
sereis, porque eu, o Senhor vosso Deus, sou santo. Cada um
temerá a sua mãe e a seu pai, e guardará os meus sábados. Eu sou o
Senhor vosso Deus. Não vos virareis para os ídolos nem vos
fareis deuses de fundição. Eu sou o Senhor vosso Deus. E, quando
oferecerdes sacrifício pacífico ao Senhor, da vossa própria vontade
o oferecereis. No dia em que o sacrificardes, e no dia seguinte,
se comerá; mas o que sobejar ao terceiro dia, será queimado com
fogo. E se alguma coisa dele for comida ao terceiro dia, coisa
abominável é; não será aceita. E qualquer que o comer levará a
sua iniqüidade, porquanto profanou a santidade do Senhor; por isso
tal alma será extirpada do seu povo. Quando também fizerdes a
colheita da vossa terra, o canto do teu campo não segarás
totalmente, nem as espigas caídas colherás da tua sega.
Semelhantemente não rabiscarás a tua vinha, nem colherás os
bagos caídos da tua vinha; deixá-los-ás ao pobre e ao estrangeiro.
Eu sou o Senhor vosso Deus.

Não furtareis, nem mentireis, nem usareis de falsidade cada um
com o seu próximo; nem jurareis falso pelo meu nome, pois
profanarás o nome do teu Deus. Eu sou o Senhor. Não oprimirás o
teu próximo, nem o roubarás; a paga do diarista não ficará contigo
até pela manhã. Não amaldiçoarás ao surdo, nem porás tropeço
diante do cego; mas temerás o teu Deus. Eu sou o Senhor. Não
farás injustiça no juízo; não respeitarás o pobre\footnote{AV: Ye
shall do no unrighteousness in judgment: thou shalt not respect the
person of the poor, nor honour the person of the mighty: but in
righteousness shalt thou judge thy neighbour. RA: Não farás
injustiça no juízo, nem favorecendo o pobre, nem comprazendo ao
grande; com justiça julgarás o teu próximo. RC: Não fareis injustiça
no juízo; não aceitarás o pobre, nem respeitarás o grande; com
justiça julgarás o teu próximo.}, nem honrarás o poderoso; com
justiça julgarás o teu próximo. Não andarás como mexeriqueiro
entre o teu povo; não te porás contra o sangue do teu próximo. Eu
sou o Senhor. Não odiarás a teu irmão no teu coração; não
deixarás de repreender o teu próximo, e por causa dele não sofrerás
pecado. Não te vingarás nem guardarás ira contra os filhos do
teu povo; mas amarás o teu próximo como a ti mesmo. Eu sou o Senhor.

Guardarás os meus estatutos; não permitirás que se ajuntem
misturadamente os teus animais de diferentes espécies; no teu campo
não semearás sementes diversas, e não vestirás roupa de diversos
estofos misturados. E, quando um homem se deitar com uma
mulher que for serva desposada com outro homem, e não for resgatada
nem se lhe houver dado liberdade, então serão açoitados; não
morrerão, pois ela não foi libertada. E, por expiação da sua
culpa, trará ao Senhor, à porta da tenda da congregação, um carneiro
da expiação, e, com o carneiro da expiação da culpa, o
sacerdote fará propiciação por ele perante o Senhor, pelo pecado que
cometeu; e este lhe será perdoado. E, quando tiverdes entrado
na terra, e plantardes toda a árvore de comer, ser-vos-á
incircunciso o seu fruto; três anos vos será incircunciso; dele não
se comerá. Porém no quarto ano todo o seu fruto será santo
para dar louvores ao Senhor. E no quinto ano comereis o seu
fruto, para que vos faça aumentar a sua produção. Eu sou o Senhor
vosso Deus. Não comereis coisa alguma com o sangue; não
agourareis nem adivinhareis. Não cortareis o cabelo,
arredondando os cantos da vossa cabeça, nem danificareis as
extremidades da tua barba. Pelos mortos não dareis golpes na
vossa carne; nem fareis marca alguma sobre vós. Eu sou o Senhor.
Não contaminarás a tua filha, fazendo-a prostituir-se; para
que a terra não se prostitua, nem se encha de maldade.

Guardareis os meus sábados, e o meu santuário reverenciareis. Eu
sou o Senhor. Não vos virareis para os adivinhadores e
encantadores; não os busqueis, contaminando-vos com eles. Eu sou o
Senhor vosso Deus. Diante das cãs te levantarás, e honrarás a
face do ancião; e temerás o teu Deus. Eu sou o Senhor. E
quando o estrangeiro peregrinar convosco na vossa terra, não o
oprimireis. Como um natural entre vós será o estrangeiro que
peregrina convosco; amá-lo-ás como a ti mesmo, pois estrangeiros
fostes na terra do Egito. Eu sou o Senhor vosso Deus. Não
cometereis injustiça no juízo, nem na vara, nem no peso, nem na
medida. Balanças justas, pesos justos, efa justo, e justo him
tereis. Eu sou o Senhor vosso Deus, que vos tirei da terra do Egito.
Por isso guardareis todos os meus estatutos, e todos os meus
juízos, e os cumprireis. Eu sou o Senhor.

\medskip

\lettrine{20} Falou mais o Senhor a Moisés, dizendo:
Também dirás aos filhos de Israel: Qualquer que, dos filhos de
Israel, ou dos estrangeiros que peregrinam em Israel, der da sua
descendência a Moloque, certamente morrerá; o povo da terra o
apedrejará. E eu porei a minha face contra esse homem, e o
extirparei do meio do seu povo, porquanto deu da sua descendência a
Moloque, para contaminar o meu santuário e profanar o meu santo
nome. E, se o povo da terra de alguma maneira esconder os seus
olhos daquele homem, quando der da sua descendência a Moloque, para
não o matar, então eu porei a minha face contra aquele homem, e
contra a sua família, e o extirparei do meio do seu povo, bem como a
todos que forem após ele, prostituindo-se com Moloque. Quando
alguém se virar para os adivinhadores e encantadores, para se
prostituir com eles, eu porei a minha face contra ele, e o
extirparei do meio do seu povo. Portanto santificai-vos, e sede
santos, pois eu sou o Senhor vosso Deus. E guardai os meus
estatutos, e cumpri-os. Eu sou o Senhor que vos santifica.
Quando um homem amaldiçoar a seu pai ou a sua mãe, certamente
morrerá; amaldiçoou a seu pai ou a sua mãe; o seu sangue será sobre
ele.

Também o homem que adulterar com a mulher de outro, havendo
adulterado com a mulher do seu próximo, certamente morrerá o
adúltero e a adúltera. E o homem que se deitar com a mulher
de seu pai descobriu a nudez de seu pai; ambos certamente morrerão;
o seu sangue será sobre eles. Semelhantemente, quando um
homem se deitar com a sua nora, ambos certamente morrerão; fizeram
confusão; o seu sangue será sobre eles. Quando também um
homem se deitar com outro homem, como com mulher, ambos fizeram
abominação; certamente morrerão; o seu sangue será sobre eles.
E, quando um homem tomar uma mulher e a sua mãe, maldade é; a
ele e a elas queimarão com fogo, para que não haja maldade no meio
de vós. Quando também um homem se deitar com um animal,
certamente morrerá; e matareis o animal. Também a mulher que
se chegar a algum animal, para ajuntar-se com ele, aquela mulher
matarás bem assim como o animal; certamente morrerão; o seu sangue
será sobre eles. E, quando um homem tomar a sua irmã, filha
de seu pai, ou filha de sua mãe, e vir a nudez dela, e ela a sua,
torpeza é; portanto serão extirpados aos olhos dos filhos do seu
povo; descobriu a nudez de sua irmã, levará sobre si a sua
iniqüidade. E, quando um homem se deitar com uma mulher no
tempo da sua enfermidade, e descobrir a sua nudez, descobrindo a sua
fonte, e ela descobrir a fonte do seu sangue, ambos serão extirpados
do meio do seu povo. Também a nudez da irmã de tua mãe, ou da
irmã de teu pai não descobrirás; porquanto descobriu a sua parenta,
sobre si levarão a sua iniqüidade. Quando também um homem se
deitar com a sua tia descobriu a nudez de seu tio; seu pecado sobre
si levarão; sem filhos morrerão. E quando um homem tomar a
mulher de seu irmão, imundícia é; a nudez de seu irmão descobriu;
sem filhos ficarão.

Guardai, pois, todos os meus estatutos, e todos os meus juízos, e
cumpri-os, para que não vos vomite a terra, para a qual eu vos levo
para habitar nela. E não andeis nos costumes das nações que
eu expulso de diante de vós, porque fizeram todas estas coisas;
portanto fui enfadado deles. E a vós vos tenho dito: Em
herança possuireis a sua terra, e eu a darei a vós, para a
possuirdes, terra que mana leite e mel. Eu sou o Senhor vosso Deus,
que vos separei dos povos. Fareis, pois, diferença entre os
animais limpos e imundos, e entre as aves imundas e as limpas; e as
vossas almas não fareis abomináveis por causa dos animais, ou das
aves, ou de tudo o que se arrasta sobre a terra; as quais coisas
apartei de vós, para tê-las por imundas. E ser-me-eis santos,
porque eu, o Senhor, sou santo, e vos separei dos povos, para serdes
meus. Quando, pois, algum homem ou mulher em si tiver um
espírito de necromancia ou espírito de adivinhação, certamente
morrerá; serão apedrejados; o seu sangue será sobre eles.

\medskip

\lettrine{21} Depois disse o Senhor a Moisés: Fala aos
sacerdotes, filhos de Arão, e dize-lhes: O sacerdote não se
contaminará por causa de um morto entre o seu povo, salvo por
seu parente mais chegado: por sua mãe, e por seu pai, e por seu
filho, e por sua filha, e por seu irmão. E por sua irmã virgem,
chegada a ele, que ainda não teve marido; por ela também se
contaminará. Ele sendo principal entre o seu povo, não se
contaminará, pois que se profanaria. Não farão calva na sua
cabeça, e não raparão as extremidades da sua barba, nem darão golpes
na sua carne. Santos serão a seu Deus, e não profanarão o nome
do seu Deus, porque oferecem as ofertas queimadas do Senhor, e o pão
do seu Deus; portanto serão santos. Não tomarão mulher
prostituta ou desonrada, nem tomarão mulher repudiada de seu marido;
pois santo é a seu Deus. Portanto o santificarás, porquanto
oferece o pão do teu Deus; santo será para ti, pois eu, o Senhor que
vos santifica, sou santo. E quando a filha de um sacerdote
começar a prostituir-se, profana a seu pai; com fogo será queimada.

E o sumo sacerdote entre seus irmãos, sobre cuja cabeça foi
derramado o azeite da unção, e que for consagrado para vestir as
vestes, não descobrirá a sua cabeça nem rasgará as suas vestes;
e não se chegará a cadáver algum, nem por causa de seu pai
nem por sua mãe se contaminará; nem sairá do santuário, para
que não profane o santuário do seu Deus, pois a coroa do azeite da
unção do seu Deus está sobre ele. Eu sou o Senhor. E ele
tomará por esposa uma mulher na sua virgindade. Viúva, ou
repudiada ou desonrada ou prostituta, estas não tomará; mas virgem
do seu povo tomará por mulher. E não profanará a sua
descendência entre o seu povo; porque eu sou o Senhor que o
santifico.

Falou mais o Senhor a Moisés, dizendo: Fala a Arão,
dizendo: Ninguém da tua descendência, nas suas gerações, em que
houver algum defeito, se chegará a oferecer o pão do seu Deus.
Pois nenhum homem em quem houver alguma deformidade se
chegará; como homem cego, ou coxo, ou de nariz chato, ou de membros
demasiadamente compridos, ou homem que tiver quebrado o pé,
ou a mão quebrada, ou corcunda, ou anão, ou que tiver defeito
no olho, ou sarna, ou impigem, ou que tiver testículo mutilado.
Nenhum homem da descendência de Arão, o sacerdote, em quem
houver alguma deformidade, se chegará para oferecer as ofertas
queimadas do Senhor; defeito nele há; não se chegará para oferecer o
pão do seu Deus. Ele comerá do pão do seu Deus, tanto do
santíssimo como do santo. Porém até ao véu não entrará, nem
se chegará ao altar, porquanto defeito há nele, para que não profane
os meus santuários; porque eu sou o Senhor que os santifico.
E Moisés falou isto a Arão e a seus filhos, e a todos os
filhos de Israel.

\medskip

\lettrine{22} Depois falou o Senhor a Moisés, dizendo:
Dize a Arão e a seus filhos que se apartem das coisas santas dos
filhos de Israel, que a mim me santificam, para que não profanem o
meu santo nome. Eu sou o Senhor. Dize-lhes: Todo o homem, que
entre as vossas gerações, de toda a vossa descendência, se chegar às
coisas santas que os filhos de Israel santificam ao Senhor, tendo
sobre si a sua imundícia, aquela alma será extirpada de diante da
minha face. Eu sou o Senhor. Ninguém da descendência de Arão,
que for leproso, ou tiver fluxo, comerá das coisas santas, até que
seja limpo; como também o que tocar alguma coisa imunda de cadáver,
ou aquele de que sair sêmen da cópula, ou qualquer que tocar a
algum réptil, pelo qual se fez imundo, ou a algum homem, pelo qual
se fez imundo, segundo toda a sua imundícia; o homem que o tocar
será imundo até à tarde, e não comerá das coisas santas, mas banhará
a sua carne em água. E havendo-se o sol já posto, então será
limpo, e depois comerá das coisas santas; porque este é o seu pão.
O corpo morto e o dilacerado não comerá, para que não se
contamine com ele. Eu sou o Senhor. Guardarão, pois, o meu
mandamento, para que por isso não levem pecado, e morram nele,
havendo-o profanado. Eu sou o Senhor que os santifico.

Também nenhum estranho comerá das coisas santas; nem o hóspede do
sacerdote, nem o diarista comerá das coisas santas. Mas
quando o sacerdote comprar alguma pessoa com o seu dinheiro, aquela
comerá delas, e os nascidos na sua casa, estes comerão do seu pão.
E, quando a filha do sacerdote se casar com homem estranho,
ela não comerá da oferta das coisas santas. Mas quando a
filha do sacerdote for viúva ou repudiada, e não tiver filho, e se
houver tornado à casa de seu pai, como na sua mocidade, do pão de
seu pai comerá; mas nenhum estranho comerá dele. E quando
alguém por erro comer a coisa santa, sobre ela acrescentará uma
quinta parte, e a dará ao sacerdote com a coisa santa. Assim
não profanarão as coisas santas dos filhos de Israel, que oferecem
ao Senhor, nem os farão levar a iniqüidade da culpa, comendo
as suas coisas santas; pois eu sou o Senhor que as santifico.

Falou mais o Senhor a Moisés, dizendo: Fala a Arão, e a seus
filhos, e a todos os filhos de Israel, e dize-lhes: Qualquer que, da
casa de Israel, ou dos estrangeiros em Israel, oferecer a sua
oferta, quer dos seus votos, quer das suas ofertas voluntárias, que
oferecem ao Senhor em holocausto, segundo a sua vontade,
oferecerá macho sem defeito, ou dos bois, ou dos cordeiros, ou das
cabras. Nenhuma coisa em que haja defeito oferecereis, porque
não seria aceita em vosso favor. E, quando alguém oferecer
sacrifício pacífico ao Senhor, separando dos bois ou das ovelhas um
voto, ou oferta voluntária, sem defeito será, para que seja aceito;
nenhum defeito haverá nele. O cego, ou quebrado, ou aleijado, o
verrugoso, ou sarnoso, ou cheio de impigens, estes não oferecereis
ao Senhor, e deles não poreis oferta queimada ao Senhor sobre o
altar. Porém boi, ou gado miúdo, comprido ou curto de membros,
poderás oferecer por oferta voluntária, mas por voto não será
aceito. O machucado, ou moído, ou despedaçado, ou cortado, não
oferecereis ao Senhor; não fareis isto na vossa terra. Também
da mão do estrangeiro nenhum alimento oferecereis ao vosso Deus, de
todas estas coisas, pois a sua corrupção está nelas; defeito nelas
há; não serão aceitas em vosso favor. Falou mais o Senhor a
Moisés, dizendo: Quando nascer o boi, ou cordeiro, ou cabra,
sete dias estará debaixo de sua mãe; depois, desde o oitavo dia em
diante, será aceito por oferta queimada ao Senhor. Também boi
ou gado miúdo, a ele e a seu filho não degolareis no mesmo dia. 
E, quando oferecerdes sacrifícios de louvores ao Senhor, o
oferecereis da vossa vontade. No mesmo dia se comerá; dele nada
deixareis ficar até pela manhã. Eu sou o Senhor. Por isso
guardareis os meus mandamentos, e os cumprireis. Eu sou o Senhor.
E não profanareis o meu santo nome, para que eu seja
santificado no meio dos filhos de Israel. Eu sou o Senhor que vos
santifico; que vos tirei da terra do Egito, para ser o vosso
Deus. Eu sou o Senhor.

\medskip

\lettrine{23} Depois falou o Senhor a Moisés, dizendo:
Fala aos filhos de Israel, e dize-lhes: As solenidades do
Senhor, que convocareis, serão santas convocações; estas são as
minhas solenidades: Seis dias trabalho se fará, mas o sétimo dia
será o sábado do descanso, santa convocação; nenhum trabalho fareis;
sábado do Senhor é em todas as vossas habitações.

Estas são as solenidades do Senhor, as santas convocações, que
convocareis ao seu tempo determinado: No mês primeiro, aos
catorze do mês, pela tarde, é a páscoa do Senhor. E aos quinze
dias deste mês é a festa dos pães ázimos do Senhor; sete dias
comereis pães ázimos. No primeiro dia tereis santa convocação;
nenhum trabalho servil fareis; mas sete dias oferecereis oferta
queimada ao Senhor; ao sétimo dia haverá santa convocação; nenhum
trabalho servil fareis. E falou o Senhor a Moisés, dizendo:
Fala aos filhos de Israel, e dize-lhes: Quando houverdes
entrado na terra, que vos hei de dar, e fizerdes a sua colheita,
então trareis um molho das primícias da vossa sega ao sacerdote;
e ele moverá o molho perante o Senhor, para que sejais
aceitos; no dia seguinte ao sábado o sacerdote o moverá. E no
dia em que moverdes o molho, preparareis um cordeiro sem defeito, de
um ano, em holocausto ao Senhor, e a sua oferta de alimentos,
será de duas dízimas de flor de farinha, amassada com azeite, para
oferta queimada em cheiro suave ao Senhor, e a sua libação será de
vinho, um quarto de him. E não comereis pão, nem trigo
tostado, nem espigas verdes, até aquele mesmo dia em que trouxerdes
a oferta do vosso Deus; estatuto perpétuo é por vossas gerações, em
todas as vossas habitações.

Depois para vós contareis desde o dia seguinte ao sábado, desde o
dia em que trouxerdes o molho da oferta movida; sete semanas
inteiras serão. Até ao dia seguinte ao sétimo sábado,
contareis cinqüenta dias; então oferecereis nova oferta de alimentos
ao Senhor. Das vossas habitações trareis dois pães de
movimento; de duas dízimas de farinha serão, levedados se cozerão;
primícias são ao Senhor. Também com o pão oferecereis sete
cordeiros sem defeito, de um ano, e um novilho, e dois carneiros;
holocausto serão ao Senhor, com a sua oferta de alimentos, e as suas
libações, por oferta queimada de cheiro suave ao Senhor.
Também oferecereis um bode para expiação do pecado, e dois
cordeiros de um ano por sacrifício pacífico. Então o
sacerdote os moverá com o pão das primícias por oferta movida
perante o Senhor, com os dois cordeiros; santos serão ao Senhor para
uso do sacerdote. E naquele mesmo dia apregoareis que tereis
santa convocação; nenhum trabalho servil fareis; estatuto perpétuo é
em todas as vossas habitações pelas vossas gerações. E,
quando fizerdes a colheita da vossa terra, não acabarás de segar os
cantos do teu campo, nem colherás as espigas caídas da tua sega;
para o pobre e para o estrangeiro as deixarás. Eu sou o Senhor vosso
Deus.

E falou o Senhor a Moisés, dizendo: 24 Fala aos filhos de Israel,
dizendo: No mês sétimo, ao primeiro do mês, tereis descanso,
memorial com sonido de trombetas, santa convocação. Nenhum
trabalho servil fareis, mas oferecereis oferta queimada ao Senhor.
Falou mais o Senhor a Moisés, dizendo: Mas aos dez
dias desse sétimo mês será o dia da expiação; tereis santa
convocação, e afligireis as vossas almas; e oferecereis oferta
queimada ao Senhor. E naquele mesmo dia nenhum trabalho
fareis, porque é o dia da expiação, para fazer expiação por vós
perante o Senhor vosso Deus. Porque toda a alma, que naquele
mesmo dia se não afligir, será extirpada do seu povo. Também
toda a alma, que naquele mesmo dia fizer algum trabalho, eu a
destruirei do meio do seu povo. Nenhum trabalho fareis;
estatuto perpétuo é pelas vossas gerações em todas as vossas
habitações. Sábado de descanso vos será; então afligireis as
vossas almas; aos nove do mês à tarde, de uma tarde a outra tarde,
celebrareis o vosso sábado.

E falou o Senhor a Moisés, dizendo: Fala aos filhos de
Israel, dizendo: Aos quinze dias deste mês sétimo será a festa dos
tabernáculos ao Senhor por sete dias. Ao primeiro dia haverá
santa convocação; nenhum trabalho servil fareis. Sete dias
oferecereis ofertas queimadas ao Senhor; ao oitavo dia tereis santa
convocação, e oferecereis ofertas queimadas ao Senhor; dia de
proibição é, nenhum trabalho servil fareis. Estas são as
solenidades do Senhor, que apregoareis para santas convocações, para
oferecer ao Senhor oferta queimada, holocausto e oferta de
alimentos, sacrifício e libações, cada qual em seu dia próprio;
além dos sábados do Senhor, e além dos vossos dons, e além de
todos os vossos votos, e além de todas as vossas ofertas
voluntárias, que dareis ao Senhor. Porém aos quinze dias do
mês sétimo, quando tiverdes recolhido do fruto da terra, celebrareis
a festa do Senhor por sete dias; no primeiro dia haverá descanso, e
no oitavo dia haverá descanso. E no primeiro dia tomareis
para vós ramos de formosas árvores, ramos de palmeiras, ramos de
árvores frondosas, e salgueiros de ribeiras; e vos alegrareis
perante o Senhor vosso Deus por sete dias. E celebrareis esta
festa ao Senhor por sete dias cada ano; estatuto perpétuo é pelas
vossas gerações; no mês sétimo a celebrareis. Sete dias
habitareis em tendas; todos os naturais em Israel habitarão em
tendas; para que saibam as vossas gerações que eu fiz habitar
os filhos de Israel em tendas, quando os tirei da terra do Egito. Eu
sou o Senhor vosso Deus. Assim pronunciou Moisés as
solenidades do Senhor aos filhos de Israel.

\medskip

\lettrine{24} E falou o Senhor a Moisés, dizendo: Ordena
aos filhos de Israel que te tragam azeite de oliveira, puro, batido,
para a luminária, para manter as lâmpadas acesas continuamente.
Arão as porá em ordem perante o Senhor continuamente, desde a
tarde até à manhã, fora do véu do testemunho, na tenda da
congregação; estatuto perpétuo é pelas vossas gerações. Sobre o
candelabro de ouro puro porá em ordem as lâmpadas perante o Senhor
continuamente. Também tomarás da flor de farinha, e dela cozerás
doze pães; cada pão será de duas dízimas de um efa. E os porás
em duas fileiras, seis em cada fileira, sobre a mesa pura, perante o
Senhor. E sobre cada fileira porás incenso puro, para que seja,
para o pão, por oferta memorial; oferta queimada é ao Senhor. Em
cada dia de sábado, isto se porá em ordem perante o Senhor
continuamente, pelos filhos de Israel, por aliança perpétua. E
será de Arão e de seus filhos, os quais o comerão no lugar santo,
porque uma coisa santíssima é para eles, das ofertas queimadas ao
Senhor, por estatuto perpétuo.

E apareceu, no meio dos filhos de Israel, o filho de uma mulher
israelita, o qual era filho de um homem egípcio; e o filho da
israelita e um homem israelita discutiram no arraial. Então o
filho da mulher israelita blasfemou o nome do Senhor, e o
amaldiçoou, por isso o trouxeram a Moisés; e o nome de sua mãe era
Selomite, filha de Dibri, da tribo de Dã. E eles o puseram na
prisão, até que a vontade do Senhor lhes pudesse ser declarada.
E falou o Senhor a Moisés, dizendo: Tira o que tem
blasfemado para fora do arraial; e todos os que o ouviram porão as
suas mãos sobre a sua cabeça; então toda a congregação o apedrejará.
E aos filhos de Israel falarás, dizendo: Qualquer que
amaldiçoar o seu Deus, levará sobre si o seu pecado. E aquele
que blasfemar o nome do Senhor, certamente morrerá; toda a
congregação certamente o apedrejará; assim o estrangeiro como o
natural, blasfemando o nome do Senhor, será morto. E quem
matar a alguém certamente morrerá. Mas quem matar um animal,
o restituirá, vida por vida. Quando também alguém desfigurar
o seu próximo, como ele fez, assim lhe será feito: Quebradura
por quebradura, olho por olho, dente por dente; como ele tiver
desfigurado a algum homem, assim se lhe fará. Quem, pois,
matar um animal, restituí-lo-á, mas quem matar um homem será morto.
Uma mesma lei tereis; assim será para o estrangeiro como para
o natural; pois eu sou o Senhor vosso Deus. E disse Moisés,
aos filhos de Israel que levassem o que tinha blasfemado para fora
do arraial, e o apedrejassem; e fizeram os filhos de Israel como o
Senhor ordenara a Moisés.

\medskip

\lettrine{25} Falou mais o Senhor a Moisés no monte Sinai,
dizendo: Fala aos filhos de Israel, e dize-lhes: Quando tiverdes
entrado na terra, que eu vos dou, então a terra descansará um sábado
ao Senhor. Seis anos semearás a tua terra, e seis anos podarás a
tua vinha, e colherás os seus frutos; porém ao sétimo ano haverá
sábado de descanso para a terra, um sábado ao Senhor; não semearás o
teu campo nem podarás a tua vinha. O que nascer de si mesmo da
tua sega, não colherás, e as uvas da tua separação não vindimarás;
ano de descanso será para a terra. Mas os frutos do sábado da
terra vos serão por alimento, a ti, e ao teu servo, e à tua serva, e
ao teu diarista, e ao estrangeiro que peregrina contigo; e ao
teu gado, e aos teus animais, que estão na tua terra, todo o seu
produto será por mantimento.

Também contarás sete semanas de anos, sete vezes sete anos; de
maneira que os dias das sete semanas de anos te serão quarenta e
nove anos. Então no mês sétimo, aos dez do mês, farás passar a
trombeta do jubileu; no dia da expiação fareis passar a trombeta por
toda a vossa terra, e santificareis o ano qüinquagésimo, e
apregoareis liberdade na terra a todos os seus moradores; ano de
jubileu vos será, e tornareis, cada um à sua possessão, e cada um à
sua família. O ano qüinquagésimo vos será jubileu; não
semeareis nem colhereis o que nele nascer de si mesmo, nem nele
vindimareis as uvas das separações, porque jubileu é, santo
será para vós; a novidade do campo comereis. Neste ano do
jubileu tornareis cada um à sua possessão. E quando venderdes
alguma coisa ao vosso próximo, ou a comprardes da mão do vosso
próximo, ninguém engane a seu irmão; conforme ao número dos
anos, desde o jubileu, comprarás ao teu próximo; e conforme o número
dos anos das colheitas, ele a venderá a ti. Conforme se
multipliquem os anos, aumentarás o seu preço, e conforme à
diminuição dos anos abaixarás o seu preço; porque conforme o número
das colheitas é que ele te vende. Ninguém, pois, engane ao
seu próximo; mas terás temor do teu Deus; porque eu sou o Senhor
vosso Deus. E observareis os meus estatutos, e guardareis os
meus juízos, e os cumprireis; assim habitareis seguros na terra.
E a terra dará o seu fruto, e comereis a fartar, e nela
habitareis seguros. E se disserdes: Que comeremos no ano
sétimo? eis que não havemos de semear nem fazer a nossa colheita;
então eu mandarei a minha bênção sobre vós no sexto ano, para
que dê fruto por três anos, e no oitavo ano semeareis, e
comereis da colheita velha até ao ano nono; até que venha a nova
colheita, comereis a velha.

Também a terra não se venderá em perpetuidade, porque a terra é
minha; pois vós sois estrangeiros e peregrinos comigo.
Portanto em toda a terra da vossa possessão dareis resgate à
terra. Quando teu irmão empobrecer e vender alguma parte da
sua possessão, então virá o seu resgatador, seu parente, e resgatará
o que vendeu seu irmão. E se alguém não tiver resgatador,
porém conseguir o suficiente para o seu resgate, então
contará os anos desde a sua venda, e o que ficar restituirá ao homem
a quem a vendeu, e tornará à sua possessão. Mas se não
conseguir o suficiente para restituir-lha, então a que foi vendida
ficará na mão do comprador até ao ano do jubileu; porém no ano do
jubileu sairá, e ele tornará à sua possessão. E, quando
alguém vender uma casa de moradia em cidade murada, então poderá
resgatá-la até que se cumpra o ano da sua venda; durante um ano
inteiro será lícito o seu resgate. Mas, se, cumprindo-se-lhe
um ano inteiro, ainda não for resgatada, então a casa, que estiver
na cidade que tem muro, em perpetuidade ficará ao que a comprou,
pelas suas gerações; não sairá no jubileu. Mas as casas das
aldeias que não têm muro ao redor, serão estimadas como o campo da
terra; para elas haverá resgate, e sairão no jubileu. Mas, no
tocante às cidades dos levitas, às casas das cidades da sua
possessão, direito perpétuo de resgate terão os levitas. E se
alguém comprar dos levitas, uma casa, a casa comprada e a cidade da
sua possessão sairão do poder do comprador no jubileu; porque as
casas das cidades dos levitas são a sua possessão no meio dos filhos
de Israel. Mas o campo do arrabalde das suas cidades não se
venderá, porque lhes é possessão perpétua. E, quando teu
irmão empobrecer, e as suas forças decaírem, então sustentá-lo-ás,
como estrangeiro e peregrino viverá contigo. Não tomarás dele
juros, nem ganho; mas do teu Deus terás temor, para que teu irmão
viva contigo. Não lhe darás teu dinheiro com usura, nem darás
do teu alimento por interesse. Eu sou o Senhor vosso Deus,
que vos tirei da terra do Egito, para vos dar a terra de Canaã, para
ser vosso Deus.

Quando também teu irmão empobrecer, estando ele contigo, e
vender-se a ti, não o farás servir como escravo. Como
diarista, como peregrino estará contigo; até ao ano do jubileu te
servirá; então sairá do teu serviço, ele e seus filhos com
ele, e tornará à sua família e à possessão de seus pais.
Porque são meus servos, que tirei da terra do Egito; não
serão vendidos como se vendem os escravos. Não te
assenhorearás dele com rigor, mas do teu Deus terás temor. E
quanto a teu escravo ou a tua escrava que tiveres, serão das nações
que estão ao redor de vós; deles comprareis escravos e escravas.
Também os comprareis dos filhos dos forasteiros que
peregrinam entre vós, deles e das suas famílias que estiverem
convosco, que tiverem gerado na vossa terra; e vos serão por
possessão. E possuí-los-eis por herança para vossos filhos
depois de vós, para herdarem a possessão; perpetuamente os fareis
servir; mas sobre vossos irmãos, os filhos de Israel, não vos
assenhoreareis com rigor, uns sobre os outros. E se o
estrangeiro ou peregrino que está contigo alcançar riqueza, e teu
irmão, que está com ele, empobrecer, e vender-se ao estrangeiro ou
peregrino que está contigo, ou a alguém da família do estrangeiro,
depois que se houver vendido, haverá resgate para ele; um de
seus irmãos o poderá resgatar; ou seu tio, ou o filho de seu
tio o poderá resgatar; ou um dos seus parentes, da sua família, o
poderá resgatar; ou, se alcançar riqueza, se resgatará a si mesmo.
E acertará com aquele que o comprou, desde o ano que se
vendeu a ele até ao ano do jubileu, e o preço da sua venda será
conforme o número dos anos; conforme os dias de um diarista estará
com ele. Se ainda faltarem muitos anos, conforme a eles
restituirá, para seu resgate, parte do dinheiro pelo qual foi
vendido, e se ainda restarem poucos anos até ao ano do
jubileu, então fará contas com ele; segundo os seus anos restituirá
o seu resgate. Como diarista, de ano em ano, estará com ele;
não se assenhoreará sobre ele com rigor diante dos teus olhos.
E, se desta sorte não se resgatar, sairá no ano do jubileu,
ele e seus filhos com ele. Porque os filhos de Israel me são
servos; meus servos são eles, que tirei da terra do Egito. Eu sou o
Senhor vosso Deus.


\medskip

\lettrine{26} Não fareis para vós ídolos, nem vos levantareis
imagem de escultura, nem estátua, nem poreis pedra figurada na vossa
terra, para inclinar-vos a ela; porque eu sou o Senhor vosso Deus.
Guardareis os meus sábados, e reverenciareis o meu santuário. Eu
sou o Senhor. Se andardes nos meus estatutos, e guardardes os
meus mandamentos, e os cumprirdes, então eu vos darei as chuvas
a seu tempo; e a terra dará a sua colheita, e a árvore do campo dará
o seu fruto; e a debulha se vos chegará à vindima, e a vindima
se chegará à sementeira; e comereis o vosso pão a fartar, e
habitareis seguros na vossa terra. Também darei paz na terra, e
dormireis seguros, e não haverá quem vos espante; e farei cessar os
animais nocivos da terra, e pela vossa terra não passará espada.
E perseguireis os vossos inimigos, e cairão à espada diante de
vós. Cinco de vós perseguirão a um cento deles, e cem de vós
perseguirão a dez mil; e os vossos inimigos cairão à espada diante
de vós. E para vós olharei, e vos farei frutificar, e vos
multiplicarei, e confirmarei a minha aliança convosco. E
comereis da colheita velha, há muito tempo guardada, e tirareis fora
a velha por causa da nova. E porei o meu tabernáculo no meio
de vós, e a minha alma de vós não se enfadará. E andarei no
meio de vós, e eu vos serei por Deus, e vós me sereis por povo.
Eu sou o Senhor vosso Deus, que vos tirei da terra dos
egípcios, para que não fôsseis seus escravos; e quebrei os
timões\footnote{Timão: peça longa do arado ou do carro à qual se
atrelam os animais que os puxam; tiradoura.} do vosso jugo, e vos
fiz andar eretos.

Mas, se não me ouvirdes, e não cumprirdes todos estes
mandamentos, e se rejeitardes os meus estatutos, e a vossa
alma se enfadar dos meus juízos, não cumprindo todos os meus
mandamentos, para invalidar a minha aliança, então eu também
vos farei isto: porei sobre vós terror, a tísica e a febre ardente,
que consumam os olhos e atormentem a alma; e semeareis em vão a
vossa semente, pois os vossos inimigos a comerão. E porei a
minha face contra vós, e sereis feridos diante de vossos inimigos; e
os que vos odeiam, de vós se assenhorearão, e fugireis, sem ninguém
vos perseguir. E, se ainda com estas coisas não me ouvirdes,
então eu prosseguirei a castigar-vos sete vezes mais, por causa dos
vossos pecados. Porque quebrarei a soberba da vossa força; e
farei que os vossos céus sejam como ferro e a vossa terra como
cobre. E em vão se gastará a vossa força; a vossa terra não
dará a sua colheita, e as árvores da terra não darão o seu fruto.
E se andardes contrariamente para comigo, e não me quiserdes
ouvir, trar-vos-ei pragas sete vezes mais, conforme os vossos
pecados. Porque enviarei entre vós as feras do campo, as
quais vos desfilharão, e desfarão o vosso gado, e vos diminuirão; e
os vossos caminhos serão desertos. Se ainda com estas coisas
não vos corrigirdes voltando para mim, mas ainda andardes
contrariamente para comigo, eu também andarei contrariamente
para convosco, e eu, eu mesmo, vos ferirei sete vezes mais por causa
dos vossos pecados. Porque trarei sobre vós a espada, que
executará a vingança da aliança; e ajuntados sereis nas vossas
cidades; então enviarei a peste entre vós, e sereis entregues na mão
do inimigo. Quando eu vos quebrar o sustento do pão, então
dez mulheres cozerão o vosso pão num só forno, e devolver-vos-ão o
vosso pão por peso; e comereis, mas não vos fartareis. E se
com isto não me ouvirdes, mas ainda andardes contrariamente para
comigo, também eu para convosco andarei contrariamente em
furor; e vos castigarei sete vezes mais por causa dos vossos
pecados. Porque comereis a carne de vossos filhos, e a carne
de vossas filhas. E destruirei os vossos altos, e desfarei as
vossas imagens, e lançarei os vossos cadáveres sobre os cadáveres
dos vossos deuses; a minha alma se enfadará de vós. E
reduzirei as vossas cidades a deserto, e assolarei os vossos
santuários, e não cheirarei o vosso cheiro suave. E assolarei
a terra e se espantarão disso os vossos inimigos que nela morarem.
E espalhar-vos-ei entre as nações, e desembainharei a espada
atrás de vós; e a vossa terra será assolada, e as vossas cidades
serão desertas. Então a terra folgará nos seus sábados, todos
os dias da sua assolação, e vós estareis na terra dos vossos
inimigos; então a terra descansará, e folgará nos seus sábados.
Todos os dias da assolação descansará, porque não descansou
nos vossos sábados, quando habitáveis nela. E, quanto aos que
de vós ficarem, eu porei tal pavor nos seus corações, nas terras dos
seus inimigos, que o ruído de uma folha movida os perseguirá; e
fugirão como quem foge da espada; e cairão sem ninguém os perseguir.
E cairão uns sobre os outros como diante da espada, sem
ninguém os perseguir; e não podereis resistir diante dos vossos
inimigos. E perecereis entre as nações, e a terra dos vossos
inimigos vos consumirá. E aqueles que entre vós ficarem se
consumirão pela sua iniqüidade nas terras dos vossos inimigos, e
pela iniqüidade de seus pais com eles se consumirão.

Então confessarão a sua iniqüidade, e a iniqüidade de seus pais,
com as suas transgressões, com que transgrediram contra mim; como
também eles andaram contrariamente para comigo. Eu também
andei para com eles contrariamente, e os fiz entrar na terra dos
seus inimigos; se então o seu coração incircunciso se humilhar, e
então tomarem por bem o castigo da sua iniqüidade, também eu
me lembrarei da minha aliança com Jacó, e também da minha aliança
com Isaque, e também da minha aliança com Abraão me lembrarei, e da
terra me lembrarei. E a terra será abandonada por eles, e
folgará nos seus sábados, sendo assolada por causa deles; e tomarão
por bem o castigo da sua iniqüidade, em razão mesmo de que
rejeitaram os meus juízos e a sua alma se enfastiou dos meus
estatutos. E, demais disto também, estando eles na terra dos
seus inimigos, não os rejeitarei nem me enfadarei deles, para
consumi-los e invalidar a minha aliança com eles, porque eu sou o
Senhor seu Deus. Antes por amor deles me lembrarei da aliança
com os seus antepassados, que tirei da terra do Egito perante os
olhos dos gentios, para lhes ser por Deus. Eu sou o Senhor.
Estes são os estatutos, e os juízos, e as leis que deu o
Senhor entre si e os filhos de Israel, no monte Sinai, pela mão de
Moisés.

\medskip

\lettrine{27} Falou mais o Senhor a Moisés, dizendo: Fala
aos filhos de Israel, e dize-lhes: Quando alguém fizer particular
voto, segundo a tua avaliação serão as pessoas ao Senhor. Se for
a tua avaliação de um homem, da idade de vinte anos até a idade de
sessenta, será a tua avaliação de cinqüenta siclos de prata, segundo
o siclo do santuário. Porém, se for mulher, a tua avaliação será
de trinta siclos. E, se for de cinco anos até vinte, a tua
avaliação de um homem será vinte siclos e da mulher dez siclos.
E, se for de um mês até cinco anos, a tua avaliação de um homem
será de cinco siclos de prata, e a tua avaliação pela mulher será de
três siclos de prata. E, se for de sessenta anos e acima, pelo
homem a tua avaliação será de quinze siclos e pela mulher dez
siclos. Mas, se for mais pobre do que a tua avaliação, então
apresentar-se-á diante do sacerdote, para que o sacerdote o avalie;
conforme as posses daquele que fez o voto, o avaliará o sacerdote.
E, se for animal dos que se oferecem em oferta ao Senhor, tudo
quanto der dele ao Senhor será santo. Não o mudará, nem o
trocará bom por mau, ou mau por bom; se porém de alguma maneira
trocar animal por animal, tanto um como o outro, será santo.
E, se for algum animal imundo, dos que não se oferecem em
oferta ao Senhor, então apresentará o animal diante do sacerdote,
e o sacerdote o avaliará, seja bom ou seja mau; segundo a
avaliação do sacerdote, assim será. Porém, se de alguma
maneira o resgatar, então acrescentará a sua quinta parte sobre a
tua avaliação.

E quando alguém santificar a sua casa para ser santa ao Senhor, o
sacerdote a avaliará, seja boa ou seja má; como o sacerdote a
avaliar, assim será. Mas, se o que a santificou resgatar a
sua casa, então acrescentará a quinta parte do dinheiro sobre a tua
avaliação, e será sua. Se também alguém santificar ao Senhor
uma parte do campo da sua possessão, então a tua avaliação será
segundo a sua semente: um ômer de semente de cevada será avaliado
por cinqüenta siclos de prata. Se santificar o seu campo
desde o ano do jubileu, conforme à tua avaliação ficará. Mas,
se santificar o seu campo depois do ano do jubileu, então o
sacerdote lhe contará o dinheiro conforme aos anos restantes até ao
ano do jubileu, e isto se abaterá da tua avaliação. E se
aquele que santificou o campo de alguma maneira o resgatar, então
acrescentará a quinta parte do dinheiro da tua avaliação, e ficará
seu. E se não resgatar o campo, ou se vender o campo a outro
homem, nunca mais se resgatará. Porém havendo o campo saído
no ano do jubileu, será santo ao Senhor, como campo consagrado; a
possessão dele será do sacerdote. E se alguém santificar ao
Senhor o campo que comprou, e não for parte do campo da sua
possessão, então o sacerdote lhe contará o valor da tua
avaliação até ao ano do jubileu; e no mesmo dia dará a tua avaliação
como coisa santa ao Senhor. No ano do jubileu o campo tornará
àquele de quem o comprou, àquele de quem era a possessão do campo.
E toda a tua avaliação se fará conforme ao siclo do
santuário; o siclo será de vinte geras.

Mas o primogênito de um animal, por já ser do Senhor ninguém o
santificará; seja boi ou gado miúdo, do Senhor é. Mas, se for
de um animal imundo, o resgatará, segundo a tua estimação, e sobre
ele acrescentará a sua quinta parte; e se não se resgatar,
vender-se-á segundo a tua estimação. Todavia, nenhuma coisa
consagrada, que alguém consagrar ao Senhor de tudo o que tem, de
homem, ou de animal, ou do campo da sua possessão, se venderá nem
resgatará; toda a coisa consagrada será santíssima ao Senhor.
Toda a coisa consagrada que for consagrada do homem, não será
resgatada; certamente morrerá. Também todas as dízimas do
campo, da semente do campo, do fruto das árvores, são do Senhor;
santas são ao Senhor. Porém, se alguém das suas dízimas
resgatar alguma coisa, acrescentará a sua quinta parte sobre ela.
No tocante a todas as dízimas do gado e do rebanho, tudo o
que passar debaixo da vara, o dízimo será santo ao Senhor.
Não se investigará entre o bom e o mau, nem o trocará; mas,
se de alguma maneira o trocar, tanto um como o outro será santo; não
serão resgatados. Estes são os mandamentos que o Senhor
ordenou a Moisés, para os filhos de Israel, no monte Sinai.

