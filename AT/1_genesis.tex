\addchap{Gênesis}

\lettrine{1} No princípio criou Deus os céus e a terra. E
a terra era sem forma e vazia; e havia trevas sobre a face do
abismo; e o Espírito de Deus se movia sobre a face das águas.

E disse Deus: Haja luz; e houve luz. E viu Deus que era boa a
luz; e fez Deus separação entre a luz e as trevas. E Deus chamou
à luz Dia; e às trevas chamou Noite. E foi a tarde e a manhã, o dia
primeiro.

E disse Deus: Haja uma expansão\footnote{KJ: firmament.} no meio
das águas, e haja separação entre águas e águas. E fez Deus a
expansão, e fez separação entre as águas que estavam debaixo da
expansão e as águas que estavam sobre a expansão; e assim foi. E
chamou Deus à expansão Céus, e foi a tarde e a manhã, o dia segundo.

E disse Deus: Ajuntem-se as águas debaixo dos céus num lugar; e
apareça a porção seca; e assim foi. E chamou Deus à porção
seca Terra; e ao ajuntamento das águas chamou Mares; e viu Deus que
era bom. E disse Deus: Produza a terra erva verde, erva que
dê semente, árvore frutífera que dê fruto segundo a sua espécie,
cuja semente está nela sobre a terra; e assim foi. E a terra
produziu erva, erva dando semente conforme a sua espécie, e a árvore
frutífera, cuja semente está nela conforme a sua espécie; e viu Deus
que era bom. E foi a tarde e a manhã, o dia terceiro.

E disse Deus: Haja luminares na expansão dos céus, para haver
separação entre o dia e a noite; e sejam eles para sinais e para
tempos determinados e para dias e anos. E sejam para
luminares na expansão dos céus, para iluminar a terra; e assim foi.
E fez Deus os dois grandes luminares: o luminar maior para
governar o dia, e o luminar menor para governar a noite; e fez as
estrelas. E Deus os pôs na expansão dos céus para iluminar a
terra, e para governar o dia e a noite, e para fazer
separação entre a luz e as trevas; e viu Deus que era bom. E
foi a tarde e a manhã, o dia quarto.

E disse Deus: Produzam as águas abundantemente répteis de alma
vivente; e voem as aves sobre a face da expansão dos céus. E
Deus criou as grandes baleias, e todo o réptil de alma vivente que
as águas abundantemente produziram conforme as suas espécies; e toda
a ave de asas conforme a sua espécie; e viu Deus que era bom.
E Deus os abençoou, dizendo: Frutificai e multiplicai-vos, e
enchei as águas nos mares; e as aves se multipliquem na terra.
E foi a tarde e a manhã, o dia quinto.

E disse Deus: Produza a terra alma vivente conforme a sua
espécie; gado, e répteis e feras da terra conforme a sua espécie; e
assim foi. E fez Deus as feras da terra conforme a sua
espécie, e o gado conforme a sua espécie, e todo o réptil da terra
conforme a sua espécie; e viu Deus que era bom.

E disse Deus: \textbf{Façamos o homem à nossa imagem, conforme a
nossa semelhança}; e domine sobre os peixes do mar, e sobre as aves
dos céus, e sobre o gado, e sobre toda a terra, e sobre todo o
réptil que se move sobre a terra. E criou Deus o homem à sua
imagem; à imagem de Deus o criou; homem e mulher os criou. E
Deus os abençoou, e Deus lhes disse: Frutificai e multiplicai-vos, e
enchei a terra, e sujeitai-a; e dominai sobre os peixes do mar e
sobre as aves dos céus, e sobre todo o animal que se move sobre a
terra.

E disse Deus: Eis que vos tenho dado toda a erva que dê semente,
que está sobre a face de toda a terra; e toda a árvore, em que há
fruto que dê semente, ser-vos-á para mantimento. E a todo o
animal da terra, e a toda a ave dos céus, e a todo o réptil da
terra, em que há alma vivente, toda a erva verde será para
mantimento; e assim foi.

E viu Deus tudo quanto tinha feito, e eis que era muito bom; e
foi a tarde e a manhã, o dia sexto.

\smallskip

\lettrine{2} Assim os céus, a terra e todo o seu exército
foram acabados. E havendo Deus acabado no dia sétimo a obra que
fizera, descansou no sétimo dia de toda a sua obra, que tinha feito.
E abençoou Deus o dia sétimo, e o santificou; porque nele
descansou de toda a sua obra que Deus criara e fizera.

Estas são as origens dos céus e da terra, quando foram criados; no
dia em que o Senhor Deus fez a terra e os céus, e toda a planta
do campo que ainda não estava na terra, e toda a erva do campo que
ainda não brotava; porque ainda o Senhor Deus não tinha feito chover
sobre a terra, e não havia homem para lavrar a terra. Um vapor,
porém, subia da terra, e regava toda a face da terra. \textbf{E
formou o Senhor Deus o homem do pó da terra, e soprou em suas
narinas o fôlego da vida; e o homem foi feito alma vivente}.

E plantou o Senhor Deus um jardim no Éden, do lado oriental; e pôs
ali o homem que tinha formado. E o Senhor Deus fez brotar da
terra toda a árvore agradável à vista, e boa para comida; e a
\textbf{árvore da vida} no meio do jardim, e a árvore do
conhecimento do bem e do mal. E saía um rio do Éden para
regar o jardim; e dali se dividia e se tornava em quatro braços.
O nome do primeiro é Pisom; este é o que rodeia toda a terra
de Havilá, onde há ouro. E o ouro dessa terra é bom; ali há o
bdélio\footnote{Mirra (resina, óleo)}, e a pedra
sardônica\footnote{Sárdonix: variedade vermelho-parda de calcedônia,
muito us. nos camafeus helenísticos e romanos; sardônia, sardônica}.
E o nome do segundo rio é Giom; este é o que rodeia toda a
terra de Cuxe. E o nome do terceiro rio é Tigre; este é o que
vai para o lado oriental da Assíria; e o quarto rio é o Eufrates.
E tomou o Senhor Deus o homem, e o pôs no jardim do Éden para
o lavrar e o guardar.

E ordenou o Senhor Deus ao homem, dizendo: De toda a árvore do
jardim comerás livremente, mas da árvore do conhecimento do
bem e do mal, dela não comerás; porque no dia em que dela comeres,
certamente morrerás.

E disse o Senhor Deus: Não é bom que o homem esteja só;
far-lhe-ei uma ajudadora idônea para ele. Havendo, pois, o
Senhor Deus formado da terra todo o animal do campo, e toda a ave
dos céus, os trouxe a Adão, para este ver como lhes chamaria; e tudo
o que \textbf{Adão} chamou a toda a alma vivente, isso foi o seu
nome. E Adão pôs os nomes a todo o gado, e às aves dos céus,
e a todo o animal do campo; mas para o homem não se achava ajudadora
idônea.

Então o Senhor Deus fez cair um sono pesado sobre Adão, e este
adormeceu; e tomou uma das suas costelas, e cerrou a carne em seu
lugar; e da costela que o Senhor Deus tomou do homem, formou
uma mulher, e trouxe-a a Adão. E disse Adão: Esta é agora
osso dos meus ossos, e carne da minha carne; esta será chamada
\textbf{mulher}, porquanto do homem foi tomada. Portanto
deixará o homem o seu pai e a sua mãe, e apegar-se-á à sua mulher, e
serão ambos uma carne. E ambos estavam nus, o homem e a sua
mulher; e não se envergonhavam.

\smallskip

\lettrine{3} Ora, a serpente era mais astuta que todas as
alimárias\footnote{Qualquer animal, esp. quadrúpede. Besta de carga
Sentido figurado: pessoa estúpida e grosseira; bruto} do campo que o
Senhor Deus tinha feito. E esta disse à mulher: É assim que Deus
disse: Não comereis de toda a árvore do jardim? E disse a mulher
à serpente: Do fruto das árvores do jardim comeremos, mas do
fruto da árvore que está no meio do jardim, disse Deus: Não comereis
dele, nem nele tocareis para que não morrais. Então a serpente
disse à mulher: Certamente não morrereis. Porque Deus sabe que
no dia em que dele comerdes se abrirão os vossos olhos, e sereis
como Deus, sabendo o bem e o mal.

E viu a mulher que aquela árvore era boa para se comer, e
agradável aos olhos, e árvore desejável para dar entendimento; tomou
do seu fruto, e comeu, e deu também a seu marido, e ele comeu com
ela. Então foram abertos os olhos de ambos, e conheceram que
estavam nus; e coseram folhas de figueira, e fizeram para si
aventais. E ouviram a voz do Senhor Deus, que passeava no jardim
pela viração do dia; e esconderam-se Adão e sua mulher da presença
do Senhor Deus, entre as árvores do jardim.

E chamou o Senhor Deus a Adão, e disse-lhe: Onde estás? E
ele disse: Ouvi a tua voz soar no jardim, e temi, porque estava nu,
e escondi-me.

E Deus disse: Quem te mostrou que estavas nu? Comeste tu da
árvore de que te ordenei que não comesses? Então disse Adão:
A mulher que me deste por companheira, ela me deu da árvore, e comi.
E disse o Senhor Deus à mulher: Por que fizeste isto? E disse
a mulher: A serpente me enganou, e eu comi.

Então o Senhor Deus disse à serpente: Porquanto fizeste isto,
maldita serás mais que toda a fera, e mais que todos os animais do
campo; sobre o teu ventre andarás, e pó comerás todos os dias da tua
vida. E porei inimizade entre ti e a mulher, e entre a tua
semente e a sua semente; esta te ferirá a cabeça, e tu lhe ferirás o
calcanhar.

E à mulher disse: Multiplicarei grandemente a tua dor, e a tua
conceição; com dor darás à luz filhos; e o teu desejo será para o
teu marido, e ele te dominará.

E a Adão disse: Porquanto deste ouvidos à voz de tua mulher, e
comeste da árvore de que te ordenei, dizendo: Não comerás dela,
maldita é a terra por causa de ti; com dor comerás dela todos os
dias da tua vida. Espinhos, e cardos também, te produzirá; e
comerás a erva do campo. No suor do teu rosto comerás o teu
pão, até que te tornes à terra; porque dela foste tomado; porquanto
és pó e em pó te tornarás.

E chamou Adão o nome de sua mulher \textbf{Eva}; porquanto era a
mãe de todos os viventes.

E fez o Senhor Deus a Adão e à sua mulher túnicas de peles, e os
vestiu.

Então disse o Senhor Deus: Eis que o homem é como um de
\textbf{nós}, sabendo o bem e o mal; ora, para que não estenda a sua
mão, e tome também da árvore da vida, e coma e viva eternamente,
o Senhor Deus, pois, o lançou fora do jardim do Éden, para
lavrar a terra de que fora tomado. E havendo lançado fora o
homem, pôs querubins ao oriente do jardim do Éden, e uma espada
inflamada que andava ao redor, para guardar o caminho da árvore da
vida.

\smallskip

\lettrine{4} E conheceu Adão a Eva, sua mulher, e ela concebeu
e deu à luz a Caim, e disse: Alcancei do Senhor um homem. E deu
à luz mais a seu irmão Abel; e Abel foi pastor de ovelhas, e Caim
foi lavrador da terra.

E aconteceu ao cabo de dias que Caim trouxe do fruto da terra uma
oferta ao Senhor. E Abel também trouxe dos primogênitos das suas
ovelhas, e da sua gordura; e atentou o Senhor para Abel e para a sua
oferta. Mas para Caim e para a sua oferta não atentou. E irou-se
Caim fortemente, e descaiu-lhe o semblante.

E o Senhor disse a Caim: Por que te iraste? E por que descaiu o
teu semblante? Se bem fizeres, não é certo que serás aceito? E
se não fizeres bem, o pecado jaz à porta, e sobre ti será o seu
desejo, mas sobre ele deves dominar.

E falou Caim com o seu irmão Abel; e sucedeu que, estando eles no
campo, se levantou Caim contra o seu irmão Abel, e o matou.

E disse o Senhor a Caim: Onde está Abel, teu irmão? E ele disse:
Não sei; sou eu guardador do meu irmão? E disse Deus: Que
fizeste? A voz do sangue do teu irmão clama a mim desde a terra.
E agora maldito és tu desde a terra, que abriu a sua boca
para receber da tua mão o sangue do teu irmão. Quando
lavrares a terra, não te dará mais a sua força; fugitivo e vagabundo
serás na terra.

Então disse Caim ao Senhor: É maior a minha maldade que a que
possa ser perdoada. Eis que hoje me lanças da face da terra,
e da tua face me esconderei; e serei fugitivo e vagabundo na terra,
e será que todo aquele que me achar, me matará. O Senhor,
porém, disse-lhe: Portanto qualquer que matar a Caim, sete vezes
será castigado. E pôs o Senhor um sinal em Caim, para que o não
ferisse qualquer que o achasse.

E saiu Caim de diante da face do Senhor, e habitou na terra de
Node, do lado oriental do Éden. E conheceu Caim a sua mulher,
e ela concebeu, e deu à luz a Enoque; e ele edificou uma cidade, e
chamou o nome da cidade conforme o nome de seu filho Enoque;
e a Enoque nasceu Irade, e Irade gerou a Meujael, e Meujael
gerou a Metusael e Metusael gerou a Lameque.

E tomou Lameque para si duas mulheres; o nome de uma era Ada, e o
nome da outra, Zilá. E Ada deu à luz a Jabal; este foi o pai
dos que habitam em tendas e têm gado. E o nome do seu irmão
era Jubal; este foi o pai de todos os que tocam harpa e órgão.
E Zilá também deu à luz a Tubalcaim, mestre de toda a obra de
cobre e ferro; e a irmã de Tubalcaim foi Noema.

E disse Lameque a suas mulheres Ada e Zilá: Ouvi a minha voz;
vós, mulheres de Lameque, escutai as minhas palavras; porque eu
matei um homem por me ferir, e um jovem por me pisar. Porque
sete vezes Caim será castigado; mas Lameque setenta vezes sete.

E tornou Adão a conhecer a sua mulher; e ela deu à luz um filho,
e chamou o seu nome \textbf{Sete}; porque, disse ela, Deus me deu
outro filho em lugar de Abel; porquanto Caim o matou. E a
Sete também nasceu um filho; e chamou o seu nome Enos; então se
começou a invocar o nome do Senhor.

\smallskip

\lettrine{5} Este é o livro das gerações de Adão. No dia em
que Deus criou o homem, à semelhança de Deus o fez. Homem e
mulher os criou; e os abençoou e chamou o seu nome Adão, no dia em
que foram criados. E Adão viveu cento e trinta anos, e gerou um
filho à sua semelhança, conforme a sua imagem, e pôs-lhe o nome de
Sete. E foram os dias de Adão, depois que gerou a Sete,
oitocentos anos, e gerou filhos e filhas. E foram todos os dias
que Adão viveu, novecentos e trinta anos, e morreu.

E viveu Sete cento e cinco anos, e gerou a Enos. E viveu Sete,
depois que gerou a Enos, oitocentos e sete anos, e gerou filhos e
filhas. E foram todos os dias de Sete novecentos e doze anos, e
morreu. E viveu Enos noventa anos, e gerou a Cainã. E
viveu Enos, depois que gerou a Cainã, oitocentos e quinze anos, e
gerou filhos e filhas. E foram todos os dias de Enos
novecentos e cinco anos, e morreu. E viveu Cainã setenta
anos, e gerou a Maalaleel. E viveu Cainã, depois que gerou a
Maalaleel, oitocentos e quarenta anos, e gerou filhos e filhas.
E foram todos os dias de Cainã novecentos e dez anos, e
morreu. E viveu Maalaleel sessenta e cinco anos, e gerou a
Jerede. E viveu Maalaleel, depois que gerou a Jerede,
oitocentos e trinta anos, e gerou filhos e filhas. E foram
todos os dias de Maalaleel oitocentos e noventa e cinco anos, e
morreu. E viveu Jerede cento e sessenta e dois anos, e gerou
a Enoque. E viveu Jerede, depois que gerou a Enoque,
oitocentos anos, e gerou filhos e filhas. E foram todos os
dias de Jerede novecentos e sessenta e dois anos, e morreu.

E viveu \textbf{Enoque} sessenta e cinco anos, e gerou a
Matusalém. E \textbf{andou Enoque com Deus}, depois que gerou
a Matusalém, trezentos anos, e gerou filhos e filhas. E foram
todos os dias de Enoque trezentos e sessenta e cinco anos. E
andou Enoque com Deus; e não apareceu mais, porquanto Deus para si o
tomou.

E viveu Matusalém cento e oitenta e sete anos, e gerou a Lameque.
E viveu Matusalém, depois que gerou a Lameque, setecentos e
oitenta e dois anos, e gerou filhos e filhas. E foram todos
os dias de Matusalém novecentos e sessenta e nove anos, e morreu.

E viveu Lameque cento e oitenta e dois anos, e gerou um filho,
a quem chamou \textbf{Noé}, dizendo: Este nos consolará
acerca de nossas obras e do trabalho de nossas mãos, por causa da
terra que o Senhor amaldiçoou. E viveu Lameque, depois que
gerou a Noé, quinhentos e noventa e cinco anos, e gerou filhos e
filhas. E foram todos os dias de Lameque setecentos e setenta
e sete anos, e morreu. E era Noé da idade de quinhentos anos,
e gerou Noé a \textbf{Sem, Cão e Jafé}.

\smallskip

\lettrine{6} E aconteceu que, como os homens começaram a
multiplicar-se sobre a face da terra, e lhes nasceram filhas,
viram os \textbf{filhos de Deus} que as filhas dos homens eram
formosas; e tomaram para si mulheres de todas as que escolheram.

Então disse o Senhor: Não contenderá o meu Espírito para sempre
com o homem; porque ele também é carne; porém os seus dias serão
cento e vinte anos.

Havia naqueles dias gigantes na terra; e também depois, quando os
filhos de Deus entraram às filhas dos homens e delas geraram filhos;
estes eram os valentes que houve na antigüidade, os homens de fama.
E viu o Senhor que a maldade do homem se multiplicara sobre a
terra e que toda a imaginação dos pensamentos de seu coração era só
má continuamente.

Então arrependeu-se o Senhor de haver feito o homem sobre a terra
e pesou-lhe em seu coração. E disse o Senhor: Destruirei o homem
que criei de sobre a face da terra, desde o homem até ao animal, até
ao réptil, e até à ave dos céus; porque me arrependo de os haver
feito.

\textbf{Noé, porém, achou graça aos olhos do Senhor}. Estas
são as gerações de Noé. Noé era homem justo e perfeito em suas
gerações; \textbf{Noé andava com Deus}. E gerou Noé três
filhos: Sem, Cão e Jafé.

A terra, porém, estava corrompida diante da face de Deus; e
encheu-se a terra de violência. E viu Deus a terra, e eis que
estava corrompida; porque toda a carne havia corrompido o seu
caminho sobre a terra.

Então disse Deus a Noé: O fim de toda a carne é vindo perante a
minha face; porque a terra está cheia de violência; e eis que os
desfarei com a terra. Faze para ti uma arca da madeira de
gofer; farás compartimentos na arca e a betumarás por dentro e por
fora com betume. E desta maneira a farás: De trezentos
côvados\footnote{Do cotovelo à ponta dos dedos: 45 centímetros} o
comprimento da arca, e de cinqüenta côvados a sua largura, e de
trinta côvados a sua altura.\footnote{Media, aproximadamente, em
metros: 133m x 22m x 13m. Cf. Apologia da Fé Cristã, de Abraão de
Almeida, p.20.} Farás na arca uma janela, e de um côvado a
acabarás em cima; e a porta da arca porás ao seu lado; far-lhe-ás
andares, baixo, segundo e terceiro. Porque eis que eu trago
um \textbf{dilúvio} de águas sobre a terra, para desfazer toda a
carne em que há espírito de vida debaixo dos céus; tudo o que há na
terra expirará. Mas \textbf{contigo estabelecerei a minha
aliança}; e entrarás na arca, tu e os teus filhos, tua mulher e as
mulheres de teus filhos contigo. E de tudo o que vive, de
toda a carne, dois de cada espécie, farás entrar na arca, para os
conservar vivos contigo; macho e fêmea serão. Das aves
conforme a sua espécie, e dos animais conforme a sua espécie, de
todo o réptil da terra conforme a sua espécie, dois de cada espécie
virão a ti, para os conservar em vida. E leva contigo de toda
a comida que se come e ajunta-a para ti; e te será para mantimento,
a ti e a eles.

Assim fez Noé; conforme a tudo o que Deus lhe mandou, assim o
fez.

\smallskip

\lettrine{7} Depois disse o Senhor a Noé: Entra tu e toda a
tua casa na arca, porque tenho visto que és justo diante de mim
nesta geração. De todos os animais limpos tomarás para ti sete e
sete, o macho e sua fêmea; mas dos animais que não são limpos, dois,
o macho e sua fêmea. Também das aves dos céus sete e sete, macho
e fêmea, para conservar em vida sua espécie sobre a face de toda a
terra. Porque, passados ainda sete dias, farei chover sobre a
terra quarenta dias e quarenta noites; e desfarei de sobre a face da
terra toda a substância que fiz.

E fez Noé conforme a tudo o que o Senhor lhe ordenara.
\textbf{E era Noé da idade de seiscentos anos, quando o dilúvio
das águas veio sobre a terra}. Noé entrou na arca, e com ele
seus filhos, sua mulher e as mulheres de seus filhos, por causa das
águas do dilúvio. Dos animais limpos e dos animais que não são
limpos, e das aves, e de todo o réptil sobre a terra, entraram
de dois em dois para junto de Noé na arca, macho e fêmea, como Deus
ordenara a Noé. E aconteceu que passados sete dias, vieram
sobre a terra as águas do dilúvio.

No ano seiscentos da vida de Noé, no mês segundo, aos dezessete
dias do mês, naquele mesmo dia se romperam todas as fontes do grande
abismo, e as janelas dos céus se abriram, e houve chuva sobre
a terra quarenta dias e quarenta noites.

E no mesmo dia entraram na arca Noé, seus filhos Sem, Cão e Jafé,
sua mulher e as mulheres de seus filhos. Eles, e todo o
animal conforme a sua espécie, e todo o gado conforme a sua espécie,
e todo o réptil que se arrasta sobre a terra conforme a sua espécie,
e toda a ave conforme a sua espécie, pássaros de toda qualidade.
E de toda a carne, em que havia espírito de vida, entraram de
dois em dois para junto de Noé na arca. E os que entraram
eram macho e fêmea de toda a carne, como Deus lhe tinha ordenado; e
o Senhor o fechou dentro.

E durou o dilúvio quarenta dias sobre a terra, e cresceram as
águas e levantaram a arca, e ela se elevou sobre a terra. E
prevaleceram as águas e cresceram grandemente sobre a terra; e a
arca andava sobre as águas. E as águas prevaleceram
excessivamente sobre a terra; e todos os altos montes que havia
debaixo de todo o céu, foram cobertos. Quinze côvados acima
prevaleceram as águas; e os montes foram cobertos.

E expirou toda a carne que se movia sobre a terra, tanto de ave
como de gado e de feras, e de todo o réptil que se arrasta sobre a
terra, e todo o homem. Tudo o que tinha fôlego de espírito de
vida em suas narinas, tudo o que havia em terra seca, morreu.
Assim foi destruído todo o ser vivente que havia sobre a face
da terra, desde o homem até ao animal, até ao réptil, e até à ave
dos céus; e foram extintos da terra; e ficou somente Noé, e os que
com ele estavam na arca. \textbf{E prevaleceram as águas
sobre a terra cento e cinqüenta dias}.

\smallskip

\lettrine{8} E lembrou-se Deus de Noé, e de todos os seres
viventes, e de todo o gado que estavam com ele na arca; e Deus fez
passar um vento sobre a terra, e aquietaram-se as águas.
Cerraram-se também as fontes do abismo e as janelas dos céus, e
a chuva dos céus deteve-se. E as águas iam-se escoando
continuamente de sobre a terra, e ao fim de cento e cinqüenta dias
minguaram.

E \textbf{a arca repousou no sétimo mês, no dia dezessete do mês},
sobre os montes de Ararate. E foram as águas indo e minguando
até ao décimo mês; no décimo mês, no primeiro dia do mês, apareceram
os cumes dos montes.

E aconteceu que ao cabo de quarenta dias, abriu Noé a janela da
arca que tinha feito. E soltou um corvo, que saiu, indo e
voltando, até que as águas se secaram de sobre a terra. Depois
soltou uma pomba, para ver se as águas tinham minguado de sobre a
face da terra. A pomba, porém, não achou repouso para a planta
do seu pé, e voltou a ele para a arca; porque as águas estavam sobre
a face de toda a terra; e ele estendeu a sua mão, e tomou-a, e
recolheu-a consigo na arca. E esperou ainda outros sete dias,
e tornou a enviar a pomba fora da arca. E a pomba voltou a
ele à tarde; e eis, arrancada, uma folha de oliveira no seu bico; e
conheceu Noé que as águas tinham minguado de sobre a terra.
Então esperou ainda outros sete dias, e enviou fora a pomba;
mas não tornou mais a ele.

E aconteceu que no ano seiscentos e um, no mês primeiro, no
primeiro dia do mês, as águas se secaram de sobre a terra. Então Noé
tirou a cobertura da arca, e olhou, e eis que a face da terra estava
enxuta. E no segundo mês, aos vinte e sete dias do mês, a
terra estava seca.

Então falou Deus a Noé dizendo: Sai da arca, tu com tua
mulher, e teus filhos e as mulheres de teus filhos. Todo o
animal que está contigo, de toda a carne, de ave, e de gado, e de
todo o réptil que se arrasta sobre a terra, traze fora contigo; e
povoem abundantemente a terra e frutifiquem, e se multipliquem sobre
a terra. Então saiu Noé, e seus filhos, e sua mulher, e as
mulheres de seus filhos com ele. Todo o animal, todo o
réptil, e toda a ave, e tudo o que se move sobre a terra, conforme
as suas famílias, saiu\footnote{SBTB: ``saiu para fora''. Ed.
Contemp.: saíram da arca; KJ: went forth out of the ark.} da arca.

E edificou Noé um altar ao Senhor; e tomou de todo o animal limpo
e de toda a ave limpa, e ofereceu holocausto sobre o altar. E
o Senhor sentiu o suave cheiro, e o Senhor disse em seu coração: Não
tornarei mais a amaldiçoar a terra por causa do homem; porque a
imaginação do coração do homem é má desde a sua meninice, nem
tornarei mais a ferir todo o vivente, como fiz. Enquanto a
terra durar, sementeira e sega, e frio e calor, e verão e inverno, e
dia e noite, não cessarão.

\smallskip

\lettrine{9} E abençoou Deus a Noé e a seus filhos, e
disse-lhes: Frutificai e multiplicai-vos e enchei a terra. E o
temor de vós e o pavor de vós virão sobre todo o animal da terra, e
sobre toda a ave dos céus; tudo o que se move sobre a terra, e todos
os peixes do mar, nas vossas mãos são entregues. Tudo quanto se
move, que é vivente, será para vosso mantimento; tudo vos tenho dado
como a erva verde. A carne, porém, com sua vida, isto é, com seu
sangue, não comereis. Certamente requererei o vosso sangue, o
sangue das vossas vidas; da mão de todo o animal o requererei; como
também da mão do homem, e da mão do irmão de cada um requererei a
vida do homem. Quem derramar o sangue do homem, pelo homem o seu
sangue será derramado; porque Deus fez o homem conforme a sua
imagem. Mas vós frutificai e multiplicai-vos; povoai
abundantemente a terra, e multiplicai-vos nela.

E falou Deus a Noé e a seus filhos com ele, dizendo: E eu, eis
que \textbf{estabeleço a minha aliança} convosco e com a vossa
descendência depois de vós. E com toda a alma vivente, que
convosco está, de aves, de gado, e de todo o animal da terra
convosco; com todos que saíram da arca, até todo o animal da terra.
E eu convosco \textbf{estabeleço a minha aliança}, que não
será mais destruída toda a carne pelas águas do dilúvio, e que não
haverá mais dilúvio, para destruir a terra.

E disse Deus: Este é o sinal da aliança que ponho entre mim e
vós, e entre toda a alma vivente, que está convosco, por gerações
eternas. O meu arco tenho posto nas nuvens; este será por
sinal da aliança entre mim e a terra. E acontecerá que,
quando eu trouxer nuvens sobre a terra, aparecerá o arco nas nuvens.
Então me lembrarei da minha aliança, que está entre mim e
vós, e entre toda a alma vivente de toda a carne; e as águas não se
tornarão mais em dilúvio para destruir toda a carne. E estará
o arco nas nuvens, e eu o verei, para me lembrar da \textbf{aliança
eterna} entre Deus e toda a alma vivente de toda a carne, que está
sobre a terra. E disse Deus a Noé: Este é o sinal da aliança
que tenho estabelecido entre mim e entre toda a carne, que está
sobre a terra.

E os filhos de Noé, que da arca saíram, foram Sem, Cão e Jafé; e
Cão é o pai de Canaã. Estes três foram os filhos de Noé; e
destes se povoou toda a terra. E começou Noé a ser lavrador
da terra, e plantou uma vinha. E bebeu do vinho, e
embebedou-se; e descobriu-se no meio de sua tenda. E viu Cão,
o pai de Canaã, a nudez do seu pai, e fê-lo saber a ambos seus
irmãos no lado de fora. Então tomaram Sem e Jafé uma capa, e
puseram-na sobre ambos os seus ombros, e indo virados para trás,
cobriram a nudez do seu pai, e os seus rostos estavam virados, de
maneira que não viram a nudez do seu pai.

E despertou Noé do seu vinho, e soube o que seu filho menor lhe
fizera. E disse: Maldito seja Canaã; servo dos servos seja
aos seus irmãos. E disse: Bendito seja o Senhor Deus de Sem;
e seja-lhe Canaã por servo. Alargue Deus a Jafé, e habite nas
tendas de Sem; e seja-lhe Canaã por servo.

E viveu Noé, depois do dilúvio, trezentos e cinqüenta anos.
E foram todos os dias de Noé novecentos e cinqüenta anos, e
morreu.

\smallskip

\lettrine{10} Estas, pois, são as gerações dos filhos de Noé:
Sem, Cão e Jafé; e nasceram-lhes filhos depois do dilúvio. Os
filhos de Jafé são: Gomer, Magogue, Madai, Javã, Tubal, Meseque e
Tiras. E os filhos de Gomer são: Asquenaz, Rifate e Togarma.
E os filhos de Javã são: Elisá, Társis, Quitim e Dodanim.
Por estes foram repartidas as ilhas dos gentios nas suas terras,
cada qual segundo a sua língua, segundo as suas famílias, entre as
suas nações.

E os filhos de Cão são: Cuxe, Mizraim, Pute e Canaã. E os
filhos de Cuxe são: Sebá, Havilá, Sabtá, Raamá e Sabtecá; e os
filhos de Raamá: Sebá e Dedã. E Cuxe gerou a Ninrode; este
começou a ser poderoso na terra. E este foi poderoso caçador
diante da face do Senhor; por isso se diz: Como Ninrode, poderoso
caçador diante do Senhor. E o princípio do seu reino foi
Babel, Ereque, Acade e Calné, na terra de Sinar. Desta mesma
terra saiu à Assíria e edificou a Nínive, Reobote-Ir, Calá, e
Resen, entre Nínive e Calá (esta é a grande cidade). E
Mizraim gerou a Ludim, a Anamim, a Leabim, a Naftuim, a
Patrusim e a Casluim (donde saíram os filisteus) e a Caftorim.

E Canaã gerou a Sidom, seu primogênito, e a Hete; e ao
jebuseu, ao amorreu, ao girgaseu, e ao heveu, ao arqueu, ao
sineu, e ao arvadeu, ao zemareu, e ao hamateu, e depois se
espalharam as famílias dos cananeus. E foi o termo dos
cananeus desde Sidom, indo para Gerar, até Gaza; indo para Sodoma e
Gomorra, Admá e Zeboim, até Lasa. Estes são os filhos de Cão
segundo as suas famílias, segundo as suas línguas, em suas terras,
em suas nações.

E a Sem nasceram filhos, e ele é o pai de todos os filhos de
Éber, o irmão mais velho de Jafé.\footnote{Ed. Contemp.: A Sem,
irmão mais velho de Jafé, nasceram filhos; ele foi o pai de todos os
filhos de Éber. KJ: Unto Shem also, the father of all children of
Eber, the brother of Japheth the elder, even to him were
\emph{children} born.} Os filhos de Sem são: Elão, Assur,
Arfaxade, Lude e Arã. E os filhos de Arã são: Uz, Hul, Geter
e Más. E Arfaxade gerou a Selá; e Selá gerou a Éber. E
a Éber nasceram dois filhos: o nome de um foi Pelegue, porquanto em
seus dias se repartiu a terra, e o nome do seu irmão foi Joctã.
E Joctã gerou a Almodá, a Selefe, a Hazarmavé, a Jerá,
a Hadorão, a Usal, a Dicla, a Obal, a Abimael, a Sebá,
a Ofir, a Havilá e a Jobabe; todos estes foram filhos de
Joctã. E foi a sua habitação desde Messa, indo para Sefar,
montanha do oriente. Estes são os filhos de Sem segundo as
suas famílias, segundo as suas línguas, nas suas terras, segundo as
suas nações. Estas são as famílias dos filhos de Noé segundo
as suas gerações, nas suas nações; e destes foram divididas as
nações na terra depois do dilúvio.

\smallskip

\lettrine{11} E era toda a terra de uma mesma língua e de uma
mesma fala. E aconteceu que, partindo eles do oriente, acharam
um vale na terra de Sinar; e habitaram ali. E disseram uns aos
outros: Eia\footnote{Ed. Contemp.: Vinde, edifiquemos para nós
\ldots{}. KJ: Go to, let us make brick\ldots{}.}, façamos tijolos e
queimemo-los bem. E foi-lhes o tijolo por pedra, e o betume por cal.
E disseram: Eia, edifiquemos nós uma cidade e uma torre cujo
cume toque nos céus, e façamo-nos um nome, para que não sejamos
espalhados sobre a face de toda a terra.

Então desceu o Senhor para ver a cidade e a torre que os filhos
dos homens edificavam; e o Senhor disse: Eis que o povo é um, e
todos têm uma mesma língua; e isto é o que começam a fazer; e agora,
não haverá restrição para tudo o que eles intentarem fazer. Eia,
desçamos e confundamos ali a sua língua, para que não entenda um a
língua do outro. Assim o Senhor os espalhou dali sobre a face de
toda a terra; e cessaram de edificar a cidade. Por isso se
chamou o seu nome Babel, porquanto ali confundiu o Senhor a língua
de toda a terra, e dali os espalhou o Senhor sobre a face de toda a
terra.

Estas são as gerações de Sem: Sem era da idade de cem anos e
gerou a Arfaxade, dois anos depois do dilúvio. E viveu Sem,
depois que gerou a Arfaxade, quinhentos anos, e gerou filhos e
filhas. E viveu Arfaxade trinta e cinco anos, e gerou a Selá.
E viveu Arfaxade depois que gerou a Selá, quatrocentos e três
anos, e gerou filhos e filhas. E viveu Selá trinta anos, e
gerou a Éber; e viveu Selá, depois que gerou a Éber,
quatrocentos e três anos, e gerou filhos e filhas. E viveu
Éber trinta e quatro anos, e gerou a Pelegue. E viveu Éber,
depois que gerou a Pelegue, quatrocentos e trinta anos, e gerou
filhos e filhas. E viveu Pelegue trinta anos, e gerou a Reú.
E viveu Pelegue, depois que gerou a Reú, duzentos e nove
anos, e gerou filhos e filhas. E viveu Reú trinta e dois
anos, e gerou a Serugue. E viveu Reú, depois que gerou a
Serugue, duzentos e sete anos, e gerou filhos e filhas. E
viveu Serugue trinta anos, e gerou a Naor. E viveu Serugue,
depois que gerou a Naor, duzentos anos, e gerou filhos e filhas.
E viveu Naor vinte e nove anos, e gerou a Terá. E
viveu Naor, depois que gerou a Terá, cento e dezenove anos, e gerou
filhos e filhas. E viveu Terá setenta anos, e gerou a
\textbf{Abrão}, a Naor, e a Harã.

E estas são as gerações de Terá: Terá gerou a Abrão, a Naor, e a
Harã; e Harã gerou a Ló. E morreu Harã estando seu pai Terá
ainda vivo, na terra do seu nascimento, em Ur dos caldeus. E
tomaram Abrão e Naor mulheres para si: o nome da mulher de Abrão era
Sarai, e o nome da mulher de Naor era Milca, filha de Harã, pai de
Milca e pai de Iscá. E Sarai foi estéril, não tinha filhos.
E tomou Terá a Abrão seu filho, e a Ló, filho de Harã, filho
de seu filho, e a Sarai sua nora, mulher de seu filho Abrão, e saiu
com eles de Ur dos caldeus, para ir à terra de Canaã; e vieram até
Harã, e habitaram ali. E foram os dias de Terá duzentos e
cinco anos, e morreu Terá em Harã.

\smallskip

\lettrine{12} Ora, o Senhor disse a Abrão: Sai-te da tua
terra, da tua parentela e da casa de teu pai, para a terra que eu te
mostrarei. E far-te-ei uma grande nação, e abençoar-te-ei e
engrandecerei o teu nome; e tu serás uma bênção. E abençoarei os
que te abençoarem, e amaldiçoarei os que te amaldiçoarem; \textbf{e
em ti serão benditas todas as famílias da terra}. Assim partiu
Abrão como o Senhor lhe tinha dito, e foi Ló com ele; e era Abrão da
idade de setenta e cinco anos quando saiu de Harã. E tomou Abrão
a Sarai, sua mulher, e a Ló, filho de seu irmão, e todos os bens que
haviam adquirido, e as almas que lhe acresceram em Harã; e saíram
para irem à terra de Canaã; e chegaram à terra de Canaã.

E passou Abrão por aquela terra até ao lugar de Siquém, até ao
carvalho de Moré; e estavam então os cananeus na terra. E
apareceu o Senhor a Abrão, e disse: \textbf{À tua descendência darei
esta terra}. E edificou ali um altar ao Senhor, que lhe aparecera.
E moveu-se dali para a montanha do lado oriental de Betel, e
armou a sua tenda, tendo Betel ao ocidente, e Ai ao oriente; e
edificou ali um altar ao Senhor, e invocou o nome do Senhor.
Depois caminhou Abrão dali, seguindo ainda para o lado do sul.

E havia fome naquela terra; e desceu Abrão ao Egito, para
peregrinar ali, porquanto a fome era grande na terra. E
aconteceu que, chegando ele para entrar no Egito, disse a Sarai, sua
mulher: Ora, bem sei que és mulher formosa à vista; e será
que, quando os egípcios te virem, dirão: Esta é sua mulher. E
matar-me-ão a mim, e a ti te guardarão em vida. Dize,
peço-te, que és minha irmã, para que me vá bem por tua causa, e que
viva a minha alma por amor de ti.

E aconteceu que, entrando Abrão no Egito, viram os egípcios a
mulher, que era mui formosa. E viram-na os príncipes de
Faraó, e gabaram-na diante de Faraó; e foi a mulher tomada para a
casa de Faraó. E fez bem a Abrão por amor dela; e ele teve
ovelhas, vacas, jumentos, servos e servas, jumentas e camelos.
Feriu, porém, o Senhor a Faraó e a sua casa, com grandes
pragas, por causa de Sarai, mulher de Abrão. Então chamou
Faraó a Abrão, e disse: Que é isto que me fizeste? Por que não me
disseste que ela era tua mulher? Por que disseste: É minha
irmã? Por isso a tomei por minha mulher; agora, pois, eis aqui tua
mulher; toma-a e vai-te. E Faraó deu ordens aos seus homens a
respeito dele; e acompanharam-no, a ele, e a sua mulher, e a tudo o
que tinha.

\smallskip

\lettrine{13} Subiu, pois, Abrão do Egito para o lado do sul,
ele e sua mulher, e tudo o que tinha, e com ele Ló. E era Abrão
muito rico em gado, em prata e em ouro. E fez as suas jornadas
do sul até Betel, até ao lugar onde a princípio estivera a sua
tenda, entre Betel e Ai; até ao lugar do altar que outrora ali
tinha feito; e Abrão invocou ali o nome do Senhor.

E também Ló, que ia com Abrão, tinha rebanhos, gado e tendas.
E não tinha capacidade a terra para poderem habitar juntos;
porque os seus bens eram muitos; de maneira que não podiam habitar
juntos. E houve contenda entre os pastores do gado de Abrão e os
pastores do gado de Ló; e os cananeus e os perizeus habitavam então
na terra. E disse Abrão a Ló: Ora, não haja contenda entre mim e
ti, e entre os meus pastores e os teus pastores, porque somos
irmãos. Não está toda a terra diante de ti? Eia, pois, aparta-te
de mim; e se escolheres a esquerda, irei para a direita; e se a
direita escolheres, eu irei para a esquerda.

E levantou Ló os seus olhos, e viu toda a campina do Jordão, que
era toda bem regada, antes do Senhor ter destruído Sodoma e Gomorra,
e era como o jardim do Senhor, como a terra do Egito, quando se
entra em Zoar. Então Ló escolheu para si toda a campina do
Jordão, e partiu Ló para o oriente, e apartaram-se um do outro.
Habitou Abrão na terra de Canaã e Ló habitou nas cidades da
campina, e armou as suas tendas até Sodoma. Ora, eram maus os
homens de Sodoma, e grandes pecadores contra o Senhor.

E disse o Senhor a Abrão, depois que Ló se apartou dele: Levanta
agora os teus olhos, e olha desde o lugar onde estás, para o lado do
norte, e do sul, e do oriente, e do ocidente; porque
\textbf{toda esta terra que vês, te hei de dar a ti, e à tua
descendência, para sempre}. E farei a tua descendência como o
pó da terra; de maneira que se alguém puder contar o pó da terra,
também a tua descendência será contada. Levanta-te, percorre
essa terra, no seu comprimento e na sua largura; porque a ti a
darei. E Abrão mudou as suas tendas, e foi, e habitou nos
carvalhais de Manre, que estão junto a Hebrom; e edificou ali um
altar ao Senhor.

\smallskip

\lettrine{14} E aconteceu nos dias de Anrafel, rei de Sinar,
Arioque, rei de Elasar, Quedorlaomer, rei de Elão, e Tidal, rei de
Goim, que estes fizeram guerra a Bera, rei de Sodoma, a Birsa,
rei de Gomorra, a Sinabe, rei de Admá, e a Semeber, rei de Zeboim, e
ao rei de Belá (esta é Zoar). Todos estes se ajuntaram no vale
de Sidim (que é o Mar Salgado). Doze anos haviam servido a
Quedorlaomer, mas ao décimo terceiro ano rebelaram-se. E ao
décimo quarto ano veio Quedorlaomer, e os reis que estavam com ele,
e feriram aos refains em Asterote-Carnaim, e aos zuzins em Hã, e aos
emins em Savé-Quiriataim, e aos horeus no seu monte Seir, até
El-Parã que está junto ao deserto. Depois tornaram e vieram a
En-Mispate (que é Cades), e feriram toda a terra dos amalequitas, e
também aos amorreus, que habitavam em Hazazom-Tamar. Então saiu
o rei de Sodoma, e o rei de Gomorra, e o rei de Admá, e o rei de
Zeboim, e o rei de Belá (esta é Zoar), e ordenaram batalha contra
eles no vale de Sidim, contra Quedorlaomer, rei de Elão, e
Tidal, rei de Goim, e Anrafel, rei de Sinar, e Arioque, rei de
Elasar; quatro reis contra cinco. E o vale de Sidim estava
cheio de poços de betume; e fugiram os reis de Sodoma e de Gomorra,
e caíram ali; e os restantes fugiram para um monte. E tomaram
todos os bens de Sodoma, e de Gomorra, e todo o seu mantimento e
foram-se. Também tomaram a Ló, que habitava em Sodoma, filho
do irmão de Abrão, e os seus bens, e foram-se.

Então veio um, que escapara, e o contou a Abrão, o hebreu; ele
habitava junto dos carvalhais de Manre, o amorreu, irmão de Escol, e
irmão de Aner; eles eram confederados de Abrão. Ouvindo,
pois, Abrão que o seu irmão estava preso, armou os seus criados,
nascidos em sua casa, trezentos e dezoito, e os perseguiu até Dã.
E dividiu-se contra eles de noite, ele e os seus criados, e
os feriu, e os perseguiu até Hobá, que fica à esquerda de Damasco.
E tornou a trazer todos os seus bens, e tornou a trazer
também a Ló, seu irmão, e os seus bens, e também as mulheres, e o
povo.

E o rei de Sodoma saiu-lhe ao encontro (depois que voltou de
ferir a Quedorlaomer e aos reis que estavam com ele) até ao Vale de
Savé, que é o vale do rei. E \textbf{Melquisedeque}, rei de
Salém, trouxe pão e vinho; e era este sacerdote do Deus Altíssimo.
E abençoou-o, e disse: \textbf{Bendito seja Abrão pelo Deus
Altíssimo, o Possuidor dos céus e da terra}; e
\textbf{bendito seja o Deus Altíssimo, que entregou os teus inimigos
nas tuas mãos}. E Abrão deu-lhe o dízimo de tudo.

E o rei de Sodoma disse a Abrão: Dá-me a mim as pessoas, e os
bens toma para ti. Abrão, porém, disse ao rei de Sodoma:
Levantei minha mão ao Senhor, o Deus Altíssimo, o Possuidor dos céus
e da terra, jurando que desde um fio até à correia de um
sapato, não tomarei coisa alguma de tudo o que é teu; para que não
digas: Eu enriqueci a Abrão; salvo tão-somente o que os
jovens comeram, e a parte que toca aos homens que comigo foram,
Aner, Escol e Manre; estes que tomem a sua parte.

\smallskip

\lettrine{15} Depois destas coisas veio a palavra do Senhor a
Abrão em visão, dizendo: \textbf{Não temas, Abrão, eu sou o teu
escudo, o teu grandíssimo galardão}.

Então disse Abrão: Senhor Deus, que me hás de dar, pois ando sem
filhos, e o mordomo da minha casa é o damasceno Eliézer? Disse
mais Abrão: Eis que não me tens dado filhos, e eis que um nascido na
minha casa será o meu herdeiro. E eis que veio a palavra do
Senhor a ele dizendo: Este não será o teu herdeiro; mas aquele que
de tuas entranhas sair, este será o teu herdeiro. Então o levou
fora, e disse: Olha agora para os céus, e conta as estrelas, se as
podes contar. E disse-lhe: Assim será a tua descendência.
\textbf{E creu ele no Senhor, e imputou-lhe isto por justiça}.

Disse-lhe mais: Eu sou o Senhor, que te tirei de Ur dos caldeus,
para dar-te a ti esta terra, para herdá-la. E disse ele: Senhor
Deus, como saberei que hei de herdá-la? E disse-lhe: Toma-me uma
bezerra de três anos, e uma cabra de três anos, e um carneiro de
três anos, uma rola e um pombinho. E trouxe-lhe todos estes,
e partiu-os pelo meio, e pôs cada parte deles em frente da outra;
mas as aves não partiu. E as aves desciam sobre os cadáveres;
Abrão, porém, as enxotava.

E pondo-se o sol, um profundo sono caiu sobre Abrão; e eis que
grande espanto e grande escuridão caiu sobre ele. Então disse
a Abrão: Sabes, de certo, que peregrina será a tua descendência em
terra alheia, e será reduzida à escravidão, e será afligida por
quatrocentos anos, mas também eu julgarei a nação, à qual ela
tem de servir, e depois sairá com grande riqueza. E tu irás a
teus pais em paz; em boa velhice serás sepultado. E a quarta
geração tornará para cá; porque a medida da injustiça dos amorreus
não está ainda cheia.

E sucedeu que, posto o sol, houve escuridão, e eis um forno de
fumaça, e uma tocha de fogo, que passou por aquelas metades.
Naquele mesmo dia \textbf{fez o Senhor uma aliança com
Abrão}, dizendo: \textbf{À tua descendência tenho dado esta terra,
desde o rio do Egito até ao grande rio Eufrates}; e o queneu,
e o quenezeu, e o cadmoneu, e o heteu, e o perizeu, e os
refains, e o amorreu, e o cananeu, e o girgaseu, e o jebuseu.

\smallskip

\lettrine{16} Ora Sarai, mulher de Abrão, não lhe dava filhos,
e ele tinha uma serva egípcia, cujo nome era Agar. E disse Sarai
a Abrão: Eis que o Senhor me tem impedido de dar à luz; toma, pois,
a minha serva; porventura terei filhos dela. E ouviu Abrão a voz de
Sarai. Assim tomou Sarai, mulher de Abrão, a Agar egípcia, sua
serva, e deu-a por mulher a Abrão seu marido, ao fim de dez anos que
Abrão habitara na terra de Canaã.

E ele possuiu a Agar, e ela concebeu; e vendo ela que concebera,
foi sua senhora desprezada aos seus olhos. Então disse Sarai a
Abrão: Meu agravo seja sobre ti; minha serva pus eu em teu regaço;
vendo ela agora que concebeu, sou menosprezada aos seus olhos; o
Senhor julgue entre mim e ti. E disse Abrão a Sarai: Eis que tua
serva está na tua mão; faze-lhe o que bom é aos teus olhos. E
afligiu-a Sarai, e ela fugiu de sua face.

E o anjo do Senhor a achou junto a uma fonte de água no deserto,
junto à fonte no caminho de Sur. E disse: Agar, serva de Sarai,
donde vens, e para onde vais? E ela disse: Venho fugida da face de
Sarai minha senhora. Então lhe disse o anjo do Senhor: Torna-te
para tua senhora, e humilha-te debaixo de suas mãos.

Disse-lhe mais o anjo do Senhor: Multiplicarei sobremaneira a tua
descendência, que não será contada, por numerosa que será.
Disse-lhe também o anjo do Senhor: Eis que concebeste, e
darás à luz um filho, e chamarás o seu nome Ismael; porquanto o
Senhor ouviu a tua aflição. E ele será homem feroz, e a sua
mão será contra todos, e a mão de todos contra ele; e habitará
diante da face de todos os seus irmãos. E ela chamou o nome
do Senhor, que com ela falava: Tu és Deus que me vê; porque disse:
Não olhei eu também para aquele que me vê? Por isso se chama
aquele poço de Beer-Laai-Rói; eis que está entre Cades e Berede.

E Agar deu à luz um filho a Abrão; e Abrão chamou o nome do seu
filho que Agar tivera, Ismael. E era Abrão da idade de
oitenta e seis anos, quando Agar deu à luz Ismael.

\smallskip

\lettrine{17} Sendo, pois, Abrão da idade de noventa e nove
anos, apareceu o Senhor a Abrão, e disse-lhe: \textbf{Eu sou o Deus
Todo-Poderoso, anda em minha presença e sê perfeito}. \textbf{E
porei a minha aliança entre mim e ti, e te multiplicarei
grandissimamente.} Então caiu Abrão sobre o seu rosto, e falou
Deus com ele, dizendo:

Quanto a mim, \textbf{eis a minha aliança contigo: serás o pai de
muitas nações}; e não se chamará mais o teu nome Abrão, mas
\textbf{Abraão} será o teu nome; porque por pai de muitas nações te
tenho posto; e te farei frutificar grandissimamente, e de ti
farei nações, e reis sairão de ti.

E estabelecerei a minha aliança entre mim e ti e a tua
descendência depois de ti em suas gerações, por aliança perpétua,
para te ser a ti por Deus, e à tua descendência depois de ti.
\textbf{E te darei a ti e à tua descendência depois de ti, a
terra de tuas peregrinações, toda a terra de Canaã em perpétua
possessão e ser-lhes-ei o seu Deus}. Disse mais Deus a Abraão:
Tu, porém, guardarás a minha aliança, tu, e a tua descendência
depois de ti, nas suas gerações. Esta é a minha aliança, que
guardareis entre mim e vós, e a tua descendência depois de ti: Que
todo o homem entre vós será circuncidado. E circuncidareis a
carne do vosso prepúcio; e isto será por sinal da aliança entre mim
e vós. O filho de oito dias, pois, será circuncidado, todo o
homem nas vossas gerações; o nascido na casa, e o comprado por
dinheiro a qualquer estrangeiro, que não for da tua descendência.
Com efeito será circuncidado o nascido em tua casa, e o
comprado por teu dinheiro; e estará a minha aliança na vossa carne
por aliança perpétua. E o homem incircunciso, cuja carne do
prepúcio não estiver circuncidada, aquela alma será extirpada do seu
povo; quebrou a minha aliança.

Disse Deus mais a Abraão: A Sarai tua mulher não chamarás mais
pelo nome de Sarai, mas \textbf{Sara} será o seu nome. Porque
eu a hei de abençoar, e te darei dela um filho; e a abençoarei, e
será mãe das nações; reis de povos sairão dela. Então caiu
Abraão sobre o seu rosto, e riu-se, e disse no seu coração: A um
homem de cem anos há de nascer um filho? E dará à luz Sara da idade
de noventa anos? E disse Abraão a Deus: Quem dera que viva
Ismael diante de teu rosto! E disse Deus: Na verdade, Sara,
tua mulher, te dará um filho, e chamarás o seu nome \textbf{Isaque,
e com ele estabelecerei a minha aliança, por aliança perpétua para a
sua descendência depois dele}. E quanto a Ismael, também te
tenho ouvido; eis aqui o tenho abençoado, e fá-lo-ei frutificar, e
fá-lo-ei multiplicar grandissimamente; doze príncipes gerará, e dele
farei uma grande nação. \textbf{A minha aliança, porém,
estabelecerei com Isaque}, o qual Sara dará à luz neste tempo
determinado, no ano seguinte. Ao acabar de falar com Abraão,
subiu Deus de diante dele.

Então tomou Abraão a seu filho Ismael, e a todos os nascidos na
sua casa, e a todos os comprados por seu dinheiro, todo o homem
entre os da casa de Abraão; e circuncidou a carne do seu prepúcio,
naquele mesmo dia, como Deus falara com ele. E era Abraão da
idade de noventa e nove anos, quando lhe foi circuncidada a carne do
seu prepúcio. E Ismael, seu filho, era da idade de treze
anos, quando lhe foi circuncidada a carne do seu prepúcio.
Naquele mesmo dia foram circuncidados Abraão e Ismael seu
filho, e todos os homens da sua casa, os nascidos em casa, e
os comprados por dinheiro ao estrangeiro, foram circuncidados com
ele.

\smallskip

\lettrine{18} Depois apareceu-lhe o Senhor nos carvalhais de
Manre, estando ele assentado à porta da tenda, no calor do dia.
E levantou os seus olhos, e olhou, e eis \textbf{três homens} em
pé junto a ele. E vendo-os, correu da porta da tenda ao seu encontro
e inclinou-se à terra, e disse: Meu Senhor, se agora tenho
achado graça aos teus olhos, rogo-te que não passes de teu servo.
Que se traga já um pouco de água, e lavai os vossos pés, e
recostai-vos debaixo desta árvore; e trarei um bocado de pão,
para que esforceis\footnote{KJ: And I will fetch a morsel of bread,
and comfort ye your hearts; after that ye shall pass on: for
therefore are ye come to your servant. And they said, So do, as thou
hast said. Ed.Contemp.: Trarei um bocado de pão,´para que possais
refazer as vossas forças, e depois passareis adiante --- visto que
chegastes até o vosso servo. Responderam: Faze como disseste.} o
vosso coração; depois passareis adiante, porquanto por isso
chegastes até vosso servo. E disseram: Assim faze como disseste.
E Abraão apressou-se em ir ter com Sara à tenda, e disse-lhe:
Amassa depressa três medidas de flor de farinha, e faze bolos. E
correu Abraão às vacas, e tomou uma vitela tenra e boa, e deu-a ao
moço, que se apressou em prepará-la. E tomou manteiga e leite, e
a vitela que tinha preparado, e pôs tudo diante deles, e ele estava
em pé junto a eles debaixo da árvore; e comeram.

E disseram-lhe: Onde está Sara, tua mulher? E ele disse: Ei-la aí
na tenda. E disse: Certamente tornarei a ti por este tempo da
vida; e eis que Sara tua mulher terá um filho. E Sara escutava à
porta da tenda, que estava atrás dele. E eram Abraão e Sara
já velhos, e adiantados em idade; já a Sara havia cessado o costume
das mulheres. Assim, pois, riu-se Sara consigo, dizendo:
Terei ainda deleite depois de haver envelhecido, sendo também o meu
senhor já velho? E disse o Senhor a Abraão: Por que se riu
Sara, dizendo: Na verdade darei eu à luz ainda, havendo já
envelhecido? \textbf{Haveria coisa alguma difícil ao Senhor?}
Ao tempo determinado tornarei a ti por este tempo da vida, e Sara
terá um filho. E Sara negou, dizendo: Não me ri; porquanto
temeu. E ele disse: Não digas isso, porque te riste.

E levantaram-se aqueles homens dali, e olharam para o lado de
Sodoma; e Abraão ia com eles, acompanhando-os. E disse o
Senhor: Ocultarei eu a Abraão o que faço, visto que Abraão
certamente virá a ser uma grande e poderosa nação, e nele serão
benditas todas as nações da terra? Porque eu o tenho
conhecido, e sei que ele há de ordenar a seus filhos e à sua casa
depois dele, para que guardem o caminho do Senhor, para agir com
justiça e juízo; para que o Senhor faça vir sobre Abraão o que
acerca dele tem falado. Disse mais o Senhor: Porquanto o
clamor de Sodoma e Gomorra se tem multiplicado, e porquanto o seu
pecado se tem agravado muito, descerei agora, e verei se com
efeito têm praticado segundo o seu clamor, que é vindo até mim; e se
não, sabê-lo-ei. Então viraram aqueles homens os rostos dali,
e foram-se para Sodoma; mas Abraão ficou ainda em pé diante da face
do Senhor.

E chegou-se Abraão, dizendo: Destruirás também o justo com o
ímpio? Se porventura houver cinqüenta justos na cidade,
destruirás também, e não pouparás o lugar por causa dos cinqüenta
justos que estão dentro dela? Longe de ti que faças tal
coisa, que mates o justo com o ímpio; que o justo seja como o ímpio,
longe de ti. Não faria justiça o Juiz de toda a terra? Então
disse o Senhor: Se eu em Sodoma achar cinqüenta justos dentro da
cidade, pouparei a todo o lugar por amor deles. E respondeu
Abraão dizendo: Eis que agora me atrevi a falar ao Senhor, ainda que
sou pó e cinza. Se porventura de cinqüenta justos faltarem
cinco, destruirás por aqueles cinco toda a cidade? E disse: Não a
destruirei, se eu achar ali quarenta e cinco. E continuou
ainda a falar-lhe, e disse: Se porventura se acharem ali quarenta? E
disse: Não o farei por amor dos quarenta. Disse mais: Ora,
não se ire o Senhor, se eu ainda falar: Se porventura se acharem ali
trinta? E disse: Não o farei se achar ali trinta. E disse:
Eis que agora me atrevi a falar ao Senhor: Se porventura se acharem
ali vinte? E disse: Não a destruirei por amor dos vinte.
Disse mais: Ora, não se ire o Senhor, que ainda só mais esta
vez falo: Se porventura se acharem ali dez? E disse: Não a
destruirei por amor dos dez. E retirou-se o Senhor, quando
acabou de falar a Abraão; e Abraão tornou-se ao seu lugar.

\smallskip

\lettrine{19} E vieram os dois anjos a Sodoma à tarde, e
estava Ló assentado à porta de Sodoma; e vendo-os Ló, levantou-se ao
seu encontro e inclinou-se com o rosto à terra; e disse: Eis
agora, meus senhores, entrai, peço-vos, em casa de vosso servo, e
passai nela a noite, e lavai os vossos pés; e de madrugada vos
levantareis e ireis vosso caminho. E eles disseram: Não, antes na
rua passaremos a noite. E porfiou com eles muito, e vieram com
ele, e entraram em sua casa; e fez-lhes banquete, e cozeu bolos sem
levedura, e comeram.

E antes que se deitassem, cercaram a casa, os homens daquela
cidade, os homens de Sodoma, desde o moço até ao velho; todo o povo
de todos os bairros. E chamaram a Ló, e disseram-lhe: Onde estão
os homens que a ti vieram nesta noite? Traze-os fora a nós, para que
os conheçamos. Então saiu Ló a eles à porta, e fechou a porta
atrás de si, e disse: Meus irmãos, rogo-vos que não façais mal;
eis aqui, duas filhas tenho, que ainda não conheceram homens;
fora vo-las trarei, e fareis delas como bom for aos vossos olhos;
somente nada façais a estes homens, porque por isso vieram à sombra
do meu telhado. Eles, porém, disseram: Sai daí. Disseram mais:
Como estrangeiro este indivíduo veio aqui habitar, e quereria ser
juiz em tudo? Agora te faremos mais mal a ti do que a eles. E
arremessaram-se sobre o homem, sobre Ló, e aproximaram-se para
arrombar a porta. Aqueles homens porém estenderam as suas
mãos e fizeram entrar a Ló consigo na casa, e fecharam a porta;
e feriram de cegueira os homens que estavam à porta da casa,
desde o menor até ao maior, de maneira que se cansaram para achar a
porta.

Então disseram aqueles homens a Ló: Tens alguém mais aqui? Teu
genro, e teus filhos, e tuas filhas, e todos quantos tens nesta
cidade, tira-os fora deste lugar; porque nós vamos destruir
este lugar, porque o seu clamor tem aumentado diante da face do
Senhor, e o Senhor nos enviou a destruí-lo. Então saiu Ló, e
falou a seus genros, aos que haviam de tomar as suas filhas, e
disse: Levantai-vos, saí deste lugar, porque o Senhor há de destruir
a cidade. Foi tido porém por zombador aos olhos de seus genros.

E ao amanhecer os anjos apertaram com Ló, dizendo: Levanta-te,
toma tua mulher e tuas duas filhas que aqui estão, para que não
pereças na injustiça desta cidade. Ele, porém, demorava-se, e
aqueles homens lhe pegaram pela mão, e pela mão de sua mulher e de
suas duas filhas, sendo-lhe o \textbf{Senhor misericordioso}, e
tiraram-no, e puseram-no fora da cidade. E aconteceu que,
tirando-os fora, disse: Escapa-te por tua vida; não olhes para trás
de ti, e não pares em toda esta campina; escapa lá para o monte,
para que não pereças. E Ló disse-lhe: Ora, não, meu Senhor!
Eis que agora o teu servo tem achado graça aos teus olhos, e
engrandeceste a tua misericórdia que a mim me fizeste, para guardar
a minha alma em vida; mas eu não posso escapar no monte, para que
porventura não me apanhe este mal, e eu morra. Eis que agora
aquela cidade está perto, para fugir para lá, e é pequena; ora,
deixe-me escapar para lá (não é pequena?), para que minha alma viva.
E disse-lhe: Eis aqui, tenho-te aceitado também neste
negócio, para não destruir aquela cidade, de que falaste;
apressa-te, escapa-te para ali; porque nada poderei fazer,
enquanto não tiveres ali chegado. Por isso se chamou o nome da
cidade Zoar. Saiu o sol sobre a terra, quando Ló entrou em
Zoar.

Então o Senhor fez chover enxofre e fogo, do Senhor desde os
céus, sobre Sodoma e Gomorra; e destruiu aquelas cidades e
toda aquela campina, e todos os moradores daquelas cidades, e o que
nascia da terra.

E a mulher de Ló olhou para trás e ficou convertida numa estátua
de sal.

E Abraão levantou-se aquela mesma manhã, de madrugada, e foi para
aquele lugar onde estivera diante da face do Senhor; e olhou
para Sodoma e Gomorra e para toda a terra da campina; e viu, que a
fumaça da terra subia, como a de uma fornalha. E aconteceu
que, destruindo Deus as cidades da campina, lembrou-se Deus de
Abraão, e tirou a Ló do meio da destruição, derrubando aquelas
cidades em que Ló habitara.

E subiu Ló de Zoar, e habitou no monte, e as suas duas filhas com
ele; porque temia habitar em Zoar; e habitou numa caverna, ele e as
suas duas filhas. Então a primogênita disse à menor: Nosso
pai já é velho, e não há homem na terra que entre a nós, segundo o
costume de toda a terra; vem, demos de beber vinho a nosso
pai, e deitemo-nos com ele, para que em vida conservemos a
descendência de nosso pai. E deram de beber vinho a seu pai
naquela noite; e veio a primogênita e deitou-se com seu pai, e não
sentiu ele quando ela se deitou, nem quando se levantou. E
sucedeu, no outro dia, que a primogênita disse à menor: Vês aqui, eu
já ontem à noite me deitei com meu pai; demos-lhe de beber vinho
também esta noite, e então entra tu, deita-te com ele, para que em
vida conservemos a descendência de nosso pai. E deram de
beber vinho a seu pai também naquela noite; e levantou-se a menor, e
deitou-se com ele; e não sentiu ele quando ela se deitou, nem quando
se levantou. E conceberam as duas filhas de Ló de seu pai.
E a primogênita deu à luz um filho, e chamou-lhe
\textbf{Moabe}; este é o pai dos moabitas até ao dia de hoje.
E a menor também deu à luz um filho, e chamou-lhe
\textbf{Ben-Ami}; este é o pai dos filhos de \textbf{Amom} até o dia
de hoje.

\smallskip

\lettrine{20} E partiu Abraão dali para a terra do sul, e
habitou entre Cades e Sur; e peregrinou em Gerar. E havendo
Abraão dito de Sara, sua mulher: É minha irmã; enviou Abimeleque,
rei de Gerar, e tomou a Sara.

Deus, porém, veio a Abimeleque em sonhos de noite, e disse-lhe:
Eis que morto serás por causa da mulher que tomaste; porque ela tem
marido. Mas Abimeleque ainda não se tinha chegado a ela; por
isso disse: Senhor, matarás também uma nação justa? Não me disse
ele mesmo: É minha irmã? E ela também disse: É meu irmão. Em
sinceridade do coração e em pureza das minhas mãos tenho feito isto.
E disse-lhe Deus em sonhos: Bem sei eu que na sinceridade do teu
coração fizeste isto; e também eu te tenho impedido de pecar contra
mim; por isso não te permiti tocá-la. Agora, pois, restitui a
mulher ao seu marido, porque \textbf{profeta é}, e rogará por ti,
para que vivas; porém se não lha restituíres, sabe que certamente
morrerás, tu e tudo o que é teu.

E levantou-se Abimeleque pela manhã de madrugada, chamou a todos
os seus servos, e falou todas estas palavras em seus ouvidos; e
temeram muito aqueles homens. Então chamou Abimeleque a Abraão e
disse-lhe: Que nos fizeste? E em que pequei contra ti, para trazeres
sobre o meu reino tamanho pecado? Tu me fizeste aquilo que não
deverias ter feito. Disse mais Abimeleque a Abraão: Que tens
visto, para fazer tal coisa? E disse Abraão: Porque eu dizia
comigo: Certamente não há temor de Deus neste lugar, e eles me
matarão por causa da minha mulher. E, na verdade, é ela
também minha irmã, filha de meu pai, mas não filha da minha mãe; e
veio a ser minha mulher; e aconteceu que, fazendo-me Deus
sair errante da casa de meu pai, eu lhe disse: Seja esta a graça que
me farás em todo o lugar aonde chegarmos, dize de mim: É meu irmão.

Então tomou Abimeleque ovelhas e vacas, e servos e servas, e os
deu a Abraão; e restituiu-lhe Sara, sua mulher. E disse
Abimeleque: Eis que a minha terra está diante da tua face; habita
onde for bom aos teus olhos. E a Sara disse: Vês que tenho
dado ao teu irmão mil moedas de prata; eis que ele te seja por véu
dos olhos para com todos os que contigo estão, e até para com todos
os outros; e estás advertida. E orou Abraão a Deus, e sarou
Deus a Abimeleque, e à sua mulher, e às suas servas, de maneira que
tiveram filhos; porque o Senhor havia fechado totalmente
todas as madres da casa de Abimeleque, por causa de Sara, mulher de
Abraão.

\smallskip

\lettrine{21} E o Senhor visitou a Sara, como tinha dito; e
fez o Senhor a Sara como tinha prometido. E concebeu Sara, e deu
a Abraão um filho na sua velhice, ao tempo determinado, que Deus lhe
tinha falado. E Abraão pôs no filho que lhe nascera, que Sara
lhe dera, o nome de \textbf{Isaque}. E Abraão circuncidou o seu
filho Isaque, quando era da idade de oito dias, como Deus lhe tinha
ordenado. E era Abraão da idade de cem anos, quando lhe nasceu
Isaque seu filho. E disse Sara: Deus me tem feito riso; todo
aquele que o ouvir se rirá comigo. Disse mais: Quem diria a
Abraão que Sara daria de mamar a filhos? Pois lhe dei um filho na
sua velhice. E cresceu o menino, e foi desmamado; então Abraão
fez um grande banquete no dia em que Isaque foi desmamado.

E viu Sara que o filho de Agar, a egípcia, o qual tinha dado a
Abraão, zombava. E disse a Abraão: Ponha fora esta serva e o
seu filho; porque o filho desta serva não herdará com Isaque, meu
filho. E pareceu esta palavra muito má aos olhos de Abraão,
por causa de seu filho. Porém Deus disse a Abraão: Não te
pareça mal aos teus olhos acerca do moço e acerca da tua serva; em
tudo o que Sara te diz, ouve a sua voz; porque em \textbf{Isaque
será chamada a tua descendência}. Mas também do filho desta
serva farei uma nação, porquanto é tua descendência.

Então se levantou Abraão pela manhã de madrugada, e tomou pão e
um odre de água e os deu a Agar, pondo-os sobre o seu ombro; também
lhe deu o menino e despediu-a; e ela partiu, andando errante no
deserto de Berseba. E consumida a água do odre, lançou o
menino debaixo de uma das árvores. E foi assentar-se em
frente, afastando-se à distância de um tiro de arco; porque dizia:
Que eu não veja morrer o menino. E assentou-se em frente, e levantou
a sua voz, e chorou. E ouviu Deus a voz do menino, e bradou o
anjo de Deus a Agar desde os céus, e disse-lhe: Que tens, Agar? Não
temas, porque Deus ouviu a voz do menino desde o lugar onde está.
Ergue-te, levanta o menino e pega-lhe pela mão, porque dele
farei uma grande nação. E abriu-lhe Deus os olhos, e viu um
poço de água; e foi encher o odre de água, e deu de beber ao menino.
E era Deus com o menino, que cresceu; e habitou no deserto, e
foi flecheiro. E habitou no deserto de Parã; e sua mãe
tomou-lhe mulher da terra do Egito.

E aconteceu naquele mesmo tempo que Abimeleque, com Ficol,
príncipe do seu exército, falou com Abraão, dizendo: Deus é contigo
em tudo o que fazes; agora, pois, jura-me aqui por Deus, que
não mentirás a mim, nem a meu filho, nem a meu neto; segundo a
beneficência que te fiz, me farás a mim, e à terra onde
peregrinaste. E disse Abraão: Eu jurarei. Abraão,
porém, repreendeu a Abimeleque por causa de um poço de água, que os
servos de Abimeleque haviam tomado à força. Então disse
Abimeleque: Eu não sei quem fez isto; e também tu não mo fizeste
saber, nem eu o ouvi senão hoje. E tomou Abraão ovelhas e
vacas, e deu-as a Abimeleque; e fizeram ambos uma aliança.
Pôs Abraão, porém, à parte sete cordeiras do rebanho.
E Abimeleque disse a Abraão: Para que estão aqui estas sete
cordeiras, que puseste à parte? E disse: Tomarás estas sete
cordeiras de minha mão, para que sejam em testemunho que eu cavei
este poço. Por isso se chamou aquele lugar Berseba, porquanto
ambos juraram ali. Assim fizeram aliança em Berseba. Depois
se levantou Abimeleque e Ficol, príncipe do seu exército, e
tornaram-se para a terra dos filisteus.

E plantou um bosque em Berseba, e invocou lá o nome do Senhor,
Deus eterno. E peregrinou Abraão na terra dos filisteus
muitos dias.

\smallskip

\lettrine{22} E aconteceu depois destas coisas, que provou
Deus a Abraão, e disse-lhe: Abraão! E ele disse: Eis-me aqui. E
disse: Toma agora o teu filho, o teu único filho, Isaque, a quem
amas, e vai-te à terra de Moriá, e oferece-o ali em holocausto sobre
uma das montanhas, que eu te direi.

Então se levantou Abraão pela manhã de madrugada, e
albardou\footnote{Albardar: aparelhar (besta de carga) com albarda
ou albardão (sela grosseira).} o seu jumento, e tomou consigo dois
de seus moços e Isaque seu filho; e cortou lenha para o holocausto,
e levantou-se, e foi ao lugar que Deus lhe dissera. Ao terceiro
dia levantou Abraão os seus olhos, e viu o lugar de longe. E
disse Abraão a seus moços: Ficai-vos aqui com o jumento, e eu e o
moço iremos até ali; e havendo adorado, tornaremos a vós. E
tomou Abraão a lenha do holocausto, e pô-la sobre Isaque seu filho;
e ele tomou o fogo e o cutelo na sua mão, e foram ambos juntos.
Então falou Isaque a Abraão seu pai, e disse: Meu pai! E ele
disse: Eis-me aqui, meu filho! E ele disse: Eis aqui o fogo e a
lenha, mas onde está o cordeiro para o holocausto? E disse
Abraão: Deus proverá para si o cordeiro para o holocausto, meu
filho. Assim caminharam ambos juntos. E chegaram ao lugar que
Deus lhe dissera, e edificou Abraão ali um altar e pôs em ordem a
lenha, e amarrou a Isaque seu filho, e deitou-o sobre o altar em
cima da lenha. E estendeu Abraão a sua mão, e tomou o cutelo
para imolar o seu filho.

Mas o anjo do Senhor lhe bradou desde os céus, e disse: Abraão,
Abraão! E ele disse: Eis-me aqui. Então disse: Não estendas a
tua mão sobre o moço, e não lhe faças nada; porquanto agora sei que
temes a Deus, e não me negaste o teu filho, o teu único filho.
Então levantou Abraão os seus olhos e olhou; e eis um
carneiro detrás dele, travado pelos seus chifres, num mato; e foi
Abraão, e tomou o carneiro, e ofereceu-o em holocausto, em lugar de
seu filho. E chamou Abraão o nome daquele lugar: \textbf{O
Senhor PROVERÁ}; donde se diz até ao dia de hoje: No monte do Senhor
se proverá.

Então o anjo do Senhor bradou a Abraão pela segunda vez desde os
céus, e disse: Por mim mesmo jurei, diz o Senhor: Porquanto
fizeste esta ação, e não me negaste o teu filho, o teu único filho,
que deveras te abençoarei, e grandissimamente multiplicarei a
tua descendência como as estrelas dos céus, e como a areia que está
na praia do mar; e a tua descendência possuirá a porta dos seus
inimigos; e em tua descendência serão benditas todas as
nações da terra; porquanto obedeceste à minha voz. Então
Abraão tornou aos seus moços, e levantaram-se, e foram juntos para
Berseba; e Abraão habitou em Berseba.

E sucedeu depois destas coisas, que anunciaram a Abraão, dizendo:
Eis que também Milca deu filhos a Naor teu irmão. Uz o seu
primogênito, e Buz seu irmão, e Quemuel, pai de Arã, e
Quésede, e Hazo, e Pildas, e Jidlafe, e Betuel. E Betuel
gerou \textbf{Rebeca}. Estes oito deu à luz Milca a Naor, irmão de
Abraão. E a sua concubina, cujo nome era Reumá, ela lhe deu
também a Tebá, Gaã, Taás e Maaca.

\smallskip

\lettrine{23} E foi a vida de Sara cento e vinte e sete anos;
estes foram os anos da vida de Sara. E morreu Sara em
Quiriate-Arba, que é Hebrom, na terra de Canaã; e veio Abraão
lamentar Sara e chorar por ela.

Depois se levantou Abraão de diante de sua morta, e falou aos
filhos de Hete, dizendo: Estrangeiro e peregrino sou entre vós;
dai-me possessão de sepultura convosco, para que eu sepulte a minha
morta de diante da minha face. E responderam os filhos de Hete a
Abraão, dizendo-lhe: Ouve-nos, meu senhor; príncipe poderoso és
no meio de nós; enterra a tua morta na mais escolhida de nossas
sepulturas; nenhum de nós te vedará a sua sepultura, para enterrar a
tua morta. Então se levantou Abraão, inclinou-se diante do povo
da terra, diante dos filhos de Hete, e falou com eles, dizendo:
Se é de vossa vontade que eu sepulte a minha morta de diante de
minha face, ouvi-me e falai por mim a Efrom, filho de Zoar, que
ele me dê a cova de \textbf{Macpela}, que ele tem no fim do seu
campo; que ma dê pelo devido preço em herança de sepulcro no meio de
vós. Ora Efrom habitava no meio dos filhos de Hete; e
respondeu Efrom, heteu, a Abraão, aos ouvidos dos filhos de Hete, de
todos os que entravam pela porta da sua cidade, dizendo: Não,
meu senhor, ouve-me: O campo te dou, também te dou a cova que nele
está, diante dos olhos dos filhos do meu povo ta dou; sepulta a tua
morta. Então Abraão se inclinou diante da face do povo da
terra, e falou a Efrom, aos ouvidos do povo da terra,
dizendo: Mas se tu estás por isto, ouve-me, peço-te. O preço do
campo o darei; toma-o de mim e sepultarei ali a minha morta.
E respondeu Efrom a Abraão, dizendo-lhe: Meu senhor,
ouve-me, a terra é de quatrocentos siclos de prata; que é isto entre
mim e ti? Sepulta a tua morta.

E Abraão deu ouvidos a Efrom, e Abraão pesou a Efrom a prata de
que tinha falado aos ouvidos dos filhos de Hete, quatrocentos siclos
de prata, corrente entre mercadores. Assim o campo de Efrom,
que estava em Macpela, em frente de Manre, o campo e a cova que nele
estava, e todo o arvoredo que no campo havia, que estava em todo o
seu contorno ao redor, se confirmou a Abraão em possessão
diante dos olhos dos filhos de Hete, de todos os que entravam pela
porta da cidade. E depois sepultou Abraão a Sara sua mulher
na cova do campo de Macpela, em frente de Manre, que é Hebrom, na
terra de Canaã. Assim o campo e a cova que nele estava foram
confirmados a Abraão, pelos filhos de Hete, em possessão de
sepultura.

\smallskip

\lettrine{24} E era Abraão já velho e adiantado em idade, e o
Senhor havia abençoado a Abraão em tudo. E disse Abraão ao seu
servo, o mais velho da casa, que tinha o governo sobre tudo o que
possuía: Põe agora a tua mão debaixo da minha coxa, para que eu
te faça jurar pelo Senhor Deus dos céus e Deus da terra, que não
tomarás para meu filho mulher das filhas dos cananeus, no meio dos
quais eu habito. Mas que irás à minha terra e à minha parentela,
e dali tomarás mulher para meu filho Isaque. E disse-lhe o
servo: Se porventura não quiser seguir-me a mulher a esta terra,
farei, pois, tornar o teu filho à terra donde saíste? E Abraão
lhe disse: Guarda-te, que não faças lá tornar o meu filho. O
Senhor Deus dos céus, que me tomou da casa de meu pai e da terra da
minha parentela, e que me falou, e que me jurou, dizendo: À tua
descendência darei esta terra; ele enviará o seu anjo adiante da tua
face, para que tomes mulher de lá para meu filho. Se a mulher,
porém, não quiser seguir-te, serás livre deste meu juramento;
somente não faças lá tornar a meu filho. Então pôs o servo a sua
mão debaixo da coxa de Abraão seu senhor, e jurou-lhe sobre este
negócio.

E o servo tomou dez camelos, dos camelos do seu senhor, e partiu,
pois que todos os bens de seu senhor estavam em sua mão, e
levantou-se e partiu para \textbf{Mesopotâmia}, para a cidade de
Naor. E fez ajoelhar os camelos fora da cidade, junto a um
poço de água, pela tarde, ao tempo que as moças saíam a tirar água.
E disse: Ó Senhor, Deus de meu senhor Abraão, dá-me hoje bom
encontro, e faze beneficência ao meu senhor Abraão! Eis que
eu estou em pé junto à fonte de água e as filhas dos homens desta
cidade saem para tirar água; seja, pois, que a donzela, a
quem eu disser: Abaixa agora o teu cântaro para que eu beba; e ela
disser: Bebe, e também darei de beber aos teus camelos; esta seja a
quem designaste ao teu servo Isaque, e que eu conheça nisso que
usaste de benevolência com meu senhor. E sucedeu que, antes
que ele acabasse de falar, eis que Rebeca, que havia nascido a
Betuel, filho de Milca, mulher de Naor, irmão de Abraão, saía com o
seu cântaro sobre o seu ombro. E a donzela era mui formosa à
vista, virgem, a quem homem não havia conhecido; e desceu à fonte, e
encheu o seu cântaro e subiu. Então o servo correu-lhe ao
encontro, e disse: Peço-te, deixa-me beber um pouco de água do teu
cântaro. E ela disse: Bebe, meu Senhor. E apressou-se e
abaixou o seu cântaro sobre a sua mão e deu-lhe de beber. E,
acabando ela de lhe dar de beber, disse: Tirarei também água para os
teus camelos, até que acabem de beber. E apressou-se, e
despejou o seu cântaro no bebedouro, e correu outra vez ao poço para
tirar água, e tirou para todos os seus camelos. E o homem
estava admirado de vê-la, calando-se, para saber se o Senhor havia
prosperado a sua jornada ou não. E aconteceu que, acabando os
camelos de beber, tomou o homem um pendente de ouro de meio siclo de
peso, e duas pulseiras para as suas mãos, do peso de dez siclos de
ouro; e disse: De quem és filha? Faze-mo saber, peço-te. Há
também em casa de teu pai lugar para nós pousarmos? E ela lhe
disse: Eu sou a filha de Betuel, filho de Milca, o qual ela deu a
Naor. Disse-lhe mais: Também temos palha e muito pasto, e
lugar para passar a noite. Então inclinou-se aquele homem e
adorou ao Senhor, e disse: Bendito seja o Senhor Deus de meu
senhor Abraão, que não retirou a sua benevolência e a sua verdade de
meu senhor; quanto a mim, o Senhor me guiou no caminho à casa dos
irmãos de meu senhor. E a donzela correu, e fez saber estas
coisas na casa de sua mãe.

E Rebeca tinha um irmão cujo nome era Labão, o qual correu ao
encontro daquele homem até a fonte. E aconteceu que, quando
ele viu o pendente, e as pulseiras sobre as mãos de sua irmã, e
quando ouviu as palavras de sua irmã Rebeca, que dizia: Assim me
falou aquele homem; foi ter com o homem, que estava em pé junto aos
camelos, à fonte, e disse: Entra, bendito do Senhor; por que
estás fora? pois eu já preparei a casa, e o lugar para os camelos.
Então veio aquele homem à casa, e desataram os camelos, e
deram palha e pasto aos camelos, e água para lavar os pés dele, e os
pés dos homens que estavam com ele. Depois puseram comida
diante dele. Ele, porém, disse: Não comerei, até que tenha dito as
minhas palavras. E ele disse: Fala. Então disse: Eu sou o
servo de Abraão. E o Senhor abençoou muito o meu senhor, de
maneira que foi engrandecido, e deu-lhe ovelhas e vacas, e prata e
ouro, e servos e servas, e camelos e jumentos. E Sara, a
mulher do meu senhor, deu à luz um filho a meu senhor depois da sua
velhice, e ele deu-lhe tudo quanto tem. E meu senhor me fez
jurar, dizendo: Não tomarás mulher para meu filho das filhas dos
cananeus, em cuja terra habito; irás, porém, à casa de meu
pai, e à minha família, e tomarás mulher para meu filho.
Então disse eu ao meu senhor: Porventura não me seguirá a
mulher. E ele me disse: O Senhor, em cuja presença tenho
andado, enviará o seu anjo contigo, e prosperará o teu caminho, para
que tomes mulher para meu filho da minha família e da casa de meu
pai; então serás livre do meu juramento, quando fores à minha
família; e se não te derem, livre serás do meu juramento. E
hoje cheguei à fonte, e disse: Ó Senhor, Deus de meu senhor Abraão,
se tu agora prosperas o meu caminho, no qual eu ando, eis que
estou junto à fonte de água; seja, pois, que a donzela que sair para
tirar água e à qual eu disser: Peço-te, dá-me um pouco de água do
teu cântaro; e ela me disser: Bebe tu e também tirarei água
para os teus camelos; esta seja a mulher que o Senhor designou ao
filho de meu senhor. E antes que eu acabasse de falar no meu
coração, eis que Rebeca saía com o seu cântaro sobre o seu ombro,
desceu à fonte e tirou água; e eu lhe disse: Peço-te, dá-me de
beber. E ela se apressou, e abaixou o seu cântaro de sobre
si, e disse: Bebe, e também darei de beber aos teus camelos; e bebi,
e ela deu também de beber aos camelos. Então lhe perguntei, e
disse: De quem és filha? E ela disse: Filha de Betuel, filho de
Naor, que lhe deu Milca. Então eu pus o pendente no seu rosto, e as
pulseiras sobre as suas mãos; e inclinando-me adorei ao
Senhor, e bendisse ao Senhor, Deus do meu senhor Abraão, que me
havia encaminhado pelo caminho da verdade, para tomar a filha do
irmão de meu senhor para seu filho. Agora, pois, se vós
haveis de fazer benevolência e verdade a meu senhor, fazei-mo saber;
e se não, também mo fazei saber, para que eu vá à direita, ou à
esquerda. Então responderam Labão e Betuel, e disseram: Do
Senhor procedeu este negócio; não podemos falar-te mal ou bem.
Eis que Rebeca está diante da tua face; toma-a, e vai-te;
seja a mulher do filho de teu senhor, como tem dito o Senhor.
E aconteceu que, o servo de Abraão, ouvindo as suas palavras,
inclinou-se à terra diante do Senhor. E tirou o servo jóias
de prata e jóias de ouro, e vestidos, e deu-os a Rebeca; também deu
coisas preciosas a seu irmão e à sua mãe.

Então comeram e beberam, ele e os homens que com ele estavam, e
passaram a noite. E levantaram-se pela manhã, e disse: Deixai-me ir
a meu senhor. Então disseram seu irmão e sua mãe: Fique a
donzela conosco alguns dias, ou pelo menos dez dias, depois irá.
Ele, porém, lhes disse: Não me detenhais, pois o Senhor tem
prosperado o meu caminho; deixai-me partir, para que eu volte a meu
senhor. E disseram: Chamemos a donzela, e perguntemos-lho.
E chamaram a Rebeca, e disseram-lhe: Irás tu com este homem?
Ela respondeu: Irei. Então despediram a Rebeca, sua irmã, e
sua ama, e o servo de Abraão, e seus homens. E abençoaram a
Rebeca, e disseram-lhe: Ó nossa irmã, sê tu a mãe de milhares de
milhares, e que a tua descendência possua a porta de seus
aborrecedores! E Rebeca se levantou com as suas moças, e
subiram sobre os camelos, e seguiram o homem; e tomou aquele servo a
Rebeca, e partiu.

Ora, Isaque vinha de onde se vem do poço de Beer-Laai-Rói; porque
habitava na terra do sul. E Isaque saíra a orar no campo, à
tarde; e levantou os seus olhos, e olhou, e eis que os camelos
vinham. Rebeca também levantou seus olhos, e viu a Isaque, e
desceu do camelo. E disse ao servo: Quem é aquele homem que
vem pelo campo ao nosso encontro? E o servo disse: Este é meu
Senhor. Então tomou ela o véu e cobriu-se. E o servo contou a
Isaque todas as coisas que fizera. E Isaque trouxe-a para a
tenda de sua mãe Sara, e tomou a Rebeca, e foi-lhe por mulher, e
amou-a. Assim Isaque foi consolado depois da morte de sua mãe.

\smallskip

\lettrine{25} E Abraão tomou outra mulher; e o seu nome era
Quetura; e deu-lhe à luz Zinrã, Jocsã, Medã, Midiã, Jisbaque e
Suá. E Jocsã gerou Seba e Dedã; e os filhos de Dedã foram
Assurim, Letusim e Leumim. E os filhos de Midiã foram Efá, Efer,
Enoque, Abida e Elda. Estes todos foram filhos de Quetura. Porém
Abraão deu tudo o que tinha a Isaque; mas aos filhos das
concubinas que Abraão tinha, deu Abraão presentes e, vivendo ele
ainda, despediu-os do seu filho Isaque, enviando-os ao oriente, para
a terra oriental. Estes, pois, são os dias dos anos da vida de
Abraão, que viveu cento e setenta e cinco anos. E Abraão
expirou, morrendo em boa velhice, velho e farto de dias; e foi
congregado ao seu povo; e Isaque e Ismael, seus filhos,
sepultaram-no na cova de Macpela, no campo de Efrom, filho de Zoar,
heteu, que estava em frente de Manre, o campo que Abraão
comprara aos filhos de Hete. Ali está sepultado Abraão e Sara, sua
mulher.

E aconteceu depois da morte de Abraão, que Deus abençoou a Isaque
seu filho; e habitava Isaque junto ao poço Beer-Laai-Rói.
Estas, porém, são as gerações de Ismael filho de Abraão, que
a serva de Sara, Agar, egípcia, deu a Abraão. E estes são os
nomes dos filhos de Ismael, pelos seus nomes, segundo as suas
gerações: O primogênito de Ismael era Nebaiote, depois Quedar,
Adbeel e Mibsão, Misma, Dumá, Massá, Hadade, Tema,
Jetur, Nafis e Quedemá. Estes são os filhos de Ismael, e
estes são os seus nomes pelas suas vilas e pelos seus castelos; doze
príncipes segundo as suas famílias. E estes são os anos da
vida de Ismael, cento e trinta e sete anos, e ele expirou e,
morrendo, foi congregado ao seu povo. E habitaram desde
Havilá até Sur, que está em frente do Egito, como quem vai para a
Assíria; e fez o seu assento diante da face de todos os seus irmãos.

E estas são as gerações de Isaque, filho de Abraão: Abraão gerou
a Isaque; e era Isaque da idade de quarenta anos, quando
tomou por mulher a Rebeca, filha de Betuel, arameu de Padã-Arã, irmã
de Labão, arameu. E Isaque orou insistentemente ao Senhor por
sua mulher, porquanto era estéril; e o Senhor ouviu as suas orações,
e Rebeca sua mulher concebeu. E os filhos lutavam dentro
dela; então disse: Se assim é, por que sou eu assim? E foi perguntar
ao Senhor. E o Senhor lhe disse: Duas nações há no teu
ventre, e dois povos se dividirão das tuas entranhas, e um povo será
mais forte do que o outro povo, e o maior servirá ao menor. E
cumprindo-se os seus dias para dar à luz, eis gêmeos no seu ventre.
E saiu o primeiro ruivo e todo como um vestido de pêlo; por
isso chamaram o seu nome \textbf{Esaú}. E depois saiu o seu
irmão, agarrada sua mão ao calcanhar de Esaú; por isso se chamou o
seu nome \textbf{Jacó}. E era Isaque da idade de sessenta anos
quando os gerou. E cresceram os meninos, e Esaú foi homem
perito na caça, homem do campo; mas Jacó era homem simples,
habitando em tendas. E amava Isaque a Esaú, porque a caça era
de seu gosto, mas Rebeca amava a Jacó.

E Jacó cozera um guisado; e veio Esaú do campo, e estava ele
cansado; e disse Esaú a Jacó: Deixa-me, peço-te, comer desse
guisado vermelho, porque estou cansado. Por isso se chamou
\textbf{Edom}. Então disse Jacó: Vende-me hoje a tua
primogenitura. E disse Esaú: Eis que estou a ponto de morrer;
para que me servirá a primogenitura? Então disse Jacó:
Jura-me hoje. E jurou-lhe e vendeu a sua primogenitura a Jacó.
E Jacó deu pão a Esaú e o guisado de lentilhas; e ele comeu,
e bebeu, e levantou-se, e saiu. Assim \textbf{desprezou Esaú a sua
primogenitura}.

\smallskip

\lettrine{26} E havia fome na terra, além da primeira fome,
que foi nos dias de Abraão; por isso foi Isaque a Abimeleque, rei
dos filisteus, em Gerar. E apareceu-lhe o Senhor, e disse: Não
desças ao Egito; habita na terra que eu te disser; peregrina
nesta terra, e serei contigo, e te abençoarei; porque a ti e à tua
descendência darei todas estas terras, e confirmarei o juramento que
tenho jurado a Abraão teu pai; e multiplicarei a tua
descendência como as estrelas dos céus, e darei à tua descendência
todas estas terras; e por meio dela serão benditas todas as nações
da terra; porquanto Abraão obedeceu à minha voz, e guardou o meu
mandado, os meus preceitos, os meus estatutos, e as minhas leis.

Assim habitou Isaque em Gerar. E perguntando-lhe os homens
daquele lugar acerca de sua mulher, disse: É minha irmã; porque
temia dizer: É minha mulher; para que porventura (dizia ele) não me
matem os homens daquele lugar por amor de Rebeca; porque era formosa
à vista. E aconteceu que, como ele esteve ali muito tempo,
Abimeleque, rei dos filisteus, olhou por uma janela, e viu, e eis
que Isaque estava brincando com Rebeca sua mulher. Então chamou
Abimeleque a Isaque, e disse: Eis que na verdade é tua mulher; como
pois disseste: É minha irmã? E disse-lhe Isaque: Porque eu dizia:
Para que eu porventura não morra por causa dela. E disse
Abimeleque: Que é isto que nos fizeste? Facilmente se teria deitado
alguém deste povo com a tua mulher, e tu terias trazido sobre nós um
delito. E mandou Abimeleque a todo o povo, dizendo: Qualquer
que tocar neste homem ou em sua mulher, certamente morrerá.

E semeou Isaque naquela mesma terra, e colheu naquele mesmo ano
cem medidas, porque o Senhor o abençoava. E engrandeceu-se o
homem, e ia enriquecendo-se, até que se tornou mui poderoso.
E tinha possessão de ovelhas, e possessão de vacas, e muita
gente de serviço, de maneira que os filisteus o invejavam. E
todos os poços, que os servos de seu pai tinham cavado nos dias de
seu pai Abraão, os filisteus entulharam e encheram de terra.
Disse também Abimeleque a Isaque: Aparta-te de nós; porque
muito mais poderoso te tens feito do que nós. Então Isaque
partiu dali e fez o seu acampamento no vale de Gerar, e habitou lá.
E tornou Isaque e cavou os poços de água que cavaram nos dias
de Abraão seu pai, e que os filisteus entulharam depois da morte de
Abraão, e chamou-os pelos nomes que os chamara seu pai.
Cavaram, pois, os servos de Isaque naquele vale, e acharam
ali um poço de águas vivas. E os pastores de Gerar porfiaram
com os pastores de Isaque, dizendo: Esta água é nossa. Por isso
chamou aquele poço Eseque, porque contenderam com ele. Então
cavaram outro poço, e também porfiaram sobre ele; por isso chamou-o
Sitna. E partiu dali, e cavou outro poço, e não porfiaram
sobre ele; por isso chamou-o Reobote, e disse: Porque agora nos
alargou o Senhor, e crescemos nesta terra. Depois subiu dali
a Berseba. \textbf{E apareceu-lhe o Senhor} naquela mesma
noite, e disse: Eu sou o Deus de Abraão teu pai; não temas, porque
eu sou contigo, e abençoar-te-ei, e multiplicarei a tua descendência
por amor de Abraão meu servo. Então edificou ali um altar, e
invocou o nome do Senhor, e armou ali a sua tenda; e os servos de
Isaque cavaram ali um poço.

E Abimeleque veio a ele de Gerar, com Auzate seu amigo, e Ficol,
príncipe do seu exército. E disse-lhes Isaque: Por que
viestes a mim, pois que vós me odiais e me repelistes de vós?
E eles disseram: Havemos visto, na verdade, que o Senhor é
contigo, por isso dissemos: Haja agora juramento entre nós, entre
nós e ti; e façamos aliança contigo. Que não nos faças mal,
como nós te não temos tocado, e como te fizemos somente bem, e te
deixamos ir em paz. Agora tu és o bendito do Senhor. Então
lhes fez um banquete, e comeram e beberam; e levantaram-se de
madrugada e juraram um ao outro; depois os despediu Isaque, e
despediram-se dele em paz. E aconteceu, naquele mesmo dia,
que vieram os servos de Isaque, e anunciaram-lhe acerca do negócio
do poço, que tinham cavado; e disseram-lhe: Temos achado água.
E chamou-o Seba; por isso é o nome daquela cidade
\textbf{Berseba} até o dia de hoje.

Ora, sendo Esaú da idade de quarenta anos, tomou por mulher a
Judite, filha de Beeri, heteu, e a Basemate, filha de Elom, heteu.
E estas foram para Isaque e Rebeca uma amargura de espírito.

\smallskip

\lettrine{27} E aconteceu que, como Isaque envelheceu, e os
seus olhos se escureceram, de maneira que não podia ver, chamou a
Esaú, seu filho mais velho, e disse-lhe: Meu filho. E ele lhe disse:
Eis-me aqui. E ele disse: Eis que já agora estou velho, e não
sei o dia da minha morte; agora, pois, toma as tuas armas, a tua
aljava e o teu arco, e sai ao campo, e apanha para mim alguma caça.
E faze-me um guisado saboroso, como eu gosto, e traze-mo, para
que eu coma; para que minha alma te abençoe, antes que morra. E
Rebeca escutou quando Isaque falava ao seu filho Esaú. E foi Esaú ao
campo para apanhar a caça que havia de trazer.

Então falou Rebeca a Jacó seu filho, dizendo: Eis que tenho ouvido
o teu pai que falava com Esaú teu irmão, dizendo: Traze-me caça,
e faze-me um guisado saboroso, para que eu coma, e te abençoe diante
da face do Senhor, antes da minha morte. Agora, pois, filho meu,
ouve a minha voz naquilo que eu te mando: Vai agora ao rebanho,
e traze-me de lá dois bons cabritos, e eu farei deles um guisado
saboroso para teu pai, como ele gosta; e levá-lo-ás a teu
pai, para que o coma; para que te abençoe antes da sua morte.
Então disse Jacó a Rebeca, sua mãe: Eis que Esaú meu irmão é
homem cabeludo, e eu homem liso; porventura me apalpará o meu
pai, e serei aos seus olhos como enganador; assim trarei eu sobre
mim maldição, e não bênção. E disse-lhe sua mãe: Meu filho,
sobre mim seja a tua maldição; somente obedece à minha voz, e vai,
traze-mos. E foi, e tomou-os, e trouxe-os a sua mãe; e sua
mãe fez um guisado saboroso, como seu pai gostava. Depois
tomou Rebeca os vestidos de gala de Esaú, seu filho mais velho, que
tinha consigo em casa, e vestiu a Jacó, seu filho menor; e
com as peles dos cabritos cobriu as suas mãos e a lisura do seu
pescoço; e deu o guisado saboroso e o pão que tinha
preparado, na mão de Jacó seu filho.

E foi ele a seu pai, e disse: Meu pai! E ele disse: Eis-me aqui;
quem és tu, meu filho? E Jacó disse a seu pai: Eu sou Esaú,
teu primogênito; tenho feito como me disseste; levanta-te agora,
assenta-te e come da minha caça, para que a tua alma me abençoe.
Então disse Isaque a seu filho: Como é isto, que tão cedo a
achaste, filho meu? E ele disse: Porque o Senhor teu Deus a mandou
ao meu encontro. E disse Isaque a Jacó: Chega-te agora, para
que te apalpe, meu filho, se és meu filho Esaú mesmo, ou não.
Então se chegou Jacó a Isaque seu pai, que o apalpou, e
disse: A voz é a voz de Jacó, porém as mãos são as mãos de Esaú.
E não o conheceu, porquanto as suas mãos estavam cabeludas,
como as mãos de Esaú seu irmão; e abençoou-o. E disse: És tu
meu filho Esaú mesmo? E ele disse: Eu sou. Então disse: Faze
chegar isso perto de mim, para que coma da caça de meu filho; para
que a minha alma te abençoe. E chegou-lhe, e comeu; trouxe-lhe
também vinho, e bebeu. E disse-lhe Isaque seu pai: Ora
chega-te, e beija-me, filho meu. E chegou-se, e beijou-o;
então sentindo o cheiro das suas vestes, abençoou-o, e disse: Eis
que o cheiro do meu filho é como o cheiro do campo, que o Senhor
abençoou; assim, pois, te dê Deus do orvalho dos céus, e das
gorduras da terra, e abundância de trigo e de mosto.
Sirvam-te povos, e nações se encurvem a ti; sê senhor de teus
irmãos, e os filhos da tua mãe se encurvem a ti; malditos sejam os
que te amaldiçoarem, e benditos sejam os que te abençoarem.

E aconteceu que, acabando Isaque de abençoar a Jacó, apenas Jacó
acabava de sair da presença de Isaque seu pai, veio Esaú, seu irmão,
da sua caça; e fez também ele um guisado saboroso, e trouxe-o
a seu pai; e disse a seu pai: Levanta-te, meu pai, e come da caça de
teu filho, para que me abençoe a tua alma. E disse-lhe Isaque
seu pai: Quem és tu? E ele disse: Eu sou teu filho, o teu
primogênito Esaú. Então estremeceu Isaque de um
estremecimento muito grande, e disse: Quem, pois, é aquele que
apanhou a caça, e ma trouxe? E comi de tudo, antes que tu viesses, e
abençoei-o, e ele será bendito. Esaú, ouvindo as palavras de
seu pai, bradou com grande e mui amargo brado, e disse a seu pai:
Abençoa-me também a mim, meu pai. E ele disse: Veio teu irmão
com sutileza, e tomou a tua bênção. Então disse ele: Não é o
seu nome justamente Jacó, tanto que já duas vezes me enganou? A
minha primogenitura me tomou, e eis que agora me tomou a minha
bênção. E perguntou: Não reservaste, pois, para mim nenhuma bênção?
Então respondeu Isaque a Esaú dizendo: Eis que o tenho posto
por senhor sobre ti, e todos os seus irmãos lhe tenho dado por
servos; e de trigo e de mosto o tenho fortalecido; que te farei,
pois, agora, meu filho? E disse Esaú a seu pai: Tens uma só
bênção, meu pai? Abençoa-me também a mim, meu pai. E levantou Esaú a
sua voz, e chorou. Então respondeu Isaque, seu pai, e
disse-lhe: Eis que a tua habitação será nas gorduras da terra e no
orvalho dos altos céus. E pela tua espada viverás, e ao teu
irmão servirás. Acontecerá, porém, que quando te assenhoreares,
então sacudirás o seu jugo do teu pescoço.

E Esaú odiou a Jacó por causa daquela bênção, com que seu pai o
tinha abençoado; e Esaú disse no seu coração: Chegar-se-ão os dias
de luto de meu pai; e matarei a Jacó meu irmão. E foram
denunciadas a Rebeca estas palavras de Esaú, seu filho mais velho; e
ela mandou chamar a Jacó, seu filho menor, e disse-lhe: Eis que Esaú
teu irmão se consola a teu respeito, propondo matar-te.
Agora, pois, meu filho, ouve a minha voz, e levanta-te;
acolhe-te a Labão meu irmão, em Harã, e mora com ele alguns
dias, até que passe o furor de teu irmão; até que se desvie
de ti a ira de teu irmão, e se esqueça do que lhe fizeste; então
mandarei trazer-te de lá; por que seria eu desfilhada também de vós
ambos num mesmo dia? E disse Rebeca a Isaque: Enfadada estou
da minha vida, por causa das filhas de Hete; se Jacó tomar mulher
das filhas de Hete, como estas são, das filhas desta terra, para que
me servirá a vida?

\smallskip

\lettrine{28} E Isaque chamou a Jacó, e abençoou-o, e
ordenou-lhe, e disse-lhe: Não tomes mulher de entre as filhas de
Canaã; levanta-te, vai a Padã-Arã, à casa de Betuel, pai de tua
mãe, e toma de lá uma mulher das filhas de Labão, irmão de tua mãe;
e Deus Todo-Poderoso te abençoe, e te faça frutificar, e te
multiplique, para que sejas uma multidão de povos; e te dê a
bênção de Abraão, a ti e à tua descendência contigo, para que em
herança possuas a terra de tuas peregrinações, que Deus deu a
Abraão. Assim despediu Isaque a Jacó, o qual se foi a Padã-Arã,
a Labão, filho de Betuel, arameu, irmão de Rebeca, mãe de Jacó e de
Esaú.

Vendo, pois, Esaú que Isaque abençoara a Jacó, e o enviara a
Padã-Arã, para tomar mulher dali para si, e que, abençoando-o, lhe
ordenara, dizendo: Não tomes mulher das filhas de Canaã; e que
Jacó obedecera a seu pai e a sua mãe, e se fora a Padã-Arã;
vendo também Esaú que as filhas de Canaã eram más aos olhos de
Isaque seu pai, foi Esaú a Ismael, e tomou para si por mulher,
além das suas mulheres, a Maalate filha de Ismael, filho de Abraão,
irmã de Nebaiote.

Partiu, pois, Jacó de Berseba, e foi a Harã; e chegou a um
lugar onde passou a noite, porque já o sol era posto; e tomou uma
das pedras daquele lugar, e a pôs por seu travesseiro, e deitou-se
naquele lugar. E sonhou: e eis uma escada posta na terra,
cujo topo tocava nos céus; e eis que os anjos de Deus subiam e
desciam por ela; e eis que o Senhor estava em cima dela, e
disse: \textbf{Eu sou o Senhor Deus de Abraão teu pai, e o Deus de
Isaque; esta terra, em que estás deitado, darei a ti e à tua
descendência};  e a tua descendência será como o pó da terra,
e estender-se-á ao ocidente, e ao oriente, e ao norte, e ao sul, e
em ti e na tua descendência serão benditas todas as famílias da
terra; e eis que estou contigo, e te guardarei por onde quer
que fores, e te farei tornar a esta terra; porque não te deixarei,
até que haja cumprido o que te tenho falado.

Acordando, pois, Jacó do seu sono, disse: Na verdade o Senhor
está neste lugar; e eu não o sabia. E temeu, e disse: Quão
terrível é este lugar! Este não é outro lugar senão a casa de Deus;
e esta é a porta dos céus. Então levantou-se Jacó pela manhã
de madrugada, e tomou a pedra que tinha posto por seu travesseiro, e
a pôs por coluna, e derramou azeite em cima dela. E chamou o
nome daquele lugar \textbf{Betel}; o nome porém daquela cidade antes
era Luz. E Jacó fez um voto, dizendo: Se Deus for comigo, e
me guardar nesta viagem que faço, e me der pão para comer, e vestes
para vestir; e eu em paz tornar à casa de meu pai, o Senhor
me será por Deus; e esta pedra que tenho posto por coluna
será casa de Deus; e de tudo quanto me deres, certamente te darei o
dízimo.

\smallskip

\lettrine{29} Então pôs-se Jacó a caminho e foi à terra do
povo do oriente; e olhou, e eis um poço no campo, e eis três
rebanhos de ovelhas que estavam deitados junto a ele; porque daquele
poço davam de beber aos rebanhos; e havia uma grande pedra sobre a
boca do poço. E ajuntavam ali todos os rebanhos, e removiam a
pedra de sobre a boca do poço, e davam de beber às ovelhas; e
tornavam a pôr a pedra sobre a boca do poço, no seu lugar. E
disse-lhes Jacó: Meus irmãos, donde sois? E disseram: Somos de Harã.
E ele lhes disse: Conheceis a Labão, filho de Naor? E disseram:
Conhecemos. Disse-lhes mais: Está ele bem? E disseram: Está bem,
e eis aqui \textbf{Raquel} sua filha, que vem com as ovelhas. E
ele disse: Eis que ainda é pleno dia, não é tempo de ajuntar o gado;
dai de beber às ovelhas, e ide apascentá-las. E disseram: Não
podemos, até que todos os rebanhos se ajuntem, e removam a pedra de
sobre a boca do poço, para que demos de beber às ovelhas.

Estando ele ainda falando com eles, veio Raquel com as ovelhas de
seu pai; porque ela era pastora. E aconteceu que, vendo Jacó
a Raquel, filha de Labão, irmão de sua mãe, e as ovelhas de Labão,
irmão de sua mãe, chegou Jacó, e revolveu a pedra de sobre a boca do
poço e deu de beber às ovelhas de Labão, irmão de sua mãe. E
Jacó beijou a Raquel, e levantou a sua voz e chorou. E Jacó
anunciou a Raquel que era irmão de seu pai, e que era filho de
Rebeca; então ela correu, e o anunciou a seu pai. E aconteceu
que, ouvindo Labão as novas de Jacó, filho de sua irmã, correu-lhe
ao encontro, e abraçou-o, e beijou-o, e levou-o à sua casa; e ele
contou a Labão todas estas coisas. Então Labão disse-lhe:
Verdadeiramente és tu o meu osso e a minha carne. E ficou com ele um
mês inteiro.

Depois disse Labão a Jacó: Porque tu és meu irmão, hás de
servir-me de graça? Declara-me qual será o teu salário. E
Labão tinha duas filhas; o nome da mais velha era \textbf{Lia}, e o
nome da menor Raquel. Lia tinha olhos tenros, mas Raquel era
de formoso semblante e formosa à vista. E Jacó amava a
Raquel, e disse: Sete anos te servirei por Raquel, tua filha menor.
Então disse Labão: Melhor é que eu a dê a ti, do que eu a dê
a outro homem; fica comigo. Assim serviu Jacó sete anos por
Raquel; e estes lhe pareceram como poucos dias, pelo muito que a
amava. E disse Jacó a Labão: Dá-me minha mulher, porque meus
dias são cumpridos, para que eu me case com ela. Então reuniu
Labão a todos os homens daquele lugar, e fez um banquete. E
aconteceu, à tarde, que tomou Lia, sua filha, e trouxe-a a Jacó que
a possuiu. E Labão deu sua serva Zilpa a Lia, sua filha, por
serva. E aconteceu que pela manhã, viu que era Lia; pelo que
disse a Labão: Por que me fizeste isso? Não te tenho servido por
Raquel? Por que então me enganaste? E disse Labão: Não se faz
assim no nosso lugar, que a menor se dê antes da primogênita.
Cumpre a semana desta; então te daremos também a outra, pelo
serviço que ainda outros sete anos comigo servires. E Jacó
fez assim, e cumpriu a semana de Lia; então lhe deu por mulher
Raquel sua filha. E Labão deu sua serva Bila por serva a
Raquel, sua filha. E possuiu também a Raquel, e amou também a
Raquel mais do que a Lia e serviu com ele ainda outros sete anos.

Vendo, pois, o Senhor que Lia era desprezada, abriu a sua madre;
porém Raquel era estéril. E concebeu Lia, e deu à luz um
filho, e chamou-o \textbf{Rúben}; pois disse: Porque o Senhor
atendeu à minha aflição, por isso agora me amará o meu marido.
E concebeu outra vez, e deu à luz um filho, dizendo:
Porquanto o Senhor ouviu que eu era desprezada, e deu-me também
este. E chamou-o \textbf{Simeão}. E concebeu outra vez, e deu
à luz um filho, dizendo: Agora esta vez se unirá meu marido a mim,
porque três filhos lhe tenho dado. Por isso chamou-o \textbf{Levi}.
E concebeu outra vez e deu à luz um filho, dizendo: Esta vez
louvarei ao Senhor. Por isso chamou-o \textbf{Judá}; e cessou de dar
à luz.

\smallskip

\lettrine{30} Vendo Raquel que não dava filhos a Jacó, teve
inveja de sua irmã, e disse a Jacó: Dá-me filhos, se não morro.
Então se acendeu a ira de Jacó contra Raquel, e disse: Estou eu
no lugar de Deus, que te impediu o fruto de teu ventre? E ela
disse: Eis aqui minha serva Bila; coabita com ela, para que dê à luz
sobre meus joelhos, e eu assim receba filhos por ela. Assim lhe
deu a Bila, sua serva, por mulher; e Jacó a possuiu. E concebeu
Bila, e deu a Jacó um filho. Então disse Raquel: Julgou-me Deus,
e também ouviu a minha voz, e me deu um filho; por isso chamou-lhe
\textbf{Dã}. E Bila, serva de Raquel, concebeu outra vez, e deu
a Jacó o segundo filho. Então disse Raquel: Com grandes lutas
tenho lutado com minha irmã; também venci; e chamou-lhe
\textbf{Naftali}. Vendo, pois, Lia que cessava de ter filhos,
tomou também a \textbf{Zilpa}, sua serva, e deu-a a Jacó por mulher.
E deu Zilpa, serva de Lia, um filho a Jacó. Então
disse Lia: Afortunada! e chamou-lhe \textbf{Gade}. Depois deu
Zilpa, serva de Lia, um segundo filho a Jacó. Então disse
Lia: Para minha ventura; porque as filhas me terão por
bem-aventurada; e chamou-lhe \textbf{Aser}.

E foi Rúben nos dias da ceifa do trigo, e achou
mandrágoras\footnote{Design. comum às plantas do gên. Mandragora, da
fam. das solanáceas, que reúne seis spp. que encerram alcalóides,
nativas do Mediterrâneo até o Himalaia [Plantas muito us. em rituais
de magia, pois a forma de suas raízes tuberosas provoca o imaginário
por se assemelhar ao ser humano.] Erva tóxica (Mandragora
officinarum), nativa da Europa, de folhas ovais, flores campanuladas
e purpúreas, e frutos bacáceos, outrora us. em feitiçarias.} no
campo. E trouxe-as a Lia sua mãe. Então disse Raquel a Lia: Ora
dá-me das mandrágoras de teu filho. E ela lhe disse: É já
pouco que hajas tomado o meu marido, tomarás também as mandrágoras
do meu filho? Então disse Raquel: Por isso ele se deitará contigo
esta noite pelas mandrágoras de teu filho. Vindo, pois, Jacó
à tarde do campo, saiu-lhe Lia ao encontro, e disse: A mim
possuirás, esta noite, porque certamente te aluguei com as
mandrágoras do meu filho. E deitou-se com ela aquela noite. E
ouviu Deus a Lia, e concebeu, e deu à luz um quinto filho.
Então disse Lia: Deus me tem dado o meu galardão, pois tenho
dado minha serva ao meu marido. E chamou-lhe \textbf{Issacar}.
E Lia concebeu outra vez, e deu a Jacó um sexto filho.
E disse Lia: Deus me deu uma boa dádiva; desta vez morará o
meu marido comigo, porque lhe tenho dado seis filhos. E chamou-lhe
\textbf{Zebulom}. E depois teve uma filha, e chamou-lhe
\textbf{Diná}. E lembrou-se Deus de Raquel; e Deus a ouviu, e
abriu a sua madre. E ela concebeu, e deu à luz um filho, e
disse: Tirou-me Deus a minha vergonha. E chamou-lhe
\textbf{José}, dizendo: O Senhor me acrescente outro filho.

E aconteceu que, como Raquel deu à luz a José, disse Jacó a
Labão: Deixa-me ir, que me vá ao meu lugar, e à minha terra.
Dá-me as minhas mulheres, e os meus filhos, pelas quais te
tenho servido, e ir-me-ei; pois tu sabes o serviço que te tenho
feito. Então lhe disse Labão: Se agora tenho achado graça em
teus olhos, fica comigo. Tenho experimentado que o Senhor me
abençoou por amor de ti. E disse mais: Determina-me o teu
salário, que to darei. Então lhe disse: Tu sabes como te
tenho servido, e como passou o teu gado comigo. Porque o
pouco que tinhas antes de mim tem aumentado em grande número; e o
Senhor te tem abençoado por meu trabalho. Agora, pois, quando hei de
trabalhar também por minha casa? E disse ele: Que te darei?
Então disse Jacó: Nada me darás. Se me fizeres isto, tornarei a
apascentar e a guardar o teu rebanho; passarei hoje por todo
o teu rebanho, separando dele todos os salpicados e malhados, e
todos os morenos entre os cordeiros, e os malhados e salpicados
entre as cabras; e isto será o meu salário. Assim testificará
por mim a minha justiça no dia de amanhã, quando vieres e o meu
salário estiver diante de tua face; tudo o que não for salpicado e
malhado entre as cabras e moreno entre os cordeiros, ser-me-á por
furto. Então disse Labão: Quem dera seja conforme a tua
palavra. E separou naquele mesmo dia os bodes listrados e
malhados e todas as cabras salpicadas e malhadas, todos em que havia
brancura, e todos os morenos entre os cordeiros; e deu-os nas mãos
dos seus filhos. E pôs três dias de caminho entre si e Jacó;
e Jacó apascentava o restante dos rebanhos de Labão.

Então tomou Jacó varas verdes de álamo e de aveleira e de
castanheiro, e descascou nelas riscas brancas, descobrindo a
brancura que nas varas havia, e pôs estas varas, que tinha
descascado, em frente aos rebanhos, nos canos e nos bebedouros de
água, aonde os rebanhos vinham beber, para que concebessem quando
vinham beber. E concebiam os rebanhos diante das varas, e as
ovelhas davam crias listradas, salpicadas e malhadas. Então
separou Jacó os cordeiros, e pôs as faces do rebanho para os
listrados, e todo o moreno entre o rebanho de Labão; e pôs o seu
rebanho à parte, e não o pôs com o rebanho de Labão. E
sucedia que cada vez que concebiam as ovelhas fortes, punha Jacó as
varas nos canos, diante dos olhos do rebanho, para que concebessem
diante das varas. Mas, quando era fraco o rebanho, não as
punha. Assim as fracas eram de Labão, e as fortes de Jacó. E
cresceu o homem em grande maneira, e teve muitos rebanhos, e servas,
e servos, e camelos e jumentos.

\smallskip

\lettrine{31} Então ouvia as palavras dos filhos de Labão, que
diziam: Jacó tem tomado tudo o que era de nosso pai, e do que era de
nosso pai fez ele toda esta glória. Viu também Jacó o rosto de
Labão, e eis que não era para com ele como anteriormente. E
disse o Senhor a Jacó: \textbf{Torna-te à terra dos teus pais, e à
tua parentela, e eu serei contigo}. Então mandou Jacó chamar a
Raquel e a Lia ao campo, para junto do seu rebanho, e
disse-lhes: Vejo que o rosto de vosso pai não é para comigo como
anteriormente; porém o Deus de meu pai tem estado comigo; e vós
mesmas sabeis que com todo o meu esforço tenho servido a vosso pai;
mas vosso pai me enganou e mudou o salário dez vezes; porém Deus
não lhe permitiu que me fizesse mal. Quando ele dizia assim: Os
salpicados serão o teu salário; então todos os rebanhos davam
salpicados. E quando ele dizia assim: Os listrados serão o teu
salário, então todos os rebanhos davam listrados. Assim Deus
tirou o gado de vosso pai, e deu-o a mim. E sucedeu que, ao
tempo em que o rebanho concebia, eu levantei os meus olhos e vi em
sonhos, e eis que os bodes, que cobriam as ovelhas, eram listrados,
salpicados e malhados. E disse-me o anjo de Deus em sonhos:
Jacó! E eu disse: Eis-me aqui. E disse ele: Levanta agora os
teus olhos e vê todos os bodes que cobrem o rebanho, que são
listrados, salpicados e malhados; porque tenho visto tudo o que
Labão te fez. Eu sou o Deus de Betel, onde tens ungido uma
coluna, onde me fizeste um voto; levanta-te agora, sai-te desta
terra e torna-te à terra da tua parentela. Então responderam
Raquel e Lia e disseram-lhe: Há ainda para nós parte ou herança na
casa de nosso pai? Não nos considera ele como estranhas? Pois
vendeu-nos, e comeu de todo o nosso dinheiro. Porque toda a
riqueza, que Deus tirou de nosso pai, é nossa e de nossos filhos;
agora, pois, faze tudo o que Deus te mandou.

Então se levantou Jacó, pondo os seus filhos e as suas mulheres
sobre os camelos; e levou todo o seu gado, e todos os seus
bens, que havia adquirido, o gado que possuía, que alcançara em
Padã-Arã, para ir a Isaque, seu pai, à terra de Canaã. E
havendo Labão ido a tosquiar as suas ovelhas, furtou Raquel os
ídolos que seu pai tinha. E Jacó logrou a Labão, o arameu,
porque não lhe fez saber que fugia. E fugiu ele com tudo o
que tinha, e levantou-se e passou o rio; e se dirigiu para a
montanha de Gileade. E no terceiro dia foi anunciado a Labão
que Jacó tinha fugido. Então tomou consigo os seus irmãos, e
atrás dele seguiu o seu caminho por sete dias; e alcançou-o na
montanha de Gileade. Veio, porém, Deus a Labão, o arameu, em
sonhos, de noite, e disse-lhe: Guarda-te, que não fales com Jacó nem
bem nem mal.

Alcançou, pois, Labão a Jacó, e armara Jacó a sua tenda naquela
montanha; armou também Labão com os seus irmãos a sua, na montanha
de Gileade. Então disse Labão a Jacó: Que fizeste, que me
lograste e levaste as minhas filhas como cativas pela espada?
Por que fugiste ocultamente, e lograste-me, e não me fizeste
saber, para que eu te enviasse com alegria, e com cânticos, e com
tamboril e com harpa? Também não me permitiste beijar os meus
filhos e as minhas filhas. Loucamente agiste, agora, fazendo assim.
Poder havia em minha mão para vos fazer mal, mas o Deus de
vosso pai me falou ontem à noite, dizendo: Guarda-te, que não fales
com Jacó nem bem nem mal. E agora se querias ir embora,
porquanto tinhas saudades de voltar à casa de teu pai, por que
furtaste os meus deuses? Então respondeu Jacó, e disse a
Labão: Porque temia; pois que dizia comigo, se porventura não me
arrebatarias as tuas filhas. Com quem achares os teus deuses,
esse não viva; reconhece diante de nossos irmãos o que é teu do que
está comigo, e toma-o para ti. Pois Jacó não sabia que Raquel os
tinha furtado. Então entrou Labão na tenda de Jacó, e na
tenda de Lia, e na tenda de ambas as servas, e não os achou; e
saindo da tenda de Lia, entrou na tenda de Raquel. Mas tinha
tomado Raquel os ídolos e os tinha posto na albarda de um camelo, e
assentara-se sobre eles; e apalpou Labão toda a tenda, e não os
achou. E ela disse a seu pai: Não se acenda a ira aos olhos
de meu senhor, que não posso levantar-me diante da tua face;
porquanto tenho o costume das mulheres. E ele procurou, mas não
achou os ídolos.

Então irou-se Jacó e contendeu com Labão; e respondeu Jacó, e
disse a Labão: Qual é a minha transgressão? Qual é o meu pecado, que
tão furiosamente me tens perseguido? Havendo apalpado todos
os meus móveis, que achaste de todos os móveis de tua casa? Põe-no
aqui diante dos meus irmãos e de teus irmãos; e que julguem entre
nós ambos. Estes vinte anos eu estive contigo; as tuas
ovelhas e as tuas cabras nunca abortaram, e não comi os carneiros do
teu rebanho. Não te trouxe eu o despedaçado; eu o pagava; o
furtado de dia e o furtado de noite da minha mão o requerias.
Estava eu assim: De dia me consumia o calor, e de noite a
geada; e o meu sono fugiu dos meus olhos. Tenho estado agora
vinte anos na tua casa; catorze anos te servi por tuas duas filhas,
e seis anos por teu rebanho; mas o meu salário tens mudado dez
vezes. Se o Deus de meu pai, o Deus de Abraão e o temor de
Isaque não fora comigo, por certo me despedirias agora vazio. Deus
atendeu à minha aflição, e ao trabalho das minhas mãos, e
repreendeu-te ontem à noite.

Então respondeu Labão, e disse a Jacó: Estas filhas são minhas
filhas, e estes filhos são meus filhos, e este rebanho é o meu
rebanho, e tudo o que vês, é meu; e que farei hoje a estas minhas
filhas, ou a seus filhos, que deram à luz? Agora pois vem, e
façamos aliança eu e tu, que seja por testemunho entre mim e ti.
Então tomou Jacó uma pedra, e erigiu-a por coluna. E
disse Jacó a seus irmãos: Ajuntai pedras. E tomaram pedras, e
fizeram um montão, e comeram ali sobre aquele montão. E
chamou-o Labão Jegar-Saaduta; porém Jacó chamou-o Galeede.
Então disse Labão: Este montão seja hoje por testemunha entre
mim e ti. Por isso se lhe chamou Galeede, e Mispá, porquanto
disse: Atente o Senhor entre mim e ti, quando nós estivermos
apartados um do outro. Se afligires as minhas filhas, e se
tomares mulheres além das minhas filhas, ninguém está conosco;
atenta que Deus é testemunha entre mim e ti. Disse mais Labão
a Jacó: Eis aqui este mesmo montão, e eis aqui essa coluna que
levantei entre mim e ti. Este montão seja testemunha, e esta
coluna seja testemunha, que eu não passarei este montão a ti, e que
tu não passarás este montão e esta coluna a mim, para mal. O
Deus de Abraão e o Deus de Naor, o Deus de seu pai, julgue entre
nós. E jurou Jacó pelo temor de seu pai Isaque. E ofereceu
Jacó um sacrifício na montanha, e convidou seus irmãos, para comer
pão; e comeram pão e passaram a noite na montanha. E
levantou-se Labão pela manhã de madrugada, e beijou seus filhos e
suas filhas e abençoou-os e partiu; e voltou Labão ao seu lugar.

\smallskip

\lettrine{32} Jacó também seguiu o seu caminho, e
\textbf{encontraram-no os anjos de Deus}. E Jacó disse, quando
os viu: Este é o exército de Deus. E chamou aquele lugar Maanaim.

E enviou Jacó mensageiros adiante de si a Esaú, seu irmão, à terra
de Seir, território de Edom. E ordenou-lhes, dizendo: Assim
direis a meu senhor Esaú: Assim diz Jacó, teu servo: Como peregrino
morei com Labão, e me detive lá até agora; e tenho bois e
jumentos, ovelhas, e servos e servas; e enviei para o anunciar a meu
senhor, para que ache graça em teus olhos. E os mensageiros
voltaram a Jacó, dizendo: Fomos a teu irmão Esaú; e também ele vem
para encontrar-te, e quatrocentos homens com ele. Então Jacó
temeu muito e angustiou-se; e repartiu o povo que com ele estava, e
as ovelhas, e as vacas, e os camelos, em dois bandos. Porque
dizia: Se Esaú vier a um bando e o ferir, o outro bando escapará.

Disse mais Jacó: Deus de meu pai Abraão, e Deus de meu pai Isaque,
o Senhor, que me disseste: Torna-te à tua terra, e a tua parentela,
e far-te-ei bem; menor sou eu que todas as beneficências, e
que toda a fidelidade que fizeste ao teu servo; porque com meu
cajado passei este Jordão, e agora me tornei em dois bandos.
Livra-me, peço-te, da mão de meu irmão, da mão de Esaú;
porque eu o temo; porventura não venha, e me fira, e a mãe com os
filhos. E tu o disseste: Certamente te farei bem, e farei a
tua descendência como a areia do mar, que pela multidão não se pode
contar.

E passou ali aquela noite; e tomou do que lhe veio à sua mão, um
presente para seu irmão Esaú: duzentas cabras e vinte bodes;
duzentas ovelhas e vinte carneiros; trinta camelas de leite
com suas crias, quarenta vacas e dez novilhos; vinte jumentas e dez
jumentinhos; e deu-os na mão dos seus servos, cada rebanho à
parte, e disse a seus servos: Passai adiante de mim e ponde espaço
entre rebanho e rebanho. E ordenou ao primeiro, dizendo:
Quando Esaú, meu irmão, te encontrar, e te perguntar, dizendo: De
quem és, e para onde vais, e de quem são estes diante de ti?
Então dirás: São de teu servo Jacó, presente que envia a meu
senhor, a Esaú; e eis que ele mesmo vem também atrás de nós.
E ordenou também ao segundo, e ao terceiro, e a todos os que
vinham atrás dos rebanhos, dizendo: Conforme a esta mesma palavra
falareis a Esaú, quando o achardes. E direis também: Eis que
o teu servo Jacó vem atrás de nós. Porque dizia: Eu o aplacarei com
o presente, que vai adiante de mim, e depois verei a sua face;
porventura ele me aceitará. Assim, passou o presente adiante
dele; ele, porém, passou aquela noite no arraial. E
levantou-se aquela mesma noite, e tomou as suas duas mulheres, e as
suas duas servas, e os seus onze filhos, e passou o vau de Jaboque.
E tomou-os e fê-los passar o ribeiro; e fez passar tudo o que
tinha.

Jacó, porém, ficou só; e lutou com ele um homem, até que a alva
subiu. E vendo este que não prevalecia contra ele, tocou a
juntura de sua coxa, e se deslocou a juntura da coxa de Jacó,
lutando com ele. E disse: Deixa-me ir, porque já a alva
subiu. Porém ele disse: Não te deixarei ir, se não me abençoares.
E disse-lhe: Qual é o teu nome? E ele disse: Jacó.
Então disse: \textbf{Não te chamarás mais Jacó, mas Israel;
pois como príncipe lutaste com Deus e com os homens, e
prevaleceste}. E Jacó lhe perguntou, e disse: Dá-me, peço-te,
a saber o teu nome. E disse: Por que perguntas pelo meu nome? E
abençoou-o ali. E chamou Jacó o nome daquele lugar
\textbf{Peniel}, porque dizia: \textbf{Tenho visto a Deus face a
face, e a minha alma foi salva}. E saiu-lhe o sol, quando
passou a Peniel; e manquejava da sua coxa. Por isso os filhos
de Israel não comem o nervo encolhido, que está sobre a juntura da
coxa, até o dia de hoje; porquanto tocara a juntura da coxa de Jacó
no nervo encolhido.

\smallskip

\lettrine{33} E levantou Jacó os seus olhos, e olhou, e eis
que vinha Esaú, e quatrocentos homens com ele. Então repartiu os
filhos entre Lia, e Raquel, e as duas servas. E pôs as servas e
seus filhos na frente, e a Lia e seus filhos atrás; porém a Raquel e
José os derradeiros. E ele mesmo passou adiante deles e
inclinou-se à terra sete vezes, até que chegou a seu irmão.
Então Esaú correu-lhe ao encontro, e abraçou-o, e lançou-se
sobre o seu pescoço, e beijou-o; e choraram.

Depois levantou os seus olhos, e viu as mulheres, e os meninos, e
disse: Quem são estes contigo? E ele disse: Os filhos que Deus
graciosamente tem dado a teu servo. Então chegaram as servas;
elas e os seus filhos, e inclinaram-se. E chegou também Lia com
seus filhos, e inclinaram-se; e depois chegou José e Raquel e
inclinaram-se. E disse Esaú: De que te serve todo este bando que
tenho encontrado? E ele disse: Para achar graça aos olhos de meu
senhor. Mas Esaú disse: Eu tenho bastante, meu irmão; seja para
ti o que tens. Então disse Jacó: Não, se agora tenho achado
graça em teus olhos, peço-te que tomes o meu presente da minha mão;
porquanto tenho visto o teu rosto, como se tivesse visto o rosto de
Deus, e tomaste contentamento em mim. Toma, peço-te, a minha
bênção, que te foi trazida; porque Deus graciosamente ma tem dado; e
porque tenho de tudo. E instou com ele, até que a tomou. E
disse: Caminhemos, e andemos, e eu partirei adiante de ti.
Porém ele lhe disse: Meu senhor sabe que estes filhos são
tenros, e que tenho comigo ovelhas e vacas de leite; se as
afadigarem somente um dia, todo o rebanho morrerá. Ora passe
o meu senhor adiante de seu servo; e eu irei como guia pouco a
pouco, conforme ao passo do gado que vai adiante de mim, e conforme
ao passo dos meninos, até que chegue a meu senhor em Seir. E
Esaú disse: Permite então que eu deixe contigo alguns da minha
gente. E ele disse: Para que é isso? Basta que ache graça aos olhos
de meu senhor.

Assim voltou Esaú aquele dia pelo seu caminho a Seir.
Jacó, porém, partiu para Sucote e edificou para si uma casa;
e fez cabanas para o seu gado; por isso chamou aquele lugar Sucote.
E chegou Jacó salvo à Salém, cidade de Siquém, que está na
terra de Canaã, quando vinha de Padã-Arã; e armou a sua tenda diante
da cidade. E comprou uma parte do campo em que estendera a
sua tenda, da mão dos filhos de Hamor, pai de Siquém, por cem peças
de dinheiro\footnote{RA: cem peças de dinheiro; Edição
Contemporânea: ``cem peças de prata''. King James: ``pieces of
money''.}. E levantou ali um altar, e chamou-lhe: Deus, o
Deus de Israel.


\smallskip

\lettrine{34} E saiu Diná, filha de Lia, que esta dera a Jacó,
para ver as filhas da terra. E Siquém, filho de Hamor, heveu,
príncipe daquela terra, viu-a, e tomou-a, e deitou-se com ela, e
humilhou-a. E apegou-se a sua alma com Diná, filha de Jacó, e
amou a moça e falou afetuosamente à moça. Falou também Siquém a
Hamor, seu pai, dizendo: Toma-me esta moça por mulher. Quando
Jacó ouviu que Diná, sua filha, fora violada, estavam os seus filhos
no campo com o gado; e calou-se Jacó até que viessem.

E saiu Hamor, pai de Siquém, a Jacó, para falar com ele. E
vieram os filhos de Jacó do campo, ouvindo isso, e entristeceram-se
os homens, e iraram-se muito, porquanto Siquém cometera uma
insensatez em Israel, deitando-se com a filha de Jacó; o que não se
devia fazer assim. Então falou Hamor com eles, dizendo: A alma
de Siquém, meu filho, está enamorada da vossa filha; dai-lha,
peço-vos, por mulher; e aparentai-vos conosco, dai-nos as vossas
filhas, e tomai as nossas filhas para vós; e habitareis
conosco; e a terra estará diante de vós; habitai e negociai nela, e
tomai possessão nela. E disse Siquém ao pai dela, e aos
irmãos dela: Ache eu graça em vossos olhos, e darei o que me
disserdes; aumentai muito sobre mim o dote e a dádiva e darei
o que me disserdes; dai-me somente a moça por mulher. Então
responderam os filhos de Jacó a Siquém e a Hamor, seu pai,
enganosamente, e falaram, porquanto havia violado a Diná, sua irmã.
E disseram-lhe: Não podemos fazer isso, dar a nossa irmã a um
homem não circuncidado; porque isso seria uma vergonha para nós;
nisso, porém, consentiremos a vós: se fordes como nós; que se
circuncide todo o homem entre vós; então dar-vos-emos as
nossas filhas, e tomaremos nós as vossas filhas, e habitaremos
convosco, e seremos um povo; mas se não nos ouvirdes, e não
vos circuncidardes, tomaremos a nossa filha e ir-nos-emos.

E suas palavras foram boas aos olhos de Hamor, e aos olhos de
Siquém, filho de Hamor. E não tardou o jovem em fazer isto;
porque a filha de Jacó lhe contentava; e ele era o mais honrado de
toda a casa de seu pai. Veio, pois, Hamor e Siquém, seu
filho, à porta da sua cidade, e falaram aos homens da sua cidade,
dizendo: Estes homens são pacíficos conosco; portanto
habitarão nesta terra, e negociarão nela; eis que a terra é larga de
espaço para eles; tomaremos nós as suas filhas por mulheres, e lhes
daremos as nossas filhas. Nisto, porém, consentirão aqueles
homens, em habitar conosco, para que sejamos um povo, se todo o
homem entre nós se circuncidar, como eles são circuncidados.
E seu gado, as suas possessões, e todos os seus animais não
serão nossos? Consintamos somente com eles e habitarão conosco.
E deram ouvidos a Hamor e a Siquém, seu filho, todos os que
saíam da porta da cidade; e foi circuncidado todo o homem, de todos
os que saíam pela porta da sua cidade.

E aconteceu que, ao terceiro dia, quando estavam com a mais
violenta dor, os dois filhos de Jacó, Simeão e Levi, irmãos de Diná,
tomaram cada um a sua espada, e entraram afoitamente na cidade, e
mataram todos os homens. Mataram também ao fio da espada a
Hamor, e a seu filho Siquém; e tomaram a Diná da casa de Siquém, e
saíram. Vieram os filhos de Jacó aos mortos e saquearam a
cidade; porquanto violaram a sua irmã. As suas ovelhas, e as
suas vacas, e os seus jumentos, e o que havia na cidade e no campo,
tomaram. E todos os seus bens, e todos os seus meninos, e as
suas mulheres, levaram presos, e saquearam tudo o que havia em casa.
Então disse Jacó a Simeão e a Levi: Tendes-me turbado,
fazendo-me cheirar mal entre os moradores desta terra, entre os
cananeus e perizeus; tendo eu pouco povo em número, eles
ajuntar-se-ão, e serei destruído, eu e minha casa. E eles
disseram: Devia ele tratar a nossa irmã como a uma prostituta?

\smallskip

\lettrine{35} Depois disse Deus a Jacó: Levanta-te, sobe a
Betel, e habita ali; e faze ali um altar ao Deus que te apareceu,
quando fugiste da face de Esaú teu irmão. Então disse Jacó à sua
família, e a todos os que com ele estavam: Tirai os deuses
estranhos, que há no meio de vós, e purificai-vos, e mudai as vossas
vestes. E levantemo-nos, e subamos a Betel; e ali farei um altar
ao Deus que me respondeu no dia da minha angústia, e que foi comigo
no caminho que tenho andado. Então deram a Jacó todos os deuses
estranhos, que tinham em suas mãos, e as arrecadas\footnote{Houaiss
--- Regionalismo: Portugal. Brinco de ouro em forma de simples
argola ou em rebuscado trabalho de filigrana (mais us. no pl.).
Ed.Contemp.: argolas que lhes pendiam das orelhas. KJ: earrings
which were in their ears.} que estavam em suas orelhas; e Jacó os
escondeu debaixo do carvalho que está junto a Siquém. E
partiram; e o terror de Deus foi sobre as cidades que estavam ao
redor deles, e não seguiram após os filhos de Jacó.

Assim chegou Jacó a Luz, que está na terra de Canaã (esta é
Betel), ele e todo o povo que com ele havia. E edificou ali um
altar, e chamou aquele lugar El-Betel; porquanto Deus ali se lhe
tinha manifestado, quando fugia da face de seu irmão. E
\textbf{morreu Débora}, a ama de Rebeca, e foi sepultada ao pé de
Betel, debaixo do carvalho cujo nome chamou Alom-Bacute. E
apareceu Deus outra vez a Jacó, vindo de Padã-Arã, e abençoou-o.
E disse-lhe Deus: O teu nome é Jacó; não te chamarás mais
Jacó, mas \textbf{Israel será o teu nome}. E chamou-lhe Israel.
Disse-lhe mais Deus: \textbf{Eu sou o Deus Todo-Poderoso};
frutifica e multiplica-te; uma nação, sim, uma multidão de nações
sairá de ti, e reis procederão dos teus lombos; e te darei a
ti a terra que tenho dado a Abraão e a Isaque, e à tua descendência
depois de ti darei a terra. E Deus subiu dele, do lugar onde
falara com ele. E Jacó pôs uma coluna no lugar onde falara
com ele, uma coluna de pedra; e derramou sobre ela uma
libação\footnote{Houaiss: ato que consiste na aspersão de um líquido
em intenção de uma divindade. O líquido espargido. Ato de tomar
bebidas, esp. alcoólicas, por prazer ou para se fazerem brindes. A
bebida tomada por prazer ou para brindar (us. exclusivamente no
pl.).}, e deitou sobre ela azeite. E chamou Jacó aquele
lugar, onde Deus falara com ele, \textbf{Betel}.

E partiram de Betel; e havia ainda um pequeno espaço de terra
para chegar a Efrata, e deu à luz Raquel, e ela teve trabalho em seu
parto. E aconteceu que, tendo ela trabalho em seu parto, lhe
disse a parteira: Não temas, porque também este filho terás.
E aconteceu que, saindo-se-lhe a alma (porque morreu),
chamou-lhe Benoni; mas seu pai chamou-lhe Benjamim. Assim
\textbf{morreu Raquel}, e foi sepultada no caminho de Efrata; que é
Belém. E Jacó pôs uma coluna sobre a sua sepultura; esta é a
coluna da sepultura de Raquel até o dia de hoje.

Então partiu Israel, e estendeu a sua tenda além de Migdal Eder.
E aconteceu que, habitando Israel naquela terra, foi
\textbf{Rúben e deitou-se com Bila}, concubina de seu pai; e Israel
o soube. E \textbf{eram doze os filhos de Jacó}. Os filhos de
Lia: Rúben, o primogênito de Jacó, depois Simeão e Levi, e Judá, e
Issacar e Zebulom; os filhos de Raquel: José e Benjamim;
e os filhos de Bila, serva de Raquel: Dã e Naftali; e
os filhos de Zilpa, serva de Lia: Gade e Aser. Estes são os filhos
de Jacó, que lhe nasceram em Padã-Arã. E Jacó veio a seu pai
Isaque, a Manre, a Quiriate-Arba (que é Hebrom), onde peregrinaram
Abraão e Isaque. E foram os dias de Isaque cento e oitenta
anos. E Isaque expirou, e morreu, e foi recolhido ao seu
povo, velho e farto de dias; e Esaú e Jacó, seus filhos, o
sepultaram.

\smallskip

\lettrine{36} E estas são as gerações de Esaú (que é Edom).
Esaú tomou suas mulheres das filhas de Canaã; a Ada, filha de
Elom, heteu, e a Aolibama, filha de Aná, filho de Zibeão, heveu.
E a Basemate, filha de Ismael, irmã de Nebaiote. E Ada teve
de Esaú a Elifaz; e Basemate teve a Reuel; e Aolibama deu à luz
a Jeús, Jalão e Coré; estes são os filhos de Esaú, que lhe nasceram
na terra de Canaã. E Esaú tomou suas mulheres, e seus filhos, e
suas filhas, e todas as almas de sua casa, e seu gado, e todos os
seus animais, e todos os seus bens, que havia adquirido na terra de
Canaã; e foi para outra terra apartando-se de Jacó, seu irmão;
porque os bens deles eram muitos para habitarem juntos; e a
terra de suas peregrinações não os podia sustentar por causa do seu
gado. Portanto Esaú habitou na montanha de Seir; Esaú é Edom.

Estas, pois, são as gerações de Esaú, pai dos edomeus, na montanha
de Seir. Estes são os nomes dos filhos de Esaú: Elifaz, filho
de Ada, mulher de Esaú; Reuel, filho de Basemate, mulher de Esaú.
E os filhos de Elifaz foram: Temã, Omar, Zefô, Gaetã e
Quenaz. E Timna era concubina de Elifaz, filho de Esaú, e
teve de Elifaz a \textbf{Amaleque}. Estes são os filhos de Ada,
mulher de Esaú. E estes foram os filhos de Reuel: Naate,
Zerá, Samá e Mizá; estes foram os filhos de Basemate, mulher de
Esaú. E estes foram os filhos de Aolibama, mulher de Esaú,
filha de Aná, filho de Zibeão; ela teve de Esaú: Jeús, Jalão e Coré.
Estes são os príncipes dos filhos de Esaú: os filhos de
Elifaz, o primogênito de Esaú, o príncipe Temã, o príncipe Omar, o
príncipe Zefô, o príncipe Quenaz. O príncipe Coré, o príncipe
Gaetã, o príncipe Amaleque; estes são os príncipes de Elifaz na
terra de Edom; estes são os filhos de Ada. E estes são os
filhos de Reuel, filhos de Esaú: o príncipe Naate, o príncipe Zerá,
o príncipe Samá, o príncipe Mizá; estes são os príncipes de Reuel,
na terra de Edom; estes são os filhos de Basemate, mulher de Esaú.
E estes são os filhos de Aolibama, mulher de Esaú: o príncipe
Jeús, o príncipe Jalão, o príncipe Coré; estes são os príncipes de
Aolibama, filha de Aná, mulher de Esaú. Estes são os filhos
de Esaú, e estes são seus príncipes: Ele é Edom.

Estes são os filhos de Seir, horeu, moradores daquela terra:
Lotã, Sobal, Zibeão e Aná, 21 Disom, Eser e Disã; estes são os
príncipes dos horeus, filhos de Seir, na terra de Edom. E os
filhos de Lotã foram Hori e Homã; e a irmã de Lotã era Timna.
Estes são os filhos de Sobal: Alvã, Manaate, Ebal, Sefô e
Onã. E estes são os filhos de Zibeão: Aiá e Aná; este é o Aná
que achou as fontes termais no deserto, quando apascentava os
jumentos de Zibeão, seu pai. E estes são os filhos de Aná:
Disom e Aolibama, a filha de Aná. E estes são os filhos de
Disã: Hendã, Esbã, Itrã e Querã. Estes são os filhos de Eser:
Bilã, Zaavã e Acã. Estes são os filhos de Disã: Uz e Arã.
Estes são os príncipes dos horeus: o príncipe Lotã, o
príncipe Sobal, o príncipe Zibeão, o príncipe Aná. O príncipe
Disom, o príncipe Eser, o príncipe Disã: estes são os príncipes dos
horeus segundo os seus principados na terra de Seir.

E estes são os reis que reinaram na terra de Edom, antes que
reinasse rei algum sobre os filhos de Israel. Reinou, pois,
em Edom, Bela, filho de Beor, e o nome da sua cidade foi Dinabá.
E morreu Bela; e Jobabe, filho de Zerá, de Bozra, reinou em
seu lugar. E morreu Jobabe; e Husão, da terra dos temanitas,
reinou em seu lugar. E morreu Husão, e em seu lugar reinou
Hadade, filho de Bedade, o que feriu a Midiã, no campo de Moabe; e o
nome da sua cidade foi Avite. E morreu Hadade; e Samlá de
Masreca reinou em seu lugar. E morreu Samlá; e Saul de
Reobote, junto ao rio, reinou em seu lugar. E morreu Saul; e
Baal-Hanã, filho de Acbor, reinou em seu lugar. E morreu
Baal-Hanã, filho de Acbor; e Hadar reinou em seu lugar, e o nome de
sua cidade foi Pau; e o nome de sua mulher foi Meetabel, filha de
Matrede, filha de Me-Zaabe. E estes são os nomes dos
príncipes de Esaú, segundo as suas gerações, segundo os seus
lugares, com os seus nomes: o príncipe Timna, o príncipe Alva, o
príncipe Jetete, o príncipe Aolibama, o príncipe Ela, o
príncipe Pinom, o príncipe Quenaz, o príncipe Temã, o
príncipe Mibzar, o príncipe Magdiel, o príncipe Irã: estes
são os príncipes de Edom, segundo as suas habitações, na terra da
sua possessão. Este é Esaú, pai de Edom.

\smallskip

\lettrine{37} E Jacó habitou na terra das peregrinações de seu
pai, na terra de Canaã. Estas são as gerações de Jacó. Sendo
José de dezessete anos, apascentava as ovelhas com seus irmãos;
sendo ainda jovem, andava com os filhos de Bila, e com os filhos de
Zilpa, mulheres de seu pai; e José trazia más notícias deles a seu
pai. E Israel amava a José mais do que a todos os seus filhos,
porque era filho da sua velhice; e fez-lhe uma túnica de várias
cores. Vendo, pois, seus irmãos que seu pai o amava mais do que
a todos eles, odiaram-no, e não podiam falar com ele pacificamente.

Teve José um sonho, que contou a seus irmãos; por isso o odiaram
ainda mais. E disse-lhes: Ouvi, peço-vos, este sonho, que tenho
sonhado: Eis que estávamos atando molhos no meio do campo, e eis
que o meu molho se levantava, e também ficava em pé, e eis que os
vossos molhos o rodeavam, e se inclinavam ao meu molho. Então
lhe disseram seus irmãos: Tu, pois, deveras reinarás sobre nós? Tu
deveras terás domínio sobre nós? Por isso ainda mais o odiavam por
seus sonhos e por suas palavras. E teve José outro sonho, e o
contou a seus irmãos, e disse: Eis que tive ainda outro sonho; e eis
que o sol, e a lua, e onze estrelas se inclinavam a mim. E
contando-o a seu pai e a seus irmãos, repreendeu-o seu pai, e
disse-lhe: Que sonho é este que tiveste? Porventura viremos, eu e
tua mãe, e teus irmãos, a inclinar-nos perante ti em terra?
Seus irmãos, pois, o invejavam; seu pai porém guardava este
negócio no seu coração.

E seus irmãos foram apascentar o rebanho de seu pai, junto de
Siquém. Disse, pois, Israel a José: Não apascentam os teus
irmãos junto de Siquém? Vem, e enviar-te-ei a eles. E ele respondeu:
Eis-me aqui. E ele lhe disse: Ora vai, vê como estão teus
irmãos, e como está o rebanho, e traze-me resposta. Assim o enviou
do vale de Hebrom, e foi a Siquém. E achou-o um homem, porque
eis que andava errante pelo campo, e perguntou-lhe o homem, dizendo:
Que procuras? E ele disse: Procuro meus irmãos; dize-me,
peço-te, onde eles apascentam. E disse aquele homem: Foram-se
daqui; porque ouvi-os dizer: Vamos a Dotã. José, pois, seguiu atrás
de seus irmãos, e achou-os em Dotã. E viram-no de longe e,
antes que chegasse a eles, conspiraram contra ele para o matarem.
E disseram um ao outro: Eis lá vem o sonhador-mor!
Vinde, pois, agora, e matemo-lo, e lancemo-lo numa destas
covas, e diremos: Uma fera o comeu; e veremos que será dos seus
sonhos. E ouvindo-o Rúben, livrou-o das suas mãos, e disse:
Não lhe tiremos a vida. Também lhes disse Rúben: Não
derrameis sangue; lançai-o nesta cova, que está no deserto, e não
lanceis mãos nele; isto disse para livrá-lo das mãos deles e para
torná-lo a seu pai.

E aconteceu que, chegando José a seus irmãos, tiraram de José a
sua túnica, a túnica de várias cores, que trazia. E
tomaram-no, e lançaram-no na cova; porém a cova estava vazia, não
havia água nela. Depois assentaram-se a comer pão; e
levantaram os seus olhos, e olharam, e eis que uma companhia de
ismaelitas vinha de Gileade; e seus camelos traziam especiarias e
bálsamo e mirra, e iam levá-los ao Egito. Então Judá disse
aos seus irmãos: Que proveito haverá que matemos a nosso irmão e
escondamos o seu sangue? Vinde e vendamo-lo a estes
ismaelitas, e não seja nossa mão sobre ele; porque ele é nosso
irmão, nossa carne. E seus irmãos obedeceram. Passando, pois,
os mercadores midianitas, tiraram e alçaram a José da cova, e
venderam José por vinte moedas de prata, aos ismaelitas, os quais
levaram José ao Egito. Voltando, pois, Rúben à cova, eis que
José não estava na cova; então rasgou as suas vestes. E
voltou a seus irmãos e disse: O menino não está; e eu aonde irei?

Então tomaram a túnica de José, e mataram um cabrito, e tingiram
a túnica no sangue. E enviaram a túnica de várias cores,
mandando levá-la a seu pai, e disseram: Temos achado esta túnica;
conhece agora se esta será ou não a túnica de teu filho. E
conheceu-a, e disse: É a túnica de meu filho; uma fera o comeu;
certamente José foi despedaçado. Então Jacó rasgou as suas
vestes, pôs saco sobre os seus lombos e lamentou a seu filho muitos
dias. E levantaram-se todos os seus filhos e todas as suas
filhas, para o consolarem; recusou porém ser consolado, e disse:
Porquanto com choro hei de descer ao meu filho até à sepultura.
Assim o chorou seu pai. E os midianitas venderam-no no Egito
a Potifar, oficial de Faraó, capitão da guarda.

\smallskip

\lettrine{38} E aconteceu no mesmo tempo que Judá desceu de
entre seus irmãos e entrou na casa de um homem de Adulão, cujo nome
era Hira, e viu Judá ali a filha de um homem cananeu, cujo nome
era Sua; e tomou-a por mulher, e a possuiu. E ela concebeu e deu
à luz um filho, e chamou-lhe Er. E tornou a conceber e deu à luz
um filho, e chamou-lhe Onã. E continuou ainda e deu à luz um
filho, e chamou-lhe Selá; e Judá estava em Quezibe, quando ela o deu
à luz. Judá, pois, tomou uma mulher para Er, o seu primogênito,
e o seu nome era \textbf{Tamar}. Er, porém, o primogênito de
Judá, era mau aos olhos do Senhor, por isso o Senhor o matou.
Então disse Judá a Onã: Toma a mulher do teu irmão, e casa-te
com ela, e suscita descendência a teu irmão. Onã, porém, soube
que esta descendência não havia de ser para ele; e aconteceu que,
quando possuía a mulher de seu irmão, derramava o sêmen na terra,
para não dar descendência a seu irmão. E o que fazia era mau
aos olhos do Senhor, pelo que também o matou. Então disse
Judá a Tamar sua nora: Fica-te viúva na casa de teu pai, até que
Selá, meu filho, seja grande. Porquanto disse: Para que porventura
não morra também este, como seus irmãos. Assim se foi Tamar e ficou
na casa de seu pai.

Passando-se pois muitos dias, morreu a filha de Sua, mulher de
Judá; e depois de consolado Judá subiu aos tosquiadores das suas
ovelhas em Timna, ele e Hira, seu amigo, o adulamita. E deram
aviso a Tamar, dizendo: Eis que o teu sogro sobe a Timna, a tosquiar
as suas ovelhas. Então ela tirou de sobre si os vestidos da
sua viuvez e cobriu-se com o véu, e envolveu-se, e assentou-se à
entrada das duas fontes que estão no caminho de Timna, porque via
que Selá já era grande, e ela não lhe fora dada por mulher. E
vendo-a Judá, teve-a por uma prostituta, porque ela tinha coberto o
seu rosto. E dirigiu-se a ela no caminho, e disse: Vem,
peço-te, deixa-me possuir-te. Porquanto não sabia que era sua nora.
E ela disse: Que darás, para que possuas a mim? E ele disse:
Eu te enviarei um cabrito do rebanho. E ela disse: Dar-me-ás penhor
até que o envies? Então ele disse: Que penhor é que te darei?
E ela disse: O teu selo, e o teu cordão, e o cajado que está em tua
mão. O que ele lhe deu, e possuiu-a, e ela concebeu dele. E
ela se levantou, e se foi e tirou de sobre si o seu véu, e vestiu os
vestidos da sua viuvez. E Judá enviou o cabrito por mão do
seu amigo, o adulamita, para tomar o penhor da mão da mulher; porém
não a achou. E perguntou aos homens daquele lugar, dizendo:
Onde está a prostituta que estava no caminho junto às duas fontes? E
disseram: Aqui não esteve prostituta alguma. E tornou-se a
Judá e disse: Não a achei; e também disseram os homens daquele
lugar: Aqui não esteve prostituta. Então disse Judá: Deixa-a
ficar com o penhor, para que porventura não caiamos em desprezo; eis
que tenho enviado este cabrito; mas tu não a achaste.

E aconteceu que, quase três meses depois, deram aviso a Judá,
dizendo: Tamar, tua nora, adulterou, e eis que está grávida do
adultério. Então disse Judá: Tirai-a fora para que seja queimada.
E tirando-a fora, ela mandou dizer a seu sogro: Do homem de
quem são estas coisas eu concebi. E ela disse mais: Conhece,
peço-te, de quem é este selo, e este cordão, e este cajado. E
conheceu-os Judá e disse: Mais justa é ela do que eu, porquanto não
a tenho dado a Selá meu filho. E nunca mais a conheceu. E
aconteceu ao tempo de dar à luz que havia gêmeos em seu ventre;
e sucedeu que, dando ela à luz, que um pôs fora a mão, e a
parteira tomou-a, e atou em sua mão um fio encarnado\footnote{Ed.
Contemp.: fio escarlate; KJ: scarlet thread}, dizendo: Este saiu
primeiro. Mas aconteceu que, tornando ele a recolher a sua
mão, eis que saiu o seu irmão, e ela disse: Como tu tens rompido,
sobre ti é a rotura. E chamaram-lhe \textbf{Perez}. E depois
saiu o seu irmão, em cuja mão estava o fio encarnado; e chamaram-lhe
\textbf{Zerá}.

\smallskip

\lettrine{39} E José foi levado ao Egito, e Potifar, oficial
de Faraó, capitão da guarda, homem egípcio, comprou-o da mão dos
ismaelitas que o tinham levado lá. E o Senhor estava com José, e
foi homem próspero; e estava na casa de seu senhor egípcio.
Vendo, pois, o seu senhor que o Senhor estava com ele, e tudo o
que fazia o Senhor prosperava em sua mão, José achou graça em
seus olhos, e servia-o; e ele o pôs sobre a sua casa, e entregou na
sua mão tudo o que tinha. E aconteceu que, desde que o pusera
sobre a sua casa e sobre tudo o que tinha, o Senhor abençoou a casa
do egípcio por amor de José; e a bênção do Senhor foi sobre tudo o
que tinha, na casa e no campo. E deixou tudo o que tinha na mão
de José, de maneira que nada sabia do que estava com ele, a não ser
do pão que comia. E José era formoso de porte, e de semblante.

E aconteceu depois destas coisas que a mulher do seu senhor pôs os
seus olhos em José, e disse: Deita-te comigo. Porém ele recusou,
e disse à mulher do seu senhor: Eis que o meu senhor não sabe do que
há em casa comigo, e entregou em minha mão tudo o que tem;
ninguém há maior do que eu nesta casa, e nenhuma coisa me vedou,
senão a ti, porquanto tu és sua mulher; como pois faria eu tamanha
maldade, e pecaria contra Deus? E aconteceu que, falando ela
cada dia a José, e não lhe dando ele ouvidos, para deitar-se com
ela, e estar com ela, sucedeu num certo dia que ele veio à
casa para fazer seu serviço; e nenhum dos da casa estava ali;
e ela lhe pegou pela sua roupa, dizendo: Deita-te comigo. E
ele deixou a sua roupa na mão dela, e fugiu, e saiu\footnote{SBTB:
``saiu para fora''. Pleonasmo vicioso, novamente. Ed. Contemp.: e
fugiu, escapando para fora. RA: saiu, fugindo para fora. KJ: and
fled, and got him out.}.

E aconteceu que, vendo ela que deixara a sua roupa em sua mão, e
fugira para fora, chamou aos homens de sua casa, e
falou-lhes, dizendo: Vede, meu marido trouxe-nos um homem hebreu
para escarnecer de nós; veio a mim para deitar-se comigo, e eu
gritei com grande voz; e aconteceu que, ouvindo ele que eu
levantava a minha voz e gritava, deixou a sua roupa comigo, e fugiu,
e saiu\footnote{Idem.}. E ela pôs a sua roupa perto de si,
até que o seu senhor voltou à sua casa. Então falou-lhe
conforme as mesmas palavras, dizendo: Veio a mim o servo hebreu, que
nos trouxeste, para escarnecer de mim; e aconteceu que,
levantando eu a minha voz e gritando, ele deixou a sua roupa comigo,
e fugiu para fora.

E aconteceu que, ouvindo o seu senhor as palavras de sua mulher,
que lhe falava, dizendo: Conforme a estas mesmas palavras me fez teu
servo, a sua ira se acendeu. E o senhor de José o tomou, e o
entregou na casa do cárcere, no lugar onde os presos do rei estavam
encarcerados; assim esteve ali na casa do cárcere. O Senhor,
porém, estava com José, e estendeu sobre ele a sua benignidade, e
deu-lhe graça aos olhos do carcereiro-mor. E o carcereiro-mor
entregou na mão de José todos os presos que estavam na casa do
cárcere, e ele ordenava tudo o que se fazia ali. E o
carcereiro-mor não teve cuidado de nenhuma coisa que estava na mão
dele, porquanto o Senhor estava com ele, e tudo o que fazia o Senhor
prosperava.

\smallskip

\lettrine{40} E aconteceu, depois destas coisas, que o copeiro
do rei do Egito, e o seu padeiro, ofenderam o seu Senhor, o rei do
Egito. E indignou-se Faraó muito contra os seus dois oficiais,
contra o copeiro-mor e contra o padeiro-mor. E entregou-os à
prisão, na casa do capitão da guarda, na casa do cárcere, no lugar
onde José estava preso. E o capitão da guarda pô-los a cargo de
José, para que os servisse; e estiveram muitos dias na prisão.

E ambos tiveram um sonho, cada um seu sonho, na mesma noite, cada
um conforme a interpretação do seu sonho, o copeiro e o padeiro do
rei do Egito, que estavam presos na casa do cárcere. E veio José
a eles pela manhã, e olhou para eles, e viu que estavam perturbados.
Então perguntou aos oficiais de Faraó, que com ele estavam no
cárcere da casa de seu senhor, dizendo: Por que estão hoje tristes
os vossos semblantes? E eles lhe disseram: Tivemos um sonho, e
ninguém há que o interprete. E José disse-lhes: \textbf{Não são de
Deus as interpretações?} Contai-mo, peço-vos. Então contou o
copeiro-mor o seu sonho a José, e disse-lhe: Eis que em meu sonho
havia uma vide diante da minha face. E na vide três
sarmentos\footnote{Rebento novo da vide e de outras plantas ainda
não podadas. Rama de vide, seca para o fogo. Haste comprida das
trepadeiras. Ed. Contem.: ramos.}, e brotando ela, a sua flor saía,
e os seus cachos amadureciam em uvas; e o copo de Faraó
estava na minha mão, e eu tomava as uvas, e as espremia no copo de
Faraó, e dava o copo na mão de Faraó. Então disse-lhe José:
Esta é a sua interpretação: Os três sarmentos são três dias;
dentro ainda de três dias Faraó levantará a tua cabeça, e te
restaurará ao teu estado, e darás o copo de Faraó na sua mão,
conforme o costume antigo, quando eras seu copeiro. Porém
lembra-te de mim, quando te for bem; e rogo-te que uses comigo de
compaixão, e que faças menção de mim a Faraó, e faze-me sair desta
casa; porque, de fato, fui roubado da terra dos hebreus; e
tampouco aqui nada tenho feito para que me pusessem nesta cova.
Vendo então o padeiro-mor que tinha interpretado bem, disse a
José: Eu também sonhei, e eis que três cestos brancos estavam sobre
a minha cabeça; e no cesto mais alto havia de todos os
manjares de Faraó, obra de padeiro; e as aves o comiam do cesto, de
sobre a minha cabeça. Então respondeu José, e disse: Esta é a
sua interpretação: Os três cestos são três dias; dentro ainda
de três dias Faraó tirará a tua cabeça e te pendurará num pau, e as
aves comerão a tua carne de sobre ti.

E aconteceu ao terceiro dia, o dia do nascimento de Faraó, que
fez um banquete a todos os seus servos; e levantou a cabeça do
copeiro-mor, e a cabeça do padeiro-mor, no meio dos seus servos.
E fez tornar o copeiro-mor ao seu ofício de copeiro, e este
deu o copo na mão de Faraó, mas ao padeiro-mor enforcou, como
José havia interpretado. O copeiro-mor, porém, não se lembrou
de José, antes se esqueceu dele.

\smallskip

\lettrine{41} E aconteceu que, ao fim de dois anos inteiros,
Faraó sonhou, e eis que estava em pé junto ao rio. E eis que
subiam do rio sete vacas, formosas à vista e gordas de carne, e
pastavam no prado. E eis que subiam do rio após elas outras sete
vacas, feias à vista e magras de carne; e paravam junto às outras
vacas na praia do rio. E as vacas feias à vista e magras de
carne, comiam as sete vacas formosas à vista e gordas. Então acordou
Faraó. Depois dormiu e sonhou outra vez, e eis que brotavam de
um mesmo pé sete espigas cheias e boas. E eis que sete espigas
miúdas, e queimadas do vento oriental, brotavam após elas. E as
espigas miúdas devoravam as sete espigas grandes e cheias. Então
acordou Faraó, e eis que era um sonho. E aconteceu que pela
manhã o seu espírito perturbou-se, e enviou e chamou todos os
adivinhadores do Egito, e todos os seus sábios; e Faraó contou-lhes
os seus sonhos, mas ninguém havia que lhos interpretasse.

Então falou o copeiro-mor a Faraó, dizendo: Das minhas ofensas me
lembro hoje: Estando Faraó muito indignado contra os seus
servos, e pondo-me sob prisão na casa do capitão da guarda, a mim e
ao padeiro-mor, então tivemos um sonho na mesma noite, eu e
ele; sonhamos, cada um conforme a interpretação do seu sonho.
E estava ali conosco um jovem hebreu, servo do capitão da
guarda, e contamos-lhe os nossos sonhos e ele no-los interpretou, a
cada um conforme o seu sonho. E como ele nos interpretou,
assim aconteceu; a mim me foi restituído o meu cargo, e ele foi
enforcado. Então mandou Faraó chamar a José, e o fizeram sair
logo do cárcere; e barbeou-se e mudou as suas roupas e apresentou-se
a Faraó. E Faraó disse a José: Eu tive um sonho, e ninguém há
que o interprete; mas de ti ouvi dizer que quando ouves um sonho o
interpretas. E respondeu José a Faraó, dizendo: Isso não está
em mim; Deus dará resposta de paz a Faraó.

Então disse Faraó a José: Eis que em meu sonho estava eu em pé na
margem do rio, 18 e eis que subiam do rio sete vacas gordas de carne
e formosas à vista, e pastavam no prado. E eis que outras
sete vacas subiam após estas, muito feias à vista e magras de carne;
não tenho visto outras tais, quanto à fealdade\footnote{Qualidade de
feio.}, em toda a terra do Egito. E as vacas magras e feias
comiam as primeiras sete vacas gordas; e entravam em suas
entranhas, mas não se conhecia que houvessem entrado; porque o seu
parecer era feio como no princípio. Então acordei. Depois vi
em meu sonho, e eis que de um mesmo pé subiam sete espigas cheias e
boas; 23 e eis que sete espigas secas, miúdas e queimadas do vento
oriental, brotavam após elas. E as sete espigas miúdas
devoravam as sete espigas boas. E eu contei isso aos magos, mas
ninguém houve que mo interpretasse. Então disse José a Faraó:
O sonho de Faraó é um só; o que Deus há de fazer, mostrou-o a Faraó.
As sete vacas formosas são sete anos, as sete espigas
formosas também são sete anos, o sonho é um só. E as sete
vacas feias à vista e magras, que subiam depois delas, são sete
anos, e as sete espigas miúdas e queimadas do vento oriental, serão
sete anos de fome. Esta é a palavra que tenho dito a Faraó; o
que Deus há de fazer, mostrou-o a Faraó. E eis que vêm sete
anos, e haverá grande fartura em toda a terra do Egito. E
depois deles levantar-se-ão sete anos de fome, e toda aquela fartura
será esquecida na terra do Egito, e a fome consumirá a terra;
e não será conhecida a abundância na terra, por causa daquela
fome que haverá depois; porquanto será gravíssima. E que o
sonho foi repetido duas vezes a Faraó, é porque esta coisa é
determinada por Deus, e Deus se apressa em fazê-la.

Portanto, Faraó previna-se agora de um homem entendido e sábio, e
o ponha sobre a terra do Egito. Faça isso Faraó e ponha
governadores sobre a terra, e tome a quinta parte da terra do Egito
nos sete anos de fartura, e ajuntem toda a comida destes bons
anos, que vêm, e amontoem o trigo debaixo da mão de Faraó, para
mantimento nas cidades, e o guardem. Assim será o mantimento
para provimento da terra, para os sete anos de fome, que haverá na
terra do Egito; para que a terra não pereça de fome. E esta
palavra foi boa aos olhos de Faraó, e aos olhos de todos os seus
servos. E disse Faraó a seus servos: Acharíamos um homem como
este em quem haja o espírito de Deus? Depois disse Faraó a
José: Pois que Deus te fez saber tudo isto, ninguém há tão entendido
e sábio como tu. Tu estarás sobre a minha casa, e por tua
boca se governará todo o meu povo, somente no trono eu serei maior
que tu. Disse mais Faraó a José: Vês aqui te tenho posto
sobre toda a terra do Egito. E tirou Faraó o anel da sua mão,
e o pôs na mão de José, e o fez vestir de roupas de linho fino, e
pôs um colar de ouro no seu pescoço. E o fez subir no segundo
carro que tinha, e clamavam diante dele: Ajoelhai. Assim o pôs sobre
toda a terra do Egito. E disse Faraó a José: Eu sou Faraó;
porém sem ti ninguém levantará a sua mão ou o seu pé em toda a terra
do Egito. E Faraó chamou a José de \textbf{Zafenate-Panéia},
e deu-lhe por mulher a \textbf{Azenate}, filha de Potífera,
sacerdote de Om; e saiu José por toda a terra do Egito.

E \textbf{José era da idade de trinta anos quando se apresentou a
Faraó, rei do Egito}. E saiu José da presença de Faraó e passou por
toda a terra do Egito. E nos sete anos de fartura a terra
produziu abundantemente. E ele ajuntou todo o mantimento dos
sete anos, que houve na terra do Egito; e guardou o mantimento nas
cidades, pondo nas mesmas o mantimento do campo que estava ao redor
de cada cidade. Assim ajuntou José muitíssimo trigo, como a
areia do mar, até que cessou de contar; porquanto não havia
numeração. E nasceram a José dois filhos (antes que viesse um
ano de fome), que lhe deu Azenate, filha de Potífera, sacerdote de
Om. E chamou José ao primogênito \textbf{Manassés}, porque
disse: Deus me fez esquecer de todo o meu trabalho, e de toda a casa
de meu pai. E ao segundo chamou \textbf{Efraim}; porque
disse: Deus me fez crescer na terra da minha aflição. Então
acabaram-se os sete anos de fartura que havia na terra do Egito.
E começaram a vir os sete anos de fome, como José tinha dito;
e havia fome em todas as terras, mas em toda a terra do Egito havia
pão. E tendo toda a terra do Egito fome, clamou o povo a
Faraó por pão; e Faraó disse a todos os egípcios: Ide a José; o que
ele vos disser, fazei. Havendo, pois, fome sobre toda a
terra, abriu José tudo em que havia mantimento, e vendeu aos
egípcios; porque a fome prevaleceu na terra do Egito. E de
todas as terras vinham ao Egito, para comprar de José; porquanto a
fome prevaleceu em todas as terras.

\smallskip

\lettrine{42} Vendo então Jacó que havia mantimento no Egito,
disse a seus filhos: Por que estais olhando uns para os outros?
Disse mais: Eis que tenho ouvido que há mantimentos no Egito;
descei para lá, e comprai-nos dali, para que vivamos e não morramos.
Então desceram os dez irmãos de José, para comprarem trigo no
Egito. A Benjamim, porém, irmão de José, não enviou Jacó com os
seus irmãos, porque dizia: Para que lhe não suceda, porventura,
algum desastre. Assim, entre os que iam lá foram os filhos de
Israel para comprar, porque havia fome na terra de Canaã. José,
pois, era o governador daquela terra; ele vendia a todo o povo da
terra; e os irmãos de José chegaram e inclinaram-se a ele, com o
rosto em terra.

E José, vendo os seus irmãos, conheceu-os; porém mostrou-se
estranho para com eles, e falou-lhes asperamente, e disse-lhes: De
onde vindes? E eles disseram: Da terra de Canaã, para comprarmos
mantimento. José, pois, conheceu os seus irmãos; mas eles não o
conheceram. Então José lembrou-se dos sonhos que havia tido
deles e disse-lhes: Vós sois espias, e viestes para ver a nudez da
terra. E eles lhe disseram: Não, senhor meu; mas teus servos
vieram comprar mantimento. Todos nós somos filhos de um mesmo
homem; somos homens de retidão; os teus servos não são espias.
E ele lhes disse: Não; antes viestes para ver a nudez da
terra. E eles disseram: Nós, teus servos, somos doze irmãos,
filhos de um homem na terra de Canaã; e eis que o mais novo está com
nosso pai hoje; mas um já não existe. Então lhes disse José:
Isso é o que vos tenho dito, sois espias; nisto sereis
provados; pela vida de Faraó, não saireis daqui senão quando vosso
irmão mais novo vier aqui. Enviai um dentre vós, que traga
vosso irmão, mas vós ficareis presos, e vossas palavras sejam
provadas, se há verdade convosco; e se não, pela vida de Faraó, vós
sois espias. E pô-los juntos, em prisão, três dias. E
ao terceiro dia disse-lhes José: Fazei isso, e vivereis; porque eu
temo a Deus. Se sois homens de retidão, que fique um de
vossos irmãos preso na casa de vossa prisão; e vós ide, levai
mantimento para a fome de vossa casa, e trazei-me o vosso
irmão mais novo, e serão verificadas vossas palavras, e não
morrereis. E eles assim fizeram.

Então disseram uns aos outros: Na verdade, somos culpados acerca
de nosso irmão, pois vimos a angústia da sua alma, quando nos
rogava; nós porém não ouvimos, por isso vem sobre nós esta angústia.
E Rúben respondeu-lhes, dizendo: Não vo-lo dizia eu: Não
pequeis contra o menino; mas não ouvistes; e vedes aqui, o seu
sangue também é requerido. E eles não sabiam que José os
entendia, porque havia intérprete entre eles. E retirou-se
deles e chorou. Depois tornou a eles, e falou-lhes, e tomou a Simeão
dentre eles, e amarrou-o perante os seus olhos. E ordenou
José, que enchessem os seus sacos de trigo, e que lhes restituíssem
o seu dinheiro a cada um no seu saco, e lhes dessem comida para o
caminho; e fizeram-lhes assim. E carregaram o seu trigo sobre
os seus jumentos e partiram dali. E, abrindo um deles o seu
saco, para dar pasto ao seu jumento na estalagem, viu o seu
dinheiro; porque eis que estava na boca do seu saco. E disse
a seus irmãos: Devolveram o meu dinheiro, e ei-lo também aqui no
saco. Então lhes desfaleceu o coração, e pasmavam, dizendo um ao
outro: Que é isto que Deus nos tem feito?

E vieram para Jacó, seu pai, na terra de Canaã; e contaram-lhe
tudo o que lhes aconteceu, dizendo: O homem, o senhor da
terra, falou conosco asperamente, e tratou-nos como espias da terra;
mas dissemos-lhe: Somos homens de retidão; não somos espias;
somos doze irmãos, filhos de nosso pai; um não mais existe, e
o mais novo está hoje com nosso pai na terra de Canaã. E
aquele homem, o senhor da terra, nos disse: Nisto conhecerei que vós
sois homens de retidão; deixai comigo um de vossos irmãos, e tomai
para a fome de vossas casas, e parti, e trazei-me vosso irmão
mais novo; assim saberei que não sois espias, mas homens de retidão;
então vos darei o vosso irmão e negociareis na terra. E
aconteceu que, despejando eles os seus sacos, eis que cada um tinha
o pacote com seu dinheiro no seu saco; e viram os pacotes com seu
dinheiro, eles e seu pai, e temeram. Então Jacó, seu pai,
disse-lhes: Tendes-me desfilhado; José já não existe e Simeão não
está aqui; agora levareis a Benjamim. Todas estas coisas vieram
sobre mim. Mas Rúben falou a seu pai, dizendo: Mata os meus
dois filhos, se eu não tornar a trazê-lo para ti; entrega-o em minha
mão, e tornarei a trazê-lo. Ele porém disse: Não descerá meu
filho convosco; porquanto o seu irmão é morto, e só ele ficou. Se
lhe suceder algum desastre no caminho por onde fordes, fareis descer
minhas cãs com tristeza à sepultura.

\smallskip

\lettrine{43} E a fome era gravíssima na terra. E
aconteceu que, como acabaram de comer o mantimento que trouxeram do
Egito, disse-lhes seu pai: Voltai, comprai-nos um pouco de alimento.
Mas Judá respondeu-lhe, dizendo: Fortemente nos protestou aquele
homem, dizendo: Não vereis a minha face, se o vosso irmão não vier
convosco. Se enviares conosco o nosso irmão, desceremos e te
compraremos alimento; mas se não o enviares, não desceremos;
porquanto aquele homem nos disse: Não vereis a minha face, se o
vosso irmão não vier convosco. E disse Israel: Por que me
fizeste tal mal, fazendo saber àquele homem que tínheis ainda outro
irmão? E eles disseram: Aquele homem particularmente nos
perguntou por nós, e pela nossa parentela, dizendo: Vive ainda vosso
pai? Tendes mais um irmão? E respondemos-lhe conforme as mesmas
palavras. Podíamos nós saber que diria: Trazei vosso irmão?
Então disse Judá a Israel, seu pai: Envia o jovem comigo, e
levantar-nos-emos, e iremos, para que vivamos e não morramos, nem
nós, nem tu, nem os nossos filhos. Eu serei fiador por ele, da
minha mão o requererás; se eu não o trouxer, e não o puser perante a
tua face, serei réu de crime para contigo para sempre. E se
não nos tivéssemos detido, certamente já estaríamos segunda vez de
volta.

Então disse-lhes Israel, seu pai: Pois que assim é, fazei isso;
tomai do mais precioso desta terra em vossos vasos, e levai ao homem
um presente: um pouco do bálsamo e um pouco de mel, especiarias e
mirra, terebinto\footnote{O mesmo que \emph{almecegueira}. Almécega:
resina de aroeira goma-do-brasil}. e amêndoas; e tomai em
vossas mãos dinheiro em dobro, e o dinheiro que voltou na boca dos
vossos sacos tornai a levar em vossas mãos; bem pode ser que fosse
erro. Tomai também a vosso irmão, e levantai-vos e voltai
àquele homem; e Deus Todo-Poderoso vos dê misericórdia diante
do homem, para que deixe vir convosco vosso outro irmão, e Benjamim;
e eu, se for desfilhado, desfilhado ficarei.

E os homens tomaram aquele presente, e dinheiro em dobro em suas
mãos, e a Benjamim; e levantaram-se, e desceram ao Egito, e
apresentaram-se diante de José. Vendo, pois, José a Benjamim
com eles, disse ao que estava sobre a sua casa: Leva estes homens à
casa, e mata reses, e prepara tudo; porque estes homens comerão
comigo ao meio-dia. E o homem fez como José dissera, e
levou-os à casa de José. Então temeram aqueles homens,
porquanto foram levados à casa de José, e diziam: Por causa do
dinheiro que dantes voltou nos nossos sacos, fomos trazidos aqui,
para nos incriminar e cair sobre nós, para que nos tome por servos,
e a nossos jumentos. Por isso chegaram-se ao homem que estava
sobre a casa de José, e falaram com ele à porta da casa, e
disseram: Ai! senhor meu, certamente descemos dantes a comprar
mantimento; e aconteceu que, chegando à estalagem, e abrindo
os nossos sacos, eis que o dinheiro de cada um estava na boca do seu
saco, nosso dinheiro por seu peso; e tornamos a trazê-lo em nossas
mãos; também trouxemos outro dinheiro em nossas mãos, para
comprar mantimento; não sabemos quem tenha posto o nosso dinheiro
nos nossos sacos. E ele disse: Paz seja convosco, não temais;
o vosso Deus, e o Deus de vosso pai, vos tem dado um tesouro nos
vossos sacos; o vosso dinheiro me chegou a mim. E trouxe-lhes fora a
Simeão. Depois levou os homens à casa de José, e deu-lhes
água, e lavaram os seus pés; também deu pasto aos seus jumentos.
E prepararam o presente, para quando José viesse ao meio dia;
porque tinham ouvido que ali haviam de comer pão.

Vindo, pois, José à casa, trouxeram-lhe ali o presente que tinham
em suas mãos; e inclinaram-se a ele até à terra. E ele lhes
perguntou como estavam, e disse: Vosso pai, o ancião de quem
falastes, está bem? Ainda vive? E eles disseram: Bem está o
teu servo, nosso pai vive ainda. E abaixaram a cabeça, e
inclinaram-se. E ele levantou os seus olhos, e viu a
Benjamim, seu irmão, filho de sua mãe, e disse: Este é vosso irmão
mais novo de quem falastes? Depois ele disse: Deus te dê a sua
graça, meu filho. E José apressou-se, porque as suas
entranhas comoveram-se por causa do seu irmão, e procurou onde
chorar; e entrou na câmara, e chorou ali. Depois lavou o seu
rosto, e saiu; e conteve-se, e disse: Ponde pão. E
serviram-lhe à parte, e a eles também à parte, e aos egípcios, que
comiam com ele, à parte; porque os egípcios não podem comer pão com
os hebreus, porquanto é abominação para os egípcios. E
assentaram-se diante dele, o primogênito segundo a sua
primogenitura, e o menor segundo a sua menoridade; do que os homens
se maravilhavam entre si. E apresentou-lhes as porções que
estavam diante dele; porém a porção de Benjamim era cinco vezes
maior do que as porções deles todos. E eles beberam, e se regalaram
com ele.

\smallskip

\lettrine{44} E deu ordem ao que estava sobre a sua casa,
dizendo: Enche de mantimento os sacos destes homens, quanto puderem
levar, e põe o dinheiro de cada um na boca do seu saco. E o meu
copo, o copo de prata, porás na boca do saco do mais novo, com o
dinheiro do seu trigo. E fez conforme a palavra que José tinha dito.
Vinda a luz da manhã, despediram-se estes homens, eles com os
seus jumentos. Saindo eles da cidade, e não se havendo ainda
distanciado, disse José ao que estava sobre a sua casa: Levanta-te,
e persegue aqueles homens; e, alcançando-os, lhes dirás: Por que
pagastes mal por bem? Não é este o copo em que bebe meu senhor e
pelo qual bem adivinha? Procedestes mal no que fizestes. E
alcançou-os, e falou-lhes as mesmas palavras. E eles
disseram-lhe: Por que diz meu senhor tais palavras? Longe estejam
teus servos de fazerem semelhante coisa. Eis que o dinheiro, que
temos achado nas bocas dos nossos sacos, te tornamos a trazer desde
a terra de Canaã; como, pois, furtaríamos da casa do teu senhor
prata ou ouro? Aquele, com quem de teus servos for achado,
morra; e ainda nós seremos escravos do meu senhor. E ele
disse: Ora seja também assim conforme as vossas palavras; aquele com
quem se achar será meu escravo, porém vós sereis desculpados.
E eles apressaram-se e cada um pôs em terra o seu saco, e
cada um abriu o seu saco. E buscou, começando do maior, e
acabando no mais novo; e achou-se o copo no saco de Benjamim.
Então rasgaram as suas vestes, e carregou cada um o seu
jumento, e tornaram à cidade. E veio Judá com os seus irmãos
à casa de José, porque ele ainda estava ali; e prostraram-se diante
dele em terra. E disse-lhes José: Que é isto que fizestes?
Não sabeis vós que um homem como eu pode, muito bem, adivinhar?
Então disse Judá: Que diremos a meu senhor? Que falaremos? E
como nos justificaremos? Achou Deus a iniqüidade de teus servos; eis
que somos escravos de meu senhor, tanto nós como aquele em cuja mão
foi achado o copo. Mas ele disse: Longe de mim que eu tal
faça; o homem em cuja mão o copo foi achado, esse será meu servo;
porém vós, subi em paz para vosso pai.

Então Judá se chegou a ele, e disse: Ai! senhor meu, deixa,
peço-te, o teu servo dizer uma palavra aos ouvidos de meu senhor, e
não se acenda a tua ira contra o teu servo; porque tu és como Faraó.
Meu senhor perguntou a seus servos, dizendo: Tendes vós pai,
ou irmão? E dissemos a meu senhor: Temos um velho pai, e um
filho da sua velhice, o mais novo, cujo irmão é morto; e só ele
ficou de sua mãe, e seu pai o ama. Então tu disseste a teus
servos: Trazei-mo a mim, e porei os meus olhos sobre ele. E
nós dissemos a meu senhor: Aquele moço não poderá deixar a seu pai;
se deixar a seu pai, este morrerá. Então tu disseste a teus
servos: Se vosso irmão mais novo não descer convosco, nunca mais
vereis a minha face. E aconteceu que, subindo nós a teu servo
meu pai, e contando-lhe as palavras de meu senhor, disse
nosso pai: Voltai, comprai-nos um pouco de mantimento. E nós
dissemos: Não poderemos descer; mas, se nosso irmão menor for
conosco, desceremos; pois não poderemos ver a face do homem se este
nosso irmão menor não estiver conosco. Então disse-nos teu
servo, meu pai: Vós sabeis que minha mulher me deu dois filhos; 28 e
um ausentou-se de mim, e eu disse: Certamente foi despedaçado, e não
o tenho visto até agora. Se agora também tirardes a este da
minha face, e lhe acontecer algum desastre, fareis descer as minhas
cãs com aflição à sepultura. Agora, pois, indo eu a teu
servo, meu pai, e o moço não indo conosco, como a sua alma está
ligada com a alma dele, acontecerá que, vendo ele que o moço
ali não está, morrerá; e teus servos farão descer as cãs de teu
servo, nosso pai, com tristeza à sepultura. Porque teu servo
se deu por fiador por este moço para com meu pai, dizendo: Se eu o
não tornar para ti, serei culpado para com meu pai por todos os
dias. Agora, pois, fique teu servo em lugar deste moço por
escravo de meu senhor, e que suba o moço com os seus irmãos.
Porque, como subirei eu a meu pai, se o moço não for comigo?
para que não veja eu o mal que sobrevirá a meu pai.

\smallskip

\lettrine{45} Então José não se podia conter diante de todos
os que estavam com ele; e clamou: Fazei sair daqui a todo o homem; e
ninguém ficou com ele, quando José se deu a conhecer a seus irmãos.
E levantou a sua voz com choro, de maneira que os egípcios o
ouviam, e a casa de Faraó o ouviu. E disse José a seus irmãos:
Eu sou José; vive ainda meu pai? E seus irmãos não lhe puderam
responder, porque estavam pasmados diante da sua face. E disse
José a seus irmãos: Peço-vos, chegai-vos a mim. E chegaram-se; então
disse ele: Eu sou José vosso irmão, a quem vendestes para o Egito.
Agora, pois, não vos entristeçais, nem vos pese aos vossos olhos
por me haverdes vendido para cá; porque \textbf{para conservação da
vida, Deus me enviou adiante de vós}. Porque já houve dois anos
de fome no meio da terra, e ainda restam cinco anos em que não
haverá lavoura nem sega. Pelo que \textbf{Deus me enviou adiante
de vós, para conservar vossa sucessão na terra, e para guardar-vos
em vida por um grande livramento}. Assim não fostes vós que me
enviastes para cá, senão Deus, que me tem posto por pai de Faraó, e
por senhor de toda a sua casa, e como regente em toda a terra do
Egito. Apressai-vos, e subi a meu pai, e dizei-lhe: Assim tem
dito o teu filho José: Deus me tem posto por senhor em toda a terra
do Egito; desce a mim, e não te demores; e habitarás na terra
de Gósen, e estarás perto de mim, tu e os teus filhos, e os filhos
dos teus filhos, e as tuas ovelhas, e as tuas vacas, e tudo o que
tens. E ali te sustentarei, porque ainda haverá cinco anos de
fome, para que não pereças de pobreza, tu e tua casa, e tudo o que
tens. E eis que vossos olhos, e os olhos de meu irmão
Benjamim, vêem que é minha boca que vos fala. E fazei saber a
meu pai toda a minha glória no Egito, e tudo o que tendes visto, e
apressai-vos a fazer descer meu pai para cá. E lançou-se ao
pescoço de Benjamim seu irmão, e chorou; e Benjamim chorou também ao
seu pescoço. E beijou a todos os seus irmãos, e chorou sobre
eles; e depois seus irmãos falaram com ele.

E esta notícia ouviu-se na casa de Faraó: Os irmãos de José são
vindos; e pareceu bem aos olhos de Faraó, e aos olhos de seus
servos. E disse Faraó a José: Dize a teus irmãos: Fazei isto:
carregai os vossos animais e parti, tornai à terra de Canaã.
E tornai a vosso pai, e às vossas famílias, e vinde a mim; e
eu vos darei o melhor da terra do Egito, e comereis da fartura da
terra. A ti, pois, é ordenado: Fazei isto: tomai vós da terra
do Egito carros para vossos meninos, para vossas mulheres, e para
vosso pai, e vinde. E não vos pese coisa alguma dos vossos
utensílios; porque o melhor de toda a terra do Egito será vosso.
E os filhos de Israel fizeram assim. E José deu-lhes carros,
conforme o mandado de Faraó; também lhes deu comida para o caminho.
A todos lhes deu, a cada um, mudas de roupas; mas a Benjamim
deu trezentas peças de prata, e cinco mudas de roupas. E a
seu pai enviou semelhantemente dez jumentos carregados do melhor do
Egito, e dez jumentos carregados de trigo e pão, e comida para seu
pai, para o caminho. E despediu os seus irmãos, e partiram; e
disse-lhes: Não contendais pelo caminho.

E subiram do Egito, e vieram à terra de Canaã, a Jacó seu pai.
Então lhe anunciaram, dizendo: José ainda vive, e ele também
é regente em toda a terra do Egito. E o seu coração desmaiou, porque
não os acreditava. Porém, havendo-lhe eles contado todas as
palavras de José, que ele lhes falara, e vendo ele os carros que
José enviara para levá-lo, reviveu o espírito de Jacó seu pai.
E disse Israel: Basta; ainda vive meu filho José; eu irei e o
verei antes que morra.

\smallskip

\lettrine{46} E partiu Israel com tudo quanto tinha, e veio a
Berseba, e ofereceu sacrifícios ao Deus de seu pai Isaque. E
falou Deus a Israel em visões de noite, e disse: \textbf{Jacó,
Jacó!} E ele disse: Eis-me aqui. E disse: Eu sou Deus, o Deus de
teu pai; não temas descer ao Egito, porque eu te farei ali uma
grande nação. E descerei contigo ao Egito, e certamente te farei
tornar a subir, e José porá a sua mão sobre os teus olhos.

Então levantou-se Jacó de Berseba; e os filhos de Israel levaram a
seu pai Jacó, e seus meninos, e as suas mulheres, nos carros que
Faraó enviara para o levar. E tomaram o seu gado e os seus bens
que tinham adquirido na terra de Canaã, e vieram ao Egito, Jacó e
toda a sua descendência com ele; os seus filhos e os filhos de
seus filhos com ele, as filhas, e as filhas de seus filhos, e toda a
sua descendência levou consigo ao Egito. E estes são os nomes
dos filhos de Israel, que vieram ao Egito, Jacó e seus filhos:
Rúben, o primogênito de Jacó. E os filhos de Rúben: Enoque,
Palu, Hezrom e Carmi. E os filhos de Simeão: Jemuel, Jamim,
Oade, Jaquim, Zoar e Saul, filho de uma mulher cananéia. E os
filhos de Levi: Gérson, Coate e Merari. E os filhos de Judá:
Er, Onã, Selá, Perez e Zerá; Er e Onã, porém, morreram na terra de
Canaã; e os filhos de Perez foram Hezrom e Hamul. E os filhos
de Issacar: Tola, Puva, Jó e Sinrom. E os filhos de Zebulom:
Serede, Elom e Jaleel. Estes são os filhos de Lia, que ela
deu a Jacó em Padã-Arã, além de Diná, sua filha; todas as almas de
seus filhos e de suas filhas foram trinta e três. E os filhos
de Gade: Zifiom, Hagi, Suni, Esbom, Eri, Arodi e Areli. E os
filhos de Aser: Imna, Isvá, Isvi, Berias e Sera, a irmã deles; e os
filhos de Berias: Héber e Malquiel. Estes são os filhos de
Zilpa, a qual Labão deu à sua filha Lia; e deu a Jacó estas
dezesseis almas. Os filhos de Raquel, mulher de Jacó: José e
Benjamim. E nasceram a José na terra do Egito, Manassés e
Efraim, que lhe deu Azenate, filha de Potífera, sacerdote de Om.
E os filhos de Benjamim: Belá, Bequer, Asbel, Gera, Naamã,
Eí, Rôs, Mupim, Hupim e Arde. Estes são os filhos de Raquel,
que nasceram a Jacó, ao todo catorze almas. E o filho de Dã:
Husim. E os filhos de Naftali: Jazeel, Guni, Jezer e Silém.
Estes são os filhos de Bila, a qual Labão deu à sua filha
Raquel; e deu estes a Jacó; todas as almas foram sete. Todas
as almas que vieram com Jacó ao Egito, que saíram dos seus lombos,
fora as mulheres dos filhos de Jacó, todas foram \textbf{sessenta e
seis almas}. E os filhos de José, que lhe nasceram no Egito,
eram duas almas. \textbf{Todas as almas da casa de Jacó, que vieram
ao Egito, eram setenta}.

E Jacó enviou Judá adiante de si a José, para o encaminhar a
Gósen; e chegaram à terra de Gósen. Então José aprontou o seu
carro, e subiu ao encontro de Israel, seu pai, a Gósen. E,
apresentando-se-lhe, lançou-se ao seu pescoço, e chorou sobre o seu
pescoço longo tempo. E Israel disse a José: Morra eu agora,
pois já tenho visto o teu rosto, que ainda vives. Depois
disse José a seus irmãos, e à casa de seu pai: Eu subirei e
anunciarei a Faraó, e lhe direi: Meus irmãos e a casa de meu pai,
que estavam na terra de Canaã, vieram a mim! E os homens são
pastores de ovelhas, porque são homens de gado, e trouxeram consigo
as suas ovelhas, e as suas vacas, e tudo o que têm. Quando,
pois, acontecer que Faraó vos chamar, e disser: Qual é o vosso
negócio? Então direis: Teus servos foram homens de gado desde
a nossa mocidade até agora, tanto nós como os nossos pais; para que
habiteis na terra de Gósen, porque todo o pastor de ovelhas é
abominação aos egípcios.

\smallskip

\lettrine{47} Então veio José e anunciou a Faraó, e disse: Meu
pai e os meus irmãos e as suas ovelhas, e as suas vacas, com tudo o
que têm, são vindos da terra de Canaã, e eis que estão na terra de
Gósen. E tomou uma parte de seus irmãos, a saber, cinco homens,
e os pôs diante de Faraó. Então disse Faraó a seus irmãos: Qual
é o vosso negócio? E eles disseram a Faraó: Teus servos são pastores
de ovelhas, tanto nós como nossos pais. Disseram mais a Faraó:
Viemos para peregrinar nesta terra; porque não há pasto para as
ovelhas de teus servos, porquanto a fome é grave na terra de Canaã;
agora, pois, rogamos-te que teus servos habitem na terra de Gósen.
Então falou Faraó a José, dizendo: Teu pai e teus irmãos vieram
a ti; a terra do Egito está diante de ti; no melhor da terra
faze habitar teu pai e teus irmãos; habitem na terra de Gósen, e se
sabes que entre eles há homens valentes, os porás por maiorais do
gado, sobre o que eu tenho. E trouxe José a Jacó, seu pai, e o
apresentou a Faraó; e Jacó abençoou a Faraó. E Faraó disse a
Jacó: Quantos são os dias dos anos da tua vida? E Jacó disse a
Faraó: Os dias dos anos das minhas peregrinações são \textbf{cento e
trinta anos}, poucos e maus foram os dias dos anos da minha vida, e
não chegaram aos dias dos anos da vida de meus pais nos dias das
suas peregrinações. E Jacó abençoou a Faraó, e saiu da sua
presença. E José fez habitar a seu pai e seus irmãos e
deu-lhes possessão na terra do Egito, no melhor da terra, na terra
de \textbf{Ramessés}, como Faraó ordenara. E José sustentou
de pão a seu pai, seus irmãos e toda a casa de seu pai, segundo as
suas famílias.

E não havia pão em toda a terra, porque a fome era muito grave;
de modo que a terra do Egito e a terra de Canaã desfaleciam por
causa da fome. Então José recolheu todo o dinheiro que se
achou na terra do Egito, e na terra de Canaã, pelo trigo que
compravam; e José trouxe o dinheiro à casa de Faraó.
Acabando-se, pois, o dinheiro da terra do Egito, e da terra
de Canaã, vieram todos os egípcios a José, dizendo: Dá-nos pão; por
que morreremos em tua presença? porquanto o dinheiro nos falta.
E José disse: Dai o vosso gado, e eu vo-lo darei por vosso
gado, se falta o dinheiro. Então trouxeram o seu gado a José;
e José deu-lhes pão em troca de cavalos, e das ovelhas, e das vacas
e dos jumentos; e os sustentou de pão aquele ano por todo o seu
gado. E acabado aquele ano, vieram a ele no segundo ano e
disseram-lhe: Não ocultaremos ao meu senhor que o dinheiro acabou; e
meu senhor possui os animais, e nenhuma outra coisa nos ficou diante
de meu senhor, senão o nosso corpo e a nossa terra; por que
morreremos diante dos teus olhos, tanto nós como a nossa terra?
Compra-nos a nós e a nossa terra por pão, e nós e a nossa terra
seremos servos de Faraó; e dá-nos semente, para que vivamos, e não
morramos, e a terra não se desole. Assim José comprou toda a
terra do Egito para Faraó, porque os egípcios venderam cada um o seu
campo, porquanto a fome prevaleceu sobre eles; e a terra ficou sendo
de Faraó. E, quanto ao povo, fê-lo passar às cidades, desde
uma extremidade da terra do Egito até a outra extremidade.
Somente a terra dos sacerdotes não a comprou, porquanto os
sacerdotes tinham porção de Faraó, e eles comiam a sua porção que
Faraó lhes tinha dado; por isso não venderam a sua terra.
Então disse José ao povo: Eis que hoje tenho comprado a vós e
a vossa terra para Faraó; eis aí tendes semente para vós, para que
semeeis a terra. Há de ser, porém, que das colheitas dareis o
quinto a Faraó, e as quatro partes serão vossas, para semente do
campo, e para o vosso mantimento, e dos que estão nas vossas casas,
e para que comam vossos filhos. E disseram: A vida nos tens
dado; achemos graça aos olhos de meu senhor, e seremos servos de
Faraó. José, pois, estabeleceu isto por estatuto, até ao dia
de hoje, sobre a terra do Egito, que Faraó tirasse o quinto; só a
terra dos sacerdotes não ficou sendo de Faraó.

Assim habitou Israel na terra do Egito, na terra de Gósen, e nela
tomaram possessão, e frutificaram, e multiplicaram-se muito.
E Jacó viveu na terra do Egito dezessete anos, de sorte que
\textbf{os dias de Jacó, os anos da sua vida, foram cento e quarenta
e sete anos}. Chegando-se, pois, o tempo da morte de Israel,
chamou a José, seu filho, e disse-lhe: Se agora tenho achado graça
em teus olhos, rogo-te que ponhas a tua mão debaixo da minha coxa, e
usa comigo de beneficência e verdade; rogo-te que não me enterres no
Egito, mas que eu jaza com os meus pais; por isso me levarás
do Egito e me enterrarás na sepultura deles. E ele disse: Farei
conforme a tua palavra. Então, lhe disse Jacó: Jura-me. E ele
jurou-lhe; e Israel se inclinou sobre a cabeceira da cama.

\smallskip

\lettrine{48} E aconteceu, depois destas coisas, que alguém
disse a José: Eis que teu pai está enfermo. Então tomou consigo os
seus dois filhos, Manassés e Efraim. E alguém participou a Jacó,
e disse: Eis que José teu filho vem a ti. E esforçou-se Israel, e
assentou-se sobre a cama. E Jacó disse a José: O \textbf{Deus
Todo-Poderoso} me apareceu em Luz, na terra de Canaã, e me abençoou.
E me disse: Eis que te farei frutificar e multiplicar, e
tornar-te-ei uma multidão de povos e darei esta terra à tua
descendência depois de ti, em possessão perpétua. Agora, pois,
os teus dois filhos, que te nasceram na terra do Egito, antes que eu
viesse a ti no Egito, são meus: Efraim e Manassés serão meus, como
Rúben e Simeão; mas a tua geração, que gerarás depois deles,
será tua; segundo o nome de seus irmãos serão chamados na sua
herança. Vindo, pois, eu de Padã, morreu-me Raquel no caminho,
na terra de Canaã, havendo ainda pequena distância para chegar a
Efrata; e eu a sepultei ali, no caminho de Efrata, que é Belém.

E Israel viu os filhos de José, e disse: Quem são estes? E
José disse a seu pai: Eles são meus filhos, que Deus me tem dado
aqui. E ele disse: Peço-te, traze-mos aqui, para que os abençoe.
Os olhos de Israel, porém, estavam carregados de velhice, já
não podia ver; e fê-los chegar a ele, e beijou-os, e abraçou-os.
E Israel disse a José: Eu não cuidara ver o teu rosto; e eis
que Deus me fez ver também a tua descendência. Então José os
tirou dos joelhos de seu pai, e inclinou-se à terra diante da sua
face. E tomou José a ambos, a Efraim na sua mão direita, à
esquerda de Israel, e Manassés na sua mão esquerda, à direita de
Israel, e fê-los chegar a ele. Mas Israel estendeu a sua mão
direita e a pôs sobre a cabeça de Efraim, que era o menor, e a sua
esquerda sobre a cabeça de Manassés, dirigindo as suas mãos
propositadamente, não obstante Manassés ser o primogênito. E
abençoou a José, e disse: O Deus, em cuja presença andaram os meus
pais Abraão e Isaque, o Deus que me sustentou, desde que eu nasci
até este dia; o anjo que me livrou de todo o mal, abençoe
estes rapazes, e seja chamado neles o meu nome, e o nome de meus
pais Abraão e Isaque, e multipliquem-se como peixes, em multidão, no
meio da terra. Vendo, pois, José que seu pai punha a sua mão
direita sobre a cabeça de Efraim, foi mau aos seus olhos; e tomou a
mão de seu pai, para a transpor de sobre a cabeça de Efraim à cabeça
de Manassés. E José disse a seu pai: Não assim, meu pai,
porque este é o primogênito; põe a tua mão direita sobre a sua
cabeça. Mas seu pai recusou, e disse: Eu o sei, meu filho, eu
o sei; também ele será um povo, e também ele será grande; contudo o
seu irmão menor será maior que ele, e a sua descendência será uma
multidão de nações. Assim os abençoou naquele dia, dizendo:
Em ti abençoará Israel, dizendo: Deus te faça como a Efraim e como a
Manassés. E pôs a Efraim diante de Manassés. Depois disse
Israel a José: Eis que eu morro, mas Deus será convosco, e vos fará
tornar à terra de vossos pais. E eu tenho dado a ti um pedaço
da terra a mais do que a teus irmãos, que tomei com a minha espada e
com o meu arco, da mão dos amorreus.

\smallskip

\lettrine{49} Depois chamou Jacó a seus filhos, e disse:
Ajuntai-vos, e \textbf{anunciar-vos-ei o que vos há de acontecer nos
dias vindouros}; ajuntai-vos, e ouvi, filhos de Jacó; e ouvi a
Israel vosso pai. \textbf{Rúben}, tu és meu primogênito, minha
força e o princípio de meu vigor, o mais excelente em alteza e o
mais excelente em poder. Impetuoso como a água, não serás o mais
excelente, porquanto subiste ao leito de teu pai. Então o
contaminaste; subiu à minha cama.

\textbf{Simeão} e \textbf{Levi} são irmãos; as suas espadas são
instrumentos de violência. No seu secreto conselho não entre
minha alma, com a sua congregação minha glória não se ajunte; porque
no seu furor mataram homens, e na sua teima arrebataram bois.
Maldito seja o seu furor, pois era forte, e a sua ira, pois era
dura; eu os dividirei em Jacó, e os espalharei em Israel.

\textbf{Judá}, a ti te louvarão os teus irmãos; a tua mão será
sobre o pescoço de teus inimigos; os filhos de teu pai a ti se
inclinarão. Judá é um leãozinho, da presa subiste, filho meu;
encurva-se, e deita-se como um leão, e como um leão velho; quem o
despertará? O cetro não se arredará de Judá, nem o legislador
dentre seus pés, até que venha Siló; e a ele se congregarão os
povos. Ele amarrará o seu jumentinho à vide, e o filho da sua
jumenta à cepa mais excelente; ele lavará a sua roupa no vinho, e a
sua capa em sangue de uvas. Os olhos serão vermelhos de
vinho, e os dentes brancos de leite.

\textbf{Zebulom} habitará no porto dos mares, e será como porto
dos navios, e o seu termo será para Sidom. \textbf{Issacar} é
jumento de fortes ossos, deitado entre dois fardos. E viu ele
que o descanso era bom, e que a terra era deliciosa e abaixou seu
ombro para acarretar\footnote{Transportar em carreta ou carro.
Transportar à cabeça, às costas ou de qualquer maneira. Ed.
Contemp.: baixou o ombro à carga.}, e serviu debaixo de tributo.
\textbf{Dã} julgará o seu povo, como uma das tribos de
Israel. Dã será serpente junto ao caminho, uma víbora junto à
vereda, que morde os calcanhares do cavalo, e faz cair o seu
cavaleiro por detrás. \textbf{A tua salvação espero, ó
Senhor!} Quanto a \textbf{Gade}, uma tropa o acometerá; mas
ele a acometerá por fim. De \textbf{Aser}, o seu pão será
gordo, e ele dará delícias reais. \textbf{Naftali} é uma
gazela solta; ele dá palavras formosas.

\textbf{José} é um ramo frutífero, ramo frutífero junto à fonte;
seus ramos correm sobre o muro. Os flecheiros lhe deram
amargura, e o flecharam e odiaram. O seu arco, porém,
susteve-se no forte, e os braços de suas mãos foram fortalecidos
pelas mãos do \textbf{Valente de Jacó} (de onde é o pastor e a
\textbf{pedra de Israel}). Pelo Deus de teu pai, o qual te
ajudará, e pelo Todo-Poderoso, o qual te abençoará com bênçãos dos
altos céus, com bênçãos do abismo que está embaixo, com bênçãos dos
seios e da madre. As bênçãos de teu pai excederão as bênçãos
de meus pais, até à extremidade dos outeiros eternos; elas estarão
sobre a cabeça de José, e sobre o alto da cabeça do que foi separado
de seus irmãos. \textbf{Benjamim} é lobo que despedaça; pela
manhã comerá a presa, e à tarde repartirá o despojo.

Todas estas são as doze tribos de Israel; e isto é o que lhes
falou seu pai quando os abençoou; a cada um deles abençoou segundo a
sua bênção. Depois ordenou-lhes, e disse-lhes: Eu me congrego
ao meu povo; sepultai-me com meus pais, na cova que está no campo de
Efrom, o heteu, na cova que está no campo de Macpela, que
está em frente de Manre, na terra de Canaã, a qual Abraão comprou
com aquele campo de Efrom, o heteu, por herança de sepultura.
Ali sepultaram a Abraão e a Sara sua mulher; ali sepultaram a
Isaque e a Rebeca sua mulher; e ali eu sepultei a Lia. O
campo e a cova que está nele, foram comprados aos filhos de Hete.
Acabando, pois, Jacó de dar instruções a seus filhos,
encolheu os pés na cama, e expirou, e foi congregado ao seu povo.

\smallskip

\lettrine{50} Então José se lançou sobre o rosto de seu pai e
chorou sobre ele, e o beijou. E José ordenou aos seus servos, os
médicos, que embalsamassem a seu pai; e os médicos embalsamaram a
Israel. E cumpriram-se-lhe quarenta dias; porque assim se
cumprem os dias daqueles que se embalsamam; e os egípcios o choraram
setenta dias. Passados, pois, os dias de seu choro, falou José à
casa de Faraó, dizendo: Se agora tenho achado graça aos vossos
olhos, rogo-vos que faleis aos ouvidos de Faraó, dizendo: Meu
pai me fez jurar, dizendo: Eis que eu morro; em meu sepulcro, que
cavei para mim na terra de Canaã, ali me sepultarás. Agora, pois, te
peço, que eu suba, para que sepulte a meu pai; então voltarei. E
Faraó disse: Sobe, e sepulta a teu pai como ele te fez jurar.

E José subiu para sepultar a seu pai; e subiram com ele todos os
servos de Faraó, os anciãos da sua casa, e todos os anciãos da terra
do Egito. Como também toda a casa de José, e seus irmãos, e a
casa de seu pai; somente deixaram na terra de Gósen os seus meninos,
e as suas ovelhas e as suas vacas. E subiram também com ele,
tanto carros como gente a cavalo; e o cortejo foi grandíssimo.
Chegando eles, pois, à eira de Atade, que está além do
Jordão, fizeram um grande e dolorido pranto; e fez a seu pai uma
grande lamentação por sete dias. E vendo os moradores da
terra, os cananeus, o luto na eira de Atade, disseram: É este o
pranto grande dos egípcios. Por isso chamou-se-lhe Abel-Mizraim, que
está além do Jordão. E fizeram-lhe os seus filhos assim como
ele lhes ordenara. Pois os seus filhos o levaram à terra de
Canaã, e o sepultaram na cova do campo de Macpela, que Abraão tinha
comprado com o campo, por herança de sepultura de Efrom, o heteu, em
frente de Manre. Depois de haver sepultado seu pai, voltou
José para o Egito, ele e seus irmãos, e todos os que com ele subiram
a sepultar seu pai.

Vendo então os irmãos de José que seu pai já estava morto,
disseram: Porventura nos odiará José e certamente nos retribuirá
todo o mal que lhe fizemos. Portanto mandaram dizer a José:
Teu pai ordenou, antes da sua morte, dizendo: Assim direis a
José: Perdoa, rogo-te, a transgressão de teus irmãos, e o seu
pecado, porque te fizeram mal; agora, pois, rogamos-te que perdoes a
transgressão dos servos do Deus de teu pai. E José chorou quando
eles lhe falavam. Depois vieram também seus irmãos, e
prostraram-se diante dele, e disseram: Eis-nos aqui por teus servos.
E José lhes disse: Não temais; porventura estou eu em lugar
de Deus? Vós bem intentastes mal contra mim; porém Deus o
intentou para bem, para fazer como se vê neste dia, para conservar
muita gente com vida. Agora, pois, não temais; eu vos
sustentarei a vós e a vossos filhos. Assim os consolou, e falou
segundo o coração deles.

José, pois, habitou no Egito, ele e a casa de seu pai; \textbf{e
viveu José cento e dez anos}. E viu José os filhos de Efraim,
da terceira geração; também os filhos de Maquir, filho de Manassés,
nasceram sobre os joelhos de José. E disse José a seus
irmãos: Eu morro; mas Deus certamente vos visitará, e vos fará subir
desta terra à terra que jurou a Abraão, a Isaque e a Jacó. E
José fez jurar os filhos de Israel, dizendo: Certamente vos visitará
Deus, e fareis transportar os meus ossos daqui. E morreu José
da idade de cento e dez anos, e o embalsamaram e o puseram num
caixão no Egito.

