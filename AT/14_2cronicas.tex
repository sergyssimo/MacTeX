\addchap{Segundo livro de Crônicas}

\lettrine{1} Salomão, filho de Davi, fortaleceu-se no seu
reino; e o Senhor seu Deus era com ele, e o engrandeceu
sobremaneira. E falou Salomão a todo o Israel, aos capitães de
mil e de cem, aos juízes e a todos os governadores em todo o Israel,
chefes das famílias. E foi Salomão, e toda a congregação com
ele, ao alto que estava em Gibeom, porque ali estava a tenda da
congregação de Deus, que Moisés, servo do Senhor, tinha feito no
deserto. Mas Davi tinha feito subir a arca de Deus de
Quiriate-Jearim ao lugar que lhe preparara; porque lhe tinha armado
uma tenda em Jerusalém. Também o altar de cobre que tinha feito
Bezaleel, filho de Uri, filho de Hur, estava ali diante do
tabernáculo do Senhor; e Salomão e a congregação o buscavam. E
Salomão ofereceu ali sacrifícios perante o Senhor, sobre o altar de
cobre que estava na tenda da congregação; e ofereceu sobre ele mil
holocaustos. Naquela mesma noite Deus apareceu a Salomão, e
disse-lhe: Pede o que queres que eu te dê. E Salomão disse a
Deus: Tu usaste de grande benignidade com meu pai Davi, e a mim me
fizeste rei em seu lugar. Agora, pois, ó Senhor Deus,
confirme-se a tua palavra, dada a meu pai Davi; porque tu me fizeste
reinar sobre um povo numeroso como o pó da terra. Dá-me,
pois, agora, sabedoria e conhecimento, para que possa sair e entrar
perante este povo; pois quem poderia julgar a este tão grande povo?
Então Deus disse a Salomão: Porquanto houve isto no teu
coração, e não pediste riquezas, bens, ou honra, nem a morte dos que
te odeiam, nem tampouco pediste muitos dias de vida, mas pediste
para ti sabedoria e conhecimento, para poderes julgar a meu povo,
sobre o qual te constituí rei, sabedoria e conhecimento te
são dados; e te darei riquezas, bens e honra, quais não teve nenhum
rei antes de ti, e nem depois de ti haverá.

Assim Salomão veio a Jerusalém, do alto que estava em Gibeom, de
diante da tenda da congregação; e reinou sobre Israel. E
Salomão ajuntou carros e cavaleiros, e teve mil e quatrocentos
carros, e doze mil cavaleiros; os quais pôs nas cidades dos carros,
e junto ao rei em Jerusalém. E fez o rei que houvesse ouro e
prata em Jerusalém como pedras; e cedros em tanta abundância como
sicômoros que há pelas campinas. E os cavalos, que tinha
Salomão, eram trazidos do Egito; e os mercadores do rei os recebiam
em tropas, cada uma pelo seu preço. E faziam subir e sair do
Egito cada carro por seiscentos siclos de prata, e cada cavalo por
cento e cinqüenta; e assim, por meio deles eram para todos os reis
dos heteus, e para os reis da Síria.

\medskip

\lettrine{2} E determinou Salomão edificar uma casa ao nome do
Senhor, como também uma casa para o seu reino. E designou
Salomão setenta mil homens de carga, e oitenta mil, que talhavam
pedras na montanha e três mil e seiscentos inspetores sobre eles.
E Salomão mandou dizer a Hirão, rei de Tiro: Como fizeste com
Davi meu pai, mandando-lhe cedros, para edificar uma casa em que
morasse, assim também faze comigo. Eis que estou para edificar
uma casa ao nome do Senhor meu Deus, para lhe consagrar, para
queimar perante ele incenso aromático, e para a apresentação
contínua do pão da proposição, para os holocaustos da manhã e da
tarde, nos sábados e nas luas novas, e nas festividades do Senhor
nosso Deus; o que é obrigação perpétua de Israel. E a casa que
estou para edificar há de ser grande; porque o nosso Deus é maior do
que todos os deuses. Porém, quem seria capaz de lhe edificar uma
casa, visto que os céus e até os céus dos céus o não podem conter? E
quem sou eu, que lhe edificasse casa, salvo para queimar incenso
perante ele? Manda-me, pois, agora, um homem hábil para
trabalhar em ouro, em prata, em bronze, em ferro, em púrpura, em
carmesim\footnote{Diz-se de, ou cor vermelha muito viva. Carmim:
matéria corante, de um vermelho muito vivo, ligeiramente arroxeado,
extraída, originariamente, da cochonilha-do-carmim.} e em azul; e
que saiba lavrar ao buril\footnote{Instrumento de gravador, usado na
execução de gravuras em metal e em madeira de topo, constituído de
barra de aço especialmente temperado, de seção quadrada, triangular
ou romboidal, com uma extremidade biselada, losângica, e a outra
metida em cabo achatado. Instrumento semelhante para lavrar pedra.},
juntamente com os peritos que estão comigo em Judá e em Jerusalém,
os quais Davi, meu pai, preparou. Manda-me também madeiras de
cedro, de cipreste, e algumins\footnote{RA: ``Manda-me também
madeira de cedros, ciprestes e sândalo do Líbano''. Sândalo: Bot.
Gênero de árvores da família das santaláceas, de folhas simples,
oblongas, pequenas flores brancas, campanuladas, e que fornecem
madeira amarelada, odorífera, e óleo essencial, obtido da raiz. Bot.
Qualquer espécie desse gênero, como, p. ex., a Santalum album,
originária da Índia e adjacências, e de cuja madeira, resistente e
aromática, se extrai um óleo de uso clássico, em perfumaria, para o
fabrico do sândalo. Bot. Qualquer espécime desse gênero. Essência
perfumada e balsâmica, extraída do tronco e das raízes do sândalo, e
utilizada em preparados farmacêuticos. Perfume fabricado com essa
essência.} do Líbano; porque bem sei eu que os teus servos sabem
cortar madeira no Líbano; e eis que os meus servos estarão com os
teus servos. E isso para prepararem muita madeira; porque a casa
que estou para fazer há de ser grande e maravilhosa. E eis
que a teus servos, os cortadores, que cortarem a madeira, darei
vinte mil coros de trigo malhado, vinte mil coros de cevada, vinte
mil batos de vinho e vinte mil batos de azeite.

E Hirão, rei de Tiro, respondeu por escrito que enviou a Salomão,
dizendo: Porque o Senhor tem amado o seu povo, te constituiu sobre
ele rei. Disse mais Hirão: Bendito seja o Senhor Deus de
Israel, que fez os céus e a terra; o que deu ao rei Davi um filho
sábio, de grande prudência e entendimento, que edifique casa ao
Senhor, e para o seu reino. Agora, pois, envio um homem sábio
de grande entendimento, a saber, Hirão meu pai, filho de uma
mulher das filhas de Dã, e cujo pai foi homem de Tiro; este sabe
trabalhar em ouro, em prata, em bronze, em ferro, em pedras e em
madeira, em púrpura, em azul, e em linho fino, e em carmesim, e é
hábil para toda a obra do buril, e para toda a espécie de invenções,
qualquer coisa que se lhe propuser, juntamente com os teus peritos,
e os peritos de Davi, meu senhor, teu pai. Agora, pois, meu
senhor, mande para os seus servos o trigo, a cevada, o azeite e o
vinho, de que falou; e nós cortaremos tanta madeira no
Líbano, quanta houveres mister, e ta traremos em jangadas pelo mar
até Jope, e tu a farás subir a Jerusalém. E Salomão contou
todos os homens estrangeiros, que havia na terra de Israel, conforme
o censo com que os contara Davi seu pai; e acharam-se cento e
cinqüenta e três mil e seiscentos. E designou deles setenta
mil carregadores, e oitenta mil cortadores na montanha; como também
três mil e seiscentos inspetores, para fazerem trabalhar o povo.

\medskip

\lettrine{3} E começou Salomão a edificar a casa do Senhor em
Jerusalém, no monte Moriá, onde o Senhor aparecera a Davi seu pai,
no lugar que Davi tinha preparado na eira de Ornã, o jebuseu. E
começou a edificar no segundo mês, no segundo dia, no ano quarto do
seu reinado. E estes foram os fundamentos que Salomão pôs para
edificar a casa de Deus: o comprimento em côvados, segundo a
primeira medida, era de sessenta côvados, e a largura de vinte
côvados. E o pátio, que estava na frente, tinha vinte côvados de
comprimento, segundo a largura da casa, e a altura era de cento e
vinte; e por dentro o revestiu com ouro puro. E a casa grande
forrou com madeira de faia; e então a revestiu com ouro fino; e fez
sobre ela palmas e cadeias. Também a casa adornou de pedras
preciosas, para ornamento; e o ouro era ouro de Parvaim. Também
na casa revestiu, com ouro, as traves, os umbrais, as suas paredes e
as suas portas; e lavrou querubins nas paredes. Fez mais a casa
do lugar santíssimo, cujo comprimento, segundo a largura da casa,
era de vinte côvados, e também a largura de vinte côvados; e
revestiu-a de ouro fino, do peso de seiscentos talentos. O peso
dos pregos era de cinqüenta siclos de ouro; e as câmaras cobriu de
ouro.

Também fez na casa do lugar santíssimo dois querubins de obra
móvel, e cobriu-os de ouro. E, quanto às asas dos querubins,
o seu comprimento era de vinte côvados; a asa de um deles, de cinco
côvados, e tocava na parede da casa; e a outra asa de cinco côvados,
e tocava na asa do outro querubim. Também a asa do outro
querubim era de cinco côvados, e tocava na parede da casa; era
também a outra asa de cinco côvados, que tocava na asa do outro
querubim. E as asas destes querubins se estendiam vinte
côvados; e estavam postos em pé, e os seus rostos virados para a
casa. Também fez o véu de azul, púrpura, carmesim e linho
fino; e pôs sobre ele querubins. Fez também, diante da casa,
duas colunas de trinta e cinco côvados de altura; e o capitel, que
estava sobre cada uma, era de cinco côvados. Também fez
cadeias no oráculo\footnote{RA: ``Também fez cadeias, como no Santo
dos Santos,(\ldots)''.}, e as pôs sobre as cabeças das colunas; fez
também cem romãs, as quais pôs entre as cadeias. E levantou
as colunas diante do templo, uma à direita, e outra à esquerda; e
chamou o nome da que estava à direita Jaquim, e o nome da que estava
à esquerda Boaz.

\medskip

\lettrine{4} Também fez um altar de metal, de vinte côvados de
comprimento, de vinte côvados de largura e de dez côvados de altura.
Fez também o mar de fundição, de dez côvados de uma borda até a
outra, redondo, e de cinco côvados de altura; cingia-o ao redor um
cordão de trinta côvados. E por baixo dele havia figuras de
bois, que cingiam o mar ao redor, dez em cada côvado, contornando-o;
e tinha duas fileiras de bois, fundidos juntamente com o mar. E
o mar estava posto sobre doze bois; três que olhavam para o norte,
três que olhavam para o ocidente, três que olhavam para o sul e três
que olhavam para o oriente; e o mar estava posto sobre eles; e as
suas partes posteriores estavam todas para o lado de dentro. E
tinha um palmo de grossura, e a sua borda foi feita como a borda de
um copo, ou como uma flor-de-lis, da capacidade de três mil batos.
Também fez dez pias; e pôs cinco à direita e cinco à esquerda,
para lavarem nelas; o que pertencia ao holocausto o lavavam nelas;
porém o mar era para que os sacerdotes se lavassem nele. Fez
também dez castiçais de ouro, segundo a sua forma, e pô-los no
templo, cinco à direita, e cinco à esquerda. Também fez dez
mesas, e pô-las no templo, cinco à direita e cinco à esquerda;
também fez cem bacias de ouro. Fez mais o pátio dos sacerdotes,
e o grande átrio; como também as portas para o pátio, as quais
revestiu de cobre. E pôs o mar ao lado direito, para o lado
do oriente, na direção do sul.

Também Hirão fez as caldeiras, as pás e as bacias. Assim acabou
Hirão de fazer a obra, que fazia para o rei Salomão, na casa de
Deus. As duas colunas, os globos, e os dois capitéis sobre as
cabeças das colunas; e as duas redes, para cobrir os dois globos dos
capitéis, que estavam sobre a cabeça das colunas. E as
quatrocentas romãs para as duas redes; duas carreiras de romãs para
cada rede, para cobrirem os dois globos dos capitéis que estavam em
cima das colunas. Também fez as bases; e as pias pôs sobre as
bases; um mar, e os doze bois debaixo dele;
semelhantemente as caldeiras, as pás, os garfos e todos os
seus utensílios, fez Hirão Abiú ao rei Salomão, para a casa do
Senhor, de cobre polido. Na campina do Jordão os fundiu o
rei, na terra argilosa, entre Sucote e Zeredá. E fez Salomão
todos estes objetos em grande abundância, que não se podia averiguar
o peso do cobre. Fez também Salomão todos os objetos que eram
para a casa de Deus, como também o altar de ouro, e as mesas, sobre
as quais estavam os pães da proposição. E os castiçais com as
suas lâmpadas de ouro finíssimo, para as acenderem segundo o
costume, perante o oráculo. E as flores, as lâmpadas e os
espevitadores eram de ouro, do mais finíssimo ouro. Como
também os apagadores, as bacias, as colheres e os incensários de
ouro finíssimo; e quanto à entrada da casa, as suas portas de dentro
do lugar santíssimo, e as portas da casa do templo, eram de ouro.

\medskip

\lettrine{5} Assim se acabou toda a obra que Salomão fez para
a casa do Senhor; então trouxe Salomão as coisas que seu pai Davi
havia consagrado, a prata, o ouro e todos os objetos, e pô-los entre
os tesouros da casa de Deus. Então Salomão congregou em
Jerusalém os anciãos de Israel, e todos os chefes das tribos, os
chefes dos pais entre os filhos de Israel, para fazerem subir a arca
da aliança do Senhor, da cidade de Davi, que é Sião. E todos os
homens de Israel se congregaram ao rei na ocasião da festa, que foi
no sétimo mês. E vieram todos os anciãos de Israel; e os levitas
levantaram a arca. E fizeram subir a arca, e a tenda da
congregação, com todos os objetos sagrados, que estavam na tenda; os
sacerdotes e os levitas os fizeram subir. Então o rei Salomão e
toda a congregação de Israel, que se tinha reunido com ele diante da
arca, sacrificaram carneiros e bois, que não se podiam contar, nem
numerar, por causa da sua abundância. Assim trouxeram os
sacerdotes a arca da aliança do Senhor ao seu lugar, ao oráculo da
casa, ao lugar santíssimo, até debaixo das asas dos querubins.
Porque os querubins estendiam ambas as asas sobre o lugar da
arca, e os querubins cobriam, por cima, a arca e os seus varais.
Então os varais sobressaíam para que as pontas dos varais da
arca se vissem perante o oráculo, mas não se vissem de fora; e ali
tem estado até ao dia de hoje. Na arca não havia coisa alguma
senão as duas tábuas, que Moisés tinha posto em Horebe, quando o
Senhor fez aliança com os filhos de Israel, saindo eles do Egito.

E sucedeu que, saindo os sacerdotes do santuário (porque todos os
sacerdotes, que ali se acharam, se santificaram, sem respeitarem as
suas turmas, e os levitas, que eram cantores, todos eles, de
Asafe, de Hemã, de Jedutum, de seus filhos e de seus irmãos,
vestidos de linho fino, com címbalos, com saltérios e com harpas,
estavam em pé para o oriente do altar; e com eles até cento e vinte
sacerdotes, que tocavam as trombetas). E aconteceu que,
quando eles uniformemente tocavam as trombetas, e cantavam, para
fazerem ouvir uma só voz, bendizendo e louvando ao Senhor; e
levantando eles a voz com trombetas, címbalos, e outros instrumentos
musicais, e louvando ao Senhor, dizendo: Porque ele é bom, porque a
sua benignidade dura para sempre, então a casa se encheu de uma
nuvem, a saber, a casa do Senhor; e os sacerdotes não podiam
permanecer em pé, para ministrar, por causa da nuvem; porque a
glória do Senhor encheu a casa de Deus.

\medskip

\lettrine{6} Então falou Salomão: O Senhor disse que habitaria
nas trevas. E eu te tenho edificado uma casa para morada, e um
lugar para a tua eterna habitação. Então o rei virou o seu
rosto, e abençoou a toda a congregação de Israel, e toda a
congregação de Israel estava em pé. E ele disse: Bendito seja o
Senhor Deus de Israel, que falou pela sua boca a Davi meu pai; e
pelas suas mãos o cumpriu, dizendo: Desde o dia em que tirei a
meu povo da terra do Egito, não escolhi cidade alguma de todas as
tribos de Israel, para edificar nela uma casa em que estivesse o meu
nome; nem escolhi homem algum para ser líder do meu povo, Israel.
Porém escolhi a Jerusalém para que ali estivesse o meu nome; e
escolhi a Davi, para que estivesse sobre o meu povo Israel.
Também Davi, meu pai, teve no seu coração o edificar uma casa ao
nome do Senhor Deus de Israel. Porém o Senhor disse a Davi, meu
pai: Porquanto tiveste no teu coração o edificar uma casa ao meu
nome, bem fizeste de ter isto no teu coração. Contudo tu não
edificarás a casa, mas teu filho, que há de proceder de teus lombos,
esse edificará a casa ao meu nome. Assim confirmou o Senhor a
sua palavra, que falou; porque eu me levantei em lugar de Davi meu
pai, e me assentei sobre o trono de Israel, como o Senhor disse, e
edifiquei a casa ao nome do Senhor Deus de Israel. E pus nela
a arca, em que está a aliança que o Senhor fez com os filhos de
Israel.

E pôs-se em pé, perante o altar do Senhor, na presença de toda a
congregação de Israel, e estendeu as suas mãos. Porque
Salomão tinha feito uma plataforma de metal, de cinco côvados de
comprimento, de cinco côvados de largura e de três côvados de
altura, e a tinha posto no meio do pátio, e pôs-se em pé sobre ela,
e ajoelhou-se em presença de toda a congregação de Israel, e
estendeu as suas mãos para o céu. E disse: Ó Senhor Deus de
Israel, não há Deus semelhante a ti, nem nos céus nem na terra; que
guardas a aliança e a beneficência aos teus servos que caminham
perante ti de todo o seu coração. Que guardaste ao teu servo
Davi, meu pai, o que lhe falaste; porque tu pela tua boca o
disseste, e pela tua mão o cumpriste, como se vê neste dia.
Agora, pois, Senhor Deus de Israel, guarda ao teu servo Davi,
meu pai, o que falaste, dizendo: Nunca homem algum será cortado de
diante de mim, que se assente sobre o trono de Israel; tão-somente
que teus filhos guardem seu caminho, andando na minha lei, como tu
andaste diante de mim. E agora, Senhor Deus de Israel,
cumpra-se a tua palavra, que disseste ao teu servo Davi. Mas,
na verdade, habitará Deus com os homens na terra? Eis que os céus, e
o céu dos céus, não te podem conter, quanto menos esta casa que
tenho edificado? Atende, pois, à oração do teu servo, e à sua
súplica, ó Senhor meu Deus; para ouvires o clamor, e a oração, que o
teu servo faz perante ti. Que os teus olhos estejam dia e
noite abertos sobre este lugar, de que disseste que ali porias o teu
nome; para ouvires a oração que o teu servo orar neste lugar.
Ouve, pois, as súplicas do teu servo, e do teu povo Israel,
que fizerem neste lugar; e ouve tu do lugar da tua habitação, desde
os céus; ouve pois, e perdoa. Quando alguém pecar contra o
seu próximo, e lhe impuser juramento de maldição, fazendo-o jurar, e
o juramento de maldição vier perante o teu altar, nesta casa,
ouve tu, então, desde os céus, e age e julga a teus servos,
condenando ao ímpio, retribuindo o seu proceder sobre a sua cabeça;
e justificando ao justo, dando-lhe segundo a sua justiça.
Quando também o teu povo Israel for ferido diante do inimigo,
por ter pecado contra ti, e eles se converterem, e confessarem o teu
nome, e orarem e suplicarem perante ti nesta casa, então,
ouve tu desde os céus, e perdoa os pecados do teu povo Israel; e
torna a levá-los à terra que lhes tens dado e a seus pais.
Quando os céus se fecharem, e não houver chuva, por terem
pecado contra ti, e orarem neste lugar, e confessarem teu nome, e se
converterem dos seus pecados, quando tu os afligires, então,
ouve tu desde os céus, e perdoa o pecado de teus servos, e do teu
povo Israel, ensinando-lhes o bom caminho, em que andem; e dá chuva
sobre a tua terra, que deste ao teu povo em herança. Quando
houver fome na terra, quando houver peste, quando houver queima de
seara, ou ferrugem\footnote{Bot. Doença de gramíneas, esp. do trigo,
aveia e milho, causada por fungos dos gêneros Puccinia, Tillelia e
Ustilago.}, gafanhotos, ou lagarta, cercando-a algum dos seus
inimigos nas terras das suas portas, ou quando houver qualquer
praga, ou qualquer enfermidade, toda a oração, e toda a
súplica, que qualquer homem fizer, ou todo o teu povo Israel,
conhecendo cada um a sua praga, e a sua dor, e estendendo as suas
mãos para esta casa, então, ouve tu desde os céus, do assento
da tua habitação, e perdoa, e dá a cada um conforme a todos os seus
caminhos, segundo conheces o seu coração (pois só tu conheces o
coração dos filhos dos homens), a fim de que te temam, para
andarem nos teus caminhos, todos os dias que viverem na terra que
deste a nossos pais. Assim também ao estrangeiro, que não for
do teu povo Israel, quando vier de terras remotas por amor do teu
grande nome, e da tua poderosa mão, e do teu braço estendido, vindo
eles e orando nesta casa; então, ouve tu desde os céus, do
assento da tua habitação, e faze conforme a tudo o que o estrangeiro
te suplicar; a fim de que todos os povos da terra conheçam o teu
nome, e te temam, como o teu povo Israel; e a fim de saberem que
pelo teu nome é chamada esta casa que edifiquei. Quando o teu
povo sair à guerra contra os seus inimigos, pelo caminho que os
enviares, e orarem a ti para o lado desta cidade que escolheste, e
desta casa, que edifiquei ao teu nome, ouve, então, desde os
céus a sua oração, e a sua súplica, e faze-lhes justiça.
Quando pecarem contra ti (pois não há homem que não peque), e
tu te indignares contra eles, e os entregares diante do inimigo,
para que os que os cativarem os levem em cativeiro para alguma
terra, remota ou vizinha\footnote{``para que os levem cativos a uma
terra, longe ou perto'' (Edição Contemporânea).}, e na terra,
para onde forem levados em cativeiro, caírem em si, e se
converterem, e na terra do seu cativeiro, a ti suplicarem, dizendo:
Pecamos, perversamente procedemos e impiamente agimos; e se
converterem a ti com todo o seu coração e com toda a sua alma, na
terra do seu cativeiro, a que os levaram presos, e orarem para o
lado da sua terra, que deste a seus pais, e para esta cidade que
escolheste, e para esta casa que edifiquei ao teu nome, ouve,
então, desde os céus, do assento da tua habitação, a sua oração e as
suas súplicas, e executa o seu direito; e perdoa ao teu povo que
houver pecado contra ti. Agora, pois, ó meu Deus, estejam os
teus olhos abertos, e os teus ouvidos atentos à oração deste lugar.
Levanta-te, pois, agora, Senhor Deus, para o teu repouso, tu
e a arca da tua fortaleza; os teus sacerdotes, ó Senhor Deus, sejam
vestidos de salvação, e os teus santos se alegrem do bem. Ó
Senhor Deus, não faças virar o rosto do teu ungido; lembra-te das
misericórdias de Davi teu servo.

\medskip

\lettrine{7} E acabando Salomão de orar, desceu o fogo do céu,
e consumiu o holocausto e os sacrifícios; e a glória do Senhor
encheu a casa. E os sacerdotes não podiam entrar na casa do
Senhor, porque a glória do Senhor tinha enchido a casa do Senhor.
E todos os filhos de Israel vendo descer o fogo, e a glória do
Senhor sobre a casa, encurvaram-se com o rosto em terra sobre o
pavimento, e adoraram e louvaram ao Senhor, dizendo: Porque ele é
bom, porque a sua benignidade dura para sempre. E o rei e todo o
povo ofereciam sacrifícios perante o Senhor. E o rei Salomão
ofereceu sacrifícios de bois, vinte e dois mil, e de ovelhas, cento
e vinte mil; e o rei e todo o povo consagraram a casa de Deus. E
os sacerdotes, serviam em seus ofícios; como também os levitas com
os instrumentos musicais do Senhor, que o rei Davi tinha feito, para
louvarem ao Senhor, porque a sua benignidade dura para sempre,
quando Davi o louvava pelo ministério deles; e os sacerdotes tocavam
as trombetas diante deles, e todo o Israel estava em pé. E
Salomão santificou o meio do átrio, que estava diante da casa do
Senhor; porquanto ali tinha ele oferecido os holocaustos e a gordura
dos sacrifícios pacíficos; porque no altar de metal, que Salomão
tinha feito, não podia caber o holocausto, e a oferta de alimentos,
e a gordura. E, assim, naquele mesmo tempo celebrou Salomão a
festa por sete dias e todo o Israel com ele, uma grande congregação,
desde a entrada de Hamate, até ao rio do Egito. E no dia oitavo
realizaram uma assembléia solene; porque sete dias celebraram a
consagração do altar, e sete dias a festa. E no dia vigésimo
terceiro do sétimo mês, despediu o povo para as suas tendas, alegres
e de bom ânimo, pelo bem que o Senhor tinha feito a Davi, e a
Salomão, e a seu povo Israel. Assim Salomão acabou a casa do
Senhor, e a casa do rei, e tudo quanto Salomão intentou fazer na
casa do Senhor e na sua casa prosperamente o efetuou.

E o Senhor apareceu de noite a Salomão, e disse-lhe: Ouvi a tua
oração, e escolhi para mim este lugar para casa de sacrifício.
Se eu fechar os céus, e não houver chuva; ou se ordenar aos
gafanhotos que consumam a terra; ou se enviar a peste entre o meu
povo; e se o meu povo, que se chama pelo meu nome, se
humilhar, e orar, e buscar a minha face e se converter dos seus maus
caminhos, então eu ouvirei dos céus, e perdoarei os seus pecados, e
sararei a sua terra. Agora estarão abertos os meus olhos e
atentos os meus ouvidos à oração deste lugar. Porque agora
escolhi e santifiquei esta casa, para que o meu nome esteja nela
perpetuamente; e nela estarão fixos os meus olhos e o meu coração
todos os dias. E, quanto a ti, se andares diante de mim, como
andou Davi teu pai, e fizeres conforme a tudo o que te ordenei, e
guardares os meus estatutos e os meus juízos, também
confirmarei o trono do teu reino, conforme a aliança que fiz com
Davi, teu pai, dizendo: Não te faltará sucessor que domine em
Israel. Porém se vós vos desviardes, e deixardes os meus
estatutos, e os meus mandamentos, que vos tenho proposto, e fordes,
e servirdes a outros deuses, e vos prostrardes a eles, então
os arrancarei da minha terra que lhes dei, e lançarei da minha
presença esta casa que consagrei ao meu nome, e farei com que seja
por provérbio e motejo entre todos os povos. E desta casa,
que é tão exaltada, qualquer que passar por ela se espantará e dirá:
Por que fez o Senhor assim com esta terra e com esta casa? E
dirão: Porque deixaram ao Senhor Deus de seus pais, que os tirou da
terra do Egito, e se deram a outros deuses, e se prostraram a eles,
e os serviram; por isso ele trouxe sobre eles todo este mal.


\medskip

\lettrine{8} E sucedeu, ao fim de vinte anos, nos quais
Salomão edificou a casa do Senhor, e a sua própria casa, que
Salomão edificou as cidades que Hirão lhe tinha dado; e fez habitar
nelas os filhos de Israel. Depois foi Salomão a Hamate-Zobá, e a
tomou. Também edificou a Tadmor no deserto, e todas as cidades
de provisões, que edificou em Hamate. Edificou também a alta
Bete-Horom, e a baixa Bete-Horom; cidades fortes, com muros, portas
e ferrolhos; como também a Baalate, e todas as cidades de
provisões, que Salomão tinha, e todas as cidades dos carros e as
cidades dos cavaleiros; e tudo quanto, conforme ao seu desejo,
Salomão quis edificar em Jerusalém, e no Líbano, e em toda a terra
do seu domínio. Quanto a todo o povo, que tinha ficado dos
heteus, amorreus, perizeus, heveus e jebuseus, que não eram de
Israel, dos seus filhos, que ficaram depois deles na terra, os
quais os filhos de Israel não destruíram, Salomão os fez
tributários, até ao dia de hoje. Porém, dos filhos de Israel,
Salomão não fez servos para sua obra (mas eram homens de guerra,
chefes dos seus capitães, e capitães dos seus carros e cavaleiros),
destes, pois, eram os chefes dos oficiais que o rei Salomão
tinha, duzentos e cinqüenta, que presidiam sobre o povo. E
Salomão fez subir a filha de Faraó da cidade de Davi para a casa que
lhe tinha edificado; porque disse: Minha mulher não morará na casa
de Davi, rei de Israel, porquanto santos são os lugares nos quais
entrou a arca do Senhor.

Então Salomão ofereceu holocaustos ao Senhor, sobre o altar do
Senhor, que tinha edificado diante do pórtico, e isto segundo
a ordem de cada dia, fazendo ofertas conforme o mandamento de
Moisés, nos sábados e nas luas novas, e nas solenidades, três vezes
no ano; na festa dos pães ázimos, na festa das semanas, e na festa
das tendas. Também, conforme à ordem de Davi seu pai,
designou as turmas dos sacerdotes para seus ministérios, como também
as dos levitas acerca dos seus cargos, para louvarem e ministrarem
diante dos sacerdotes, segundo o que estava ordenado para cada dia,
e os porteiros pelas suas turmas a cada porta; porque assim tinha
mandado Davi, o homem de Deus. E não se desviaram do mandado
do rei aos sacerdotes e levitas, em negócio nenhum, nem acerca dos
tesouros. Assim se preparou toda a obra de Salomão, desde o
dia da fundação da casa do Senhor, até se acabar; e assim se
concluiu a casa do Senhor. Então foi Salomão a Eziom-Geber, e
a Elote, à praia do mar, na terra de Edom. E enviou-lhe
Hirão, por meio de seus servos, navios, e servos
práticos\footnote{Homem experimentado. Náut. Homem que conhece
minuciosamente os acidentes hidrográficos de áreas restritas, e que
com esses conhecimentos conduz embarcação através dessas áreas.} do
mar, e foram com os servos de Salomão a Ofir, e tomaram de lá
quatrocentos e cinqüenta talentos de ouro; e os trouxeram ao rei
Salomão.

\medskip

\lettrine{9} E ouvindo a rainha de Sabá a fama de Salomão,
veio a Jerusalém, para prová-lo com questões difíceis, com um grande
séquito, e com camelos carregados de especiarias; ouro em abundância
e pedras preciosas; e foi a Salomão, e falou com ele de tudo o que
tinha no seu coração. E Salomão lhe respondeu a todas as suas
questões; e não houve nada que não lhe pudesse esclarecer.
Vendo, pois, a rainha de Sabá a sabedoria de Salomão, e a casa
que edificara; e as iguarias da sua mesa, o assentar dos seus
servos, o estar dos seus criados, e as vestes deles; e os seus
copeiros e as vestes deles; e a sua subida pela qual ele chegava à
casa do Senhor, ela ficou como fora de si. Então disse ao rei:
Era verdade a palavra que ouvi na minha terra acerca dos teus feitos
e da tua sabedoria. Porém não cria naquelas palavras, até que
vim, e meus olhos o viram, e eis que não me disseram a metade da
grandeza da tua sabedoria; sobrepujaste a fama que ouvi.
Bem-aventurados os teus homens, e bem-aventurados estes teus
servos, que estão sempre diante de ti, e ouvem a tua sabedoria!
Bendito seja o Senhor teu Deus, que se agradou de ti para te
colocar no seu trono como rei para o Senhor teu Deus; porque teu
Deus ama a Israel, para estabelecê-lo perpetuamente; por isso te
constituiu rei sobre eles para fazeres juízo e justiça. E deu ao
rei cento e vinte talentos de ouro, e especiarias em grande
abundância, e pedras preciosas; e nunca houve tais especiarias,
quais a rainha de Sabá deu ao rei Salomão. E também os servos
de Hirão e os servos de Salomão, que de Ofir tinham trazido ouro,
trouxeram madeira de algumins, e pedras preciosas. E, da
madeira de algumins, o rei fez balaústres, para a casa do Senhor, e
para a casa do rei, como também harpas e saltérios para os cantores,
quais nunca dantes se viram na terra de Judá. E o rei Salomão
deu à rainha de Sabá tudo quanto ela desejou, e tudo quanto lhe
pediu, mais do que ela mesma trouxera ao rei. Assim voltou e foi
para a sua terra, ela e os seus servos.

E o peso do ouro, que vinha em um ano a Salomão, era de
seiscentos e sessenta e seis talentos de ouro, afora o que os
negociantes e mercadores traziam; também todos os reis da Arábia, e
os governadores da mesma terra traziam a Salomão ouro e prata.
Também fez o rei Salomão duzentos paveses de ouro batido;
para cada pavês destinou seiscentos siclos de ouro batido.
Como também trezentos escudos de ouro batido; para cada
escudo destinou trezentos siclos de ouro; e Salomão os pôs na casa
do bosque do Líbano. Fez mais o rei um grande trono de
marfim, e o revestiu de ouro puro. E o trono tinha seis
degraus, e um estrado de ouro, que eram ligados ao trono, e encostos
de ambos os lados no lugar do assento; e dois leões estavam junto
aos encostos. E doze leões estavam ali de ambos os lados,
sobre os seis degraus; outro tal não se fez em nenhum reino.
Também todas as taças do rei Salomão eram de ouro, e todos os
vasos da casa do bosque do Líbano, de ouro puro; a prata reputava-se
por nada nos dias de Salomão. Porque, indo os navios do rei
com os servos de Hirão, a Társis, voltavam os navios de Társis, uma
vez em três anos, e traziam ouro e prata, marfim, bugios e pavões.
Assim excedeu o rei Salomão a todos os reis da terra, em
riquezas e sabedoria. E todos os reis da terra buscavam a
presença de Salomão, para ouvirem a sabedoria que Deus tinha posto
no seu coração. E cada um trazia o seu presente, vasos de
prata, e vasos de ouro, e roupas, armaduras, especiarias, cavalos e
mulas; assim faziam de ano em ano. Teve também Salomão quatro
mil estrebarias para os cavalos de seus carros, e doze mil
cavaleiros; e colocou-os nas cidades dos carros, e junto ao rei em
Jerusalém. E dominava sobre todos os reis, desde o rio até à
terra dos filisteus, e até ao termo do Egito. Também o rei
fez que houvesse prata em Jerusalém como pedras, e cedros em tanta
abundância como os sicômoros que há pelas campinas. E do
Egito e de todas aquelas terras traziam cavalos a Salomão. Os
demais atos de Salomão, tanto os primeiros como os últimos,
porventura não estão escritos no livro das crônicas de Natã, o
profeta, e na profecia de Aías, o silonita, e nas visões de Ido, o
vidente, acerca de Jeroboão, filho de Nebate? E reinou
Salomão em Jerusalém quarenta anos sobre todo o Israel. E
dormiu Salomão com seus pais, e o sepultaram na cidade de Davi seu
pai; e Roboão, seu filho, reinou em seu lugar.

\medskip

\lettrine{10} E foi Roboão a Siquém, porque todo o Israel se
reunira ali, para fazê-lo rei. Sucedeu que, ouvindo-o Jeroboão,
filho de Nebate (o qual estava então no Egito para onde fugira da
presença do rei Salomão), voltou do Egito, porque enviaram a
ele, e o chamaram; e vieram, Jeroboão e todo o Israel, e falaram a
Roboão dizendo: Teu pai fez duro o nosso jugo; agora, pois,
alivia tu a dura servidão de teu pai, e o pesado jugo que nos impôs,
e nós te serviremos. E ele lhes disse: Daqui a três dias voltai
a mim. Então o povo se foi. E tomou Roboão conselho com os
anciãos, que estiveram perante Salomão seu pai, enquanto viveu,
dizendo: Como aconselhais vós que se responda a este povo? E
eles lhe falaram, dizendo: Se te fizeres benigno e afável para com
este povo, e lhes falares boas palavras, todos os dias serão teus
servos. Porém ele deixou o conselho que os anciãos lhe deram; e
tomou conselho com os jovens, que haviam crescido com ele, e estavam
perante ele. E disse-lhes: Que aconselhais vós, que respondamos
a este povo, que me falou, dizendo: Alivia o jugo que teu pai nos
impôs? E os jovens, que com ele haviam crescido, lhe falaram,
dizendo: Assim dirás a este povo, que te falou: Teu pai agravou o
nosso jugo, tu porém alivia-nos; assim, pois, lhe falarás: O meu
dedo mínimo é mais grosso do que os lombos de meu pai. Assim
que, se meu pai vos carregou de um jugo pesado, eu ainda aumentarei
o vosso jugo; meu pai vos castigou com açoites, porém eu vos
castigarei com escorpiões.

Veio, pois, Jeroboão, e todo o povo, ao terceiro dia, a Roboão,
como o rei havia ordenado, dizendo: Voltai a mim ao terceiro dia.
E o rei lhes respondeu asperamente; porque o rei Roboão
deixara o conselho dos anciãos. E falou-lhes conforme o
conselho dos jovens, dizendo: Meu pai agravou o vosso jugo, porém eu
o aumentarei mais; meu pai vos castigou com açoites, porém eu vos
castigarei com escorpiões. Assim o rei não deu ouvidos ao
povo, porque esta mudança vinha de Deus, para que o Senhor
confirmasse a sua palavra, a qual falara pelo ministério de Aías, o
silonita, a Jeroboão, filho de Nebate. Vendo, pois, todo o
Israel, que o rei não lhe dava ouvidos, tornou-lhe o povo a
responder, dizendo: Que parte temos nós com Davi? Já não temos
herança no filho de Jessé. Cada um à sua tenda, ó Israel! Olha agora
pela tua casa, ó Davi. Assim todo o Israel se foi para as suas
tendas. Porém, quanto aos filhos de Israel, que habitavam nas
cidades de Judá, sobre eles reinou Roboão. Então o rei Roboão
enviou a Hadorão, que tinha cargo dos tributos; porém os filhos de
Israel o apedrejaram, e ele morreu. Então o rei Roboão se esforçou
para subir ao seu carro, e fugiu para Jerusalém. Assim se
rebelaram os israelitas contra a casa de Davi, até ao dia de hoje.

\medskip

\lettrine{11} Vindo, pois, Roboão a Jerusalém, reuniu, da casa
de Judá e Benjamim, cento e oitenta mil escolhidos, destros na
guerra, para pelejarem contra Israel, e para restituírem o reino a
Roboão. Porém a palavra do Senhor veio a Semaías, homem de Deus,
dizendo: Fala a Roboão, filho de Salomão, rei de Judá, e a todo
o Israel, em Judá e Benjamim, dizendo: Assim diz o Senhor: Não
subireis, nem pelejareis contra os vossos irmãos; volte cada um à
sua casa; porque de mim proveio isto. E ouviram as palavras do
Senhor, e desistiram de ir contra Jeroboão. E Roboão habitou em
Jerusalém; e para defesa, edificou cidades em Judá. Edificou,
pois, a Belém, a Etã, e a Tecoa, e a Bete-Zur, a Socó, a Adulão,
e a Gate, a Maressa, a Zife, e a Adoraim, a Laquis, e a
Azeca, e a Zorá, a Aijalom, e a Hebrom, que estavam em Judá e
em Benjamim; cidades fortes. E fortificou estas fortalezas e
pôs nelas capitães, e armazéns de víveres, de azeite, e de vinho.
E pôs em cada cidade paveses e lanças; fortificou-as
grandemente; e Judá e Benjamim pertenceram-lhe.

Também os sacerdotes e os levitas, que havia em todo o Israel, se
reuniram a ele de todos os seus termos. Porque os levitas
deixaram os seus arrabaldes, e a sua possessão, e vieram a Judá e a
Jerusalém, porque Jeroboão e seus filhos os lançaram fora para que
não ministrassem ao Senhor. E ele constituiu para si
sacerdotes, para os altos, para os demônios, e para os bezerros, que
fizera. Depois desses também, de todas as tribos de Israel,
os que deram o seu coração a buscarem ao Senhor Deus de Israel,
vieram a Jerusalém, para oferecerem sacrifícios ao Senhor Deus de
seus pais. Assim fortaleceram o reino de Judá e corroboraram
a Roboão, filho de Salomão, por três anos; porque três anos andaram
no caminho de Davi e Salomão. E Roboão tomou para si, por
mulher, a Maalate, filha de Jerimote, filho de Davi; e Abiail, filha
de Eliabe, filho de Jessé. A qual lhe deu filhos: Jeús,
Samarias e Zaã. E depois dela tomou a Maaca, filha de
Absalão; esta lhe deu Abias, Atai, Ziza e Selomite. E amava
Roboão mais a Maaca, filha de Absalão, do que a todas as suas outras
mulheres e concubinas; porque ele tinha tomado dezoito mulheres, e
sessenta concubinas; e gerou vinte e oito filhos, e sessenta filhas.
E Roboão designou Abias, filho de Maaca, para ser chefe e
líder entre os seus irmãos, porque queria fazê-lo rei. E usou
de prudência e, de todos os seus filhos, alguns espalhou por todas
as terras de Judá, e Benjamim, por todas as cidades fortes; e
deu-lhes víveres em abundância; e lhes procurou muitas mulheres.

\medskip

\lettrine{12} Sucedeu que, havendo Roboão confirmado o reino,
e havendo-se fortalecido, deixou a lei do Senhor, e com ele todo o
Israel. E sucedeu que, no quinto ano do rei Roboão, Sisaque, rei
do Egito, subiu contra Jerusalém (porque tinham transgredido contra
o Senhor) com mil e duzentos carros e com sessenta mil
cavaleiros; e era inumerável o povo que vinha com ele do Egito, de
líbios, suquitas e etíopes. E tomou as cidades fortificadas, que
Judá tinha; e chegou até Jerusalém. Então veio Semaías, o
profeta, a Roboão e aos príncipes de Judá que se ajuntaram em
Jerusalém por causa de Sisaque, e disse-lhes: Assim diz o Senhor:
Vós me deixastes a mim, por isso também eu vos deixei na mão de
Sisaque. Então se humilharam os príncipes de Israel, e o rei, e
disseram: O Senhor é justo. Vendo, pois, o Senhor que se
humilhavam, veio a palavra do Senhor a Semaías, dizendo:
Humilharam-se, não os destruirei; antes em breve lhes darei algum
socorro, para que o meu furor não se derrame sobre Jerusalém, por
mão de Sisaque. Porém serão seus servos; para que conheçam a
diferença da minha servidão e da servidão dos reinos da terra.
Subiu, pois, Sisaque, rei do Egito, contra Jerusalém, e tomou os
tesouros da casa do Senhor, e os tesouros da casa do rei; levou
tudo; também tomou os escudos de ouro, que Salomão fizera. E
fez o rei Roboão em lugar deles escudos de cobre, e os entregou na
mão dos chefes da guarda, que guardavam a porta da casa do rei.
E todas as vezes que o rei entrava na casa do Senhor, vinham
os da guarda, e os levavam; depois tornavam a pô-los na câmara da
guarda. E humilhando-se ele, a ira do Senhor se desviou dele,
para que não o destruísse de todo; porque em Judá ainda havia boas
coisas.

Fortificou-se, pois, o rei Roboão em Jerusalém, e reinou; porque
Roboão era da idade de quarenta e um anos, quando começou a reinar;
e reinou dezessete anos em Jerusalém, a cidade que o Senhor
escolheu, dentre todas as tribos de Israel, para pôr ali o seu nome;
e era o nome de sua mãe Naamá, amonita. E fez o que era mau;
porquanto não preparou o seu coração para buscar ao Senhor.
Os atos, pois, de Roboão, assim os primeiros, como os
últimos, porventura não estão escritos nos livros de Semaías, o
profeta, e de Ido, o vidente, na relação das genealogias? E houve
guerras entre Roboão e Jeroboão em todos os seus dias. E
Roboão dormiu com seus pais, e foi sepultado na cidade de Davi; e
Abias, seu filho, reinou em seu lugar.

\medskip

\lettrine{13} No ano décimo oitavo do rei Jeroboão, Abias
começou a reinar sobre Judá. Três anos reinou em Jerusalém; e
era o nome de sua mãe Micaía, filha de Uriel de Gibeá; e houve
guerra entre Abias e Jeroboão. E Abias ordenou a peleja com um
exército de valentes guerreiros, quatrocentos mil homens escolhidos;
e Jeroboão dispôs contra ele a batalha com oitocentos mil homens
escolhidos, todos homens corajosos. E pôs-se Abias em pé em cima
do monte de Zemaraim, que está na montanha de Efraim, e disse:
Ouvi-me, Jeroboão e todo o Israel: Porventura não vos convém
saber que o Senhor Deus de Israel deu para sempre a Davi a soberania
sobre Israel, a ele e a seus filhos, por uma aliança de sal?
Contudo levantou-se Jeroboão, filho de Nebate, servo de Salomão,
filho de Davi, e se rebelou contra seu senhor. E ajuntaram-se a
ele homens vadios, filhos de Belial; e fortificaram-se contra
Roboão, filho de Salomão, sendo Roboão ainda jovem, e terno de
coração, e não lhes podia resistir. E agora julgais que podeis
resistir ao reino do Senhor, que está na mão dos filhos de Davi,
visto que sois uma grande multidão, e tendes convosco os bezerros de
ouro que Jeroboão vos fez para deuses. Não lançastes vós fora os
sacerdotes do Senhor, os filhos de Arão, e os levitas, e não
fizestes para vós sacerdotes, como os povos das outras terras?
Qualquer que vem a consagrar-se com um novilho e sete carneiros logo
se faz sacerdote daqueles que não são deuses. Porém, quanto a
nós, o Senhor é nosso Deus, e nunca o deixamos; e os sacerdotes que
ministram ao Senhor são filhos de Arão, e os levitas se ocupam na
sua obra. E queimam ao Senhor cada manhã e cada tarde
holocaustos, incenso aromático, com os pães da proposição sobre a
mesa pura, e o candelabro de ouro, e as suas lâmpadas para se
acenderem cada tarde, porque nós temos cuidado do serviço do Senhor
nosso Deus; porém vós o deixastes. E eis que Deus está
conosco, à nossa frente, como também os seus sacerdotes, tocando com
as trombetas, para dar alarme contra vós. Ó filhos de Israel, não
pelejeis contra o Senhor Deus de vossos pais; porque não
prosperareis.

Mas Jeroboão armou uma emboscada, para dar sobre eles pela
retaguarda; de maneira que estavam em frente de Judá e a emboscada
por detrás deles. Então Judá olhou, e eis que tinham que
pelejar por diante e por detrás; então clamaram ao Senhor; e os
sacerdotes tocaram as trombetas. E os homens de Judá
gritaram; e sucedeu que, gritando os homens de Judá, Deus feriu a
Jeroboão e a todo o Israel diante de Abias e de Judá. E os
filhos de Israel fugiram de diante de Judá; e Deus os entregou na
sua mão.
 De maneira que Abias e o seu povo fizeram grande matança entre
eles; porque caíram feridos de Israel quinhentos mil homens
escolhidos. E foram humilhados os filhos de Israel naquele
tempo; e os filhos de Judá prevaleceram, porque confiaram no Senhor
Deus de seus pais. E Abias perseguiu Jeroboão; e tomou-lhe a
Betel com os lugares da sua jurisdição, e a Jesana com os lugares da
sua jurisdição, e a Efrom com os lugares da sua jurisdição. E
Jeroboão não recobrou mais o seu poder nos dias de Abias; porém o
Senhor o feriu, e morreu. Abias, porém, se fortificou, e
tomou para si catorze mulheres, e gerou vinte e dois filhos e
dezesseis filhas. Os demais atos de Abias, tanto os seus
caminhos como as suas palavras, estão escritos na história do
profeta Ido.

\medskip

\lettrine{14} E Abias dormiu com seus pais, e o sepultaram na
cidade de Davi, e Asa, seu filho, reinou em seu lugar; nos seus dias
esteve a terra em paz dez anos. E Asa fez o que era bom e reto
aos olhos do Senhor seu Deus. Porque tirou os altares dos deuses
estranhos, e os altos; e quebrou as imagens, e cortou os bosques.
E mandou a Judá que buscasse ao Senhor Deus de seus pais, e que
observasse a lei e o mandamento. Também tirou de todas as
cidades de Judá os altos e as imagens; e sob ele o reino esteve em
paz. E edificou cidades fortificadas em Judá; porque a terra
estava quieta, e não havia guerra contra ele naqueles anos;
porquanto o Senhor lhe dera repouso. Disse, pois, a Judá:
Edifiquemos estas cidades, e cerquemo-las de muros e torres, portas
e ferrolhos, enquanto a terra ainda é nossa, pois buscamos ao Senhor
nosso Deus; buscamo-lo, e deu-nos repouso de todos os lados.
Edificaram, pois, e prosperaram. Tinha Asa um exército de
trezentos mil de Judá, que traziam pavês e lança; e duzentos e
oitenta mil de Benjamim, que traziam escudo e atiravam com arco;
todos estes eram homens valentes.

E Zerá, o etíope, saiu contra eles, com um exército de um milhão e
com trezentos carros, e chegou até Maressa. Então Asa saiu
contra ele; e ordenaram a batalha no vale de Zefatá, junto a
Maressa. E Asa clamou ao Senhor seu Deus, e disse: Senhor,
nada para ti é ajudar, quer o poderoso quer o de nenhuma força;
ajuda-nos, pois, Senhor nosso Deus, porque em ti confiamos, e no teu
nome viemos contra esta multidão. Senhor, tu és nosso Deus, não
prevaleça contra ti o homem. E o Senhor feriu os etíopes
diante de Asa e diante de Judá; e os etíopes fugiram. E Asa,
e o povo que estava com ele os perseguiram até Gerar, e caíram
tantos dos etíopes, que já não havia neles resistência alguma;
porque foram destruídos diante do Senhor, e diante do seu exército;
e levaram dali mui grande despojo. E feriram todas as cidades
nos arredores de Gerar, porque o terror do Senhor veio sobre elas; e
saquearam todas as cidades, porque havia nelas muita presa.
Também feriram as malhadas do gado\footnote{RC-1969:
``malhadas do gado''. RA: Também feriram as tendas dos donos do
gado, levaram ovelhas em abundância e camelos e voltaram para
Jerusalém. AV: They smote also the tents of cattle, and carried away
sheep and camels in abundance, and returned to Jerusalem. Edição
Contemporânea: ``acampamento dos pastores''.}; e levaram ovelhas em
abundância, e camelos, e voltaram para Jerusalém.

\medskip

\lettrine{15} Então veio o Espírito de Deus sobre Azarias,
filho de Odede. E saiu ao encontro de Asa, e disse-lhe: Ouvi-me,
Asa, e todo o Judá e Benjamim: O Senhor está convosco, enquanto vós
estais com ele, e, se o buscardes, o achareis; porém, se o
deixardes, vos deixará. E Israel esteve por muitos dias sem o
verdadeiro Deus, e sem sacerdote que o ensinasse, e sem lei. Mas
quando na sua angústia voltaram para o Senhor Deus de Israel, e o
buscaram, o acharam. E naqueles tempos não havia paz, nem para o
que saía, nem para o que entrava, mas muitas perturbações sobre
todos os habitantes daquelas terras. Porque nação contra nação e
cidade contra cidade se despedaçavam; porque Deus os perturbara com
toda a angústia. Mas esforçai-vos, e não desfaleçam as vossas
mãos; porque a vossa obra tem uma recompensa.

Ouvindo, pois, Asa estas palavras, e a profecia do profeta Odede,
cobrou ânimo e tirou as abominações de toda a terra, de Judá e de
Benjamim, como também das cidades que tomara nas montanhas de
Efraim, e renovou o altar do Senhor, que estava diante do pórtico do
Senhor. E reuniu a todo o Judá, e Benjamim, e com eles os
estrangeiros de Efraim e Manassés, e de Simeão; porque muitos de
Israel tinham passado a ele, vendo que o Senhor seu Deus era com
ele. E ajuntaram-se em Jerusalém no terceiro mês; no ano
décimo do reinado de Asa. E no mesmo dia ofereceram em
sacrifício ao Senhor, do despojo que trouxeram, setecentos bois e
sete mil ovelhas. E entraram na aliança para buscarem o
Senhor Deus de seus pais, com todo o seu coração, e com toda a sua
alma; e de que todo aquele que não buscasse ao Senhor Deus de
Israel, morresse; assim o menor como o maior, tanto o homem como a
mulher. E juraram ao Senhor, em alta voz, com júbilo e com
trombetas e buzinas. E todo o Judá se alegrou deste
juramento; porque de todo o seu coração juraram, e de toda a sua
vontade o buscaram, e o acharam; e o Senhor lhes deu repouso ao
redor. E também a Maaca, sua mãe, o rei Asa depôs, para que
não fosse mais rainha, porquanto fizera um horrível ídolo, a Asera;
e Asa destruiu o seu horrível ídolo, e o despedaçou, e o queimou
junto ao ribeiro de Cedrom. Os altos, porém, não foram
tirados de Israel; contudo o coração de Asa foi perfeito todos os
seus dias. E trouxe, à casa de Deus, as coisas consagradas
por seu pai, e as coisas que ele mesmo tinha consagrado: prata, ouro
e vasos. E não houve guerra até ao ano trigésimo quinto do
reinado de Asa.

\medskip

\lettrine{16} No trigésimo sexto ano do reinado de Asa, Baasa,
rei de Israel, subiu contra Judá e edificou a Ramá, para não deixar
ninguém sair, nem chegar a Asa, rei de Judá. Então Asa tirou a
prata e o ouro dos tesouros da casa do Senhor, e da casa do rei; e
enviou servos a Ben-Hadade, rei da Síria, que habitava em Damasco,
dizendo: Acordo há entre mim e ti, como houve entre meu pai e o
teu; eis que te envio prata e ouro; vai, pois, e anula o teu acordo
com Baasa, rei de Israel, para que se retire de sobre mim. E
Ben-Hadade deu ouvidos ao rei Asa, e enviou os capitães dos seus
exércitos, contra as cidades de Israel, e eles feriram a Ijom, a Dã,
a Abel-Maim, e a todas as cidades-armazéns de Naftali. E sucedeu
que, ouvindo-o Baasa, deixou de edificar a Ramá, e não continuou a
sua obra. Então o rei Asa tomou a todo o Judá, e levaram as
pedras de Ramá, e a sua madeira, com que Baasa edificara; e com elas
edificou a Geba e a Mizpá.

Naquele mesmo tempo veio Hanani, o vidente, a Asa, rei de Judá, e
disse-lhe: Porquanto confiaste no rei da Síria, e não confiaste no
Senhor teu Deus, por isso o exército do rei da Síria escapou da tua
mão. Porventura não foram os etíopes e os líbios um grande
exército, com muitíssimos carros e cavaleiros? Confiando tu, porém,
no Senhor, ele os entregou nas tuas mãos. Porque, quanto ao
Senhor, seus olhos passam por toda a terra, para mostrar-se forte
para com aqueles cujo coração é perfeito para com ele; nisto, pois,
procedeste loucamente porque desde agora haverá guerras contra ti.
Porém Asa se indignou contra o vidente, e lançou-o na casa do
tronco; porque estava enfurecido contra ele, por causa disto; também
Asa, no mesmo tempo, oprimiu a alguns do povo. E eis que os
atos de Asa, tanto os primeiros, como os últimos, estão escritos no
livro dos reis de Judá e Israel. E, no ano trinta e nove do
seu reinado, Asa caiu doente de seus pés, a sua doença era em
extremo grave; contudo, na sua enfermidade, não buscou ao Senhor,
mas antes os médicos. E Asa dormiu com seus pais; e morreu no
ano quarenta e um do seu reinado. E o sepultaram no seu
sepulcro, que tinha cavado para si na cidade de Davi, havendo-o
deitado na cama, que se enchera de perfumes e especiarias preparadas
segundo a arte dos perfumistas; e, destas coisas fizeram-lhe uma
grande queima.

\medskip

\lettrine{17} E Jeosafá, seu filho, reinou em seu lugar; e
fortificou-se contra Israel. E pôs soldados em todas as cidades
fortificadas de Judá, e estabeleceu guarnições na terra de Judá,
como também nas cidades de Efraim, que Asa seu pai tinha tomado.
E o Senhor era com Jeosafá; porque andou nos primeiros caminhos
de Davi seu pai, e não buscou a Baalins. Antes buscou ao Deus de
seu pai, andou nos seus mandamentos, e não segundo as obras de
Israel. E o Senhor confirmou o reino na sua mão, e todo o Judá
deu presentes a Jeosafá, o qual teve riquezas e glória em
abundância. E exaltou-se o seu coração nos caminhos do Senhor e,
ainda mais, tirou os altos e os bosques de Judá. E no terceiro
ano do seu reinado enviou ele os seus príncipes, a Bene-Hail, a
Obadias, a Zacarias, a Natanael e a Micaías, para ensinarem nas
cidades de Judá. E com eles os levitas, Semaías, Netanias,
Zebadias, Asael, Semiramote, Jônatas, Adonias, Tobias e Tobe-Adonias
e, com estes levitas, os sacerdotes, Elisama e Jeorão. E
ensinaram em Judá, levando consigo o livro da lei do Senhor; e foram
a todas as cidades de Judá, ensinando entre o povo.

E veio o temor do Senhor sobre todos os reinos das terras, que
estavam ao redor de Judá, e não guerrearam contra Jeosafá. E
alguns dentre os filisteus traziam presentes a Jeosafá, e prata como
tributo; também os árabes lhe trouxeram gado miúdo; sete mil e
setecentos carneiros, e sete mil e setecentos bodes. Cresceu,
pois, Jeosafá grandemente em extremo e edificou fortalezas e cidades
de provisões em Judá. E teve muitas obras nas cidades de
Judá, e homens de guerra e valentes, em Jerusalém. E este é o
número deles segundo as suas casas paternas; em Judá eram capitães
dos milhares: o chefe Adna, e com ele trezentos mil homens valentes;
e depois dele o capitão Joanã, e com ele duzentos e oitenta
mil; e depois Amasias, filho de Zicri, que voluntariamente se
entregou ao Senhor, e com ele duzentos mil homens valentes; e
de Benjamim, Eliada, homem valente, e com ele duzentos mil, armados
de arco e de escudo; e depois dele Jozabade, e com ele cento
e oitenta mil, armados para a guerra. Estes estavam no
serviço do rei; afora os que o rei tinha posto nas cidades fortes
por todo o Judá.

\medskip

\lettrine{18} Tinha, pois, Jeosafá riquezas e glória em
abundância, e aparentou-se com Acabe. E depois de alguns anos
desceu ele para Acabe em Samaria; e Acabe matou ovelhas e bois em
abundância, para ele e para o povo que vinha com ele; e o persuadiu
a subir com ele a Ramote de Gileade. Porque Acabe, rei de
Israel, disse a Jeosafá, rei de Judá: Irás tu comigo a Ramote de
Gileade? E ele lhe disse: Como tu és, serei eu; e o meu povo, como o
teu povo; iremos contigo à guerra.

Disse mais Jeosafá ao rei de Israel: Peço-te, consulta hoje a
palavra do Senhor. Então o rei de Israel reuniu os profetas,
quatrocentos homens, e disse-lhes: Iremos à guerra contra Ramote de
Gileade, ou deixarei de ir? E eles disseram: Sobe; porque Deus a
entregará na mão do rei. Disse, porém, Jeosafá: Não há ainda
aqui algum profeta do Senhor, para que o consultemos? Então o
rei de Israel disse a Jeosafá: Ainda há um homem por quem podemos
consultar ao Senhor; porém eu o odeio, porque nunca profetiza de mim
o que é bom, senão sempre o mal; este é Micaías, filho de Inlá. E
disse Jeosafá: Não fale o rei assim. Então o rei de Israel
chamou um oficial, e disse: Traze aqui depressa a Micaías, filho de
Inlá. E o rei de Israel, e Jeosafá, rei de Judá, estavam
assentados cada um no seu trono, vestidos com suas roupas reais, e
estavam assentados na praça à entrada da porta de Samaria; e todos
os profetas profetizavam na sua presença. E Zedequias, filho
de Quenaaná, fez para si uns chifres de ferro, e disse: Assim diz o
Senhor: Com estes ferirás aos sírios, até de todo os consumires.
E todos os profetas profetizavam o mesmo, dizendo: Sobe a
Ramote de Gileade, e triunfarás; porque o Senhor a dará na mão do
rei. E o mensageiro, que foi chamar a Micaías, falou-lhe,
dizendo: Eis que as palavras dos profetas, a uma voz, predizem
coisas boas para o rei; seja, pois, também a tua palavra como a de
um deles, e fala o que é bom. Porém Micaías disse: Vive o
Senhor, que o que meu Deus me disser, isso falarei. Vindo,
pois, ele ao rei, este lhe disse: Micaías, iremos a Ramote de
Gileade à guerra, ou deixaremos de ir? E ele disse: Subi, e
triunfarás; e serão dados na vossa mão. E o rei lhe disse:
Até quantas vezes, te conjurarei, para que não me fales senão a
verdade em nome do Senhor? Então disse ele: Vi a todo o
Israel disperso pelos montes, como ovelhas que não têm pastor; e
disse o Senhor: Estes não têm senhor; torne cada um em paz para sua
casa. Então o rei de Israel disse a Jeosafá: Não te disse eu,
que ele não profetizaria de mim o que é bom, porém sempre o mal?
Disse mais: Ouvi, pois, a palavra do Senhor: Vi o
Senhor\footnote{SBTB: ``Vi \emph{ao} Senhor'': português antigo,
influência espanhola. Melhor: ``Vi \textbf{o} Senhor''.} assentado
no seu trono, e todo o exército celestial em pé à sua mão direita, e
à sua esquerda. E disse o Senhor: Quem persuadirá a Acabe rei
de Israel, para que suba, e caia em Ramote de Gileade? Um dizia
desta maneira, e outro de outra. Então saiu um espírito e se
apresentou diante do Senhor, e disse: Eu o persuadirei. E o Senhor
lhe disse: Com quê? E ele disse: Eu sairei, e serei um
espírito de mentira na boca de todos os seus profetas. E disse o
Senhor: Tu o persuadirás, e ainda prevalecerás; sai, e faze-o assim.
Agora, pois, eis que o Senhor pôs um espírito de mentira na
boca destes teus profetas; e o Senhor falou o mal a teu respeito.
Então Zedequias, filho de Quenaaná, chegando-se, feriu a
Micaías no queixo, e disse: Por que caminho passou de mim o Espírito
do Senhor para falar a ti? E disse Micaías: Eis que o verás
naquele dia, quando andares de câmara em câmara, para te esconderes.
 Então disse o rei de Israel: Tomai a Micaías, e tornai a levá-lo
a Amom, o governador da cidade, e a Joás, filho do rei. E
direis: Assim diz o rei: Colocai este homem na casa do cárcere; e
sustentai-o com pão de angústia, e com água de angústia, até que eu
volte em paz. E disse Micaías: Se voltares em paz, o Senhor
não tem falado por mim. Disse mais: Ouvi, povos todos!

Subiram, pois, o rei de Israel e Jeosafá, rei de Judá, a Ramote
de Gileade. E disse o rei de Israel a Jeosafá: Disfarçando-me
eu, então entrarei na peleja; tu, porém, veste as tuas roupas reais.
Disfarçou-se, pois, o rei de Israel, e entraram na peleja.
Deu ordem, porém, o rei da Síria aos capitães dos carros que
tinha, dizendo: Não pelejareis nem contra pequeno, nem contra
grande; senão só contra o rei de Israel. Sucedeu que, vendo
os capitães dos carros a Jeosafá, disseram: Este é o rei de Israel,
e o cercaram para pelejar; porém Jeosafá clamou, e o Senhor o
ajudou. E Deus os desviou dele. Porque sucedeu que, vendo os
capitães dos carros, que não era o rei de Israel, deixaram de
segui-lo. Então um homem armou o arco e atirou a esmo, e
feriu o rei de Israel entre as junturas e a couraça; então disse ao
carreteiro: Dá volta, e tira-me do exército, porque estou gravemente
ferido. E aquele dia cresceu a peleja, mas o rei de Israel
susteve-se em pé no carro defronte dos sírios até à tarde; e morreu
ao tempo do pôr do sol.

\medskip

\lettrine{19} E Jeosafá, rei de Judá, voltou em paz à sua casa
em Jerusalém. E Jeú, filho de Hanani, o vidente, saiu ao
encontro do rei Jeosafá e lhe disse: Devias tu ajudar ao ímpio, e
amar aqueles que odeiam ao Senhor? Por isso virá sobre ti grande ira
da parte do Senhor. Boas coisas contudo se acharam em ti; porque
tiraste os bosques da terra, e preparaste o teu coração para buscar
a Deus. Habitou, pois, Jeosafá em Jerusalém; e tornou a passar
pelo povo desde Berseba até as montanhas de Efraim, e fez com que
tornassem ao Senhor Deus de seus pais.

E estabeleceu juízes na terra, em todas as cidades fortificadas,
de cidade em cidade. E disse aos juízes: Vede o que fazeis;
porque não julgais da parte do homem, senão da parte do Senhor, e
ele está convosco quando julgardes. Agora, pois, seja o temor do
Senhor convosco; guardai-o, e fazei-o; porque não há no Senhor nosso
Deus iniqüidade nem acepção de pessoas, nem aceitação de suborno.
E também estabeleceu Jeosafá a alguns dos levitas e dos
sacerdotes e dos chefes dos pais de Israel sobre o juízo do Senhor,
e sobre as causas judiciais; e voltaram a Jerusalém. E deu-lhes
ordem, dizendo: Assim fazei no temor do Senhor, com fidelidade, e
com coração íntegro. E em toda a diferença que vier a vós de
vossos irmãos que habitam nas suas cidades, entre sangue e sangue,
entre lei e mandamento, entre estatutos e juízos, admoestai-os, que
não se façam culpados para com o Senhor, e não venha grande ira
sobre vós, e sobre vossos irmãos; fazei assim, e não vos fareis
culpados. E eis que Amarias, o sumo sacerdote, presidirá
sobre vós em todo o negócio do Senhor; e Zebadias, filho de Ismael,
líder da casa de Judá, em todo o negócio do rei; também os oficiais,
os levitas, estão perante vós; esforçai-vos, pois, e fazei-o; e o
Senhor será com os bons.

\medskip

\lettrine{20} E sucedeu que, depois disto, os filhos de Moabe,
e os filhos de Amom, e com eles outros dos amonitas, vieram à peleja
contra Jeosafá. Então vieram alguns que avisaram a Jeosafá,
dizendo: Vem contra ti uma grande multidão dalém do mar e da Síria;
e eis que já estão em Hazazom-Tamar, que é En-Gedi. Então
Jeosafá temeu, e pôs-se a buscar o Senhor, e apregoou jejum em todo
o Judá. E Judá se ajuntou, para pedir socorro ao Senhor; também
de todas as cidades de Judá vieram para buscar ao Senhor. E
pôs-se Jeosafá em pé na congregação de Judá e de Jerusalém, na casa
do Senhor, diante do pátio novo. E disse: Ah! Senhor Deus de
nossos pais, porventura não és tu Deus nos céus? Não és tu que
dominas sobre todos os reinos das nações? Na tua mão há força e
potência, e não há quem te possa resistir. Porventura, ó nosso
Deus, não lançaste fora os moradores desta terra de diante do teu
povo Israel, e não a deste para sempre à descendência de Abraão, teu
amigo? E habitaram nela e edificaram-te nela um santuário ao teu
nome, dizendo: Se algum mal nos sobrevier, espada, juízo, peste,
ou fome, nós nos apresentaremos diante desta casa e diante de ti,
pois teu nome está nesta casa, e clamaremos a ti na nossa angústia,
e tu nos ouvirás e livrarás. Agora, pois, eis que os filhos
de Amom, e de Moabe e os das montanhas de Seir, pelos quais não
permitiste passar a Israel, quando vinham da terra do Egito, mas
deles se desviaram e não os destruíram, eis que nos dão o
pago, vindo para lançar-nos fora da tua herança, que nos fizeste
herdar. Ah! nosso Deus, porventura não os julgarás? Porque em
nós não há força perante esta grande multidão que vem contra nós, e
não sabemos o que faremos; porém os nossos olhos estão postos em ti.
E todo o Judá estava em pé perante o Senhor, como também as
suas crianças, as suas mulheres, e os seus filhos.

Então veio o Espírito do Senhor, no meio da congregação, sobre
Jaaziel, filho de Zacarias, filho de Benaia, filho de Jeiel, filho
de Matanias, levita, dos filhos de Asafe, e disse: Dai
ouvidos todo o Judá, e vós, moradores de Jerusalém, e tu, ó rei
Jeosafá; assim o Senhor vos diz: Não temais, nem vos assusteis por
causa desta grande multidão; pois a peleja não é vossa, mas de Deus.
Amanhã descereis contra eles; eis que sobem pela ladeira de
Ziz, e os achareis no fim do vale, diante do deserto de Jeruel.
Nesta batalha não tereis que pelejar; postai-vos, ficai
parados, e vede a salvação do Senhor para convosco, ó Judá e
Jerusalém. Não temais, nem vos assusteis; amanhã saí-lhes ao
encontro, porque o Senhor será convosco. Então Jeosafá se
prostrou com o rosto em terra, e todo o Judá e os moradores de
Jerusalém se lançaram perante o Senhor, adorando-o. E
levantaram-se os levitas, dos filhos dos coatitas, e dos filhos dos
coratitas, para louvarem ao Senhor Deus de Israel, com voz muito
alta.

E pela manhã cedo se levantaram e saíram ao deserto de Tecoa; e,
ao saírem, Jeosafá pôs-se em pé, e disse: Ouvi-me, ó Judá, e vós,
moradores de Jerusalém: Crede no Senhor vosso Deus, e estareis
seguros; crede nos seus profetas, e prosperareis; e
aconselhou-se com o povo, e ordenou cantores para o Senhor, que
louvassem à Majestade santa, saindo diante dos armados, e dizendo:
Louvai ao Senhor porque a sua benignidade dura para sempre.
 E, quando começaram a cantar e a dar louvores, o Senhor pôs
emboscadas contra os filhos de Amom e de Moabe e os das montanhas de
Seir, que vieram contra Judá, e foram desbaratados. Porque os
filhos de Amom e de Moabe se levantaram contra os moradores das
montanhas de Seir, para os destruir e exterminar; e, acabando eles
com os moradores de Seir, ajudaram uns aos outros a destruir-se.
Nisso chegou Judá à atalaia\footnote{Vigia, guarda,
sentinela. Ponto alto de onde se vigia. Torre de vigia.} do deserto;
e olharam para a multidão, e eis que eram corpos mortos, que jaziam
em terra, e nenhum escapou. E vieram Jeosafá e o seu povo
para saquear os seus despojos, e acharam entre eles riquezas e
cadáveres em abundância, assim como objetos preciosos; e tomaram
para si tanto, que não podiam levar; e três dias saquearam o
despojo, porque era muito. E ao quarto dia se ajuntaram no
vale de Beraca; pois ali louvaram ao Senhor. Por isso chamaram
aquele lugar o vale de Beraca, até ao dia de hoje. Então
voltaram todos os homens de Judá e de Jerusalém, e Jeosafá à frente
deles, e tornaram a Jerusalém com alegria; porque o Senhor os
alegrara sobre os seus inimigos. E vieram a Jerusalém com
saltérios, com harpas e com trombetas, para a casa do Senhor.
E veio o temor de Deus sobre todos os reinos daquelas terras,
ouvindo eles que o Senhor havia pelejado contra os inimigos de
Israel. E o reino de Jeosafá ficou quieto; e o seu Deus lhe
deu repouso ao redor.

E Jeosafá reinou sobre Judá; era da idade de trinta e cinco anos
quando começou a reinar e vinte e cinco anos reinou em Jerusalém; e
o nome de sua mãe era Azuba, filha de Sili. E andou no
caminho de Asa, seu pai, e não se desviou dele, fazendo o que era
reto aos olhos do Senhor. Contudo os altos não foram tirados
porque o povo não tinha ainda disposto o seu coração para com o Deus
de seus pais. Ora, o restante dos atos de Jeosafá, assim,
desde os primeiros até os últimos, eis que está escrito nas notas de
Jeú, filho de Hanani, que as inseriu no livro dos reis de Israel.
Porém, depois disto, Jeosafá, rei de Judá, se aliou com
Acazias, rei de Israel, que procedeu com toda a impiedade. E
aliou-se com ele, para fazerem navios que fossem a Társis; e fizeram
os navios em Eziom-Geber. Porém Eliezer, filho de Dodava, de
Maressa, profetizou contra Jeosafá, dizendo: Porquanto te aliaste
com Acazias, o Senhor despedaçou as tuas obras. E os navios se
quebraram, e não puderam ir a Társis.

\medskip

\lettrine{21} Depois Jeosafá dormiu com seus pais, e foi
sepultado junto a eles na cidade de Davi; e Jeorão, seu filho,
reinou em seu lugar. E teve irmãos, filhos de Jeosafá: Azarias,
Jeiel, Zacarias, Asarias, Micael e Sefatias; todos estes foram
filhos de Jeosafá, rei de Israel. E seu pai lhes deu muitos
presentes de prata, de ouro e de coisas preciosíssimas, juntamente
com cidades fortificadas em Judá; porém, o reino, deu a Jeorão,
porquanto era o primogênito. E, subindo Jeorão ao reino de seu
pai, e havendo-se fortificado, matou a todos os seus irmãos à
espada, como também a alguns dos príncipes de Israel. Da idade
de trinta e dois anos era Jeorão, quando começou a reinar; e reinou
oito anos em Jerusalém. E andou no caminho dos reis de Israel,
como fazia a casa de Acabe; porque tinha a filha de Acabe por
mulher; e fazia o que era mau aos olhos do Senhor. Porém o
Senhor não quis destruir a casa de Davi, em atenção à aliança que
tinha feito com Davi; e porque também tinha falado que lhe daria por
todos os dias uma lâmpada, a ele e a seus filhos. Nos seus dias
se revoltaram os edomitas contra o mando de Judá, e constituíram
para si um rei. Por isso Jeorão passou adiante com os seus
príncipes, e todos os carros com ele; levantou-se de noite, e feriu
aos edomeus, que o tinham cercado, como também aos capitães dos
carros. Todavia os edomitas se revoltaram contra o mando de
Judá até ao dia de hoje; então no mesmo tempo Libna se revoltou
contra o seu mando; porque deixara ao Senhor Deus de seus pais.
Ele também fez altos nos montes de Judá; e fez com que se
corrompessem os moradores de Jerusalém, e até a Judá impeliu a isso.

Então lhe veio um escrito da parte de Elias, o profeta, que
dizia: Assim diz o Senhor Deus de Davi teu pai: Porquanto não
andaste nos caminhos de Jeosafá, teu pai, e nos caminhos de Asa, rei
de Judá, mas andaste no caminho dos reis de Israel, e fizeste
prostituir a Judá e aos moradores de Jerusalém, segundo a
prostituição da casa de Acabe, e também mataste a teus irmãos da
casa de teu pai, melhores do que tu; eis que o Senhor ferirá
com um grande flagelo ao teu povo, aos teus filhos, às tuas mulheres
e a todas as tuas fazendas. Tu também terás grande
enfermidade por causa de uma doença em tuas entranhas, até que elas
saiam, de dia em dia, por causa do mal. Despertou, pois, o
Senhor, contra Jeorão o espírito dos filisteus e dos árabes, que
estavam do lado dos etíopes. Estes subiram a Judá, e deram
sobre ela, e levaram todos os bens que se achou na casa do rei, como
também a seus filhos e a suas mulheres; de modo que não lhe deixaram
filho algum, senão a Joacaz, o mais moço de seus filhos. E
depois de tudo isto o Senhor o feriu nas suas entranhas com uma
enfermidade incurável. E sucedeu que, depois de muito tempo,
ao fim de dois anos, saíram-lhe as entranhas por causa da doença; e
morreu daquela grave enfermidade; e o seu povo não lhe queimou aroma
como queimara a seus pais. Era da idade de trinta e dois anos
quando começou a reinar, e reinou oito anos em Jerusalém; e foi sem
deixar de si saudades; e sepultaram-no na cidade de Davi, porém não
nos sepulcros dos reis.

\medskip

\lettrine{22} E os moradores de Jerusalém, em lugar de Jeorão,
fizeram rei a Acazias, seu filho mais moço, porque a tropa, que
viera com os árabes ao arraial, tinha matado a todos os mais velhos.
Assim reinou Acazias, filho de Jeorão, rei de Judá. Era da idade
de quarenta e dois anos, quando começou a reinar, e reinou um ano em
Jerusalém; e era o nome de sua mãe Atalia, filha de Onri. Também
ele andou nos caminhos da casa de Acabe, porque sua mãe era sua
conselheira, para proceder impiamente. E fez o que era mau aos
olhos do Senhor, como a casa de Acabe, porque eles eram seus
conselheiros depois da morte de seu pai, para a sua perdição.
Também andou nos conselhos deles, e foi com Jorão, filho de
Acabe, rei de Israel, à peleja contra Hazael, rei da Síria, junto a
Ramote de Gileade; e os sírios feriram a Jorão. E voltou para
curar-se em Jizreel, das feridas que lhe fizeram em Ramá, pelejando
contra Hazael, rei da Síria; e Acazias, filho de Jeorão, rei de
Judá, desceu para ver a Jorão, filho de Acabe, em Jizreel, porque
estava doente. Foi, pois, da vontade de Deus, que Acazias, para
sua ruína, visitou Jorão; porque chegando ele, saiu com Jorão contra
Jeú, filho de Ninsi, a quem o Senhor tinha ungido para desarraigar a
casa de Acabe. E sucedeu que, executando Jeú juízo contra a casa
de Acabe, achou os príncipes de Judá e os filhos dos irmãos de
Acazias, que serviam a Acazias, e os matou. Depois buscou a
Acazias (porque se tinha escondido em Samaria), e o alcançaram, e o
trouxeram a Jeú, e o mataram, e o sepultaram; porque disseram: Filho
é de Jeosafá, que buscou ao Senhor com todo o seu coração. E já não
tinha a casa de Acazias ninguém que tivesse força para o reino.

Vendo, pois, Atalia, mãe de Acazias, que seu filho era morto,
levantou-se e destruiu toda a descendência real da casa de Judá.
Porém Jeosabeate, filha do rei, tomou a Joás, filho de
Acazias, furtando-o dentre os filhos do rei, aos quais matavam, e o
pôs com a sua ama na câmara dos leitos; assim Jeosabeate, filha do
rei Jeorão, mulher do sacerdote Joiada (porque era irmã de Acazias),
o escondeu de Atalia, de modo que ela não o matou. E esteve
com eles seis anos escondido na casa de Deus; e Atalia reinou sobre
a terra.

\medskip

\lettrine{23} Porém no sétimo ano Joiada se animou, e tomou
consigo em aliança os chefes de cem, a Azarias, filho de Jeroão, a
Ismael, filho de Joanã, a Azarias, filho de Obede, a Maaséias, filho
de Adaías, e a Elisafate, filho de Zicri. Estes percorreram a
Judá e ajuntaram os levitas de todas as cidades de Judá e os chefes
dos pais de Israel, e vieram para Jerusalém. E toda aquela
congregação fez aliança com o rei na casa de Deus; e Joiada lhes
disse: Eis que o filho do rei reinará, como o Senhor falou a
respeito dos filhos de Davi. Isto é o que haveis de fazer; uma
terça parte de vós, ou seja, dos sacerdotes e dos levitas que entram
no sábado, serão guardas das portas; e uma terça parte estará na
casa do rei; e a outra terça parte à porta do fundamento; e todo o
povo estará nos pátios da casa do Senhor. Porém ninguém entre na
casa do Senhor, senão os sacerdotes e os levitas que ministram;
estes entrarão, porque são santos; mas todo o povo fará a guarda
diante do Senhor. E os levitas cercarão o rei de todos os lados,
cada um com as suas armas na mão; e qualquer que entrar na casa será
morto; porém vós estareis com o rei, quando entrar e quando sair.
E fizeram os levitas e todo o Judá conforme a tudo o que
ordenara o sacerdote Joiada; e tomou cada um os seus homens, tanto
os que entravam no sábado como os que saíam no sábado; porque o
sacerdote Joiada não tinha despedido as turmas. Também o
sacerdote Joiada deu aos capitães de cem as lanças, os escudos e as
rodelas\footnote{Escudo redondo.} que foram do rei Davi, os quais
estavam na casa de Deus. E dispôs todo o povo, a cada um com
as suas armas na mão, desde o lado direito da casa até o lado
esquerdo da casa, do lado do altar e da casa, em redor do rei.
Então tiraram para fora ao filho do rei, e lhe puseram a
coroa; deram-lhe o testemunho, e o fizeram rei; e Joiada e seus
filhos o ungiram, e disseram: Viva o rei!

Ouvindo, pois, Atalia a voz do povo que concorria e louvava o
rei, veio ao povo, à casa do Senhor. E olhou, e eis que o rei
estava junto à coluna, à entrada, e os príncipes e as trombetas
junto ao rei; e todo o povo da terra estava alegre e tocava as
trombetas; e também os cantores tocavam instrumentos musicais, e
dirigiam o cantar de louvores; então Atalia rasgou os seus vestidos,
e clamou: Traição, traição! Porém o sacerdote Joiada trouxe
para fora os centuriões que estavam postos sobre o exército e
disse-lhes: Tirai-a para fora das fileiras, e o que a seguir,
morrerá à espada; porque dissera o sacerdote: Não a mateis na casa
do Senhor. E lançaram mão dela; e ela foi pelo caminho da
entrada da porta dos cavalos, à casa do rei, e ali a mataram.
E Joiada fez aliança entre si e o povo e o rei, para que
fossem o povo do Senhor. Depois todo o povo entrou na casa de
Baal, e a derrubaram, e quebraram os seus altares, e as suas
imagens, e a Matã, sacerdote de Baal, mataram diante dos altares.
E Joiada ordenou os ofícios na casa do Senhor, sob a direção
dos sacerdotes levitas a quem Davi designara na casa do Senhor, para
oferecerem os holocaustos do Senhor, como está escrito na lei de
Moisés, com alegria e com canto, conforme a instituição de Davi.
E pôs porteiros às portas da casa do Senhor, para que nela
não entrasse ninguém imundo em coisa alguma. E tomou os
centuriões, os poderosos, os que tinham domínio entre o povo e todo
o povo da terra, e conduziram o rei da casa do Senhor, e entraram na
casa do rei passando pela porta maior, e assentaram-no no trono
real. E todo o povo da terra se alegrou, e a cidade ficou em
paz, depois que mataram a Atalia à espada.

\medskip

\lettrine{24} Tinha Joás sete anos de idade quando começou a
reinar, e quarenta anos reinou em Jerusalém; e era o nome da sua mãe
Zibia, de Berseba. E fez Joás o que era reto aos olhos do
Senhor, todos os dias do sacerdote Joiada. E tomou-lhe Joiada
duas mulheres, e gerou filhos e filhas. E, depois disto, Joás
resolveu renovar a casa do Senhor. Reuniu, pois, os sacerdotes e
os levitas, e disse-lhes: Saí pelas cidades de Judá, e levantai
dinheiro de todo o Israel para reparar a casa do vosso Deus de ano
em ano; e vós, apressai este negócio. Porém os levitas não se
apressaram. E o rei chamou a Joiada, o chefe, e disse-lhe: Por
que não requereste dos levitas, que trouxessem de Judá e de
Jerusalém o tributo que Moisés, servo do Senhor, ordenou à
congregação de Israel, para a tenda do testemunho? Porque, sendo
Atalia ímpia, seus filhos arruinaram a casa de Deus, e até todas as
coisas sagradas da casa do Senhor empregaram em Baalins. E o
rei, pois, deu ordem e fizeram um cofre, e o puseram fora, à porta
da casa do Senhor. E publicou-se em Judá e em Jerusalém que
trouxessem ao Senhor o tributo de Moisés, o servo de Deus, ordenado
a Israel no deserto. Então todos os príncipes e todo o povo
se alegraram, e o trouxeram e o lançaram no cofre, até que ficou
cheio. E sucedia que, quando levavam o cofre pelas mãos dos
levitas, segundo o mandado do rei, e vendo-se que já havia muito
dinheiro, vinha o escrivão do rei, e o oficial do sumo sacerdote, e
esvaziavam o cofre, e tomavam-no e levavam-no de novo ao seu lugar;
assim faziam de dia em dia, e ajuntaram dinheiro em abundância,
o qual o rei e Joiada davam aos que tinham o encargo da obra
do serviço da casa do Senhor; e contrataram pedreiros e
carpinteiros, para renovarem a casa do Senhor; como também ferreiros
e serralheiros, para repararem a casa do Senhor. E os que
tinham o encargo da obra faziam com que o trabalho de reparação
fosse crescendo pelas suas mãos; e restauraram a casa de Deus no seu
estado, e a fortaleceram. E, depois de acabarem, trouxeram ao
rei e a Joiada o resto do dinheiro, e dele fizeram utensílios para a
casa do Senhor, objetos para ministrar e oferecer, colheres, vasos
de ouro e de prata. E continuamente sacrificaram holocaustos na casa
do Senhor, todos os dias de Joiada.

E envelheceu Joiada, e morreu farto de dias; era da idade de
cento e trinta anos quando morreu. E o sepultaram na cidade
de Davi com os reis; porque tinha feito bem em Israel, e para com
Deus e a sua casa. Porém, depois da morte de Joiada vieram os
príncipes de Judá e prostraram-se perante o rei; e o rei os ouviu.
E deixaram a casa do Senhor Deus de seus pais, e serviram às
imagens do bosque e aos ídolos. Então, por causa desta sua culpa,
veio grande ira sobre Judá e Jerusalém. Porém enviou profetas
entre eles, para os reconduzir ao Senhor, os quais protestaram
contra eles; mas eles não deram ouvidos. E o Espírito de Deus
revestiu a Zacarias, filho do sacerdote Joiada, o qual se pôs em pé
acima do povo, e lhes disse: Assim diz Deus: Por que transgredis os
mandamentos do Senhor, de modo que não possais prosperar? Porque
deixastes ao Senhor, também ele vos deixará. E eles
conspiraram contra ele, e o apedrejaram por mandado do rei, no pátio
da casa do Senhor. Assim o rei Joás não se lembrou da
beneficência que Joiada, pai de Zacarias, lhe fizera; porém
matou-lhe o filho, o qual, morrendo, disse: O Senhor o verá, e o
requererá. E sucedeu que, decorrido um ano, o exército da
Síria subiu contra ele; e vieram a Judá e a Jerusalém, e destruíram
dentre o povo a todos os seus príncipes; e enviaram todo o seu
despojo ao rei de Damasco. Porque ainda que o exército dos
sírios viera com poucos homens, contudo o Senhor entregou na sua mão
um exército mui numeroso, porquanto deixaram ao Senhor Deus de seus
pais. Assim executaram juízos contra Joás. E, quando os
sírios se retiraram, deixaram-no gravemente ferido; então seus
servos conspiraram contra ele por causa do sangue do filho do
sacerdote Joiada, e o feriram na sua cama, e morreu; e o sepultaram
na cidade de Davi, porém não nos sepulcros dos reis. Estes,
pois, foram os que conspiraram contra ele: Zabade, filho de Simeate,
a amonita, e Jeozabade, filho de Sinrite, a moabita. E,
quanto a seus filhos, e à grandeza do cargo que se lhe impôs, e à
restauração da casa de Deus, eis que estão escritos no livro da
história dos reis; e Amazias, seu filho, reinou em seu lugar.

\medskip

\lettrine{25} Era Amazias da idade de vinte e cinco anos,
quando começou a reinar, e reinou vinte e nove anos em Jerusalém; e
era o nome de sua mãe Joadã, de Jerusalém. E fez o que era reto
aos olhos do Senhor, porém não com inteireza de coração. Sucedeu
que, sendo-lhe o reino já confirmado, matou a seus servos que
mataram o rei seu pai; porém não matou os filhos deles; mas fez
segundo está escrito na lei, no livro de Moisés, como o Senhor
ordenou, dizendo: Não morrerão os pais pelos filhos, nem os filhos
pelos pais; mas cada um morrerá pelo seu pecado. E Amazias
reuniu a Judá e os pôs segundo as casas dos pais, sob capitães de
milhares, e sob capitães de cem, por todo o Judá e Benjamim; e os
contou, de vinte anos para cima, e achou deles trezentos mil
escolhidos que podiam sair à guerra, e manejar lança e escudo.
Também de Israel tomou a soldo cem mil homens valentes, por cem
talentos de prata. Porém um homem de Deus veio a ele, dizendo: Ó
rei, não deixes ir contigo o exército de Israel; porque o Senhor não
é com Israel, a saber com os filhos de Efraim. Se quiseres ir,
faze-o assim, esforça-te para a peleja. Deus, porém, te fará cair
diante do inimigo; porque força há em Deus para ajudar e para fazer
cair. E disse Amazias ao homem de Deus: Que se fará, pois, dos
cem talentos de prata que dei às tropas de Israel? E disse o homem
de Deus: Mais tem o Senhor que te dar do que isso. Então
separou Amazias as tropas que lhe tinham vindo de Efraim, para que
se fossem ao seu lugar; pelo que se acendeu a sua ira contra Judá, e
voltaram para as suas casas ardendo em ira. Esforçou-se,
pois, Amazias, e conduziu o seu povo, e foi ao Vale do Sal; onde
feriu a dez mil dos filhos de Seir. Também os filhos de Judá
prenderam vivos dez mil, e os levaram ao cume da rocha; e do mais
alto da rocha os lançaram abaixo, e todos se despedaçaram.
Porém os homens das tropas que Amazias despedira, para que
não fossem com ele à peleja, deram sobre as cidades de Judá desde
Samaria, até Bete-Horom; e feriram deles três mil, e saquearam
grande despojo.

E sucedeu que, depois que Amazias veio da matança dos edomitas e
trouxe consigo os deuses dos filhos de Seir, tomou-os por seus
deuses, e prostrou-se diante deles, e queimou-lhes incenso.
Então a ira do Senhor se acendeu contra Amazias, e mandou-lhe
um profeta que lhe disse: Por que buscaste deuses deste povo, os
quais não livraram o seu próprio povo da tua mão? E sucedeu
que, falando ele ao rei, este lhe respondeu: Puseram-te por
conselheiro do rei? Cala-te! Por que haveria de ser ferido? Então
parou o profeta, e disse: Bem vejo eu que já Deus deliberou
destruir-te; porquanto fizeste isto, e não deste ouvidos ao meu
conselho.

E, tendo tomado conselho, Amazias, rei de Judá, mandou dizer a
Jeoás, filho de Jeoacaz, filho de Jeú, rei de Israel: Vem,
vejamo-nos face a face. Porém Jeoás, rei de Israel, mandou
dizer a Amazias, rei de Judá: O cardo que estava no Líbano mandou
dizer ao cedro que estava no Líbano: Dá tua filha por mulher a meu
filho; porém os animais do campo, que estavam no Líbano passaram e
pisaram o cardo. Tu dizes: Eis que tenho ferido os edomitas;
e elevou-se o teu coração, para te gloriares; agora, pois, fica em
tua casa; por que te entremeterias no mal, para caíres tu e Judá
contigo? Porém Amazias não lhe deu ouvidos, porque isto vinha
de Deus, para entregá-los na mão dos seus inimigos; porquanto
buscaram os deuses dos edomitas. E Jeoás, rei de Israel,
subiu; e ele e Amazias, rei de Judá, viram-se face a face em
Bete-Semes, que está em Judá. E Judá foi ferido diante de
Israel; e fugiu cada um para a sua tenda. E Jeoás, rei de
Israel, prendeu a Amazias, rei de Judá, filho de Joás, o filho de
Jeoacaz\footnote{AV: And Joash the king of Israel took Amaziah king
of Judah, the son of Joash, the son of Jehoahaz, at Bethshemesh, and
brought him to Jerusalem, and brake down the wall of Jerusalem from
the gate of Ephraim to the corner gate, four hundred cubits.}, em
Bete-Semes, e o trouxe a Jerusalém; e derrubou o muro de Jerusalém,
desde a porta de Efraim até à porta da esquina, quatrocentos
côvados. Também tomou todo o ouro, a prata, e todos os
utensílios que se acharam na casa de Deus com Obede-Edom, e os
tesouros da casa do rei, e os reféns; e voltou para Samaria.
E viveu Amazias, filho de Joás, rei de Judá, depois da morte
de Jeoás, filho de Jeoacaz, rei de Israel, quinze anos.
Quanto ao mais dos atos de Amazias, tanto os primeiros como
os últimos, eis que, porventura, não estão escritos no livro dos
reis de Judá e de Israel? E desde o tempo em que Amazias se
desviou do Senhor, conspiraram contra ele em Jerusalém, porém ele
fugiu para Laquis; mas perseguiram-no até Laquis, e o mataram ali.
E trouxeram-no sobre cavalos e sepultaram-no com seus pais na
cidade de Judá.

\medskip

\lettrine{26} Então todo o povo tomou a Uzias, que tinha
dezesseis anos, e o fizeram rei em lugar de Amazias seu pai.
Este edificou a Elote, e a restituiu a Judá, depois que o rei
dormiu com seus pais. Tinha Uzias dezesseis anos quando começou
a reinar, e cinqüenta e cinco anos reinou em Jerusalém; e era o nome
de sua mãe Jecolia, de Jerusalém. E fez o que era reto aos olhos
do Senhor; conforme a tudo o que fizera Amazias seu pai. Porque
deu-se a buscar a Deus nos dias de Zacarias, que era entendido nas
visões de Deus; e nos dias em que buscou ao Senhor, Deus o fez
prosperar. Porque saiu e guerreou contra os filisteus, e quebrou
o muro de Gate, o muro de Jabne, e o muro de Asdode; e edificou
cidades em Asdode, e entre os filisteus. E Deus o ajudou contra
os filisteus e contra os árabes que habitavam em Gur-Baal, e contra
os meunitas. E os amonitas deram presentes a Uzias; e o seu nome
foi espalhado até à entrada do Egito, porque se fortificou
altamente. Também Uzias edificou torres em Jerusalém, à porta da
esquina, e à porta do vale, e à porta do ângulo, e as fortificou.
Também edificou torres no deserto, e cavou muitos poços,
porque tinha muito gado, tanto nos vales como nas campinas; tinha
lavradores, e vinhateiros, nos montes e nos campos férteis; porque
era amigo da agricultura. Tinha também Uzias um exército de
homens destros na guerra, que saíam à guerra em tropas, segundo o
número da resenha\footnote{Ato ou efeito de resenhar (fazer resenha
de; relatar minuciosamente). Enumerar por partes. Descrição
pormenorizada. Contagem, conferência. Notícia que abarca certo
número de nomes ou fatos similares. Recensão.} feita por mão de
Jeiel, o escrivão, e Maaséias, oficial, sob a direção de Hananias,
um dos capitães do rei. O total dos chefes dos pais, homens
valentes, era de dois mil e seiscentos. E debaixo das suas
ordens havia um exército guerreiro de trezentos e sete mil e
quinhentos homens, que faziam a guerra com força belicosa, para
ajudar o rei contra os inimigos. E preparou Uzias, para todo
o exército, escudos, lanças, capacetes, couraças e arcos, e até
fundas para atirar pedras. Também fez em Jerusalém máquinas
da invenção de engenheiros, que estivessem nas torres e nos cantos,
para atirarem flechas e grandes pedras; e propagou a sua fama até
muito longe; porque foi maravilhosamente ajudado, até que se
fortificou.

Mas, havendo-se já fortificado, exaltou-se o seu coração até se
corromper; e transgrediu contra o Senhor seu Deus, porque entrou no
templo do Senhor para queimar incenso no altar do incenso.
Porém o sacerdote Azarias entrou após ele, e com ele oitenta
sacerdotes do Senhor, homens valentes. E resistiram ao rei
Uzias, e lhe disseram: A ti, Uzias, não compete queimar incenso
perante o Senhor, mas aos sacerdotes, filhos de Arão, que são
consagrados para queimar incenso; sai do santuário, porque
transgrediste; e não será isto para honra tua da parte do Senhor
Deus. Então Uzias se indignou; e tinha o incensário na sua
mão para queimar incenso. Indignando-se ele, pois, contra os
sacerdotes, a lepra lhe saiu à testa perante os sacerdotes, na casa
do Senhor, junto ao altar do incenso. Então o sumo sacerdote
Azarias olhou para ele, como também todos os sacerdotes, e eis que
já estava leproso na sua testa, e apressuradamente o lançaram fora;
e até ele mesmo se deu pressa a sair, visto que o Senhor o ferira.
Assim ficou leproso o rei Uzias até ao dia da sua morte; e
morou, por ser leproso, numa casa separada, porque foi excluído da
casa do Senhor. E Jotão, seu filho, tinha o encargo da casa do rei,
julgando o povo da terra. Quanto ao mais dos atos de Uzias,
tanto os primeiros como os últimos, o profeta Isaías, filho de Amós,
o escreveu. E dormiu Uzias com seus pais, e o sepultaram com
eles no campo do sepulcro que era dos reis; porque disseram: Leproso
é. E Jotão, seu filho, reinou em seu lugar.

\medskip

\lettrine{27} Tinha Jotão vinte e cinco anos de idade, quando
começou a reinar, e reinou dezesseis anos em Jerusalém; e era o nome
de sua mãe Jerusa, filha de Zadoque. E fez o que era reto aos
olhos do Senhor, conforme a tudo o que fizera Uzias, seu pai, exceto
que não entrou no templo do Senhor. E o povo ainda se corrompia.
Ele edificou a porta superior da casa do Senhor, e também
edificou muitas obras sobre o muro de Ofel. Também edificou
cidades nas montanhas de Judá, e castelos e torres nos bosques.
Ele também guerreou contra o rei dos filhos de Amom, e
prevaleceu sobre eles, de modo que os filhos de Amom naquele ano lhe
deram cem talentos de prata, e dez mil coros de trigo, e dez mil de
cevada; isto lhe trouxeram os filhos de Amom também no segundo e no
terceiro ano. Assim se fortificou Jotão, porque dirigiu os seus
caminhos na presença do Senhor seu Deus. Ora, o restante dos
atos de Jotão, e todas as suas guerras e os seus caminhos, eis que
estão escritos no livro dos reis de Israel e de Judá. Tinha
vinte e cinco anos de idade, quando começou a reinar, e reinou
dezesseis anos em Jerusalém. E dormiu Jotão com seus pais, e
sepultaram-no na cidade de Davi; e Acaz, seu filho, reinou em seu
lugar.

\medskip

\lettrine{28} Tinha Acaz vinte anos de idade, quando começou a
reinar, e dezesseis anos reinou em Jerusalém; e não fez o que era
reto aos olhos do Senhor, como Davi, seu pai. Antes andou nos
caminhos dos reis de Israel, e, além disso, fez imagens fundidas a
Baalins. Também queimou incenso no vale do filho de Hinom, e
queimou a seus filhos no fogo, conforme as abominações dos gentios
que o Senhor tinha expulsado de diante dos filhos de Israel.
Também sacrificou, e queimou incenso nos altos e nos outeiros,
como também debaixo de toda a árvore verde. Por isso o Senhor
seu Deus o entregou na mão do rei dos sírios, os quais o feriram, e
levaram dele em cativeiro uma grande multidão de presos, que
trouxeram a Damasco; também foi entregue na mão do rei de Israel, o
qual lhe infligiu grande derrota.

Porque Peca, filho de Remalias, matou em Judá, num só dia, cento e
vinte mil, todos homens valentes; porquanto deixaram ao Senhor Deus
de seus pais. E Zicri, homem valente de Efraim, matou a Maasias,
filho do rei, e a Azricão, o mordomo, e a Elcana, o segundo depois
do rei. E os filhos de Israel levaram presos de seus irmãos
duzentos mil, mulheres, filhos e filhas; e também saquearam deles
grande despojo, que levaram para Samaria. Mas estava ali um
profeta do Senhor, cujo nome era Obede, o qual saiu ao encontro do
exército que vinha para Samaria, e lhe disse: Eis que, irando-se o
Senhor Deus de vossos pais contra Judá, os entregou na vossa mão, e
vós os matastes com uma raiva tal, que chegou até aos céus. E
agora vós cuidais em sujeitar a vós os filhos de Judá e Jerusalém,
como cativos e cativas; porventura não sois vós mesmos culpados
contra o Senhor vosso Deus? Agora, pois, ouvi-me, e tornai a
enviar os prisioneiros que trouxestes cativos de vossos irmãos;
porque o ardor da ira do Senhor está sobre vós. Então se
levantaram alguns homens dentre os cabeças dos filhos de Efraim, a
saber, Azarias, filho de Joanã, Berequias, filho de Mesilemote,
Jeizquias, filho de Salum, e Amasa, filho de Hadlai, contra os que
voltavam da batalha. E lhes disseram: Não fareis entrar aqui
estes cativos, porque, além da nossa culpa contra o Senhor, vós
intentais acrescentar mais a nossos pecados e a nossas culpas, sendo
que já temos grande culpa, e já o ardor da ira está sobre Israel.
Então os homens armados deixaram os cativos e o despojo
diante dos príncipes e de toda a congregação. E os homens que
foram apontados por seus nomes se levantaram, e tomaram os cativos,
e vestiram do despojo a todos os que dentre eles estavam nus; e
vestiram-nos, e calçaram-nos, e deram-lhes de comer e de beber, e os
ungiram, e a todos os que estavam fracos levaram sobre jumentos, e
conduziram-nos a Jericó, à cidade das palmeiras, a seus irmãos.
Depois voltaram para Samaria.

Naquele tempo o rei Acaz mandou pedir aos reis da Assíria que o
ajudassem. Porque outra vez os edomitas vieram, e feriram a
Judá, e levaram presos em cativeiro. Também os filisteus
deram sobre as cidades da campina e do sul de Judá, e tomaram a
Bete-Semes, e a Aijalom, e a Gederote e a Socó, e os lugares da sua
jurisdição, e a Timna, e os lugares da sua jurisdição, e a Ginzo, e
os lugares da sua jurisdição; e habitaram ali. Porque o
Senhor humilhou a Judá por causa de Acaz, rei de Israel; porque este
se houve desenfreadamente em Judá, havendo prevaricado grandemente
contra o Senhor. E veio a ele Tiglate-Pileser, rei da
Assíria; porém o pôs em aperto, e não o fortaleceu. Porque
Acaz tomou despojos da casa do Senhor, e da casa do rei, e dos
príncipes, e os deu ao rei da Assíria; porém não o ajudou. E
ao tempo em que este o apertou, então ainda mais transgrediu contra
o Senhor, tal era o rei Acaz. Porque sacrificou aos deuses de
Damasco, que o feriram e disse: Visto que os deuses dos reis da
Síria os ajudam, eu lhes sacrificarei, para que me ajudem a mim.
Porém eles foram a sua ruína, e de todo o Israel. E ajuntou
Acaz os utensílios da casa de Deus, e fez em pedaços os utensílios
da casa de Deus, e fechou as portas da casa do Senhor, e fez para si
altares em todos os cantos de Jerusalém. Também em cada
cidade de Judá fez altos para queimar incenso a outros deuses; assim
provocou à ira o Senhor Deus de seus pais. Ora, o restante
dos seus atos e de todos os seus caminhos, tanto os primeiros como
os últimos, eis que estão escritos no livro dos reis de Judá e de
Israel. E dormiu Acaz com seus pais, e o sepultaram na
cidade, em Jerusalém; porém não o puseram nos sepulcros dos reis de
Israel; e Ezequias, seu filho, reinou em seu lugar.

\medskip

\lettrine{29} Tinha Ezequias vinte e cinco anos de idade,
quando começou a reinar, e reinou vinte e nove anos em Jerusalém; e
era o nome de sua mãe Abia, filha de Zacarias. E fez o que era
reto aos olhos do Senhor, conforme a tudo quanto fizera Davi, seu
pai. Ele, no primeiro ano do seu reinado, no primeiro mês, abriu
as portas da casa do Senhor, e as reparou. E trouxe os
sacerdotes, e os levitas, e ajuntou-os na praça oriental, e lhes
disse: Ouvi-me, ó levitas, santificai-vos agora, e santificai a casa
do Senhor Deus de vossos pais, e tirai do santuário a imundícia.
Porque nossos pais transgrediram, e fizeram o que era mau aos
olhos do Senhor nosso Deus, e o deixaram, e desviaram os seus rostos
do tabernáculo do Senhor, e lhe deram as costas. Também fecharam
as portas do alpendre, e apagaram as lâmpadas, e não queimaram
incenso nem ofereceram holocaustos no santuário ao Deus de Israel.
Por isso veio grande ira do Senhor sobre Judá e Jerusalém, e os
entregou à perturbação, à assolação, e ao escárnio, como vós o
estais vendo com os vossos olhos. Porque eis que nossos pais
caíram à espada, e nossos filhos, e nossas filhas, e nossas
mulheres; por isso estiveram em cativeiro. Agora me tem vindo
ao coração, que façamos uma aliança com o Senhor Deus de Israel,
para que se desvie de nós o ardor da sua ira. Agora, filhos
meus, não sejais negligentes; pois o Senhor vos tem escolhido para
estardes diante dele para o servirdes, e para serdes seus ministros
e queimadores de incenso.

Então se levantaram os levitas, Maate, filho de Amasai, e Joel,
filho de Azarias, dos filhos dos coatitas; e dos filhos de Merari,
Quis, filho de Abdi, e Azarias, filho de Jealelel; e dos gersonitas,
Joá, filho de Zima, e Éden, filho de Joá; e dentre os filhos
de Elisafã, Sinri e Jeuel; dentre os filhos de Asafe, Zacarias e
Matanias; e dentre os filhos de Hemam, Jeuel e Simei; e
dentre os filhos de Jedutum, Semaías e Uziel. E ajuntaram a
seus irmãos, e santificaram-se e vieram conforme ao mandado do rei,
pelas palavras do Senhor, para purificarem a casa do Senhor.
E os sacerdotes entraram na casa do Senhor, para a purificar,
e tiraram para fora, ao pátio da casa do Senhor, toda a imundícia
que acharam no templo do Senhor; e os levitas a tomaram, para a
levarem para fora, ao ribeiro de Cedrom. Começaram, pois, a
santificar no primeiro dia, do primeiro mês; e ao oitavo dia do mês
vieram ao alpendre do Senhor, e santificaram a casa do Senhor em
oito dias; e no dia décimo sexto do primeiro mês acabaram.
Então foram ter com o rei Ezequias, e disseram: Já
purificamos toda a casa do Senhor, como também o altar do holocausto
com todos os seus utensílios e a mesa da proposição com todos os
seus utensílios. Também todos os objetos que o rei Acaz no
seu reinado lançou fora, na sua transgressão, já preparamos e
santificamos; e eis que estão diante do altar do Senhor.

Então o rei Ezequias se levantou de madrugada, e reuniu os
líderes da cidade, e subiu à casa do Senhor. E trouxeram sete
novilhos e sete carneiros, e sete cordeiros e sete bodes, para
sacrifício pelo pecado, pelo reino, e pelo santuário, e por Judá, e
disse aos filhos de Arão, os sacerdotes, que os oferecessem sobre o
altar do Senhor. E eles mataram os bois, e os sacerdotes
tomaram o sangue e o espargiram sobre o altar; também mataram os
carneiros, e espargiram o sangue sobre o altar; semelhantemente
mataram os cordeiros, e espargiram o sangue sobre o altar.
Então trouxeram os bodes para sacrifício pelo pecado, perante
o rei e a congregação, e lhes impuseram as suas mãos. E os
sacerdotes os mataram, e com o seu sangue fizeram expiação do pecado
sobre o altar, para reconciliar a todo o Israel; porque o rei tinha
ordenado que se fizesse aquele holocausto e sacrifício pelo pecado,
por todo o Israel. E pôs os levitas na casa do Senhor com
címbalos, com saltérios, e com harpas, conforme ao mandado de Davi e
de Gade, o vidente do rei, e do profeta Natã; porque este mandado
veio do Senhor, por mão de seus profetas. Estavam, pois, os
levitas em pé com os instrumentos de Davi, e os sacerdotes com as
trombetas. E Ezequias deu ordem que oferecessem o holocausto
sobre o altar; e ao tempo em que começou o holocausto, começou
também o canto do Senhor, com as trombetas e com os instrumentos de
Davi, rei de Israel. E toda a congregação se prostrou, quando
entoavam o canto, e as trombetas eram tocadas; tudo isto até o
holocausto se acabar. E acabando de o oferecer, o rei e todos
quantos com ele se achavam se prostraram e adoraram. Então o
rei Ezequias e os príncipes disseram aos levitas que louvassem ao
Senhor com as palavras de Davi, e de Asafe, o vidente. E o louvaram
com alegria e se inclinaram e adoraram. E respondeu Ezequias,
dizendo: Agora vos consagrastes a vós mesmos ao Senhor; chegai-vos e
trazei sacrifícios e ofertas de louvor à casa do Senhor. E a
congregação trouxe sacrifícios e ofertas de louvor, e todos os
dispostos de coração trouxeram holocaustos. E o número dos
holocaustos, que a congregação trouxe, foi de setenta bois, cem
carneiros, duzentos cordeiros; tudo isto em holocausto para o
Senhor. Houve, também, de coisas consagradas, seiscentos bois
e três mil ovelhas. Eram, porém, os sacerdotes mui poucos, e
não podiam esfolar a todos os holocaustos; pelo que seus irmãos os
levitas os ajudaram, até a obra se acabar, e até que os outros
sacerdotes se santificaram; porque os levitas foram mais retos de
coração, para se santificarem, do que os sacerdotes. E houve
também holocaustos em abundância, com a gordura das ofertas
pacíficas, e com as ofertas de libação para os holocaustos. Assim se
restabeleceu o ministério da casa do Senhor. E Ezequias, e
todo o povo se alegraram, por causa daquilo que Deus tinha preparado
para o povo; porque apressuradamente se fez esta obra.

\medskip

\lettrine{30} Depois disto Ezequias enviou mensageiros por
todo o Israel e Judá, e escreveu também cartas a Efraim e a Manassés
para que viessem à casa do Senhor em Jerusalém, para celebrarem a
páscoa ao Senhor Deus de Israel. Porque o rei tivera conselho
com os seus príncipes, e com toda a congregação em Jerusalém, para
celebrarem a páscoa no segundo mês. Porquanto não a puderam
celebrar no tempo próprio, porque não se tinham santificado
sacerdotes em número suficiente, e o povo não se tinha ajuntado em
Jerusalém. E isto pareceu bem aos olhos do rei, e de toda a
congregação. E ordenaram que se fizesse passar pregão por todo o
Israel, desde Berseba até Dã, para que viessem a celebrar a páscoa
ao Senhor Deus de Israel, em Jerusalém; porque muitos não a tinham
celebrado como estava escrito. Foram, pois, os correios com as
cartas, do rei e dos seus príncipes, por todo o Israel e Judá,
segundo o mandado do rei, dizendo: Filhos de Israel, convertei-vos
ao Senhor Deus de Abraão, de Isaque e de Israel; para que ele se
volte para o restante de vós que escapou da mão dos reis da Assíria.
E não sejais como vossos pais e como vossos irmãos, que
transgrediram contra o Senhor Deus de seus pais, pelo que os
entregou à desolação como vedes. Não endureçais agora a vossa
cerviz, como vossos pais; dai a mão ao Senhor, e vinde ao seu
santuário que ele santificou para sempre, e servi ao Senhor vosso
Deus, para que o ardor da sua ira se desvie de vós. Porque, em
vos convertendo ao Senhor, vossos irmãos e vossos filhos acharão
misericórdia perante os que os levaram cativos, e tornarão a esta
terra; porque o Senhor vosso Deus é misericordioso e compassivo, e
não desviará de vós o seu rosto, se vos converterdes a ele. E
os correios foram passando de cidade em cidade, pela terra de Efraim
e Manassés até Zebulom; porém riram-se e zombaram deles.
Todavia alguns de Aser, e de Manassés, e de Zebulom, se
humilharam, e vieram a Jerusalém. E a mão de Deus esteve com
Judá, dando-lhes um só coração, para fazerem o mandado do rei e dos
príncipes, conforme a palavra do Senhor.

E ajuntou-se em Jerusalém muito povo, para celebrar a festa dos
pães ázimos, no segundo mês; uma congregação mui grande. E
levantaram-se, e tiraram os altares que havia em Jerusalém; também
tiraram todos os altares de incenso, e os lançaram no ribeiro de
Cedrom. Então sacrificaram a páscoa no dia décimo quarto do
segundo mês; e os sacerdotes e levitas se envergonharam e se
santificaram e trouxeram holocaustos à casa do Senhor. E
puseram-se no seu posto, segundo o seu costume, conforme a lei de
Moisés, o homem de Deus; e os sacerdotes espargiam o sangue,
tomando-o da mão dos levitas. Porque havia muitos na
congregação que não se tinham santificado; pelo que os levitas
tinham o encargo de matarem os cordeiros da páscoa por todo aquele
que não estava limpo, para o santificarem ao Senhor. Porque
uma multidão do povo, muitos de Efraim e Manassés, Issacar e
Zebulom, não se tinham purificado, e contudo comeram a páscoa, não
como está escrito; porém Ezequias orou por eles, dizendo: O Senhor,
que é bom, perdoa todo aquele que tem preparado o seu coração
para buscar ao Senhor Deus, o Deus de seus pais, ainda que não
esteja purificado segundo a purificação do santuário. E ouviu
o Senhor a Ezequias, e sarou o povo.

E os filhos de Israel, que se acharam em Jerusalém, celebraram a
festa dos pães ázimos sete dias com grande alegria; e os levitas e
os sacerdotes louvaram ao Senhor de dia em dia, com estrondosos
instrumentos ao Senhor. E Ezequias falou benignamente a todos
os levitas, que tinham bom entendimento no conhecimento do Senhor; e
comeram as ofertas da solenidade por sete dias, oferecendo ofertas
pacíficas, e louvando ao Senhor Deus de seus pais. E, tendo
toda a congregação conselho para celebrarem outros sete dias,
celebraram ainda sete dias com alegria. Porque Ezequias, rei
de Judá, ofereceu à congregação mil novilhos e sete mil ovelhas; e
os príncipes ofereceram à congregação mil novilhos e dez mil
ovelhas; e os sacerdotes se santificaram em grande número. E
alegraram-se, toda a congregação de Judá, e os sacerdotes, e os
levitas, toda a congregação de todos os que vieram de Israel, como
também os estrangeiros que vieram da terra de Israel e os que
habitavam em Judá. E houve grande alegria em Jerusalém;
porque desde os dias de Salomão, filho de Davi, rei de Israel, tal
não houve em Jerusalém. Então os sacerdotes e os levitas se
levantaram e abençoaram o povo; e a sua voz foi ouvida; porque a sua
oração chegou até à santa habitação de Deus, até aos céus.

\medskip

\lettrine{31} E acabando tudo isto, todos os israelitas que
ali se achavam saíram às cidades de Judá e quebraram as estátuas,
cortaram os bosques, e derrubaram os altos e altares por toda Judá e
Benjamim, como também em Efraim e Manassés, até que tudo destruíram;
então tornaram todos os filhos de Israel, cada um para sua
possessão, para as cidades deles. E estabeleceu Ezequias as
turmas dos sacerdotes e levitas, segundo as suas turmas, a cada um
segundo o seu ministério; aos sacerdotes e levitas para o holocausto
e para as ofertas pacíficas, para ministrarem, louvarem, e cantarem,
às portas dos arraiais do Senhor. Também estabeleceu a parte da
fazenda do rei para os holocaustos; para os holocaustos da manhã e
da tarde, e para os holocaustos dos sábados, e das luas novas, e das
solenidades; como está escrito na lei do Senhor. E ordenou ao
povo, que morava em Jerusalém, que desse a parte dos sacerdotes e
levitas, para que eles pudessem se dedicar à lei do Senhor. E,
depois que se divulgou esta ordem, os filhos de Israel trouxeram
muitas primícias de trigo, mosto, azeite, mel, e de todo o produto
do campo; também os dízimos de tudo trouxeram em abundância. E
os filhos de Israel e de Judá, que habitavam nas cidades de Judá,
também trouxeram dízimos dos bois e das ovelhas, e dízimos das
coisas dedicadas que foram consagradas ao Senhor seu Deus; e fizeram
muitos montões. No terceiro mês começaram a fazer os primeiros
montões; e no sétimo mês acabaram. Vindo, pois, Ezequias e os
príncipes, e vendo aqueles montões, bendisseram ao Senhor e ao seu
povo Israel. E perguntou Ezequias aos sacerdotes e aos levitas
acerca daqueles montões. E Azarias, o sumo sacerdote da casa
de Zadoque, lhe respondeu, dizendo: Desde que se começou a trazer
estas ofertas à casa do Senhor, temos comido e temos fartado, e
ainda sobejou em abundância; porque o Senhor abençoou ao seu povo, e
sobejou esta abastança.

Então ordenou Ezequias que se preparassem câmaras na casa do
Senhor, e as prepararam. Ali recolheram fielmente as ofertas,
e os dízimos, e as coisas consagradas; e tinham cargo disto
Conanias, o levita principal, e Simei, seu irmão, o segundo.
E Jeiel, Azarias, Naate, Asael, Jerimote, Jozabade, Eliel,
Ismaquias, Maate, e Benaia, eram superintendentes sob a direção de
Conanias e Simei, seu irmão, por mandado do rei Ezequias, e de
Azarias, líder da casa de Deus. E Coré, filho de Imna, o
levita, porteiro do lado do oriente, estava encarregado das ofertas
voluntárias que se faziam a Deus, para distribuir as ofertas alçadas
do Senhor e as coisas santíssimas. E debaixo das suas ordens
estavam Éden, Miniamim, Jesua, Semaías, Amarias e Secanias, nas
cidades dos sacerdotes, para distribuírem com fidelidade a seus
irmãos, segundo as suas turmas, tanto aos pequenos como aos grandes;
exceto os que estavam contados pelas genealogias dos homens,
da idade de três anos para cima, a todos os que entravam na casa do
Senhor, para a obra de cada dia no seu dia, pelo seu ministério nas
suas guardas, segundo as suas turmas. Quanto ao registro dos
sacerdotes foi ele feito segundo as suas famílias, e o dos levitas,
da idade de vinte anos para cima, foi feito segundo as suas guardas
nas suas turmas; como também conforme às genealogias, com
todas as suas crianças, suas mulheres, e seus filhos, e suas filhas,
por toda a congregação. Porque com fidelidade estes se santificavam
nas coisas consagradas. Também dentre os filhos de Arão, os
sacerdotes, que estavam nos campos dos arrabaldes das suas cidades,
em cada cidade, havia homens que foram designados pelos seus nomes
para distribuírem as porções a todo o homem entre os sacerdotes e a
todos os que estavam contados entre os levitas. E assim fez
Ezequias em todo o Judá; e fez o que era bom, e reto, e verdadeiro,
perante o Senhor seu Deus. E toda a obra que começou no
serviço da casa de Deus, e na lei, e nos mandamentos, para buscar a
seu Deus, ele a fez de todo o seu coração, e prosperou.

\medskip

\lettrine{32} Depois destas coisas e desta verdade, veio
Senaqueribe, rei da Assíria, e entrou em Judá, e acampou-se contra
as cidades fortificadas, e intentou apoderar-se delas. Vendo,
pois, Ezequias que Senaqueribe vinha, e que estava resolvido contra
Jerusalém, teve conselho com os seus príncipes e os seus homens
valentes, para que se tapassem as fontes das águas que havia fora da
cidade; e eles o ajudaram. Assim muito povo se ajuntou, e tapou
todas as fontes, como também o ribeiro que se estendia pelo meio da
terra, dizendo: Por que viriam os reis da Assíria, e achariam tantas
águas? E ele se animou, e edificou todo o muro quebrado até às
torres, e levantou o outro muro por fora; e fortificou a Milo na
cidade de Davi, e fez armas e escudos em abundância. E pôs
capitães de guerra sobre o povo, e reuniu-os na praça da porta da
cidade, e falou-lhes ao coração, dizendo: Esforçai-vos, e tende
bom ânimo; não temais, nem vos espanteis, por causa do rei da
Assíria, nem por causa de toda a multidão que está com ele, porque
há um maior conosco do que com ele. Com ele está o braço de
carne, mas conosco o Senhor nosso Deus, para nos ajudar, e para
guerrear por nós. E o povo descansou nas palavras de Ezequias, rei
de Judá.

Depois disto Senaqueribe, rei da Assíria, enviou os seus servos a
Jerusalém (ele porém estava diante de Laquis, com todas as suas
forças), a Ezequias, rei de Judá, e a todo o Judá que estava em
Jerusalém, dizendo: Assim diz Senaqueribe, rei da Assíria: Em
que confiais vós, para vos deixardes sitiar em Jerusalém?
Porventura não vos incita Ezequias, para morrerdes à fome e à
sede, dizendo: O Senhor nosso Deus nos livrará das mãos do rei da
Assíria? Não é Ezequias o mesmo que tirou os seus altos e os
seus altares, e falou a Judá e a Jerusalém, dizendo: Diante de um
único altar vos prostrareis, e sobre ele queimareis incenso?
Não sabeis vós o que eu e meus pais fizemos a todos os povos
das terras? Porventura puderam de qualquer maneira os deuses das
nações daquelas terras livrar o seu país da minha mão? Qual
é, de todos os deuses daquelas nações que meus pais destruíram, o
que pôde livrar o seu povo da minha mão, para que vosso Deus vos
possa livrar da minha mão? Agora, pois, não vos engane
Ezequias, nem vos incite assim, nem lhe deis crédito; porque nenhum
deus de nação alguma, nem de reino algum, pôde livrar o seu povo da
minha mão, nem da mão de meus pais; quanto menos vos poderá livrar o
vosso Deus da minha mão? Também seus servos falaram ainda
mais contra o Senhor Deus, e contra Ezequias, o seu servo.
Escreveu também cartas, para blasfemar do Senhor Deus de
Israel, e para falar contra ele, dizendo: Assim como os deuses das
nações das terras não livraram o seu povo da minha mão, assim também
o Deus de Ezequias não livrará o seu povo da minha mão. E
clamaram em alta voz em judaico contra o povo de Jerusalém, que
estava em cima do muro, para os atemorizar e os perturbar, para que
tomassem a cidade. E falaram do Deus de Jerusalém, como dos
deuses dos povos da terra, obras das mãos dos homens. Porém o
rei Ezequias e o profeta Isaías, filho de Amós, oraram contra isso,
e clamaram ao céu. Então o Senhor enviou um anjo que destruiu
a todos os homens valentes, e os líderes, e os capitães no arraial
do rei da Assíria; e \emph{este}, envergonhado\footnote{RC-1969:
Então o Senhor enviou um anjo que destruiu a todos os varões
valentes, e os príncipes, e os chefes no arraial do rei da Assíria;
e este tornou com vergonha de rosto à sua terra; e, entrando na casa
de seu deus, os mesmos, que saíram das suas entranhas, o mataram ali
à espada.}, voltou à sua terra; e, entrando na casa de seu deus,
alguns dos seus próprios filhos o mataram ali à espada. Assim
livrou o Senhor a Ezequias, e aos moradores de Jerusalém, da mão de
Senaqueribe, rei da Assíria, e da mão de todos; e de todos os lados
os guiou. E muitos traziam a Jerusalém presentes ao Senhor, e
coisas preciosíssimas a Ezequias, rei de Judá, de modo que depois
disto foi exaltado perante os olhos de todas as nações.

Naqueles dias Ezequias adoeceu mortalmente; e orou ao Senhor, o
qual lhe falou, e lhe deu um sinal. Mas não correspondeu
Ezequias ao benefício que lhe fora feito; porque o seu coração se
exaltou; por isso veio grande ira sobre ele, e sobre Judá e
Jerusalém. Ezequias, porém, se humilhou pela exaltação do seu
coração, ele e os habitantes de Jerusalém; e a grande ira do Senhor
não veio sobre eles, nos dias de Ezequias. E teve Ezequias
riquezas e glória em grande abundância; proveu-se de tesouraria para
prata, ouro, pedras preciosas, especiarias, escudos, e toda a
espécie de objetos desejáveis. Também de armazéns para a
colheita do trigo, e do vinho, e do azeite; e de estrebarias para
toda a espécie de animais e de currais para os rebanhos.
Edificou também cidades, e possuiu ovelhas e vacas em
abundância; porque Deus lhe tinha dado muitíssimas possessões.
Também o mesmo Ezequias tapou o manancial superior das águas
de Giom, e as fez correr por baixo para o ocidente da cidade de
Davi; porque Ezequias prosperou em todas as suas obras.
Contudo, no tocante aos embaixadores dos príncipes de
Babilônia, que foram enviados a ele, a perguntarem acerca do
prodígio que se fez naquela terra, Deus o desamparou, para tentá-lo,
para saber tudo o que havia no seu coração. Quanto aos demais
atos de Ezequias, e as suas boas obras, eis que estão escritos na
visão do profeta Isaías, filho de Amós, e no livro dos reis de Judá
e de Israel. E dormiu Ezequias com seus pais, e o sepultaram
no mais alto dos sepulcros dos filhos de Davi; e todo o Judá e os
habitantes de Jerusalém lhe fizeram honras na sua morte; e Manassés,
seu filho, reinou em seu lugar.

\medskip

\lettrine{33} Tinha Manassés doze anos de idade, quando
começou a reinar, e cinqüenta e cinco anos reinou em Jerusalém.
E fez o que era mau aos olhos do Senhor, conforme às abominações
dos gentios que o Senhor lançara fora de diante dos filhos de
Israel. Porque tornou a edificar os altos que Ezequias, seu pai,
tinha derrubado; e levantou altares aos Baalins, e fez bosques, e
prostrou-se diante de todo o exército dos céus, e o serviu. E
edificou altares na casa do Senhor, da qual o Senhor tinha falado:
Em Jerusalém estará o meu nome eternamente. Edificou altares a
todo o exército dos céus, em ambos os átrios da casa do Senhor.
Fez ele também passar seus filhos pelo fogo no vale do filho de
Hinom, e usou de adivinhações e de agouros, e de feitiçarias, e
consultou adivinhos e encantadores, e fez muitíssimo mal aos olhos
do Senhor, para o provocar à ira. Também pôs uma imagem de
escultura do ídolo que tinha feito, na casa de Deus, da qual Deus
tinha falado a Davi e a Salomão seu filho: Nesta casa e em
Jerusalém, que escolhi de todas as tribos de Israel, porei o meu
nome para sempre. E nunca mais removerei o pé de Israel da terra
que destinei a vossos pais; contanto que tenham cuidado de fazer
tudo o que eu lhes ordenei, conforme a toda a lei, e estatutos, e
juízos, dados pela mão de Moisés. E Manassés tanto fez errar a
Judá e aos moradores de Jerusalém, que fizeram pior do que as nações
que o Senhor tinha destruído de diante dos filhos de Israel.
E falou o Senhor a Manassés e ao seu povo, porém não deram
ouvidos.

Assim o Senhor trouxe sobre eles os capitães do exército do rei
da Assíria, os quais prenderam a Manassés com ganchos e, amarrando-o
com cadeias, o levaram para Babilônia. E ele, angustiado,
orou deveras ao Senhor seu Deus, e humilhou-se muito perante o Deus
de seus pais; e fez-lhe oração, e Deus se aplacou para com
ele, e ouviu a sua súplica, e tornou a trazê-lo a Jerusalém, ao seu
reino. Então conheceu Manassés que o Senhor era Deus. E
depois disto edificou o muro de fora da cidade de Davi, ao ocidente
de Giom, no vale, e à entrada da porta do peixe, e ao redor de Ofel,
e o levantou muito alto; também pôs capitães de guerra em todas as
cidades fortificadas de Judá.
 E tirou da casa do Senhor os deuses estranhos e o ídolo, como
também todos os altares que tinha edificado no monte da casa do
Senhor, e em Jerusalém, e os lançou fora da cidade. E reparou
o altar do Senhor e ofereceu sobre ele sacrifícios de ofertas
pacíficas e de louvor; e ordenou a Judá que servisse ao Senhor Deus
de Israel. Contudo o povo ainda sacrificava nos altos, mas
somente ao Senhor seu Deus. O restante dos atos de Manassés,
e a sua oração ao seu Deus, e as palavras dos videntes que lhe
falaram no nome do Senhor Deus de Israel, eis que estão nas crônicas
dos reis de Israel. E a sua oração, e como Deus se aplacou
para com ele, e todo o seu pecado, e a sua transgressão, e os
lugares onde edificou altos, e pôs bosques e imagens de escultura,
antes que se humilhasse, eis que estão escritos nos livros dos
videntes. E dormiu Manassés com seus pais, e o sepultaram em
sua casa. Amom, seu filho, reinou em seu lugar.

Tinha Amom vinte e dois anos de idade quando começou a reinar, e
dois anos reinou em Jerusalém. E fez o que era mau aos olhos
do Senhor, como havia feito Manassés, seu pai; porque Amom
sacrificou a todas as imagens de escultura que Manassés, seu pai
tinha feito, e as serviu. Mas não se humilhou perante o
Senhor, como Manassés, seu pai, se humilhara; antes multiplicou Amom
os seus delitos. E conspiraram contra ele os seus servos, e o
mataram em sua casa. Porém o povo da terra feriu a todos
quantos conspiraram contra o rei Amom; e o povo da terra fez reinar
em seu lugar a Josias, seu filho.

\medskip

\lettrine{34} Tinha Josias oito anos quando começou a reinar,
e trinta e um anos reinou em Jerusalém. E fez o que era reto aos
olhos do Senhor; e andou nos caminhos de Davi, seu pai, sem se
desviar deles nem para a direita nem para a esquerda. Porque no
oitavo ano do seu reinado, sendo ainda moço, começou a buscar o Deus
de Davi, seu pai; e no duodécimo ano começou a purificar a Judá e a
Jerusalém, dos altos, e dos bosques, e das imagens de escultura e de
fundição. E derrubaram perante ele os altares de Baalins; e
despedaçou as imagens, que estavam acima deles; e os bosques, e as
imagens de escultura e de fundição quebrou e reduziu a pó, e o
espargiu sobre as sepulturas dos que lhes tinham sacrificado. E
os ossos dos sacerdotes queimou sobre os seus altares; e purificou a
Judá e a Jerusalém. O mesmo fez nas cidades de Manassés, e de
Efraim, e de Simeão, e ainda até Naftali, em seus lugares assolados
ao redor. E, tendo derrubado os altares, e os bosques, e as
imagens de escultura, até reduzi-los a pó, e tendo despedaçado todas
as imagens do sol em toda a terra de Israel, então voltou para
Jerusalém.

E no ano décimo oitavo do seu reinado, havendo já purificado a
terra e a casa, enviou a Safã, filho de Azalias, e a Maaséias,
governador da cidade, e a Joá, filho de Joacaz, cronista, para
repararem a casa do Senhor seu Deus. E foram a Hilquias, sumo
sacerdote, e deram o dinheiro que se tinha trazido à casa de Deus, e
que os levitas, que guardavam a entrada tinham recebido da mão de
Manassés, e de Efraim, e de todo o restante de Israel, como também
de todo o Judá e Benjamim, e dos habitantes de Jerusalém. E
eles o entregaram aos que tinham o encargo da obra, e superintendiam
a casa do Senhor; e estes o deram aos que faziam a obra, e
trabalhavam na casa do Senhor, para consertarem e repararem a casa.
E deram-no aos carpinteiros e aos edificadores, para
comprarem pedras lavradas, e madeiras para as junturas e para
servirem de vigas para as casas que os reis de Judá tinham
destruído. E estes homens trabalhavam fielmente na obra; e os
superintendentes sobre eles eram: Jaate e Obadias, levitas, dos
filhos de Merari, como também Zacarias e Mesulão, dos filhos dos
coatitas, para adiantarem a obra; e todos os levitas que eram
entendidos em instrumentos de música. Estavam também sobre os
carregadores e dirigiam todos os que trabalhavam em alguma obra; e
dentre os levitas havia escrivães, oficiais e porteiros.

E, tirando eles o dinheiro que se tinha trazido à casa do Senhor,
Hilquias, o sacerdote, achou o livro da lei do Senhor, dada pela mão
de Moisés. E Hilquias disse a Safã, o escrivão: Achei o livro
da lei na casa do Senhor. E Hilquias deu o livro a Safã. E
Safã levou o livro ao rei, e deu-lhe conta, dizendo: Teus servos
fazem tudo quanto se lhes encomendou. E ajuntaram o dinheiro
que se achou na casa do Senhor, e o deram na mão dos
superintendentes e na mão dos que faziam a obra. Além disto,
Safã, o escrivão, fez saber ao rei, dizendo: O sacerdote Hilquias
entregou-me um livro. E Safã leu nele perante o rei. Sucedeu
que, ouvindo o rei as palavras da lei, rasgou as suas vestes.
E o rei ordenou a Hilquias, e a Aicão, filho de Safã, e a
Abdom, filho de Mica, e a Safã, o escrivão, e a Asaías, servo do
rei, dizendo: Ide, consultai ao Senhor por mim, e pelos que
restam em Israel e em Judá, sobre as palavras deste livro que se
achou; porque grande é o furor do Senhor, que se derramou sobre nós;
porquanto nossos pais não guardaram a palavra do Senhor, para
fazerem conforme a tudo quanto está escrito neste livro.
Então Hilquias, e os enviados do rei, foram ter com a
profetiza Hulda, mulher de Salum, filho de Tocate, filho de Harás,
guarda das vestimentas (e habitava ela em Jerusalém na segunda
parte); e falaram-lhe a esse respeito. E ela lhes disse:
Assim diz o Senhor Deus de Israel: Dizei ao homem que vos enviou a
mim: Assim diz o Senhor: Eis que trarei mal sobre este lugar,
e sobre os seus habitantes, a saber, todas as maldições que estão
escritas no livro que se leu perante o rei de Judá. Porque me
deixaram, e queimaram incenso perante outros deuses, para me
provocarem à ira com todas as obras das suas mãos; portanto o meu
furor se derramou sobre este lugar, e não se apagará. Porém
ao rei de Judá, que vos enviou a consultar ao Senhor, assim lhe
direis: Assim diz o Senhor Deus de Israel, quanto às palavras que
ouviste: Porquanto o teu coração se enterneceu, e te
humilhaste perante Deus, ouvindo as suas palavras contra este lugar,
e contra os seus habitantes, e te humilhaste perante mim, e rasgaste
as tuas vestes, e choraste perante mim, também eu te ouvi, diz o
Senhor. Eis que te reunirei a teus pais, e tu serás recolhido
ao teu sepulcro em paz, e os teus olhos não verão todo o mal que hei
de trazer sobre este lugar e sobre os seus habitantes. E tornaram
com esta resposta ao rei.

Então o rei mandou reunir todos os anciãos de Judá e Jerusalém.
E o rei subiu à casa do Senhor, com todos os homens de Judá,
e os habitantes de Jerusalém, e os sacerdotes, e os levitas, e todo
o povo, desde o maior até ao menor; e ele leu aos ouvidos deles
todas as palavras do livro da aliança que fora achado na casa do
Senhor. E pôs-se o rei em pé em seu lugar, e fez aliança
perante o Senhor, para seguirem ao Senhor, e para guardar os seus
mandamentos, e os seus testemunhos, e os seus estatutos, com todo o
seu coração, e com toda a sua alma, cumprindo as palavras da
aliança, que estão escritas naquele livro. E fez com que
todos quantos se achavam em Jerusalém e em Benjamim o firmassem; e
os habitantes de Jerusalém fizeram conforme a aliança de Deus, o
Deus de seus pais. E Josias tirou todas as abominações de
todas as terras que eram dos filhos de Israel; e a todos quantos se
achavam em Israel obrigou a que servissem ao Senhor seu Deus.
Enquanto ele viveu não se desviaram de seguir o Senhor, o Deus de
seus pais.

\medskip

\lettrine{35} Então Josias celebrou a páscoa ao Senhor em
Jerusalém; e mataram o cordeiro da páscoa no décimo quarto dia do
primeiro mês. E estabeleceu os sacerdotes nos seus cargos, e os
animou ao ministério da casa do Senhor. E disse aos levitas que
ensinavam a todo o Israel e estavam consagrados ao Senhor: Ponde a
arca sagrada na casa que edificou Salomão, filho de Davi, rei de
Israel; não tereis mais esta carga aos ombros; agora servi ao Senhor
vosso Deus, e ao seu povo Israel. E preparai-vos segundo as
vossas casas paternas e segundo as vossas turmas, conforme à
prescrição de Davi, rei de Israel, e a de Salomão, seu filho.
 E estai no santuário segundo as divisões das casas paternas de
vossos irmãos, os filhos do povo; e haja para cada divisão uma parte
de uma família de levitas. E imolai a páscoa, e santificai-vos,
e preparai-a para vossos irmãos, fazendo conforme a palavra do
Senhor, dada pela mão de Moisés. E ofereceu Josias, aos filhos
do povo, cordeiros e cabritos do rebanho, todos para os sacrifícios
da páscoa, em número de trinta mil, por todos os que ali se achavam,
e de bois três mil; isto era da fazenda do rei. Também
apresentaram os seus príncipes ofertas voluntárias ao povo, aos
sacerdotes e aos levitas: Hilquias, e Zacarias, e Jeiel, líderes da
casa de Deus, deram aos sacerdotes para os sacrifícios da páscoa
duas mil e seiscentas reses\footnote{Rês: qualquer quadrúpede usado
na alimentação humana.} de gado miúdo, e trezentos bois. E
Conanias, e Semaías, e Natanael, seus irmãos, como também Hasabias,
e Jeiel, e Jozabade, chefe dos levitas, apresentaram aos levitas,
para os sacrifícios da páscoa, cinco mil reses de gado miúdo, e
quinhentos bois. Assim se preparou o serviço, e puseram-se os
sacerdotes nos seus postos, e os levitas nas suas turmas, conforme a
ordem do rei, então imolaram a páscoa; e os sacerdotes
espargiram o sangue recebido das mãos dos levitas que esfolavam as
reses. E puseram de parte os holocaustos para os darem aos
filhos do povo, segundo as divisões das casas paternas, para o
oferecerem ao Senhor, como está escrito no livro de Moisés; e assim
fizeram com os bois. E assaram a páscoa no fogo, segundo o
rito; e as ofertas sagradas cozeram em panelas, e em caldeirões e em
sertãs\footnote{Sertã: frigideira larga e rasa.}; e prontamente as
repartiram entre todo o povo. Depois prepararam para si e
para os sacerdotes; porque os sacerdotes, filhos de Arão, se
ocuparam até à noite com o sacrifício dos holocaustos e da gordura;
por isso os levitas prepararam para si e para os sacerdotes, filhos
de Arão. E os cantores, filhos de Asafe, estavam no seu
posto, segundo o mandado de Davi, e de Asafe, e de Hemã, e de
Jedutum, vidente do rei, como também os porteiros a cada porta; não
necessitaram de se desviarem do seu ministério; porquanto seus
irmãos, os levitas, preparavam o necessário para eles. Assim
se estabeleceu todo o serviço do Senhor naquele dia, para celebrar a
páscoa, e oferecer holocaustos sobre o altar do Senhor, segundo a
ordem do rei Josias. E os filhos de Israel que ali se acharam
celebraram a páscoa naquele tempo, e a festa dos pães ázimos,
durante sete dias. Nunca, pois, se celebrou tal páscoa em
Israel, desde os dias do profeta Samuel; nem nenhum rei de Israel
celebrou tal páscoa como a que celebrou Josias com os sacerdotes, e
levitas, e todo o Judá e Israel, que ali se acharam, e os habitantes
de Jerusalém. No décimo oitavo ano do reinado de Josias se
celebrou esta páscoa.

Depois de tudo isto, havendo Josias já preparado o templo, subiu
Neco, rei do Egito, para guerrear contra Carquemis, junto ao
Eufrates; e Josias lhe saiu ao encontro. Então ele lhe mandou
mensageiros, dizendo: Que tenho eu contigo, rei de Judá? Não é
contra ti que venho hoje, mas contra a casa que me faz guerra; e
disse Deus que me apressasse; guarda-te de te opores a Deus, que é
comigo, para que ele não te destrua. Porém Josias não virou
dele o seu rosto, antes se disfarçou, para pelejar contra ele; e não
deu ouvidos às palavras de Neco, que saíram da boca de Deus; antes
veio pelejar no vale de Megido. E os flecheiros atiraram
contra o rei Josias. Então o rei disse a seus servos: Tirai-me
daqui, porque estou gravemente ferido. E seus servos o
tiraram do carro, e o levaram no segundo carro que tinha, e o
trouxeram a Jerusalém; e morreu, e o sepultaram nos sepulcros de
seus pais; e todo o Judá e Jerusalém prantearam a Josias. E
Jeremias fez uma lamentação sobre Josias; e todos os cantores e
cantoras, nas suas lamentações, têm falado de Josias, até ao dia de
hoje; porque as estabeleceram por estatuto em Israel; e eis que
estão escritas nas lamentações. Quanto ao mais dos atos de
Josias, e as suas boas obras, conforme o que está escrito na lei do
Senhor, e os seus atos, tanto os primeiros como os últimos,
eis que estão escritos no livro dos reis de Israel e de Judá.

\medskip

\lettrine{36} Então o povo da terra tomou a Jeoacaz, filho de
Josias, e o fez rei em lugar de seu pai, em Jerusalém. Tinha
Jeoacaz a idade de vinte e três anos, quando começou a reinar; e
três meses reinou em Jerusalém, porque o rei do Egito o depôs em
Jerusalém, e condenou a terra à contribuição de cem talentos de
prata e um talento de ouro. E o rei do Egito pôs a Eliaquim,
irmão de Jeoacaz, rei sobre Judá e Jerusalém, e mudou-lhe o nome em
Jeoiaquim; mas a seu irmão Jeoacaz tomou Neco, e levou-o para o
Egito. Tinha Jeoiaquim vinte e cinco anos de idade, quando
começou a reinar, e reinou onze anos em Jerusalém; e fez o que era
mau aos olhos do Senhor seu Deus. Subiu, pois, contra ele
Nabucodonosor, rei de Babilônia, e o amarrou com cadeias, para o
levar a Babilônia. Também alguns dos vasos da casa do Senhor
levou Nabucodonosor a Babilônia, e pô-los no seu templo em
Babilônia. Quanto ao mais dos atos de Jeoiaquim, e as
abominações que fez, e o mais que se achou nele, eis que estão
escritos no livro dos reis de Israel e de Judá; e Joaquim, seu
filho, reinou em seu lugar. Tinha Joaquim a idade de oito anos,
quando começou a reinar; e reinou três meses e dez dias em
Jerusalém; e fez o que era mau aos olhos do Senhor. E no
decurso de um ano enviou o rei Nabucodonosor, e mandou trazê-lo a
Babilônia, com os mais preciosos vasos da casa do Senhor; e pôs a
Zedequias, seu irmão, rei sobre Judá e Jerusalém.

Tinha Zedequias a idade de vinte e cinco anos, quando começou a
reinar; e onze anos reinou em Jerusalém. E fez o que era mau
aos olhos do Senhor seu Deus; nem se humilhou perante o profeta
Jeremias, que falava da parte do Senhor. Além disto, também
se rebelou contra o rei Nabucodonosor, que o tinha ajuramentado por
Deus. Mas endureceu a sua cerviz, e tanto se obstinou no seu
coração, que não se converteu ao Senhor Deus de Israel.
Também todos os chefes dos sacerdotes e o povo aumentavam de
mais em mais as transgressões, segundo todas as abominações dos
gentios; e contaminaram a casa do Senhor, que ele tinha santificado
em Jerusalém. E o Senhor, Deus de seus pais, falou-lhes
constantemente por intermédio dos mensageiros, porque se compadeceu
do seu povo e da sua habitação. Eles, porém, zombaram dos
mensageiros de Deus, e desprezaram as suas palavras, e
mofaram\footnote{Zombar, troçar, escarnecer.} dos seus profetas; até
que o furor do Senhor tanto subiu contra o seu povo, que mais nenhum
remédio houve. Porque fez subir contra eles o rei dos
caldeus, o qual matou os seus jovens à espada, na casa do seu
santuário, e não teve piedade nem dos jovens, nem das donzelas, nem
dos velhos, nem dos decrépitos; a todos entregou na sua mão.
E todos os vasos da casa de Deus, grandes e pequenos, os
tesouros da casa do Senhor, e os tesouros do rei e dos seus
príncipes, tudo levou para Babilônia. E queimaram a casa de
Deus, e derrubaram os muros de Jerusalém, e todos os seus palácios
queimaram a fogo, destruindo também todos os seus preciosos vasos.
E os que escaparam da espada levou para Babilônia; e
fizeram-se servos dele e de seus filhos, até ao tempo do reino da
Pérsia. Para que se cumprisse a palavra do Senhor, pela boca
de Jeremias, até que a terra se agradasse dos seus sábados; todos os
dias da assolação repousou, até que os setenta anos se cumpriram.

Porém, no primeiro ano de Ciro, rei da Pérsia (para que se
cumprisse a palavra do Senhor pela boca de Jeremias), despertou o
Senhor o espírito de Ciro, rei da Pérsia, o qual fez passar pregão
por todo o seu reino, como também por escrito, dizendo: Assim
diz Ciro, rei da Pérsia: O Senhor Deus dos céus me deu todos os
reinos da terra, e me encarregou de lhe edificar uma casa em
Jerusalém, que está em Judá. Quem há entre vós, de todo o seu povo,
o Senhor seu Deus seja com ele, e suba.

