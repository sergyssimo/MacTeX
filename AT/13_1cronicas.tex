\addchap{Primeiro livro de Crônicas}

\lettrine{1} Adão, Sete, Enos, Cainã, Maalaleel, Jerede,
Enoque, Matusalém, Lameque, Noé, Sem, Cão e Jafé. Os
filhos de Jafé foram: Gomer, Magogue, Madai, Javã, Tubal, Meseque e
Tiras. E os filhos de Gomer: Asquenaz, Rifate, Togarma. E os
filhos de Javã: Elisá, Társis, Quitim e Dodanim. Os filhos de
Cão: Cuxe, Mizraim, Pute e Canaã. E os filhos de Cuxe eram:
Sebá, Havilá, Sabtá, Raamá e Sabtecá; os filhos de Raamá: Sebá e
Dedã. E Cuxe gerou a Ninrode, que começou a ser poderoso na
terra. E Mizraim gerou aos ludeus e aos anameus e aos leabeus
e aos naftueus, e aos patruseus e aos caslueus (dos quais
procedem os filisteus) e aos caftoreus. E Canaã gerou a Sidom
seu primogênito, e a Hete, e aos jebuseus e aos amorreus e
aos girgaseus, e aos heveus e aos arqueus e aos sineus,
e aos arvadeus e aos zemareus e aos hamateus. E foram
os filhos de Sem: Elão, Assur, Arfaxade, Lude, Arã, Uz, Hul, Geter e
Meseque. E Arfaxade gerou a Selá e Selá gerou a Éber.
E a Éber nasceram dois filhos: o nome de um foi Pelegue,
porquanto nos seus dias se repartiu a terra, e o nome de seu irmão
era Joctã. E Joctã gerou a Almoda, a Selefe, a Hazarmavé, e a
Jerá, e a Hadorão, a Usal, e a Dicla, e a Obal, a
Abimael, a Sebá, e a Ofir, a Havilá, e a Jobabe: todos estes
foram filhos de Joctã. Sem, Arfaxade, Selá, Éber,
Pelegue, Reú, Serugue, Naor, Terá, Abrão, que é
Abraão.

Os filhos de Abraão foram: Isaque e Ismael. Estas são as
suas gerações: o primogênito de Ismael foi Nebaiote, e, depois,
Quedar, Adbeel, Mibsão, Misma, Dumá, Massá, Hadade, Tema,
Jetur, Nafis e Quedemá; estes foram os filhos de Ismael.
Quanto aos filhos de Quetura, concubina de Abraão, esta deu à
luz a Zinrã, a Jocsã, a Medã, a Midiã, a Jisbaque e a Suá; e os
filhos de Jocsã foram Seba e Dedã. E os filhos de Midiã: Efá,
Efer, Enoque, Abida e Elda; todos estes foram filhos de Quetura.
Abraão, pois, gerou a Isaque; e foram os filhos de Isaque:
Esaú e Israel. Os filhos de Esaú: Elifaz, Reuel, Jeús, Jalão
e Coré. Os filhos de Elifaz: Temã, Omar, Zefi, Gaetã, Quenaz,
Timna e Amaleque. Os filhos de Reuel: Naate, Zerá, Samá e
Mizá. E os filhos de Seir: Lotã, Sobal, Zibeão, Aná, Disom,
Eser e Disã. E os filhos de Lotã: Hori e Homã; e a irmã de
Lotã foi Timna. Os filhos de Sobal eram Alvã, Manaate, Ebal,
Sefi e Onã; e os filhos de Zibeão eram Aiá e Aná. O filho de
Aná foi Disom; e os filhos de Disom foram Hanrão, Esbã, Itrã e
Querã. Os filhos de Eser eram: Bilã, Zaavã e Jaacã; os filhos
de Disã eram: Uz e Arã. E estes são os reis que reinaram na
terra de Edom, antes que reinasse rei sobre os filhos de Israel:
Bela, filho de Beor, e era o nome da sua cidade Dinabá. E
morreu Bela, e reinou em seu lugar Jobabe, filho de Zerá, de Bozra.
E morreu Jobabe, e reinou em seu lugar Husão, da terra dos
temanitas. E morreu Husão, e reinou em seu lugar Hadade,
filho de Bedade; este feriu os midianitas no campo de Moabe; e era o
nome da sua cidade Avite. E morreu Hadade, e reinou em seu
lugar Samlá, de Masreca. E morreu Samlá, e reinou em seu
lugar Saul, de Reobote, junto ao rio. E morreu Saul, e reinou
em seu lugar Baal-Hanã, filho de Acbor. E, morrendo
Baal-Hanã, Hadade reinou em seu lugar; e era o nome da sua cidade
Paí; e o nome de sua mulher era Meetabel, filha de Matrede, filha de
Me-Zaabe. E, morrendo Hadade, foram príncipes em Edom o
príncipe Timna, o príncipe Alva, o príncipe Jetete, o
príncipe Oolibama, o príncipe Elá, o príncipe Pinom, o
príncipe Quenaz, o príncipe Temã, o príncipe Mibzar, o
príncipe Magdiel, o príncipe Irã, estes foram os príncipes de Edom.

\medskip

\lettrine{2} Estes são os filhos de Israel: Rúben, Simeão,
Levi, Judá, Issacar e Zebulom; Dã, José e Benjamim, Naftali,
Gade e Aser. Os filhos de Judá foram Er, e Onã, e Selá, estes
três lhe nasceram da filha de Suá, a cananéia; e Er, o primogênito
de Judá, foi mau aos olhos do Senhor, pelo que o matou. Porém
Tamar, sua nora, lhe deu à luz Perez e Zerá; todos os filhos de Judá
foram cinco. Os filhos de Perez foram Hezrom e Hamul. E os
filhos de Zerá: Zinri, e Etã, e Hemã, e Calcol, e Dara: cinco ao
todo. E os filhos de Carmi foram Acar, o perturbador de Israel,
que pecou no anátema. E o filho de Etã foi Azarias. E os
filhos de Hezrom, que lhe nasceram, foram Jerameel, e Rão, e
Quelubai. E Rão gerou a Aminadabe, e Aminadabe gerou a
Naassom, príncipe dos filhos de Judá. E Naassom gerou a
Salma, e Salma gerou a Boaz. E Boaz gerou a Obede, e Obede
gerou a Jessé. E Jessé gerou a Eliabe, seu primogênito, e
Abinadabe, o segundo, e Siméia, o terceiro. Natanael, o
quarto, Radai, o quinto. Ozem, o sexto, Davi, o sétimo.
E foram suas irmãs Zeruia e Abigail; e foram os filhos de
Zeruia: Abisai e Joabe, e Asael, três. E Abigail deu à luz a
Amasa; e o pai de Amasa foi Jeter, o ismaelita.

E Calebe, filho de Hezrom, gerou filhos de Azuba, sua mulher, e
de Jeriote; e os filhos desta foram estes: Jeser, Sobabe, e Ardom.
E morreu Azuba; e Calebe tomou para si a Efrate, da qual lhe
nasceu Hur. E Hur gerou a Uri, e Uri gerou a Bezaleel.
Então Hezrom coabitou com a filha de Maquir, pai de Gileade,
e, sendo ele de sessenta anos, a tomou; e ela deu à luz a Segube.
E Segube gerou a Jair; e este tinha vinte e três cidades na
terra de Gileade. E Gesur e Arã tomaram deles as aldeias de
Jair, e Quenate, e seus lugares, sessenta cidades; todos estes foram
filhos de Maquir, pai de Gileade. E, depois da morte de
Hezrom, em Calebe de Efrata, Abia, mulher de Hezrom, deu à luz a
Asur, pai de Tecoa. E os filhos de Jerameel, primogênito de
Hezrom, foram Rão, o primogênito, Buna, Orem, Ozem e Aías.
Teve também Jerameel ainda outra mulher cujo nome era Atara;
esta foi a mãe de Onã. E foram os filhos de Rão, primogênito
de Jerameel: Maaz, Jamim, e Equer. E foram os filhos de Onã:
Samai e Jada; e os filhos de Samai: Nadabe e Abisur. E o nome
da mulher de Abisur era Abiail, que lhe deu a Abã e a Molide.
E foram os filhos de Nadabe, Selede e Apaim; e Selede morreu
sem filhos. E o filho de Apaim foi Isi; e o filho de Isi,
Sesã. E o filho de Sesã, Alai. E os filhos de Jada, irmão de
Samai, foram Jeter e Jônatas; e Jeter morreu sem filhos. E os
filhos de Jônatas foram: Pelete e Zaza; estes foram os filhos de
Jerameel. E Sesã não teve filhos, mas filhas; e tinha Sesã um
servo egípcio, cujo nome era Jará. Deu, pois, Sesã sua filha
por mulher a Jará, seu servo; e lhe deu à luz a Atai. E Atai
gerou a Natã, e Natã gerou a Zabade. E Zabade gerou a Eflal,
e Eflal gerou a Obede. E Obede gerou a Jeú, e Jeú gerou a
Azarias. E Azarias gerou a Helez, e Helez gerou a Eleasá.
E Eleasá gerou a Sismai, e Sismai gerou a Salum. E
Salum gerou a Jecamias, e Jecamias gerou a Elisama. E foram
os filhos de Calebe, irmão de Jerameel, Messa, seu primogênito (este
foi o pai de Zife), e os filhos de Maressa, pai de Hebrom. E
foram os filhos de Hebrom: Coré, Tápua, Requém e Sema. E Sema
gerou a Raão, pai de Jorqueão; e Requém gerou a Samai. E foi
o filho de Samai, Maom; e Maom foi pai de Bete-Zur. E Efá, a
concubina de Calebe, deu à luz a Harã, a Mosa, e a Gazez; e Harã
gerou a Gazez. E foram filhos de Jadai: Regém, Jotão, Gesã,
Pelete, Efá e Saafe. De Maaca, concubina, Calebe gerou a
Seber e a Tiraná. E a mulher de Saafe, pai de Madmana, deu à
luz a Seva, pai de Macbena e pai de Gibeá; e foi a filha de Calebe,
Acsa. Estes foram os filhos de Calebe, filho de Hur, o
primogênito de Efrata: Sobal, pai de Quiriate-Jearim, e
Salma, pai dos belemitas, Harefe, pai de Bete-Gader. E foram
os filhos de Sobal, pai de Quiriate-Jearim: Haroé e metade dos
menuítas. E as famílias de Quiriate-Jearim foram os jitreus,
e os puteus, e os sumateus, e os misraeus; destes saíram os
zorateus, e os estaoleus. Os filhos de Salma foram Belém e os
netofatitas, Atarote, Bete-Joabe, e metade dos manaatitas, e os
zoritas. E as famílias dos escribas que habitavam em Jabez
foram os tiratitas, os simeatitas e os sucatitas; estes são os
queneus, que vieram de Hamate, pai da casa de Recabe.

\medskip

\lettrine{3} E estes foram os filhos de Davi, que lhe nasceram
em Hebrom: o primogênito, Amnom, de Ainoã, a jizreelita; o segundo
Daniel, de Abigail, a carmelita; o terceiro, Absalão, filho de
Maaca, filha de Talmai, rei de Gesur; o quarto, Adonias, filho de
Hagite; o quinto, Sefatias, de Abital; o sexto, Itreão, de Eglá,
sua mulher. Seis filhos lhe nasceram em Hebrom, porque ali
reinou sete anos e seis meses; e trinta e três anos reinou em
Jerusalém. E estes lhe nasceram em Jerusalém: Siméia, e Sobabe,
e Natã, e Salomão; estes quatro lhe nasceram de Bate-Sua, filha de
Amiel. Nasceram-lhe mais Ibar, Elisama, Elifelete, Nogá,
Nefegue, Jafia, Elisama, Eliada, e Elifelete, nove. Todos
estes foram filhos de Davi, afora os filhos das concubinas e Tamar,
irmã deles.

E o filho de Salomão foi Roboão; de quem foi filho Abias; de quem
foi filho Asa; de quem foi filho Jeosafá; de quem foi filho
Jorão; de quem foi filho Acazias; de quem foi filho Joás; de
quem foi filho Amazias; de quem foi filho Azarias\footnote{SBTB
omite: ``de quem foi filho Azarias''. Hélio de Menezes Silva: ``De
quem foi filho Amazias; DE QUEM FOI FILHO AZARIAS, de quem foi filho
Jotão;\ldots'' Como no T. Massorético, na KJV (e até mesmo na
Revista e Atualizada!). Lembrar que Azarias = Ozias = Uzias. RA: de
quem foi filho Amazias, de quem foi filho Azarias, de quem foi filho
Jotão. AV:  Amaziah his son, Azariah his son, Jotham his son.}; de
quem foi filho Jotão; de quem foi filho Acaz; de quem foi
filho Ezequias; de quem foi filho Manassés; de quem foi filho
Amom; de quem foi filho Josias. E os filhos de Josias foram:
o primogênito, Joanã; o segundo, Jeoiaquim; o terceiro, Zedequias; o
quarto, Salum. E os filhos de Jeoiaquim: Jeconias, seu filho,
e Zedequias, seu filho. E os filhos de Jeconias: Assir, e seu
filho Sealtiel. Os filhos deste foram: Malquirão, Pedaías,
Senazar, Jecamias, Hosama, e Nedabias. E os filhos de
Pedaías: Zorobabel e Simei; e os filhos de Zorobabel: Mesulão,
Hananias, e Selomite, sua irmã, e Hasubá, Oel, Berequias,
Hasadias, Jusabe-Hesede, cinco. E os filhos de Hananias:
Pelatias e Jesaías; os filhos de Refaías, os filhos de Arnã, os
filhos de Obadias, e os filhos de Secanias. E o filho de
Secanias foi Semaías; e os filhos de Semaías: Hatus, e Igeal, e
Bariá, e Nearias, e Safate, seis. E os filhos de Nearias:
Elioenai, e Ezequias, e Azricão, três. E os filhos de
Elioenai: Hodavias, Eliasibe, Pelaías, Acube, Joanã, Delaías, e
Anani, sete.

\medskip

\lettrine{4} Os filhos de Judá foram: Perez, Hezrom, Carmi,
Hur, e Sobal. E Reaías, filho de Sobal gerou a Jaate, e Jaate
gerou a Aumai e a Laade; estas são as famílias dos zoratitas. E
estes foram os filhos do pai de Etã: Jizreel, Isma e Idbas; e era o
nome de sua irmã Hazelelponi. E mais Penuel, pai de Gedor, e
Ezer, pai de Husá; estes foram os filhos de Hur, o primogênito de
Efrata, pai de Belém. E tinha Asur, pai de Tecoa, duas mulheres:
Helá e Naará. E Naará deu à luz a Auzão, e a Hefer, e a Temeni,
e a Haastari; estes foram os filhos de Naará. E os filhos de
Helá: Zerete, Izar e Etnã. E Coz gerou a Anube e a Zobeba e as
famílias de Aarel, filho de Harum. E foi Jabez mais ilustre do
que seus irmãos; e sua mãe deu-lhe o nome de Jabez, dizendo:
Porquanto com dores o dei à luz. Porque Jabez invocou o Deus
de Israel, dizendo: Se me abençoares muitíssimo, e meus termos
ampliares, e a tua mão for comigo, e fizeres que do mal não seja
afligido! E Deus lhe concedeu o que lhe tinha pedido.

E Quelube, irmão de Suá, gerou a Meir; este é o pai de Estom.
E Estom gerou a Bete-Rafa, a Pasea, e a Teina, pai de
Ir-Naás; estes foram os homens de Reca. E foram os filhos de
Quenaz: Otniel e Seraías; o filho de Otniel: Hatate. E
Meonotai gerou a Ofra, e Seraías gerou a Joabe, pai dos do vale dos
artífices; porque os dali eram artífices. E foram os filhos
de Calebe, filho de Jefoné: Iru, Elá e Naã; e o filho de Elá:
Quenaz. E os filhos de Jealelel: Zife, Zifa, Tiria e Asareel.
E os filhos de Ezra: Jeter, Merede, Efer, e Jalom; e teve
mais a Miriã, e Samai, e Isbá, pai de Estemoa. E sua mulher,
Judia, deu à luz a Jerede, pai de Gedor, e a Héber, pai de Socó, e a
Jecutiel, pai de Zanoa; e estes foram os filhos de Bitia, filha de
Faraó, que Merede tomou. E foram os filhos da mulher de
Hodias, irmã de Naã: Abiqueila, o garmita, Estemoa, o maacatita.
E os filhos de Simeão: Amom, Rina, Bene-Hanã, e Tilom; e os
filhos de Isi: Zoete e Bene-Zoete. Os filhos de Selá, filho
de Judá: Er, pai de Leca, e Lada, pai de Maressa, e as famílias da
casa dos que fabricavam linho fino, em casa de Asbéia. Como
também Joquim, e os homens de Cozeba, e Joás, e Sarafe (que
dominaram sobre os moabitas), e Jasubi-Leém; porém estas coisas já
são antigas. Estes foram oleiros, e habitavam nas hortas e
nos cerrados; estes ficaram ali com o rei na sua obra.

Os filhos de Simeão foram Nemuel, Jamim, Jaribe, Zerá, e Saul,
cujo filho foi Salum, de quem foi filho Mibsão, de quem foi
filho Misma. E os filhos de Misma foram: Hamuel, de quem foi
filho Zacur, de quem foi filho Simei. E Simei teve dezesseis
filhos, e seis filhas, porém seus irmãos não tiveram muitos filhos;
e toda a sua família não se multiplicou tanto como as dos filhos de
Judá. E habitaram em Berseba, e em Moladá, e em Hazar-Sual,
e em Bila, e em Ezém, e em Tolade, e em Betuel, e em
Hormá, e em Ziclague, e em Bete-Marcabote, e em Hazar-Susim,
e em Bete-Biri, e em Saaraim; estas foram as suas cidades, até que
Davi reinou. E foram as suas aldeias: Etã, Aim, Rimom,
Toquém, e Asã, cinco cidades, e todas as suas aldeias, que
estavam em redor destas cidades, até Baal. Estas foram as suas
habitações e suas genealogias. Porém Mesobabe, e Janleque e
Josa, filho de Amazias, e Joel, e Jeú, filho de Josibias,
filho de Seraías, filho de Asiel, e Elioenai e Jaacobá,
Jesoaías, Asaías, Adiel, Jesimiel, Benaias, e Ziza, filho de
Sifi, filho de Alom, filho de Jedaías, filho de Sinri, filho de
Semaías; estes, registrados por seus nomes, foram príncipes
nas suas famílias; e as famílias de seus pais se multiplicaram
abundantemente. E chegaram até à entrada de Gedor, ao oriente
do vale, a buscar pasto para os seus rebanhos. E acharam
pasto fértil e terra espaçosa, e quieta, e descansada; porque os de
Cão haviam habitado ali antes. Estes, pois, que estão
descritos por seus nomes, vieram nos dias de Ezequias, rei de Judá,
e derrubaram as tendas e habitações dos que se acharam ali, e as
destruíram totalmente até o dia de hoje, e habitaram em seu lugar;
porque ali havia pasto para os seus rebanhos. Também deles,
dos filhos de Simeão, quinhentos homens foram às montanhas de Seir;
levaram por cabeças a Pelatias, e a Nearias, e a Refaías, e a Uziel,
filhos de Isi. E feriram o restante dos que escaparam dos
amalequitas, e habitaram ali até o dia de hoje.

\medskip

\lettrine{5} Quanto aos filhos de Rúben, o primogênito de
Israel (pois ele era o primogênito; mas porque profanara a cama de
seu pai, deu-se a sua primogenitura aos filhos de José, filho de
Israel; de modo que não foi contado, na genealogia da primogenitura,
porque Judá foi poderoso entre seus irmãos, e dele veio o
soberano; porém a primogenitura foi de José). Foram, pois, os
filhos de Rúben, o primogênito de Israel: Enoque, Palu, Hezrom, e
Carmi. Os filhos de Joel: Semaías, seu filho; Gogue, seu filho;
Simei, seu filho; Mica, seu filho; Reaías, seu filho; Baal, seu
filho; Beera, seu filho, o qual Tiglate-Pilneser, rei da
Assíria, levou preso; este foi príncipe dos rubenitas. Quanto a
seus irmãos pelas suas famílias, quando foram postos nas
genealogias, segundo as suas descendências, tiveram por chefes Jeiel
e Zacarias, e Bela, filho de Azaz, filho de Sema, filho de Joel,
que habitou em Aroer, até Nebo e Baal-Meom, também habitou do
lado do oriente, até à entrada do deserto, desde o rio Eufrates;
porque seu gado se tinha multiplicado na terra de Gileade. E
nos dias de Saul fizeram guerra aos hagarenos, que caíram pela sua
mão; e eles habitaram nas suas tendas defronte de todo o lado
oriental de Gileade. E os filhos de Gade habitaram defronte
deles, na terra de Basã, até Salca. Joel foi chefe, e Safã o
segundo; também Janai e Safate estavam em Basã. E seus
irmãos, segundo as suas casas paternas, foram: Micael, Mesulão,
Seba, Jorai, Jacã, Zia, e Éber, sete. Estes foram os filhos
de Abiail filho de Huri, filho de Jaroa, filho de Gileade, filho de
Micael, filho de Jesisai, filho de Jado, filho de Buz; Aí,
filho de Abdiel, filho de Guni, foi chefe da casa de seus pais.
E habitaram em Gileade, em Basã e nos lugares da sua
jurisdição; como também em todos os arrabaldes de Sarom, até aos
seus termos. Todos estes foram registrados, segundo as suas
genealogias, nos dias de Jotão, rei de Judá, e nos dias de Jeroboão,
rei de Israel.

Dos filhos de Rúben, e dos gaditas, e da meia tribo de Manassés,
homens muito valentes, que traziam escudo e espada, e entesavam o
arco, e eram destros na guerra; houve quarenta e quatro mil e
setecentos e sessenta, que saíam à peleja. E fizeram guerra
aos hagarenos, como a Jetur, e a Nafis e a Nodabe. E foram
ajudados contra eles, e os hagarenos e todos quantos estavam com
eles foram entregues em sua mão; porque, na peleja, clamaram a Deus
que lhes deu ouvidos, porquanto confiaram nele. E levaram
preso o seu gado; seus camelos, cinqüenta mil, e duzentas e
cinqüenta mil ovelhas, e dois mil jumentos, e cem mil homens.
Porque muitos caíram feridos, porque de Deus era a peleja; e
habitaram em seu lugar, até ao cativeiro. E os filhos da meia
tribo de Manassés habitaram naquela terra; multiplicaram-se desde
Basã até Baal-Hermom, e Senir, e o monte de Hermom. E estes
foram cabeças de suas casas paternas, a saber: Hefer, Isi, Eliel,
Azriel, Jeremias, Hodavias, e Jadiel, homens valentes, homens de
nome, e chefes das casas de seus pais. Porém transgrediram
contra o Deus de seus pais; e se prostituíram, seguindo os deuses
dos povos da terra, os quais Deus destruíra de diante deles.
Por isso o Deus de Israel suscitou o espírito de Pul, rei da
Assíria, e o espírito de Tiglate-Pilneser, rei da Assíria, que os
levaram presos, a saber: os rubenitas e gaditas, e a meia tribo de
Manassés; e os trouxeram a Hala, e a Habor, e a Hara, e ao rio de
Gozã, até ao dia de hoje.

\medskip

\lettrine{6} Os filhos de Levi foram: Gérson, Coate e Merari,
e os filhos de Coate: Anrão, e Izar, e Hebrom, e Uziel. E os
filhos de Anrão: Arão, Moisés, e Miriã; e os filhos de Arão: Nadabe,
Abiú, Eleazar, e Itamar. E Eleazar gerou a Finéias, e Finéias
gerou a Abisua, e Abisua gerou a Buqui, e Buqui gerou a Uzi,
e Uzi gerou a Zeraías, e Zeraías gerou a Meraiote. E
Meraiote gerou a Amarias, e Amarias gerou a Aitube. E Aitube
gerou a Zadoque, e Zadoque gerou a Aimaás, e Aimaás gerou a
Azarias, e Azarias gerou a Joanã, e Joanã gerou a Azarias; e
este é o que exerceu o sacerdócio na casa que Salomão tinha
edificado em Jerusalém. E Azarias gerou a Amarias, e Amarias
gerou a Aitube, e Aitube gerou a Zadoque, e Zadoque gerou a
Salum, e Salum gerou a Hilquias, e Hilquias gerou a Azarias,
e Azarias gerou a Seraías, e Seraías gerou a Jeozadaque,
e Jeozadaque foi levado cativo quando o Senhor levou presos a
Judá e a Jerusalém pela mão de Nabucodonosor. Os filhos de
Levi foram, pois, Gérson, Coate, e Merari. E estes são os
nomes dos filhos de Gérson: Libni e Simei. E os filhos de
Coate: Anrão, Izar, Hebrom, e Uziel. Os filhos de Merari:
Mali e Musi; estas são as famílias dos levitas, segundo seus pais.
De Gérson: Libni, seu filho; Jaate, seu filho; Zima, seu
filho; Joá, seu filho; Ido, seu filho; Zerá, seu filho;
Jeatarai, seu filho. Os filhos de Coate foram: Aminadabe, seu
filho; Coré, seu filho; Assir, seu filho; Elcana, seu filho;
Ebiasafe, seu filho; Assir, seu filho; Taate, seu filho;
Uriel, seu filho; Uzias, seu filho; e Saul, seu filho. E os
filhos de Elcana: Amasai e Aimote. Quanto a Elcana: os filhos
de Elcana foram Zofai, seu filho; e seu filho Naate. Seu
filho Eliabe, seu filho Jeroão, seu filho Elcana. E os filhos
de Samuel: Joel, seu primogênito, e o segundo Abias. Os
filhos de Merari: Mali, seu filho Libni, seu filho Simei, seu filho
Uzá. Seu filho Siméia, seu filho Hagias, seu filho Asaías.

Estes são, pois, os que Davi constituiu para o ofício do canto na
casa do Senhor, depois que a arca teve repouso. E ministravam
diante do tabernáculo da tenda da congregação com cantares, até que
Salomão edificou a casa do Senhor em Jerusalém; e estiveram, segundo
o seu costume, no seu ministério. Estes são, pois, os que ali
estavam com seus filhos: dos filhos dos coatitas, Hemã, o cantor,
filho de Joel, filho de Samuel, filho de Elcana, filho de
Jeroão, filho de Eliel, filho de Toá, filho de Zufe, filho de
Elcana, filho de Maate, filho de Amasai, filho de Elcana,
filho de Joel, filho de Azarias, filho de Sofonias. Filho de
Taate, filho de Assir, filho de Ebiasafe, filho de Coré,
filho de Isar, filho de Coate, filho de Levi, filho de
Israel. E seu irmão Asafe estava à sua direita; e era Asafe
filho de Berequias, filho de Siméia, filho de Micael, filho
de Baaséias, filho de Malquias, filho de Etni, filho de Zerá,
filho de Adaías, filho de Etã, filho de Zima, filho de Simei,
filho de Jaate, filho de Gérson, filho de Levi. E seus
irmãos, os filhos de Merari, estavam à esquerda; a saber: Etã, filho
de Quisi, filho de Abdi, filho de Maluque, filho de Hasabias,
filho de Amazias, filho de Hilquias, filho de Anzi, filho de
Bani, filho de Semer, filho de Mali, filho de Musi, filho de
Merari, filho de Levi. E seus irmãos, os levitas, foram
postos para todo o ministério do tabernáculo da casa de Deus.
E Arão e seus filhos ofereceram sobre o altar do holocausto e
sobre o altar do incenso, por todo o serviço do lugar santíssimo, e
para fazer expiação por Israel, conforme tudo quanto Moisés, servo
de Deus, tinha ordenado. E estes foram os filhos de Arão: seu
filho Eleazar, seu filho Finéias, seu filho Abisua. Seu filho
Buqui, seu filho Uzi, seu filho Seraías, seu filho Meraiote,
seu filho Amarias, seu filho Aitube, seu filho Zadoque, seu
filho Aimaás.

E estas foram as suas habitações, segundo os seus acampamentos,
nos seus termos, a saber: dos filhos de Arão, da família dos
coatitas, porque a eles caiu a sorte. Deram-lhes, pois, a
Hebrom, na terra de Judá, e os arrabaldes que a rodeiam.
Porém o território da cidade e as suas aldeias deram a
Calebe, filho de Jefoné. E aos filhos de Arão deram as
cidades de refúgio: Hebrom e Libna e os seus arrabaldes, e Jatir e
Estemoa e os seus arrabaldes. E Hilém, e os seus arrabaldes,
Debir e os seus arrabaldes, e Asã e os seus arrabaldes, e
Bete-Semes e os seus arrabaldes. E da tribo de Benjamim, Geba
e os seus arrabaldes, Alemete e os seus arrabaldes, e Anatote e os
seus arrabaldes; todas as suas cidades, pelas suas famílias, foram
treze. Mas os filhos de Coate, que restaram da sua família,
tiveram, por sorte, dez cidades da meia tribo de Manassés. E
os filhos de Gérson, segundo as suas famílias, tiveram treze cidades
da tribo de Issacar, e da tribo de Aser, e da tribo de Naftali e da
tribo de Manassés, em Basã. Os filhos de Merari, segundo as
suas famílias, tiveram, por sorte, doze cidades da tribo de Rúben, e
da tribo de Gade, e da tribo de Zebulom. Assim os filhos de
Israel deram aos levitas estas cidades e os seus arrabaldes.
E deram-lhes por sorte estas cidades, da tribo dos filhos de
Judá, da tribo dos filhos de Simeão, e da tribo dos filhos de
Benjamim, às quais deram os seus nomes. E quanto ao mais das
famílias dos filhos de Coate, se lhes deram, da tribo de Efraim as
cidades dos seus termos. Porque lhes deram as cidades de
refúgio, Siquém e os seus arrabaldes, nas montanhas de Efraim, como
também Gezer e os seus arrabaldes, e Jocmeão e os seus
arrabaldes, Bete-Horom e os seus arrabaldes, e Aijalom e os
seus arrabaldes, Gate-Rimom e os seus arrabaldes. E da meia
tribo de Manassés, Aner e os seus arrabaldes, e Bileã e os seus
arrabaldes; estas cidades tiveram os que ficaram da família dos
filhos de Coate. Os filhos de Gérson tiveram, da família da
meia tribo de Manassés, Golã, em Basã, e os seus arrabaldes, e
Astarote e os seus arrabaldes. E da tribo de Issacar, Quedes
e os seus arrabaldes, e Daberate e os seus arrabaldes. E
Ramote e os seus arrabaldes, e Aném e os seus arrabaldes. E
da tribo de Aser, Masal e os seus arrabaldes, e Abdom e os seus
arrabaldes, e Hucoque e os seus arrabaldes, e Reobe e os seus
arrabaldes. E da tribo de Naftali, Quedes, em Galiléia, e os
seus arrabaldes, Hamom e os seus arrabaldes e Quiriataim e os seus
arrabaldes. Os que ficaram dos filhos de Merari tiveram, da
tribo de Zebulom, a Rimom e os seus arrabaldes, a Tabor e os seus
arrabaldes. E dalém do Jordão, na altura de Jericó, ao
oriente do Jordão, da tribo de Rúben, a Bezer, no deserto, e os seus
arrabaldes, e a Jaza e os seus arrabaldes, e a Quedemote e os
seus arrabaldes, e a Mefaate e os seus arrabaldes. E da tribo
de Gade, a Ramote, em Gileade, e os seus arrabaldes, e Maanaim e os
seus arrabaldes, e a Hesbom e os seus arrabaldes, e a Jazer e
os seus arrabaldes.

\medskip

\lettrine{7} E quanto aos filhos de Issacar, foram: Tola, Pua,
Jasube e Sinrom, quatro. E os filhos de Tola foram: Uzi,
Refaías, Jeriel, Jamai, Ibsão e Semuel, chefes das casas de seus
pais, descendentes de Tola, homens valentes nas suas gerações; o seu
número, nos dias de Davi, foi de vinte e dois mil e seiscentos.
E o filho de Uzi: Izraías; e os filhos de Izraías foram: Mical,
Obadias, Joel e Issias; todos estes cinco chefes. E houve com
eles nas suas gerações, segundo as suas casas paternas, em tropas de
guerra, trinta e seis mil; porque tiveram muitas mulheres e filhos.
E seus irmãos, em todas as famílias de Issacar, homens valentes,
foram oitenta e sete mil, todos contados pelas suas genealogias.
Os filhos de Benjamim foram: Belá, e Bequer, e Jediael, três.
E os filhos de Belá: Esbom, e Uzi, e Uziel, e Jerimote, e Iri,
cinco chefes da casa dos pais, homens valentes que foram contados
pelas suas genealogias, vinte e dois mil e trinta e quatro. E os
filhos de Bequer: Zemira, Joás, Eliezer, Elioenai, Onri, Jerimote,
Abias, Anatote, e Alemete; todos estes foram filhos de Bequer. E
foram contados pelas suas genealogias, segundo as suas gerações, e
chefes das casas de seus pais, homens valentes, vinte mil e
duzentos. E foi o filho de Jediael, Bilã; e os filhos de Bilã
foram Jeús, Benjamim, Eude, Quenaaná, Zetã, Társis e Aisaar.
Todos estes filhos de Jediael foram chefes das famílias dos
pais, homens valentes, dezessete mil e duzentos, que saíam no
exército à peleja. E Supim, e Hupim, filhos de Ir, e Husim,
dos filhos de Aer. Os filhos de Naftali: Jaziel, e Guni, e
Jezer, e Salum, filhos de Bila. Os filhos de Manassés:
Asriel, que a mulher de Gileade gerou (porém a sua concubina, a
síria, gerou a Maquir, pai de Gileade; e Maquir tomou a irmã
de Hupim e Supim por mulher, e era o seu nome Maaca), e foi o nome
do segundo Zelofeade; e Zelofeade teve filhas. E Maaca,
mulher de Maquir, deu à luz um filho, e chamou-o Perez; e o nome de
seu irmão foi Seres; e foram seus filhos Ulão e Raquém. E o
filho de Ulão, Bedã; estes foram os filhos de Gileade, filho de
Maquir, filho de Manassés. E quanto à sua irmã Hamolequete,
teve a Is-Hode, a Abiezer, e a Maalá. E foram os filhos de
Semida: Aiã, Siquém, Liqui, e Anião.

E os filhos de Efraim: Sutela, e seu filho Berede, e seu filho
Taate, e seu filho Elada e seu filho Taate. E seu filho
Zabade, e seu filho Sutela, e Ezer, e Elade; e os homens de Gate,
naturais da terra, os mataram, porque desceram para tomar os seus
gados. Por isso Efraim, seu pai, por muitos dias os chorou; e
vieram seus irmãos para o consolar. Depois coabitou com sua
mulher, e ela concebeu, e teve um filho; e chamou-o Berias; porque
ia mal na sua casa. E sua filha foi Seerá, que edificou a
Bete-Horom, a baixa e a alta, como também a Uzém-Seerá. E foi
seu filho Refa, e Resefe, de quem foi filho Tela, de quem foi filho
Taã, de quem foi filho Ladã, de quem foi filho Amiúde, de
quem foi filho Elisama, de quem foi filho Num, de quem foi
filho Josué. E foi a sua possessão e habitação Betel e os
lugares da sua jurisdição; e ao oriente Naarã, e ao ocidente Gezer e
os lugares da sua jurisdição, e Siquém e os lugares da sua
jurisdição, até Gaza e os lugares da sua jurisdição; e do
lado dos filhos de Manassés, Bete-Seã e os lugares da sua
jurisdição, Taanaque e os lugares da sua jurisdição, Megido e os
lugares da sua jurisdição, Dor e os lugares da sua jurisdição;
nestas habitaram os filhos de José, filho de Israel. Os
filhos de Aser foram: Imná, Isvá, Isvi, Berias, e Sera, irmã deles.
E os filhos de Berias: Héber e Malquiel; este foi o pai de
Birzavite. E Héber gerou a Jaflete, e a Somer, e a Hotão, e a
Suá, irmã deles. E foram os filhos de Jaflete: Pasaque, e
Bimal e Asvate; estes foram os filhos de Jaflete. E os filhos
de Semer: Ai, Roga, Jeubá, e Arã. E os filhos de seu irmão
Helém: Zofa, e Imna, e Seles, e Amal. Os filhos de Zofa: Suá,
e Harnefer, e Sual, e Beri, e Inra, Bezer, Hode, Samá, Silsa,
Itrã, e Beera. E os filhos de Jeter: Jefoné, Pispa e Ara.
E os filhos de Ula: Ará e Haniel e Rizia. Todos estes
foram filhos de Aser, chefes das casas paternas, homens escolhidos e
valentes, chefes dos príncipes, e contados nas suas genealogias, no
exército para a guerra; foi seu número de vinte e seis mil homens.

\medskip

\lettrine{8} E Benjamim gerou a Belá, seu primogênito, a Asbel
o segundo, e a Aará o terceiro, a Noá o quarto, e a Rafa o
quinto. E Belá teve estes filhos: Adar, Gera, Abiúde,
Abisua, Naamã, Aoá, Gera, Sefufá e Hurão. E estes foram
os filhos de Eúde; que foram chefes dos pais dos moradores de Geba,
e os levaram cativos a Manaate: e Naamã, e Aías e Gera; este os
transportou, e gerou a Uzá e a Aiúde. E Saaraim (depois de os
enviar), na terra de Moabe, gerou filhos de Husim e Baara, suas
mulheres. E de Hodes, sua mulher, gerou a Jobabe, a Zibia, a
Mesa, a Malcã, a Jeuz, a Saquias e a Mirma; estes foram seus
filhos, chefes dos pais. E de Husim gerou a Abitude e a
Elpaal. E foram os filhos de Elpaal: Éber, Misã e Semede;
este edificou a Ono e a Lode e os lugares da sua jurisdição.
E Berias e Sema foram cabeças dos pais dos moradores de
Aijalom; estes afugentaram os moradores de Gate. E Aiô,
Sasaque, Jerimote, Zebadias, Arade, Eder, Micael, Ispa
e Joa foram filhos de Berias. Zebadias, Mesulão, Hizque,
Héber, Ismerai, Izlias e Jobabe, filhos de Elpaal.
Jaquim, Zicri, Zabdi, Elienai, Ziletai, Eliel,
Adaías, Beraías e Sinrate, filhos de Simei. E Ispã,
Éber, Eliel, Abdom, Zicri, Hanã, Hananias, Elão,
Antotias, e Ifdéias, e Penuel, filhos de Sasaque; e
Sanserai, e Searias, e Atalias, e Jaaresias, e Elias e Zicri,
filhos de Jeroão. Estes foram cabeças dos pais, segundo as
suas gerações, chefes, e habitaram em Jerusalém. E em Gibeão
habitou o pai de Gibeão; e era o nome de sua mulher Maaca; e
seu filho primogênito, Abdom; depois Zur, e Quis, Baal, e Nadabe,
e Gedor, Aiô, e Zequer, e Miclote gerou a Siméia; e
também estes, defronte de seus irmãos, habitaram em Jerusalém com
eles.

E Ner gerou a Quis, e Quis gerou a Saul; e Saul gerou a Jônatas,
a Malquisua, a Abinadabe, e a Esbaal. E filho de Jônatas foi
Meribe-Baal; e Meribe-Baal gerou a Mica. E os filhos de Mica
foram: Pitom, Meleque, Tareá, e Acaz. E Acaz gerou a Jeoada;
e Jeoada gerou a Alemete, e a Azmavete, e a Zinri; e Zinri gerou a
Moza, e Moza gerou a Bineá, cujo filho foi Rafa, de quem foi
filho Eleazá, cujo filho foi Azel. E teve Azel seis filhos, e
estes foram os seus nomes: Azricão, Bocru, Ismael, Searias, Obadias,
e Hanã; todos estes foram filhos de Azel. E os filhos de
Ezeque, seu irmão: Ulão, seu primogênito, Jeús o segundo e Elifelete
o terceiro. E foram os filhos de Ulão homens heróis,
valentes, e flecheiros destros; e tiveram muitos filhos, e filhos de
filhos, cento e cinqüenta; todos estes foram dos filhos de Benjamim.

\medskip

\lettrine{9} E todo o Israel foi contado por genealogias, que
estão escritas no livro dos reis de Israel; e os de Judá foram
transportados a Babilônia, por causa da sua transgressão. E os
primeiros habitantes, que moravam na sua possessão e nas suas
cidades, foram os israelitas, os sacerdotes, os levitas, e os
netineus. Porém alguns dos filhos de Judá, e dos filhos de
Benjamim, e dos filhos de Efraim e Manassés, habitaram em Jerusalém:
Utai, filho de Amiúde, filho de Onri, filho de Inri, filho de
Bani, dos filhos de Perez, filho de Judá; e dos silonitas:
Asaías o primogênito, e seus filhos; e dos filhos de Zerá:
Jeuel, e seus irmãos, seiscentos e noventa; e dos filhos de
Benjamim: Salu, filho de Mesulão, filho de Hodavias, filho de
Hassenua, e Ibnéias, filho de Jeroão, e Elá, filho de Uzi, filho
de Micri, e Mesulão, filho de Sefatias, filho de Reuel, filho de
Ibnijas; e seus irmãos, segundo as suas gerações, novecentos e
cinqüenta e seis; todos estes homens foram chefes dos pais nas casas
de seus pais. E dos sacerdotes: Jedaías, e Jeoiaribe, e
Jaquim, e Azarias, filho de Hilquias, filho de Mesulão, filho
de Zadoque, filho de Meraiote, filho de Aitube, maioral da casa de
Deus; Adaías, filho de Jeroão, filho de Pasur, filho de
Malquias, e Masai, filho de Adiel, filho de Jazera, filho de
Mesulão, filho de Mesilemite, filho de Imer; como também seus
irmãos, cabeças nas casas de seus pais, mil setecentos e sessenta,
homens valentes para a obra do ministério da casa de Deus.

E dos levitas: Semaías, filho de Hassube, filho de Azricão, filho
de Hasabias, dos filhos de Merari; e Baquebacar, Heres e
Galal; e Matanias, filho de Mica, filho de Zicri, filho de Asafe;
e Obadias, filho de Semaías, filho de Galal, filho de
Jedutum; e Berequias, filho de Asa, filho de Elcana, morador das
aldeias dos netofatitas. E foram porteiros: Salum, Acube,
Talmom, Aimã, e seus irmãos, cujo chefe era Salum. E até
aquele tempo estavam de guarda à porta do rei, do lado do oriente;
estes foram os porteiros dos arraiais dos filhos de Levi. E
Salum, filho de Coré, filho de Ebiasafe, filho de Corá, e seus
irmãos da casa de seu pai, os coraítas, tinham cargo da obra do
ministério, e eram guardas das portas do tabernáculo, como seus pais
foram responsáveis pelo arraial do Senhor, e guardas da entrada.
Finéias, filho de Eleazar, antes era líder entre eles; e o
Senhor era com ele. E Zacarias, filho de Meselemias, porteiro
da entrada da tenda da congregação. Todos estes, escolhidos
para serem guardas das portas, foram duzentos e doze; e foram estes,
segundo as suas aldeias, postos em suas genealogias; e Davi e
Samuel, o vidente, os constituíram nos seus respectivos cargos.
Estavam, pois, eles, e seus filhos, às portas da casa do
Senhor, na casa da tenda, junto aos guardas, os porteiros
estavam aos quatro lados; ao oriente, ao ocidente, ao norte, e ao
sul. E seus irmãos, que estavam nas suas aldeias, deviam, de
tempo em tempo, vir por sete dias para servirem com eles.
Porque havia naquele ofício quatro porteiros principais que
eram levitas, e tinham o encargo das câmaras e dos tesouros da casa
de Deus. E de noite ficavam em redor da casa de Deus, cuja
guarda lhes tinha sido confiada, e tinham o encargo de abri-la cada
manhã. E alguns deles estavam encarregados dos utensílios do
ministério, porque por conta os traziam e por conta os tiravam.
Porque deles havia alguns que tinham o encargo dos objetos e
de todos os utensílios do santuário; como também da flor de farinha,
do vinho, do azeite, do incenso, e das especiarias. E alguns
dos filhos dos sacerdotes eram os obreiros da confecção das
especiarias. E Matitias, dentre os levitas, o primogênito de
Salum, o coraíta, tinha o encargo da obra que se fazia em
sertãs\footnote{Assadeiras.}. E alguns dos seus irmãos, dos
filhos dos coatitas, tinham o encargo de preparar os pães da
proposição para todos os sábados. Destes foram também os
cantores, chefes dos pais entre os levitas, habitando nas câmaras,
isentos de serviços; porque de dia e de noite estava a seu cargo
ocuparem-se naquela obra. Estes foram cabeças dos pais entre
os levitas, chefes em suas gerações; estes habitaram em Jerusalém.

Porém em Gibeão habitaram Jeiel, pai de Gibeão (e era o nome de
sua mulher Maaca). E seu filho primogênito Abdom; depois Zur,
Quis, Baal, Ner e Nadabe, e Gedor, Aiô, Zacarias e Miclote.
Miclote gerou a Simeão; e também estes habitaram em
Jerusalém, defronte de seus irmãos, com eles. E Ner gerou a
Quis; e Quis gerou a Saul, Saul gerou a Jônatas, a Malquisua, a
Abinadabe e a Esbaal. E o filho de Jônatas foi Meribe-Baal, e
Meribe-Baal gerou a Mica. E os filhos de Mica foram: Pitom,
Meleque e Taréia. E Acaz gerou a Jaerá, e Jaerá gerou a
Alemete, a Azmavete e a Zinri; e Zinri gerou a Moza. E Moza
gerou a Bineá, cujo filho foi Refaías, de quem foi filho Eleasá,
cujo filho foi Azel. E teve Azel seis filhos, e estes foram
os seus nomes: Azricão, Bocru, Ismael, Seraías, Obadias e Hanã;
estes foram os filhos de Azel.

\medskip

\lettrine{10} E os filisteus pelejaram com Israel; e os homens
de Israel fugiram de diante dos filisteus, e caíram mortos nas
montanhas de Gilboa. E os filisteus perseguiram a Saul e aos
seus filhos e mataram a Jônatas, a Abinadabe e a Malquisua, filhos
de Saul. E a peleja se agravou contra Saul, e os flecheiros o
alcançaram; e temeu muito aos flecheiros. Então disse Saul ao
seu escudeiro: Arranca a tua espada, e atravessa-me com ela; para
que porventura não venham estes incircuncisos e escarneçam de mim.
Porém o seu escudeiro não quis, porque temia muito; então tomou Saul
a espada, e se lançou sobre ela. Vendo, pois, o seu escudeiro
que Saul estava morto, também ele se lançou sobre a espada e morreu.
Assim morreram Saul e seus três filhos; e toda a sua casa morreu
juntamente. E, vendo todos os homens de Israel, que estavam no
vale, que haviam fugido, e que Saul e seus filhos eram mortos,
deixaram as suas cidades, e fugiram; então vieram os filisteus, e
habitaram nelas.

E sucedeu que, no dia seguinte, vindo os filisteus a despojar os
mortos, acharam a Saul e a seus filhos estirados nas montanhas de
Gilboa. E o despojaram, e tomaram a sua cabeça e as suas armas,
e as enviaram pela terra dos filisteus em redor, para o anunciarem a
seus ídolos e ao povo. E puseram as suas armas na casa do seu
deus, e a sua cabeça afixaram na casa de Dagom. Ouvindo,
pois, toda a Jabes de Gileade tudo quanto os filisteus fizeram a
Saul, então todos os homens valorosos se levantaram, e
tomaram o corpo de Saul, e os corpos de seus filhos, e os trouxeram
a Jabes; e sepultaram os seus ossos debaixo de um carvalho em Jabes,
e jejuaram sete dias. Assim morreu Saul por causa da
transgressão que cometeu contra o Senhor, por causa da palavra do
Senhor, a qual não havia guardado; e também porque buscou a
adivinhadora para a consultar. E não buscou ao Senhor, que
por isso o matou, e transferiu o reino a Davi, filho de Jessé.

\medskip

\lettrine{11} Então todo o Israel se ajuntou a Davi em Hebrom,
dizendo: Eis que somos teus ossos e tua carne. E também outrora,
sendo Saul ainda rei, eras tu o que fazias sair e entrar a Israel;
também o Senhor teu Deus te disse: Tu apascentarás o meu povo
Israel, e tu serás chefe sobre o meu povo Israel. Também vieram
todos os anciãos de Israel ao rei, a Hebrom, e Davi fez com eles
aliança em Hebrom, perante o Senhor; e ungiram a Davi rei sobre
Israel, conforme a palavra do Senhor pelo ministério de Samuel.
E partiu Davi e todo o Israel para Jerusalém, que é Jebus;
porque ali estavam os jebuseus, habitantes da terra. E disseram
os habitantes de Jebus a Davi: Tu não entrarás aqui. Porém Davi
ganhou a fortaleza de Sião, que é a cidade de Davi. Porque disse
Davi: Qualquer que primeiro ferir os jebuseus será chefe e capitão.
Então Joabe, filho de Zeruia, subiu primeiro a ela; pelo que foi
feito chefe. E Davi habitou na fortaleza; por isso foi chamada a
cidade de Davi. E edificou a cidade ao redor, desde Milo até ao
circuito; e Joabe renovou o restante da cidade. E Davi
tornava-se cada vez mais forte; porque o Senhor dos Exércitos era
com ele.

E estes foram os chefes dos poderosos que Davi tinha, e que o
apoiaram fortemente no seu reino, com todo o Israel, para o fazerem
rei, conforme a palavra do Senhor, no tocante a Israel. E
este é o número dos poderosos que Davi tinha: Jasobeão, hacmonita,
chefe dos capitães, o qual, brandindo a sua lança contra trezentos,
de uma vez os matou. E, depois dele Eleazar, filho de Dodó, o
aoíta; ele estava entre os três poderosos. Este esteve com
Davi em Pas-Damim, quando os filisteus ali se ajuntaram à peleja,
onde havia um pedaço de campo cheio de cevada; e o povo fugiu de
diante dos filisteus. E puseram-se no meio daquele campo, e o
defenderam, e feriram os filisteus; e o Senhor efetuou um grande
livramento. E três dos trinta capitães desceram à penha, a
ter com Davi, na caverna de Adulão; e o exército dos filisteus
estava acampado no vale de Refaim. E Davi estava então no
lugar forte; e o alojamento dos filisteus estava então em Belém.
E desejou Davi, e disse: Quem me dera beber da água do poço
de Belém, que está junto à porta! Então aqueles três romperam
pelo acampamento dos filisteus, e tiraram água do poço de Belém, que
estava junto à porta, e tomaram dela e a trouxeram a Davi; porém
Davi não a quis beber, mas a derramou ao Senhor, e disse:
Nunca meu Deus permita que faça tal! Beberia eu o sangue destes
homens com as suas vidas? Pois com perigo das suas vidas a
trouxeram. E ele não a quis beber. Isto fizeram aqueles três homens.
E também Abisai, irmão de Joabe, era chefe de três, o qual,
brandindo a sua lança contra trezentos, os feriu; e teve nome entre
os três. Ele foi o mais ilustre dos três, pelo que foi
capitão deles; porém não igualou aos primeiros três. Também
Benaia, filho de Joiada, filho de um homem poderoso de Cabzeel,
grande em obras; ele feriu a dois heróis de Moabe; e também desceu,
e feriu um leão dentro de uma cova, no tempo da neve. Também
feriu ele a um homem egípcio, homem de grande altura, de cinco
côvados; e trazia o egípcio uma lança na mão, como o órgão do
tecelão; mas Benaia desceu contra ele com uma vara, e arrancou a
lança da mão do egípcio, e com ela o matou. Estas coisas fez
Benaia, filho de Joiada; pelo que teve nome entre aqueles três
poderosos. Eis que dos trinta foi ele o mais ilustre; contudo
não chegou aos primeiros três; e Davi o pôs sobre os da sua guarda.
E foram os poderosos dos exércitos: Asael, irmão de Joabe,
El-Hanã, filho de Dodó, de Belém; Samote, o harorita; Helez,
o pelonita; Ira, filho de Iques, o tecoíta; Abiezer, o
anatotita; Sibecai, o husatita; Ilai, o aoíta; Maarai,
o netofatita; Helede, filho de Baaná, o netofatita; Itai,
filho de Ribai, de Gileade, dos filhos de Benjamim; Benaia, o
piratonita; Hurai, do ribeiro de Gaás; Abiel, o arbatita;
Azmavete, o baarumita; Eliabe, o saalbonita; dos
filhos de Hasem, o gizonita: Jônatas, filho de Sage, o hararita;
Aião, filho de Sacar, o hararita; Elifal, filho de Ur;
Hefer, o mequeratita; Aías, o pelonita; Hezro, o
carmelita; Naarai, filho de Ezbai; Joel, irmão de Natã;
Mibar, filho de Hagri; Zeleque, o amonita; Naarai, o
beerotita, escudeiro de Joabe, filho de Zeruia; Ira, o
itrita; Garebe, o itrita; Urias, o heteu; Zabade, filho de
Alai; Adina, filho de Siza, o rubenita, capitão dos
rubenitas, e com ele trinta; Hanã, filho de Maaca; e Josafá,
o mitatita; Uzias, o asteratita; Sama e Jeiel, filhos de
Hotão, o aroerita; Jediael, filho de Sinri; e Joa, seu irmão,
o tizita; Eliel, o maavita; e Jeribai e Josavias, filhos de
Elnaão; e Itma, o moabita; Eliel, Obede, e Jaasiel, o
mesobaíta.

\medskip

\lettrine{12} Estes, porém, são os que vieram a Davi, a
Ziclague, estando ele ainda escondido, por causa de Saul, filho de
Quis; e eram dos valentes que o ajudaram na guerra. Estavam
armados de arco, e usavam tanto da mão direita como da esquerda em
atirar pedras e em atirar flechas com o arco; eram dos irmãos de
Saul, benjamitas. Aiezer, o chefe, e Joás, filho de Semaa, o
gibeatita, e Jeziel e Pelete, filhos de Azmavete; e Beraca, e Jeú, o
anatotita, e Ismaías, o gibeonita, valente entre os trinta,
líder deles; e Jeremias, e Jaaziel, e Joanã, e Jozabade, o
gederatita, Eluzai, e Jerimote, e Bealias, e Samarias, e
Sefatias, o harufita, Elcana, Issias, Azarel, Joezer, e
Jasobeão, os coraítas, e Joela, e Zabadias, filhos de Jeroão de
Gedor. E dos gaditas se desertaram para Davi, ao lugar forte no
deserto, valentes, homens de guerra para pelejar, armados com escudo
e lança; e seus rostos eram como rostos de leões, e ligeiros como
corças sobre os montes: Ezer, o primeiro; Obadias, o segundo;
Eliabe, o terceiro; Mismana, o quarto; Jeremias, o quinto;
Atai, o sexto; Eliel, o sétimo; Joanã, o oitavo;
Elzabade, o nono; Jeremias, o décimo; Macbanai, o undécimo;
estes, dos filhos de Gade, foram os capitães do exército; o
menor tinha o encargo de cem homens e o maior de mil. Estes
são os que passaram o Jordão no primeiro mês, quando ele
transbordava por todas as suas ribanceiras, e fizeram fugir a todos
os dos vales ao oriente e ao ocidente. Também alguns dos
filhos de Benjamim e de Judá vieram a Davi, ao lugar forte. E
Davi lhes saiu ao encontro, e lhes falou, dizendo: Se vós vindes a
mim pacificamente e para me ajudar, o meu coração se unirá convosco;
porém, se é para me entregar aos meus inimigos, sem que haja
deslealdade nas minhas mãos, o Deus de nossos pais o veja e o
repreenda. Então veio o espírito sobre Amasai, chefe de
trinta, e disse: Nós somos teus, ó Davi, e contigo estamos, ó filho
de Jessé! Paz, paz contigo, e paz com quem te ajuda, pois que teu
Deus te ajuda. E Davi os recebeu, e os fez capitães das tropas.
Também de Manassés alguns passaram para Davi, quando veio com
os filisteus para a batalha contra Saul; todavia Davi não os ajudou,
porque os príncipes dos filisteus, tendo feito conselho, o
despediram, dizendo: À custa de nossas cabeças passará a Saul, seu
senhor. Voltando ele, pois, a Ziclague, passaram-se para ele,
de Manassés, Adna, Jozabade, Jediael, Micael, Jozabade, Eliú, e
Ziletai, capitães de milhares dos de Manassés. E estes
ajudaram a Davi contra aquela tropa, porque todos eles eram heróis
poderosos, e foram capitães no exército. Porque naquele
tempo, dia após dia, vinham a Davi para o ajudar, até que se fez um
grande exército, como o exército de Deus.

Ora este é o número dos chefes armados para a peleja, que vieram
a Davi em Hebrom, para transferir a ele o reino de Saul, conforme a
palavra do Senhor. Dos filhos de Judá, que traziam escudo e
lança, seis mil e oitocentos, armados para a peleja; dos
filhos de Simeão, homens poderosos para pelejar, sete mil e cem;
dos filhos de Levi, quatro mil e seiscentos. Joiada,
que era o líder dos de Arão, e com ele três mil e setecentos.
E Zadoque, sendo ainda jovem, homem poderoso, com vinte e
dois capitães da família de seu pai; e dos filhos de
Benjamim, irmãos de Saul, três mil; porque até então havia ainda
muitos deles que eram pela casa de Saul. E dos filhos de
Efraim, vinte mil e oitocentos homens poderosos, homens de nome nas
casas de seus pais. E da meia tribo de Manassés, dezoito mil,
que foram apontados pelos seus nomes para virem fazer rei a Davi.
E dos filhos de Issacar, duzentos de seus chefes, destros na
ciência dos tempos, para saberem o que Israel devia fazer, e todos
os seus irmãos seguiam suas ordens. De Zebulom, dos que
podiam sair no exército, cinqüenta mil ordenados para a peleja com
todas as armas de guerra; como também destros para ordenarem uma
batalha, e não eram de coração dobre\footnote{Fingido, enganoso,
falso.}. E de Naftali, mil capitães, e com eles trinta e sete
mil com escudo e lança. E dos danitas, ordenados para a
peleja, vinte e oito mil e seiscentos. E de Aser, dos que
podiam sair no exército, para ordenarem a batalha, quarenta mil.
E do outro lado do Jordão, dos rubenitas e gaditas, e da meia
tribo de Manassés, com toda a sorte de instrumentos de guerra para
pelejar, cento e vinte mil. Todos estes homens de guerra,
postos em ordem de batalha, vieram a Hebrom, com corações decididos,
para constituírem a Davi rei sobre todo o Israel; e também todo o
restante de Israel tinha o mesmo coração para constituir a Davi rei.
E estiveram ali com Davi três dias, comendo e bebendo; porque
seus irmãos lhes tinham preparado as provisões. E também seus
vizinhos de mais perto, até Issacar, e Zebulom, e Naftali,
trouxeram, sobre jumentos, e sobre camelos, e sobre mulos, e sobre
bois, pão, provisões de farinha, pastas de figos e cachos de passas,
e vinho, e azeite, e bois, gado miúdo em abundância; porque havia
alegria em Israel.

\medskip

\lettrine{13} E Davi tomou conselho com os capitães dos
milhares, e das centenas, e com todos os líderes. E disse Davi a
toda a congregação de Israel: Se bem vos parece, e se isto vem do
Senhor nosso Deus, enviemos depressa mensageiros a todos os nossos
outros irmãos em todas as terras de Israel, e aos sacerdotes, e aos
levitas nas suas cidades e nos seus arrabaldes, para que se reúnam
conosco; e tornemos a trazer para nós a arca do nosso Deus;
porque não a buscamos nos dias de Saul. Então disse toda a
congregação que se fizesse assim; porque este negócio pareceu reto
aos olhos de todo o povo. Convocou, pois, Davi a todo o Israel
desde Sior do Egito até chegar a Hamate; para trazer a arca de Deus
de Quiriate-Jearim. E então Davi com todo o Israel subiu a Baalá
de Quiriate-Jearim, que está em Judá, para fazer subir dali a arca
de Deus, o Senhor que habita entre os querubins, sobre a qual é
invocado o seu nome. E levaram a arca de Deus, da casa de
Abinadabe, sobre um carro novo; e Uzá e Aiô guiavam o carro. E
Davi e todo o Israel, alegraram-se perante Deus com todas as suas
forças; com cânticos, e com harpas, e com saltérios, e com
tamborins, e com címbalos, e com trombetas.

E, chegando à eira de Quidom, estendeu Uzá a sua mão, para segurar
a arca, porque os bois tropeçavam. Então se acendeu a ira do
Senhor contra Uzá, e o feriu, por ter estendido a sua mão à arca; e
morreu ali perante Deus. E Davi se encheu de tristeza porque
o Senhor havia aberto brecha em Uzá; pelo que chamou aquele lugar
Perez-Uzá, até ao dia de hoje. E aquele dia temeu Davi a
Deus, dizendo: Como trarei a mim a arca de Deus? Por isso
Davi não trouxe a arca a si, à cidade de Davi; porém a fez levar à
casa de Obede-Edom, o giteu. Assim ficou a arca de Deus com a
família de Obede-Edom, três meses em sua casa; e o Senhor abençoou a
casa de Obede-Edom, e tudo quanto tinha.

\medskip

\lettrine{14} Então Hirão, rei de Tiro, mandou mensageiros a
Davi, e madeira de cedro, e pedreiros, e carpinteiros, para lhe
edificarem uma casa. E entendeu Davi que o Senhor o tinha
confirmado rei sobre Israel; porque o seu reino tinha sido muito
exaltado por amor do seu povo Israel. E Davi tomou ainda mais
mulheres em Jerusalém; e gerou Davi ainda mais filhos e filhas.
E estes são os nomes dos filhos que teve em Jerusalém: Samua,
Sobabe, Natã, Salomão, e Ibar, Elisua, Elpelete, e Nogá,
Nefegue, Jafia, e Elisama, Eliada, e Elifelete.

Ouvindo, pois, os filisteus que Davi havia sido ungido rei sobre
todo o Israel, todos os filisteus subiram em busca de Davi; o que
ouvindo Davi, logo saiu contra eles. E vindo os filisteus, se
estenderam pelo vale de Refaim. Então consultou Davi a Deus,
dizendo: Subirei contra os filisteus, e nas minhas mãos os
entregarás? E o Senhor lhe disse: Sobe, porque os entregarei nas
tuas mãos. E, subindo a Baal-Perazim, Davi ali os feriu; e
disse Davi: Por minha mão Deus derrotou a meus inimigos, como se
rompem as águas. Pelo que chamaram aquele lugar, Baal-Perazim.
E deixaram ali seus deuses; e ordenou Davi que se queimassem
a fogo; porém os filisteus tornaram, e se estenderam pelo
vale. E tornou Davi a consultar a Deus; e disse-lhe Deus: Não
subirás atrás deles; mas rodeia-os por detrás, e vem a eles por
defronte das amoreiras; e há de ser que, ouvindo tu um ruído
de marcha pelas copas das amoreiras, então sairás à peleja; porque
Deus terá saído diante de ti, para ferir o exército dos filisteus.
E fez Davi como Deus lhe ordenara; e feriram o exército dos
filisteus desde Gibeom até Gezer. Assim se espalhou o nome de
Davi por todas aquelas terras; e o Senhor pôs o temor dele sobre
todas aquelas nações.

\medskip

\lettrine{15} Davi também fez casa para si na cidade de Davi;
e preparou um lugar para a arca de Deus, e armou-lhe uma tenda.
Então disse Davi: Ninguém pode levar a arca de Deus, senão os
levitas; porque o Senhor os escolheu, para levar a arca de Deus, e
para o servirem eternamente. E Davi convocou a todo o Israel em
Jerusalém, para fazer subir a arca do Senhor ao seu lugar, que lhe
tinha preparado. E Davi reuniu os filhos de Arão e os levitas:
Dos filhos de Coate: Uriel, o chefe, e de seus irmãos cento e
vinte. Dos filhos de Merari: Asaías, o chefe, e de seus irmãos
duzentos e vinte. Dos filhos de Gérson: Joel, o chefe, e de seus
irmãos cento e trinta. Dos filhos de Elizafã: Semaías, o chefe,
e de seus irmãos duzentos. Dos filhos de Hebrom: Eliel, o chefe,
e de seus irmãos oitenta. Dos filhos de Uziel: Aminadabe, o
chefe, e de seus irmãos cento e doze. E chamou Davi os
sacerdotes Zadoque e Abiatar, e os levitas, Uriel, Asaías, Joel,
Semaías, Eliel, e Aminadabe. E disse-lhes: Vós sois os chefes
dos pais entre os levitas; santificai-vos, vós e vossos irmãos, para
que façais subir a arca do Senhor Deus de Israel, ao lugar que lhe
tenho preparado. Porquanto vós não a levastes na primeira
vez, o Senhor nosso Deus fez rotura em nós, porque não o buscamos
segundo a ordenança. Santificaram-se, pois, os sacerdotes e
os levitas, para fazerem subir a arca do Senhor Deus de Israel.
E os filhos dos levitas trouxeram a arca de Deus sobre os
seus ombros, pelas varas que nela havia, como Moisés tinha ordenado
conforme a palavra do Senhor. E disse Davi aos chefes dos
levitas que constituíssem, de seus irmãos, cantores, para que com
instrumentos musicais, com alaúdes, harpas e címbalos, se fizessem
ouvir, levantando a voz com alegria. Designaram, pois, os
levitas a Hemã, filho de Joel; e dos seus irmãos, Asafe, filho de
Berequias; e dos filhos de Merari, seus irmãos, Etã, filho de
Cusaías. E com eles a seus irmãos da segunda ordem: a
Zacarias, Bene, Jaaziel, Semiramote, Jeiel, Uni, Eliabe, Benaia,
Maaséias, Matitias, Elifeleu, Micnéias, Obede-Edom, e Jeiel, os
porteiros. E os cantores, Hemã, Asafe e Etã, se faziam ouvir
com címbalos de metal; e Zacarias, Aziel, Semiramote, Jeiel,
Uni, Eliabe, Maaséias, e Benaia, com alaúdes, sobre Alamote:
E Matitias, Elifeleu, Micnéias, Obede-Edom, Jeiel, e Azazias,
com harpas, sobre Seminite, para sobressaírem. E Quenanias,
chefe dos levitas, tinha o encargo de dirigir o canto; ensinava-os a
entoá-lo, porque era entendido. E Berequias e Elcana eram
porteiros da arca. E Sebanias, Jeosafá, Netanel, Amasai,
Zacarias, Benaia, e Eliezer, os sacerdotes, tocavam as trombetas
perante a arca de Deus; e Obede-Edom e Jeías eram porteiros da arca.

Sucedeu, pois, que Davi e os anciãos de Israel, e os capitães dos
milhares, foram, com alegria, para fazer subir a arca da aliança do
Senhor, da casa de Obede-Edom. E sucedeu que, ajudando Deus
os levitas que levavam a arca da aliança do Senhor, sacrificaram
sete novilhos e sete carneiros. E Davi ia vestido de um manto
de linho fino, como também todos os levitas que levavam a arca, e os
cantores, e Quenanias, mestre dos cantores; também Davi levava sobre
si um éfode de linho. E todo o Israel fez subir a arca da
aliança do Senhor, com júbilo, e ao som de buzinas, e de trombetas,
e de címbalos, fazendo ressoar alaúdes e harpas. E sucedeu
que, chegando a arca da aliança do Senhor à cidade de Davi, Mical, a
filha de Saul, olhou de uma janela, e, vendo a Davi dançar e tocar,
o desprezou no seu coração.

\medskip

\lettrine{16} Trouxeram, pois, a arca de Deus, e a puseram no
meio da tenda que Davi lhe tinha armado; e ofereceram holocaustos e
sacrifícios pacíficos perante Deus. E, acabando Davi de oferecer
os holocaustos e sacrifícios pacíficos, abençoou o povo em nome do
Senhor. E repartiu a todos em Israel, tanto a homens como a
mulheres, a cada um, um pão, e um bom pedaço de carne, e um frasco
de vinho. E pôs alguns dos levitas por ministros perante a arca
do Senhor; isto para recordarem, e louvarem, e celebrarem ao Senhor
Deus de Israel. Era Asafe, o chefe, e Zacarias o segundo depois
dele; Jeiel, e Semiramote, e Jeiel, e Matitias, e Eliabe, e Benaia,
e Obede-Edom, e Jeiel, com alaúdes e com harpas; e Asafe se fazia
ouvir com címbalos; também Benaia, e Jaaziel, os sacerdotes,
continuamente tocavam trombetas, perante a arca da aliança de Deus.

Então naquele mesmo dia Davi, em primeiro lugar, deu o seguinte
salmo para que, pelo ministério de Asafe e de seus irmãos, louvassem
ao Senhor. Louvai ao Senhor, invocai o seu nome, fazei
conhecidas as suas obras entre os povos. Cantai-lhe,
salmodiai-lhe, atentamente falai de todas as suas maravilhas.
Gloriai-vos no seu santo nome; alegre-se o coração dos que
buscam ao Senhor. Buscai ao Senhor e a sua força; buscai a
sua face continuamente. Lembrai-vos das maravilhas que fez,
de seus prodígios, e dos juízos da sua boca; vós, semente de
Israel, seus servos, vós, filhos de Jacó, seus escolhidos.
Ele é o Senhor nosso Deus; os seus juízos estão em toda a
terra. Lembrai-vos perpetuamente da sua aliança e da palavra
que prescreveu para mil gerações; da aliança que fez com
Abraão, e do seu juramento a Isaque; o qual também a Jacó
confirmou por estatuto, e a Israel por aliança eterna,
dizendo: A ti te darei a terra de Canaã, quinhão da vossa
herança. Quando eram poucos homens em número, sim, mui
poucos, e estrangeiros nela, quando andavam de nação em
nação, e de um reino para outro povo, a ninguém permitiu que
os oprimisse, e por amor deles repreendeu reis, dizendo: Não
toqueis os meus ungidos, e aos meus profetas não façais mal.
Cantai ao Senhor em toda a terra; anunciai de dia em dia a
sua salvação. Contai entre as nações a sua glória, entre
todos os povos as suas maravilhas. Porque grande é o Senhor,
e mui digno de louvor, e mais temível é do que todos os deuses.
Porque todos os deuses dos povos são ídolos; porém o Senhor
fez os céus. Louvor e majestade há diante dele, força e
alegria no seu lugar. Tributai ao Senhor, ó famílias dos
povos, tributai ao Senhor glória e força. Tributai ao Senhor
a glória de seu nome; trazei presentes, e vinde perante ele; adorai
ao Senhor na beleza da sua santidade. Trema perante ele,
trema toda a terra; pois o mundo se firmará, para que não se abale.
Alegrem-se os céus, e regozije-se a terra; e diga-se entre as
nações: O Senhor reina. Brame o mar com a sua plenitude;
exulte o campo com tudo o que nele há; então jubilarão as
árvores dos bosques perante o Senhor; porquanto vem julgar a terra.
Louvai ao Senhor, porque é bom; pois a sua benignidade dura
perpetuamente. E dizei: Salva-nos, ó Deus da nossa salvação,
e ajunta-nos, e livra-nos das nações, para que louvemos o teu santo
nome, e nos gloriemos no teu louvor. Bendito seja o Senhor
Deus de Israel, de eternidade a eternidade. E todo o povo disse:
Amém! E louvou ao Senhor.

Então Davi deixou ali, diante da arca da aliança do Senhor, a
Asafe e a seus irmãos, para ministrarem continuamente perante a
arca, segundo se ordenara para cada dia. E mais a Obede-Edom,
com seus irmãos, sessenta e oito; a este Obede-Edom, filho de
Jedutum, e a Hosa, deixou por porteiros. E deixou a Zadoque,
o sacerdote, e a seus irmãos, os sacerdotes, diante do tabernáculo
do Senhor, no alto que está em Gibeom, para oferecerem
holocaustos ao Senhor continuamente, pela manhã e à tarde, sobre o
altar dos holocaustos; e isto segundo tudo o que está escrito na lei
do Senhor que tinha prescrito a Israel. E com eles a Hemã, e
a Jedutum, e aos mais escolhidos, que foram apontados pelos seus
nomes, para louvarem ao Senhor, porque a sua benignidade dura
perpetuamente. Com eles, pois, estavam Hemã e Jedutum, com
trombetas e címbalos, para os que haviam de tocar, e com outros
instrumentos de música de Deus; porém os filhos de Jedutum estavam à
porta. Então todo o povo se retirou, cada um para a sua casa;
e voltou Davi, para abençoar a sua casa.

\medskip

\lettrine{17} Sucedeu, pois, que, morando Davi já em sua casa,
disse ao profeta Natã: Eis que moro em casa de cedro, mas a arca da
aliança do Senhor está debaixo de cortinas. Então Natã disse a
Davi: Tudo quanto tens no teu coração faze, porque Deus é contigo.
Mas sucedeu, na mesma noite, que a palavra de Deus veio a Natã,
dizendo: Vai, e dize a Davi meu servo: Assim diz o Senhor: Tu
não me edificarás uma casa para eu morar; porque em casa nenhuma
morei, desde o dia em que fiz subir a Israel até ao dia de hoje; mas
fui de tenda em tenda, e de tabernáculo em tabernáculo. Por
todas as partes por onde andei com todo o Israel, porventura falei
alguma palavra a algum dos juízes de Israel, a quem ordenei que
apascentasse o meu povo, dizendo: Por que não me edificais uma casa
de cedro? Agora, pois, assim dirás a meu servo Davi: Assim diz o
Senhor dos Exércitos: Eu te tirei do curral, de detrás das ovelhas,
para que fosses chefe do meu povo Israel. E estive contigo por
toda a parte, por onde foste, e de diante de ti exterminei todos os
teus inimigos, e te fiz um nome como o nome dos grandes que estão na
terra, e ordenarei um lugar para o meu povo Israel, e o
plantarei, para que habite no seu lugar, e nunca mais seja removido
de uma para outra parte; e nunca mais os filhos da perversidade o
debilitarão como dantes, e desde os dias em que ordenei
juízes sobre o meu povo Israel. Assim abaterei a todos os teus
inimigos; também te faço saber que o Senhor te edificará uma casa.
E há de ser que, quando forem cumpridos os teus dias, para
ires a teus pais, suscitarei a tua descendência depois de ti, um dos
teus filhos, e estabelecerei o seu reino. Este me edificará
casa; e eu confirmarei o seu trono para sempre. Eu lhe serei
por pai, e ele me será por filho; e a minha benignidade não
retirarei dele, como a tirei daquele, que foi antes de ti.
Mas o confirmarei na minha casa e no meu reino para sempre, e
o seu trono será firme para sempre. Conforme todas estas
palavras, e conforme toda esta visão, assim falou Natã a Davi.

Então entrou o rei Davi, e ficou perante o Senhor; e disse: Quem
sou eu, Senhor Deus? e qual é a minha casa, para que me tenhas
trazido até aqui? E ainda isto, ó Deus, foi pouco aos teus
olhos; pelo que falaste da casa de teu servo para tempos distantes;
e trataste-me como a um homem ilustre, ó Senhor Deus. Que
mais te dirá Davi, acerca da honra feita a teu servo? Porém tu
conheces bem a teu servo. Ó Senhor, por amor de teu servo, e
segundo o teu coração, fizeste toda esta grandeza, para fazer
notória todas estas grandes coisas. Senhor, ninguém há como
tu, e não há Deus fora de ti, segundo tudo quanto ouvimos com os
nossos ouvidos. E quem há como o teu povo Israel, única gente
na terra, a quem Deus foi resgatar para seu povo, fazendo-te nome
com coisas grandes e temerosas, lançando as nações de diante do teu
povo, que resgataste do Egito? E confirmaste o teu povo
Israel para ser teu povo para sempre; e tu, Senhor, lhe foste por
Deus. Agora, pois, Senhor, a palavra que falaste de teu
servo, e acerca da sua casa, confirma-a para sempre; e faze como
falaste. Confirme-se e engrandeça-se o teu nome para sempre,
e diga-se: O Senhor dos Exércitos é o Deus de Israel, é Deus para
Israel; e permaneça firme diante de ti a casa de Davi, teu servo.
Porque tu, Deus meu, revelaste ao ouvido de teu servo que lhe
edificarias casa; pelo que o teu servo achou confiança para orar em
tua presença. Agora, pois, Senhor, tu és o mesmo Deus, e
falaste este bem acerca de teu servo. Agora, pois, foste
servido abençoar a casa de teu servo, para que permaneça para sempre
diante de ti: porque tu, Senhor, a abençoaste, e ficará abençoada
para sempre.

\medskip

\lettrine{18} E depois disto aconteceu que Davi derrotou os
filisteus, e os sujeitou; e tomou a Gate, e os lugares da sua
jurisdição, da mão dos filisteus. Também derrotou os moabitas; e
os moabitas ficaram por servos de Davi, pagando tributos. Também
Davi derrotou a Hadar-Ezer, rei de Zobá, junto a Hamate, quando ele
ia estabelecer o seu domínio sobre o rio Eufrates. E Davi lhe
tomou mil carros, e sete mil cavaleiros, e vinte mil homens de pé; e
Davi jarretou todos os cavalos dos carros; porém reservou deles para
cem carros. E vieram os sírios de Damasco a socorrer a
Hadar-Ezer, rei de Zobá; porém Davi feriu dos sírios vinte e dois
mil homens. E Davi pôs guarnições na Síria de Damasco, e os
sírios ficaram por servos de Davi, pagando-lhe tributo; e o Senhor
guardava a Davi, por onde quer que ia. E Davi tomou os escudos
de ouro, que tinham os servos de Hadar-Ezer, e os trouxe a
Jerusalém. Também de Tibate, e de Cum, cidades de Hadar-Ezer,
tomou Davi muitíssimo cobre, de que Salomão fez o mar de cobre, e as
colunas, e os utensílios de cobre.

E ouvindo Toí, rei de Hamate, que Davi destruíra todo o exército
de Hadar-Ezer, rei de Zobá, mandou seu filho Hadorão a Davi,
para lhe perguntar como estava, e para o abençoar, por haver
pelejado contra Hadar-Ezer, e por havê-lo ferido (porque Hadar-Ezer
fazia guerra a Toí), enviando-lhe também toda a sorte de vasos de
ouro, e de prata, e de cobre. Os quais Davi também consagrou
ao Senhor, juntamente com a prata e ouro que trouxera de todas as
demais nações: dos edomeus, e dos moabitas, e dos filhos de Amom, e
dos filisteus, e dos amalequitas. Também Abisai, filho de
Zeruia, feriu a dezoito mil edomeus no Vale do Sal. E pôs
guarnições em Edom, e todos os edomeus ficaram por servos de Davi; e
o Senhor guardava a Davi, por onde quer que ia. E Davi reinou
sobre todo o Israel; e fazia juízo e justiça a todo o seu povo.
E Joabe, filho de Zeruia, comandava o exército; Jeosafá,
filho de Ailude, era cronista. E Zadoque, filho de Aitube, e
Abimeleque, filho de Abiatar, eram sacerdotes; e Savsa escrivão.
E Benaia, filho de Joiada, estava sobre os quereteus e
peleteus; porém os filhos de Davi, eram os primeiros junto ao rei.

\medskip

\lettrine{19} E aconteceu, depois disso, que Naás, rei dos
filhos de Amom, morreu; e seu filho reinou em seu lugar. Então
disse Davi: Usarei de benevolência com Hanum, filho de Naás, porque
seu pai usou de benevolência comigo. Por isso Davi enviou
mensageiros para o consolarem acerca de seu pai. E, chegando os
servos de Davi à terra dos filhos de Amom, a Hanum, para o
consolarem, disseram os príncipes dos filhos de Amom a Hanum:
Pensas, porventura, que foi para honrar teu pai aos teus olhos, que
Davi te mandou consoladores? Não vieram seus servos a ti, a
esquadrinhar, e a transtornar, e a espiar a terra? Por isso
Hanum tomou os servos de Davi, e raspou-os, e cortou-lhes as vestes
no meio até à coxa da perna, e os despediu. E foram-se, e
avisaram a Davi acerca daqueles homens; e enviou ele mensageiros a
encontrá-los; porque aqueles homens estavam sobremaneira
envergonhados. Disse, pois, o rei: Deixai-vos ficar em Jericó, até
que vos torne a crescer a barba, e então voltai.

Vendo, pois, os filhos de Amom que se tinham feito odiosos para
com Davi, enviou Hanum, e os filhos de Amom, mil talentos de prata
para alugarem para si carros e cavaleiros da Mesopotâmia, e da Síria
de Maaca, e de Zobá. E alugaram para si trinta e dois mil
carros, e o rei de Maaca e o seu povo, e eles vieram, e se acamparam
diante de Medeba; também os filhos de Amom se ajuntaram das suas
cidades, e vieram para a guerra. O que ouvindo Davi, enviou
Joabe e todo o exército dos homens valentes. E, saindo os filhos
de Amom, ordenaram a batalha à porta da cidade; porém os reis que
vieram se puseram à parte no campo. E, vendo Joabe que a
batalha estava preparada contra ele, pela frente e pela retaguarda,
separou dentre os mais escolhidos de Israel, e os ordenou contra os
sírios. E o resto do povo entregou na mão de Abisai, seu
irmão; e puseram-se em ordem de batalha contra os filhos de Amom.
E disse: Se os sírios forem mais fortes do que eu, tu virás
socorrer-me; e, se os filhos de Amom forem mais fortes do que tu,
então eu te socorrerei. Esforça-te, e esforcemo-nos pelo
nosso povo, e pelas cidades do nosso Deus, e faça o Senhor o que
parecer bem aos seus olhos. Então se chegou Joabe, e o povo
que tinha consigo, diante dos sírios, para a batalha; e fugiram de
diante dele. Vendo, pois, os filhos de Amom que os sírios
fugiram, também eles fugiram de diante de Abisai, seu irmão, e
entraram na cidade; e veio Joabe para Jerusalém. E, vendo os
sírios que foram derrotados diante de Israel, enviaram mensageiros,
e fizeram sair os sírios que habitavam do outro lado do rio; e
Sofaque, capitão do exército de Hadar-Ezer, marchava diante deles.
Do que avisado Davi, ajuntou a todo o Israel, e passou o
Jordão, e foi ter com eles, e ordenou contra eles a batalha; e,
tendo Davi ordenado a batalha contra os sírios, pelejaram contra
ele. Porém os sírios fugiram de diante de Israel, e feriu
Davi, dos sírios, os homens de sete mil carros, e quarenta mil
homens de pé; e a Sofaque, capitão do exército, matou. Vendo,
pois, os servos de Hadar-Ezer que tinham sido feridos diante de
Israel, fizeram paz com Davi, e o serviram; e os sírios nunca mais
quiseram socorrer os filhos de Amom.

\medskip

\lettrine{20} Aconteceu que, no decurso de um ano, no tempo em
que os reis costumam sair para a guerra, Joabe levou o exército, e
destruiu a terra dos filhos de Amom, e veio, e cercou a Rabá; porém
Davi ficou em Jerusalém; e Joabe feriu a Rabá, e a destruiu. E
Davi tirou a coroa da cabeça do rei deles, e achou nela o peso de um
talento de ouro, e havia nela pedras preciosas; e foi posta sobre a
cabeça de Davi; e levou da cidade mui grande despojo. Também
levou o povo que estava nela, e os fez trabalhar com a serra, e com
talhadeiras de ferro e com machados; e assim fez Davi com todas as
cidades dos filhos de Amom; então voltou Davi, com todo o povo, para
Jerusalém.

E, depois disto, aconteceu que, levantando-se guerra em Gezer, com
os filisteus, então Sibecai, o husatita, feriu a Sipai, dos filhos
do gigante; e ficaram subjugados. E tornou a haver guerra com os
filisteus; e El-Hanã, filho de Jair, feriu a Lami, irmão de Golias,
o giteu, cuja haste da lança era como órgão de tecelão. E houve
ainda outra guerra em Gate; onde havia um homem de grande estatura,
e tinha vinte e quatro dedos, seis em cada mão, e seis em cada pé, e
que também era filho do gigante. E injuriou a Israel; porém
Jônatas, filho de Simei, irmão de Davi, o feriu. Estes nasceram
ao gigante em Gate; e caíram pela mão de Davi e pela mão dos seus
servos.

\medskip

\lettrine{21} Então Satanás se levantou contra Israel, e
incitou Davi a numerar a Israel. E disse Davi a Joabe e aos
maiorais do povo: Ide, numerai a Israel, desde Berseba até Dã; e
trazei-me a conta para que saiba o número deles. Então disse
Joabe: O Senhor acrescente ao seu povo cem vezes tanto como é;
porventura, ó rei meu senhor, não são todos servos de meu senhor?
Por que procura isto o meu senhor? Porque seria isto causa de delito
para com Israel. Porém a palavra do rei prevaleceu contra Joabe;
por isso saiu Joabe, e passou por todo o Israel; então voltou para
Jerusalém. E Joabe deu a Davi a soma do número do povo; e era
todo o Israel um milhão e cem mil homens, dos que arrancavam da
espada; e de Judá quatrocentos e setenta mil homens, dos que
arrancavam da espada. Porém os de Levi e Benjamim não contou
entre eles, porque a palavra do rei foi abominável a Joabe.

E este negócio também pareceu mau aos olhos de Deus; por isso
feriu a Israel. Então disse Davi a Deus: Gravemente pequei em
fazer este negócio; porém agora sê servido tirar a iniqüidade de teu
servo, porque procedi mui loucamente. Falou, pois, o Senhor a
Gade, o vidente de Davi, dizendo: Vai, e fala a Davi,
dizendo: Assim diz o Senhor: Três coisas te proponho; escolhe uma
delas, para que eu ta faça. E Gade veio a Davi, e lhe disse:
Assim diz o Senhor: Escolhe para ti, ou três anos de fome, ou
que três meses sejas consumido diante dos teus adversários, e a
espada de teus inimigos te alcance, ou que três dias a espada do
Senhor, isto é, a peste na terra, e o anjo do Senhor destrua todos
os termos de Israel; vê, pois, agora, que resposta hei de levar a
quem me enviou. Então disse Davi a Gade: Estou em grande
angústia; caia eu, pois, nas mãos do Senhor, porque são muitíssimas
as suas misericórdias; mas que eu não caia nas mãos dos homens.
Mandou, pois, o Senhor a peste a Israel; e caíram de Israel
setenta mil homens. E Deus mandou um anjo a Jerusalém para a
destruir; e, destruindo-a ele, o Senhor olhou, e se arrependeu
daquele mal, e disse ao anjo destruidor: Basta, agora retira a tua
mão. E o anjo do Senhor estava junto à eira de Ornã, o jebuseu.
E, levantando Davi os seus olhos, viu o anjo do Senhor, que
estava entre a terra e o céu, com a sua espada desembainhada na sua
mão estendida contra Jerusalém; então Davi e os anciãos, cobertos de
sacos, se prostraram sobre os seus rostos. E disse Davi a
Deus: Não sou eu o que disse que se contasse o povo? E eu mesmo sou
o que pequei, e fiz muito mal; mas estas ovelhas que fizeram? Ah!
Senhor, meu Deus, seja a tua mão contra mim, e contra a casa de meu
pai, e não para castigo de teu povo.

Então o anjo do Senhor ordenou a Gade que dissesse a Davi para
subir e levantar um altar ao Senhor na eira de Ornã, o jebuseu.
Subiu, pois, Davi, conforme a palavra de Gade, que falara em
nome do Senhor. E, virando-se Ornã, viu o anjo, e
esconderam-se seus quatro filhos que estavam com ele; e Ornã estava
trilhando o trigo. E Davi veio a Ornã; e olhou Ornã, e viu a
Davi, e saiu da eira, e se prostrou perante Davi com o rosto em
terra. E disse Davi a Ornã: Dá-me este lugar da eira, para
edificar nele um altar ao Senhor; dá-mo pelo seu valor, para que
cesse este castigo sobre o povo. Então disse Ornã a Davi:
Toma-o para ti, e faça o rei meu senhor dele o que parecer bem aos
seus olhos; eis que dou os bois para holocaustos, e os
trilhos\footnote{Utensílio de lavoura para debulhar cereais.} para
lenha, e o trigo para oferta de alimentos; tudo dou. E disse
o rei Davi a Ornã: Não, antes, pelo seu valor, a quero comprar;
porque não tomarei o que é teu, para o Senhor, para que não ofereça
holocausto sem custo. E Davi deu a Ornã, por aquele lugar, o
peso de seiscentos siclos de ouro. Então Davi edificou ali um
altar ao Senhor, e ofereceu nele holocaustos e sacrifícios
pacíficos; e invocou o Senhor, o qual lhe respondeu com fogo do céu
sobre o altar do holocausto. E o Senhor deu ordem ao anjo, e
ele tornou a sua espada à bainha. Vendo Davi, no mesmo tempo,
que o Senhor lhe respondera na eira de Ornã, o jebuseu, sacrificou
ali. Porque o tabernáculo do Senhor, que Moisés fizera no
deserto, e o altar do holocausto, estavam naquele tempo no alto de
Gibeom. E não podia Davi ir perante ele consultar a Deus;
porque estava aterrorizado por causa da espada do anjo do Senhor.

\medskip

\lettrine{22} E disse Davi: Esta será a casa do Senhor Deus, e
este será o altar do holocausto para Israel. E deu ordem Davi
que se ajuntassem os estrangeiros que estavam na terra de Israel; e
ordenou cortadores de pedras, para que lavrassem pedras de
cantaria\footnote{Pedra para construção, esquadrejada segundo as
normas de estereotomia; alistão.}, para edificar a casa de Deus.
E aparelhou Davi ferro em abundância, para os pregos das portas
das entradas, e para as junturas; como também cobre em abundância,
que não foi pesado; e madeira de cedro sem conta; porque os
sidônios e tírios traziam a Davi madeira de cedro em abundância.
Porque dizia Davi: Salomão, meu filho, ainda é moço e tenro, e a
casa que se há de edificar para o Senhor deve ser magnífica em
excelência, para nome e glória em todas as terras; eu, pois, agora
lhe prepararei materiais. Assim preparou Davi materiais em
abundância, antes da sua morte.

Então chamou a Salomão seu filho, e lhe ordenou que edificasse uma
casa ao Senhor Deus de Israel. E disse Davi a Salomão: Filho
meu, quanto a mim, tive em meu coração o propósito de edificar uma
casa ao nome do Senhor meu Deus. Porém, veio a mim a palavra do
Senhor, dizendo: Tu derramaste sangue em abundância, e fizeste
grandes guerras; não edificarás casa ao meu nome; porquanto muito
sangue tens derramado na terra, perante mim. Eis que o filho que
te nascer será homem de repouso; porque repouso lhe hei de dar de
todos os seus inimigos ao redor; portanto, Salomão será o seu nome,
e paz e descanso darei a Israel nos seus dias. Ele edificará
uma casa ao meu nome, e me será por filho, e eu lhe serei por pai, e
confirmarei o trono de seu reino sobre Israel, para sempre.
Agora, pois, meu filho, o Senhor seja contigo; e prospera, e
edifica a casa do Senhor teu Deus, como ele disse de ti. O
Senhor te dê tão-somente prudência e entendimento, e te instrua
acerca de Israel; e isso para guardar a lei do Senhor teu Deus.
Então prosperarás, se tiveres cuidado de cumprir os estatutos
e os juízos, que o Senhor mandou a Moisés acerca de Israel;
esforça-te, e tem bom ânimo; não temas, nem tenhas pavor. Eis
que na minha aflição preparei para a casa do Senhor cem mil talentos
de ouro, e um milhão de talentos de prata, e de cobre e de ferro que
não se pesou, porque era em abundância; também madeira e pedras
preparei, e tu suprirás o que faltar. Também tens contigo
obreiros em grande número, cortadores e artífices em obra de pedra e
madeira; e toda a sorte de peritos em toda a espécie de obra.
Do ouro, da prata, e do cobre, e do ferro não há conta.
Levanta-te, pois, e faze a obra, e o Senhor seja contigo.

E Davi deu ordem a todos os príncipes de Israel que ajudassem a
Salomão, seu filho, dizendo: Porventura não está convosco o
Senhor vosso Deus, e não vos deu repouso ao redor? Porque entregou
na minha mão os moradores da terra; e a terra foi sujeita perante o
Senhor e perante o seu povo. Disponde, pois, agora o vosso
coração e a vossa alma para buscardes ao Senhor vosso Deus; e
levantai-vos, e edificai o santuário do Senhor Deus, para que a arca
da aliança do Senhor, e os vasos sagrados de Deus se tragam a esta
casa, que se há de edificar ao nome do Senhor.

\medskip

\lettrine{23} Sendo, pois, Davi já velho, e cheio de dias, fez
a Salomão, seu filho, rei sobre Israel. E reuniu a todos os
príncipes de Israel, como também aos sacerdotes e levitas. E
foram contados os levitas de trinta anos para cima; e foi o número
deles, segundo as suas cabeças, trinta e oito mil homens. Destes
havia vinte e quatro mil, para promoverem a obra da casa do Senhor,
e seis mil oficiais e juízes, e quatro mil porteiros, e quatro
mil para louvarem ao Senhor com os instrumentos, que eu fiz para o
louvar, disse Davi. E Davi os repartiu por turnos, segundo os
filhos de Levi, Gérson, Coate e Merari. Dos gersonitas: Ladã e
Simei. Os filhos de Ladã: Jeiel, o chefe, e Zetã, e Joel, três.
Os filhos de Simei: Selomite, Haziel, e Harã, três; estes foram
os chefes dos pais de Ladã. E os filhos de Simei: Jaate,
Ziza, Jeús, e Berias; estes foram os filhos de Simei, quatro.
E Jaate era o chefe, e Ziza o segundo, mas Jeús e Berias não
tiveram muitos filhos; por isso estes, sendo contados juntos, se
tornaram uma só família. Os filhos de Coate: Anrão, Izar,
Hebrom, e Uziel, quatro. Os filhos de Anrão: Arão e Moisés; e
Arão foi separado para santificar o santo dos santos, ele e seus
filhos, eternamente; para incensar diante do Senhor, para o
servirem, e para darem a bênção em seu nome eternamente. E,
quanto a Moisés, homem de Deus, seus filhos foram contados entre os
da tribo de Levi. Foram, pois, os filhos de Moisés, Gérson e
Eliézer. Dos filhos de Gérson foi Sebuel o chefe. E,
quanto aos filhos de Eliézer, foi Reabias o chefe; e Eliézer não
teve outros filhos; porém os filhos de Reabias foram muitos.
Dos filhos de Izar foi Selomite o chefe. Quanto aos
filhos de Hebrom, foram Jerias o primeiro, Amarias o segundo,
Jaaziel o terceiro, e Jecameão o quarto. Quanto aos filhos de
Uziel, Mica o chefe, e Issias o segundo. Os filhos de Merari:
Mali, e Musi; os filhos de Mali: Eleazar e Quis. E morreu
Eleazar, e não teve filhos, porém filhas; e os filhos de Quis, seus
parentes, as tomaram por mulheres. Os filhos de Musi: Mali, e
Eder, e Jeremote, três.

Estes são os filhos de Levi, segundo a casa de seus pais, chefes
dos pais, conforme foram contados pelos seus nomes, segundo as suas
cabeças, que faziam a obra do ministério da casa do Senhor, desde a
idade de vinte anos para cima. Porque disse Davi: O Senhor
Deus de Israel deu repouso ao seu povo, e habitará em Jerusalém para
sempre. E também, quanto aos levitas, que nunca mais levassem
o tabernáculo, nem algum de seus aparelhos pertencentes ao seu
ministério. Porque, segundo as últimas palavras de Davi,
foram contados os filhos de Levi da idade de vinte anos para cima.
Porque o seu cargo era assistir aos filhos de Arão no
ministério da casa do Senhor, nos átrios, e nas câmaras, e na
purificação de todas as coisas sagradas, e na obra do ministério da
casa de Deus. A saber: para os pães da proposição, e para a
flor de farinha, para a oferta de alimentos, e para os coscorões
ázimos, e para as sertãs, e para o tostado, e para todo o peso e
medida; e para estarem cada manhã em pé para louvarem e
celebrarem ao Senhor; e semelhantemente à tarde; e para
oferecerem os holocaustos do Senhor, aos sábados, nas luas novas, e
nas solenidades, segundo o seu número e costume, continuamente
perante o Senhor; e para que tivessem cuidado da guarda da
tenda da congregação, e da guarda do santuário, e da guarda dos
filhos de Arão, seus irmãos, no ministério da casa do Senhor.

\medskip

\lettrine{24} E quanto aos filhos de Arão, estas foram as suas
divisões: os filhos de Arão: Nadabe, Abiú, Eleazar e Itamar. E
morreram Nadabe e Abiú antes de seu pai, e não tiveram filhos; e
Eleazar e Itamar administravam o sacerdócio. E Davi, com
Zadoque, dos filhos de Eleazar, e Aimeleque, dos filhos de Itamar,
dividiu-os segundo o seu ofício no seu ministério. E acharam-se
muito mais chefes dos pais entre os filhos de Eleazar do que entre
os filhos de Itamar, quando os repartiram; dos filhos de Eleazar
dezesseis chefes das casas paternas, mas dos filhos de Itamar,
segundo as casas paternas, oito. E os repartiram por sortes, uns
com os outros; porque houve governadores do santuário e governadores
da casa de Deus, assim dentre os filhos de Eleazar, como dentre os
filhos de Itamar. E Semaías, filho de Natanael, o escrivão
dentre os levitas, os registrou perante o rei, e os príncipes, e
Zadoque, o sacerdote, e Aimeleque, filho de Abiatar, e os chefes dos
pais entre os sacerdotes, e entre os levitas; dentre as casas dos
pais tomou-se uma para Eleazar, e outra para Itamar. E saiu a
primeira sorte a Jeoiaribe, a segunda a Jedaías, a terceira a
Harim, a quarta a Seorim, a quinta a Malquias, a sexta a Miamim,
a sétima a Hacoz, a oitava a Abias, a nona a Jesua, a
décima a Secanias, a undécima a Eliasibe, a duodécima a
Jaquim, a décima terceira a Hupa, a décima quarta a
Jesebeabe, a décima quinta a Bilga, a décima sexta a Imer,
a décima sétima a Hezir, a décima oitava a Hapizes, a
décima nona a Petaías, a vigésima a Jeezquel, a vigésima
primeira a Jaquim, a vigésima segunda a Gamul, a vigésima
terceira a Delaías, a vigésima quarta a Maazias. O ofício
destes no seu ministério era entrar na casa do Senhor, segundo lhes
fora ordenado por Arão seu pai, como o Senhor Deus de Israel lhe
tinha mandado.

E do restante dos filhos de Levi: dos filhos de Anrão, Subael;
dos filhos de Subael, Jedias. Quanto a Reabias: dos filhos de
Reabias, Issias era o primeiro; dos izaritas, Selomote; dos
filhos de Selomote, Jaate; e dos filhos de Hebrom, Jerias o
primeiro, Amarias o segundo, Jaaziel o terceiro, Jecameão o quarto;
dos filhos de Uziel, Mica; dos filhos de Mica, Samir;
o irmão de Mica, Issias; dos filhos de Issias, Zacarias;
os filhos de Merari, Mali e Musi; dos filhos de Jaazias,
Beno; os filhos de Merari: de Jaazias, Beno, e Soão, e Zacur,
e Ibri; de Mali, Eleazar; e este não teve filhos.
Quanto a Quis: dos filhos de Quis, Jerameel; e os
filhos de Musi: Mali, e Eder, e Jerimote; estes foram os filhos dos
levitas, segundo as suas casas paternas. Estes também
lançaram sortes como seus irmãos, os filhos de Arão, perante o rei
Davi, e Zadoque, e Aimeleque, e os chefes das famílias entre os
sacerdotes e entre os levitas; assim fizeram, tanto os pais
principais como os irmãos menores.

\medskip

\lettrine{25} E Davi, juntamente com os capitães do exército,
separou para o ministério os filhos de Asafe, e de Hemã, e de
Jedutum, para profetizarem com harpas, com címbalos, e com
saltérios; e este foi o número dos homens aptos para a obra do seu
ministério: Dos filhos de Asafe: Zacur, José, Netanias, e
Asarela, filhos de Asafe; a cargo de Asafe, que profetizava debaixo
das ordens do rei Davi. Quanto a Jedutum, os filhos: Gedalias,
Zeri, Jesaías, Hasabias, e Matitias, seis, a cargo de seu pai,
Jedutum, o qual profetizava com a harpa, louvando e dando graças ao
Senhor. Quanto a Hemã, os filhos: Buquias, Matanias, Uziel,
Sebuel, Jerimote, Hananias, Hanani, Eliata, Gidalti, Romanti-Ezer,
Josbecasa, Maloti, Hotir, e Maaziote. Todos estes foram filhos
de Hemã, o vidente do rei nas palavras de Deus, para exaltar o seu
poder; porque Deus dera a Hemã catorze filhos e três filhas.
Todos estes estavam sob a direção de seu pai, para a música da
casa do Senhor, com saltérios, címbalos e harpas, para o ministério
da casa de Deus; e Asafe, Jedutum, e Hemã, estavam sob as ordens do
rei. E era o número deles, juntamente com seus irmãos instruídos
no canto ao Senhor, todos eles mestres, duzentos e oitenta e oito.

E deitaram sortes acerca da guarda igualmente, assim o pequeno
como o grande, o mestre juntamente com o discípulo. Saiu, pois,
a primeira sorte a Asafe, a saber a José; a segunda a Gedalias; e
ele, e seus irmãos, e seus filhos, ao todo eram doze. A
terceira a Zacur, seus filhos, e seus irmãos, doze. A quarta
a Izri, seus filhos, e seus irmãos, doze. A quinta a
Netanias, seus filhos, e seus irmãos, doze. A sexta a
Buquias, seus filhos, e seus irmãos, doze. A sétima a
Jesarela, seus filhos, e seus irmãos, doze. A oitava a
Jesaías, seus filhos, e seus irmãos, doze. A nona a Matanias,
seus filhos, e seus irmãos, doze. A décima a Simei, seus
filhos, e seus irmãos, doze. A undécima a Azareel, seus
filhos, e seus irmãos, doze. A duodécima a Hasabias, seus
filhos, e seus irmãos, doze. A décima terceira a Subael, seus
filhos, e seus irmãos, doze. A décima quarta a Matitias, seus
filhos, e seus irmãos, doze. A décima quinta a Jeremote, seus
filhos, e seus irmãos, doze. A décima sexta a Hananias, seus
filhos, e seus irmãos, doze. A décima sétima a Josbecasa,
seus filhos, e seus irmãos, doze. A décima oitava a Hanani,
seus filhos, e seus irmãos, doze. A décima nona a Maloti,
seus filhos, e seus irmãos, doze. A vigésima a Eliata, seus
filhos, e seus irmãos, doze. A vigésima primeira a Hotir,
seus filhos, e seus irmãos, doze. A vigésima segunda a
Gidalti, seus filhos, e seus irmãos, doze. A vigésima
terceira a Maaziote, seus filhos, e seus irmãos, doze. A
vigésima quarta a Romanti-Ezer, seus filhos, e seus irmãos, doze.

\medskip

\lettrine{26} Quanto às divisões dos porteiros: dos coraítas:
Meselemias, filho de Coré, dos filhos de Asafe. E foram os
filhos de Meselemias: Zacarias o primogênito, Jediael o segundo,
Zebadias o terceiro, Jatniel o quarto, Elão o quinto, Joanã o
sexto, Elioenai o sétimo. E os filhos de Obede-Edom foram:
Semaías o primogênito, Jozabade o segundo, Joá o terceiro, e Sacar o
quarto, e Natanael o quinto, Amiel o sexto, Issacar o sétimo,
Peuletai o oitavo; porque Deus o tinha abençoado. Também a seu
filho Semaías nasceram filhos, que dominaram sobre a casa de seu
pai; porque foram homens valentes. Os filhos de Semaías: Otni,
Rafael, Obede, e Elzabade, com seus irmãos, homens valentes, Eliú e
Semaquias. Todos estes foram dos filhos de Obede-Edom; eles e
seus filhos, e seus irmãos, homens valentes e de força para o
ministério; ao todo sessenta e dois, de Obede-Edom. E os filhos
e os irmãos de Meselemias, homens valentes, foram dezoito. E
de Hosa, dentre os filhos de Merari, foram filhos: Sinri o chefe
(ainda que não era o primogênito, contudo seu pai o constituiu
chefe), Hilquias o segundo, Tebalias o terceiro, Zacarias o
quarto; todos os filhos e irmãos de Hosa foram treze. Destes
se fizeram as turmas dos porteiros, alternando os principais dos
homens da guarda, juntamente com os seus irmãos, para ministrarem na
casa do Senhor. E lançaram sortes, assim os pequenos como os
grandes, segundo as casas de seus pais, para cada porta. E
caiu a sorte do oriente a Selemias; e lançou-se a sorte por seu
filho Zacarias, conselheiro entendido, e saiu-lhe a do norte.
E para Obede-Edom a do sul; e para seus filhos a casa dos
depósitos. Para Supim e Hosa a do ocidente, junto a porta
Salequete, perto do caminho da subida; uma guarda defronte de outra
guarda. Ao oriente seis levitas; ao norte quatro por dia, ao
sul quatro por dia, porém para as casas dos depósitos de dois em
dois. Em Parbar, ao ocidente, quatro junto ao caminho, e dois
junto a Parbar. Estas são as turmas dos porteiros dentre os
filhos dos coraítas, e dentre os filhos de Merari.

E dos levitas: Aías tinha cargo dos tesouros da casa de Deus e
dos tesouros das coisas sagradas. Quanto aos filhos de Ladã,
os filhos dos gersonitas que pertencem a Ladã, chefes das casas
paternas de Ladã: Jeieli. Os filhos de Jeieli: Zetã e Joel,
seu irmão; estes tinham cargo dos tesouros da casa do Senhor,
dos anramitas, dos izaritas, dos hebronitas, dos uzielitas.
E Sebuel, filho de Gérson, o filho de Moisés, era o chefe dos
tesouros. E seus irmãos foram, do lado de Eliézer, Reabias
seu filho, e Jesaías seu filho, e Jorão seu filho, e Zicri seu
filho, e Selomite, seu filho. Este Selomite e seus irmãos
tinham a seu cargo todos os tesouros das coisas dedicadas que o rei
Davi e os chefes das casas paternas, capitães de milhares, e de
centenas, e capitães do exército tinham consagrado. Dos
despojos das guerras dedicaram ofertas para repararem a casa do
Senhor. Como também tudo quanto tinha consagrado Samuel, o
vidente, e Saul filho de Quis, e Abner filho de Ner, e Joabe filho
de Zeruia; tudo que qualquer havia dedicado estava debaixo da mão de
Selomite e seus irmãos.

Dos izaritas, Quenanias e seus filhos foram postos sobre Israel
como oficiais e como juízes, dos negócios externos. Dos
hebronitas foram Hasabias e seus irmãos, homens valentes, mil e
setecentos, que tinham a superintendência sobre Israel, além do
Jordão para o ocidente, em toda a obra do Senhor, e para o serviço
do rei. Dos hebronitas Jerias era o chefe, segundo as suas
gerações conforme as suas famílias. No ano quarenta do reino de Davi
se buscaram e acharam entre eles homens valentes em Jazer de
Gileade. E seus irmãos, homens valentes, dois mil e
setecentos, chefes dos pais; e o rei Davi os constituiu sobre os
rubenitas e os gaditas, e a meia tribo dos manassitas, para todos os
negócios de Deus, e para todos os negócios do rei.

\medskip

\lettrine{27} Estes são os filhos de Israel segundo o seu
número, os chefes dos pais, e os capitães dos milhares e das
centenas, com os seus oficiais, que serviam ao rei em todos os
negócios das turmas que entravam e saíam de mês em mês, em todos os
meses do ano; cada turma de vinte e quatro mil. Sobre a primeira
turma do primeiro mês estava Jasobeão, filho de Zabdiel; e em sua
turma havia vinte e quatro mil. Era este dos filhos de Perez,
chefe de todos os capitães dos exércitos, para o primeiro mês, e
sobre a turma do segundo mês estava Dodai, o aoíta, com a sua turma,
cujo líder era Miclote; também em sua turma havia vinte e quatro
mil. O terceiro capitão do exército, para o terceiro mês, era
Benaia, filho de Joiada, chefe dos sacerdotes; também em sua turma
havia vinte e quatro mil. Era este Benaia valente entre os
trinta, e sobre os trinta; e na sua turma estava Amizabade, seu
filho. O quarto, do quarto mês, era Asael, irmão de Joabe, e
depois dele Zebadias, seu filho; também em sua turma havia vinte e
quatro mil. O quinto, do quinto mês, Samute, o israíta; também
em sua turma havia vinte e quatro mil. O sexto, do sexto mês,
Ira, filho de Iques, o tecoíta; também em sua turma havia vinte e
quatro mil. O sétimo, do sétimo mês, Helez, o pelonita, dos
filhos de Efraim; também em sua turma havia vinte e quatro mil.
O oitavo, do oitavo mês, Sibecai, o husatita, dos zeraítas;
também em sua turma havia vinte e quatro mil. O nono, do nono
mês, Abiezer, o anatotita, dos benjamitas; também em sua turma havia
vinte e quatro mil. O décimo, do décimo mês, Maarai, o
netofatita, dos zeraítas; também em sua turma havia vinte e quatro
mil. O undécimo, do undécimo mês, Benaia, o piratonita, dos
filhos de Efraim; também em sua turma havia vinte e quatro mil.
O duodécimo, do duodécimo mês, Heldai, o netofatita, de
Otniel; também em sua turma havia vinte e quatro mil.

Sobre as tribos de Israel estavam: sobre os rubenitas era líder
Eliezer, filho de Zicri; sobre os simeonitas, Sefatias, filho de
Maaca. Sobre os levitas, Hasabias, filho de Quemuel; sobre os
aronitas, Zadoque; sobre Judá, Eliú, dos irmãos de Davi;
sobre Issacar, Onri, filho de Micael; sobre Zebulom, Ismaías,
filho de Obadias; sobre Naftali, Jerimote, filho de Azriel;
sobre os filhos de Efraim, Oséias, filho de Azazias; sobre a
meia tribo de Manassés, Joel, filho de Pedaías; sobre a outra
meia tribo de Manassés em Gileade, Ido, filho de Zacarias; sobre
Benjamim, Jaasiel, filho de Abner; sobre Dã, Azarel, filho de
Jeroão. Estes eram os príncipes das tribos de Israel. Não
tomou, porém, Davi o número dos de vinte anos para baixo, porquanto
o Senhor tinha falado que havia de multiplicar a Israel como as
estrelas do céu. Joabe, filho de Zeruia, tinha começado a
numerá-los, porém não acabou; porquanto viera por isso grande ira
sobre Israel; assim o número não se pôs no registro das crônicas do
rei Davi. E sobre os tesouros do rei estava Azmavete, filho
de Adiel; e sobre os tesouros dos campos, das cidades, e das
aldeias, e das torres, Jônatas, filho de Uzias. E sobre os
que faziam a obra do campo, na lavoura da terra, Ezri, filho de
Quelube. E sobre as vinhas, Simei, o ramatita; porém sobre o
que das vides entrava nas adegas do vinho, Zabdi, o sifmita.
E sobre os olivais e sicômoros que havia nas campinas,
Baal-Hanã, o gederita; porém Joás sobre os armazéns do azeite.
E sobre os gados que pastavam em Sarom, Sitrai, o saronita;
porém, sobre os gados dos vales, Safate, filho de Adlai. E
sobre os camelos, Obil, o ismaelita; e sobre as jumentas, Jedias, o
meronotita. E sobre o gado miúdo, Jaziz, o hagrita; todos
esses eram administradores da fazenda que tinha o rei Davi. E
Jônatas, tio de Davi, era do conselho, homem entendido, e também
escriba; e Jeiel, filho de Hacmoni, estava com os filhos do rei.
E Aitofel era do conselho do rei; e Husai, o arquita, amigo
do rei. E depois de Aitofel, Joiada, filho de Benaia, e
Abiatar; porém Joabe era o general do exército do rei.

\medskip

\lettrine{28} Então Davi reuniu em Jerusalém todos os
príncipes de Israel, os príncipes das tribos, e os capitães das
turmas, que serviam o rei, e os capitães dos milhares, e os capitães
das centenas, e os administradores de toda a fazenda e possessão do
rei, e de seus filhos, como também os oficiais, os poderosos, e todo
o homem valente. E pôs-se o rei Davi em pé, e disse: Ouvi-me,
irmãos meus, e povo meu: em meu coração propus eu edificar uma casa
de repouso para a arca da aliança do Senhor e para o estrado dos pés
do nosso Deus, e eu tinha feito o preparo para a edificar. Porém
Deus me disse: Não edificarás casa ao meu nome, porque és homem de
guerra, e derramaste muito sangue. E o Senhor Deus de Israel
escolheu-me de toda a casa de meu pai, para que eternamente fosse
rei sobre Israel; porque a Judá escolheu por soberano, e a casa de
meu pai na casa de Judá; e entre os filhos de meu pai se agradou de
mim para me fazer reinar sobre todo o Israel. E, de todos os
meus filhos (porque muitos filhos me deu o Senhor), escolheu ele o
meu filho Salomão para se assentar no trono do reino do Senhor sobre
Israel. E me disse: Teu filho Salomão, ele edificará a minha
casa e os meus átrios; porque o escolhi para filho, e eu lhe serei
por pai. E estabelecerei o seu reino para sempre, se perseverar
em cumprir os meus mandamentos e os meus juízos, como até ao dia de
hoje. Agora, pois, perante os olhos de todo o Israel, a
congregação do Senhor, e perante os ouvidos de nosso Deus, guardai e
buscai todos os mandamentos do Senhor vosso Deus, para que possuais
esta boa terra, e a façais herdar a vossos filhos depois de vós,
para sempre. E tu, meu filho Salomão, conhece o Deus de teu pai,
e serve-o com um coração perfeito e com uma alma voluntária; porque
esquadrinha o Senhor todos os corações, e entende todas as
imaginações dos pensamentos; se o buscares, será achado de ti;
porém, se o deixares, rejeitar-te-á para sempre. Olha, pois,
agora, porque o Senhor te escolheu para edificares uma casa para o
santuário; esforça-te, e faze a obra.

E deu Davi a Salomão, seu filho, a planta do alpendre com as suas
casas, e as suas tesourarias, e os seus cenáculos, e as suas
recâmaras interiores, como também da casa do propiciatório. E
também a planta de tudo quanto tinha em mente, a saber: dos átrios
da casa do Senhor, e de todas as câmaras ao redor, para os tesouros
da casa de Deus, e para os tesouros das coisas sagradas; e
para as turmas dos sacerdotes, e para os levitas, e para toda a obra
do ministério da casa do Senhor, e para todos os utensílios do
ministério da casa do Senhor. E deu ouro, segundo o peso do
ouro, para todos os utensílios de cada ministério; também a prata,
por peso, para todos os utensílios de prata, para todos os
utensílios de cada ministério. E o peso para os castiçais de
ouro, e suas candeias de ouro segundo o peso de cada castiçal e as
suas candeias; também para os castiçais de prata, segundo o peso do
castiçal e as suas candeias, segundo o uso de cada castiçal.
Também deu o ouro por peso para as mesas da proposição, para
cada mesa; como também a prata para as mesas de prata. E ouro
puro para os garfos, e para as bacias, e para os jarros, e para as
taças de ouro, para cada taça seu peso; como também para as taças de
prata, para cada taça seu peso. E para o altar do incenso,
ouro purificado, por seu peso; como também o ouro para o modelo do
carro, a saber, dos querubins, que haviam de estender as asas, e
cobrir a arca da aliança do Senhor. Tudo isto, disse Davi,
fez-me entender o Senhor, por escrito da sua mão, a saber, todas as
obras desta planta. E disse Davi a Salomão seu filho:
Esforça-te e tem bom ânimo, e faze a obra; não temas, nem te
apavores; porque o Senhor Deus, meu Deus, há de ser contigo; não te
deixará, nem te desamparará, até que acabes toda a obra do serviço
da casa do Senhor. E eis que aí tens as turmas dos sacerdotes
e dos levitas para todo o ministério da casa de Deus; estão também
contigo, para toda a obra, voluntários com sabedoria de toda a
espécie para todo o ministério; como também todos os príncipes, e
todo o povo, para todos os teus mandados.

\medskip

\lettrine{29} Disse mais o rei Davi a toda a congregação:
Salomão, meu filho, a quem só Deus escolheu, é ainda moço e tenro, e
esta obra é grande; porque não é o palácio para homem, mas para o
Senhor Deus. Eu, pois, com todas as minhas forças já tenho
preparado para a casa de meu Deus ouro para as obras de ouro, e
prata para as de prata, e cobre para as de cobre, ferro para as de
ferro e madeira para as de madeira, pedras de
ônix\footnote{Variedade de ágata entre cujas camadas se observa
sensível destaque de cor. Mármore com camadas policrômicas.}, e as
de engaste, e pedras ornamentais, e pedras de diversas cores, e toda
a sorte de pedras preciosas, e pedras de mármore em abundância.
E ainda, porque tenho afeto à casa de meu Deus, o ouro e prata
particular que tenho eu dou para a casa do meu Deus, afora tudo
quanto tenho preparado para a casa do santuário: três mil
talentos de ouro de Ofir; e sete mil talentos de prata purificada,
para cobrir as paredes das casas. Ouro para os objetos de ouro,
e prata para os de prata; e para toda a obra de mão dos artífices.
Quem, pois, está disposto a encher a sua mão, para oferecer hoje
voluntariamente ao Senhor? Então os chefes dos pais, e os
príncipes das tribos de Israel, e os capitães de mil e de cem, até
os chefes da obra do rei, voluntariamente contribuíram. E deram
para o serviço da casa de Deus cinco mil talentos de ouro, e dez mil
dracmas, e dez mil talentos de prata, e dezoito mil talentos de
cobre, e cem mil talentos de ferro. E os que possuíam pedras
preciosas, deram-nas para o tesouro da casa do Senhor, a cargo de
Jeiel o gersonita. E o povo se alegrou porque contribuíram
voluntariamente; porque, com coração perfeito, voluntariamente deram
ao Senhor; e também o rei Davi se alegrou com grande alegria.

Por isso Davi louvou ao Senhor na presença de toda a congregação;
e disse Davi: Bendito és tu, Senhor Deus de Israel, nosso pai, de
eternidade em eternidade. Tua é, Senhor, a magnificência, e o
poder, e a honra, e a vitória, e a majestade; porque teu é tudo
quanto há nos céus e na terra; teu é, Senhor, o reino, e tu te
exaltaste por cabeça sobre todos. E riquezas e glória vêm de
diante de ti, e tu dominas sobre tudo, e na tua mão há força e
poder; e na tua mão está o engrandecer e o dar força a tudo.
Agora, pois, ó Deus nosso, graças te damos, e louvamos o nome
da tua glória. Porque quem sou eu, e quem é o meu povo, para
que pudéssemos oferecer voluntariamente coisas semelhantes? Porque
tudo vem de ti, e do que é teu to damos. Porque somos
estrangeiros diante de ti, e peregrinos como todos os nossos pais;
como a sombra são os nossos dias sobre a terra, e sem ti não há
esperança. Senhor, nosso Deus, toda esta abundância, que
preparamos, para te edificar uma casa ao teu santo nome, vem da tua
mão, e é toda tua. E bem sei eu, Deus meu, que tu provas os
corações, e que da sinceridade te agradas; eu também na sinceridade
de meu coração voluntariamente dei todas estas coisas; e agora vi
com alegria que o teu povo, que se acha aqui, voluntariamente te
deu. Senhor Deus de Abraão, Isaque, e Israel, nossos pais,
conserva isto para sempre no intento dos pensamentos do coração de
teu povo; e encaminha o seu coração para ti. E a Salomão, meu
filho, dá um coração perfeito, para guardar os teus mandamentos, os
teus testemunhos, e os teus estatutos; e para fazer tudo, e para
edificar este palácio que tenho preparado. Então disse Davi a
toda a congregação: Agora louvai ao Senhor vosso Deus. Então toda a
congregação louvou ao Senhor Deus de seus pais, e inclinaram-se, e
prostraram-se perante o Senhor, e o rei. E ao outro dia
imolaram sacrifícios ao Senhor, e ofereceram holocaustos ao Senhor,
mil bezerros, mil carneiros, mil cordeiros, com as suas libações; e
sacrifícios em abundância por todo o Israel. E comeram e
beberam naquele dia perante o Senhor, com grande gozo; e a segunda
vez fizeram rei a Salomão filho de Davi, e o ungiram ao Senhor por
líder, e a Zadoque por sacerdote. Assim Salomão se assentou
no trono do Senhor, como rei, em lugar de Davi seu pai, e prosperou;
e todo o Israel lhe obedecia. E todos os príncipes, e os
grandes, e até todos os filhos do rei Davi, se submeteram ao rei
Salomão. E o Senhor magnificou a Salomão grandíssimamente,
perante os olhos de todo o Israel; e deu-lhe majestade real, qual
antes dele não teve nenhum rei em Israel. Assim Davi, filho
de Jessé, reinou sobre todo o Israel. E foram os dias que
reinou sobre Israel, quarenta anos; em Hebrom reinou sete anos, e em
Jerusalém reinou trinta e três. E morreu numa boa velhice,
cheio de dias, riquezas e glória; e Salomão, seu filho, reinou em
seu lugar. Os atos, pois, do rei Davi, assim os primeiros
como os últimos, eis que estão escritos nas crônicas de Samuel, o
vidente, e nas crônicas do profeta Natã, e nas crônicas de Gade, o
vidente, juntamente com todo o seu reinado e o seu poder; e
os tempos que passaram sobre ele, e sobre Israel, e sobre todos os
reinos daquelas terras.

