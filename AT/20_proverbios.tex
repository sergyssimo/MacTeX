\addchap{Provérbios}

\lettrine{1}{}Provérbios de Salomão, filho de Davi, rei de
Israel; para se conhecer a sabedoria e a instrução; para se
entenderem as palavras da prudência. Para se receber a instrução
do entendimento, a justiça, o juízo e a eqüidade; para dar aos
simples, prudência, e aos moços, conhecimento e bom siso; o
sábio ouvirá e crescerá em conhecimento, e o entendido adquirirá
sábios conselhos; para entender os provérbios e sua
interpretação; as palavras dos sábios e as suas proposições.

O temor do Senhor é o princípio do conhecimento; os loucos
desprezam a sabedoria e a instrução. Filho meu, ouve a instrução
de teu pai, e não deixes o ensinamento de tua mãe, porque serão
como diadema gracioso em tua cabeça, e colares ao teu pescoço.

Filho meu, se os pecadores procuram te atrair com agrados, não
aceites. Se disserem: Vem conosco a tocaias de sangue;
embosquemos o inocente sem motivo; traguemo-los vivos, como a
sepultura; e inteiros, como os que descem à cova; acharemos
toda sorte de bens preciosos; encheremos as nossas casas de
despojos; lança a tua sorte conosco; teremos todos uma só
bolsa! Filho meu, não te ponhas a caminho com eles; desvia o
teu pé das suas veredas; porque os seus pés correm para o
mal, e se apressam a derramar sangue. Na verdade é inútil
estender-se a rede ante os olhos de qualquer ave. No entanto
estes armam ciladas contra o seu próprio sangue; e espreitam suas
próprias vidas. São assim as veredas de todo aquele que usa
de cobiça: ela põe a perder a alma dos que a possuem.

A sabedoria clama lá fora; pelas ruas levanta a sua voz.
Nas esquinas movimentadas ela brada; nas entradas das portas
e nas cidades profere as suas palavras: Até quando, ó
simples, amareis a simplicidade? E vós escarnecedores, desejareis o
escárnio? E vós insensatos, odiareis o conhecimento? Atentai
para a minha repreensão; pois eis que vos derramarei abundantemente
do meu espírito e vos farei saber as minhas palavras.
Entretanto, porque eu clamei e recusastes; e estendi a minha
mão e não houve quem desse atenção, antes rejeitastes todo o
meu conselho, e não quisestes a minha repreensão, também de
minha parte eu me rirei na vossa perdição e zombarei, em vindo o
vosso temor. Vindo o vosso temor como a assolação, e vindo a
vossa perdição como uma tormenta, sobrevirá a vós aperto e angústia.
Então clamarão a mim, mas eu não responderei; de madrugada me
buscarão, porém não me acharão. Porquanto odiaram o
conhecimento; e não preferiram o temor do Senhor; não
aceitaram o meu conselho, e desprezaram toda a minha repreensão.
Portanto comerão do fruto do seu caminho, e fartar-se-ão dos
seus próprios conselhos. Porque o erro dos simples os matará,
e o desvario dos insensatos os destruirá. Mas o que me der
ouvidos habitará em segurança, e estará livre do temor do mal.

\medskip

\lettrine{2}{}Filho meu, se aceitares as minhas palavras, e
esconderes contigo os meus mandamentos, para fazeres o teu
ouvido atento à sabedoria; e inclinares o teu coração ao
entendimento; se clamares por conhecimento, e por inteligência
alçares a tua voz, se como a prata a buscares e como a tesouros
escondidos a procurares, então entenderás o temor do Senhor, e
acharás o conhecimento de Deus. Porque o Senhor dá a sabedoria;
da sua boca é que vem o conhecimento e o entendimento. Ele
reserva a verdadeira sabedoria para os retos. Escudo é para os que
caminham na sinceridade, para que guardem as veredas do juízo.
Ele preservará o caminho dos seus santos. Então entenderás a
justiça, o juízo, a eqüidade e todas as boas veredas.

Pois quando a sabedoria entrar no teu coração, e o conhecimento
for agradável à tua alma, o bom siso te guardará e a
inteligência te conservará; para te afastar do mau caminho, e
do homem que fala coisas perversas; dos que deixam as veredas
da retidão, para andarem pelos caminhos escusos; que se
alegram de fazer mal, e folgam com as perversidades dos maus,
cujas veredas são tortuosas e que se desviam nos seus
caminhos; para te afastar da mulher estranha, sim da estranha
que lisonjeia com suas palavras; que deixa o guia da sua
mocidade e se esquece da aliança do seu Deus; porque a sua
casa se inclina para a morte, e as suas veredas para os mortos.
Todos os que se dirigem a ela não voltarão e não atinarão com
as veredas da vida. Para andares pelos caminhos dos bons, e
te conservares nas veredas dos justos. Porque os retos
habitarão a terra, e os íntegros permanecerão nela. Mas os
ímpios serão arrancados da terra, e os aleivosos\footnote{Em que há
aleive; fraudulento. Que procede com aleive; desleal, traidor,
pérfido. Caluniador.} serão dela exterminados.

\medskip

\lettrine{3}{}Filho meu, não te esqueças da minha lei, e o teu
coração guarde os meus mandamentos. Porque eles aumentarão os
teus dias e te acrescentarão anos de vida e paz. Não te
desamparem a benignidade e a fidelidade; ata-as ao teu pescoço;
escreve-as na tábua do teu coração. E acharás graça e bom
entendimento aos olhos de Deus e do homem. Confia no Senhor de
todo o teu coração, e não te estribes no teu próprio entendimento.
Reconhece-o em todos os teus caminhos, e ele endireitará as tuas
veredas.

Não sejas sábio a teus próprios olhos; teme ao Senhor e aparta-te
do mal. Isto será saúde para o teu âmago, e medula para os teus
ossos. Honra ao Senhor com os teus bens, e com a primeira parte
de todos os teus ganhos; e se encherão os teus celeiros, e
transbordarão de vinho os teus lagares. Filho meu, não
rejeites a correção do Senhor, nem te enojes da sua repreensão.
Porque o Senhor repreende aquele a quem ama, assim como o pai
ao filho a quem quer bem.

Bem-aventurado o homem que acha sabedoria, e o homem que adquire
conhecimento; porque é melhor a sua mercadoria do que artigos
de prata, e maior o seu lucro que o ouro mais fino. Mais
preciosa é do que os rubis, e tudo o que mais possas desejar não se
pode comparar a ela. Vida longa de dias está na sua mão
direita; e na esquerda, riquezas e honra. Os seus caminhos
são caminhos de delícias, e todas as suas veredas de paz. É
árvore de vida para os que dela tomam, e são bem-aventurados todos
os que a retêm. O Senhor, com sabedoria fundou a terra; com
entendimento preparou os céus. Pelo seu conhecimento se
fenderam os abismos, e as nuvens destilam o orvalho.

Filho meu, não se apartem estas coisas dos teus olhos: guarda a
verdadeira sabedoria e o bom siso; porque serão vida para a
tua alma, e adorno ao teu pescoço. Então andarás confiante
pelo teu caminho, e o teu pé não tropeçará. Quando te
deitares, não temerás; ao contrário, o teu sono será suave ao te
deitares. Não temas o pavor repentino, nem a investida dos
perversos quando vier. Porque o Senhor será a tua esperança;
guardará os teus pés de serem capturados.

Não deixes de fazer bem a quem o merece, estando em tuas mãos a
capacidade de fazê-lo. Não digas ao teu próximo: Vai, e volta
amanhã que to darei, se já o tens contigo. Não maquines o mal
contra o teu próximo, pois que habita contigo confiadamente.
Não contendas com alguém sem causa, se não te fez nenhum mal.
Não tenhas inveja do homem violento, nem escolhas nenhum dos
seus caminhos. Porque o perverso é abominável ao Senhor, mas
com os sinceros ele tem intimidade. A maldição do Senhor
habita na casa do ímpio, mas a habitação dos justos abençoará.
Certamente ele escarnecerá dos escarnecedores, mas dará graça
aos mansos. Os sábios herdarão honra, mas os loucos tomam
sobre si vergonha.

\medskip

\lettrine{4}{}Ouvi, filhos, a instrução do pai, e estai atentos
para conhecerdes a prudência. Pois dou-vos boa doutrina; não
deixeis a minha lei. Porque eu era filho tenro na companhia de
meu pai, e único diante de minha mãe. E ele me ensinava e me
dizia: Retenha o teu coração as minhas palavras; guarda os meus
mandamentos, e vive. Adquire sabedoria, adquire inteligência, e
não te esqueças nem te apartes das palavras da minha boca. Não a
abandones e ela te guardará; ama-a, e ela te protegerá. A
sabedoria é a coisa principal; adquire pois a sabedoria, emprega
tudo o que possuis na aquisição de entendimento. Exalta-a, e ela
te exaltará; e, abraçando-a tu, ela te honrará. Dará à tua
cabeça um diadema de graça e uma coroa de glória te entregará.
Ouve, filho meu, e aceita as minhas palavras, e se
multiplicarão os anos da tua vida. No caminho da sabedoria te
ensinei, e por veredas de retidão te fiz andar. Por elas
andando, não se embaraçarão os teus passos; e se correres não
tropeçarás. Apega-te à instrução e não a largues; guarda-a,
porque ela é a tua vida.

Não entres pela vereda dos ímpios, nem andes no caminho dos maus.
Evita-o; não passes por ele; desvia-te dele e passa de largo.
Pois não dormem, se não fizerem mal, e foge deles o sono se
não fizerem alguém tropeçar. Porque comem o pão da impiedade,
e bebem o vinho da violência. Mas a vereda dos justos é como
a luz da aurora, que vai brilhando mais e mais até ser dia perfeito.
O caminho dos ímpios é como a escuridão; nem sabem em que
tropeçam.

Filho meu, atenta para as minhas palavras; às minhas razões
inclina o teu ouvido. Não as deixes apartar-se dos teus
olhos; guarda-as no íntimo do teu coração. Porque são vida
para os que as acham, e saúde para todo o seu corpo. Sobre
tudo o que se deve guardar, guarda o teu coração, porque dele
procedem as fontes da vida. Desvia de ti a falsidade da boca,
e afasta de ti a perversidade dos lábios. Os teus olhos olhem
para a frente, e as tuas pálpebras olhem direto diante de ti.
Pondera a vereda de teus pés, e todos os teus caminhos sejam
bem ordenados! Não declines nem para a direita nem para a
esquerda; retira o teu pé do mal.

\medskip

\lettrine{5}{}Filho meu, atende à minha sabedoria; à minha
inteligência inclina o teu ouvido; para que guardes os meus
conselhos e os teus lábios observem o conhecimento. Porque os
lábios da mulher estranha destilam favos de mel, e o seu paladar é
mais suave do que o azeite. Mas o seu fim é amargoso como o
absinto, agudo como a espada de dois gumes. Os seus pés descem
para a morte; os seus passos estão impregnados do inferno. Para
que não ponderes os caminhos da vida, as suas andanças são errantes:
jamais os conhecerás. Agora, pois, filhos, dai-me ouvidos, e não
vos desvieis das palavras da minha boca. Longe dela seja o teu
caminho, e não te chegues à porta da sua casa; para que não dês
a outrem a tua honra, e não entregues a cruéis os teus anos de vida;
para que não farte a estranhos o teu esforço, e todo o fruto
do teu trabalho vá parar em casa alheia; e no fim venhas a
gemer, no consumir-se da tua carne e do teu corpo. E então
digas: Como odiei a correção! e o meu coração desprezou a
repreensão! E não escutei a voz dos que me ensinavam, nem aos
meus mestres inclinei o meu ouvido! No meio da congregação e
da assembléia foi que eu me achei em quase todo o mal.

Bebe água da tua fonte, e das correntes do teu poço.
Derramar-se-iam as tuas fontes por fora, e pelas ruas os
ribeiros de águas? Sejam para ti só, e não para os estranhos
contigo. Seja bendito o teu manancial, e alegra-te com a
mulher da tua mocidade. Como cerva amorosa, e gazela
graciosa, os seus seios te saciem todo o tempo; e pelo seu amor
sejas atraído perpetuamente. E porque, filho meu, te
deixarias atrair por outra mulher, e te abraçarias ao peito de uma
estranha? Eis que os caminhos do homem estão perante os olhos
do Senhor, e ele pesa todas as suas veredas. Quanto ao ímpio,
as suas iniqüidades o prenderão, e com as cordas do seu pecado será
detido. Ele morrerá, porque desavisadamente andou, e pelo
excesso da sua loucura se perderá.

\medskip

\lettrine{6}{}Filho meu, se ficaste por fiador do teu
companheiro, se deste a tua mão ao estranho, e te deixaste
enredar pelas próprias palavras; e te prendeste nas palavras da tua
boca; faze pois isto agora, filho meu, e livra-te, já que caíste
nas mãos do teu companheiro: vai, humilha-te, e importuna o teu
companheiro. Não dês sono aos teus olhos, nem deixes adormecer
as tuas pálpebras. Livra-te, como a gazela da mão do caçador, e
como a ave da mão do passarinheiro.

Vai ter com a formiga, ó preguiçoso; olha para os seus caminhos, e
sê sábio. Pois ela, não tendo chefe, nem guarda, nem dominador,
prepara no verão o seu pão; na sega ajunta o seu mantimento.
Ó preguiçoso, até quando ficarás deitado? Quando te levantarás
do teu sono? Um pouco a dormir, um pouco a tosquenejar; um
pouco a repousar de braços cruzados; assim sobrevirá a tua
pobreza como o meliante, e a tua necessidade como um homem armado.

O homem mau, o homem iníquo tem a boca pervertida. Acena
com os olhos, fala com os pés e faz sinais com os dedos. Há
no seu coração perversidade, todo o tempo maquina mal; anda semeando
contendas. Por isso a sua destruição virá repentinamente;
subitamente será quebrantado, sem que haja cura. Estas seis
coisas o Senhor odeia, e a sétima a sua alma abomina: olhos
altivos, língua mentirosa, mãos que derramam sangue inocente,
o coração que maquina pensamentos perversos, pés que se
apressam a correr para o mal, a testemunha falsa que profere
mentiras, e o que semeia contendas entre irmãos.

Filho meu, guarda o mandamento de teu pai, e não deixes a lei da
tua mãe; ata-os perpetuamente ao teu coração, e pendura-os ao
teu pescoço. Quando caminhares, te guiará; quando te
deitares, te guardará; quando acordares, falará contigo.
Porque o mandamento é lâmpada, e a lei é luz; e as
repreensões da correção são o caminho da vida, para te
guardarem da mulher vil, e das lisonjas da estranha. Não
cobices no teu coração a sua formosura, nem te prendas aos seus
olhos. Porque por causa duma prostituta se chega a pedir um
bocado de pão; e a adúltera anda à caça da alma preciosa.
Porventura tomará alguém fogo no seu seio, sem que suas
vestes se queimem? Ou andará alguém sobre brasas, sem que se
queimem os seus pés? Assim ficará o que entrar à mulher do
seu próximo; não será inocente todo aquele que a tocar. Não
se injuria o ladrão, quando furta para saciar-se, tendo fome;
e se for achado pagará o tanto sete vezes; terá de dar todos
os bens da sua casa. Assim, o que adultera com uma mulher é
falto de entendimento; aquele que faz isso destrói a sua alma.
Achará castigo e vilipêndio, e o seu opróbrio nunca se
apagará. Porque os ciúmes enfurecerão o marido; de maneira
nenhuma perdoará no dia da vingança. Não aceitará nenhum
resgate, nem se conformará por mais que aumentes os presentes.

\medskip

\lettrine{7}{}Filho meu, guarda as minhas palavras, e esconde
dentro de ti os meus mandamentos. Guarda os meus mandamentos e
vive; e a minha lei, como a menina dos teus olhos. Ata-os aos
teus dedos, escreve-os na tábua do teu coração. Dize à
sabedoria: Tu és minha irmã; e à prudência chama de tua parenta,
para que elas te guardem da mulher alheia, da estranha que
lisonjeia com as suas palavras.

Porque da janela da minha casa, olhando eu por minhas frestas,
vi entre os simples, descobri entre os moços, um moço falto de
juízo, que passava pela rua junto à sua esquina, e seguia o
caminho da sua casa; no crepúsculo, à tarde do dia, na tenebrosa
noite e na escuridão. E eis que uma mulher lhe saiu ao
encontro com enfeites de prostituta, e astúcia de coração.
Estava alvoroçada e irriquieta; não paravam em sua casa os
seus pés. Foi para fora, depois pelas ruas, e ia espreitando
por todos os cantos; e chegou-se para ele e o beijou. Com
face impudente lhe disse: Sacrifícios pacíficos tenho comigo;
hoje paguei os meus votos. Por isto saí ao teu encontro a
buscar diligentemente a tua face, e te achei. Já cobri a
minha cama com cobertas de tapeçaria, com obras lavradas, com linho
fino do Egito. Já perfumei o meu leito com mirra, aloés e
canela. Vem, saciemo-nos de amores até à manhã; alegremo-nos
com amores. Porque o marido não está em casa; foi fazer uma
longa viagem; levou na sua mão um saquitel de dinheiro;
voltará para casa só no dia marcado. Assim, o seduziu com
palavras muito suaves e o persuadiu com as lisonjas dos seus lábios.
E ele logo a segue, como o boi que vai para o matadouro, e
como vai o insensato para o castigo das prisões; até que a
flecha lhe atravesse o fígado; ou como a ave que se apressa para o
laço, e não sabe que está armado contra a sua vida.

Agora pois, filhos, dai-me ouvidos, e estai atentos às palavras
da minha boca. Não se desvie para os caminhos dela o teu
coração, e não te deixes perder nas suas veredas. Porque a
muitos feridos derrubou; e são muitíssimos os que por causa dela
foram mortos. A sua casa é caminho do inferno que desce para
as câmaras da morte.

\medskip

\lettrine{8}{}Não clama porventura a sabedoria, e a
inteligência não faz ouvir a sua voz? No cume das alturas, junto
ao caminho, nas encruzilhadas das veredas se posta. Do lado das
portas da cidade, à entrada da cidade, e à entrada das portas está
gritando: A vós, ó homens, clamo; e a minha voz se dirige aos
filhos dos homens. Entendei, ó simples, a prudência; e vós,
insensatos, entendei de coração. Ouvi, porque falarei coisas
excelentes; os meus lábios se abrirão para a eqüidade. Porque a
minha boca proferirá a verdade, e os meus lábios abominam a
impiedade. São justas todas as palavras da minha boca: não há
nelas nenhuma coisa tortuosa nem pervertida. Todas elas são
retas para aquele que as entende bem, e justas para os que acham o
conhecimento. Aceitai a minha correção, e não a prata; e o
conhecimento, mais do que o ouro fino escolhido. Porque
melhor é a sabedoria do que os rubis; e tudo o que mais se deseja
não se pode comparar com ela.

Eu, a sabedoria, habito com a prudência, e acho o conhecimento
dos conselhos. O temor do Senhor é odiar o mal; a soberba e a
arrogância, o mau caminho e a boca perversa, eu odeio. Meu é
o conselho e a verdadeira sabedoria; eu sou o entendimento; minha é
a fortaleza. Por mim reinam os reis e os príncipes decretam
justiça. Por mim governam príncipes e nobres; sim, todos os
juízes da terra.
 Eu amo aos que me amam, e os que cedo me buscarem, me acharão. Riquezas e honra estão comigo; assim como os bens duráveis e a
justiça. Melhor é o meu fruto do que o ouro, do que o ouro
refinado, e os meus ganhos mais do que a prata escolhida.
Faço andar pelo caminho da justiça, no meio das veredas do
juízo. Para que faça herdar bens permanentes aos que me amam,
e eu encha os seus tesouros.

O Senhor me possuiu no princípio de seus caminhos, desde então, e
antes de suas obras. Desde a eternidade fui ungida, desde o
princípio, antes do começo da terra. Quando ainda não havia
abismos, fui gerada, quando ainda não havia fontes carregadas de
águas. Antes que os montes se houvessem assentado, antes dos
outeiros, eu fui gerada. Ainda ele não tinha feito a terra,
nem os campos, nem o princípio do pó do mundo. Quando ele
preparava os céus, aí estava eu, quando traçava o horizonte sobre a
face do abismo; quando firmava as nuvens acima, quando
fortificava as fontes do abismo, quando fixava ao mar o seu
termo, para que as águas não traspassassem o seu mando, quando
compunha os fundamentos da terra. Então eu estava com ele, e
era seu arquiteto; era cada dia as suas delícias, alegrando-me
perante ele em todo o tempo; regozijando-me no seu mundo
habitável e enchendo-me de prazer com os filhos dos homens.

Agora, pois, filhos, ouvi-me, porque bem-aventurados serão os que
guardarem os meus caminhos. Ouvi a instrução, e sede sábios,
não a rejeiteis. Bem-aventurado o homem que me dá ouvidos,
velando às minhas portas cada dia, esperando às ombreiras da minha
entrada. Porque o que me achar, achará a vida, e alcançará o
favor do Senhor. Mas o que pecar contra mim violentará a sua
própria alma; todos os que me odeiam amam a morte.

\medskip

\lettrine{9}{}A sabedoria já edificou a sua casa, já lavrou as
suas sete colunas. Já abateu os seus animais e misturou o seu
vinho, e já preparou a sua mesa. Já ordenou às suas criadas, e
está convidando desde as alturas da cidade, dizendo: Quem é
simples, volte-se para cá. Aos faltos de senso diz: Vinde, comei
do meu pão, e bebei do vinho que tenho misturado. Deixai os
insensatos e vivei; e andai pelo caminho do entendimento. O que
repreende o escarnecedor, toma afronta para si; e o que censura o
ímpio recebe a sua mancha. Não repreendas o escarnecedor, para
que não te odeie; repreende o sábio, e ele te amará. Dá
instrução ao sábio, e ele se fará mais sábio; ensina o justo e ele
aumentará em doutrina. O temor do Senhor é o princípio da
sabedoria, e o conhecimento do Santo a prudência. Porque por
meu intermédio se multiplicam os teus dias, e anos de vida se te
aumentarão. Se fores sábio, para ti serás sábio; e, se fores
escarnecedor, só tu o suportarás.

A mulher louca é alvoroçadora; é simples e nada sabe.
Assenta-se à porta da sua casa numa cadeira, nas alturas da
cidade, e põe-se a chamar aos que vão pelo caminho, e que
passam reto pelas veredas, dizendo: Quem é simples, volte-se
para cá. E aos faltos de entendimento ela diz: As águas
roubadas são doces, e o pão tomado às escondidas é agradável.
Mas não sabem que ali estão os mortos; os seus convidados
estão nas profundezas do inferno.

\medskip

\lettrine{10}{}Provérbios de Salomão: O filho sábio alegra a
seu pai, mas o filho insensato é a tristeza de sua mãe.
\textparagraph 2 Os tesouros da impiedade de nada aproveitam; mas a
justiça livra da morte. O Senhor não deixa o justo passar fome,
mas rechaça a aspiração dos perversos.

O que trabalha com mão displicente empobrece, mas a mão dos
diligentes enriquece.

O que ajunta no verão é filho ajuizado, mas o que dorme na sega é
filho que envergonha.

Bênçãos há sobre a cabeça do justo, mas a violência cobre a boca
dos perversos.

A memória do justo é abençoada, mas o nome dos perversos
apodrecerá.

O sábio de coração aceita os mandamentos, mas o insensato de
lábios ficará transtornado.

Quem anda em sinceridade, anda seguro; mas o que perverte os seus
caminhos ficará conhecido.

O que acena com os olhos causa dores, e o tolo de lábios ficará
transtornado.

A boca do justo é fonte de vida, mas a violência cobre a boca dos
perversos.

O ódio excita contendas, mas o amor cobre todos os pecados.

Nos lábios do entendido se acha a sabedoria, mas a vara é para as
costas do falto de entendimento.

Os sábios entesouram a sabedoria; mas a boca do tolo o aproxima
da ruína.

Os bens do rico são a sua cidade forte, a pobreza dos pobres a
sua ruína.

A obra do justo conduz à vida, o fruto do perverso, ao pecado.

O caminho para a vida é daquele que guarda a instrução, mas o que
deixa a repreensão comete erro.

O que encobre o ódio tem lábios falsos, e o que divulga má fama é
um insensato.

Na multidão de palavras não falta pecado, mas o que modera os
seus lábios é sábio.

Prata escolhida é a língua do justo; o coração dos perversos é de
nenhum valor. Os lábios do justo apascentam a muitos, mas os
tolos morrem por falta de entendimento.

A bênção do Senhor é que enriquece; e não traz consigo dores.

Para o tolo, o cometer desordem é divertimento; mas para o homem
entendido é o ter sabedoria.

Aquilo que o perverso teme sobrevirá a ele, mas o desejo dos
justos será concedido. Como passa a tempestade, assim
desaparece o perverso, mas o justo tem fundamento perpétuo.

Como vinagre para os dentes, como fumaça para os olhos, assim é o
preguiçoso para aqueles que o mandam.

O temor do Senhor aumenta os dias, mas os perversos terão os anos
da vida abreviados. A esperança dos justos é alegria, mas a
expectação dos perversos perecerá.

O caminho do Senhor é fortaleza para os retos, mas ruína para os
que praticam a iniqüidade. O justo nunca jamais será abalado,
mas os perversos não habitarão a terra.

A boca do justo jorra sabedoria, mas a língua da perversidade
será cortada. Os lábios do justo sabem o que agrada, mas a
boca dos perversos, só perversidades.

\medskip

\lettrine{11}{}Balança enganosa é abominação para o Senhor, mas
o peso justo é o seu prazer. \textparagraph 2 Em vindo a soberba,
virá também a afronta; mas com os humildes está a sabedoria.

A sinceridade dos íntegros os guiará, mas a perversidade dos
aleivosos os destruirá.

De nada aproveitam as riquezas no dia da ira, mas a justiça livra
da morte.

A justiça do sincero endireitará o seu caminho, mas o perverso
pela sua falsidade cairá. A justiça dos virtuosos os livrará,
mas na sua perversidade serão apanhados os iníquos.

Morrendo o homem perverso perece sua esperança, e acaba-se a
expectação de riquezas.

O justo é libertado da angústia, e vem o ímpio para o seu lugar.

O hipócrita com a boca destrói o seu próximo, mas os justos se
libertam pelo conhecimento.

No bem dos justos exulta a cidade; e perecendo os ímpios, há
júbilo. Pela bênção dos homens de bem a cidade se exalta, mas
pela boca dos perversos é derrubada.

O que despreza o seu próximo carece de entendimento, mas o homem
entendido se mantém calado. O mexeriqueiro revela o segredo,
mas o fiel de espírito o mantém em oculto.

Não havendo sábios conselhos, o povo cai, mas na multidão de
conselhos há segurança.

Decerto sofrerá severamente aquele que fica por fiador do
estranho, mas o que evita a fiança estará seguro.

A mulher graciosa guarda a honra como os violentos guardam as
riquezas.

O homem bom cuida bem de si mesmo, mas o cruel prejudica o seu
corpo.

O ímpio faz obra falsa, mas para o que semeia justiça haverá
galardão fiel.

Como a justiça encaminha para a vida, assim o que segue o mal vai
para a sua morte.

Abominação ao Senhor são os perversos de coração, mas os de
caminho sincero são o seu deleite.

Ainda que junte as mãos, o mau não ficará impune, mas a semente
dos justos será liberada.

Como jóia de ouro no focinho de uma porca, assim é a mulher
formosa que não tem discrição.

O desejo dos justos é tão somente para o bem, mas a esperança dos
ímpios é criar contrariedades.

Ao que distribui mais se lhe acrescenta, e ao que retém mais do
que é justo, é para a sua perda.

A alma generosa prosperará e aquele que atende também será
atendido.

Ao que retém o trigo o povo amaldiçoa, mas bênção haverá sobre a
cabeça do que o vende.

O que cedo busca o bem, busca favor, mas o que procura o mal,
esse lhe sobrevirá.

Aquele que confia nas suas riquezas cairá, mas os justos
reverdecerão como a folhagem.

O que perturba a sua casa herdará o vento, e o tolo será servo do
sábio de coração.

O fruto do justo é árvore de vida, e o que ganha almas é sábio.

Eis que o justo recebe na terra a retribuição; quanto mais o
ímpio e o pecador!

\medskip

\lettrine{12}{}O que ama a instrução ama o conhecimento, mas o
que odeia a repreensão é estúpido. \textparagraph 2 O homem de bem
alcançará o favor do Senhor, mas ao homem de intenções perversas ele
condenará.

O homem não se estabelecerá pela impiedade, mas a raiz dos justos
não será removida.

A mulher virtuosa é a coroa do seu marido, mas a que o envergonha
é como podridão nos seus ossos.

Os pensamentos dos justos são retos, mas os conselhos dos ímpios,
engano.

As palavras dos ímpios são ciladas para derramar sangue, mas a
boca dos retos os livrará.

Os ímpios serão transtornados e não subsistirão, mas a casa dos
justos permanecerá.

Cada qual será louvado segundo o seu entendimento, mas o perverso
de coração estará em desprezo.

Melhor é o que se estima em pouco, e tem servos, do que o que se
vangloria e tem falta de pão.

O justo tem consideração pela vida dos seus animais, mas as
afeições dos ímpios são cruéis.

O que lavra a sua terra se fartará de pão; mas o que segue os
ociosos é falto de juízo.

O ímpio deseja a rede dos maus, mas a raiz dos justos produz o
seu fruto.

O ímpio se enlaça na transgressão dos lábios, mas o justo sairá
da angústia.

Cada um se fartará do fruto da sua boca, e da obra das suas mãos
o homem receberá a recompensa.

O caminho do insensato é reto aos seus próprios olhos, mas o que
dá ouvidos ao conselho é sábio.

A ira do insensato se conhece no mesmo dia, mas o prudente
encobre a afronta.

O que diz a verdade manifesta a justiça, mas a falsa testemunha
diz engano.

Há alguns que falam como que espada penetrante, mas a língua dos
sábios é saúde.

O lábio da verdade permanece para sempre, mas a língua da
falsidade, dura por um só momento.

No coração dos que maquinam o mal há engano, mas os que
aconselham a paz têm alegria.

Nenhum agravo sobrevirá ao justo, mas os ímpios ficam cheios de
problemas.

Os lábios mentirosos são abomináveis ao Senhor, mas os que agem
fielmente são o seu deleite.

O homem prudente encobre o conhecimento, mas o coração dos tolos
proclama a estultícia.

A mão dos diligentes dominará, mas os negligentes serão
tributários.

A ansiedade no coração deixa o homem abatido, mas uma boa palavra
o alegra.

O justo é mais excelente do que o seu próximo, mas o caminho dos
ímpios faz errar.

O preguiçoso deixa de assar a sua caça, mas ser diligente é o
precioso bem do homem.

Na vereda da justiça está a vida, e no caminho da sua carreira
não há morte.

\medskip

\lettrine{13}{}O filho sábio atende à instrução do pai; mas o
escarnecedor não ouve a repreensão. \textparagraph 2 Do fruto da
boca cada um comerá o bem, mas a alma dos prevaricadores comerá a
violência.

O que guarda a sua boca conserva a sua alma, mas o que abre muito
os seus lábios se destrói.

A alma do preguiçoso deseja, e coisa nenhuma alcança, mas a alma
dos diligentes se farta.

O justo odeia a palavra de mentira, mas o ímpio faz vergonha e se
confunde.

A justiça guarda ao que é de caminho certo, mas a impiedade
transtornará o pecador.

Há alguns que se fazem de ricos, e não têm coisa nenhuma, e outros
que se fazem de pobres e têm muitas riquezas.

O resgate da vida de cada um são as suas riquezas, mas o pobre não
ouve ameaças.

A luz dos justos alegra, mas a candeia dos ímpios se apagará.

Da soberba só provém a contenda, mas com os que se aconselham se
acha a sabedoria.

A riqueza de procedência vã diminuirá, mas quem a ajunta com o
próprio trabalho a aumentará.

A esperança adiada desfalece o coração, mas o desejo atendido é
árvore de vida.

O que despreza a palavra perecerá, mas o que teme o mandamento
será galardoado.

A doutrina do sábio é uma fonte de vida para se desviar dos laços
da morte.

O bom entendimento favorece, mas o caminho dos prevaricadores é
áspero.

Todo prudente procede com conhecimento, mas o insensato
espraia\footnote{Espraiar: Derramar, estender, alastrar. Irradiar,
emitir. Desenvolver, dilatar. Propagar.} a sua loucura.

O que prega a maldade cai no mal, mas o embaixador fiel é saúde.

Pobreza e afronta virão ao que rejeita a instrução, mas o que
guarda a repreensão será honrado.

O desejo que se alcança deleita a alma, mas apartar-se do mal é
abominável para os insensatos.

O que anda com os sábios ficará sábio, mas o companheiro dos
tolos será destruído.

O mal perseguirá os pecadores, mas os justos serão galardoados
com o bem.

O homem de bem deixa uma herança aos filhos de seus filhos, mas a
riqueza do pecador é depositada para o justo.

O pobre, do sulco da terra, tira mantimento em abundância; mas há
os que se consomem por falta de juízo.

O que não faz uso da vara odeia seu filho, mas o que o ama, desde
cedo o castiga.

O justo come até ficar satisfeito, mas o ventre dos ímpios
passará necessidade.

\medskip

\lettrine{14}{}Toda mulher sábia edifica a sua casa; mas a tola
a derruba com as próprias mãos. \textparagraph 2 O que anda na
retidão teme ao Senhor, mas o que se desvia de seus caminhos o
despreza.

Na boca do tolo está a punição da soberba, mas os sábios se
conservam pelos próprios lábios.

Não havendo bois o estábulo fica limpo, mas pela força do boi há
abundância de colheita.

A verdadeira testemunha não mentirá, mas a testemunha falsa se
desboca em mentiras.

O escarnecedor busca sabedoria e não acha nenhuma, para o
prudente, porém, o conhecimento é fácil.

Desvia-te do homem insensato, porque nele não acharás lábios de
conhecimento.

A sabedoria do prudente é entender o seu caminho, mas a estultícia
dos insensatos é engano.

Os insensatos zombam do pecado, mas entre os retos há
benevolência.

O coração conhece a sua própria amargura, e o estranho não
participará no íntimo da sua alegria.

A casa dos ímpios se desfará, mas a tenda dos retos florescerá.

Há um caminho que ao homem parece direito, mas o fim dele são os
caminhos da morte.

Até no riso o coração sente dor e o fim da alegria é tristeza.

O que no seu coração comete deslize, se enfada dos seus caminhos,
mas o homem bom fica satisfeito com o seu proceder.

O simples dá crédito a cada palavra, mas o prudente atenta para
os seus passos.

O sábio teme, e desvia-se do mal, mas o tolo se encoleriza, e
dá-se por seguro.

O que se indigna à toa fará doidices, e o homem de maus intentos
será odiado.

Os simples herdarão a estultícia, mas os prudentes serão coroados
de conhecimento.

Os maus inclinam-se diante dos bons, e os ímpios diante das
portas dos justos.

O pobre é odiado até pelo seu próximo, porém os amigos dos ricos
são muitos.

O que despreza ao seu próximo peca, mas o que se compadece dos
humildes é bem-aventurado.

Porventura não erram os que praticam o mal? mas beneficência e
fidelidade haverá para os que praticam o bem.

Em todo trabalho há proveito, mas ficar só em palavras leva à
pobreza.

A coroa dos sábios é a sua riqueza, a estultícia dos tolos é só
estultícia.

A testemunha verdadeira livra as almas, mas o que se desboca em
mentiras é enganador.

No temor do Senhor há firme confiança e ele será um refúgio para
seus filhos. O temor do Senhor é fonte de vida, para desviar
dos laços da morte.

Na multidão do povo está a glória do rei, mas na falta de povo a
ruína do príncipe.

O longânimo é grande em entendimento, mas o que é de espírito
impaciente mostra a sua loucura.

O sentimento sadio é vida para o corpo, mas a inveja é podridão
para os ossos.

O que oprime o pobre insulta àquele que o criou, mas o que se
compadece do necessitado o honra.

Pela sua própria malícia é lançado fora o perverso, mas o justo
até na morte se mantém confiante.

No coração do prudente a sabedoria permanece, mas o que está no
interior dos tolos se faz conhecido.

A justiça exalta os povos, mas o pecado é a vergonha das nações.

O rei se alegra no servo prudente, mas sobre o que o envergonha
cairá o seu furor.

\medskip

\lettrine{15}{}A resposta branda desvia o furor, mas a palavra
dura suscita a ira. \textparagraph 2 A língua dos sábios adorna a
sabedoria, mas a boca dos tolos derrama a estultícia.

Os olhos do Senhor estão em todo lugar, contemplando os maus e os
bons.

A língua benigna é árvore de vida, mas a perversidade nela deprime
o espírito.

O tolo despreza a instrução de seu pai, mas o que observa a
repreensão se haverá prudentemente.

Na casa do justo há um grande tesouro, mas nos ganhos do ímpio há
perturbação.

Os lábios dos sábios derramam o conhecimento, mas o coração dos
tolos não faz assim.

O sacrifício dos ímpios é abominável ao Senhor, mas a oração dos
retos é o seu contentamento.

O caminho do ímpio é abominável ao Senhor, mas ao que segue a
justiça ele ama.

Correção severa há para o que deixa a vereda, e o que odeia a
repreensão morrerá.

O inferno e a perdição estão perante o Senhor; quanto mais os
corações dos filhos dos homens?

O escarnecedor não ama aquele que o repreende, nem se chegará aos
sábios.

O coração alegre aformoseia o rosto, mas pela dor do coração o
espírito se abate.

O coração entendido buscará o conhecimento, mas a boca dos tolos
se apascentará de estultícia.

Todos os dias do oprimido são maus, mas o coração alegre é um
banquete contínuo.

Melhor é o pouco com o temor do Senhor, do que um grande tesouro
onde há inquietação. Melhor é a comida de hortaliça, onde há
amor, do que o boi cevado, e com ele o ódio.

O homem iracundo\footnote{Propenso à ira; irascível. Irado,
colérico, enfurecido.} suscita contendas, mas o longânimo apaziguará
a luta.

O caminho do preguiçoso é cercado de espinhos, mas a vereda dos
retos é bem aplanada.

O filho sábio alegra seu pai, mas o homem insensato despreza a
sua mãe.

A estultícia é alegria para o que carece de entendimento, mas o
homem entendido anda retamente.

Quando não há conselhos os planos se dispersam, mas havendo
muitos conselheiros eles se firmam.

O homem se alegra em responder bem, e quão boa é a palavra dita a
seu tempo!

Para o entendido, o caminho da vida leva para cima, para que se
desvie do inferno em baixo.

O Senhor desarraiga a casa dos soberbos, mas estabelece o termo
da viúva.

Abomináveis são para o Senhor os pensamentos do mau, mas as
palavras dos puros são aprazíveis.

O que agir com avareza perturba a sua casa, mas o que odeia
presentes viverá.

O coração do justo medita no que há de responder, mas a boca dos
ímpios jorra coisas más.

O Senhor está longe dos ímpios, mas a oração dos justos escutará.

A luz dos olhos alegra o coração, a boa notícia fortalece os
ossos.

Os ouvidos que atendem à repreensão da vida farão a sua morada no
meio dos sábios.

O que rejeita a instrução menospreza a própria alma, mas o que
escuta a repreensão adquire entendimento.

O temor do Senhor é a instrução da sabedoria, e precedendo a
honra vai a humildade.

\medskip

\lettrine{16}{}Do homem são as preparações do coração, mas do
Senhor a resposta da língua. \textparagraph 2 Todos os caminhos do
homem são puros aos seus olhos, mas o Senhor pesa o espírito.

Confia ao Senhor as tuas obras, e teus pensamentos serão
estabelecidos.

O Senhor fez todas as coisas para atender aos seus próprios
desígnios, até o ímpio para o dia do mal.

Abominação é ao Senhor todo o altivo de coração; não ficará impune
mesmo de mãos postas.

Pela misericórdia e verdade a iniqüidade é perdoada, e pelo temor
do Senhor os homens se desviam do pecado.

Sendo os caminhos do homem agradáveis ao Senhor, até a seus
inimigos faz que tenham paz com ele.

Melhor é o pouco com justiça, do que a abundância de bens com
injustiça.

O coração do homem planeja o seu caminho, mas o Senhor lhe dirige
os passos.

Nos lábios do rei se acha a sentença divina; a sua boca não
transgride quando julga.

O peso e a balança justos são do Senhor; obra sua são os pesos da
bolsa.

Abominação é aos reis praticarem impiedade, porque com justiça é
que se estabelece o trono.

Os lábios de justiça são o contentamento dos reis; eles amarão o
que fala coisas retas.

O furor do rei é mensageiro da morte, mas o homem sábio o
apaziguará. No semblante iluminado do rei está a vida, e a
sua benevolência é como a nuvem da chuva serôdia\footnote{Serôdio:
Que vem tarde, fora do tempo; tardio. Diz-se de fruto ou flor que
aparece no fim da estação própria; extemporâneo, serotino, serôtino.
Antiquado, ultrapassado.}.

Quão melhor é adquirir a sabedoria do que o ouro! e quão mais
excelente é adquirir a prudência do que a prata!

Os retos fazem o seu caminho desviar-se do mal; o que guarda o
seu caminho preserva a sua alma.

A soberba precede a ruína, e a altivez do espírito precede a
queda.

Melhor é ser humilde de espírito com os mansos, do que repartir o
despojo com os soberbos.

O que atenta prudentemente para o assunto achará o bem, e o que
confia no Senhor será bem-aventurado.

O sábio de coração será chamado prudente, e a doçura dos lábios
aumentará o ensino.

O entendimento para aqueles que o possuem, é uma fonte de vida,
mas a instrução dos tolos é a sua estultícia.

O coração do sábio instrui a sua boca, e aumenta o ensino dos
seus lábios.

As palavras suaves são favos de mel, doces para a alma, e saúde
para os ossos.

Há um caminho que parece direito ao homem, mas o seu fim são os
caminhos da morte.

O trabalhador trabalha para si mesmo, porque a sua boca o incita.

O homem ímpio cava o mal, e nos seus lábios há como que uma
fogueira. O homem perverso instiga a contenda, e o intrigante
separa os maiores amigos.

O homem violento coage o seu próximo, e o faz deslizar por
caminhos nada bons. O que fecha os olhos para imaginar coisas
ruins, ao cerrar os lábios pratica o mal.

Coroa de honra são as cãs, quando elas estão no caminho da
justiça.

Melhor é o que tarda em irar-se do que o poderoso, e o que
controla o seu ânimo do que aquele que toma uma cidade.

A sorte se lança no regaço, mas do Senhor procede toda a
determinação.

\medskip

\lettrine{17}{}É melhor um bocado seco, e com ele a
tranqüilidade, do que a casa cheia de iguarias e com desavença.
\textparagraph 2 O servo prudente dominará sobre o filho que faz
envergonhar; e repartirá a herança entre os irmãos.

O crisol é para a prata, e o forno para o ouro; mas o Senhor é
quem prova os corações.

O ímpio atenta para o lábio iníquo, o mentiroso inclina os ouvidos
à língua maligna.

O que escarnece do pobre insulta ao seu Criador, o que se alegra
da calamidade não ficará impune.

A coroa dos velhos são os filhos dos filhos; e a glória dos filhos
são seus pais.

Não convém ao tolo a fala excelente; quanto menos ao príncipe, o
lábio mentiroso.

O presente é, aos olhos dos que o recebem, como pedra preciosa;
para onde quer que se volte servirá de proveito.

Aquele que encobre a transgressão busca a amizade, mas o que
revolve o assunto separa os maiores amigos.

A repreensão penetra mais profundamente no prudente do que cem
açoites no tolo.

Na verdade o rebelde não busca senão o mal; afinal, um mensageiro
cruel será enviado contra ele.

Encontre-se o homem com a ursa roubada dos filhos, mas não com o
louco na sua estultícia.

Quanto àquele que paga o bem com o mal, não se apartará o mal da
sua casa.

Como o soltar das águas é o início da contenda, assim, antes que
sejas envolvido afasta-te da questão.

O que justifica o ímpio, e o que condena o justo, tanto um como o
outro são abomináveis ao Senhor.

De que serviria o preço na mão do tolo para comprar sabedoria,
visto que não tem entendimento?

Em todo o tempo ama o amigo e para a hora da angústia nasce o
irmão.

O homem falto de entendimento compromete-se, ficando por fiador
na presença do seu amigo.

O que ama a transgressão ama a contenda; o que exalta a sua porta
busca a ruína.

O perverso de coração jamais achará o bem; e o que tem a língua
dobre vem a cair no mal.

O que gera um tolo para a sua tristeza o faz; e o pai do
insensato não tem alegria.

O coração alegre é como o bom remédio, mas o espírito abatido
seca até os ossos.

O ímpio toma presentes em secreto para perverter as veredas da
justiça.

No rosto do entendido se vê a sabedoria, mas os olhos do tolo
vagam pelas extremidades da terra.

O filho insensato é tristeza para seu pai, e amargura para aquela
que o deu à luz.

Também não é bom punir o justo, nem tampouco ferir aos príncipes
por eqüidade.

O que possui o conhecimento guarda as suas palavras, e o homem de
entendimento é de precioso espírito. Até o tolo, quando se
cala, é reputado por sábio; e o que cerra os seus lábios é tido por
entendido.

\medskip

\lettrine{18}{}Busca satisfazer seu próprio desejo aquele que
se isola; ele se insurge contra toda sabedoria. \textparagraph 2 O
tolo não tem prazer na sabedoria, mas só em que se manifeste aquilo
que agrada o seu coração.

Vindo o ímpio, vem também o desprezo, e com a ignomínia a
vergonha.

Águas profundas são as palavras da boca do homem, e ribeiro
transbordante é a fonte da sabedoria.

Não é bom favorecer o ímpio, e com isso, fazer o justo perder a
questão.

Os lábios do tolo entram na contenda, e a sua boca brada por
açoites. A boca do tolo é a sua própria destruição, e os seus
lábios um laço para a sua alma.

As palavras do mexeriqueiro são como doces bocados; elas descem ao
íntimo do ventre.

O que é negligente na sua obra é também irmão do desperdiçador.

Torre forte é o nome do Senhor; a ela correrá o justo, e estará
em alto refúgio.

Os bens do rico são a sua cidade forte, e como uma muralha na sua
imaginação.

O coração do homem se exalta antes de ser abatido e diante da
honra vai a humildade.

O que responde antes de ouvir comete estultícia que é para
vergonha sua.

O espírito do homem susterá a sua enfermidade, mas ao espírito
abatido, quem o suportará?

O coração do entendido adquire o conhecimento, e o ouvido dos
sábios busca a sabedoria.

Com presentes o homem alarga o seu caminho e o eleva diante dos
grandes.

O que pleiteia por algo, a princípio parece justo, porém vem o
seu próximo e o examina.

A sorte faz cessar os pleitos, e faz separação entre os
poderosos.

O irmão ofendido é mais difícil de conquistar do que uma cidade
forte; e as contendas são como os ferrolhos de um palácio.

Do fruto da boca de cada um se fartará o seu ventre; dos renovos
dos seus lábios ficará satisfeito.

A morte e a vida estão no poder da língua; e aquele que a ama
comerá do seu fruto.

Aquele que encontra uma esposa, acha o bem, e alcança a
benevolência do Senhor.

O pobre fala com rogos, mas o rico responde com dureza.

O homem de muitos amigos deve mostrar-se amigável, mas há um
amigo mais chegado do que um irmão.

\medskip

\lettrine{19}{}Melhor é o pobre que anda na sua integridade do
que o perverso de lábios e tolo. \textparagraph 2 Assim como não é
bom ficar a alma sem conhecimento, peca aquele que se apressa com
seus pés.

A estultícia do homem perverterá o seu caminho, e o seu coração se
irará contra o Senhor.

As riquezas granjeiam muitos amigos, mas ao pobre, o seu próprio
amigo o deixa.

A falsa testemunha não ficará impune e o que respira mentiras não
escapará.

Muitos se deixam acomodar pelos favores do príncipe, e cada um é
amigo daquele que dá presentes. Todos os irmãos do pobre o
odeiam; quanto mais se afastarão dele os seus amigos! Corre após
eles com palavras, que não servem de nada.

O que adquire entendimento ama a sua alma; o que cultiva a
inteligência achará o bem.

A falsa testemunha não ficará impune; e o que profere mentiras
perecerá.

Ao tolo não é certo gozar de deleites; quanto menos ao servo
dominar sobre os príncipes!

A prudência do homem faz reter a sua ira, e é glória sua o passar
por cima da transgressão.

Como o rugido do leão jovem é a indignação do rei, mas como o
orvalho sobre a relva é a sua benevolência.

O filho insensato é uma desgraça para o pai, e um gotejar
contínuo as contendas da mulher.

A casa e os bens são herança dos pais; porém do Senhor vem a
esposa prudente.

A preguiça faz cair em profundo sono, e a alma indolente padecerá
fome.

O que guardar o mandamento guardará a sua alma; porém o que
desprezar os seus caminhos morrerá.

Ao Senhor empresta o que se compadece do pobre, ele lhe pagará o
seu benefício.

Castiga o teu filho enquanto há esperança, mas não deixes que o
teu ânimo se exalte até o matar.

O homem de grande indignação deve sofrer o dano; porque se tu o
livrares ainda terás de tornar a fazê-lo.

Ouve o conselho, e recebe a correção, para que no fim sejas
sábio.

Muitos propósitos há no coração do homem, porém o conselho do
Senhor permanecerá.

O que o homem mais deseja é o que lhe faz bem; porém é melhor ser
pobre do que mentiroso.

O temor do Senhor encaminha para a vida; aquele que o tem ficará
satisfeito, e não o visitará mal nenhum.

O preguiçoso esconde a sua mão ao seio; e não tem disposição nem
de torná-la à sua boca.

Açoita o escarnecedor, e o simples tomará aviso; repreende ao
entendido, e aprenderá conhecimento.

O que aflige o seu pai, ou manda embora sua mãe, é filho que traz
vergonha e desonra.

Filho meu, ouvindo a instrução, cessa de te desviares das
palavras do conhecimento.

O ímpio escarnece do juízo, e a boca dos perversos devora a
iniqüidade.

Preparados estão os juízos para os escarnecedores, e os açoites
para as costas dos tolos.

\medskip

\lettrine{20}{}O vinho é escarnecedor, a bebida forte
alvoroçadora; e todo aquele que neles errar nunca será sábio.
\textparagraph 2 Como o rugido do leão é o terror do rei; o que o
provoca à ira peca contra a sua própria alma.

Honroso é para o homem desviar-se de questões, mas todo tolo é
intrometido.

O preguiçoso não lavrará por causa do inverno, pelo que mendigará
na sega, mas nada receberá.

Como as águas profundas é o conselho no coração do homem; mas o
homem de inteligência o trará para fora.

A multidão dos homens apregoa a sua própria bondade, porém o homem
fidedigno quem o achará?

O justo anda na sua sinceridade; bem-aventurados serão os seus
filhos depois dele.

Assentando-se o rei no trono do juízo, com os seus olhos dissipa
todo o mal.

Quem poderá dizer: Purifiquei o meu coração, limpo estou de meu
pecado?

Dois pesos diferentes e duas espécies de medida são abominação ao
Senhor, tanto um como outro.

Até a criança se dará a conhecer pelas suas ações, se a sua obra
é pura e reta.

O ouvido que ouve, e o olho que vê, o Senhor os fez a ambos.

Não ames o sono, para que não empobreças; abre os teus olhos, e
te fartarás de pão.

Nada vale, nada vale, dirá o comprador, mas, indo-se, então se
gabará.

Há ouro e abundância de rubis, mas os lábios do conhecimento são
jóia preciosa.

Ficando alguém por fiador de um estranho, tome-se-lhe a roupa; e
por penhor àquele que se obriga pela mulher estranha.

Suave é ao homem o pão da mentira, mas depois a sua boca se
encherá de cascalho.

Cada pensamento se confirma com conselho e com bons conselhos se
faz a guerra.

O que anda tagarelando revela o segredo; não te intrometas com o
que lisonjeia com os seus lábios.

O que amaldiçoa seu pai ou sua mãe, apagar-se-á a sua lâmpada em
negras trevas.

A herança que no princípio é adquirida às pressas, no fim não
será abençoada.

Não digas: Vingar-me-ei do mal; espera pelo Senhor, e ele te
livrará.

Pesos diferentes são abomináveis ao Senhor, e balança enganosa
não é boa.

Os passos do homem são dirigidos pelo Senhor; como, pois,
entenderá o homem o seu caminho?

Laço é para o homem apropriar-se do que é santo, e só refletir
depois de feitos os votos.

O rei sábio dispersa os ímpios e faz passar sobre eles a roda.

O espírito do homem é a lâmpada do Senhor, que esquadrinha todo o
interior até o mais íntimo do ventre.

Benignidade e verdade guardam ao rei, e com benignidade sustém
ele o seu trono.

A glória do jovem é a sua força; e a beleza dos velhos são as
cãs.

Os vergões\footnote{Vergão: vinco ou marca na pele, produzido por
pancada, sobretudo de vergasta ou azorrague, ou por outra causa.}
das feridas são a purificação dos maus, como também as pancadas que
penetram até o mais íntimo do ventre.

\medskip

\lettrine{21}{}Como ribeiros de águas assim é o coração do rei
na mão do Senhor, que o inclina a todo o seu querer. \textparagraph
Todo caminho do homem é reto aos seus olhos, mas o Senhor sonda os
corações.

Fazer justiça e juízo é mais aceitável ao Senhor do que
sacrifício.

Os olhos altivos, o coração orgulhoso e a lavoura dos ímpios é
pecado.

Os pensamentos do diligente tendem só para a abundância, porém os
de todo apressado, tão-somente para a pobreza.

Trabalhar com língua falsa para ajuntar tesouros é vaidade que
conduz aqueles que buscam a morte.

As rapinas dos ímpios os destruirão, porquanto se recusam a fazer
justiça.

O caminho do homem é todo perverso e estranho, porém a obra do
homem puro é reta.

É melhor morar num canto de telhado do que ter como companheira em
casa ampla uma mulher briguenta.

A alma do ímpio deseja o mal; o seu próximo não agrada aos seus
olhos.

Quando o escarnecedor é castigado, o simples torna-se sábio; e o
sábio quando é instruído recebe o conhecimento.

O justo considera com prudência a casa do ímpio; mas Deus destrói
os ímpios por causa dos seus males.

O que tapa o seu ouvido ao clamor do pobre, ele mesmo também
clamará e não será ouvido.

O presente dado em segredo aplaca a ira, e a dádiva no regaço põe
fim à maior indignação.

O fazer justiça é alegria para o justo, mas destruição para os
que praticam a iniqüidade.

O homem que anda desviado do caminho do entendimento, na
congregação dos mortos repousará.

O que ama os prazeres padecerá necessidade; o que ama o vinho e o
azeite nunca enriquecerá.

O resgate do justo é o ímpio; o do honrado é o perverso.

É melhor morar numa terra deserta do que com a mulher rixosa e
irritadiça.

Tesouro desejável e azeite há na casa do sábio, mas o homem
insensato os esgota.

O que segue a justiça e a beneficência achará a vida, a justiça e
a honra.

O sábio escala a cidade do poderoso e derruba a força da sua
confiança.

O que guarda a sua boca e a sua língua guarda a sua alma das
angústias.

O soberbo e presumido, zombador é o seu nome, trata com
indignação e soberba.

O desejo do preguiçoso o mata, porque as suas mãos recusam
trabalhar. O cobiçoso cobiça o dia todo, mas o justo dá, e
nada retém.

O sacrifício dos ímpios já é abominação; quanto mais oferecendo-o
com má intenção!

A falsa testemunha perecerá, porém o homem que dá ouvidos falará
sempre.

O homem ímpio endurece o seu rosto; mas o reto considera o seu
caminho. Não há sabedoria, nem inteligência, nem conselho
contra o Senhor. Prepara-se o cavalo para o dia da batalha,
porém do Senhor vem a vitória.

\medskip

\lettrine{22}{}Vale mais ter um bom nome do que muitas
riquezas; e o ser estimado é melhor do que a riqueza e o ouro.
\textparagraph 2 O rico e o pobre se encontram; a todos o Senhor os
fez.

O prudente prevê o mal, e esconde-se; mas os simples passam e
acabam pagando.

O galardão da humildade e o temor do Senhor são riquezas, honra e
vida.

Espinhos e laços há no caminho do perverso; o que guarda a sua
alma retira-se para longe dele.

Educa a criança no caminho em que deve andar; e até quando
envelhecer não se desviará dele.

O rico domina sobre os pobres e o que toma emprestado é servo do
que empresta.

O que semear a perversidade segará males; e com a vara da sua
própria indignação será extinto.

O que vê com bons olhos será abençoado, porque dá do seu pão ao
pobre.

Lança fora o escarnecedor, e se irá a contenda; e acabará a
questão e a vergonha.

O que ama a pureza de coração, e é amável de lábios, será amigo
do rei.

Os olhos do Senhor conservam o conhecimento, mas as palavras do
iníquo ele transtornará.

Diz o preguiçoso: Um leão está lá fora; serei morto no meio das
ruas.

Cova profunda é a boca das mulheres estranhas; aquele contra quem
o Senhor se irar, cairá nela.

A estultícia está ligada ao coração da criança, mas a vara da
correção a afugentará dela.

O que oprime ao pobre para se engrandecer a si mesmo, ou o que dá
ao rico, certamente empobrecerá.

Inclina o teu ouvido e ouve as palavras dos sábios, e aplica o
teu coração ao meu conhecimento. Porque te será agradável se
as guardares no teu íntimo, se aplicares todas elas aos teus lábios.
Para que a tua confiança esteja no Senhor, faço-te sabê-las
hoje, a ti mesmo. Porventura não te escrevi excelentes
coisas, acerca de todo conselho e conhecimento, para fazer-te
saber a certeza das palavras da verdade, e assim possas responder
palavras de verdade aos que te consultarem?

Não roubes ao pobre, porque é pobre, nem atropeles na porta o
aflito; porque o Senhor defenderá a sua causa em juízo, e aos
que os roubam ele lhes tirará a vida.

Não sejas companheiro do homem briguento nem andes com o
colérico, para que não aprendas as suas veredas, e tomes um
laço para a tua alma.

Não estejas entre os que se comprometem, e entre os que ficam por
fiadores de dívidas, pois se não tens com que pagar,
deixarias que te tirassem até a tua cama de debaixo de ti?

Não removas os antigos limites que teus pais fizeram.

Viste o homem diligente na sua obra? Perante reis será posto; não
permanecerá entre os de posição inferior.

\medskip

\lettrine{23}{}Quando te assentares a comer com um governador,
atenta bem para o que é posto diante de ti, e se és homem de
grande apetite, põe uma faca à tua garganta. Não cobices as suas
iguarias porque são comidas enganosas.

Não te fatigues para enriqueceres; e não apliques nisso a tua
sabedoria. Porventura fixarás os teus olhos naquilo que não é
nada? porque certamente criará asas e voará ao céu como a águia.

Não comas o pão daquele que tem o olhar maligno, nem cobices as
suas iguarias gostosas. Porque, como imaginou no seu coração,
assim é ele. Come e bebe, te disse ele; porém o seu coração não está
contigo. Vomitarás o bocado que comeste, e perderás as tuas
suaves palavras.

Não fales ao ouvido do tolo, porque desprezará a sabedoria das
tuas palavras.

Não removas os limites antigos nem entres nos campos dos órfãos,
porque o seu redentor é poderoso; e pleiteará a causa deles
contra ti.

Aplica o teu coração à instrução e os teus ouvidos às palavras do
conhecimento. Não retires a disciplina da criança; pois se a
fustigares com a vara, nem por isso morrerá. Tu a fustigarás
com a vara, e livrarás a sua alma do inferno. Filho meu, se o
teu coração for sábio, alegrar-se-á o meu coração, sim, o meu
próprio. E exultarão os meus rins, quando os teus lábios
falarem coisas retas.

O teu coração não inveje os pecadores; antes permanece no temor
do Senhor todo dia. Porque certamente acabará bem; não será
malograda a tua esperança.

Ouve tu, filho meu, e sê sábio, e dirige no caminho o teu
coração. Não estejas entre os beberrões de vinho, nem entre
os comilões de carne. Porque o beberrão e o comilão acabarão
na pobreza; e a sonolência os faz vestir-se de trapos. Ouve
teu pai, que te gerou, e não desprezes tua mãe, quando vier a
envelhecer. Compra a verdade, e não a vendas; e também a
sabedoria, a instrução e o entendimento. Grandemente se
regozijará o pai do justo, e o que gerar um sábio, se alegrará nele.
Alegrem-se teu pai e tua mãe, e regozije-se a que te gerou.
Dá-me, filho meu, o teu coração, e os teus olhos observem os
meus caminhos. Porque cova profunda é a prostituta, e poço
estreito a estranha. Pois ela, como um salteador, se põe à
espreita, e multiplica entre os homens os iníquos.

Para quem são os ais? Para quem os pesares? Para quem as pelejas?
Para quem as queixas? Para quem as feridas sem causa? E para quem os
olhos vermelhos? Para os que se demoram perto do vinho, para
os que andam buscando vinho misturado. Não olhes para o vinho
quando se mostra vermelho, quando resplandece no copo e se escoa
suavemente. No fim, picará como a cobra, e como o basilisco
morderá. Os teus olhos olharão para as mulheres estranhas, e
o teu coração falará perversidades. E serás como o que se
deita no meio do mar, e como o que jaz no topo do mastro. E
dirás: Espancaram-me e não me doeu; bateram-me e nem senti; quando
despertarei? aí então beberei outra vez.

\medskip

\lettrine{24}{}Não tenhas inveja dos homens malignos, nem
desejes estar com eles. Porque o seu coração medita a rapina, e
os seus lábios falam a malícia.

Com a sabedoria se edifica a casa, e com o entendimento ela se
estabelece; e pelo conhecimento se encherão as câmaras com todos
os bens preciosos e agradáveis. O homem sábio é forte, e o homem
de conhecimento consolida a força. Com conselhos prudentes tu
farás a guerra; e há vitória na multidão dos conselheiros.

A sabedoria é demasiadamente alta para o tolo, na porta não abrirá
a sua boca. Àquele que cuida em fazer mal, chamá-lo-ão de pessoa
danosa. O pensamento do tolo é pecado, e abominável aos homens é
o escarnecedor.

Se te mostrares fraco no dia da angústia, é que a tua força é
pequena.

Se tu deixares de livrar os que estão sendo levados para a morte,
e aos que estão sendo levados para a matança; se disseres:
Eis que não o sabemos; porventura não o considerará aquele que
pondera os corações? Não o saberá aquele que atenta para a tua alma?
Não dará ele ao homem conforme a sua obra?

Come mel, meu filho, porque é bom; o favo de mel é doce ao teu
paladar. Assim será para a tua alma o conhecimento da
sabedoria; se a achares, haverá galardão para ti e não será cortada
a tua esperança.

Não armes ciladas contra a habitação do justo, ó ímpio, nem
assoles o seu lugar de repouso, porque sete vezes cairá o
justo, e se levantará; mas os ímpios tropeçarão no mal.

Quando cair o teu inimigo, não te alegres, nem se regozije o teu
coração quando ele tropeçar; para que, vendo-o o Senhor, seja
isso mau aos seus olhos, e desvie dele a sua ira.

Não te indignes por causa dos malfeitores, nem tenhas inveja dos
ímpios, porque o homem maligno não terá galardão, e a lâmpada
dos ímpios se apagará.

Teme ao Senhor, filho meu, e ao rei, e não te ponhas com os que
buscam mudanças, porque de repente se levantará a sua
destruição, e a ruína de ambos, quem o sabe?

Também estes são provérbios dos sábios: Ter respeito a pessoas no
julgamento não é bom.\footnote{RC-1969: Também estes são provérbios
dos sábios: Ter respeito a pessoas no juízo não é bom. RA: São
também estes provérbios dos sábios. Parcialidade no julgar não é
bom. AV: These things also belong to the wise. It is not good to
have respect of persons in judgment.} O que disser ao ímpio:
Justo és, os povos o amaldiçoarão, as nações o detestarão.
Mas para os que o repreenderem haverá delícias, e sobre eles
virá a bênção do bem. Beijados serão os lábios do que
responde com palavras retas.

Prepara de fora a tua obra, e aparelha-a no campo, e então
edifica a tua casa.

Não sejas testemunha sem causa contra o teu próximo; e não
enganes com os teus lábios. Não digas: Como ele me fez a mim,
assim o farei eu a ele; pagarei a cada um segundo a sua obra.

Passei pelo campo do preguiçoso, e junto à vinha do homem falto
de entendimento, eis que estava toda cheia de cardos, e a sua
superfície coberta de urtiga, e o seu muro de pedras estava
derrubado. O que eu tenho visto, o guardarei no coração, e
vendo-o recebi instrução. Um pouco a dormir, um pouco a
cochilar; outro pouco deitado de mãos cruzadas, para dormir,
assim te sobrevirá a tua pobreza como um vagabundo, e a tua
necessidade como um homem armado.

\medskip

\lettrine{25}{}Também estes são provérbios de Salomão, os quais
transcreveram os homens de Ezequias, rei de Judá. \textparagraph 2 A
glória de Deus está nas coisas encobertas; mas a honra dos reis,
está em descobri-las. Os céus, pela altura, e a terra, pela
profundidade, assim o coração dos reis é insondável.

Tira da prata as escórias, e sairá vaso para o fundidor; tira
o ímpio da presença do rei, e o seu trono se firmará na justiça.

Não te glories na presença do rei, nem te ponhas no lugar dos
grandes; porque melhor é que te digam: Sobe aqui; do que seres
humilhado diante do príncipe que os teus olhos já viram.

Não te precipites em litigar\footnote{Pleitear ou questionar em
juízo. Ter litígio, demanda ou questão. Entrar em luta; pelejar,
lidar; contender.}, para que depois, ao fim, fiques sem ação, quando
teu próximo te puser em apuros. Pleiteia a tua causa com o teu
próximo, e não reveles o problema a outrem, para que não te
desonre o que o ouvir, e a tua infâmia não se aparte de ti.

Como maçãs de ouro em salvas\footnote{Tipo de bandeja redonda e
pequena. (Originariamente era a prova que se fazia da comida e da
bebida que iam ser servidas ao rei e grão-senhores para salvá-los de
possível envenenamento; o prato em que eram servidas tomou o nome de
salva).} de prata, assim é a palavra dita a seu tempo. Como
pendentes de ouro e gargantilhas de ouro fino, assim é o sábio
repreensor para o ouvido atento.

Como o frio da neve no tempo da sega, assim é o mensageiro fiel
para com os que o enviam; porque refresca a alma dos seus senhores.

Como nuvens e ventos que não trazem chuva, assim é o homem que se
gaba falsamente de dádivas.

Pela longanimidade se persuade o príncipe, e a língua branda
amolece até os ossos.

Achaste mel? come só o que te basta; para que porventura não te
fartes dele, e o venhas a vomitar.

Não ponhas muito os pés na casa do teu próximo; para que se não
enfade de ti, e passe a te odiar.

Martelo, espada e flecha aguda é o homem que profere falso
testemunho contra o seu próximo.

Como dente quebrado, e pé desconjuntado, é a confiança no
desleal, no tempo da angústia.

O que canta canções para o coração aflito é como aquele que despe
a roupa num dia de frio, ou como o vinagre sobre salitre\footnote{O
nitrato de potássio; nitro.}.

Se o teu inimigo tiver fome, dá-lhe pão para comer; e se tiver
sede, dá-lhe água para beber; porque assim lhe amontoarás
brasas sobre a cabeça; e o Senhor to retribuirá.

O vento norte afugenta a chuva, e a face irada, a língua fingida.

Melhor é morar só num canto de telhado do que com a mulher
briguenta numa casa ampla.

Como água fresca para a alma cansada, tais são as boas novas
vindas da terra distante.

Como fonte turvada, e manancial poluído, assim é o justo que cede
diante do ímpio.

Comer mel demais não é bom; assim, a busca da própria glória não
é glória.

Como a cidade derrubada, sem muro, assim é o homem que não pode
conter o seu espírito.

\medskip

\lettrine{26}{}Como a neve no verão, e como a chuva na sega,
assim não fica bem para o tolo a honra. \textparagraph 2 Como ao
pássaro o vaguear, como à andorinha o voar, assim a maldição sem
causa não virá.

O açoite é para o cavalo, o freio é para o jumento, e a vara é
para as costas dos tolos.

Não respondas ao tolo segundo a sua estultícia; para que também
não te faças semelhante a ele. Responde ao tolo segundo a sua
estultícia, para que não seja sábio aos seus próprios olhos.

Os pés corta, e o dano sorve\footnote{Haurir ou beber, aspirando.
Beber aos sorvos ou aos poucos. Embeber-se ou impregnar-se de;
chupar, sugar; absorver. Atrair para baixo; tragar. Submergir,
afundar, subverter. Fig. Abrigar, recolher. Destruir, aniquilar;
devastar.}, aquele que manda mensagem pela mão dum tolo. Como as
pernas do coxo, que pendem flácidas, assim é o provérbio na boca dos
tolos. Como o que arma a funda com pedra preciosa, assim é
aquele que concede honra ao tolo. Como o espinho que entra na
mão do bêbado, assim é o provérbio na boca dos tolos.

O Poderoso, que formou todas as coisas, paga ao tolo, e
recompensa ao transgressor.

Como o cão torna ao seu vômito, assim o tolo repete a sua
estultícia.

Tens visto o homem que é sábio a seus próprios olhos? Pode-se
esperar mais do tolo do que dele.

Diz o preguiçoso: Um leão está no caminho; um leão está nas ruas.

Como a porta gira nos seus gonzos, assim o preguiçoso na sua
cama.

O preguiçoso esconde a sua mão ao seio; e cansa-se até de
torná-la à sua boca.

Mais sábio é o preguiçoso a seus próprios olhos do que sete
homens que respondem bem.

O que, passando, se põe em questão alheia, é como aquele que pega
um cão pelas orelhas.

Como o louco que solta faíscas, flechas, e mortandades,
assim é o homem que engana o seu próximo, e diz: Fiz isso por
brincadeira.

Sem lenha, o fogo se apagará; e não havendo intrigante, cessará a
contenda. Como o carvão para as brasas, e a lenha para o
fogo, assim é o homem contencioso para acender rixas.
 As palavras do intrigante são como doces bocados; elas descem ao
mais íntimo do ventre.

Como o caco de vaso coberto de escórias de prata, assim são os
lábios ardentes com o coração maligno.

Aquele que odeia dissimula com seus lábios, mas no seu íntimo
encobre o engano; quando te suplicar com voz suave não te
fies nele, porque abriga sete abominações no seu coração,
cujo ódio se encobre com engano, a sua maldade será exposta
perante a congregação.

O que cava uma cova cairá nela; e o que revolve a pedra, esta
voltará sobre ele.

A língua falsa odeia aos que ela fere, e a boca lisonjeira
provoca a ruína.

\medskip

\lettrine{27}{}Não presumas do dia de amanhã, porque não sabes
o que ele trará. \textparagraph 2 Que um outro te louve, e não a tua
própria boca; o estranho, e não os teus lábios.

A pedra é pesada, e a areia é espessa; porém a ira do insensato é
mais pesada que ambas. O furor é cruel e a ira impetuosa, mas
quem poderá enfrentar a inveja?

Melhor é a repreensão franca do que o amor encoberto. Leais
são as feridas feitas pelo amigo, mas os beijos do inimigo são
enganosos.

A alma farta pisa o favo de mel, mas para a alma faminta todo
amargo é doce.

Qual a ave que vagueia longe do seu ninho, tal é o homem que anda
vagueando longe da sua morada.

O óleo e o perfume alegram o coração; assim o faz a doçura do
amigo pelo conselho cordial. Não deixes o teu amigo, nem o
amigo de teu pai; nem entres na casa de teu irmão no dia da tua
adversidade; melhor é o vizinho perto do que o irmão longe.

Sê sábio, filho meu, e alegra o meu coração, para que tenha
alguma coisa que responder àquele que me desprezar.

O avisado vê o mal e esconde-se; mas os simples passam e sofrem a
pena.

Quando alguém fica por fiador do estranho, toma-lhe até a sua
roupa, e por penhor àquele que se obriga pela mulher estranha.

O que, pela manhã de madrugada, abençoa o seu amigo em alta voz,
lho será imputado por maldição.

O gotejar contínuo em dia de grande chuva, e a mulher
contenciosa, uma e outra são semelhantes; tentar moderá-la
será como deter o vento, ou como conter o óleo dentro da sua mão
direita.

Como o ferro com ferro se aguça, assim o homem afia o rosto do
seu amigo.

O que cuida da figueira comerá do seu fruto; e o que atenta para
o seu senhor será honrado.

Como na água o rosto corresponde ao rosto, assim o coração do
homem ao homem.

Como o inferno e a perdição nunca se fartam, assim os olhos do
homem nunca se satisfazem.

Como o crisol é para a prata, e o forno para o ouro, assim o
homem é provado pelos louvores.

Ainda que repreendas o tolo como quem bate o trigo com a mão de
gral\footnote{Almofariz: recipiente de pedra, metal, etc., em que se
trituram e homogeneízam substâncias sólidas; pilão, gral, morteiro.}
entre grãos pilados, não se apartará dele a sua estultícia.

Procura conhecer o estado das tuas ovelhas; põe o teu coração
sobre os teus rebanhos, porque o tesouro não dura para
sempre; e durará a coroa de geração em geração? Quando brotar
a erva, e aparecerem os renovos, e se juntarem as ervas dos montes,
então os cordeiros serão para te vestires, e os bodes para o
preço do campo; e a abastança do leite das cabras para o teu
sustento, para sustento da tua casa e para sustento das tuas servas.

\medskip

\lettrine{28}{}Os ímpios fogem sem que haja ninguém a
persegui-los; mas os justos são ousados como um leão. \textparagraph
 2 Pela transgressão da terra muitos são os seus príncipes, mas por
homem prudente e entendido a sua continuidade será prolongada.

O homem pobre que oprime os pobres é como a chuva impetuosa, que
causa a falta de alimento.

Os que deixam a lei louvam o ímpio; porém os que guardam a lei
contendem com eles.

Os homens maus não entendem o juízo, mas os que buscam ao Senhor
entendem tudo.

Melhor é o pobre que anda na sua integridade do que o de caminhos
perversos ainda que seja rico.

O que guarda a lei é filho sábio, mas o companheiro dos
desregrados envergonha a seu pai.

O que aumenta os seus bens com usura e ganância ajunta-os para o
que se compadece do pobre.

O que desvia os seus ouvidos de ouvir a lei, até a sua oração será
abominável.

O que faz com que os retos errem por mau caminho, ele mesmo cairá
na sua cova; mas os bons herdarão o bem.

O homem rico é sábio aos seus próprios olhos, mas o pobre que é
entendido, o examina.

Quando os justos exultam, grande é a glória; mas quando os ímpios
sobem, os homens se escondem.

O que encobre as suas transgressões nunca prosperará, mas o que
as confessa e deixa, alcançará misericórdia.

Bem-aventurado o homem que continuamente teme; mas o que endurece
o seu coração cairá no mal.

Como leão rugidor, e urso faminto, assim é o ímpio que domina
sobre um povo pobre.

O príncipe falto de entendimento é também um grande opressor, mas
o que odeia a avareza prolongará seus dias.

O homem carregado do sangue de qualquer pessoa fugirá até à cova;
ninguém o detenha.

O que anda sinceramente salvar-se-á, mas o perverso em seus
caminhos cairá logo.

O que lavrar a sua terra virá a fartar-se de pão, mas o que segue
a ociosos se fartará de pobreza.

O homem fiel será coberto de bênçãos, mas o que se apressa a
enriquecer não ficará impune.

Dar importância à aparência das pessoas não é bom, porque até por
um bocado de pão um homem prevaricará.

O que quer enriquecer depressa é homem de olho maligno, porém não
sabe que a pobreza há de vir sobre ele.

O que repreende o homem gozará depois mais amizade do que aquele
que lisonjeia com a língua.

O que rouba a seu próprio pai, ou a sua mãe, e diz: Não é
transgressão, companheiro é do homem destruidor.

O orgulhoso de coração levanta contendas, mas o que confia no
Senhor prosperará.

O que confia no seu próprio coração é insensato, mas o que anda
em sabedoria, será salvo.

O que dá ao pobre não terá necessidade, mas o que esconde os seus
olhos terá muitas maldições.

Quando os ímpios se elevam, os homens andam se escondendo, mas
quando perecem, os justos se multiplicam.

\medskip

\lettrine{29}{}O homem que muitas vezes repreendido endurece a
cerviz, de repente será destruído sem que haja remédio.

Quando os justos se engrandecem, o povo se alegra, mas quando o
ímpio domina, o povo geme.

O homem que ama a sabedoria alegra a seu pai, mas o companheiro de
prostitutas desperdiça os bens.

O rei com juízo sustém a terra, mas o amigo de peitas a
transtorna.

O homem que lisonjeia o seu próximo arma uma rede aos seus passos.

Na transgressão do homem mau há laço, mas o justo jubila e se
alegra.

O justo se informa da causa dos pobres, mas o ímpio nem sequer
toma conhecimento.

Os homens escarnecedores alvoroçam a cidade, mas os sábios desviam
a ira.

O homem sábio que pleiteia com o tolo, quer se zangue, quer se
ria, não terá descanso.

Os homens sanguinários odeiam ao sincero, mas os justos procuram
o seu bem.

O tolo revela todo o seu pensamento, mas o sábio o guarda até o
fim.

O governador que dá atenção às palavras mentirosas, achará que
todos os seus servos são ímpios.

O pobre e o usurário\footnote{Que ou aquele que usura. Pop.
agiota.} se encontram; o Senhor ilumina os olhos de ambos.

O rei que julga os pobres conforme a verdade firmará o seu trono
para sempre.

A vara e a repreensão dão sabedoria, mas a criança entregue a si
mesma, envergonha a sua mãe.

Quando os ímpios se multiplicam, multiplicam-se as transgressões,
mas os justos verão a sua queda.

Castiga o teu filho, e te dará descanso; e dará delícias à tua
alma.

Não havendo profecia, o povo perece; porém o que guarda a lei,
esse é bem-aventurado.

O servo não se emendará com palavras, porque, ainda que entenda,
todavia não atenderá.

Tens visto um homem precipitado no falar? Maior esperança há para
um tolo do que para ele.

Quando alguém cria o seu servo com mimos desde a meninice, por
fim ele tornar-se-á seu filho.

O homem irascível levanta contendas; e o furioso multiplica as
transgressões.

A soberba do homem o abaterá, mas a honra sustentará o humilde de
espírito.

O que tem parte com o ladrão odeia a sua própria alma; ouve
maldições, e não o denuncia.

O temor do homem armará laços, mas o que confia no Senhor será
posto em alto retiro.

Muitos buscam o favor do poderoso, mas o juízo de cada um vem do
Senhor.

Abominação é, para os justos, o homem iníquo; mas abominação é
para o iníquo o de retos caminhos.

\medskip

\lettrine{30}{}Palavras de Agur, filho de Jaque, o masaíta, que
proferiu este homem a Itiel, a Itiel e a Ucal: Na verdade eu sou
o mais bruto dos homens, nem mesmo tenho o conhecimento de homem.
Nem aprendi a sabedoria, nem tenho o conhecimento do santo.
Quem subiu ao céu e desceu? Quem encerrou os ventos nos seus
punhos? Quem amarrou as águas numa roupa? Quem estabeleceu todas as
extremidades da terra? Qual é o seu nome? E qual é o nome de seu
filho, se é que o sabes? Toda a Palavra de Deus é pura; escudo é
para os que confiam nele. Nada acrescentes às suas palavras,
para que não te repreenda e sejas achado mentiroso.

Duas coisas te pedi; não mas negues, antes que morra: afasta
de mim a vaidade e a palavra mentirosa; não me dês nem a pobreza nem
a riqueza; mantém-me do pão da minha porção de costume; para
que, porventura, estando farto não te negue, e venha a dizer: Quem é
o Senhor? ou que, empobrecendo, não venha a furtar, e tome o nome de
Deus em vão.

Não acuses o servo diante de seu senhor, para que não te
amaldiçoe e tu fiques o culpado. Há uma geração que amaldiçoa
a seu pai, e que não bendiz a sua mãe. Há uma geração que é
pura aos seus próprios olhos, mas que nunca foi lavada da sua
imundícia. Há uma geração cujos olhos são altivos, e as suas
pálpebras são sempre levantadas. Há uma geração cujos dentes
são espadas, e cujas queixadas são facas, para consumirem da terra
os aflitos, e os necessitados dentre os homens.

A sanguessuga tem duas filhas: Dá e Dá. Estas três coisas nunca
se fartam; e com a quarta, nunca dizem: Basta! A sepultura; a
madre estéril; a terra que não se farta de água; e o fogo; nunca
dizem: Basta! Os olhos que zombam do pai, ou desprezam a
obediência à mãe, corvos do ribeiro os arrancarão e os filhotes da
águia os comerão.

Estas três coisas me maravilham; e quatro há que não conheço:
o caminho da águia no ar; o caminho da cobra na penha; o
caminho do navio no meio do mar; e o caminho do homem com uma
virgem. O caminho da mulher adúltera é assim: ela come,
depois limpa a sua boca e diz: Não fiz nada de mal! Por três
coisas se alvoroça a terra; e por quatro que não pode suportar:
pelo servo, quando reina; e pelo tolo, quando vive na
fartura; pela mulher odiosa, quando é casada; e pela serva,
quando fica herdeira da sua senhora.

Estas quatro coisas são das menores da terra, porém bem providas
de sabedoria: as formigas não são um povo forte; todavia no
verão preparam a sua comida; os coelhos são um povo débil; e
contudo, põem a sua casa na rocha; os gafanhotos não têm rei;
e contudo todos saem, e em bandos se repartem; a aranha se
pendura com as mãos, e está nos palácios dos reis.

Estes três têm um bom andar, e quatro passeiam airosamente;
o leão, o mais forte entre os animais, que não foge de nada;
o galo; o bode também; e o rei a quem não se pode resistir.
Se procedeste loucamente, exaltando-te, e se planejaste o
mal, leva a mão à boca; porque o mexer do leite produz
manteiga, o espremer do nariz produz sangue; assim o forçar da ira
produz contenda.

\medskip

\lettrine{31}{}Palavras do rei Lemuel, a profecia que lhe
ensinou a sua mãe. Como, filho meu? e como, filho do meu ventre?
e como, filho dos meus votos? Não dês às mulheres a tua força,
nem os teus caminhos ao que destrói os reis. Não é próprio dos
reis, ó Lemuel, não é próprio dos reis beber vinho, nem dos
príncipes o desejar bebida forte; para que bebendo, se esqueçam
da lei, e pervertam o direito de todos os aflitos. Dai bebida
forte ao que está prestes a perecer, e o vinho aos amargurados de
espírito. Que beba, e esqueça da sua pobreza, e da sua miséria
não se lembre mais. Abre a tua boca a favor do mudo, pela causa
de todos que são designados à destruição. Abre a tua boca; julga
retamente; e faze justiça aos pobres e aos necessitados.

Mulher virtuosa quem a achará? O seu valor muito excede ao de
rubis. O coração do seu marido está nela confiado; assim ele
não necessitará de despojo. Ela só lhe faz bem, e não mal,
todos os dias da sua vida. Busca lã e linho, e trabalha de
boa vontade com suas mãos. Como o navio mercante, ela traz de
longe o seu pão. Levanta-se, mesmo à noite, para dar de comer
aos da casa, e distribuir a tarefa das servas. Examina uma
propriedade e adquire-a; planta uma vinha com o fruto de suas mãos.
Cinge os seus lombos de força, e fortalece os seus braços.
Vê que é boa a sua mercadoria; e a sua lâmpada não se apaga
de noite. Estende as suas mãos ao fuso\footnote{Instrumento
roliço sobre o qual se forma, ao fiar, a maçaroca (fio que o fuso
enrolou em torno de si).}, e suas mãos pegam na roca. Abre a
sua mão ao pobre, e estende as suas mãos ao necessitado. Não
teme a neve na sua casa, porque toda a sua família está vestida de
escarlata. Faz para si cobertas de tapeçaria; seu vestido é
de seda e de púrpura. Seu marido é conhecido nas portas, e
assenta-se entre os anciãos da terra. Faz panos de linho fino
e vende-os, e entrega cintos aos mercadores. A força e a
honra são seu vestido, e se alegrará com o dia futuro. Abre a
sua boca com sabedoria, e a lei da beneficência está na sua língua.
Está atenta ao andamento da casa, e não come o pão da
preguiça. Levantam-se seus filhos e chamam-na bem-aventurada;
seu marido também, e ele a louva. Muitas filhas têm procedido
virtuosamente, mas tu és, de todas, a mais excelente!
Enganosa é a beleza e vã a formosura, mas a mulher que teme
ao Senhor, essa sim será louvada. Dai-lhe do fruto das suas
mãos, e deixe o seu próprio trabalho louvá-la nas portas.

