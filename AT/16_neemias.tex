\addchap{Neemias}

\lettrine{1} As palavras de Neemias, filho de Hacalias. E
sucedeu no mês de Quislev, no ano vigésimo, estando eu em Susã, a
fortaleza, que veio Hanani, um de meus irmãos, ele e alguns de
Judá; e perguntei-lhes pelos judeus que escaparam, e que restaram do
cativeiro, e acerca de Jerusalém. E disseram-me: Os restantes,
que ficaram do cativeiro, lá na província estão em grande miséria e
desprezo; e o muro de Jerusalém fendido e as suas portas queimadas a
fogo. E sucedeu que, ouvindo eu estas palavras, assentei-me e
chorei, e lamentei por alguns dias; e estive jejuando e orando
perante o Deus dos céus.

E disse: Ah! Senhor Deus dos céus, Deus grande e terrível! Que
guarda a aliança e a benignidade para com aqueles que o amam e
guardam os seus mandamentos; estejam, pois, atentos os teus
ouvidos e os teus olhos abertos, para ouvires a oração do teu servo,
que eu hoje faço perante ti, dia e noite, pelos filhos de Israel,
teus servos; e faço confissão pelos pecados dos filhos de Israel,
que temos cometido contra ti; também eu e a casa de meu pai temos
pecado. De todo nos corrompemos contra ti, e não guardamos os
mandamentos, nem os estatutos, nem os juízos, que ordenaste a
Moisés, teu servo. Lembra-te, pois, da palavra que ordenaste a
Moisés, teu servo, dizendo: Vós transgredireis, e eu vos espalharei
entre os povos. E vós vos convertereis a mim, e guardareis os
meus mandamentos, e os cumprireis; então, ainda que os vossos
rejeitados estejam na extremidade do céu, de lá os ajuntarei e os
trarei ao lugar que tenho escolhido para ali fazer habitar o meu
nome. Eles são teus servos e o teu povo que resgataste com a
tua grande força e com a tua forte mão. Ah! Senhor, estejam,
pois, atentos os teus ouvidos à oração do teu servo, e à oração dos
teus servos que desejam temer o teu nome; e faze prosperar hoje o
teu servo, e dá-lhe graça perante este homem. Então era eu copeiro
do rei.

\medskip

\lettrine{2} Sucedeu, pois, no mês de Nisã, no ano vigésimo do
rei Artaxerxes, que estava posto vinho diante dele, e eu peguei o
vinho e o dei ao rei; porém eu nunca estivera triste diante dele.
E o rei me disse: Por que está triste o teu rosto, pois não
estás doente? Não é isto senão tristeza de coração; então temi
sobremaneira. E disse ao rei: Viva o rei para sempre! Como não
estaria triste o meu rosto, estando a cidade, o lugar dos sepulcros
de meus pais, assolada, e tendo sido consumidas as suas portas a
fogo? E o rei me disse: Que me pedes agora? Então orei ao Deus
dos céus, e disse ao rei: Se é do agrado do rei, e se o teu
servo é aceito em tua presença, peço-te que me envies a Judá, à
cidade dos sepulcros de meus pais, para que eu a reedifique.
Então o rei me disse, estando a rainha assentada junto a ele:
Quanto durará a tua viagem, e quando voltarás? E aprouve ao rei
enviar-me, apontando-lhe eu um certo tempo. Disse mais ao rei:
Se ao rei parece bem, dêem-se-me cartas para os governadores dalém
do rio, para que me permitam passar até que chegue a Judá. Como
também uma carta para Asafe, guarda da floresta do rei, para que me
dê madeira para cobrir as portas do paço da casa, para o muro da
cidade e para a casa em que eu houver de entrar. E o rei mas deu,
segundo a boa mão de Deus sobre mim.

Então fui aos governadores dalém do rio, e dei-lhes as cartas do
rei; e o rei tinha enviado comigo capitães do exército e cavaleiros.
O que ouvindo Sambalate, o horonita, e Tobias, o servo
amonita, lhes desagradou extremamente que alguém viesse a procurar o
bem dos filhos de Israel. E cheguei a Jerusalém, e estive ali
três dias. E de noite me levantei, eu e poucos homens comigo,
e não declarei a ninguém o que o meu Deus me pôs no coração para
fazer em Jerusalém; e não havia comigo animal algum, senão aquele em
que estava montado. E de noite saí pela porta do vale, e para
o lado da fonte do dragão, e para a porta do monturo, e contemplei
os muros de Jerusalém, que estavam fendidos, e as suas portas, que
tinham sido consumidas pelo fogo. E passei à porta da fonte,
e ao tanque do rei; e não havia lugar por onde pudesse passar o
animal em que estava montado. Ainda, de noite subi pelo
ribeiro e contemplei o muro; e, virando entrei pela porta do vale;
assim voltei. E não souberam os magistrados aonde eu fora nem
o que eu fazia; porque ainda nem aos judeus, nem aos sacerdotes, nem
aos nobres, nem aos magistrados, nem aos mais que faziam a obra, até
então tinha declarado coisa alguma. Então lhes disse: Bem
vedes vós a miséria em que estamos, que Jerusalém está assolada, e
que as suas portas têm sido queimadas a fogo; vinde, pois, e
reedifiquemos o muro de Jerusalém, e não sejamos mais um
opróbrio\footnote{Abjeção extrema. Ignomínia, desonra. Afronta
infamante; injúria.}. Então lhes declarei como a mão do meu
Deus me fora favorável, como também as palavras do rei, que ele me
tinha dito. Então disseram: Levantemo-nos, e edifiquemos. E
esforçaram as suas mãos para o bem. O que ouvindo Sambalate,
o horonita, e Tobias, o servo amonita, e Gesém, o árabe, zombaram de
nós, e desprezaram-nos, e disseram: Que é isto que fazeis? Quereis
rebelar-vos contra o rei? Então lhes respondi, e disse: O
Deus dos céus é o que nos fará prosperar; e nós, seus servos, nos
levantaremos e edificaremos; mas vós não tendes parte, nem justiça,
nem memória em Jerusalém.

\medskip

\lettrine{3} E levantou-se Eliasibe, o sumo sacerdote, com os
seus irmãos, os sacerdotes, e reedificaram a porta das ovelhas, a
qual consagraram; e levantaram as suas portas, e até à torre de Meá
consagraram, e até à torre de Hananel. E junto a ele edificaram
os homens de Jericó; também ao seu lado edificou Zacur, filho de
Imri. E a porta do peixe edificaram os filhos de Hassenaá; a
qual emadeiraram, e levantaram as suas portas com as suas fechaduras
e os seus ferrolhos. E ao seu lado reparou Meremote, filho de
Urias, o filho de Coz; e ao seu lado reparou Mesulão, filho de
Berequias, o filho de Mesezabeel; e ao seu lado reparou Zadoque,
filho de Baana. E ao seu lado repararam os tecoítas; porém os
seus nobres não submeteram a cerviz ao serviço de seu senhor. E
a porta velha repararam-na Joiada, filho de Paséia, e Mesulão, filho
de Besodias; estes a emadeiraram, e levantaram as suas portas com as
suas fechaduras e os seus ferrolhos. E ao seu lado repararam
Melatias, o gibeonita, e Jadom, meronotita, homens de Gibeom e
Mizpá, que pertenciam ao domínio do governador dalém do rio. Ao
seu lado reparou Uziel, filho de Haraías, um dos ourives; e ao seu
lado reparou Hananias, filho de um dos boticários\footnote{Dono de
botica. Preparador e vendedor de medicamentos na botica;
farmacêutico. RA: perfumista.}; e fortificaram a Jerusalém até ao
muro largo. E ao seu lado reparou Refaías, filho de Hur, líder
da metade de Jerusalém. E ao seu lado reparou Jedaías, filho
de Harumafe, e defronte de sua casa e ao seu lado reparou Hatus,
filho de Hasabnéias. A outra porção reparou Malquias, filho
de Harim, e Hasube, filho de Paate-Moabe; como também a torre dos
fornos. E ao seu lado reparou Salum, filho de Haloés, líder
da outra meia parte de Jerusalém, ele e suas filhas. A porta
do vale reparou-a Hanum e os moradores de Zanoa; estes a edificaram,
e lhe levantaram as portas com as suas fechaduras e os seus
ferrolhos, como também mil côvados do muro, até a porta do monturo.
E a porta do monturo reparou-a Malquias, filho de Recabe,
líder do distrito de Bete-Haquerem; este a edificou, e lhe levantou
as portas com as suas fechaduras e os seus ferrolhos. E a
porta da fonte reparou-a Salum, filho de Col-Hosé, líder do distrito
de Mizpá; este a edificou, e a cobriu, e lhe levantou as portas com
as suas fechaduras e os seus ferrolhos, como também o muro do tanque
de Hasselá, ao pé do jardim do rei, e até aos degraus que descem da
cidade de Davi. Depois dele edificou Neemias, filho de
Azbuque, líder da metade de Bete-Zur, até defronte dos sepulcros de
Davi, até ao tanque artificial e até à casa dos valentes.
Depois dele repararam os levitas, Reum, filho de Bani; ao seu
lado reparou Hasabias, líder da metade de Queila, no seu distrito.
Depois dele repararam seus irmãos, Bavai, filho de Henadade,
líder da outra meia parte de Queila. Ao seu lado reparou
Ezer, filho de Jesuá, líder de Mizpá, outra porção, defronte da
subida à casa das armas, à esquina. Depois dele reparou com
grande ardor Baruque, filho de Zabai, outra medida, desde a esquina
até à porta da casa de Eliasibe, o sumo sacerdote. Depois
dele reparou Meremote, filho de Urias, o filho de Coz, outra porção,
desde a porta da casa de Eliasibe, até à extremidade da casa de
Eliasibe. E depois dele repararam os sacerdotes que habitavam
na campina. Depois reparou Benjamim e Hasube, defronte da sua
casa; depois dele reparou Azarias, filho de Maaséias, o filho de
Ananias, junto à sua casa. Depois dele reparou Binui, filho
de Henadade, outra porção, desde a casa de Azarias até à esquina, e
até ao canto. Palal, filho de Uzai, reparou defronte da
esquina, e a torre que sai da casa real superior, que está junto ao
pátio da prisão; depois dele Pedaías, filho de Parós. E os
servidores do templo que habitavam em Ofel, até defronte da porta
das águas, para o oriente, e até à torre alta. Depois
repararam os tecoítas outra porção, defronte da torre grande e alta,
e até ao muro de Ofel. Desde acima da porta dos cavalos
repararam os sacerdotes, cada um defronte da sua casa. Depois
deles reparou Zadoque, filho de Imer, defronte da sua casa; e depois
dele reparou Semaías, filho de Secanias, guarda da porta oriental.
Depois dele reparou Hananias, filho de Selemias, e Hanum,
filho de Zalafe, o sexto, outra porção; depois dele reparou Mesulão,
filho de Berequias, defronte da sua câmara. Depois dele
reparou Malquias, filho de um ourives, até à casa dos servidores do
templo e mercadores, defronte da porta de Mifcade, e até à câmara do
canto. E entre a câmara do canto e a porta das ovelhas,
repararam os ourives e os mercadores.

\medskip

\lettrine{4} E sucedeu que, ouvindo Sambalate que edificávamos
o muro, ardeu em ira, e se indignou muito; e escarneceu dos judeus.
E falou na presença de seus irmãos, e do exército de Samaria, e
disse: Que fazem estes fracos judeus? Permitir-se-lhes-á isto?
Sacrificarão? Acabá-lo-ão num só dia? Vivificarão dos montões do pó
as pedras que foram queimadas? E estava com ele Tobias, o
amonita, e disse: Ainda que edifiquem, contudo, vindo uma raposa,
derrubará facilmente o seu muro de pedra. Ouve, ó nosso Deus,
que somos tão desprezados, e torna o seu opróbrio sobre a sua
cabeça, e dá-los por presa, na terra do cativeiro. E não cubras
a sua iniqüidade, e não se risque de diante de ti o seu pecado, pois
que te irritaram na presença dos edificadores. Porém edificamos
o muro, e todo o muro se fechou até sua metade; porque o coração do
povo se inclinava a trabalhar.

E sucedeu que, ouvindo Sambalate e Tobias, e os árabes, os
amonitas, e os asdoditas, que tanto ia crescendo a reparação dos
muros de Jerusalém, que já as roturas se começavam a tapar,
iraram-se sobremodo, e ligaram-se entre si todos, para virem
guerrear contra Jerusalém, e para os desviarem do seu intento.
Porém nós oramos ao nosso Deus e pusemos uma guarda contra eles,
de dia e de noite, por causa deles. Então disse Judá: Já
desfaleceram as forças dos carregadores, e o pó é muito, e nós não
poderemos edificar o muro. Disseram, porém, os nossos
inimigos: Nada saberão disto, nem verão, até que entremos no meio
deles, e os matemos; assim faremos cessar a obra. E sucedeu
que, vindo os judeus que habitavam entre eles, dez vezes nos
disseram: De todos os lugares, tornarão contra nós. Então pus
guardas nos lugares baixos por detrás do muro e nos altos; e pus ao
povo pelas suas famílias com as suas espadas, com as suas lanças, e
com os seus arcos. E olhei, e levantei-me, e disse aos
nobres, aos magistrados, e ao restante do povo: Não os temais;
lembrai-vos do grande e terrível Senhor, e pelejai pelos vossos
irmãos, vossos filhos, vossas mulheres e vossas casas. E
sucedeu que, ouvindo os nossos inimigos que já o sabíamos, e que
Deus tinha dissipado o conselho deles, todos voltamos ao muro, cada
um à sua obra.

E sucedeu que, desde aquele dia, metade dos meus servos
trabalhava na obra, e metade deles tinha as lanças, os escudos, os
arcos e as couraças; e os líderes estavam por detrás de toda a casa
de Judá. Os que edificavam o muro, os que traziam as cargas e
os que carregavam, cada um com uma das mãos fazia a obra e na outra
tinha as armas. E os edificadores cada um trazia a sua espada
cingida aos lombos, e edificavam; e o que tocava a trombeta estava
junto comigo. E disse eu aos nobres, aos magistrados e ao
restante do povo: Grande e extensa é a obra, e nós estamos apartados
do muro, longe uns dos outros. No lugar onde ouvirdes o som
da buzina, ali vos ajuntareis conosco; o nosso Deus pelejará por
nós. Assim trabalhávamos na obra; e metade deles tinha as
lanças desde a subida da alva até ao sair das estrelas.
Também naquele tempo disse ao povo: Cada um com o seu servo
fique em Jerusalém, para que à noite nos sirvam de guarda, e de dia
na obra. E nem eu, nem meus irmãos, nem meus servos, nem os
homens da guarda que me seguiam largávamos as nossas vestes; cada um
tinha suas armas e água.

\medskip

\lettrine{5} Foi, porém, grande o clamor do povo e de suas
mulheres, contra os judeus, seus irmãos. Porque havia quem
dizia: Nós, nossos filhos e nossas filhas, somos muitos; então
tomemos trigo, para que comamos e vivamos. Também havia quem
dizia: As nossas terras, as nossas vinhas e as nossas casas
empenhamos, para tomarmos trigo nesta fome. Também havia quem
dizia: Tomamos emprestado dinheiro até para o tributo do rei, sobre
as nossas terras e as nossas vinhas. Agora, pois, a nossa carne
é como a carne de nossos irmãos, e nossos filhos como seus filhos; e
eis que sujeitamos nossos filhos e nossas filhas para serem servos;
e até algumas de nossas filhas são tão sujeitas, que já não estão no
poder de nossas mãos; e outros têm as nossas terras e as nossas
vinhas.

Ouvindo eu, pois, o seu clamor, e estas palavras, muito me
indignei. E considerei comigo mesmo no meu coração; depois
pelejei com os nobres e com os magistrados, e disse-lhes: Sois
usurários cada um para com seu irmão. E convoquei contra eles uma
grande assembléia. E disse-lhes: Nós resgatamos os judeus,
nossos irmãos, que foram vendidos às nações, segundo nossas posses;
e vós outra vez venderíeis a vossos irmãos, ou vender-se-iam a nós?
Então se calaram, e não acharam que responder. Disse mais: Não é
bom o que fazeis; porventura não andaríeis no temor do nosso Deus,
por causa do opróbrio das nações, os nossos inimigos? Também
eu, meus irmãos e meus servos, a juros lhes temos emprestado
dinheiro e trigo. Deixemos este ganho. Restituí-lhes hoje,
vos peço, as suas terras, as suas vinhas, os seus olivais e as suas
casas; como também a centésima parte do dinheiro, do trigo, do mosto
e do azeite, que vós exigis deles. Então disseram:
Restituir-lhes-emos, e nada procuraremos deles; faremos assim como
dizes. Então chamei os sacerdotes, e os fiz jurar que fariam
conforme a esta palavra. Também sacudi as minhas vestes, e
disse: Assim sacuda Deus todo o homem da sua casa e do seu trabalho
que não confirmar esta palavra, e assim seja sacudido e vazio. E
toda a congregação disse: Amém! E louvaram ao Senhor; e o povo fez
conforme a esta palavra.

Também desde o dia em que me mandou que eu fosse seu governador
na terra de Judá, desde o ano vinte, até ao ano trinta e dois do rei
Artaxerxes, doze anos, nem eu nem meus irmãos comemos o pão do
governador. Mas os primeiros governadores, que foram antes de
mim, oprimiram o povo, e tomaram-lhe pão e vinho e, além disso,
quarenta siclos de prata, como também os seus servos dominavam sobre
o povo; porém eu assim não fiz, por causa do temor de Deus.
Como também na obra deste muro fiz reparação, e terra nenhuma
compramos; e todos os meus servos se ajuntaram ali à obra.
Também dos judeus e dos magistrados, cento e cinqüenta
homens, e os que vinham a nós dentre as nações que estão ao redor de
nós, se punham a minha mesa. E o que se preparava para cada
dia era um boi e seis ovelhas escolhidas; também aves se me
preparavam e, de dez em dez dias, muito vinho de todas as espécies;
e nem por isso exigi o pão do governador, porquanto a servidão deste
povo era grande. Lembra-te de mim para bem, ó meu Deus, e de
tudo quanto fiz a este povo.

\medskip

\lettrine{6} Sucedeu que, ouvindo Sambalate, Tobias, Gesem, o
árabe, e o resto dos nossos inimigos, que eu tinha edificado o muro,
e que nele já não havia brecha alguma, ainda que até este tempo não
tinha posto as portas nos portais, Sambalate e Gesem mandaram
dizer-me: Vem, e congreguemo-nos juntamente nas aldeias, no vale de
Ono. Porém intentavam fazer-me mal. E enviei-lhes mensageiros a
dizer: Faço uma grande obra, de modo que não poderei descer; por que
cessaria esta obra, enquanto eu a deixasse, e fosse ter convosco?
E do mesmo modo enviaram a mim quatro vezes; e da mesma forma
lhes respondi. Então Sambalate ainda pela quinta vez me enviou
seu servo com uma carta aberta na sua mão; e na qual estava
escrito: Entre os gentios se ouviu, e Gesem diz: Tu e os judeus
intentais rebelar-vos, então edificas o muro; e tu te farás rei
deles segundo estas palavras; e que puseste profetas, para
pregarem de ti em Jerusalém, dizendo: Este é rei em Judá; de modo
que o rei o ouvirá, segundo estas palavras; vem, pois, agora, e
consultemos juntamente. Porém eu mandei dizer-lhe: De tudo o que
dizes coisa nenhuma sucedeu; mas tu, do teu coração, o inventas.
Porque todos eles procuravam atemorizar-nos, dizendo: As suas
mãos largarão a obra, e não se efetuará. Agora, pois, ó Deus,
fortalece as minhas mãos.

E, entrando eu em casa de Semaías, filho de Delaías, o filho de
Meetabel (que estava encerrado\footnote{KJ: shut up - fechar bem;
confirmar, aprisionar; tapar a boca de.}), disse ele: Vamos
juntamente à casa de Deus, ao meio do templo, e fechemos as portas
do templo; porque virão matar-te; sim, de noite virão matar-te.
Porém eu disse: Um homem como eu fugiria? E quem há, como eu,
que entre no templo para que viva? De maneira nenhuma entrarei.
E percebi que não era Deus quem o enviara; mas esta profecia
falou contra mim, porquanto Tobias e Sambalate o subornaram.
Para isto o subornaram, para me atemorizar, e para que assim
fizesse, e pecasse, para que tivessem alguma causa para me
infamarem, e assim me vituperarem. Lembra-te, meu Deus, de
Tobias e de Sambalate, conforme a estas suas obras, e também da
profetisa Noadia, e dos mais profetas que procuraram atemorizar-me.

Acabou-se, pois, o muro aos vinte e cinco do mês de Elul; em
cinqüenta e dois dias. E sucedeu que, ouvindo-o todos os
nossos inimigos, todos os povos que havia em redor de nós temeram, e
abateram-se muito a seus próprios olhos; porque reconheceram que o
nosso Deus fizera esta obra. Também naqueles dias alguns
nobres de Judá escreveram muitas cartas que iam para Tobias; e as
cartas de Tobias vinham para eles. Porque muitos em Judá lhe
eram ajuramentados, porque era genro de Secanias, filho de Ará; e
seu filho Joanã se casara com a filha de Mesulão, filho de
Berequias. Também as suas boas ações contavam perante mim, e
as minhas palavras transmitiam a ele; portanto Tobias escrevia
cartas para me atemorizar.

\medskip

\lettrine{7} Sucedeu que, depois que o muro foi edificado, eu
levantei as portas; e foram estabelecidos os porteiros, os cantores
e os levitas. Eu nomeei a Hanani, meu irmão, e a Hananias, líder
da fortaleza, em Jerusalém; porque ele era homem fiel e temente a
Deus, mais do que muitos. E disse-lhes: Não se abram as portas
de Jerusalém até que o sol aqueça, e enquanto os que assistirem ali
permanecerem, fechem as portas, e vós trancai-as; e ponham-se
guardas dos moradores de Jerusalém, cada um na sua guarda, e cada um
diante da sua casa. E era a cidade larga de espaço, e grande,
porém pouco povo havia dentro dela; e ainda as casas não estavam
edificadas.

Então o meu Deus me pôs no coração que ajuntasse os nobres, os
magistrados e o povo, para registrar as genealogias; e achei o livro
da genealogia dos que subiram primeiro e nele estava escrito o
seguinte: Estes são os filhos da província, que subiram do
cativeiro dos exilados, que transportara Nabucodonosor, rei de
Babilônia; e voltaram para Jerusalém e para Judá, cada um para a sua
cidade. Os quais vieram com Zorobabel, Jesuá, Neemias, Azarias,
Raamias, Naamani, Mordecai, Bilsã, Misperete, Bigvai, Neum, e Baana;
este é o número dos homens do povo de Israel. Foram os filhos de
Parós, dois mil, cento e setenta e dois. Os filhos de Sefatias,
trezentos e setenta e dois. Os filhos de Ará, seiscentos e
cinqüenta e dois. Os filhos de Paate-Moabe, dos filhos de
Jesuá e de Joabe, dois mil, oitocentos e dezoito. Os filhos
de Elão, mil, duzentos e cinqüenta e quatro. Os filhos de
Zatu, oitocentos e quarenta e cinco. Os filhos de Zacai,
setecentos e sessenta. Os filhos de Binui, seiscentos e
quarenta e oito. Os filhos de Bebai, seiscentos e vinte e
oito. Os filhos de Azgade, dois mil, trezentos e vinte e
dois. Os filhos de Adonicão, seiscentos e sessenta e sete.
Os filhos de Bigvai, dois mil e sessenta e sete. Os
filhos de Adim, seiscentos e cinqüenta e cinco. Os filhos de
Ater, de Ezequias, noventa e oito. Os filhos de Hassum,
trezentos e vinte e oito. Os filhos de Bezai, trezentos e
vinte e quatro. Os filhos de Harife, cento e doze. Os
filhos de Gibeom, noventa e cinco. Os homens de Belém e de
Netofa, cento e oitenta e oito. Os homens de Anatote, cento e
vinte e oito. Os homens de Bete-Azmavete, quarenta e dois.
Os homens de Quiriate-Jearim, Quefira e Beerote, setecentos e
quarenta e três. Os homens de Ramá e Geba, seiscentos e vinte
e um. Os homens de Micmás, cento e vinte e dois. Os
homens de Betel e Ai, cento e vinte e três. Os homens do
outro Nebo, cinqüenta e dois. Os filhos do outro Elão, mil,
duzentos e cinqüenta e quatro. Os filhos de Harim, trezentos
e vinte. Os filhos de Jericó, trezentos e quarenta e cinco.
Os filhos de Lode, Hadide e Ono, setecentos e vinte e um.
Os filhos de Senaá, três mil, novecentos e trinta. Os
sacerdotes: Os filhos de Jedaías, da casa de Jesuá, novecentos e
setenta e três. Os filhos de Imer, mil e cinqüenta e dois.
Os filhos de Pasur, mil, duzentos e quarenta e sete.
Os filhos de Harim, mil e dezessete. Os levitas: Os
filhos de Jesuá, de Cadmiel, dos filhos de Hodeva, setenta e quatro.
Os cantores: Os filhos de Asafe, cento e quarenta e oito.
Os porteiros: Os filhos de Salum, os filhos de Ater, os
filhos de Talmom, os filhos de Acube, os filhos de Hatita, os filhos
de Sobai, cento e trinta e oito. Os servidores do templo: Os
filhos de Zia, os filhos de Hasufa, os filhos de Tabaote, os
filhos de Queros, os filhos de Sia, os filhos de Padom, os
filhos de Lebana, os filhos de Hagaba, os filhos de Salmai,
os filhos de Hanã, os filhos de Gidel, os filhos de Gaar,
os filhos de Reaías, os filhos de Rezim, os filhos de Necoda,
os filhos de Gazão, os filhos de Uzá, os filhos de Paseá,
os filhos de Besai, os filhos de Meunim, os filhos de
Nefussim, os filhos de Bacbuque, os filhos de Hacufa, os
filhos de Harur, os filhos de Bazlite, os filhos de Meída, os
filhos de Harsa, os filhos de Barcos, os filhos de Sísera, os
filhos de Tamá, os filhos de Nezia, os filhos de Hatifa.
Os filhos dos servos de Salomão, os filhos de Sotai, os
filhos de Soferete, os filhos de Perida, os filhos de Jaalá,
os filhos de Darcom, os filhos de Gidel, os filhos de
Sefatias, os filhos de Hatil, os filhos de Poquerete-Hazebaim, os
filhos de Amom. Todos os servidores do templo e os filhos dos
servos de Salomão, trezentos e noventa e dois. Também estes
subiram de Tel-Melá, e Tel-Harsa, Querube, Adom, Imer; porém não
puderam provar que a casa de seus pais e a sua linhagem, eram de
Israel. Os filhos de Dalaías, os filhos de Tobias, os filhos
de Necoda, seiscentos e quarenta e dois. E dos sacerdotes: os
filhos de Hobaías, os filhos de Coz, os filhos de Barzilai, que
tomara uma mulher das filhas de Barzilai, o gileadita, e que foi
chamado do seu nome. Estes buscaram o seu registro nos livros
genealógicos, porém não se achou; então, como imundos, foram
excluídos do sacerdócio. E o governador lhes disse que não
comessem das coisas sagradas, até que se apresentasse o sacerdote
com Urim e Tumim. Toda esta congregação junta foi de quarenta
e dois mil, trezentos e sessenta, afora os seus servos e as
suas servas, que foram sete mil, trezentos e trinta e sete; e tinham
duzentos e quarenta e cinco cantores e cantoras. Os seus
cavalos, setecentos e trinta e seis; os seus mulos, duzentos e
quarenta e cinco. Camelos, quatrocentos e trinta e cinco;
jumentos, seis mil, setecentos e vinte. E uma parte dos
chefes dos pais contribuíram para a obra. O governador deu para o
tesouro, em ouro, mil dracmas, cinqüenta bacias, e quinhentas e
trinta vestes sacerdotais. E alguns mais dos chefes dos pais
contribuíram para o tesouro da obra, em ouro, vinte mil dracmas, e
em prata, duas mil e duzentas libras. E o que deu o restante
do povo foi, em ouro, vinte mil dracmas, e em prata, duas mil
libras; e sessenta e sete vestes sacerdotais. E habitaram os
sacerdotes, os levitas, os porteiros, os cantores, alguns do povo,
os servidores do templo, e todo o Israel nas suas cidades.

\medskip

\lettrine{8} E chegado o sétimo mês, e estando os filhos de
Israel nas suas cidades, todo o povo se ajuntou como um só homem, na
praça, diante da porta das águas; e disseram a Esdras, o escriba,
que trouxesse o livro da lei de Moisés, que o Senhor tinha ordenado
a Israel. E Esdras, o sacerdote, trouxe a lei perante a
congregação, tanto de homens como de mulheres, e todos os que podiam
ouvir com entendimento, no primeiro dia do sétimo mês. E leu no
livro diante da praça, que está diante da porta das águas, desde a
alva até ao meio dia, perante homens e mulheres, e os que podiam
entender; e os ouvidos de todo o povo estavam atentos ao livro da
lei. E Esdras, o escriba, estava sobre um púlpito de madeira,
que fizeram para aquele fim; e estava em pé junto a ele, à sua mão
direita, Matitias, Sema, Anaías, Urias, Hilquias e Maaséias; e à sua
mão esquerda, Pedaías, Misael, Melquias, Hasum, Hasbadana, Zacarias
e Mesulão. E Esdras abriu o livro perante à vista de todo o
povo; porque estava acima de todo o povo; e, abrindo-o ele, todo o
povo se pôs em pé. E Esdras louvou ao Senhor, o grande Deus; e
todo o povo respondeu: Amém, Amém! levantando as suas mãos; e
inclinaram suas cabeças, e adoraram ao Senhor, com os rostos em
terra. E Jesuá, Bani, Serebias, Jamim, Acube, Sabetai, Hodias,
Maaséias, Quelita, Azarias, Jozabade, Hanã, Pelaías, e os levitas
ensinavam o povo na lei; e o povo estava no seu lugar. E leram
no livro, na lei de Deus; e declarando, e explicando o sentido,
faziam que, lendo, se entendesse.

E Neemias, que era o governador, e o sacerdote Esdras, o escriba,
e os levitas que ensinavam ao povo, disseram a todo o povo: Este dia
é consagrado ao Senhor vosso Deus, então não vos lamenteis, nem
choreis. Porque todo o povo chorava, ouvindo as palavras da lei.
Disse-lhes mais: Ide, comei as gorduras, e bebei as doçuras,
e enviai porções aos que não têm nada preparado para si; porque este
dia é consagrado ao nosso Senhor; portanto não vos entristeçais;
porque a alegria do Senhor é a vossa força. E os levitas
fizeram calar a todo o povo, dizendo: Calai-vos; porque este dia é
santo; por isso não vos entristeçais. Então todo o povo se
foi a comer, a beber, a enviar porções e a fazer grande regozijo;
porque entenderam as palavras que lhes fizeram saber.

E no dia seguinte ajuntaram-se os chefes dos pais de todo o povo,
os sacerdotes e os levitas, a Esdras, o escriba; e isto para
atentarem nas palavras da lei. E acharam escrito na lei que o
Senhor ordenara, pelo ministério de Moisés, que os filhos de Israel
habitassem em cabanas, na solenidade da festa, no sétimo mês.
Assim publicaram, e fizeram passar pregão por todas as suas
cidades, e em Jerusalém, dizendo: Saí ao monte, e trazei ramos de
oliveiras, e ramos de zambujeiros\footnote{Ou azamjujeiro ou
azambujo: espécie de oliveira brava, de madeira rija.}, e ramos de
murtas\footnote{Ou mirto: Gênero de arbustos tropicais e
subtropicais, de folhas sempre-verdes, ovadas ou lanceoadas, e
flores com numerosos óvulos. Qualquer espécie desse gênero, como, p.
ex., a Myrtus communis, de origem mediterrânea, cultivada como
ornamental, e que se caracteriza pelas flores pequeninas e
compactas, e fruto bacáceo ovóide, purpúreo tirante ao negro; é tb.
conhecida como murta. Qualquer espécime desse gênero, ou a flor
dele.}, e ramos de palmeiras, e ramos de árvores espessas, para
fazer cabanas, como está escrito. Saiu, pois, o povo, e os
trouxeram, e fizeram para si cabanas, cada um no seu terraço, nos
seus pátios, e nos átrios da casa de Deus, na praça da porta das
águas, e na praça da porta de Efraim. E toda a congregação
dos que voltaram do cativeiro fizeram cabanas, e habitaram nas
cabanas, porque nunca fizeram assim os filhos de Israel, desde os
dias de Josué, filho de Num, até àquele dia; e houve mui grande
alegria. E, de dia em dia, Esdras leu no livro da lei de
Deus, desde o primeiro dia até ao derradeiro; e celebraram a
solenidade da festa sete dias, e no oitavo dia, houve uma assembléia
solene, segundo o rito.

\medskip

\lettrine{9} E, no dia vinte e quatro deste mês, ajuntaram-se
os filhos de Israel com jejum e com sacos, e traziam terra sobre si.
E a descendência de Israel se apartou de todos os estrangeiros,
e puseram-se em pé, e fizeram confissão pelos seus pecados e pelas
iniqüidades de seus pais. E, levantando-se no seu lugar, leram
no livro da lei do Senhor seu Deus uma quarta parte do dia; e na
outra quarta parte fizeram confissão, e adoraram ao Senhor seu Deus.

E Jesuá, Bani, Cadmiel, Sebanias, Buni, Serebias, Bani e Quenani
se puseram em pé no lugar alto dos levitas, e clamaram em alta voz
ao Senhor seu Deus. E os levitas, Jesuá, Cadmiel, Bani,
Hasabnéias, Serebias, Hodias, Sebanias e Petaías, disseram:
Levantai-vos, bendizei ao Senhor vosso Deus de eternidade em
eternidade; e bendigam o teu glorioso nome, que está exaltado sobre
toda a bênção e louvor. Só tu és Senhor; tu fizeste o céu, o céu
dos céus, e todo o seu exército, a terra e tudo quanto nela há, os
mares e tudo quanto neles há, e tu os guardas com vida a todos; e o
exército dos céus te adora. Tu és o Senhor, o Deus, que elegeste
a Abrão, e o tiraste de Ur dos caldeus, e lhe puseste por nome
Abraão. E achaste o seu coração fiel perante ti, e fizeste com
ele a aliança, de que darias à sua descendência a terra dos
cananeus, dos heteus, dos amorreus, dos perizeus, dos jebuseus e dos
girgaseus; e confirmaste as tuas palavras, porquanto és justo. E
viste a aflição de nossos pais no Egito, e ouviste o seu clamor
junto ao Mar Vermelho. E mostraste sinais e prodígios a
Faraó, e a todos os seus servos, e a todo o povo da sua terra,
porque soubeste que soberbamente os trataram; e assim adquiriste
para ti nome, como hoje se vê. E o mar fendeste perante eles,
e passaram pelo meio do mar, em seco; e lançaste os seus
perseguidores nas profundezas, como uma pedra nas águas violentas.
E guiaste-os de dia por uma coluna de nuvem, e de noite por
uma coluna de fogo, para lhes iluminar o caminho por onde haviam de
ir. E sobre o monte Sinai desceste, e dos céus falaste com
eles, e deste-lhes juízos retos e leis verdadeiras, estatutos e
mandamentos bons. E o teu santo sábado lhes fizeste conhecer;
e preceitos, estatutos e lei lhes mandaste pelo ministério de
Moisés, teu servo. E pão dos céus lhes deste na sua fome, e
água da penha lhes produziste na sua sede; e lhes disseste que
entrassem para possuírem a terra pela qual alçaste a tua mão, que
lhes havias de dar. Porém eles e nossos pais se houveram
soberbamente, e endureceram a sua cerviz, e não deram ouvidos aos
teus mandamentos. E recusaram ouvir-te, e não se lembraram
das tuas maravilhas, que lhes fizeste, e endureceram a sua cerviz e,
na sua rebelião, levantaram um capitão, a fim de voltarem para a sua
servidão; porém tu, ó Deus perdoador, clemente e misericordioso,
tardio em irar-te, e grande em beneficência, tu não os desamparaste.
Ainda mesmo quando eles fizeram para si um bezerro de
fundição, e disseram: Este é o teu Deus, que te tirou do Egito; e
cometeram grandes blasfêmias; todavia tu, pela multidão das
tuas misericórdias, não os deixaste no deserto. A coluna de nuvem
nunca se apartou deles de dia, para os guiar pelo caminho, nem a
coluna de fogo de noite, para lhes iluminar; e isto pelo caminho por
onde haviam de ir. E deste o teu bom espírito, para os
ensinar; e o teu maná não retiraste da sua boca; e água lhes deste
na sua sede. De tal modo os sustentaste quarenta anos no
deserto; nada lhes faltou; as suas roupas não se envelheceram, e os
seus pés não se incharam. Também lhes deste reinos e povos, e
os repartiste em porções; e eles possuíram a terra de Siom, a saber,
a terra do rei de Hesbom, e a terra de Ogue, rei de Basã. E
multiplicaste os seus filhos como as estrelas do céu, e trouxeste-os
à terra de que tinhas falado a seus pais que nela entrariam para a
possuírem. Assim os filhos entraram e possuíram aquela terra;
e abateste perante eles os moradores da terra, os cananeus, e lhos
entregaste na mão, como também os reis e os povos da terra, para
fazerem deles conforme a sua vontade. E tomaram cidades
fortificadas e terra fértil, e possuíram casas cheias de toda a
fartura, cisternas cavadas, vinhas e olivais, e árvores frutíferas,
em abundância; e comeram e se fartaram e engordaram e viveram em
delícias, pela tua grande bondade. Porém se obstinaram, e se
rebelaram contra ti, e lançaram a tua lei para trás das suas costas,
e mataram os teus profetas, que protestavam contra eles, para que
voltassem para ti; assim fizeram grandes abominações. Por
isso os entregaste na mão dos seus adversários, que os angustiaram;
mas no tempo de sua angústia, clamando a ti, desde os céus tu
ouviste; e segundo a tua grande misericórdia lhes deste libertadores
que os libertaram da mão de seus adversários. Porém, em tendo
repouso, tornavam a fazer o mal diante de ti; e tu os deixavas na
mão dos seus inimigos, para que dominassem sobre eles; e
convertendo-se eles, e clamando a ti, tu os ouviste desde os céus, e
segundo a tua misericórdia os livraste muitas vezes. E
testificaste contra eles, para que voltassem para a tua lei; porém
eles se houveram soberbamente, e não deram ouvidos aos teus
mandamentos, mas pecaram contra os teus juízos, pelos quais o homem
que os cumprir viverá; viraram o ombro, endureceram a sua cerviz, e
não quiseram ouvir. Porém estendeste a tua benignidade sobre
eles por muitos anos, e testificaste contra eles pelo teu Espírito,
pelo ministério dos teus profetas; porém eles não deram ouvidos; por
isso os entregaste nas mãos dos povos das terras. Mas pela
tua grande misericórdia os não destruíste nem desamparaste, porque
és um Deus clemente e misericordioso. Ago-a, pois, nosso
Deus, o grande, poderoso e terrível Deus, que guardas a aliança e a
beneficência, não tenhas em pouca conta toda a aflição que nos
alcançou a nós, aos nossos reis, aos nossos príncipes, aos nossos
sacerdotes, aos nossos profetas, aos nossos pais e a todo o teu
povo, desde os dias dos reis da Assíria até ao dia de hoje.
Porém tu és justo em tudo quanto tem vindo sobre nós; porque
tu tens agido fielmente, e nós temos agido impiamente. E os
nossos reis, os nossos príncipes, os nossos sacerdotes, e os nossos
pais não guardaram a tua lei, e não deram ouvidos aos teus
mandamentos e aos teus testemunhos, que testificaste contra eles.
Porque eles nem no seu reino, nem na muita abundância de bens
que lhes deste, nem na terra espaçosa e fértil que puseste diante
deles, te serviram, nem se converteram de suas más obras. Eis
que hoje somos servos; e até na terra que deste a nossos pais, para
comerem o seu fruto e o seu bem, eis que somos servos nela. E
ela multiplica os seus produtos para os reis, que puseste sobre nós,
por causa dos nossos pecados; e conforme a sua vontade dominam sobre
os nossos corpos e sobre o nosso gado; e estamos numa grande
angústia. E, todavia fizemos uma firme aliança, e o
escrevemos; e selaram-no os nossos príncipes, os nossos levitas e os
nossos sacerdotes.

\medskip

\lettrine{10} E os que selaram foram: Neemias, o governador,
filho de Hacalias, e Zedequias, Seraías, Azarias, Jeremias,
Pasur, Amarias, Malquias, Hatus, Sebanias, Maluque,
Harim, Meremote, Obadias, Daniel, Ginetom, Baruque,
Mesulão, Abias, Miamim, Maazias, Bilgai, Semaías; estes eram
os sacerdotes. E os levitas: Jesuá, filho de Azanias, Binui, dos
filhos de Henadade, Cadmiel, e seus irmãos: Sebanias, Hodias,
Quelita, Pelaías, Hanã, Mica, Reobe, Hasabias, Zacur,
Serebias, Sebanias, Hodias, Bani e Beninu. Os chefes
do povo: Parós, Paate-Moabe, Elão, Zatu, Bani, Buni, Azgade,
Bebai, Adonias, Bigvai, Adim, Ater, Ezequias, Azur,
Hodias, Hasum, Bezai, Harife, Anatote, Nebai,
Magpias, Mesulão, Hezir, Mesezabeel, Zadoque, Jadua,
Pelatias, Hanã, Anaías, Oséias, Hananias, Hassube,
Haloés, Pilha, Sobeque, Reum, Hasabná, Maaséias,
e Aías, Hanã, Anã, Maluque, Harim e Baaná. E o
restante do povo, os sacerdotes, os levitas, os porteiros, os
cantores, os servidores do templo, todos os que se tinham separado
dos povos das terras para a lei de Deus, suas mulheres, seus filhos
e suas filhas, todos os que tinham conhecimento e entendimento,
firmemente aderiram a seus irmãos os mais nobres dentre eles,
e convieram num anátema e num juramento, de que andariam na lei de
Deus, que foi dada pelo ministério de Moisés, servo de Deus; e de
que guardariam e cumpririam todos os mandamentos do SENHOR nosso
Senhor, e os seus juízos e os seus estatutos; e que não
daríamos as nossas filhas aos povos da terra, nem tomaríamos as
filhas deles para os nossos filhos. E que, trazendo os povos
da terra no dia de sábado qualquer mercadoria, e qualquer grão para
venderem, nada compraríamos deles no sábado, nem no dia santificado;
e no sétimo ano deixaríamos descansar a terra, e perdoaríamos toda e
qualquer cobrança.

Também sobre nós pusemos preceitos, impondo-nos cada ano a terça
parte de um siclo, para o ministério da casa do nosso Deus;
para os pães da proposição, para a contínua oferta de
alimentos, e para o contínuo holocausto dos sábados, das luas novas,
para as festas solenes, para as coisas sagradas, e para os
sacrifícios pelo pecado, para expiação de Israel, e para toda a obra
da casa do nosso Deus. Também lançamos sortes entre os
sacerdotes, levitas, e o povo, acerca da oferta da lenha que se
havia de trazer à casa do nosso Deus, segundo as casas de nossos
pais, a tempos determinados, de ano em ano, para se queimar sobre o
altar do Senhor nosso Deus, como está escrito na lei. Que
também traríamos as primícias da nossa terra, e as primícias de
todos os frutos de todas as árvores, de ano em ano, à casa do
Senhor. E os primogênitos dos nossos filhos, e os do nosso
gado, como está escrito na lei; e que os primogênitos do nosso gado
e das nossas ovelhas traríamos à casa do nosso Deus, aos sacerdotes,
que ministram na casa do nosso Deus. E que as primícias da
nossa massa, as nossas ofertas alçadas, o fruto de toda a árvore, o
mosto e o azeite, traríamos aos sacerdotes, às câmaras da casa do
nosso Deus; e os dízimos da nossa terra aos levitas; e que os
levitas receberiam os dízimos em todas as cidades, da nossa lavoura.
E que o sacerdote, filho de Arão, estaria com os levitas
quando estes recebessem os dízimos, e que os levitas trariam os
dízimos dos dízimos à casa do nosso Deus, às câmaras da casa do
tesouro. Porque àquelas câmaras os filhos de Israel e os
filhos de Levi devem trazer ofertas alçadas do grão, do mosto e do
azeite; porquanto ali estão os vasos do santuário, como também os
sacerdotes que ministram, os porteiros e os cantores; e que assim
não desampararíamos a casa do nosso Deus.

\medskip

\lettrine{11} E os líderes do povo habitaram em Jerusalém,
porém o restante do povo lançou sortes, para tirar um de dez, que
habitasse na santa cidade de Jerusalém, e as nove partes nas outras
cidades. E o povo bendisse a todos os homens que voluntariamente
se ofereciam para habitar em Jerusalém. E estes são os chefes da
província, que habitaram em Jerusalém; porém nas cidades de Judá
habitou cada um na sua possessão, nas suas cidades, Israel, os
sacerdotes, os levitas, os servidores do templo, e os filhos dos
servos de Salomão. Habitaram, pois, em Jerusalém alguns dos
filhos de Judá e dos filhos de Benjamim. Dos filhos de Judá, Ataías,
filho de Uzias, filho de Zacarias, filho de Amarias, filho de
Sefatias, filho de Maalaleel, dos filhos de Perez; e Maaséias,
filho de Baruque, filho de Col-Hoze, filho de Hazaías, filho de
Adaías, filho de Joiaribe, filho de Zacarias, filho de Siloni.
Todos os filhos de Perez, que habitaram em Jerusalém, foram
quatrocentos e sessenta e oito homens valentes. E estes são os
filhos de Benjamim: Salu, filho de Mesulão, filho de Joede, filho de
Pedaías, filho de Colaías, filho de Maaséias, filho de Itiel, filho
de Jesaías. E depois dele Gabai e Salai, ao todo novecentos e
vinte e oito. E Joel, filho de Zicri, superintendente sobre
eles; e Judá, filho de Senua, o segundo sobre a cidade. Dos
sacerdotes: Jedaías, filho de Joiaribe, Jaquim, Seraías,
filho de Hilquias, filho de Mesulão, filho de Zadoque, filho de
Meraiote, filho de Aitube, líder da casa de Deus, e seus
irmãos, que faziam a obra na casa, oitocentos e vinte e dois; e
Adaías, filho de Jeroão, filho de Pelalias, filho de Anzi, filho de
Zacarias, filho de Pasur, filho de Malquias, e seus irmãos,
chefes dos pais, duzentos e quarenta e dois; e Amassai, filho de
Azareel, filho de Azai, filho de Mesilemote, filho de Imer, e
os irmãos deles, homens valentes, cento e vinte e oito, e
superintendente sobre eles Zabdiel, filho de Gedolim. E dos
levitas: Semaías, filho de Hassube, filho de Azricão, filho de
Hasabias, filho de Buni; e Sabetai, e Jozabade, dos chefes
dos levitas, presidiam sobre a obra de fora da casa de Deus.
E Matanias, filho de Mica, filho de Zabdi, filho de Asafe, o
chefe, que iniciava as ações de graças na oração, e Bacbuquias, o
segundo de seus irmãos; depois Abda, filho de Samua, filho de Galal,
filho de Jedutum. Todos os levitas na santa cidade, foram
duzentos e oitenta e quatro. E os porteiros, Acube, Talmom,
com seus irmãos, os guardas das portas, cento e setenta e dois.

E o restante de Israel, dos sacerdotes e levitas, habitou em
todas as cidades de Judá, cada um na sua herança. E os
servidores do templo, habitaram em Ofel; e Zia e Gispa presidiam
sobre os servidores do templo. E o superintendente dos
levitas em Jerusalém foi Uzi, filho de Bani, filho de Hasabias,
filho de Matanias, filho de Mica; dos filhos de Asafe, os cantores,
ao serviço da casa de Deus. Porque havia um mandado do rei
acerca deles, e uma certa regra para os cantores, cada qual no seu
dia. E Petaías, filho de Mesezabeel, dos filhos de Zera,
filho de Judá, estava à mão do rei, em todos os negócios do povo.
E quanto às aldeias, com as suas terras, alguns dos filhos de
Judá habitaram em Quiriate-Arba e nos lugares da sua jurisdição, e
em Dibom, e nos lugares da sua jurisdição, e em Jecabzeel e nas suas
aldeias, e em Jesuá, e em Molada, e em Bete-Pelete, e
em Hazar-Sual, e em Berseba e nos lugares da sua jurisdição,
e em Ziclague, em Mecona e nos lugares da sua jurisdição,
e em En-Rimom, em Zora e em Jarmute, em Zanoa, Adulão
e nas suas aldeias, em Laquis e nas suas terras, em Azaca e nos
lugares da sua jurisdição. Acamparam-se desde Berseba até ao vale de
Hinom. E os filhos de Benjamim habitaram desde Geba, em
Micmás, Aia, Betel e nos lugares da sua jurisdição, e em
Anatote, em Nobe, em Ananias, em Hazor, em Ramá, em Gitaim,
em Hadide, em Zeboim, em Nebalate, em Lode e em Ono,
no vale dos artífices, e alguns dos levitas habitaram nas
divisões de Judá e de Benjamim.

\medskip

\lettrine{12} Estes são sacerdotes e levitas que subiram com
Zorobabel, filho de Sealtiel, e com Jesuá: Seraías, Jeremias,
Esdras, Amarias, Maluque, Hatus, Secanias, Reum, Meremote,
Ido, Ginetoi, Abias, Miamim, Maadias, Bilga, Semaías,
Joiaribe, Jedaías, Salu, Amoque, Hilquias, Jedaías; estes foram
os chefes dos sacerdotes e de seus irmãos, nos dias de Jesuá. E
os levitas: Jesuá, Binui, Cadmiel, Serebias, Judá, Matanias; este e
seus irmãos dirigiam os louvores. E Bacbuquias e Uni, seus
irmãos, estavam defronte deles, nas guardas. E Jesuá gerou a
Joiaquim, e Joiaquim gerou a Eliasibe, e Eliasibe gerou a Joiada,
e Joiada gerou a Jônatas, e Jônatas gerou a Jadua. E
nos dias de Joiaquim foram sacerdotes, chefes dos pais: de Seraías,
Meraías; de Jeremias, Hananias; de Esdras, Mesulão; de
Amarias, Joanã; de Maluqui, Jônatas; de Sebanias, José;
de Harim, Adna; de Meraiote, Helcai; de Ido, Zacarias;
de Ginetom, Mesulão. De Abias, Zicri; de Miamim e de Moadias,
Piltai; de Bilga, Samua; de Semaías, Jônatas; e de
Joiaribe, Matenai; de Jedaías, Uzi; de Salai, Calai; de
Amoque, Éber; de Hilquias, Hasabias; de Jedaías, Natanael.
Dos levitas, nos dias de Eliasibe, foram inscritos como
chefes de pais, Joiada, Joanã e Jadua; como também os sacerdotes,
até ao reinado de Dario, o persa. Os filhos de Levi foram
inscritos, como chefes de pais, no livro das crônicas, até aos dias
de Joanã, filho de Eliasibe. Foram, pois, os chefes dos
levitas: Hasabias, Serabias, e Jesuá, filho de Cadmiel; e seus
irmãos estavam defronte deles, para louvarem e darem graças, segundo
o mandado de Davi, homem de Deus; guarda contra guarda.\footnote{RA:
coro contra coro.} Matanias, Bacbuquias, Obadias, Mesulão,
Talmom e Acube, eram porteiros, que faziam a guarda às tesourarias
das portas. Estes viveram nos dias de Jeoiaquim, filho de
Jesuá, o filho de Jozadaque; como também nos dias de Neemias, o
governador, e do sacerdote Esdras, o escriba.

E na dedicação dos muros de Jerusalém buscaram os levitas de
todos os seus lugares, para trazê-los, a fim de fazerem a dedicação
com alegria, com louvores e com canto, saltérios, címbalos e com
harpas. E assim ajuntaram os filhos dos cantores, tanto da
campina dos arredores de Jerusalém, como das aldeias de Netofati;
como também da casa de Gilgal, e dos campos de Geba, e
Azmavete; porque os cantores edificaram para si aldeias nos
arredores de Jerusalém. E purificaram-se os sacerdotes e os
levitas; e logo purificaram o povo, as portas e o muro. Então
fiz subir os príncipes de Judá sobre o muro, e ordenei dois grandes
coros em procissão, um à mão direita sobre o muro do lado da porta
do monturo. E após ele ia Hosaías, e a metade dos príncipes
de Judá. E Azarias, Esdras e Mesulão, Judá, Benjamim,
Semaías e Jeremias. E dos filhos dos sacerdotes, com
trombetas: Zacarias, filho de Jônatas, filho de Semaías, filho de
Matanias, filho de Micaías, filho de Zacur, filho de Asafe. E
seus irmãos, Semaías, e Azareel, Milalai, Gilalai, Maai, Natanael,
Judá e Hanani, com os instrumentos musicais de Davi, homem de Deus;
e Esdras, o escriba, ia adiante deles. Indo assim para a
porta da fonte, defronte deles, subiram as escadas da cidade de
Davi, onde começa a subida do muro, desde cima da casa de Davi, até
à porta das águas, do lado do oriente. E o segundo coro ia em
frente, e eu após ele; e a metade do povo ia sobre o muro, desde a
torre dos fornos, até à muralha larga; e desde a porta de
Efraim, passaram por cima da porta velha, e da porta do peixe, e
pela torre de Hananeel e a torre de Meá, até à porta das ovelhas; e
pararam à porta da prisão. Então ambos os coros pararam na
casa de Deus; como também eu, e a metade dos magistrados comigo.
E os sacerdotes Eliaquim, Maaséias, Miniamim, Micaías,
Elioenai, Zacarias e Hananias, com trombetas. Como também,
Maaséias, Semaías, Eleazar, Uzi, Joanã, Malquias, Elão e Ezer; e
faziam-se ouvir os cantores, juntamente com Jezraías, o seu
superintendente. E ofereceram, no mesmo dia, grandes
sacrifícios e se alegraram; porque Deus os alegrara com grande
alegria; e até as mulheres e os meninos se alegraram, de modo que a
alegria de Jerusalém se ouviu até de longe.

Também no mesmo dia se nomearam homens sobre as câmaras, dos
tesouros, das ofertas alçadas, das primícias, dos dízimos, para
ajuntarem nelas, dos campos das cidades, as partes da lei para os
sacerdotes e para os levitas; porque Judá estava alegre por causa
dos sacerdotes e dos levitas que assistiam ali. E observava
os preceitos do seu Deus, e os da purificação; como também os
cantores e porteiros, conforme ao mandado de Davi e de seu filho
Salomão. Porque já nos dias de Davi e Asafe, desde a
antiguidade, havia chefes dos cantores, e dos cânticos de louvores e
de ação de graças a Deus. Por isso todo o Israel, já nos dias
de Zorobabel e nos dias de Neemias, dava aos cantores e aos
porteiros as porções de cada dia; e santificavam as porções aos
levitas, e os levitas as santificavam aos filhos de Arão.

\medskip

\lettrine{13} Naquele dia leu-se no livro de Moisés, aos
ouvidos do povo; e achou-se escrito nele que os amonitas e os
moabitas não entrassem jamais na congregação de Deus, porquanto
não tinham saído ao encontro dos filhos de Israel com pão e água;
antes contra eles assalariaram a Balaão para os amaldiçoar; porém o
nosso Deus converteu a maldição em bênção. Sucedeu, pois, que,
ouvindo eles esta lei, apartaram de Israel todo o elemento misto.
Ora, antes disto, Eliasibe, sacerdote, que presidia sobre a
câmara da casa do nosso Deus, se tinha aparentado com Tobias; e
fizera-lhe uma câmara grande, onde dantes se depositavam as ofertas
de alimentos, o incenso, os utensílios, os dízimos do grão, do mosto
e do azeite, que se ordenaram para os levitas, cantores e porteiros,
como também a oferta alçada para os sacerdotes. Mas durante tudo
isto não estava eu em Jerusalém, porque no ano trinta e dois de
Artaxerxes, rei de Babilônia, fui ter com o rei; mas após alguns
dias tornei a alcançar licença do rei. E voltando a Jerusalém,
compreendi o mal que Eliasibe fizera para Tobias, fazendo-lhe uma
câmara nos pátios da casa de Deus. O que muito me desagradou; de
sorte que lancei todos os móveis da casa de Tobias fora da câmara.
E, ordenando-o eu, purificaram as câmaras; e tornei a trazer
para ali os utensílios da casa de Deus, com as ofertas de alimentos
e o incenso.

Também entendi que os quinhões dos levitas não se lhes davam, de
maneira que os levitas e os cantores, que faziam a obra, tinham
fugido cada um para a sua terra. Então contendi com os
magistrados, e disse: Por que se desamparou a casa de Deus? Porém eu
os ajuntei, e os restaurei no seu posto. Então todo o Judá
trouxe os dízimos do grão, do mosto e do azeite aos celeiros.
E por tesoureiros pus sobre os celeiros a Selemias, o
sacerdote, e a Zadoque, o escrivão e a Pedaías, dentre os levitas; e
com eles Hanã, filho de Zacur, o filho de Matanias; porque foram
achados fiéis; e se lhes encarregou a eles a distribuição para seus
irmãos. Por isto, Deus meu, lembra-te de mim e não risques as
beneficências que eu fiz à casa de meu Deus e às suas observâncias.

Naqueles dias vi em Judá os que pisavam lagares ao sábado e
traziam feixes que carregavam sobre os jumentos; como também vinho,
uvas e figos, e toda a espécie de cargas, que traziam a Jerusalém no
dia de sábado; e protestei contra eles no dia em que vendiam
mantimentos. Também habitavam em Jerusalém tírios que traziam
peixe e toda a mercadoria, que vendiam no sábado aos filhos de Judá,
e em Jerusalém. E contendi com os nobres de Judá, e lhes
disse: Que mal é este que fazeis, profanando o dia de sábado?
Porventura não fizeram vossos pais assim, e não trouxe o
nosso Deus todo este mal sobre nós e sobre esta cidade? E vós ainda
mais acrescentais o ardor de sua ira sobre Israel, profanando o
sábado. Sucedeu, pois, que, dando já sombra nas portas de
Jerusalém antes do sábado, ordenei que as portas fossem fechadas; e
mandei que não as abrissem até passado o sábado; e pus às portas
alguns de meus servos, para que nenhuma carga entrasse no dia de
sábado. Então os negociantes e os vendedores de toda a
mercadoria passaram a noite fora de Jerusalém, uma ou duas vezes.
Protestei, pois, contra eles, e lhes disse: Por que passais a
noite defronte do muro? Se outra vez o fizerdes, hei de lançar mão
de vós. Daquele tempo em diante não vieram no sábado. Também
disse aos levitas que se purificassem, e viessem guardar as portas,
para santificar o sábado. Nisto também, Deus meu, lembra-te de mim e
perdoa-me segundo a abundância da tua benignidade.

Vi também naqueles dias judeus que tinham casado com mulheres
asdoditas, amonitas e moabitas. E seus filhos falavam meio
asdodita, e não podiam falar judaico, senão segundo a língua de cada
povo. E contendi com eles, e os amaldiçoei e espanquei alguns
deles, e lhes arranquei os cabelos, e os fiz jurar por Deus,
dizendo: Não dareis mais vossas filhas a seus filhos, e não tomareis
mais suas filhas, nem para vossos filhos nem para vós mesmos.
Porventura não pecou nisto Salomão, rei de Israel, não
havendo entre muitas nações rei semelhante a ele, e sendo ele amado
de seu Deus, e pondo-o Deus rei sobre todo o Israel? E contudo as
mulheres estrangeiras o fizeram pecar. E dar-vos-íamos nós
ouvidos, para fazermos todo este grande mal, prevaricando contra o
nosso Deus, casando com mulheres estrangeiras? Também um dos
filhos de Joiada, filho de Eliasibe, o sumo sacerdote, era genro de
Sambalate, o horonita, por isso o afugentei de mim. Lembra-te
deles, Deus meu, pois contaminaram o sacerdócio, como também a
aliança do sacerdócio e dos levitas. Assim os limpei de todo
o estrangeiro, e designei os cargos dos sacerdotes e dos levitas,
cada um na sua obra. Como também para com as ofertas de lenha
em tempos determinados, e para com as primícias. Lembra-te de mim,
Deus meu, para bem.

