\addchap{Números}

\lettrine{1} Falou mais o Senhor a Moisés no deserto de Sinai,
na tenda da congregação, no primeiro dia do segundo mês, no segundo
ano da sua saída da terra do Egito, dizendo: Tomai a soma de
toda a congregação dos filhos de Israel, segundo as suas famílias,
segundo a casa de seus pais, conforme o número dos nomes de todo o
homem, cabeça por cabeça; da idade de vinte anos para cima,
todos os que em Israel podem sair à guerra, a estes contareis
segundo os seus exércitos, tu e Arão. Estará convosco, de cada
tribo, um homem que seja cabeça da casa de seus pais. Estes,
pois, são os nomes dos homens que estarão convosco: De Rúben,
Elizur, filho de Sedeur; de Simeão, Selumiel, filho de
Zurisadai; de Judá, Naasson, filho de Aminadabe; de Issacar,
Natanael, filho de Zuar; de Zebulom, Eliabe, filho de Helom;
dos filhos de José: de Efraim, Elisama, filho de Amiúde; de
Manassés, Gamaliel, filho de Pedazur; de Benjamim, Abidã,
filho de Gideoni; de Dã, Aieser, filho de Amisadai; de
Aser, Pagiel, filho de Ocrã; de Gade, Eliasafe, filho de
Deuel; de Naftali, Aira, filho de Enã. Estes foram os
chamados da congregação, os príncipes das tribos de seus pais, os
cabeças dos milhares de Israel.

Então tomaram Moisés e Arão a estes homens, que foram declarados
pelos seus nomes, e reuniram toda a congregação no primeiro
dia do mês segundo, e declararam a sua descendência segundo as suas
famílias, segundo a casa de seus pais, pelo número dos nomes dos de
vinte anos para cima, cabeça por cabeça; como o Senhor
ordenara a Moisés, assim os contou no deserto de Sinai.
Foram, pois, os filhos de Rúben, o primogênito de Israel, as
suas gerações, pelas suas famílias, segundo a casa de seus pais,
pelo número dos nomes, cabeça por cabeça, todo o homem de vinte anos
para cima, todos os que podiam sair à guerra, foram contados
deles, da tribo de Rúben, quarenta e seis mil e quinhentos.
Dos filhos de Simeão, as suas gerações pelas suas famílias,
segundo a casa dos seus pais; os seus contados, pelo número dos
nomes, cabeça por cabeça, todo o homem de vinte anos para cima,
todos os que podiam sair à guerra, foram contados deles, da
tribo de Simeão, cinqüenta e nove mil e trezentos. Dos filhos
de Gade, as suas gerações, pelas suas famílias, segundo a casa de
seus pais, pelo número dos nomes dos de vinte anos para cima, todos
os que podiam sair à guerra, foram contados deles, da tribo
de Gade, quarenta e cinco mil e seiscentos e cinqüenta. Dos
filhos de Judá, as suas gerações, pelas suas famílias, segundo a
casa de seus pais; pelo número dos nomes dos de vinte anos para
cima, todos os que podiam sair à guerra, foram contados
deles, da tribo de Judá, setenta e quatro mil e seiscentos.
Dos filhos de Issacar, as suas gerações, pelas suas famílias,
segundo a casa de seus pais, pelo número dos nomes dos de vinte anos
para cima, todos os que podiam sair à guerra, foram contados
deles da tribo de Issacar, cinqüenta e quatro mil e quatrocentos.
Dos filhos de Zebulom, as suas gerações, pelas suas famílias,
segundo a casa de seus pais, pelo número dos nomes dos de vinte anos
para cima, todos os que podiam sair à guerra, foram contados
deles, da tribo de Zebulom, cinqüenta e sete mil e quatrocentos.
Dos filhos de José, dos filhos de Efraim, as suas gerações,
pelas suas famílias, segundo a casa de seus pais, pelo número dos
nomes dos de vinte anos para cima, todos os que podiam sair à
guerra, foram contados deles, da tribo de Efraim, quarenta
mil e quinhentos. Dos filhos de Manassés, as suas gerações,
pelas suas famílias, segundo a casa de seus pais, pelo número dos
nomes dos de vinte anos para cima, todos os que podiam sair à
guerra, foram contados deles, da tribo de Manassés, trinta e
dois mil e duzentos. Dos filhos de Benjamim, as suas
gerações, pelas suas famílias, segundo a casa de seus pais, pelo
número dos nomes dos de vinte anos para cima, todos os que podiam
sair à guerra, foram contados deles, da tribo de Benjamim,
trinta e cinco mil e quatrocentos. Dos filhos de Dã, as suas
gerações, pelas suas famílias, segundo a casa de seus pais, pelo
número dos nomes dos de vinte anos para cima, todos os que podiam
sair à guerra, foram contados deles, da tribo de Dã, sessenta
e dois mil e setecentos. Dos filhos de Aser, as suas
gerações, pelas suas famílias, segundo a casa de seus pais, pelo
número dos nomes dos de vinte anos para cima, todos os que podiam
sair à guerra, foram contados deles, da tribo de Aser,
quarenta e um mil e quinhentos. Dos filhos de Naftali, as
suas gerações, pelas suas famílias, segundo a casa de seus pais,
pelo número dos nomes dos de vinte anos para cima, todos os que
podiam sair à guerra, foram contados deles, da tribo de
Naftali, cinqüenta e três mil e quatrocentos.

Estes foram os contados, que contaram Moisés e Arão, e os
príncipes de Israel, doze homens, cada um era pela casa de seus
pais. Assim foram todos os contados dos filhos de Israel,
segundo a casa de seus pais, de vinte anos para cima, todos os que
podiam sair à guerra em Israel; todos os contados eram
seiscentos e três mil e quinhentos e cinqüenta.

Mas os levitas, segundo a tribo de seus pais, não foram contados
entre eles, porquanto o Senhor tinha falado a Moisés,
dizendo: Porém não contarás a tribo de Levi, nem tomarás a
soma deles entre os filhos de Israel; mas tu põe os levitas
sobre o tabernáculo do testemunho, e sobre todos os seus utensílios,
e sobre tudo o que pertence a ele; eles levarão o tabernáculo e
todos os seus utensílios; e eles o administrarão, e acampar-se-ão ao
redor do tabernáculo. E, quando o tabernáculo partir, os
levitas o desarmarão; e, quando o tabernáculo se houver de assentar
no arraial, os levitas o armarão; e o estranho que se chegar
morrerá. E os filhos de Israel armarão as suas tendas, cada
um no seu esquadrão, e cada um junto à sua bandeira, segundo os seus
exércitos. Mas os levitas armarão as suas tendas ao redor do
tabernáculo do testemunho, para que não haja indignação sobre a
congregação dos filhos de Israel, pelo que os levitas terão o
cuidado da guarda do tabernáculo do testemunho. Assim fizeram
os filhos de Israel; conforme a tudo o que o Senhor ordenara a
Moisés, assim o fizeram.

\medskip

\lettrine{2} E falou o Senhor a Moisés e a Arão, dizendo:
Os filhos de Israel armarão as suas tendas, cada um debaixo da
sua bandeira, segundo as insígnias da casa de seus pais; ao redor,
defronte da tenda da congregação, armarão as suas tendas.

Os que armarem as suas tendas do lado do oriente, para o nascente,
serão os da bandeira do exército de Judá, segundo os seus
esquadrões, e Naassom, filho de Aminadabe, será príncipe dos filhos
de Judá. E o seu exército, os que foram contados deles, era de
setenta e quatro mil e seiscentos. E junto a ele armará as suas
tendas a tribo de Issacar; e Natanael, filho de Zuar, será príncipe
dos filhos de Issacar. E o seu exército, os que foram contados
deles, era de cinqüenta e quatro mil e quatrocentos. Depois a
tribo de Zebulom; e Eliabe, filho de Helam, será príncipe dos filhos
de Zebulom. E o seu exército, os que foram contados deles, era
de cinqüenta e sete mil e quatrocentos. Todos os que foram
contados do exército de Judá, cento e oitenta e seis mil e
quatrocentos, segundo os seus esquadrões, estes marcharão primeiro.
A bandeira do exército de Rúben, segundo os seus esquadrões,
estará para o lado do sul; e Elizur, filho de Sedeur, será príncipe
dos filhos de Rúben, e o seu exército, os que foram contados
deles, era de quarenta e seis mil e quinhentos. E junto a ele
armará as suas tendas a tribo de Simeão; e Selumiel, filho de
Zurisadai, será príncipe dos filhos de Simeão. E o seu
exército, os que foram contados deles, era de cinqüenta e nove mil e
trezentos. Depois a tribo de Gade; e Eliasafe, filho de
Deuel, será príncipe dos filhos de Gade. E o seu exército, os
que foram contados deles, era de quarenta e cinco mil e seiscentos e
cinqüenta. Todos os que foram contados no exército de Rúben
foram cento e cinqüenta e um mil e quatrocentos e cinqüenta, segundo
os seus esquadrões; e estes marcharão em segundo lugar. Então
partirá a tenda da congregação com o exército dos levitas no meio
dos exércitos; como armaram as suas tendas, assim marcharão, cada um
no seu lugar, segundo as suas bandeiras. A bandeira do
exército de Efraim segundo os seus esquadrões, estará para o lado do
ocidente; e Elisama, filho de Amiúde, será príncipe dos filhos de
Efraim. E o seu exército, os que foram contados deles, era de
quarenta mil e quinhentos. E junto a ele estará a tribo de
Manassés; e Gamaliel, filho de Pedazur, será príncipe dos filhos de
Manassés. E o seu exército, os que foram contados deles, era
de trinta e dois mil e duzentos. Depois a tribo de Benjamim;
e Abidã, filho de Gideoni, será príncipe dos filhos de Benjamim,
e o seu exército, os que foram contados deles, era de trinta
e cinco mil e quatrocentos. Todos os que foram contados no
exército de Efraim foram cento e oito mil e cem, segundo os seus
esquadrões; e estes marcharão em terceiro lugar. A bandeira
do exército de Dã estará para o norte, segundo os seus esquadrões; e
Aieser, filho de Amisadai, será príncipe dos filhos de Dã. E
o seu exército, os que foram contados deles, era de sessenta e dois
mil e setecentos. E junto a ele armará as suas tendas a tribo
de Aser; e Pagiel, filho de Ocrã, será príncipe dos filhos de Aser.
E o seu exército, os que foram contados deles, era de
quarenta e um mil e quinhentos. Depois a tribo de Naftali; e
Aira, filho de Enã, será príncipe dos filhos de Naftali. E o
seu exército, os que foram contados deles, era de cinqüenta e três
mil e quatrocentos. Todos os que foram contados no exército
de Dã foram cento e cinqüenta e sete mil e seiscentos; estes
marcharão em último lugar, segundo as suas bandeiras. Estes
são os que foram contados dos filhos de Israel, segundo a casa de
seus pais; todos os que foram contados dos exércitos pelos seus
esquadrões foram seiscentos e três mil e quinhentos e cinqüenta.
Mas os levitas não foram contados entre os filhos de Israel,
como o Senhor ordenara a Moisés. E os filhos de Israel
fizeram conforme a tudo o que o Senhor ordenara a Moisés; assim
armaram o arraial segundo as suas bandeiras, e assim marcharam, cada
qual segundo as suas gerações, segundo a casa de seus pais.

\medskip

\lettrine{3} E estas são as gerações de Arão e de Moisés, no
dia em que o Senhor falou com Moisés, no monte Sinai. E estes
são os nomes dos filhos de Arão: o primogênito Nadabe; depois Abiú,
Eleazar e Itamar. Estes são os nomes dos filhos de Arão, dos
sacerdotes ungidos, cujas mãos foram consagradas para administrar o
sacerdócio. Mas Nadabe e Abiú morreram perante o Senhor, quando
ofereceram fogo estranho perante o Senhor no deserto de Sinai, e não
tiveram filhos; porém Eleazar e Itamar administraram o sacerdócio
diante de Arão, seu pai. E falou o Senhor a Moisés, dizendo:
Faze chegar a tribo de Levi, e põe-na diante de Arão, o
sacerdote, para que o sirvam, e tenham cuidado da sua guarda, e
da guarda de toda a congregação, diante da tenda da congregação,
para administrar o ministério do tabernáculo. E tenham cuidado
de todos os utensílios da tenda da congregação, e da guarda dos
filhos de Israel, para administrar o ministério do tabernáculo.
Darás, pois, os levitas a Arão e a seus filhos; dentre os filhos
de Israel lhes são dados em dádiva. Mas a Arão e a seus
filhos ordenarás que guardem o seu sacerdócio, e o estranho que se
chegar morrerá. E falou o Senhor a Moisés, dizendo: E
eu, eis que tenho tomado os levitas do meio dos filhos de Israel, em
lugar de todo o primogênito, que abre a madre, entre os filhos de
Israel; e os levitas serão meus. Porque todo o primogênito é
meu; desde o dia em que tenho ferido a todo o primogênito na terra
do Egito, santifiquei para mim todo o primogênito em Israel, desde o
homem até ao animal: meus serão; Eu sou o Senhor.

E falou o Senhor a Moisés no deserto de Sinai, dizendo:
Conta os filhos de Levi, segundo a casa de seus pais, pelas
suas famílias; contarás a todo o homem da idade de um mês para cima.
E Moisés os contou conforme ao mandado do Senhor, como lhe
foi ordenado. Estes, pois, foram os filhos de Levi pelos seus
nomes: Gérson, e Coate e Merari. E estes são os nomes dos
filhos de Gérson pelas suas famílias: Libni e Simei. E os
filhos de Coate pelas suas famílias: Amrão, e Jizar, Hebrom e Uziel.
E os filhos de Merari pelas suas famílias: Maeli e
Musi.\footnote{SBTB: usa um ``;'' em vez de ponto. AV: And the sons
of Merari by their families; Mahli, and Mushi. These are the
families of the Levites according to the house of their fathers. }
Estas são as famílias dos levitas, segundo a casa de seus pais.
De Gérson é a família dos libnitas e a família dos simeítas;
estas são as famílias dos gersonitas. Os que deles foram
contados pelo número de todo o homem da idade de um mês para cima,
sim, os que deles foram contados eram sete mil e quinhentos.
As famílias dos gersonitas armarão as suas tendas atrás do
tabernáculo, ao ocidente. E o príncipe da casa paterna dos
gersonitas será Eliasafe, filho de Lael. E os filhos de
Gérson terão a seu cargo, na tenda da congregação, o tabernáculo, a
tenda, a sua coberta, e o véu da porta da tenda da congregação.
E as cortinas do pátio, e o pavilhão da porta do pátio, que
estão junto ao tabernáculo e junto ao altar, em redor; como também
as suas cordas para todo o seu serviço. E de Coate é a
família dos amramitas, e a família dos jizaritas, e a família dos
hebronitas, e a família dos uzielitas; estas são as famílias dos
coatitas. Pelo número contado de todo o homem da idade de um
mês para cima, eram oito mil e seiscentos, que tinham cuidado da
guarda do santuário. As famílias dos filhos de Coate armarão
as suas tendas ao lado do tabernáculo, do lado do sul. E o
príncipe da casa paterna das famílias dos coatitas será Elisafã,
filho de Uziel. E a sua guarda será a arca, e a mesa, e o
candelabro, e os altares, e os utensílios do santuário com que
ministram, e o véu com todo o seu serviço. E o príncipe dos
príncipes de Levi será Eleazar, filho de Arão, o sacerdote; terá a
superintendência sobre os que têm cuidado da guarda do santuário.
De Merari é a família dos malitas e a família dos musitas;
estas são as famílias de Merari. E os que deles foram
contados pelo número de todo o homem de um mês para cima, foram seis
mil e duzentos. E o príncipe da casa paterna das famílias de
Merari será Zuriel, filho de Abiail; armarão as suas tendas ao lado
do tabernáculo, do lado do norte. E os filhos de Merari terão
a seu cargo as tábuas do tabernáculo, os seus varais, as suas
colunas, as suas bases, e todos os seus utensílios, com todo o seu
serviço. E as colunas do pátio em redor, e as suas bases, as
suas estacas e as suas cordas. E os que armarão as suas
tendas diante do tabernáculo, ao oriente, diante da tenda da
congregação, para o nascente, serão Moisés e Arão, com seus filhos,
tendo o cuidado da guarda do santuário, pela guarda dos filhos de
Israel; e o estranho que se chegar morrerá. Todos os que
foram contados dos levitas, que contaram Moisés e Arão por mandado
do Senhor, segundo as suas famílias, todo o homem de um mês para
cima, foram vinte e dois mil.

E disse o Senhor a Moisés: Conta todo o primogênito homem dos
filhos de Israel, da idade de um mês para cima, e toma o número dos
seus nomes, e para mim tomarás os levitas (eu sou o Senhor),
em lugar de todo o primogênito dos filhos de Israel, e os animais
dos levitas, em lugar de todo o primogênito entre os animais dos
filhos de Israel. E contou Moisés, como o Senhor lhe
ordenara, todo o primogênito entre os filhos de Israel. E
todos os primogênitos homens, pelo número dos nomes dos da idade de
um mês para cima, segundo os que eram contados deles, foram vinte e
dois mil e duzentos e setenta e três. E falou o Senhor a
Moisés, dizendo: Toma os levitas em lugar de todo o
primogênito entre os filhos de Israel, e os animais dos levitas em
lugar dos seus animais; porquanto os levitas serão meus: Eu sou o
Senhor. Quanto aos duzentos e setenta e três, que se houverem
de resgatar dos primogênitos dos filhos de Israel, que excedem ao
número dos levitas, tomarás, por cabeça, cinco siclos;
conforme ao siclo do santuário os tomarás, a vinte geras o siclo.
E a Arão e a seus filhos darás o dinheiro dos resgatados, dos
que sobram entre eles. Então Moisés tomou o dinheiro do
resgate dos que excederam sobre os resgatados pelos levitas.
Dos primogênitos dos filhos de Israel recebeu o dinheiro, mil
e trezentos e sessenta e cinco siclos, segundo o siclo do santuário.
E Moisés deu o dinheiro dos resgatados a Arão e a seus
filhos, segundo o mandado do Senhor, como o Senhor ordenara a
Moisés.

\medskip

\lettrine{4} E falou o Senhor a Moisés e a Arão, dizendo:
Fazei a soma dos filhos de Coate, dentre os filhos de Levi,
pelas suas famílias, segundo a casa de seus pais; da idade de
trinta anos para cima até aos cinqüenta anos, será todo aquele que
entrar neste serviço, para fazer o trabalho na tenda da congregação.
Este será o ministério dos filhos de Coate na tenda da
congregação, nas coisas santíssimas. Quando partir o arraial,
Arão e seus filhos virão e tirarão o véu da tenda, e com ele
cobrirão a arca do testemunho; e pôr-lhe-ão por cima uma coberta
de peles de texugos, e sobre ela estenderão um pano, todo azul, e
lhe colocarão os varais. Também sobre a mesa da proposição
estenderão um pano azul; e sobre ela porão os pratos, as colheres, e
as taças e os jarros para libação; também o pão contínuo estará
sobre ela. Depois estenderão em cima deles um pano de carmesim,
e com a coberta de peles de texugos o cobrirão, e lhe colocarão os
seus varais. Então tomarão um pano azul, e cobrirão o candelabro
da luminária, e as suas lâmpadas, e os seus espevitadores, e os seus
apagadores, e todos os seus vasos de azeite, com que o servem.
E envolverão, a ele e a todos os seus utensílios, na coberta
de peles de texugos; e o colocarão sobre os varais. E sobre o
altar de ouro estenderão um pano azul, e com a coberta de peles de
texugos, o cobrirão, e lhe colocarão os seus varais. Também
tomarão todos os utensílios do ministério, com que servem no
santuário; e os colocarão num pano azul, e os cobrirão com uma
coberta de peles de texugos, e os colocarão sobre os varais.
E tirarão as cinzas do altar, e por cima dele estenderão um
pano de púrpura. E sobre ele colocarão todos os seus
instrumentos com que o servem: os seus braseiros, os garfos e as
pás, e as bacias; todos os pertences do altar; e por cima dele
estenderão uma coberta de peles de texugos, e lhe colocarão os seus
varais. Havendo, pois, Arão e seus filhos, ao partir do
arraial, acabado de cobrir o santuário, e todos os instrumentos do
santuário, então os filhos de Coate virão para levá-lo; mas no
santuário não tocarão para que não morram; este é o cargo dos filhos
de Coate na tenda da congregação. Porém o cargo de Eleazar,
filho de Arão, o sacerdote, será o azeite da luminária e o incenso
aromático, e a contínua oferta dos alimentos, e o azeite da unção, o
cargo de todo o tabernáculo, e de tudo que nele há, o santuário e os
seus utensílios. E falou o Senhor a Moisés e a Arão, dizendo:
Não deixareis extirpar a tribo das famílias dos coatitas do
meio dos levitas. Mas isto lhes fareis, para que vivam e não
morram, quando se aproximarem das coisas santíssimas: Arão e seus
filhos virão, e a cada um colocarão no seu ministério e no seu
cargo, porém não entrarão a ver, quando cobrirem o santuário,
para que não morram.

Falou mais o Senhor a Moisés, dizendo: Fazei também a soma
dos filhos de Gérson, segundo a casa de seus pais, segundo as suas
famílias: Da idade de trinta anos para cima até aos
cinqüenta, contarás a todo aquele que entrar a se ocupar no seu
serviço, para executar o ministério na tenda da congregação.
Este será o ministério das famílias dos gersonitas no serviço
e no cargo. Levarão, pois, as cortinas do tabernáculo, e a
tenda da congregação, e a sua coberta, e a coberta de peles de
texugos, que está por cima dele, e a cortina da porta da tenda da
congregação, e as cortinas do pátio, e a cortina da porta do
pátio, que está junto ao tabernáculo, e junto ao altar em redor, e
as suas cordas, e todos os instrumentos do seu ministério, com tudo
o que diz respeito a eles, para que sirvam. Todo o ministério
dos filhos dos gersonitas, em todo o seu cargo, e em todo o seu
trabalho, será segundo o mandado de Arão e de seus filhos; e lhes
designareis as responsabilidades do seu cargo. Este é o
ministério das famílias dos filhos dos gersonitas na tenda da
congregação; e a sua guarda será debaixo da mão de Itamar, filho de
Arão, o sacerdote. Quanto aos filhos de Merari, segundo as
suas famílias e segundo a casa de seus pais os contarás; da
idade de trinta anos para cima, até aos cinqüenta, contarás a todo
aquele que entrar neste serviço, para administrar o ministério da
tenda da congregação. Esta, pois, será a responsabilidade do
seu cargo, segundo todo o seu ministério, na tenda da congregação:
As tábuas do tabernáculo, e os seus varais, e as suas colunas, e as
suas bases; como também as colunas do pátio em redor, e as
suas bases, e as suas estacas, e as suas cordas, com todos os seus
instrumentos, e com todo o seu ministério; e contareis os objetos
que ficarão a seu cargo, nome por nome. Este é o ministério
das famílias dos filhos de Merari, segundo todo o seu ministério, na
tenda da congregação, debaixo da mão de Itamar, filho de Arão, o
sacerdote.

Moisés, pois, e Arão e os príncipes da congregação contaram os
filhos dos coatitas, segundo as suas famílias e segundo a casa de
seus pais; da idade de trinta anos para cima, até aos
cinqüenta, todo aquele que entrou neste serviço, para o ministério
da tenda da congregação. Os que deles foram contados, pois,
segundo as suas famílias, foram dois mil e setecentos e cinqüenta.
Estes são os que foram contados das famílias dos coatitas, de
todo aquele que ministrava na tenda da congregação, os quais Moisés
e Arão contaram, conforme ao mandado do Senhor pela mão de Moisés.
Semelhantemente os que foram contados dos filhos de Gérson,
segundo as suas famílias, e segundo a casa de seus pais; da
idade de trinta anos para cima até aos cinqüenta, todo aquele que
entrou neste serviço, para o ministério na tenda da congregação.
Os que deles foram contados, segundo as suas famílias,
segundo a casa de seus pais, foram dois mil e seiscentos e trinta.
Estes são os contados das famílias dos filhos de Gérson, de
todo aquele que ministrava na tenda da congregação; os quais Moisés
e Arão contaram, conforme ao mandado do Senhor. E os que
foram contados das famílias dos filhos de Merari, segundo as suas
famílias, segundo a casa de seus pais; da idade de trinta
anos para cima, até aos cinqüenta, todo aquele que entrou neste
serviço, para o ministério na tenda da congregação. Os que
deles foram contados, segundo as suas famílias, eram três mil e
duzentos. Estes são os contados das famílias dos filhos de
Merari; os quais Moisés e Arão contaram, conforme ao mandado do
Senhor, pela mão de Moisés. Todos os que deles foram
contados, que contaram Moisés e Arão, e os príncipes de Israel, dos
levitas, segundo as suas famílias, segundo a casa de seus pais;
da idade de trinta anos para cima, até aos cinqüenta, todo
aquele que entrava a executar o ministério da administração, e o
ministério das cargas na tenda da congregação, os que deles
foram contados foram oito mil quinhentos e oitenta. Conforme
ao mandado do Senhor, pela mão de Moisés, foram contados cada qual
segundo o seu ministério, e segundo o seu cargo; assim foram
contados por ele, como o Senhor ordenara a Moisés.

\medskip

\lettrine{5} E falou o Senhor a Moisés, dizendo: Ordena
aos filhos de Israel que lancem fora do arraial a todo o leproso, e
a todo o que padece fluxo, e a todos os imundos por causa de contato
com algum morto. Desde o homem até a mulher os lançareis; fora
do arraial os lançareis; para que não contaminem os seus arraiais,
no meio dos quais eu habito. E os filhos de Israel fizeram
assim, e os lançaram fora do arraial; como o Senhor falara a Moisés,
assim fizeram os filhos de Israel. Falou mais o Senhor a Moisés,
dizendo: Dize aos filhos de Israel: Quando homem ou mulher fizer
algum de todos os pecados humanos, transgredindo contra o Senhor,
tal alma culpada é. E confessará o seu pecado que cometeu; pela
sua culpa, fará plena restituição, segundo a soma total, e lhe
acrescentará a sua quinta parte, e a dará àquele contra quem se fez
culpado. Mas, se aquele homem não tiver resgatador, a quem se
restitua a culpa, então a culpa que se restituir ao Senhor será do
sacerdote, além do carneiro da expiação pelo qual por ele se fará
expiação. Semelhantemente toda a oferta de todas as coisas
santificadas dos filhos de Israel, que trouxerem ao sacerdote, será
sua. E as coisas santificadas de cada um serão suas; o que
alguém der ao sacerdote será seu.

Falou mais o Senhor a Moisés, dizendo: Fala aos filhos de
Israel, e dize-lhes: Quando a mulher de alguém se desviar, e
transgredir contra ele, de maneira que algum homem se tenha
deitado com ela, e for oculto aos olhos de seu marido, e ela o tiver
ocultado, havendo-se ela contaminado, e contra ela não houver
testemunha, e no feito não for apanhada, e o espírito de
ciúmes vier sobre ele, e de sua mulher tiver ciúmes, por ela se
haver contaminado, ou sobre ele vier o espírito de ciúmes, e de sua
mulher tiver ciúmes, não se havendo ela contaminado, então
aquele homem trará a sua mulher perante o sacerdote, e juntamente
trará a sua oferta por ela; uma décima de efa de farinha de cevada,
sobre a qual não deitará azeite, nem sobre ela porá incenso,
porquanto é oferta de alimentos por ciúmes, oferta memorativa, que
traz a iniqüidade em memória. E o sacerdote a fará chegar, e
a porá perante a face do Senhor. E o sacerdote tomará água
santa num vaso de barro; também tomará o sacerdote do pó que houver
no chão do tabernáculo, e o deitará na água. Então o
sacerdote apresentará a mulher perante o Senhor, e descobrirá a
cabeça da mulher; e a oferta memorativa, que é a oferta por ciúmes,
porá sobre as suas mãos, e a água amarga, que traz consigo a
maldição, estará na mão do sacerdote. E o sacerdote a fará
jurar, e dirá àquela mulher: Se ninguém contigo se deitou, e se não
te apartaste de teu marido pela imundícia, destas águas amargas,
amaldiçoantes, serás livre. Mas, se te apartaste de teu
marido, e te contaminaste, e algum homem, fora de teu marido, se
deitou contigo, então o sacerdote fará jurar à mulher com o
juramento da maldição; e o sacerdote dirá à mulher: O Senhor te
ponha por maldição e por praga no meio do teu povo, fazendo-te o
Senhor consumir a tua coxa e inchar o teu ventre. E esta água
amaldiçoante entre nas tuas entranhas, para te fazer inchar o
ventre, e te fazer consumir a coxa. Então a mulher dirá: Amém, Amém.
Depois o sacerdote escreverá estas mesmas maldições num
livro, e com a água amarga as apagará. E a água amarga,
amaldiçoante, dará a beber à mulher, e a água amaldiçoante entrará
nela para amargurar. E o sacerdote tomará a oferta por ciúmes
da mão da mulher, e moverá a oferta perante o Senhor; e a oferecerá
sobre o altar. Também o sacerdote tomará um punhado da oferta
memorativa, e sobre o altar a queimará; e depois dará a beber a água
à mulher. E, havendo-lhe dado a beber aquela água, será que,
se ela se tiver contaminado, e contra seu marido tiver transgredido,
a água amaldiçoante entrará nela para amargura, e o seu ventre se
inchará, e consumirá a sua coxa; e aquela mulher será por maldição
no meio do seu povo. E, se a mulher se não tiver contaminado,
mas estiver limpa, então será livre, e conceberá filhos. Esta
é a lei dos ciúmes, quando a mulher, em poder de seu marido, se
desviar e for contaminada; ou quando sobre o homem vier o
espírito de ciúmes, e tiver ciúmes de sua mulher, apresente a mulher
perante o Senhor, e o sacerdote nela execute toda esta lei. E
o homem será livre da iniqüidade, porém a mulher levará a sua
iniqüidade.

\medskip

\lettrine{6} E falou o Senhor a Moisés, dizendo: Fala aos
filhos de Israel, e dize-lhes: Quando um homem ou mulher se tiver
separado, fazendo voto de nazireu, para se separar ao Senhor, de
vinho e de bebida forte se apartará; vinagre de vinho, nem vinagre
de bebida forte não beberá; nem beberá alguma beberagem de uvas; nem
uvas frescas nem secas comerá. Todos os dias do seu nazireado
não comerá de coisa alguma, que se faz da vinha, desde os caroços
até às cascas. Todos os dias do voto do seu nazireado sobre a
sua cabeça não passará navalha; até que se cumpram os dias, que se
separou ao Senhor, santo será, deixando crescer livremente o cabelo
da sua cabeça. Todos os dias que se separar para o Senhor não se
aproximará do corpo de um morto. Por seu pai, ou por sua mãe,
por seu irmão, ou por sua irmã, por eles se não contaminará quando
forem mortos; porquanto o nazireado do seu Deus está sobre a sua
cabeça. Todos os dias do seu nazireado santo será ao Senhor.
E se alguém vier a morrer junto a ele por acaso, subitamente,
que contamine a cabeça do seu nazireado, então no dia da sua
purificação rapará a sua cabeça, ao sétimo dia a rapará. E ao
oitavo dia trará duas rolas, ou dois pombinhos, ao sacerdote, à
porta da tenda da congregação; e o sacerdote oferecerá, um
para expiação do pecado, e o outro para holocausto; e fará expiação
por ele, do que pecou relativamente ao morto; assim naquele mesmo
dia santificará a sua cabeça. Então separará os dias do seu
nazireado ao Senhor, e para expiação da transgressão trará um
cordeiro de um ano; e os dias antecedentes serão perdidos, porquanto
o seu nazireado foi contaminado. E esta é a lei do nazireu:
no dia em que se cumprirem os dias do seu nazireado, trá-lo-ão à
porta da tenda da congregação; e ele oferecerá a sua oferta
ao Senhor, um cordeiro sem defeito de um ano em holocausto, e uma
cordeira sem defeito de um ano para expiação do pecado, e um
carneiro sem defeito por oferta pacífica; e um cesto de pães
ázimos, bolos de flor de farinha com azeite, amassados, e coscorões
ázimos untados com azeite, como também a sua oferta de alimentos, e
as suas libações. E o sacerdote os trará perante o Senhor, e
sacrificará a sua expiação do pecado, e o seu holocausto;
também sacrificará o carneiro em sacrifício pacífico ao
Senhor, com o cesto dos pães ázimos; e o sacerdote oferecerá a sua
oferta de alimentos, e a sua libação. Então o nazireu à porta
da tenda da congregação rapará a cabeça do seu nazireado, e tomará o
cabelo da cabeça do seu nazireado, e o porá sobre o fogo que está
debaixo do sacrifício pacífico. Depois o sacerdote tomará a
espádua cozida do carneiro, e um pão ázimo do cesto, e um coscorão
ázimo, e os porá nas mãos do nazireu, depois de haver rapado a
cabeça do seu nazireado. E o sacerdote os oferecerá em oferta
de movimento perante o Senhor: Isto é santo para o sacerdote,
juntamente com o peito da oferta de movimento, e com a espádua da
oferta alçada; e depois o nazireu poderá beber vinho. Esta é
a lei do nazireu, que fizer voto da sua oferta ao Senhor pelo seu
nazireado, além do que suas posses lhe permitirem; segundo o seu
voto, que fizer, assim fará conforme à lei do seu nazireado.

E falou o Senhor a Moisés, dizendo: Fala a Arão, e a seus
filhos dizendo: Assim abençoareis os filhos de Israel, dizendo-lhes:
O Senhor te abençoe e te guarde; 25 o Senhor faça
resplandecer o seu rosto sobre ti, e tenha misericórdia de ti;
o Senhor sobre ti levante o seu rosto e te dê a paz.
Assim porão o meu nome sobre os filhos de Israel, e eu os
abençoarei.

\medskip

\lettrine{7} E aconteceu, no dia em que Moisés acabou de
levantar o tabernáculo, e o ungiu, e o santificou, e todos os seus
utensílios; também o altar, e todos os seus pertences, e os ungiu, e
os santificou, que os príncipes de Israel, os cabeças da casa de
seus pais, os que foram príncipes das tribos, que estavam sobre os
que foram contados, ofereceram, e trouxeram a sua oferta perante
o Senhor, seis carros cobertos, e doze bois; por dois príncipes um
carro, e cada um deles um boi; e os apresentaram diante do
tabernáculo. E falou o Senhor a Moisés, dizendo: 5 Recebe-os
deles, e serão para servir no ministério da tenda da congregação; e
os darás aos levitas, a cada qual segundo o seu ministério.
Assim Moisés recebeu os carros e os bois, e os deu aos levitas.
Dois carros e quatro bois deu aos filhos de Gérson, segundo o
seu ministério; e quatro carros e oito bois deu aos filhos de
Merari, segundo o seu ministério, debaixo da mão de Itamar, filho de
Arão, o sacerdote. Mas aos filhos de Coate nada deu, porquanto a
seu cargo estava o santuário e o levavam aos ombros.

E ofereceram os príncipes para a consagração do altar, no dia em
que foi ungido; apresentaram, pois, os príncipes a sua oferta
perante o altar. E disse o Senhor a Moisés: Cada príncipe
oferecerá a sua oferta, cada qual no seu dia, para a consagração do
altar. O que, pois, no primeiro dia apresentou a sua oferta
foi Naassom, filho de Aminadabe, pela tribo de Judá. E a sua
oferta foi um prato de prata, do peso de cento e trinta siclos, uma
bacia de prata de setenta siclos, segundo o siclo do santuário;
ambos cheios de flor de farinha, amassada com azeite, para oferta de
alimentos; uma colher de dez siclos de ouro, cheia de
incenso; um novilho, um carneiro, um cordeiro de um ano, para
holocausto; um bode para expiação do pecado; e para
sacrifício pacífico dois bois, cinco carneiros, cinco bodes, cinco
cordeiros de um ano; esta foi a oferta de Naassom, filho de
Aminadabe. No segundo dia fez a sua oferta Natanael, filho de
Zuar, príncipe de Issacar. E como sua oferta ofereceu um
prato de prata, do peso de cento e trinta siclos, uma bacia de prata
de setenta siclos, segundo o siclo do santuário; ambos cheios de
flor de farinha amassada com azeite, para a oferta de alimentos;
uma colher de dez siclos de ouro, cheia de incenso; um
novilho, um carneiro, um cordeiro de um ano, para holocausto;
um bode para expiação do pecado; e para sacrifício
pacífico dois bois, cinco carneiros, cinco bodes, cinco cordeiros de
um ano; esta foi a oferta de Natanael, filho de Zuar. No
terceiro dia ofereceu o príncipe dos filhos de Zebulom, Eliabe,
filho de Helom. A sua oferta foi um prato de prata, do peso
de cento e trinta siclos, uma bacia de prata de setenta siclos,
segundo o siclo do santuário; ambos cheios de flor de farinha
amassada com azeite, para oferta de alimentos; uma colher de
dez siclos de ouro, cheia de incenso; um novilho, um
carneiro, um cordeiro de um ano, para holocausto; um bode
para expiação do pecado; e para sacrifício pacífico dois
bois, cinco carneiros, cinco bodes, cinco cordeiros de um ano; esta
foi a oferta de Eliabe, filho de Helom. No quarto dia
ofereceu o príncipe dos filhos de Rúben, Elizur, filho de Sedeur;
a sua oferta foi um prato de prata, do peso de cento e trinta
siclos, uma bacia de prata de setenta siclos, segundo o siclo do
santuário; ambos cheios de flor de farinha, amassada com azeite,
para oferta de alimentos; uma colher de dez siclos de ouro,
cheia de incenso; um novilho, um carneiro, um cordeiro de um
ano, para holocausto; um bode para expiação do pecado;
e para sacrifício pacífico dois bois, cinco carneiros, cinco
bodes, cinco cordeiros de um ano; esta foi a oferta de Elizur, filho
de Sedeur. No quinto dia ofereceu o príncipe dos filhos de
Simeão, Selumiel, filho de Zurisadai. A sua oferta foi um
prato de prata, do peso de cento e trinta siclos, uma bacia de prata
de setenta siclos, segundo o siclo do santuário; ambos cheios de
flor de farinha amassada com azeite, para oferta de alimentos;
uma colher de dez siclos de ouro, cheia de incenso; um
novilho, um carneiro, um cordeiro de um ano para holocausto;
um bode para expiação do pecado; e para sacrifício
pacífico dois bois, cinco carneiros, cinco bodes, cinco cordeiros de
um ano; esta foi a oferta de Selumiel, filho de Zurisadai. No
sexto dia ofereceu o príncipe dos filhos de Gade; Eliasafe, filho de
Deuel. A sua oferta foi um prato de prata, do peso de cento e
trinta siclos, uma bacia de prata de setenta siclos, segundo o siclo
do santuário; ambos cheios de flor de farinha, amassada com azeite,
para oferta de alimentos; uma colher de dez siclos de ouro,
cheia de incenso; um novilho, um carneiro, um cordeiro de um
ano, para holocausto; um bode para expiação do pecado.
E para sacrifício pacífico dois bois, cinco carneiros, cinco
bodes, cinco cordeiros de um ano; esta foi a oferta de Eliasafe,
filho de Deuel. No sétimo dia ofereceu o príncipe dos filhos
de Efraim, Elisama, filho de Amiúde. A sua oferta foi um
prato de prata, do peso de cento e trinta siclos, uma bacia de prata
de setenta siclos, segundo o siclo do santuário; ambos cheios de
flor de farinha, amassada com azeite, para oferta de alimentos;
uma colher de dez siclos de ouro, cheia de incenso; um
novilho, um carneiro, um cordeiro de um ano, para holocausto;
um bode para expiação do pecado; e para sacrifício
pacífico dois bois, cinco carneiros, cinco bodes, cinco cordeiros de
um ano; esta foi a oferta de Elisama, filho de Amiúde. No
oitavo dia ofereceu o príncipe dos filhos de Manassés, Gamaliel,
filho de Pedazur. A sua oferta foi um prato de prata, do peso
de cento e trinta siclos, uma bacia de prata de setenta siclos,
segundo o siclo do santuário; ambos cheios de flor de farinha,
amassada com azeite, para oferta de alimentos; uma colher de
dez siclos de ouro, cheia de incenso; um novilho, um
carneiro, um cordeiro de um ano, para holocausto; um bode
para expiação do pecado; e para sacrifício pacífico dois
bois, cinco carneiros, cinco bodes, cinco cordeiros de um ano; esta
foi a oferta de Gamaliel, filho de Pedazur. No dia nono
ofereceu o príncipe dos filhos de Benjamim, Abidã, filho de Gideoni;
a sua oferta foi um prato de prata, do peso de cento e trinta
siclos, uma bacia de prata de setenta siclos, segundo o siclo do
santuário; ambos cheios de flor de farinha, amassada com azeite,
para oferta de alimentos; uma colher de dez siclos de ouro,
cheia de incenso; um novilho, um carneiro, um cordeiro de um
ano, para holocausto; um bode para expiação do pecado;
e para sacrifício pacífico dois bois, cinco carneiros, cinco
bodes, cinco cordeiros de um ano; esta foi a oferta de Abidã filho
de Gideoni. No décimo dia ofereceu o príncipe dos filhos de
Dã, Aieser, filho de Amisadai. A sua oferta foi um prato de
prata, do peso de cento e trinta siclos, uma bacia de prata de
setenta siclos, segundo o siclo do santuário; ambos cheios de flor
de farinha, amassada com azeite, para oferta de alimentos;
uma colher de dez siclos de ouro, cheia de incenso; um
novilho, um carneiro, um cordeiro de um ano, para holocausto;
um bode para expiação do pecado; e para sacrifício
pacífico dois bois, cinco carneiros, cinco bodes, cinco cordeiros de
um ano; esta foi a oferta de Aieser, filho de Amisadai. No
dia undécimo ofereceu o príncipe dos filhos de Aser, Pagiel, filho
de Ocrã; a sua oferta foi um prato de prata, do peso de cento
e trinta siclos, uma bacia de prata de setenta siclos, segundo o
siclo do santuário; ambos cheios de flor de farinha, amassada com
azeite, para oferta de alimentos; uma colher de dez siclos de
ouro, cheia de incenso; um novilho, um carneiro, um cordeiro
de um ano, para holocausto; um bode para expiação do pecado;
e para sacrifício pacífico dois bois, cinco carneiros, cinco
bodes, cinco cordeiros de um ano; esta foi a oferta de Pagiel, filho
de Ocrã. No duodécimo dia ofereceu o príncipe dos filhos de
Naftali, Aira, filho de Enã. A sua oferta foi um prato de
prata, do peso de cento e trinta siclos, uma bacia de prata de
setenta siclos, segundo o siclo do santuário; ambos cheios de flor
de farinha, amassada com azeite, para oferta de alimentos;
uma colher de dez siclos de ouro, cheia de incenso; um
novilho, um carneiro, um cordeiro de um ano, para holocausto;
um bode para expiação do pecado; 83 e para sacrifício
pacífico dois bois, cinco carneiros, cinco bodes, cinco cordeiros de
um ano; esta foi a oferta de Aira, filho de Enã. Esta foi a
consagração do altar, feita pelos príncipes de Israel, no dia em que
foi ungido, doze pratos de prata, doze bacias de prata, doze
colheres de ouro. Cada prato de prata de cento e trinta
siclos, e cada bacia de setenta; toda a prata dos vasos foi dois mil
e quatrocentos siclos, segundo o siclo do santuário; doze
colheres de ouro cheias de incenso, cada colher de dez siclos,
segundo o siclo do santuário; todo o ouro das colheres foi de cento
e vinte siclos; todos os animais para holocausto foram doze
novilhos, doze carneiros, doze cordeiros de um ano, com a sua oferta
de alimentos e doze bodes para expiação do pecado. E todos os
animais para sacrifício pacífico foram vinte e quatro novilhos, os
carneiros sessenta, os bodes sessenta, os cordeiros de um ano
sessenta; esta foi a consagração do altar, depois que foi ungido.
E, quando Moisés entrava na tenda da congregação para falar
com ele, então ouvia a voz que lhe falava de cima do propiciatório,
que estava sobre a arca do testemunho entre os dois querubins; assim
com ele falava.

\medskip

\lettrine{8} E falou o Senhor a Moisés, dizendo: Fala a
Arão, e dize-lhe: Quando acenderes as lâmpadas, as sete lâmpadas
iluminarão o espaço em frente do candelabro. E Arão fez assim:
Acendeu as lâmpadas do candelabro para iluminar o espaço em frente,
como o Senhor ordenara a Moisés. E era esta a obra do
candelabro, obra de ouro batido; desde o seu pé até às suas flores
era ele de ouro batido; conforme ao modelo que o Senhor mostrara a
Moisés, assim ele fez o candelabro.

E falou o Senhor a Moisés, dizendo: Toma os levitas do meio
dos filhos de Israel e purifica-os; e assim lhes farás, para os
purificar: Esparge sobre eles a água da expiação; e sobre toda a sua
carne farão passar a navalha, e lavarão as suas vestes, e se
purificarão. Então tomarão um novilho, com a sua oferta de
alimentos de flor de farinha amassada com azeite; e tomarás tu outro
novilho, para expiação do pecado. E farás chegar os levitas
perante a tenda da congregação e ajuntarás toda a congregação dos
filhos de Israel. Farás, pois, chegar os levitas perante o
Senhor; e os filhos de Israel porão as suas mãos sobre os levitas.
E Arão oferecerá os levitas por oferta movida, perante o
Senhor, pelos filhos de Israel; e serão para servirem no ministério
do Senhor. E os levitas colocarão as suas mãos sobre a cabeça
dos novilhos; então sacrifica tu, um para expiação do pecado, e o
outro para holocausto ao Senhor, para fazer expiação pelos levitas.
E porás os levitas perante Arão, e perante os seus filhos, e
os oferecerá por oferta movida ao Senhor. E separarás os
levitas do meio dos filhos de Israel, para que os levitas sejam
meus. E depois os levitas entrarão para fazerem o serviço da
tenda da congregação; e tu os purificarás, e por oferta movida os
oferecerás. Porquanto eles, dentre os filhos de Israel, me
são dados; em lugar de todo aquele que abre a madre, do primogênito
de cada um dos filhos de Israel, para mim os tenho tomado.
Porque meu é todo o primogênito entre os filhos de Israel,
entre os homens e entre os animais; no dia em que, na terra do
Egito, feri a todo o primogênito, os santifiquei para mim. E
tomei os levitas em lugar de todo o primogênito entre os filhos de
Israel. E os levitas, dados a Arão e a seus filhos, dentre os
filhos de Israel, tenho dado para ministrarem o ministério dos
filhos de Israel na tenda da congregação e para fazer expiação pelos
filhos de Israel, para que não haja praga entre eles, chegando-se os
filhos de Israel ao santuário. E assim fizeram Moisés e Arão,
e toda a congregação dos filhos de Israel, com os levitas; conforme
a tudo o que o Senhor ordenara a Moisés acerca dos levitas, assim os
filhos de Israel lhes fizeram. E os levitas se purificaram, e
lavaram as suas vestes, e Arão os ofereceu por oferta movida perante
o Senhor, e Arão fez expiação por eles, para purificá-los. E
depois vieram os levitas, para exercerem o seu ministério na tenda
da congregação, perante Arão e perante os seus filhos; como o Senhor
ordenara a Moisés acerca dos levitas, assim lhes fizeram. E
falou o Senhor a Moisés, dizendo: Este é o ofício dos
levitas: Da idade de vinte e cinco anos para cima entrarão, para
fazerem o serviço no ministério da tenda da congregação; mas
desde a idade de cinqüenta anos sairão do serviço deste ministério,
e nunca mais servirão; porém com os seus irmãos servirão na
tenda da congregação, para terem cuidado da guarda; mas o ministério
não exercerão; assim farás com os levitas quanto aos seus deveres.

\medskip

\lettrine{9} E falou o Senhor a Moisés no deserto de Sinai, no
ano segundo da sua saída da terra do Egito, no primeiro mês,
dizendo: Celebrem os filhos de Israel a páscoa a seu tempo
determinado. No dia catorze deste mês, pela tarde, a seu tempo
determinado a celebrareis; segundo todos os seus estatutos, e
segundo todos os seus ritos, a celebrareis. Disse, pois, Moisés
aos filhos de Israel que celebrassem a páscoa. Então celebraram
a páscoa no dia catorze do primeiro mês, pela tarde, no deserto de
Sinai; conforme a tudo o que o Senhor ordenara a Moisés, assim
fizeram os filhos de Israel. E houve alguns que estavam imundos
por terem tocado o corpo de um homem morto; e não podiam celebrar a
páscoa naquele dia; por isso se chegaram perante Moisés e Arão
naquele mesmo dia; e aqueles homens disseram-lhe: Imundos
estamos nós pelo corpo de um homem morto; por que seríamos privados
de oferecer a oferta do Senhor a seu tempo determinado no meio dos
filhos de Israel? E disse-lhes Moisés: Esperai, e eu ouvirei o
que o Senhor vos ordenará. Então falou o Senhor a Moisés,
dizendo: Fala aos filhos de Israel, dizendo: Quando alguém
entre vós, ou entre as vossas gerações, for imundo por tocar corpo
morto, ou achar-se em jornada longe de vós, contudo ainda celebrará
a páscoa ao Senhor. No mês segundo, no dia catorze à tarde, a
celebrarão; com pães ázimos e ervas amargas a comerão. Dela
nada deixarão até à manhã, e dela não quebrarão osso algum; segundo
todo o estatuto da páscoa a celebrarão. Porém, quando um
homem for limpo, e não estiver em viajem, e deixar de celebrar a
páscoa, essa alma do seu povo será extirpada; porquanto não ofereceu
a oferta do Senhor a seu tempo determinado; esse homem levará o seu
pecado. E, quando um estrangeiro peregrinar entre vós, e
também celebrar a páscoa ao Senhor, segundo o estatuto da páscoa e
segundo o seu rito assim a celebrará; um mesmo estatuto haverá para
vós, assim para o estrangeiro, como para o natural da terra.

E no dia em que foi levantado o tabernáculo, a nuvem cobriu o
tabernáculo sobre a tenda do testemunho; e à tarde estava sobre o
tabernáculo com uma aparência de fogo até à manhã. Assim era
de contínuo: a nuvem o cobria, e de noite havia aparência de fogo.
Mas sempre que a nuvem se alçava de sobre a tenda, os filhos
de Israel partiam; e no lugar onde a nuvem parava, ali os filhos de
Israel se acampavam. Segundo a ordem do Senhor, os filhos de
Israel partiam, e segundo a ordem do Senhor se acampavam; todos os
dias em que a nuvem parava sobre o tabernáculo, ficavam acampados.
E, quando a nuvem se detinha muitos dias sobre o tabernáculo,
então os filhos de Israel cumpriam a ordem do Senhor, e não partiam.
E, quando a nuvem ficava poucos dias sobre o tabernáculo,
segundo a ordem do Senhor se alojavam, e segundo a ordem do Senhor
partiam. Porém, outras vezes a nuvem ficava desde a tarde até
à manhã, e quando ela se alçava pela manhã, então partiam; quer de
dia quer de noite alçando-se a nuvem, partiam. Ou, quando a
nuvem sobre o tabernáculo se detinha dois dias, ou um mês, ou um
ano, ficando sobre ele, então os filhos de Israel se alojavam, e não
partiam; e alçando-se ela, partiam. Segundo a ordem do Senhor
se alojavam, e segundo a ordem do Senhor partiam; cumpriam o seu
dever para com o Senhor, segundo a ordem do Senhor por intermédio de
Moisés.

\medskip

\lettrine{10} Falou mais o Senhor a Moisés, dizendo:
Faze-te duas trombetas de prata; de obra batida as farás, e elas
te servirão para a convocação da congregação, e para a partida dos
arraiais. E, quando as tocarem, então toda a congregação se
reunirá a ti à porta da tenda da congregação. Mas, quando tocar
uma só, então a ti se congregarão os príncipes, os cabeças dos
milhares de Israel. Quando, retinindo\footnote{Retinir: Tinir
muito ou demoradamente; produzir grande som; ressoar; impressionar
vivamente; causar profunda impressão; repercutir, ecoar.}, as
tocardes, então partirão os arraiais que estão acampados do lado do
oriente. Mas, quando a segunda vez retinindo, as tocardes, então
partirão os arraiais que estão acampados do lado do sul; retinindo,
as tocarão para as suas partidas. Porém, ajuntando a
congregação, as tocareis; mas sem retinir. E os filhos de Arão,
sacerdotes, tocarão as trombetas; e a vós serão por estatuto
perpétuo nas vossas gerações. E, quando na vossa terra sairdes a
pelejar contra o inimigo, que vos oprime, também tocareis as
trombetas retinindo, e perante o Senhor vosso Deus haverá lembrança
de vós, e sereis salvos de vossos inimigos. Semelhantemente,
no dia da vossa alegria e nas vossas solenidades, e nos princípios
de vossos meses, também tocareis as trombetas sobre os vossos
holocaustos, sobre os vossos sacrifícios pacíficos, e vos serão por
memorial perante vosso Deus: Eu sou o Senhor vosso Deus.

E aconteceu, no ano segundo, no segundo mês, aos vinte do mês,
que a nuvem se alçou de sobre o tabernáculo da congregação. E
os filhos de Israel, segundo a ordem de marcha, partiram do deserto
de Sinai; e a nuvem parou no deserto de Parã. Assim partiram
pela primeira vez segundo a ordem do Senhor, por intermédio de
Moisés. Porque primeiramente partiu a bandeira do arraial dos
filhos de Judá segundo os seus exércitos; e sobre o seu exército
estava Naassom, filho de Aminadabe. E sobre o exército da
tribo dos filhos de Issacar, Natanael, filho de Zuar. E sobre
o exército da tribo dos filhos de Zebulom, Eliabe, filho de Helom.
Então desarmaram o tabernáculo, e os filhos de Gérson e os
filhos de Merari partiram, levando o tabernáculo. Depois
partiu a bandeira do arraial de Rúben segundo os seus exércitos; e
sobre o seu exército estava Elizur, filho de Sedeur. E sobre
o exército da tribo dos filhos de Simeão, Selumiel, filho de
Zurisadai. E sobre o exército da tribo dos filhos de Gade,
Eliasafe, filho de Deuel. Então partiram os coatitas, levando
o santuário; e os outros levantaram o tabernáculo, enquanto estes
vinham. Depois partiu a bandeira do arraial dos filhos de
Efraim segundo os seus exércitos; e sobre o seu exército estava
Elisama, filho de Amiúde. E sobre o exército da tribo dos
filhos de Manassés, Gamaliel, filho de Pedazur. E sobre o
exército da tribo dos filhos de Benjamim, Abidã, filho de Gideoni.
Então partiu a bandeira do arraial dos filhos de Dã, fechando
todos os arraiais segundo os seus exércitos; e sobre o seu exército
estava Aieser, filho de Amisadai. E sobre o exército da tribo
dos filhos de Aser, Pagiel, filho de Ocrã. E sobre o exército
da tribo dos filhos de Naftali, Aira, filho de Enã. Esta era
a ordem das partidas dos filhos de Israel segundo os seus exércitos,
quando partiam.

Disse então Moisés a Hobabe, filho de Reuel, o midianita, sogro
de Moisés: Nós caminhamos para aquele lugar, de que o Senhor disse:
Vo-lo darei; vai conosco e te faremos bem; porque o Senhor falou bem
sobre Israel. Porém ele lhe disse: Não irei; antes irei à
minha terra e à minha parentela. E ele disse: Ora, não nos
deixes; porque tu sabes onde devemos acampar no deserto; nos
servirás de guia. E será que, vindo tu conosco, e sucedendo o
bem que o Senhor nos fizer, também nós te faremos bem. Assim
partiram do monte do Senhor caminho de três dias; e a arca da
aliança do Senhor caminhou diante deles caminho de três dias, para
lhes buscar lugar de descanso. E a nuvem do Senhor ia sobre
eles de dia, quando partiam do arraial. Acontecia que,
partindo a arca, Moisés dizia: Levanta-te, Senhor, e dissipados
sejam os teus inimigos, e fujam diante de ti os odiadores. E,
pousando ela, dizia: Volta, ó Senhor, para os muitos milhares de
Israel.

\medskip

\lettrine{11} E aconteceu que, queixou-se o povo falando o que
era mal aos ouvidos do Senhor; e ouvindo o Senhor a sua ira se
acendeu; e o fogo do Senhor ardeu entre eles e consumiu os que
estavam na última parte do arraial. Então o povo clamou a
Moisés, e Moisés orou ao Senhor, e o fogo se apagou. Pelo que
chamou aquele lugar Taberá, porquanto o fogo do Senhor se acendera
entre eles.

E o vulgo, que estava no meio deles, veio a ter grande desejo;
pelo que os filhos de Israel tornaram a chorar, e disseram: Quem nos
dará carne a comer? Lembramo-nos dos peixes que no Egito
comíamos de graça; e dos pepinos, e dos melões, e dos porros, e das
cebolas, e dos alhos. Mas agora a nossa alma se seca; coisa
nenhuma há senão este maná diante dos nossos olhos. E era o maná
como semente de coentro, e a sua cor como a cor de
bdélio\footnote{Goma-resina semelhante à mirra, extraída de várias
árvores burseráceas do gênero Commiphora; Quím. Material ceroso,
avermelhado, de cheiro agradável, obtido como exsudação da
Balsomodendron africanum, usado em perfumaria.}. Espalhava-se o
povo e o colhia, e em moinhos o moía, ou num
gral\footnote{Almofariz: Recipiente de pedra, metal, etc., em que se
trituram e homogeneízam substâncias sólidas; pilão, gral, morteiro.}
o pisava, e em panelas o cozia, e dele fazia bolos; e o seu sabor
era como o sabor de azeite fresco. E, quando o orvalho descia de
noite sobre o arraial, o maná descia sobre ele. Então Moisés
ouviu chorar o povo pelas suas famílias, cada qual à porta da sua
tenda; e a ira do Senhor grandemente se acendeu, e pareceu mal aos
olhos de Moisés. E disse Moisés ao Senhor: Por que fizeste
mal a teu servo, e por que não achei graça aos teus olhos, visto que
puseste sobre mim o cargo de todo este povo? Concebi eu
porventura todo este povo? Dei-o eu à luz? para que me dissesses:
leva-o ao teu colo, como a ama leva a criança que mama, à terra que
juraste a seus pais? De onde teria eu carne para dar a todo
este povo? Porquanto contra mim choram, dizendo: Dá-nos carne a
comer; eu só não posso levar a todo este povo, porque muito
pesado é para mim. E se assim fazes comigo, mata-me, peço-te,
se tenho achado graça aos teus olhos, e não me deixes ver o meu mal.

E disse o Senhor a Moisés: Ajunta-me setenta homens dos anciãos
de Israel, que sabes serem anciãos do povo e seus oficiais; e os
trarás perante a tenda da congregação, e ali estejam contigo.
Então eu descerei e ali falarei contigo, e tirarei do
espírito que está sobre ti, e o porei sobre eles; e contigo levarão
a carga do povo, para que tu não a leves sozinho. E dirás ao
povo: Santificai-vos para amanhã, e comereis carne; porquanto
chorastes aos ouvidos do Senhor, dizendo: Quem nos dará carne a
comer? Pois íamos bem no Egito; por isso o Senhor vos dará carne, e
comereis; não comereis um dia, nem dois dias, nem cinco dias,
nem dez dias, nem vinte dias; mas um mês inteiro, até vos
sair pelas narinas, até que vos enfastieis dela; porquanto
rejeitastes ao Senhor, que está no meio de vós, e chorastes diante
dele, dizendo: Por que saímos do Egito? E disse Moisés:
Seiscentos mil homens de pé é este povo, no meio do qual estou; e tu
tens dito: Dar-lhes-ei carne, e comerão um mês inteiro.
Degolar-se-ão para eles ovelhas e vacas que lhes bastem? Ou
ajuntar-se-ão para eles todos os peixes do mar, que lhes bastem?
Porém, o Senhor disse a Moisés: Teria sido encurtada a mão do
Senhor? Agora verás se a minha palavra se há de cumprir ou não.

E saiu Moisés, e falou as palavras do Senhor ao povo, e ajuntou
setenta homens dos anciãos do povo e os pôs ao redor da tenda.
Então o Senhor desceu na nuvem, e lhe falou; e, tirando do
espírito, que estava sobre ele, o pôs sobre aqueles setenta anciãos;
e aconteceu que, quando o espírito repousou sobre eles,
profetizaram; mas depois nunca mais. Porém no arraial ficaram
dois homens; o nome de um era Eldade, e do outro Medade; e repousou
sobre eles o espírito (porquanto estavam entre os inscritos, ainda
que não saíram à tenda), e profetizavam no arraial. Então
correu um moço e anunciou a Moisés e disse: Eldade e Medade
profetizam no arraial. E Josué, filho de Num, servidor de
Moisés, um dos seus jovens escolhidos, respondeu e disse: Moisés,
meu senhor, proíbe-lho. Porém, Moisés lhe disse: Tens tu
ciúmes por mim? Quem dera que todo o povo do Senhor fosse profeta, e
que o Senhor pusesse o seu espírito sobre ele! Depois Moisés
se recolheu ao arraial, ele e os anciãos de Israel.

Então soprou um vento do Senhor e trouxe codornizes do mar, e as
espalhou pelo arraial quase caminho de um dia, de um lado e de outro
lado, ao redor do arraial; quase dois côvados sobre a terra.
Então o povo se levantou todo aquele dia e toda aquela noite,
e todo o dia seguinte, e colheram as codornizes; o que menos tinha,
colhera dez ômeres; e as estenderam para si ao redor do arraial.
Quando a carne estava entre os seus dentes, antes que fosse
mastigada, se acendeu a ira do Senhor contra o povo, e feriu o
Senhor o povo com uma praga mui grande. Por isso o nome
daquele lugar se chamou Quibrote-Ataavá, porquanto ali enterraram o
povo que teve o desejo. De Quibrote-Ataavá caminhou o povo
para Hazerote, e pararam em Hazerote.

\medskip

\lettrine{12} E falaram Miriã e Arão contra Moisés, por causa
da mulher cusita, com quem casara; porquanto tinha casado com uma
mulher cusita. E disseram: Porventura falou o Senhor somente por
Moisés? Não falou também por nós? E o Senhor o ouviu. E era o
homem Moisés mui manso, mais do que todos os homens que havia sobre
a terra.

E logo o Senhor disse a Moisés, a Arão e a Miriã: Vós três saí à
tenda da congregação. E saíram eles três. Então o Senhor desceu
na coluna de nuvem, e se pôs à porta da tenda; depois chamou a Arão
e a Miriã e ambos saíram. E disse: Ouvi agora as minhas
palavras; se entre vós houver profeta, eu, o Senhor, em visão a ele
me farei conhecer, ou em sonhos falarei com ele. Não é assim com
o meu servo Moisés que é fiel em toda a minha casa. Boca a boca
falo com ele, claramente e não por enigmas; pois ele vê a semelhança
do Senhor; por que, pois, não tivestes temor de falar contra o meu
servo, contra Moisés? Assim a ira do Senhor contra eles se
acendeu; e retirou-se.

E a nuvem se retirou de sobre a tenda; e eis que Miriã ficou
leprosa como a neve; e olhou Arão para Miriã, e eis que estava
leprosa. Por isso Arão disse a Moisés: Ai, senhor meu, não
ponhas sobre nós este pecado, pois agimos loucamente, e temos
pecado. Ora, não seja ela como um morto, que saindo do ventre
de sua mãe, a metade da sua carne já esteja consumida.
Clamou, pois, Moisés ao Senhor, dizendo: Ó Deus, rogo-te que
a cures. E disse o Senhor a Moisés: Se seu pai cuspira em seu
rosto, não seria envergonhada sete dias? Esteja fechada sete dias
fora do arraial, e depois a recolham. Assim Miriã esteve
fechada fora do arraial sete dias, e o povo não partiu, até que
recolheram a Miriã. Porém, depois o povo partiu de Hazerote;
e acampou-se no deserto de Parã.

\medskip

\lettrine{13} E falou o Senhor a Moisés, dizendo: Envia
homens que espiem a terra de Canaã, que eu hei de dar aos filhos de
Israel; de cada tribo de seus pais enviareis um homem, sendo cada um
príncipe entre eles. E enviou-os Moisés do deserto de Parã,
segundo a ordem do Senhor; todos aqueles homens eram cabeças dos
filhos de Israel. E estes são os seus nomes: Da tribo de Rúben,
Samua, filho de Zacur; da tribo de Simeão, Safate, filho de
Hori; da tribo de Judá, Calebe, filho de Jefoné; da tribo de
Issacar, Jigeal, filho de José; da tribo de Efraim, Oséias,
filho de Num; da tribo de Benjamim, Palti, filho de Rafu;
da tribo de Zebulom, Gadiel, filho de Sodi; da tribo
de José, pela tribo de Manassés, Gadi filho de Susi; da tribo
de Dã, Amiel, filho de Gemali; da tribo de Aser, Setur, filho
de Micael; da tribo de Naftali, Nabi, filho de Vofsi;
da tribo de Gade, Geuel, filho de Maqui. Estes são os
nomes dos homens que Moisés enviou a espiar aquela terra; e a
Oséias, filho de Num, Moisés chamou Josué. Enviou-os, pois,
Moisés a espiar a terra de Canaã; e disse-lhes: Subi por aqui para o
lado do sul, e subi à montanha: E vede que terra é, e o povo
que nela habita; se é forte ou fraco; se pouco ou muito. E
como é a terra em que habita, se boa ou má; e quais são as cidades
em que eles habitam; se em arraiais, ou em fortalezas. Também
como é a terra, se fértil ou estéril; se nela há árvores, ou não; e
esforçai-vos, e tomai do fruto da terra. E eram aqueles dias os dias
das primícias das uvas.

Assim subiram e espiaram a terra desde o deserto de Zim, até
Reobe, à entrada de Hamate. E subiram para o lado do sul, e
vieram até Hebrom; e estavam ali Aimã, Sesai e Talmai, filhos de
Enaque (Hebrom foi edificada sete anos antes de Zoã no Egito).
Depois foram até ao vale de Escol, e dali cortaram um ramo de
vide com um cacho de uvas, o qual trouxeram dois homens, sobre uma
vara; como também das romãs e dos figos. Chamaram àquele
lugar o vale de Escol, por causa do cacho que dali cortaram os
filhos de Israel. E eles voltaram de espiar a terra, ao fim
de quarenta dias.

E caminharam, e vieram a Moisés e a Arão, e a toda a congregação
dos filhos de Israel no deserto de Parã, em Cades; e deram-lhes
notícias, a eles, e a toda a congregação, e mostraram-lhes o fruto
da terra. E contaram-lhe, e disseram: Fomos à terra a que nos
enviaste; e verdadeiramente mana leite e mel, e este é o seu fruto.
O povo, porém, que habita nessa terra é poderoso, e as
cidades fortificadas e mui grandes; e também ali vimos os filhos de
Enaque. Os amalequitas habitam na terra do sul; e os heteus,
e os jebuseus, e os amorreus habitam na montanha; e os cananeus
habitam junto do mar, e pela margem do Jordão. Então Calebe
fez calar o povo perante Moisés, e disse: Certamente subiremos e a
possuiremos em herança; porque seguramente prevaleceremos contra
ela. Porém, os homens que com ele subiram disseram: Não
poderemos subir contra aquele povo, porque é mais forte do que nós.
E infamaram a terra que tinham espiado, dizendo aos filhos de
Israel: A terra, pela qual passamos a espiá-la, é terra que consome
os seus moradores; e todo o povo que vimos nela são homens de grande
estatura. Também vimos ali gigantes, filhos de Enaque,
descendentes dos gigantes; e éramos aos nossos olhos como
gafanhotos, e assim também éramos aos seus olhos.

\medskip

\lettrine{14} Então toda a congregação levantou a sua voz; e o
povo chorou naquela noite. E todos os filhos de Israel
murmuraram contra Moisés e contra Arão; e toda a congregação lhes
disse: Quem dera tivéssemos morrido na terra do Egito! ou, mesmo
neste deserto! E por que o Senhor nos traz a esta terra, para
cairmos à espada, e para que nossas mulheres e nossas crianças sejam
por presa? Não nos seria melhor voltarmos ao Egito? E diziam uns
aos outros: Constituamos um líder, e voltemos ao Egito.

Então Moisés e Arão caíram sobre os seus rostos perante toda a
congregação dos filhos de Israel. E Josué, filho de Num, e
Calebe filho de Jefoné, dos que espiaram a terra, rasgaram as suas
vestes. E falaram a toda a congregação dos filhos de Israel,
dizendo: A terra pela qual passamos a espiar é terra muito boa.
Se o Senhor se agradar de nós, então nos porá nesta terra, e
no-la dará; terra que mana leite e mel. Tão-somente não sejais
rebeldes contra o Senhor, e não temais o povo dessa terra, porquanto
são eles nosso pão; retirou-se deles o seu amparo, e o Senhor é
conosco; não os temais. Mas toda a congregação disse que os
apedrejassem; porém a glória do Senhor apareceu na tenda da
congregação a todos os filhos de Israel.

E disse o Senhor a Moisés: Até quando me provocará este povo? e
até quando não crerá em mim, apesar de todos os sinais que fiz no
meio dele? Com pestilência o ferirei, e o rejeitarei; e te
farei a ti povo maior e mais forte do que este. E disse
Moisés ao Senhor: Assim os egípcios o ouvirão; porquanto com a tua
força fizeste subir este povo do meio deles. E dirão aos
moradores desta terra, os quais ouviram que tu, ó Senhor, estás no
meio deste povo, que face a face, ó Senhor, lhes apareces, que tua
nuvem está sobre ele e que vais adiante dele numa coluna de nuvem de
dia, e numa coluna de fogo de noite. E se matares este povo
como a um só homem, então as nações, que antes ouviram a tua fama,
falarão, dizendo: Porquanto o Senhor não podia pôr este povo
na terra que lhe tinha jurado; por isso os matou no deserto.
Agora, pois, rogo-te que a força do meu Senhor se engrandeça;
como tens falado, dizendo: O Senhor é longânimo, e grande em
misericórdia, que perdoa a iniqüidade e a transgressão, que o
culpado não tem por inocente, e visita a iniqüidade dos pais sobre
os filhos até à terceira e quarta geração. Perdoa, pois, a
iniqüidade deste povo, segundo a grandeza da tua misericórdia; e
como também perdoaste a este povo desde a terra do Egito até aqui.

E disse o Senhor: Conforme à tua palavra lhe perdoei.
Porém, tão certamente como eu vivo, e como a glória do Senhor
encherá toda a terra, e que todos os homens que viram a minha
glória e os meus sinais, que fiz no Egito e no deserto, e me
tentaram estas dez vezes, e não obedeceram à minha voz, não
verão a terra de que a seus pais jurei, e nenhum daqueles que me
provocaram a verá. Porém o meu servo Calebe, porquanto nele
houve outro espírito, e perseverou em seguir-me, eu o levarei à
terra em que entrou, e a sua descendência a possuirá em herança.
Ora, os amalequitas e os cananeus habitam no vale; tornai-vos
amanhã e caminhai para o deserto pelo caminho do Mar Vermelho.
Depois falou o Senhor a Moisés e a Arão dizendo: Até
quando sofrerei esta má congregação, que murmura contra mim? Tenho
ouvido as murmurações dos filhos de Israel, com que murmuram contra
mim. Dize-lhes: Vivo eu, diz o Senhor, que, como falastes aos
meus ouvidos, assim farei a vós outros. Neste deserto cairão
os vossos cadáveres, como também todos os que de vós foram contados
segundo toda a vossa conta, de vinte anos para cima, os que dentre
vós contra mim murmurastes; não entrareis na terra, pela qual
levantei a minha mão que vos faria habitar nela, salvo Calebe, filho
de Jefoné, e Josué, filho de Num. Mas os vossos filhos, de
que dizeis: Por presa serão, porei nela; e eles conhecerão a terra
que vós desprezastes. Porém, quanto a vós, os vossos
cadáveres cairão neste deserto. E vossos filhos pastorearão
neste deserto quarenta anos, e levarão sobre si as vossas
infidelidades, até que os vossos cadáveres se consumam neste
deserto. Segundo o número dos dias em que espiastes esta
terra, quarenta dias, cada dia representando um ano, levareis sobre
vós as vossas iniqüidades quarenta anos, e conhecereis o meu
afastamento. Eu, o Senhor, falei; assim farei a toda esta má
congregação, que se levantou contra mim; neste deserto se
consumirão, e aí falecerão.

E os homens que Moisés mandara a espiar a terra, e que, voltando,
fizeram murmurar toda a congregação contra ele, infamando a terra,
aqueles mesmos homens que infamaram a terra, morreram de
praga perante o Senhor. Mas Josué, filho de Num, e Calebe,
filho de Jefoné, que eram dos homens que foram espiar a terra,
ficaram com vida. E falou Moisés estas palavras a todos os
filhos de Israel; então o povo se contristou muito. E
levantaram-se pela manhã de madrugada, e subiram ao cume do monte,
dizendo: Eis-nos aqui, e subiremos ao lugar que o Senhor tem falado;
porquanto havemos pecado. Mas Moisés disse: Por que
transgredis o mandado do Senhor? Pois isso não prosperará.
Não subais, pois o Senhor não estará no meio de vós, para que
não sejais feridos diante dos vossos inimigos. Porque os
amalequitas e os cananeus estão ali diante da vossa face, e caireis
à espada; pois, porquanto vos desviastes do Senhor, o Senhor não
estará convosco. Contudo, temerariamente, tentaram subir ao
cume do monte; mas a arca da aliança do Senhor e Moisés não se
apartaram do meio do arraial. Então desceram os amalequitas e
os cananeus, que habitavam na montanha, e os feriram, derrotando-os
até Horma.

\medskip

\lettrine{15} Depois falou o Senhor a Moisés, dizendo:
Fala aos filhos de Israel, e dize-lhes: Quando entrardes na
terra das vossas habitações, que eu vos hei de dar, e ao Senhor
fizerdes oferta queimada, holocausto, ou sacrifício, para cumprir um
voto, ou em oferta voluntária, ou nas vossas solenidades, para
fazerdes ao Senhor um cheiro suave de ovelhas ou gado, então
aquele que apresentar a sua oferta ao Senhor, por oferta de
alimentos trará uma décima de flor de farinha misturada com a quarta
parte de um him de azeite. E de vinho para libação prepararás a
quarta parte de um him, para holocausto, ou para sacrifício para
cada cordeiro; e para cada carneiro prepararás uma oferta de
alimentos de duas décimas de flor de farinha, misturada com a terça
parte de um him de azeite. E de vinho para a libação oferecerás
a terça parte de um him ao Senhor, em cheiro suave. E, quando
preparares novilho para holocausto ou sacrifício, para cumprir um
voto, ou um sacrifício pacífico ao Senhor, com o novilho
apresentarás uma oferta de alimentos de três décimas de flor de
farinha misturada com a metade de um him de azeite. E de
vinho para a libação oferecerás a metade de um him, oferta queimada
em cheiro suave ao Senhor. Assim se fará com cada boi, ou com
cada carneiro, ou com cada um dos cordeiros ou cabritos.
Segundo o número que oferecerdes, assim o fareis com cada um,
segundo o número deles. Todo o natural assim fará estas
coisas, oferecendo oferta queimada em cheiro suave ao Senhor.
Quando também peregrinar convosco algum estrangeiro, ou que
estiver no meio de vós nas vossas gerações, e ele apresentar uma
oferta queimada de cheiro suave ao Senhor, como vós fizerdes, assim
fará ele. Um mesmo estatuto haja para vós, ó congregação, e
para o estrangeiro que entre vós peregrina, por estatuto perpétuo
nas vossas gerações; como vós, assim será o peregrino perante o
Senhor. Uma mesma lei e um mesmo direito haverá para vós e
para o estrangeiro que peregrina convosco. Falou mais o
Senhor a Moisés, dizendo: Fala aos filhos de Israel, e
dize-lhes: Quando entrardes na terra em que vos hei de introduzir,
acontecerá que, quando comerdes do pão da terra, então
oferecereis ao Senhor oferta alçada. Das primícias da vossa
massa oferecereis um bolo em oferta alçada; como a oferta da eira,
assim o oferecereis. Das primícias das vossas massas dareis
ao Senhor oferta alçada nas vossas gerações.

E, quando vierdes a errar, e não cumprirdes todos estes
mandamentos, que o Senhor falou a Moisés, tudo quanto o
Senhor vos tem mandado por intermédio de Moisés, desde o dia que o
Senhor ordenou, e dali em diante, nas vossas gerações, será
que, quando se fizer alguma coisa por ignorância, e for encoberto
aos olhos da congregação, toda a congregação oferecerá um novilho
para holocausto em cheiro suave ao Senhor, com a sua oferta de
alimentos e libação conforme ao estatuto, e um bode para expiação do
pecado. E o sacerdote fará expiação por toda a congregação
dos filhos de Israel, e lhes será perdoado, porquanto foi por
ignorância; e trouxeram a sua oferta, oferta queimada ao Senhor, e a
sua expiação do pecado perante o Senhor, por causa da sua
ignorância. Será, pois, perdoado a toda a congregação dos
filhos de Israel, e mais ao estrangeiro que peregrina no meio deles,
porquanto por ignorância sobreveio a todo o povo. E, se
alguma alma pecar por ignorância, para expiação do pecado oferecerá
uma cabra de um ano. E o sacerdote fará expiação pela pessoa
que pecou, quando pecar por ignorância, perante o Senhor, fazendo
expiação por ela, e lhe será perdoado. Para o natural dos
filhos de Israel, e para o estrangeiro que no meio deles peregrina,
uma mesma lei vos será, para aquele que pecar por ignorância.

Mas a pessoa que fizer alguma coisa temerariamente, quer seja dos
naturais quer dos estrangeiros, injuria ao Senhor; tal pessoa será
extirpada do meio do seu povo. Pois desprezou a palavra do
Senhor, e anulou o seu mandamento; totalmente será extirpada aquela
pessoa, a sua iniqüidade será sobre ela. Estando, pois, os
filhos de Israel no deserto, acharam um homem apanhando lenha no dia
de sábado. E os que o acharam apanhando lenha o trouxeram a
Moisés e a Arão, e a toda a congregação. E o puseram em
guarda; porquanto ainda não estava declarado o que se lhe devia
fazer. Disse, pois, o Senhor a Moisés: Certamente morrerá
aquele homem; toda a congregação o apedrejará fora do arraial.
Então toda a congregação o tirou para fora do arraial, e o
apedrejaram, e morreu, como o Senhor ordenara a Moisés.

E falou o Senhor a Moisés, dizendo: Fala aos filhos de
Israel, e dize-lhes: Que nas bordas das suas vestes façam franjas
pelas suas gerações; e nas franjas das bordas ponham um cordão de
azul. E as franjas vos serão para que, vendo-as, vos lembreis
de todos os mandamentos do Senhor, e os cumprais; e não seguireis o
vosso coração, nem após os vossos olhos, pelos quais andais vos
prostituindo. Para que vos lembreis de todos os meus
mandamentos, e os cumprais, e santos sejais a vosso Deus. Eu
sou o Senhor vosso Deus, que vos tirei da terra do Egito, para ser
vosso Deus. Eu sou o Senhor vosso Deus.

\medskip

\lettrine{16} E Coré, filho de Jizar, filho de Coate, filho de
Levi, tomou consigo a Datã e a Abirão, filhos de Eliabe, e a Om,
filho de Pelete, filhos de Rúben. E levantaram-se perante Moisés
com duzentos e cinqüenta homens dos filhos de Israel, príncipes da
congregação, chamados à assembléia, homens de posição, e se
congregaram contra Moisés e contra Arão, e lhes disseram: Basta-vos,
pois que toda a congregação é santa, todos são santos, e o Senhor
está no meio deles; por que, pois, vos elevais sobre a congregação
do Senhor? Quando Moisés ouviu isso, caiu sobre o seu rosto.
E falou a Coré e a toda a sua congregação, dizendo: Amanhã pela
manhã o Senhor fará saber quem é seu, e quem é o santo que ele fará
chegar a si; e aquele a quem escolher fará chegar a si. Fazei
isto: Tomai vós incensários, Coré e todo seu grupo; e, pondo
fogo neles amanhã, sobre eles deitai incenso perante o Senhor; e
será que o homem a quem o Senhor escolher, este será o santo;
basta-vos, filhos de Levi. Disse mais Moisés a Coré: Ouvi agora,
filhos de Levi: Porventura pouco para vós é que o Deus de Israel
vos tenha separado da congregação de Israel, para vos fazer chegar a
si, e administrar o ministério do tabernáculo do Senhor e estar
perante a congregação para ministrar-lhe; e te fez chegar, e
todos os teus irmãos, os filhos de Levi, contigo? ainda também
procurais o sacerdócio? Assim tu e todo o teu grupo estais
contra o Senhor; e Arão, quem é ele, que murmureis contra ele?

E Moisés mandou chamar a Datã e a Abirão, filhos de Eliabe; porém
eles disseram: Não subiremos; porventura pouco é que nos
fizeste subir de uma terra que mana leite e mel, para nos matares
neste deserto, senão que também queres fazer-te príncipe sobre nós?
Nem tampouco nos trouxeste a uma terra que mana leite e mel,
nem nos deste campo e vinhas em herança; porventura arrancarás os
olhos a estes homens? Não subiremos. Então Moisés irou-se
muito, e disse ao Senhor: Não atentes para a sua oferta; nem um só
jumento tomei deles, nem a nenhum deles fiz mal. Disse mais
Moisés a Coré: Tu e todo o teu grupo ponde-vos perante o Senhor, tu
e eles, e Arão, amanhã. E tomai cada um o seu incensário, e
neles ponde incenso; e trazei cada um o seu incensário perante o
Senhor, duzentos e cinqüenta incensários; também tu e Arão, cada um
o seu incensário. Tomaram, pois, cada um o seu incensário, e
neles puseram fogo, e neles deitaram incenso, e se puseram perante a
porta da tenda da congregação com Moisés e Arão. E Coré fez
ajuntar contra eles todo o povo à porta da tenda da congregação;
então a glória do Senhor apareceu a toda a congregação. E
falou o Senhor a Moisés e a Arão, dizendo: Apartai-vos do
meio desta congregação, e os consumirei num momento. Mas eles
se prostraram sobre os seus rostos, e disseram: Ó Deus, Deus dos
espíritos de toda a carne, pecará um só homem, e indignar-te-ás tu
contra toda esta congregação?

E falou o Senhor a Moisés, dizendo: Fala a toda esta
congregação, dizendo: Subi do derredor da habitação de Coré, Datã e
Abirão. Então Moisés levantou-se, e foi a Datã e a Abirão; e
após ele seguiram os anciãos de Israel. E falou à
congregação, dizendo: Desviai-vos, peço-vos, das tendas destes
homens ímpios, e não toqueis nada do que é seu para que porventura
não pereçais em todos os seus pecados. Subiram, pois, do
derredor da habitação de Coré, Datã e Abirão. E Datã e Abirão
saíram, e se puseram à porta das suas tendas, juntamente com as suas
mulheres, e seus filhos, e suas crianças. Então disse Moisés:
Nisto conhecereis que o Senhor me enviou a fazer todos estes feitos,
que de meu coração não procedem. Se estes morrerem como
morrem todos os homens, e se forem visitados como são visitados
todos os homens, então o Senhor não me enviou. Mas, se o
Senhor criar alguma coisa nova, e a terra abrir a sua boca e os
tragar com tudo o que é seu, e vivos descerem ao abismo, então
conhecereis que estes homens irritaram ao Senhor. E aconteceu
que, acabando ele de falar todas estas palavras, a terra que estava
debaixo deles se fendeu. E a terra abriu a sua boca, e os
tragou com as suas casas, como também a todos os homens que
pertenciam a Coré, e a todos os seus bens. E eles e tudo o
que era seu desceram vivos ao abismo, e a terra os cobriu, e
pereceram do meio da congregação. E todo o Israel, que estava
ao redor deles, fugiu ao clamor deles; porque diziam: Para que não
nos trague a terra também a nós.

Então saiu fogo do Senhor, e consumiu os duzentos e cinqüenta
homens que ofereciam o incenso. E falou o Senhor a Moisés,
dizendo: Dize a Eleazar, filho de Arão, o sacerdote, que tome
os incensários do meio do incêndio, e espalhe o fogo longe, porque
santos são; quanto aos incensários daqueles que pecaram
contra as suas almas, deles se façam folhas estendidas para
cobertura do altar; porquanto os trouxeram perante o Senhor; pelo
que santos são; e serão por sinal aos filhos de Israel. E
Eleazar, o sacerdote, tomou os incensários de metal, que trouxeram
aqueles que foram queimados, e os estenderam em folhas para
cobertura do altar, por memorial para os filhos de Israel,
que nenhum estranho, que não for da descendência de Arão, se chegue
para acender incenso perante o Senhor; para que não seja como Coré e
a sua congregação, como o Senhor lhe tinha dito por intermédio de
Moisés.

Mas no dia seguinte toda a congregação dos filhos de Israel
murmurou contra Moisés e contra Arão, dizendo: Vós matastes o povo
do Senhor. E aconteceu que, ajuntando-se a congregação contra
Moisés e Arão, e virando-se para a tenda da congregação, eis que a
nuvem a cobriu, e a glória do Senhor apareceu. Vieram, pois,
Moisés e Arão perante a tenda da congregação. Então falou o
Senhor a Moisés, dizendo: Levantai-vos do meio desta
congregação, e a consumirei num momento.\footnote{SBTB divide os
dois períodos com ``;''. AV: Get you up from among this
congregation, that I may consume them as in a moment. And they fell
upon their faces. RA: Levantai-vos do meio desta congregação, e a
consumirei num momento; então, se prostraram sobre o seu rosto. }
Então se prostraram sobre os seus rostos, e disse Moisés a
Arão: Toma o teu incensário, e põe nele fogo do altar, e deita
incenso sobre ele, e vai depressa à congregação, e faze expiação por
eles; porque grande indignação saiu de diante do Senhor; já começou
a praga. E tomou-o Arão, como Moisés tinha falado, e correu
ao meio da congregação; e eis que já a praga havia começado entre o
povo; e deitou incenso nele, e fez expiação pelo povo. E
estava em pé entre os mortos e os vivos; e cessou a praga. E
os que morreram daquela praga foram catorze mil e setecentos, fora
os que morreram pela causa de Coré. E voltou Arão a Moisés à
porta da tenda da congregação; e cessou a praga.

\medskip

\lettrine{17} Então falou o Senhor a Moisés, dizendo: Fala
aos filhos de Israel, e toma deles uma vara para cada casa paterna
de todos os seus príncipes, segundo as casas de seus pais, doze
varas; e escreverás o nome de cada um sobre a sua vara. Porém o
nome de Arão escreverás sobre a vara de Levi; porque cada cabeça da
casa de seus pais terá uma vara. E as porás na tenda da
congregação, perante o testemunho, onde eu virei a vós. E será
que a vara do homem que eu tiver escolhido florescerá; assim farei
cessar as murmurações dos filhos de Israel contra mim, com que
murmuram contra vós. Falou, pois, Moisés aos filhos de Israel; e
todos os seus príncipes deram-lhe cada um uma vara, para cada
príncipe uma vara, segundo as casas de seus pais, doze varas; e a
vara de Arão estava entre as deles. E Moisés pôs estas varas
perante o Senhor na tenda do testemunho.

Sucedeu, pois, que no dia seguinte Moisés entrou na tenda do
testemunho, e eis que a vara de Arão, pela casa de Levi, florescia;
porque produzira flores e brotara renovos e dera amêndoas. Então
Moisés tirou todas as varas de diante do Senhor a todos os filhos de
Israel; e eles o viram, e tomaram cada um a sua vara. Então o
Senhor disse a Moisés: Torna a pôr a vara de Arão perante o
testemunho, para que se guarde por sinal para os filhos rebeldes;
assim farás acabar as suas murmurações contra mim, e não morrerão.
E Moisés fez assim; como lhe ordenara o Senhor, assim fez.
Então falaram os filhos de Israel a Moisés, dizendo: Eis
aqui, nós expiramos, perecemos, nós todos perecemos. Todo
aquele que se aproximar do tabernáculo do Senhor, morrerá; seremos
pois todos consumidos?

\medskip

\lettrine{18} Então disse o Senhor a Arão: Tu, e teus filhos,
e a casa de teu pai contigo, levareis sobre vós a iniqüidade do
santuário; e tu e teus filhos contigo levareis sobre vós a
iniqüidade do vosso sacerdócio. E também farás chegar contigo a
teus irmãos, a tribo de Levi, a tribo de teu pai, para que se
ajuntem a ti, e te sirvam; mas tu e teus filhos contigo estareis
perante a tenda do testemunho. E eles cumprirão as tuas ordens e
terão o encargo de toda a tenda; mas não se chegarão aos utensílios
do santuário, nem ao altar, para que não morram, tanto eles como
vós. Mas se ajuntarão a ti, e farão o serviço da tenda da
congregação em todo o ministério da tenda; e o estranho não se
chegará a vós. Vós, pois, fareis o serviço do santuário e o
serviço do altar; para que não haja outra vez furor sobre os filhos
de Israel. E eu, eis que tenho tomado vossos irmãos, os levitas,
do meio dos filhos de Israel; são dados a vós em dádiva pelo Senhor,
para que sirvam ao ministério da tenda da congregação. Mas tu e
teus filhos contigo cumprireis o vosso sacerdócio no tocante a tudo
o que é do altar, e a tudo o que está dentro do véu, nisso
servireis; eu vos tenho dado o vosso sacerdócio em dádiva
ministerial e o estranho que se chegar morrerá.

Disse mais o Senhor a Arão: Eis que eu te tenho dado a guarda das
minhas ofertas alçadas, com todas as coisas santas dos filhos de
Israel; por causa da unção as tenho dado a ti e a teus filhos por
estatuto perpétuo. Isto terás das coisas santíssimas do
fogo:\footnote{SBTB: ``;''. Demais versões corretamente optam pelo
``:''. AV: This shall be thine of the most holy things, reserved
from the fire: every oblation of theirs, every meat \ldots{}. RA:
Isto terás das coisas santíssimas, não dadas ao fogo: todas as suas
ofertas, com todas as suas ofertas de manjares, \ldots{}. RC (1969):
Isto terás das cousas santíssimas do fogo: todas as suas ofertas,
com todas as suas ofertas de manjares, \ldots{}.} todas as suas
ofertas com todas as suas ofertas de alimentos, e com todas as suas
expiações pelo pecado, e com todas as suas expiações pela culpa, que
me apresentarão; serão coisas santíssimas para ti e para teus
filhos. No lugar santíssimo as comerás; todo o homem a
comerá; santas serão para ti. Também isto será teu: a oferta
alçada dos seus dons com todas as ofertas movidas dos filhos de
Israel; a ti, a teus filhos, e a tuas filhas contigo, as tenho dado
por estatuto perpétuo; todo o que estiver limpo na tua casa, delas
comerá. Todo o melhor do azeite, e todo o melhor do mosto e
do grão, as suas primícias que derem ao Senhor, as tenho dado a ti.
Os primeiros frutos de tudo que houver na terra, que
trouxerem ao Senhor, serão teus; todo o que estiver limpo na tua
casa os comerá. Toda a coisa consagrada em Israel será tua.
Tudo que abrir a madre, e toda a carne que trouxerem ao
Senhor, tanto de homens como de animais, será teu; porém os
primogênitos dos homens resgatarás; também os primogênitos dos
animais imundos resgatarás. Os que deles se houverem de
resgatar resgatarás, da idade de um mês, segundo a tua avaliação,
por cinco siclos de dinheiro\footnote{Ed.Contemp: siclos de prata.},
segundo o siclo do santuário, que é de vinte geras. Mas o
primogênito de vaca, ou primogênito de ovelha, ou primogênito de
cabra, não resgatarás, santos são; o seu sangue espargirás sobre o
altar, e a sua gordura queimarás em oferta queimada de cheiro suave
ao Senhor. E a carne deles será tua; assim como o peito da
oferta de movimento, e o ombro direito, teus serão. Todas as
ofertas alçadas das coisas santas, que os filhos de Israel
oferecerem ao Senhor, tenho dado a ti, e a teus filhos e a tuas
filhas contigo, por estatuto perpétuo; aliança perpétua de sal
perante o Senhor é, para ti e para a tua descendência contigo.

Disse também o Senhor a Arão: Na sua terra herança nenhuma terás,
e no meio deles, nenhuma parte terás; eu sou a tua parte e a tua
herança no meio dos filhos de Israel. E eis que aos filhos de
Levi tenho dado todos os dízimos em Israel por herança, pelo
ministério que executam, o ministério da tenda da congregação.
E nunca mais os filhos de Israel se chegarão à tenda da
congregação, para que não levem sobre si o pecado e morram.
Mas os levitas executarão o ministério da tenda da
congregação, e eles levarão sobre si a sua iniqüidade; pelas vossas
gerações estatuto perpétuo será; e no meio dos filhos de Israel
nenhuma herança terão, porque os dízimos dos filhos de
Israel, que oferecerem ao Senhor em oferta alçada, tenho dado por
herança aos levitas; porquanto eu lhes disse: No meio dos filhos de
Israel nenhuma herança terão. E falou o Senhor a Moisés,
dizendo: Também falarás aos levitas, e dir-lhes-ás: Quando
receberdes os dízimos dos filhos de Israel, que eu deles vos tenho
dado por vossa herança, deles oferecereis uma oferta alçada ao
Senhor, os dízimos dos dízimos. E contar-se-vos-á a vossa
oferta alçada, como grão da eira, e como plenitude do lagar.
Assim também oferecereis ao Senhor uma oferta alçada de todos
os vossos dízimos, que receberdes dos filhos de Israel, e deles
dareis a oferta alçada do Senhor a Arão, o sacerdote. De
todas as vossas dádivas oferecereis toda a oferta alçada do Senhor;
de tudo o melhor deles, a sua santa parte. Dir-lhes-ás pois:
Quando oferecerdes o melhor deles, como novidade da eira, e como
novidade do lagar, se contará aos levitas. E o comereis em
todo o lugar, vós e as vossas famílias, porque vosso galardão é pelo
vosso ministério na tenda da congregação. Assim, não levareis
sobre vós o pecado, quando deles oferecerdes o melhor; e não
profanareis as coisas santas dos filhos de Israel, para que não
morrais.

\medskip

\lettrine{19} Falou mais o Senhor a Moisés e a Arão dizendo:
Este é o estatuto da lei, que o Senhor ordenou, dizendo: Dize
aos filhos de Israel que te tragam uma novilha ruiva, que não tenha
defeito, e sobre a qual não tenha sido posto jugo. E a dareis a
Eleazar, o sacerdote; ele a tirará para fora do arraial, e
degolar-se-á diante dele. E Eleazar, o sacerdote, tomará do seu
sangue com o seu dedo, e dele espargirá para a frente da tenda da
congregação sete vezes. Então queimará a novilha perante os seus
olhos; o seu couro, e a sua carne, e o seu sangue, com o seu
esterco, se queimará. E o sacerdote tomará pau de cedro, e
hissopo, e carmesim, e os lançará no meio do fogo que queima a
novilha. Então o sacerdote lavará as suas vestes, e banhará a
sua carne na água, e depois entrará no arraial; e o sacerdote será
imundo até à tarde. Também o que a queimou lavará as suas vestes
com água, e em água banhará a sua carne, e imundo será até à tarde.
E um homem limpo ajuntará a cinza da novilha, e a porá fora do
arraial, num lugar limpo, e ficará ela guardada para a congregação
dos filhos de Israel, para a água da separação; expiação é. E
o que apanhou a cinza da novilha lavará as suas vestes, e será
imundo até à tarde; isto será por estatuto perpétuo aos filhos de
Israel e ao estrangeiro que peregrina no meio deles.

Aquele que tocar em algum morto, cadáver de algum homem, imundo
será sete dias. Ao terceiro dia se purificará com aquela
água, e ao sétimo dia será limpo; mas, se ao terceiro dia se não
purificar, não será limpo ao sétimo dia. Todo aquele que
tocar em algum morto, cadáver de algum homem, e não se purificar,
contamina o tabernáculo do Senhor; e aquela pessoa será extirpada de
Israel; porque a água da separação não foi espargida sobre ele,
imundo será; está nele ainda a sua imundícia. Esta é a lei,
quando morrer algum homem em alguma tenda, todo aquele que entrar
naquela tenda, e todo aquele que nela estiver, será imundo sete
dias. Também todo o vaso aberto, sobre o qual não houver pano
atado, será imundo. E todo aquele que sobre a face do campo
tocar em alguém que for morto pela espada, ou em outro morto ou nos
ossos de algum homem, ou numa sepultura, será imundo sete dias.
Para um imundo, pois, tomarão da cinza da queima da expiação,
e sobre ela colocarão água corrente num vaso. E um homem
limpo tomará hissopo, e o molhará naquela água, e a espargirá sobre
aquela tenda, e sobre todos os móveis, e sobre as pessoas que ali
estiverem, como também sobre aquele que tocar os ossos, ou em alguém
que foi morto, ou que faleceu, ou numa sepultura. E o limpo
ao terceiro e sétimo dia espargirá sobre o imundo; e ao sétimo dia o
purificará; e lavará as suas vestes, e se banhará na água, e à tarde
será limpo. Porém o que for imundo, e se não purificar, do
meio da congregação será ele extirpado; porquanto contaminou o
santuário do Senhor; água de separação sobre ele não foi espargida;
imundo é. Isto lhes será por estatuto perpétuo; e o que
espargir a água da separação lavará as suas vestes; e o que tocar a
água da separação será imundo até à tarde, e tudo o que tocar
o imundo também será imundo; e a pessoa que o tocar será imunda até
à tarde.

\medskip

\lettrine{20} Chegando os filhos de Israel, toda a
congregação, ao deserto de Zim, no mês primeiro, o povo ficou em
Cades; e Miriã morreu ali, e ali foi sepultada. E não havia água
para a congregação; então se reuniram contra Moisés e contra Arão.
E o povo contendeu com Moisés, dizendo: Quem dera tivéssemos
perecido quando pereceram nossos irmãos perante o Senhor! E por
que trouxestes a congregação do Senhor a este deserto, para que
morramos aqui, nós e os nossos animais? E por que nos fizestes
subir do Egito, para nos trazer a este lugar mau? lugar onde não há
semente, nem de figos, nem de vides, nem de romãs, nem tem água para
beber. Então Moisés e Arão se foram de diante do povo à porta da
tenda da congregação, e se lançaram sobre os seus rostos; e a glória
do Senhor lhes apareceu. E o Senhor falou a Moisés dizendo:
Toma a vara, e ajunta a congregação, tu e Arão, teu irmão, e
falai à rocha, perante os seus olhos, e dará a sua água; assim lhes
tirarás água da rocha, e darás a beber à congregação e aos seus
animais. Então Moisés tomou a vara de diante do Senhor, como lhe
tinha ordenado. E Moisés e Arão reuniram a congregação diante
da rocha, e Moisés disse-lhes: Ouvi agora, rebeldes, porventura
tiraremos água desta rocha para vós? Então Moisés levantou a
sua mão, e feriu a rocha duas vezes com a sua vara, e saiu muita
água; e bebeu a congregação e os seus animais. E o Senhor
disse a Moisés e a Arão: Porquanto não crestes em mim, para me
santificardes diante dos filhos de Israel, por isso não
introduzireis esta congregação na terra que lhes tenho dado.
Estas são as águas de Meribá, porque os filhos de Israel
contenderam com o Senhor; e se santificou neles.

Depois Moisés, de Cades, mandou mensageiros ao rei de Edom,
dizendo: Assim diz teu irmão Israel: Sabes todo o trabalho que nos
sobreveio, como nossos pais desceram ao Egito, e nós no Egito
habitamos muitos dias; e como os egípcios nos maltrataram, a nós e a
nossos pais; e clamamos ao Senhor, e ele ouviu a nossa voz, e
mandou um anjo, e nos tirou do Egito; e eis que estamos em Cades,
cidade na extremidade dos teus termos. Deixa-nos, pois,
passar pela tua terra; não passaremos pelo campo, nem pelas vinhas,
nem beberemos a água dos poços; iremos pela estrada real; não nos
desviaremos para a direita nem para a esquerda, até que passemos
pelos teus termos. Porém Edom lhe disse: Não passarás por
mim, para que eu não saia com a espada ao teu encontro. Então
os filhos de Israel lhe disseram: Subiremos pelo caminho aplanado, e
se eu e o meu gado bebermos das tuas águas, darei o preço delas; não
desejo alguma outra coisa, senão passar a pé. Porém ele
disse: Não passarás. E saiu-lhe Edom ao encontro com muita gente, e
com mão forte. Assim recusou Edom deixar passar a Israel pelo
seu termo; por isso Israel se desviou dele.

Então partiram de Cades; e os filhos de Israel, toda a
congregação, chegaram ao monte Hor. E falou o Senhor a Moisés
e a Arão no monte Hor, nos termos da terra de Edom, dizendo:
Arão será recolhido a seu povo, porque não entrará na terra
que tenho dado aos filhos de Israel, porquanto rebeldes fostes à
minha ordem, nas águas de Meribá. Toma a Arão e a Eleazar,
seu filho, e faze-os subir ao monte Hor. E despe a Arão as
suas vestes, e veste-as em Eleazar, seu filho, porque Arão será
recolhido, e morrerá ali. Fez, pois, Moisés como o Senhor lhe
ordenara; e subiram ao monte Hor perante os olhos de toda a
congregação. E Moisés despiu a Arão de suas vestes, e as
vestiu em Eleazar, seu filho; e morreu Arão ali sobre o cume do
monte; e desceram Moisés e Eleazar do monte. Vendo, pois,
toda a congregação que Arão era morto, choraram a Arão trinta dias,
toda a casa de Israel.

\medskip

\lettrine{21} Ouvindo o cananeu, rei de Arade, que habitava
para o lado sul, que Israel vinha pelo caminho dos espias, pelejou
contra Israel, e dele levou alguns prisioneiros. Então Israel
fez um voto ao Senhor, dizendo: Se de fato entregares este povo na
minha mão, destruirei totalmente as suas cidades. O Senhor,
pois, ouviu a voz de Israel, e lhe entregou os cananeus; e os
israelitas destruíram totalmente, a eles e às suas cidades; e o nome
daquele lugar chamou Hormá.

Então partiram do monte Hor, pelo caminho do Mar Vermelho, a
rodear a terra de Edom; porém a alma do povo angustiou-se naquele
caminho. E o povo falou contra Deus e contra Moisés: Por que nos
fizestes subir do Egito para que morrêssemos neste deserto? Pois
aqui nem pão nem água há; e a nossa alma tem fastio deste pão tão
vil. Então o Senhor mandou entre o povo serpentes ardentes, que
picaram o povo; e morreu muita gente em Israel. Por isso o povo
veio a Moisés, e disse: Havemos pecado porquanto temos falado contra
o Senhor e contra ti; ora ao Senhor que tire de nós estas serpentes.
Então Moisés orou pelo povo. E disse o Senhor a Moisés: Faze-te
uma serpente ardente, e põe-na sobre uma haste; e será que viverá
todo o que, tendo sido picado, olhar para ela. E Moisés fez uma
serpente de metal, e pô-la sobre uma haste; e sucedia que, picando
alguma serpente a alguém, quando esse olhava para a serpente de
metal, vivia.

Então os filhos de Israel partiram, e alojaram-se em Obote.
Depois partiram de Obote e alojaram-se nos outeiros de
Ije-Abarim, no deserto que está defronte de Moabe, ao nascente do
sol. Dali partiram, e alojaram-se junto ao ribeiro de Zerede.
E dali partiram e alojaram-se no lado de Arnom, que está no
deserto e sai dos termos dos amorreus; porque Arnom é o termo de
Moabe, entre Moabe e os amorreus. Por isso se diz no livro
das guerras do Senhor: O que fiz no Mar Vermelho e nos ribeiros de
Arnom, e à corrente dos ribeiros, que descendo para a
situação de Ar, se encosta aos termos de Moabe. E dali
partiram para Beer; este é o poço do qual o Senhor disse a Moisés:
Ajunta o povo e lhe darei água. Então Israel cantou este
cântico: Brota, ó poço! Cantai dele: Tu, poço, que cavaram os
príncipes, que escavaram os nobres do povo, e o legislador com os
seus bordões; e do deserto partiram para Mataná; e de Mataná
a Naaliel, e de Naaliel a Bamote. E de Bamote ao vale que
está no campo de Moabe, no cume de Pisga, e à vista do deserto.

Então Israel mandou mensageiros a Siom, rei dos amorreus,
dizendo: Deixa-me passar pela tua terra; não nos desviaremos
pelos campos nem pelas vinhas; as águas dos poços não beberemos;
iremos pela estrada real até que passemos os teus termos.
Porém Siom não deixou passar a Israel pelos seus termos;
antes Siom congregou todo o seu povo, e saiu ao encontro de Israel
no deserto, e veio a Jaza, e pelejou contra Israel. Mas
Israel o feriu ao fio da espada, e tomou a sua terra em possessão,
desde Arnom até Jaboque, até aos filhos de Amom; porquanto o termo
dos filhos de Amom era forte. Assim Israel tomou todas as
cidades; e habitou em todas elas, em Hesbom e em todas as suas
aldeias. Porque Hesbom era cidade de Siom, rei dos amorreus,
que tinha pelejado contra o precedente rei dos moabitas, e tinha
tomado da sua mão toda a sua terra até Arnom. Por isso dizem
os que falam em provérbios: Vinde a Hesbom; edifique-se e
estabeleça-se a cidade de Siom. Porque fogo saiu de Hesbom, e
uma chama da cidade de Siom; e consumiu a Ar dos moabitas, e os
senhores dos altos de Arnom. Ai de ti, Moabe! perdido és,
povo de Quemós! entregou seus filhos, que iam fugindo, e suas
filhas, como cativas a Siom, rei dos amorreus. E nós os
derribamos; Hesbom perdida é até Dibom, e os assolamos até Nofá, que
se estende até Medeba. Assim Israel habitou na terra dos
amorreus. Depois mandou Moisés espiar a Jazer, e tomaram as
suas aldeias, e daquela possessão lançaram os amorreus que estavam
ali. Então viraram-se, e subiram o caminho de Basã; e Ogue,
rei de Basã, saiu contra eles, ele e todo o seu povo, à peleja em
Edrei. E disse o Senhor a Moisés: Não o temas, porque eu o
tenho dado na tua mão, a ele, e a todo o seu povo, e a sua terra, e
far-lhe-ás como fizeste a Siom, rei dos amorreus, que habitava em
Hesbom. E de tal maneira o feriram, a ele e a seus filhos, e
a todo o seu povo, que nenhum deles escapou; e tomaram a sua terra
em possessão.

\medskip

\lettrine{22} Depois partiram os filhos de Israel, e
acamparam-se nas campinas de Moabe, além do Jordão na altura de
Jericó. Vendo, pois, Balaque, filho de Zipor, tudo o que Israel
fizera aos amorreus, Moabe temeu muito diante deste povo, porque
era numeroso; e Moabe andava angustiado por causa dos filhos de
Israel. Por isso Moabe disse aos anciãos dos midianitas: Agora
lamberá esta congregação tudo quanto houver ao redor de nós, como o
boi lambe a erva do campo. Naquele tempo Balaque, filho de Zipor,
era rei dos moabitas. Este enviou mensageiros a Balaão, filho de
Beor, a Petor, que está junto ao rio, na terra dos filhos do seu
povo, a chamá-lo, dizendo: Eis que um povo saiu do Egito; eis que
cobre a face da terra, e está parado defronte de mim. Vem, pois,
agora, rogo-te, amaldiçoa-me este povo, pois mais poderoso é do que
eu; talvez o poderei ferir e lançar fora da terra; porque eu sei
que, a quem tu abençoares será abençoado, e a quem tu amaldiçoares
será amaldiçoado. Então foram-se os anciãos dos moabitas e os
anciãos dos midianitas com o preço dos encantamentos nas suas mãos;
e chegaram a Balaão, e disseram-lhe as palavras de Balaque. E
ele lhes disse: Passai aqui esta noite, e vos trarei a resposta,
como o Senhor me falar; então os príncipes dos moabitas ficaram com
Balaão. E veio Deus a Balaão, e disse: Quem são estes homens que
estão contigo? E Balaão disse a Deus: Balaque, filho de
Zipor, rei dos moabitas, os enviou, dizendo: Eis que o povo
que saiu do Egito cobre a face da terra; vem agora, amaldiçoa-o;
porventura poderei pelejar contra ele e expulsá-lo. Então
disse Deus a Balaão: Não irás com eles, nem amaldiçoarás a este
povo, porquanto é bendito. Então Balaão levantou-se pela
manhã, e disse aos príncipes de Balaque: Ide à vossa terra, porque o
Senhor recusa deixar-me ir convosco. E levantaram-se os
príncipes dos moabitas, e vieram a Balaque, e disseram: Balaão
recusou vir conosco.

Porém Balaque tornou a enviar mais príncipes, mais honrados do
que aqueles. Os quais foram a Balaão, e lhe disseram: Assim
diz Balaque, filho de Zipor: Rogo-te que não te demores em vir a
mim. Porque grandemente te honrarei, e farei tudo o que me
disseres; vem pois, rogo-te, amaldiçoa-me este povo. Então
Balaão respondeu, e disse aos servos de Balaque: Ainda que Balaque
me desse a sua casa cheia de prata e de ouro, eu não poderia ir além
da ordem do Senhor meu Deus, para fazer coisa pequena ou grande;
agora, pois, rogo-vos que também aqui fiqueis esta noite,
para que eu saiba o que mais o Senhor me dirá. Veio, pois,
Deus a Balaão, de noite, e disse-lhe: Se aqueles homens te vieram
chamar, levanta-te, vai com eles; todavia, farás o que eu te disser.
Então Balaão levantou-se pela manhã, e albardou a sua
jumenta, e foi com os príncipes de Moabe.

E a ira de Deus acendeu-se, porque ele se ia; e o anjo do Senhor
pôs-se-lhe no caminho por adversário; e ele ia caminhando, montado
na sua jumenta, e dois de seus servos com ele. Viu, pois, a
jumenta o anjo do Senhor, que estava no caminho, com a sua espada
desembainhada na mão; pelo que desviou-se a jumenta do caminho, indo
pelo campo; então Balaão espancou a jumenta para fazê-la tornar ao
caminho. Mas o anjo do Senhor pôs-se numa vereda entre as
vinhas, havendo uma parede de um e de outro lado. Vendo,
pois, a jumenta, o anjo do Senhor, encostou-se contra a parede, e
apertou contra a parede o pé de Balaão; por isso tornou a
espancá-la. Então o anjo do Senhor passou mais adiante, e
pôs-se num lugar estreito, onde não havia caminho para se desviar
nem para a direita nem para a esquerda. E, vendo a jumenta o
anjo do Senhor, deitou-se debaixo de Balaão; e a ira de Balaão
acendeu-se, e espancou a jumenta com o bordão. Então o Senhor
abriu a boca da jumenta, a qual disse a Balaão: Que te fiz eu, que
me espancaste estas três vezes? E Balaão disse à jumenta: Por
que zombaste de mim; quem dera tivesse eu uma espada na mão, porque
agora te mataria. E a jumenta disse a Balaão: Porventura não
sou a tua jumenta, em que cavalgaste desde o tempo em que me tornei
tua até hoje? Acaso tem sido o meu costume fazer assim contigo? E
ele respondeu: Não. Então o Senhor abriu os olhos a Balaão, e
ele viu o anjo do Senhor, que estava no caminho e a sua espada
desembainhada na mão; pelo que inclinou a cabeça, e prostrou-se
sobre a sua face. Então o anjo do Senhor lhe disse: Por que
já três vezes espancaste a tua jumenta? Eis que eu saí para ser teu
adversário, porquanto o teu caminho é perverso diante de mim:
Porém a jumenta me viu, e já três vezes se desviou de diante
de mim; se ela não se desviasse de diante de mim, na verdade que eu
agora te haveria matado, e a ela deixaria com vida. Então
Balaão disse ao anjo do Senhor: Pequei, porque não sabia que estavas
neste caminho para te opores a mim; e agora, se parece mal aos teus
olhos, voltarei. E disse o anjo do Senhor a Balaão: Vai-te
com estes homens; mas somente a palavra que eu falar a ti, esta
falarás. Assim Balaão se foi com os príncipes de Balaque.

Ouvindo, pois, Balaque que Balaão vinha, saiu-lhe ao encontro até
à cidade de Moabe, que está no termo de Arnom, na extremidade do
termo dele. E Balaque disse a Balaão: Porventura não enviei
diligentemente a chamar-te? Por que não vieste a mim? Não posso eu
na verdade honrar-te? Então Balaão disse a Balaque: Eis que
eu tenho vindo a ti; porventura poderei eu agora de alguma forma
falar alguma coisa? A palavra que Deus puser na minha boca essa
falarei. E Balaão foi com Balaque, e chegaram a
Quiriate-Huzote. Então Balaque matou bois e ovelhas; e deles
enviou a Balaão e aos príncipes que estavam com ele. E
sucedeu que, pela manhã Balaque tomou a Balaão, e o fez subir aos
altos de Baal, e viu ele dali a última parte do povo.

\medskip

\lettrine{23} Então Balaão disse a Balaque: Edifica-me aqui
sete altares, e prepara-me aqui sete novilhos e sete carneiros.
Fez, pois, Balaque como Balaão dissera: e Balaque e Balaão
ofereceram um novilho e um carneiro sobre cada altar. Então
Balaão disse a Balaque: Fica-te junto do teu holocausto, e eu irei;
porventura o Senhor me sairá ao encontro, e o que me mostrar te
notificarei. Então foi a um lugar alto. E encontrando-se Deus
com Balaão, este lhe disse: Preparei sete altares, e ofereci um
novilho e um carneiro sobre cada altar. Então o Senhor pôs a
palavra na boca de Balaão, e disse: Torna-te para Balaque, e assim
falarás. E tornando para ele, eis que estava junto do seu
holocausto, ele e todos os príncipes dos moabitas. Então
proferiu a sua parábola, e disse: De Arã, me mandou trazer Balaque,
rei dos moabitas, das montanhas do oriente, dizendo: Vem,
amaldiçoa-me a Jacó; e vem, denuncia a Israel. Como amaldiçoarei
o que Deus não amaldiçoa? E como denunciarei, quando o Senhor não
denuncia? Porque do cume das penhas o vejo, e dos outeiros o
contemplo; eis que este povo habitará só, e entre as nações não será
contado. Quem contará o pó de Jacó e o número da quarta parte
de Israel? Que a minha alma morra da morte dos justos, e seja o meu
fim como o seu. Então disse Balaque a Balaão: Que me fizeste?
Chamei-te para amaldiçoar os meus inimigos, mas eis que inteiramente
os abençoaste. E ele respondeu, e disse: Porventura não terei
cuidado de falar o que o Senhor pôs na minha boca?

Então Balaque lhe disse: Rogo-te que venhas comigo a outro lugar,
de onde o verás; verás somente a última parte dele, mas a todo ele
não verás; e amaldiçoa-mo dali. Assim o levou consigo ao
campo de Zofim, ao cume de Pisga; e edificou sete altares, e
ofereceu um novilho e um carneiro sobre cada altar. Então
disse a Balaque: Fica aqui junto do teu holocausto, e eu irei ali ao
encontro do Senhor. E, encontrando-se o Senhor com Balaão,
pôs uma palavra na sua boca, e disse: Torna para Balaque, e assim
falarás. E, vindo a ele, eis que estava junto do holocausto,
e os príncipes dos moabitas com ele; disse-lhe pois Balaque: Que
coisa falou o Senhor? Então proferiu a sua parábola, e disse:
Levanta-te, Balaque, e ouve; inclina os teus ouvidos a mim, filho de
Zipor. Deus não é homem, para que minta; nem filho do homem,
para que se arrependa; porventura diria ele, e não o faria? Ou
falaria, e não o confirmaria? Eis que recebi mandado de
abençoar; pois ele tem abençoado, e eu não o posso revogar.
Não viu iniqüidade em Israel, nem contemplou maldade em Jacó;
o Senhor seu Deus é com ele, e no meio dele se ouve a aclamação de
um rei. Deus os tirou do Egito; as suas forças são como as do
boi selvagem. Pois contra Jacó não vale encantamento, nem
adivinhação contra Israel; neste tempo se dirá de Jacó e de Israel:
Que coisas Deus tem realizado! Eis que o povo se levantará
como leoa, e se erguerá como leão; não se deitará até que coma a
presa, e beba o sangue dos mortos. Então Balaque disse a
Balaão: Nem o amaldiçoarás, nem o abençoarás. Porém Balaão
respondeu, e disse a Balaque: Não te falei eu, dizendo: Tudo o que o
Senhor falar isso farei? Disse mais Balaque a Balaão: Ora
vem, e te levarei a outro lugar; porventura bem parecerá aos olhos
de Deus que dali mo amaldiçoes. Então Balaque levou Balaão
consigo ao cume de Peor, que dá para o lado do deserto.
Balaão disse a Balaque: Edifica-me aqui sete altares, e
prepara-me aqui sete novilhos e sete carneiros. Balaque,
pois, fez como dissera Balaão: e ofereceu um novilho e um carneiro
sobre cada altar.

\medskip

\lettrine{24} Vendo Balaão que bem parecia aos olhos do Senhor
que abençoasse a Israel, não se foi esta vez como antes ao encontro
dos encantamentos; mas voltou o seu rosto para o deserto. E,
levantando Balaão os seus olhos, e vendo a Israel, que estava
acampado segundo as suas tribos, veio sobre ele o Espírito de Deus.
E proferiu a sua parábola, e disse: Fala Balaão, filho de Beor,
e fala o homem de olhos abertos; fala aquele que ouviu as
palavras de Deus, o que vê a visão do Todo-Poderoso; que cai, e se
lhe abrem os olhos: Quão formosas são as tuas tendas, ó Jacó, as
tuas moradas, ó Israel! Como ribeiros se estendem, como jardins
à beira dos rios; como árvores de sândalo\footnote{1. Bot. Gênero de
árvores da família das santaláceas, de folhas simples, oblongas,
pequenas flores brancas, campanuladas, e que fornecem madeira
amarelada, odorífera, e óleo essencial, obtido da raiz. 2. Bot.
Qualquer espécie desse gênero, como, p. ex., a Santalum album,
originária da Índia e adjacências, e de cuja madeira, resistente e
aromática, se extrai um óleo de uso clássico, em perfumaria, para o
fabrico do sândalo. 3. Bot. Qualquer espécime desse gênero. 4.
Essência perfumada e balsâmica, extraída do tronco e das raízes do
sândalo, e utilizada em preparados farmacêuticos. 5. Perfume
fabricado com essa essência.} o Senhor os plantou, como cedros junto
às águas; de seus baldes manarão águas, e a sua semente estará
em muitas águas; e o seu rei se erguerá mais do que Agague, e o seu
reino será exaltado. Deus o tirou do Egito; as suas forças são
como as do boi selvagem; consumirá as nações, seus inimigos, e
quebrará seus ossos, e com as suas setas os atravessará.
Encurvou-se, deitou-se como leão, e como leoa; quem o
despertará? benditos os que te abençoarem, e malditos os que te
amaldiçoarem.

Então a ira de Balaque se acendeu contra Balaão, e bateu ele as
suas palmas; e Balaque disse a Balaão: Para amaldiçoar os meus
inimigos te tenho chamado; porém agora já três vezes os abençoaste
inteiramente. Agora, pois, foge para o teu lugar; eu tinha
dito que te honraria grandemente; mas eis que o Senhor te privou
desta honra. Então Balaão disse a Balaque: Não falei eu
também aos teus mensageiros, que me enviaste, dizendo: Ainda
que Balaque me desse a sua casa cheia de prata e ouro, não poderia
ir além da ordem do Senhor, fazendo bem ou mal de meu próprio
coração; o que o Senhor falar, isso falarei eu? Agora, pois,
eis que me vou ao meu povo; vem, avisar-te-ei do que este povo fará
ao teu povo nos últimos dias.

Então proferiu a sua parábola, e disse: Fala Balaão, filho de
Beor, e fala o homem de olhos abertos; fala aquele que ouviu
as palavras de Deus, e o que sabe a ciência do Altíssimo; o que viu
a visão do Todo-Poderoso, que cai, e se lhe abrem os olhos.
Vê-lo-ei, mas não agora, contemplá-lo-ei, mas não de perto;
uma estrela procederá de Jacó e um cetro subirá de Israel, que
ferirá os termos dos moabitas, e destruirá todos os filhos de Sete.
E Edom será uma possessão, e Seir, seus inimigos, também será
uma possessão; pois Israel fará proezas. E dominará um de
Jacó, e matará os que restam das cidades. E vendo os
amalequitas, proferiu a sua parábola, e disse: Amaleque é a primeira
das nações; porém o seu fim será a destruição. E vendo os
quenitas, proferiu a sua parábola, e disse: Firme está a tua
habitação, e puseste o teu ninho na penha. Todavia o quenita
será consumido, até que Assur te leve por prisioneiro. E,
proferindo ainda a sua parábola, disse: Ai, quem viverá, quando Deus
fizer isto? E as naus virão das costas de Quitim e afligirão
a Assur; também afligirão a Éber; que também será para destruição.
Então Balaão levantou-se, e se foi, e voltou ao seu lugar, e
também Balaque se foi pelo seu caminho.

\medskip

\lettrine{25} E Israel deteve-se em Sitim e o povo começou a
prostituir-se com as filhas dos moabitas. Elas convidaram o povo
aos sacrifícios dos seus deuses; e o povo comeu, e inclinou-se aos
seus deuses. Juntando-se, pois, Israel a Baal-Peor, a ira do
Senhor se acendeu contra Israel. Disse o Senhor a Moisés: Toma
todos os cabeças do povo, e enforca-os ao Senhor diante do sol, e o
ardor da ira do Senhor se retirará de Israel. Então Moisés disse
aos juízes de Israel: Cada um mate os seus homens que se juntaram a
Baal-Peor.

E eis que veio um homem dos filhos de Israel, e trouxe a seus
irmãos uma midianita, à vista de Moisés, e à vista de toda a
congregação dos filhos de Israel, chorando eles diante da tenda da
congregação. Vendo isso Finéias, filho de Eleazar, o filho de
Arão, sacerdote, se levantou do meio da congregação, e tomou uma
lança na sua mão; e foi após o homem israelita até à tenda, e os
atravessou a ambos, ao homem israelita e à mulher, pelo ventre;
então a praga cessou de sobre os filhos de Israel. E os que
morreram daquela praga foram vinte e quatro mil. Então o
Senhor falou a Moisés, dizendo: Finéias, filho de Eleazar, o
filho de Arão, sacerdote, desviou a minha ira de sobre os filhos de
Israel, pois foi zeloso com o meu zelo no meio deles; de modo que,
no meu zelo, não consumi os filhos de Israel. Portanto dize:
Eis que lhe dou a minha aliança de paz; e ele, e a sua
descendência depois dele, terá a aliança do sacerdócio perpétuo,
porquanto teve zelo pelo seu Deus, e fez expiação pelos filhos de
Israel. E o nome do israelita, que foi morto com a midianita,
era Zimri, filho de Salu, príncipe da casa paterna dos simeonitas.
E o nome da mulher midianita morta era Cosbi, filha de Zur,
cabeça do povo da casa paterna entre os midianitas.

Falou mais o Senhor a Moisés, dizendo: Afligireis os
midianitas e os ferireis, porque eles vos afligiram a vós com
os seus enganos com que vos enganaram no caso de Peor, e no caso de
Cosbi, filha do príncipe dos midianitas, irmã deles, que foi morta
no dia da praga no caso de Peor.

\medskip

\lettrine{26} Aconteceu, pois, que, depois daquela praga,
falou o Senhor a Moisés, e a Eleazar, filho de Arão, o sacerdote,
dizendo: Tomai a soma de toda a congregação dos filhos de
Israel, da idade de vinte anos para cima, segundo as casas de seus
pais; todos os que em Israel podem sair à guerra. Falaram-lhes,
pois, Moisés e Eleazar, o sacerdote, nas campinas de Moabe, junto ao
Jordão na altura de Jericó, dizendo: Conta o povo da idade de
vinte anos para cima, como o Senhor ordenara a Moisés e aos filhos
de Israel, que saíram do Egito.

Rúben, o primogênito de Israel; os filhos de Rúben: de Enoque, a
família dos enoquitas; de Palu, a família dos paluítas; de
Hezrom, a família dos hezronitas; de Carmi, a família dos carmitas.
Estas são as famílias dos rubenitas; e os que foram deles
contados foram quarenta e três mil e setecentos e trinta. E os
filhos de Palu, Eliabe; e os filhos de Eliabe, Nemuel, e Datã, e
Abirão: estes, Datã e Abirão, foram os do conselho da congregação,
que contenderam contra Moisés e contra Arão no grupo de Coré, quando
rebelaram contra o Senhor; e a terra abriu a sua boca, e os
tragou com Coré, quando morreu aquele grupo; quando o fogo consumiu
duzentos e cinqüenta homens, os quais serviram de advertência.
Mas os filhos de Coré não morreram. Os filhos de
Simeão, segundo as suas famílias: de Nemuel, a família dos
nemuelitas; de Jamim, a família dos jaminitas; de Jaquim, a família
dos jaquinitas; de Zerá, a família dos zeraítas; de Saul, a
família dos saulitas. Estas são as famílias dos simeonitas,
vinte e dois mil e duzentos. Os filhos de Gade, segundo as
suas gerações; de Zefom, a família dos zefonitas; de Hagi, a família
dos hagitas; de Suni, a família dos sunitas; de Ozni, a
família dos oznitas; de Eri, a família dos eritas; de Arode,
a família dos aroditas; de Areli, a família dos arelitas.
Estas são as famílias dos filhos de Gade, segundo os que
foram deles contados, quarenta mil e quinhentos. Os filhos de
Judá, Er e Onã; mas Er e Onã morreram na terra de Canaã.
Assim os filhos de Judá foram segundo as suas famílias; de
Selá, a família dos selanitas; de Perez, a família dos perezitas; de
Zerá, a família dos zeraítas. E os filhos de Perez foram: de
Hezrom, a família dos hezronitas; de Hamul, a família dos hamulitas.
Estas são as famílias de Judá, segundo os que foram deles
contados, setenta e seis mil e quinhentos. Os filhos de
Issacar, segundo as suas famílias, foram: de Tola, a família dos
tolaítas; de Puva, a família dos puvitas; de Jasube, a
família dos jasubitas; de Sinrom, a família dos sinronitas.
Estas são as famílias de Issacar, segundo os que foram deles
contados, sessenta e quatro mil e trezentos. Os filhos de
Zebulom, segundo as suas famílias, foram: de Serede, a família dos
sereditas; de Elom, a família dos elonitas; de Jaleel, a família dos
jaleelitas. Estas são as famílias dos zebulonitas, segundo os
que foram deles contados, sessenta mil e quinhentos. Os
filhos de José segundo as suas famílias, foram Manassés e Efraim.
Os filhos de Manassés foram; de Maquir, a família dos
maquiritas; e Maquir gerou a Gileade; de Gileade, a família dos
gileaditas. Estes são os filhos de Gileade; de Jezer, a
família dos jezeritas; de Heleque, a família dos helequitas;
e de Asriel, a família dos asrielitas; e de Siquém, a família
dos siquemitas; e de Semida, a família dos semidaítas; e de
Hefer, a família dos heferitas. Porém, Zelofeade, filho de
Hefer, não tinha filhos, senão filhas; e os nomes das filhas de
Zelofeade foram Maalá, Noa, Hogla, Milca e Tirza. Estas são
as famílias de Manassés; e os que foram deles contados, foram
cinqüenta e dois mil e setecentos. Estes são os filhos de
Efraim, segundo as suas famílias: de Sutela, a família dos
sutelaítas; de Bequer, a família dos bequeritas; de Taã, a família
dos taanitas. E estes são os filhos de Sutela: de Erã, a
família dos eranitas. Estas são as famílias dos filhos de
Efraim, segundo os que foram deles contados, trinta e dois mil e
quinhentos; estes são os filhos de José, segundo as suas famílias.
Os filhos de Benjamim, segundo as suas famílias: de Belá, a
família dos belaítas; de Asbel, a família dos asbelitas; de Airã, a
família dos airamitas; de Sufã, a família dos sufamitas; de
Hufã, a família dos hufamitas. E os filhos de Belá foram Arde
e Naamã; de Arde, a família dos arditas; de Naamã, a família dos
naamanitas. Estes são os filhos de Benjamim, segundo as suas
famílias; e os que foram deles contados, foram quarenta e cinco mil
e seiscentos. Estes são os filhos de Dã, segundo as suas
famílias; de Suã, a família dos suamitas. Estas são as famílias de
Dã, segundo as suas famílias. Todas as famílias dos suamitas,
segundo os que foram deles contados, foram sessenta e quatro mil e
quatrocentos. Os filhos de Aser, segundo as suas famílias,
foram: de Imna, a família dos imnaítas; de Isvi, a família dos
isvitas; de Berias, a família dos beriítas. Dos filhos de
Berias, foram; de Héber, a família dos heberitas; de Malquiel, a
família dos malquielitas. E o nome da filha de Aser foi Sera.
Estas são as famílias dos filhos de Aser, segundo os que
foram deles contados, cinqüenta e três mil e quatrocentos. Os
filhos de Naftali, segundo as suas famílias; de Jazeel, a família
dos jazeelitas; de Guni, a família dos gunitas; de Jezer, a
família dos jezeritas; de Silém, a família dos silemitas. Estas
são as famílias de Naftali, segundo as suas famílias; e os que foram
deles contados, foram quarenta e cinco mil e quatrocentos. 
Estes são os que foram contados dos filhos de Israel, seiscentos e
um mil e setecentos e trinta.

E falou o Senhor a Moisés, dizendo: A estes se repartirá a
terra em herança, segundo o número dos nomes. Aos muitos
aumentarás a sua herança, e aos poucos diminuirás a sua herança; a
cada um se dará a sua herança, segundo os que foram deles contados.
Todavia a terra se repartirá por sortes; segundo os nomes das
tribos de seus pais a herdarão. Segundo sair a sorte, se
repartirá a herança deles entre as tribos de muitos e as de poucos.
E estes são os que foram contados dos levitas, segundo as
suas famílias: de Gérson, a família dos gersonitas; de Coate, a
família dos coatitas; de Merari, a família dos meraritas.
Estas são as famílias de Levi: a família dos libnitas, a
família dos hebronitas, a família dos malitas, a família dos
musitas, a família dos coreítas. E Coate gerou a Anrão. E o
nome da mulher de Anrão era Joquebede, filha de Levi, a qual nasceu
a Levi no Egito; e de Anrão ela teve Arão, e Moisés, e Miriã, irmã
deles. E a Arão nasceram Nadabe, Abiú, Eleazar, e Itamar.
Porém Nadabe e Abiú morreram quando trouxeram fogo estranho
perante o Senhor. E os que deles foram contados eram vinte e
três mil, todo o homem da idade de um mês para cima; porque estes
não foram contados entre os filhos de Israel, porquanto não lhes foi
dada herança entre os filhos de Israel.

Estes são os que foram contados por Moisés e Eleazar, o
sacerdote, que contaram os filhos de Israel nas campinas de Moabe,
junto ao Jordão na direção de Jericó. E entre estes nenhum
houve dos que foram contados por Moisés e Arão, o sacerdote, quando
contaram aos filhos de Israel no deserto de Sinai. Porque o
Senhor dissera deles que certamente morreriam no deserto; e nenhum
deles ficou senão Calebe, filho de Jefoné, e Josué, filho de Num.

\medskip

\lettrine{27} E Chegaram as filhas de Zelofeade, filho de
Hefer, filho de Gileade, filho de Maquir, filho de Manassés, entre
as famílias de Manassés, filho de José; e estes são os nomes delas:
Maalá, Noa, Hogla, Milca, e Tirza; e apresentaram-se diante de
Moisés, e diante de Eleazar, o sacerdote, e diante dos príncipes e
de toda a congregação, à porta da tenda da congregação, dizendo:
Nosso pai morreu no deserto, e não estava entre os que se
congregaram contra o Senhor no grupo de Coré; mas morreu no seu
próprio pecado, e não teve filhos. Por que se tiraria o nome de
nosso pai do meio da sua família, porquanto não teve filhos? Dá-nos
possessão entre os irmãos de nosso pai. E Moisés levou a causa
delas perante o Senhor. E falou o Senhor a Moisés, dizendo:
As filhas de Zelofeade falam o que é justo; certamente lhes
darás possessão de herança entre os irmãos de seu pai; e a herança
de seu pai farás passar a elas. E falarás aos filhos de Israel,
dizendo: Quando alguém morrer e não tiver filho, então fareis passar
a sua herança à sua filha. E, se não tiver filha, então a sua
herança dareis a seus irmãos. Porém, se não tiver irmãos,
então dareis a sua herança aos irmãos de seu pai. Se também
seu pai não tiver irmãos, então dareis a sua herança a seu parente,
àquele que lhe for o mais chegado da sua família, para que a possua;
isto aos filhos de Israel será por estatuto de direito, como o
Senhor ordenou a Moisés.

Depois disse o Senhor a Moisés: Sobe a este monte de Abarim, e vê
a terra que tenho dado aos filhos de Israel. E, tendo-a
visto, então serás recolhido ao teu povo, assim como foi recolhido
teu irmão Arão; porquanto, no deserto de Zim, na contenda da
congregação, fostes rebeldes ao meu mandado de me santificar nas
águas diante dos seus olhos (estas são as águas de Meribá de Cades,
no deserto de Zim).

Então falou Moisés ao Senhor, dizendo: O Senhor, Deus dos
espíritos de toda a carne, ponha um homem sobre esta congregação,
que saia diante deles, e que entre diante deles, e que os
faça sair, e que os faça entrar; para que a congregação do Senhor
não seja como ovelhas que não têm pastor. Então disse o
Senhor a Moisés: Toma a Josué, filho de Num, homem em quem há o
Espírito, e impõe a tua mão sobre ele. E apresenta-o perante
Eleazar, o sacerdote, e perante toda a congregação, e dá-lhe as tuas
ordens na presença deles. E põe sobre ele da tua glória, para
que lhe obedeça toda a congregação dos filhos de Israel. E
apresentar-se-á perante Eleazar, o sacerdote, o qual por ele
consultará, segundo o juízo de Urim, perante o Senhor; conforme a
sua palavra sairão, e conforme a sua palavra entrarão, ele e todos
os filhos de Israel com ele, e toda a congregação. E fez
Moisés como o Senhor lhe ordenara; porque tomou a Josué, e
apresentou-o perante Eleazar, o sacerdote, e perante toda a
congregação; e sobre ele impôs as suas mãos, e lhe deu
ordens, como o Senhor falara por intermédio de Moisés.

\medskip

\lettrine{28} Falou mais o Senhor a Moisés, dizendo: Dá
ordem aos filhos de Israel, e dize-lhes: Da minha oferta, do meu
alimento para as minhas ofertas queimadas, do meu cheiro suave,
tereis cuidado, para me oferecê-las ao seu tempo determinado. E
dir-lhes-ás: Esta é a oferta queimada que oferecereis ao Senhor:
dois cordeiros de um ano, sem defeito, cada dia, em contínuo
holocausto; um cordeiro sacrificarás pela manhã, e o outro
cordeiro sacrificarás à tarde; e a décima parte de um efa de
flor de farinha em oferta de alimentos, misturada com a quarta parte
de um him de azeite batido. Este é o holocausto contínuo,
instituído no monte Sinai, em cheiro suave, oferta queimada ao
Senhor. E a sua libação será a quarta parte de um him para um
cordeiro; no santuário, oferecerás a libação de bebida forte ao
Senhor. E o outro cordeiro sacrificarás à tarde, como a oferta
de alimentos da manhã, e como a sua libação o oferecerás em oferta
queimada de cheiro suave ao Senhor.

Porém, no dia de sábado, oferecerás dois cordeiros de um ano, sem
defeito, e duas décimas de flor de farinha, misturada com azeite, em
oferta de alimentos, com a sua libação. Holocausto é de cada
sábado, além do holocausto contínuo, e a sua libação. E nos
princípios dos vossos meses oferecereis, em holocausto ao Senhor,
dois novilhos e um carneiro, sete cordeiros de um ano, sem defeito;
e três décimas de flor de farinha misturada com azeite, em
oferta de alimentos, para um novilho; e duas décimas de flor de
farinha misturada com azeite, em oferta de alimentos, para um
carneiro. E uma décima de flor de farinha misturada com
azeite em oferta de alimentos, para um cordeiro; holocausto é de
cheiro suave, oferta queimada ao Senhor. E as suas libações
serão a metade de um him de vinho para um novilho, e a terça parte
de um him para um carneiro, e a quarta parte de um him para um
cordeiro; este é o holocausto da lua nova de cada mês, segundo os
meses do ano. Também um bode para expiação do pecado ao
Senhor, além do holocausto contínuo, com a sua libação se oferecerá.

Porém no mês primeiro, aos catorze dias do mês, é a páscoa do
Senhor. E aos quinze dias do mesmo mês haverá festa; sete
dias se comerão pães ázimos. No primeiro dia haverá santa
convocação; nenhum trabalho servil fareis; mas oferecereis
oferta queimada em holocausto ao Senhor, dois novilhos e um
carneiro, e sete cordeiros de um ano; eles serão sem defeito.
E a sua oferta de alimentos será de flor de farinha misturada
com azeite; oferecereis três décimas para um novilho, e duas décimas
para um carneiro. Para cada um dos sete cordeiros oferecereis
uma décima; e um bode para expiação do pecado, para fazer
expiação por vós. Estas coisas oferecereis, além do
holocausto da manhã, que é o holocausto contínuo. Segundo
este modo, cada dia oferecereis, por sete dias, o alimento da oferta
queimada em cheiro suave ao Senhor; além do holocausto contínuo se
oferecerá isto com a sua libação. E no sétimo dia tereis
santa convocação; nenhum trabalho servil fareis.
Semelhantemente, tereis santa convocação no dia das
primícias, quando oferecerdes oferta nova de alimentos ao Senhor,
segundo as vossas semanas; nenhum trabalho servil fareis.
Então oferecereis ao Senhor por holocausto, em cheiro suave,
dois novilhos, um carneiro e sete cordeiros de um ano; e a
sua oferta de alimentos de flor de farinha misturada com azeite:
três décimas para um novilho, duas décimas para um carneiro;
e uma décima, para cada um dos sete cordeiros; um bode
para fazer expiação por vós. 31 Além do holocausto contínuo, e a sua
oferta de alimentos, os oferecereis (ser-vos-ão eles sem defeito)
com as suas libações.

\medskip

\lettrine{29} Semelhantemente, tereis santa convocação no
sétimo mês, no primeiro dia do mês; nenhum trabalho servil fareis;
será para vós dia de sonido de trombetas. Então por holocausto,
em cheiro suave ao Senhor, oferecereis um novilho, um carneiro e
sete cordeiros de um ano, sem defeito. E pela sua oferta de
alimentos de flor de farinha misturada com azeite, três décimas para
o novilho, e duas décimas para o carneiro, e uma décima para
cada um dos sete cordeiros. E um bode para expiação do pecado,
para fazer expiação por vós; além do holocausto do mês, e a sua
oferta de alimentos, e o holocausto contínuo, e a sua oferta de
alimentos, com as suas libações, segundo o seu estatuto, em cheiro
suave, oferta queimada ao Senhor. E no dia dez deste sétimo mês
tereis santa convocação, e afligireis as vossas almas; nenhum
trabalho fareis. Mas por holocausto, em cheiro suave ao Senhor,
oferecereis um novilho, um carneiro e sete cordeiros de um ano; eles
serão sem defeito. E, pela sua oferta de alimentos de flor de
farinha misturada com azeite, três décimas para o novilho, duas
décimas para o carneiro, e uma décima para cada um dos sete
cordeiros; um bode para expiação do pecado, além da expiação
do pecado pelas propiciações, e do holocausto contínuo, e da sua
oferta de alimentos com as suas libações.

Semelhantemente, aos quinze dias deste sétimo mês tereis santa
convocação; nenhum trabalho servil fareis; mas sete dias celebrareis
festa ao Senhor. E, por holocausto em oferta queimada, de
cheiro suave ao Senhor, oferecereis treze novilhos, dois carneiros e
catorze cordeiros de um ano; todos eles sem defeito. E, pela
sua oferta de alimentos de flor de farinha misturada com azeite,
três décimas para cada um dos treze novilhos, duas décimas para cada
carneiro, entre os dois carneiros; e uma décima para cada um
dos catorze cordeiros; e um bode para expiação do pecado,
além do holocausto contínuo, a sua oferta de alimentos e a sua
libação; depois, no segundo dia, doze novilhos, dois
carneiros, catorze cordeiros de um ano, sem defeito; e a sua
oferta de alimentos e as suas libações para os novilhos, para os
carneiros e para os cordeiros, conforme o seu número, segundo o
estatuto; e um bode para expiação do pecado, além do
holocausto contínuo, da sua oferta de alimentos e das suas libações.
E, no terceiro dia, onze novilhos, dois carneiros, catorze
cordeiros de um ano, sem defeito; e as suas ofertas de
alimentos, e as suas libações para os novilhos, para os carneiros e
para os cordeiros, conforme o seu número, segundo o estatuto;
e um bode para expiação do pecado, além do holocausto
contínuo, e da sua oferta de alimentos e da sua libação. E,
no quarto dia, dez novilhos, dois carneiros, catorze cordeiros de um
ano, sem defeito; a sua oferta de alimentos, e as suas
libações para os novilhos, para os carneiros, e para os cordeiros,
conforme o seu número, segundo o estatuto; e um bode para
expiação do pecado, além do holocausto contínuo, da sua oferta de
alimentos e da sua libação. E, no quinto dia, nove novilhos,
dois carneiros e catorze cordeiros de um ano, sem defeito. E
a sua oferta de alimentos, e as suas libações para os novilhos, para
os carneiros e para os cordeiros, conforme o seu número, segundo o
estatuto; e um bode para expiação do pecado além do
holocausto contínuo, e da sua oferta de alimentos e da sua libação.
E, no sexto dia, oito novilhos, dois carneiros, catorze
cordeiros de um ano, sem defeito; e a sua oferta de
alimentos, e as suas libações para os bezerros, para os carneiros e
para os cordeiros, conforme o seu número, segundo o estatuto;
e um bode para expiação do pecado, além do holocausto
contínuo, da sua oferta de alimentos e da sua libação. E, no
sétimo dia, sete novilhos, dois carneiros, catorze cordeiros de um
ano, sem defeito. E a sua oferta de alimentos, e as suas
libações para os novilhos, para os carneiros e para os cordeiros,
conforme o seu número, segundo o seu estatuto, e um bode para
expiação do pecado, além do holocausto contínuo, da sua oferta de
alimentos e da sua libação. No oitavo dia tereis dia de
solenidade; nenhum trabalho servil fareis; e por holocausto
em oferta queimada de cheiro suave ao Senhor oferecereis um novilho,
um carneiro, sete cordeiros de um ano, sem defeito; a sua
oferta de alimentos e as suas libações para o novilho, para o
carneiro e para os cordeiros, conforme o seu número, segundo o
estatuto. E um bode para expiação do pecado, além do
holocausto contínuo, e da sua oferta de alimentos e da sua libação.
Estas coisas fareis ao Senhor nas vossas solenidades além dos
vossos votos, e das vossas ofertas voluntárias, com os vossos
holocaustos, e com as vossas ofertas de alimentos, e com as vossas
libações, e com as vossas ofertas pacíficas. E falou Moisés
aos filhos de Israel, conforme a tudo o que o Senhor ordenara a
Moisés.

\medskip

\lettrine{30} E falou Moisés aos cabeças das tribos dos filhos
de Israel, dizendo: Esta é a palavra que o Senhor tem ordenado.
Quando um homem fizer voto ao Senhor, ou fizer juramento,
ligando a sua alma com obrigação, não violará a sua palavra: segundo
tudo o que saiu da sua boca, fará.

Também quando uma mulher, na sua mocidade, estando ainda na casa
de seu pai, fizer voto ao Senhor, e com obrigação se ligar, e
seu pai ouvir o seu voto e a sua obrigação, com que ligou a sua
alma; e seu pai se calar para com ela, todos os seus votos serão
válidos; e toda a obrigação com que ligou a sua alma, será válida.
Mas se seu pai lhe tolher no dia que tal ouvir, todos os seus
votos e as suas obrigações com que tiver ligado a sua alma, não
serão válidos; mas o Senhor lhe perdoará, porquanto seu pai lhos
tolheu. E se ela for casada, e for obrigada a alguns votos, ou à
pronunciação dos seus lábios, com que tiver ligado a sua alma; e
seu marido o ouvir, e se calar para com ela no dia em que o ouvir,
os seus votos serão válidos; e as suas obrigações com que ligou a
sua alma, serão válidas. Mas se seu marido lhe tolher no dia em
que o ouvir, e anular o seu voto a que estava obrigada, como também
a pronunciação dos seus lábios, com que ligou a sua alma; o Senhor
lhe perdoará. No tocante ao voto da viúva, ou da repudiada, tudo
com que ligar a sua alma, sobre ela será válido. Porém se fez
voto na casa de seu marido, ou ligou a sua alma com obrigação de
juramento; e seu marido o ouviu, e se calou para com ela, e
não lho tolheu, todos os seus votos serão válidos, e toda a
obrigação, com que ligou a sua alma, será válida. Porém se
seu marido lhos anulou no dia em que os ouviu; tudo quanto saiu dos
seus lábios, quer dos seus votos, quer da obrigação da sua alma, não
será válido; seu marido lhos anulou, e o Senhor lhe perdoará.
Todo o voto, e todo o juramento de obrigação, para humilhar a
alma, seu marido o confirmará, ou anulará. Porém se seu
marido, de dia em dia, se calar inteiramente para com ela, então
confirma todos os seus votos e todas as suas obrigações, que
estiverem sobre ela; confirmado lhos tem, porquanto se calou para
com ela no dia em que o ouviu. Porém se de todo lhos anular
depois que o ouviu, então ele levará a iniqüidade dela. Estes
são os estatutos que o Senhor ordenou a Moisés entre o marido e sua
mulher; entre o pai e sua filha, na sua mocidade, em casa de seu
pai.

\medskip

\lettrine{31} E falou o Senhor a Moisés, dizendo: Vinga os
filhos de Israel dos midianitas; depois recolhido serás ao teu povo.
Falou, pois, Moisés ao povo, dizendo: Armem-se alguns de vós
para a guerra, e saiam contra os midianitas, para fazerem a vingança
do Senhor contra eles. Mil de cada tribo, entre todas as tribos
de Israel, enviareis à guerra. Assim foram dados, dos milhares
de Israel, mil de cada tribo; doze mil armados para a peleja. E
Moisés os mandou à guerra, mil de cada tribo, e com eles Finéias,
filho de Eleazar, o sacerdote, com os vasos do santuário, e com as
trombetas do alarido na sua mão.

E pelejaram contra os midianitas, como o Senhor ordenara a Moisés;
e mataram a todos os homens. Mataram também, além dos que já
haviam sido mortos, os reis dos midianitas: a Evi, e a Requém, e a
Zur, e a Hur, e a Reba, cinco reis dos midianitas; também a Balaão,
filho de Beor, mataram à espada. Porém, os filhos de Israel
levaram presas as mulheres dos midianitas e as suas crianças; também
levaram todos os seus animais e todo o seu gado, e todos os seus
bens. E queimaram a fogo todas as suas cidades com todas as
suas habitações e todos os seus acampamentos. E tomaram todo
o despojo e toda a presa de homens e de animais. E trouxeram
a Moisés e a Eleazar, o sacerdote, e à congregação dos filhos de
Israel, os cativos, e a presa, e o despojo, para o arraial, nas
campinas de Moabe, que estão junto ao Jordão, na altura de Jericó.

Porém Moisés e Eleazar, o sacerdote, e todos os príncipes da
congregação, saíram a recebê-los fora do arraial. E
indignou-se Moisés grandemente contra os oficiais do exército,
capitães dos milhares e capitães das centenas, que vinham do serviço
da guerra. E Moisés disse-lhes: Deixastes viver todas as
mulheres? Eis que estas foram as que, por conselho de Balaão,
deram ocasião aos filhos de Israel de transgredir contra o Senhor no
caso de Peor; por isso houve aquela praga entre a congregação do
Senhor. Agora, pois, matai todo o homem entre as crianças, e
matai toda a mulher que conheceu algum homem, deitando-se com ele.
Porém, todas as meninas que não conheceram algum homem,
deitando-se com ele, deixai-as viver para vós. E alojai-vos
sete dias fora do arraial; qualquer que tiver matado alguma pessoa,
e qualquer que tiver tocado algum morto, ao terceiro dia, e ao
sétimo dia vos purificareis, a vós e a vossos cativos. Também
purificareis toda a roupa, e toda a obra de peles, e toda a obra de
pêlos de cabras, e todo o utensílio de madeira. E disse
Eleazar, o sacerdote, aos homens da guerra, que foram à peleja: Este
é o estatuto da lei que o Senhor ordenou a Moisés. Contudo o
ouro, e a prata, o cobre, o ferro, o estanho, e o chumbo,
toda a coisa que pode resistir ao fogo, fareis passar pelo
fogo, para que fique limpa, todavia se purificará com a água da
purificação; mas tudo que não pode resistir ao fogo, fareis passar
pela água. Também lavareis as vossas roupas ao sétimo dia,
para que fiqueis limpos; e depois entrareis no arraial.

Falou mais o Senhor a Moisés, dizendo: Faze a soma da
presa que foi tomada, de homens e de animais, tu e Eleazar, o
sacerdote, e os cabeças das casas dos pais da congregação, e
divide a presa em duas metades, entre os que se armaram para a
peleja, e saíram à guerra, e toda a congregação. Então para o
Senhor tomarás o tributo dos homens de guerra, que saíram a esta
peleja, de cada quinhentos uma alma, dos homens, e dos bois, e dos
jumentos e das ovelhas. Da sua metade o tomareis, e o dareis
ao sacerdote Eleazar, para a oferta alçada do Senhor. Mas, da
metade dos filhos de Israel, tomarás um de cada cinqüenta, um dos
homens, dos bois, dos jumentos, e das ovelhas, e de todos os
animais; e os darás aos levitas que têm cuidado da guarda do
tabernáculo do Senhor. E fizeram Moisés e Eleazar, o
sacerdote, como o Senhor ordenara a Moisés. Foi a presa,
restante do despojo que tomaram os homens de guerra, seiscentas e
setenta e cinco mil ovelhas; e setenta e dois mil bois;
e sessenta e um mil jumentos; e, das mulheres que não
conheceram homem algum, deitando-se com ele, todas as almas foram
trinta e duas mil. E a metade, que era a porção dos que
saíram à guerra, foi em número de trezentas e trinta e sete mil e
quinhentas ovelhas. E das ovelhas, o tributo para o Senhor
foi de seiscentas e setenta e cinco. E foram os bois trinta e
seis mil; e o seu tributo para o Senhor setenta e dois. E
foram os jumentos trinta mil e quinhentos; e o seu tributo para o
Senhor sessenta e um. E houve de pessoas dezesseis mil; e o
seu tributo para o Senhor trinta e duas pessoas. E deu Moisés
a Eleazar, o sacerdote, o tributo da oferta alçada do Senhor, como o
Senhor ordenara a Moisés. E da metade dos filhos de Israel
que Moisés separara da dos homens que pelejaram,

metade para a congregação foi, das ovelhas, trezentas e trinta
e sete mil e quinhentas; e dos bois trinta e seis mil;
e dos jumentos trinta mil e quinhentos; e das pessoas
humanas dezesseis mil). Desta metade dos filhos de Israel,
Moisés tomou um de cada cinqüenta, de homens e de animais, e os deu
aos levitas, que tinham cuidado da guarda do tabernáculo do Senhor,
como o Senhor ordenara a Moisés. Então chegaram-se a Moisés
os oficiais que estavam sobre os milhares do exército, os chefes de
mil e os chefes de cem; e disseram a Moisés: Teus servos
tomaram a soma dos homens de guerra que estiveram sob as nossas
ordens; e não falta nenhum de nós. Por isso trouxemos uma
oferta ao Senhor, cada um o que achou, objetos de ouro, cadeias, ou
manilhas, anéis, arrecadas\footnote{Arrecada: brinco, em geral de
metal e em forma de argola.}, e colares, para fazer expiação pelas
nossas almas perante o Senhor. Assim Moisés e Eleazar, o
sacerdote, receberam deles o ouro, sendo todos os objetos bem
trabalhados. E foi todo o ouro da oferta alçada, que
ofereceram ao Senhor, dezesseis mil e setecentos e cinqüenta siclos,
dos chefes de mil e dos chefes de cem (pois cada um dos
homens de guerra, tinha tomado presa para si). Receberam,
pois, Moisés e Eleazar, o sacerdote, o ouro dos chefes de mil e dos
chefes de cem, e o levaram à tenda da congregação, por memorial para
os filhos de Israel perante o Senhor.

\medskip

\lettrine{32} E os filhos de Rúben e os filhos de Gade tinham
gado em grande quantidade; e viram a terra de Jazer, e a terra de
Gileade, e eis que o lugar era lugar de gado. Vieram, pois, os
filhos de Gade, e os filhos de Rúben e falaram a Moisés e a Eleazar,
o sacerdote, e aos chefes da congregação, dizendo: Atarote, e
Dibom, e Jazer, e Ninra, e Hesbom, e Eleale, e Sebã, e Nebo, e Beom,
a terra que o Senhor feriu diante da congregação de Israel, é
terra para gado, e os teus servos têm gado. Disseram mais: Se
achamos graça aos teus olhos, dê-se esta terra aos teus servos em
possessão; e não nos faças passar o Jordão. Porém Moisés disse
aos filhos de Gade e aos filhos de Rúben: Irão vossos irmãos à
peleja, e ficareis vós aqui? Por que, pois, desencorajais o
coração dos filhos de Israel, para que não passem à terra que o
Senhor lhes tem dado? Assim fizeram vossos pais, quando os
mandei de Cades-Barnéia, a ver esta terra. Chegando eles até ao
vale de Escol, e vendo esta terra, desencorajaram o coração dos
filhos de Israel, para que não entrassem na terra que o Senhor lhes
tinha dado. Então a ira do Senhor se acendeu naquele mesmo
dia, e jurou dizendo: Que os homens, que subiram do Egito, de
vinte anos para cima, não verão a terra que jurei a Abraão, a
Isaque, e a Jacó! porquanto não perseveraram em seguir-me;
exceto Calebe, filho de Jefoné o quenezeu, e Josué, filho de
Num, porquanto perseveraram em seguir ao Senhor. Assim se
acendeu a ira do Senhor contra Israel, e fê-los andar errantes pelo
deserto quarenta anos até que se consumiu toda aquela geração, que
fizera mal aos olhos do Senhor. E eis que vós, uma geração de
homens pecadores, vos levantastes em lugar de vossos pais, para
ainda mais acrescentar o furor da ira do Senhor contra Israel.
Se vós vos virardes de segui-lo, também ele os deixará de
novo no deserto, e destruireis a todo este povo.

Então chegaram-se a ele, e disseram: Edificaremos currais aqui
para o nosso gado, e cidades para as nossas crianças; porém
nós nos armaremos, apressando-nos adiante dos de Israel, até que os
levemos ao seu lugar; e ficarão as nossas crianças nas cidades
fortes por causa dos moradores da terra. Não voltaremos para
nossas casas, até que os filhos de Israel estejam de posse, cada um,
da sua herança. Porque não herdaremos com eles além do
Jordão, nem mais adiante; porquanto nós já temos a nossa herança
aquém do Jordão, ao oriente. Então Moisés lhes disse: Se isto
fizerdes assim, se vos armardes à guerra perante o Senhor; e
cada um de vós, armado, passar o Jordão perante o Senhor, até que
haja lançado fora os seus inimigos de diante dele, e a terra
esteja subjugada perante o Senhor; então voltareis e sereis
inculpáveis perante o Senhor e perante Israel; e esta terra vos será
por possessão perante o Senhor; e se não fizerdes assim, eis
que pecastes contra o Senhor; e sabei que o vosso pecado vos há de
achar. Edificai cidades para as vossas crianças, e currais
para as vossas ovelhas; e fazei o que saiu da vossa boca.
Então falaram os filhos de Gade, e os filhos de Rúben a
Moisés, dizendo: Como ordena meu senhor, assim farão teus servos.
As nossas crianças, as nossas mulheres, o nosso gado, e todos
os nossos animais estarão aí nas cidades de Gileade. Mas os
teus servos passarão, cada um armado para a guerra, a pelejar
perante o Senhor, como tem falado o meu Senhor.

Então Moisés deu ordem acerca deles a Eleazar, o sacerdote, e a
Josué filho de Num, e aos cabeças das casas dos pais das tribos dos
filhos de Israel. E disse-lhes Moisés: Se os filhos de Gade e
os filhos de Rúben passarem convosco o Jordão, armado cada um para a
guerra, perante o Senhor, e a terra estiver subjugada diante de vós,
em possessão lhes dareis a terra de Gileade. Porém, se não
passarem armados convosco, terão possessões entre vós, na terra de
Canaã. E responderam os filhos de Gade e os filhos de Rúben,
dizendo: O que o Senhor falou a teus servos, isso faremos.
Nós passaremos, armados, perante o Senhor, à terra de Canaã,
e teremos a possessão de nossa herança aquém do Jordão. Assim
deu-lhes Moisés, aos filhos de Gade, e aos filhos de Rúben, e à meia
tribo de Manassés, filho de José, o reino de Siom, rei dos amorreus,
e o reino de Ogue, rei de Basã; a terra com as suas cidades nos seus
termos, e as cidades ao seu redor. E os filhos de Gade
edificaram a Dibom, e Atarote, e Aroer; e Atarote-Sofã, e
Jazer, e Jogbeá; e Bete-Ninra, e Bete-Harã, cidades fortes; e
currais de ovelhas. E os filhos de Rúben edificaram a Hesbom,
e Eleale, e Quiriataim; e Nebo, e Baal-Meom, mudando-lhes o
nome, e Sibma; e os nomes das cidades que edificaram chamaram por
outros nomes. E os filhos de Maquir, filho de Manassés,
foram-se para Gileade, e a tomaram; e daquela possessão expulsaram
os amorreus que estavam nela. Assim Moisés deu Gileade a
Maquir, filho de Manassés, o qual habitou nela. E foi Jair,
filho de Manassés, e tomou as suas aldeias; e chamou-as Havote-Jair.
E foi Nobá, e tomou a Quenate com as suas aldeias; e chamou-a
Nobá, segundo o seu próprio nome.

\medskip

\lettrine{33} Estas são as jornadas dos filhos de Israel, que
saíram da terra do Egito, segundo os seus exércitos, sob a direção
de Moisés e Arão. E escreveu Moisés as suas saídas, segundo as
suas jornadas, conforme ao mandado do Senhor; e estas são as suas
jornadas, segundo as suas saídas. Partiram, pois, de Ramessés no
primeiro mês, no dia quinze do primeiro mês; no dia seguinte da
páscoa saíram os filhos de Israel por alta mão, aos olhos de todos
os egípcios, enquanto os egípcios enterravam os que o Senhor
tinha ferido entre eles, a todo o primogênito, e havendo o Senhor
executado juízos também contra os seus deuses. Partiram, pois,
os filhos de Israel de Ramessés, e acamparam-se em Sucote. E
partiram de Sucote, e acamparam-se em Etã, que está no fim do
deserto. E partiram de Etã, e voltaram a Pi-Hairote, que está
defronte de Baal-Zefom, e acamparam-se diante de Migdol. E
partiram de Pi-Hairote, e passaram pelo meio do mar ao deserto, e
andaram caminho de três dias no deserto de Etã, e acamparam-se em
Mara. E partiram de Mara, e vieram a Elim, e em Elim havia doze
fontes de águas e setenta palmeiras, e acamparam-se ali. E
partiram de Elim, e acamparam-se junto ao Mar Vermelho. E
partiram do Mar Vermelho, e acamparam-se no deserto de Sim. E
partiram do deserto de Sim, e acamparam-se em Dofca. E
partiram de Dofca, e acamparam-se em Alus. E partiram de
Alus, e acamparam-se em Refidim; porém não havia ali água, para que
o povo bebesse. Partiram, pois, de Refidim, e acamparam-se no
deserto de Sinai. E partiram do deserto de Sinai, e
acamparam-se em Quibrote-Taavá. E partiram de Quibrote-Taavá,
e acamparam-se em Hazerote. E partiram de Hazerote, e
acamparam-se em Ritmá. E partiram de Ritmá, e acamparam-se em
Rimom-Perez. E partiram de Rimom-Perez, e acamparam-se em
Libna. E partiram de Libna, e acamparam-se em Rissa. E
partiram de Rissa, e acamparam-se em Queelata. E partiram de
Queelata, e acamparam-se no monte de Séfer. E partiram do
monte de Séfer, e acamparam-se em Harada. E partiram de
Harada, e acamparam-se em Maquelote. E partiram de Maquelote,
e acamparam-se em Taate. E partiram de Taate, e acamparam-se
em Tara. E partiram de Tara, e acamparam-se em Mitca.
E partiram de Mitca, e acamparam-se em Hasmona. E
partiram de Hasmona, e acamparam-se em Moserote. E partiram
de Moserote, e acamparam-se em Bene-Jaacã. E partiram de
Bene-Jaacã, e acamparam-se em Hor-Hagidgade. E partiram de
Hor-Hagidgade, e acamparam-se em Jotbatá. E partiram de
Jotbatá, e acamparam-se em Abrona. E partiram de Abrona, e
acamparam-se em Ezion-Geber. E partiram de Ezion-Geber, e
acamparam-se no deserto de Zim, que é Cades. E partiram de
Cades, e acamparam-se no monte Hor, no fim da terra de Edom.
Então Arão, o sacerdote, subiu ao monte Hor, conforme ao
mandado do Senhor; e morreu ali no quinto mês do ano quadragésimo da
saída dos filhos de Israel da terra do Egito, no primeiro dia do
mês. E era Arão da idade de cento e vinte e três anos, quando
morreu no monte Hor. E ouviu o cananeu, rei de Harade, que
habitava o sul na terra de Canaã, que chegavam os filhos de Israel.
E partiram do monte Hor, e acamparam-se em Zalmona. E
partiram de Zalmona, e acamparam-se em Punom. E partiram de
Punom, e acamparam-se em Obote. E partiram de Obote, e
acamparam-se em Ije-Abarim, no termo de Moabe. E partiram de
Ije-Abarim, e acamparam-se em Dibom-Gade. E partiram de
Dibom-Gade, e acamparam-se em Almom-Diblataim. E partiram de
Almom-Diblataim, e acamparam-se nos montes de Abarim, defronte de
Nebo. E partiram dos montes de Abarim, e acamparam-se nas
campinas de Moabe, junto ao Jordão, na direção de Jericó. E
acamparam-se junto ao Jordão, desde Bete-Jesimote até Abel-Sitim,
nas campinas de Moabe.

E falou o Senhor a Moisés, nas campinas de Moabe junto ao Jordão
na direção de Jericó, dizendo: Fala aos filhos de Israel, e
dize-lhes: Quando houverdes passado o Jordão para a terra de Canaã,
lançareis fora todos os moradores da terra de diante de vós,
e destruireis todas as suas pinturas; também destruireis todas as
suas imagens de fundição, e desfareis todos os seus altos; e
tomareis a terra em possessão, e nela habitareis; porquanto vos
tenho dado esta terra, para possuí-la. E por sortes herdareis
a terra, segundo as vossas famílias; aos muitos multiplicareis a
herança, e aos poucos diminuireis a herança; conforme a sorte sair a
alguém, ali a possuirá; segundo as tribos de vossos pais recebereis
as heranças. Mas se não lançardes fora os moradores da terra
de diante de vós, então os que deixardes ficar vos serão por
espinhos nos vossos olhos, e por aguilhões nas vossas virilhas, e
apertar-vos-ão na terra em que habitardes, e será que farei a
vós como pensei fazer-lhes a eles.

\medskip

\lettrine{34} Falou mais o Senhor a Moisés, dizendo: Dá
ordem aos filhos de Israel, e dize-lhes: Quando entrardes na terra
de Canaã, esta há de ser a terra que vos cairá em herança; a terra
de Canaã, segundo os seus termos. O lado do sul vos será desde o
deserto de Zim até aos termos de Edom; e o termo do sul vos será
desde a extremidade do Mar Salgado para o lado do oriente. E
este limite vos irá rodeando do sul para a subida de Acrabim, e
passará até Zim; e as suas saídas serão do sul a Cades-Barnéia; e
sairá a Hazar-Adar, e passará a Azmom; rodeará mais este limite
de Azmom até ao rio do Egito; e as suas saídas serão para o lado do
mar. Quanto ao limite do ocidente, o Mar Grande vos será por
limite; este vos será o limite do ocidente. E este vos será o
termo do norte: desde o Mar Grande marcareis até ao monte Hor.
Desde o monte Hor marcareis até à entrada de Hamate; e as saídas
deste termo serão até Zedade. E este limite seguirá até Zifrom,
e as suas saídas serão em Hazar-Enã; este vos será o termo do norte.
E por limite do lado do oriente marcareis de Hazar-Enã até
Sefã. E este limite descerá desde Sefã até Ribla, para o lado
do oriente de Aim; depois descerá este termo, e irá ao longo da
borda do mar de Quinerete para o lado do oriente. Descerá
também este limite ao longo do Jordão, e as suas saídas serão no Mar
Salgado; esta vos será a terra, segundo os seus limites ao redor.
E Moisés deu ordem aos filhos de Israel, dizendo: Esta é a
terra que herdareis por sorte, a qual o Senhor mandou dar às nove
tribos e à meia tribo. Porque a tribo dos filhos dos
rubenitas, segundo a casa de seus pais, e a tribo dos filhos dos
gaditas, segundo a casa de seus pais, já receberam; também a meia
tribo de Manassés recebeu a sua herança. Já duas tribos e
meia tribo receberam a sua herança aquém do Jordão, na direção de
Jericó, do lado do oriente, ao nascente.

Falou mais o Senhor a Moisés, dizendo: Estes são os nomes
dos homens que vos repartirão a terra por herança: Eleazar, o
sacerdote, e Josué, filho de Num. Tomareis mais de cada tribo
um príncipe, para repartir a terra em herança. E estes são os
nomes dos homens: Da tribo de Judá, Calebe, filho de Jefoné;
e, da tribo dos filhos de Simeão, Samuel, filho de Amiúde;
da tribo de Benjamim, Elidade, filho de Quislom; e, da
tribo dos filhos de Dã, o príncipe Buqui, filho de Jogli; dos
filhos de José, da tribo dos filhos de Manassés, o príncipe Haniel,
filho de Éfode; e, da tribo dos filhos de Efraim, o príncipe
Quemuel, filho de Siftã; e, da tribo dos filhos de Zebulom, o
príncipe Elizafã, filho de Parnaque; e, da tribo dos filhos
de Issacar, o príncipe Paltiel, filho de Azã; e, da tribo dos
filhos de Aser, o príncipe Aiúde, filho de Selomi; e, da
tribo dos filhos de Naftali, o príncipe Pedael, filho de Amiúde.
Estes são aqueles a quem o Senhor ordenou\footnote{SBTB:
interpõe ``,'' após ``ordenou''. AV: These are they whom the LORD
commanded to divide the inheritance unto the children of Israel in
the land of Canaan. RA: A estes o SENHOR ordenou que repartissem a
herança pelos filhos de Israel, na terra de Canaã. RC - 1969: Estes
são aqueles a quem o Senhor ordenou, que repartissem a herança pelos
filhos de Israel na terra de Canaã. RC: Estes são aqueles a quem o
SENHOR ordenou que repartissem a herança pelos filhos de Israel na
terra de Canaã.} que repartissem as heranças aos filhos de Israel na
terra de Canaã.

\medskip

\lettrine{35} E falou o Senhor a Moisés nas campinas de Moabe,
junto ao Jordão na direção de Jericó, dizendo: Dá ordem aos
filhos de Israel que, da herança da sua possessão, dêem cidades aos
levitas, em que habitem; e também aos levitas dareis
arrabaldes\footnote{Cercanias de uma cidade ou povoação; subúrbio.}
ao redor delas. E terão estas cidades para habitá-las; porém os
seus arrabaldes serão para o seu gado, e para os seus bens, e para
todos os seus animais. E os arrabaldes das cidades, que dareis
aos levitas, desde o muro da cidade para fora, serão de mil côvados
em redor. E de fora da cidade, do lado do oriente, medireis dois
mil côvados, e do lado do sul, dois mil côvados, e do lado do
ocidente dois mil côvados, e do lado do norte dois mil côvados, e a
cidade no meio; isto terão por arrabaldes das cidades. Das
cidades, pois, que dareis aos levitas, haverá seis cidades de
refúgio, as quais dareis para que o homicida ali se acolha; e, além
destas, lhes dareis quarenta e duas cidades. Todas as cidades
que dareis aos levitas serão quarenta e oito cidades, juntamente com
os seus arrabaldes. E quanto às cidades que derdes da herança
dos filhos de Israel, do que tiver muito tomareis muito, e do que
tiver pouco tomareis pouco; cada um dará das suas cidades aos
levitas, segundo a herança que herdar.

Falou mais o Senhor a Moisés, dizendo: Fala aos filhos de
Israel, e dize-lhes: Quando passardes o Jordão à terra de Canaã,
fazei com que vos estejam à mão cidades que vos sirvam de
cidades de refúgio, para que ali se acolha o homicida que ferir a
alguma alma por engano. E estas cidades vos serão por refúgio
do vingador do sangue; para que o homicida não morra, até que seja
apresentado à congregação para julgamento. E das cidades que
derdes haverá seis cidades de refúgio para vós. Três destas
cidades dareis além do Jordão, e três destas cidades dareis na terra
de Canaã; cidades de refúgio serão. Serão por refúgio estas
seis cidades para os filhos de Israel, e para o estrangeiro, e para
o que se hospedar no meio deles, para que ali se acolha aquele que
matar a alguém por engano. Porém, se o ferir com instrumento
de ferro e morrer, homicida é; certamente o homicida morrerá.
Ou, se lhe ferir com uma pedrada, de que possa morrer, e
morrer, homicida é; certamente o homicida morrerá. Ou, se o
ferir com instrumento de pau que tiver na mão, de que possa morrer,
e ele morrer, homicida é; certamente morrerá o homicida. O
vingador do sangue matará o homicida; encontrando-o, matá-lo-á.
Se também o empurrar com ódio, ou com mal intento lançar
contra ele alguma coisa, e morrer; ou por inimizade o ferir
com a sua mão, e morrer, certamente morrerá aquele que o ferir;
homicida é; o vingador do sangue, encontrando o homicida, o matará.
Porém, se o empurrar subitamente, sem inimizade, ou contra
ele lançar algum instrumento sem intenção; ou, sobre ele
deixar cair alguma pedra sem o ver, de que possa morrer, e ele
morrer, sem que fosse seu inimigo nem procurasse o seu mal;
então a congregação julgará entre aquele que feriu e o
vingador do sangue, segundo estas leis. E a congregação
livrará o homicida da mão do vingador do sangue, e a congregação o
fará voltar à cidade do seu refúgio, onde se tinha acolhido; e ali
ficará até à morte do sumo sacerdote, a quem ungiram com o santo
óleo. Porém, se de alguma maneira o homicida sair dos limites
da cidade de refúgio, onde se tinha acolhido, e o vingador do
sangue o achar fora dos limites da cidade de seu refúgio, e o matar,
não será culpado do sangue. Pois o homicida deverá ficar na
cidade do seu refúgio, até à morte do sumo sacerdote; mas, depois da
morte do sumo sacerdote, o homicida voltará à terra da sua
possessão. E estas coisas vos serão por estatuto de direito
às vossas gerações, em todas as vossas habitações. Todo
aquele que matar alguma pessoa, conforme depoimento de testemunhas,
será morto; mas uma só testemunha não testemunhará contra alguém,
para que morra. E não recebereis resgate pela vida do
homicida que é culpado de morte; pois certamente morrerá.
Também não tomareis resgate por aquele que se acolher à sua
cidade de refúgio, para tornar a habitar na terra, até à morte do
sumo sacerdote. Assim não profanareis a terra em que estais;
porque o sangue faz profanar a terra; e nenhuma expiação se fará
pela terra por causa do sangue que nela se derramar, senão com o
sangue daquele que o derramou. Não contaminareis pois a terra
na qual vós habitais, no meio da qual eu habito; pois eu, o Senhor,
habito no meio dos filhos de Israel.

\medskip

\lettrine{36} E chegaram os chefes dos pais da família de
Gileade, filho de Maquir, filho de Manassés, das famílias dos filhos
de José, e falaram diante de Moisés, e diante dos príncipes, chefes
dos pais dos filhos de Israel, e disseram: O Senhor mandou a meu
senhor que, por sorte, desse esta terra em herança aos filhos de
Israel; e a meu senhor foi ordenado pelo Senhor, que a herança do
nosso irmão Zelofeade se desse às suas filhas. E, casando-se
elas com alguns dos filhos das outras tribos dos filhos de Israel,
então a sua herança será diminuída da herança de nossos pais, e
acrescentada à herança da tribo a que vierem a pertencer; assim se
tirará da sorte da nossa herança. Vindo também o ano do jubileu
dos filhos de Israel, a sua herança será acrescentada à herança da
tribo daqueles com que se casarem; assim a sua herança será tirada
da herança da tribo de nossos pais.

Então Moisés deu ordem aos filhos de Israel, segundo o mandado do
Senhor, dizendo: A tribo dos filhos de José fala o que é justo.
Isto é o que o Senhor mandou acerca das filhas de Zelofeade,
dizendo: Sejam por mulheres a quem bem parecer aos seus olhos,
contanto que se casem na família da tribo de seu pai. Assim a
herança dos filhos de Israel não passará de tribo em tribo; pois os
filhos de Israel se chegarão cada um à herança da tribo de seus
pais. E qualquer filha que herdar alguma herança das tribos dos
filhos de Israel se casará com alguém da família da tribo de seu
pai; para que os filhos de Israel possuam cada um a herança de seus
pais. Assim a herança não passará de uma tribo a outra; pois as
tribos dos filhos de Israel se chegarão cada uma à sua herança.
Como o Senhor ordenara a Moisés, assim fizeram as filhas de
Zelofeade. Pois Maalá, Tirza, Hogla, Milca e Noa, filhas de
Zelofeade, se casaram com os filhos de seus tios. E elas
casaram-se nas famílias dos filhos de Manassés, filho de José; assim
a sua herança ficou na tribo da família de seu pai. Estes são
os mandamentos e os juízos que mandou o Senhor através de Moisés aos
filhos de Israel nas campinas de Moabe, junto ao Jordão, na direção
de Jericó.

