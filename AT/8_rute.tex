\addchap{Rute}

\lettrine{1} E sucedeu que, nos dias em que os juízes
julgavam, houve uma fome na terra; por isso um homem de Belém de
Judá saiu a peregrinar nos campos de Moabe, ele e sua mulher, e seus
dois filhos; era o nome deste homem Elimeleque, e o de sua
mulher Noemi, e os de seus dois filhos Malom e Quiliom, efrateus, de
Belém de Judá; e chegaram aos campos de Moabe, e ficaram ali. E
morreu Elimeleque, marido de Noemi; e ficou ela com os seus dois
filhos, os quais tomaram para si mulheres moabitas; e era o nome
de uma Orfa, e o da outra Rute; e ficaram ali quase dez anos. E
morreram também ambos, Malom e Quiliom, ficando assim a mulher
desamparada dos seus dois filhos e de seu marido.

Então se levantou ela com as suas noras, e voltou dos campos de
Moabe, porquanto na terra de Moabe ouviu que o Senhor tinha visitado
o seu povo, dando-lhe pão. Por isso saiu do lugar onde estivera,
e as suas noras com ela. E, indo elas caminhando, para voltarem para
a terra de Judá, disse Noemi às suas noras: Ide, voltai cada uma
à casa de sua mãe; e o Senhor use convosco de benevolência, como vós
usastes com os falecidos e comigo. O Senhor vos dê que acheis
descanso cada uma em casa de seu marido. E, beijando-as ela,
levantaram a sua voz e choraram. E disseram-lhe: Certamente
voltaremos contigo ao teu povo. Porém Noemi disse: Voltai,
minhas filhas. Por que iríeis comigo? Tenho eu ainda no meu ventre
mais filhos, para que vos sejam por maridos? Voltai, filhas
minhas, ide-vos embora, que já mui velha sou para ter marido; ainda
quando eu dissesse: Tenho esperança, ou ainda que esta noite tivesse
marido e ainda tivesse filhos, esperá-los-íeis até que
viessem a ser grandes? Deter-vos-íeis por eles, sem tomardes marido?
Não, filhas minhas, que mais amargo me é a mim do que a vós mesmas;
porquanto a mão do Senhor se descarregou contra mim. Então
levantaram a sua voz, e tornaram a chorar; e Orfa beijou a sua
sogra, porém Rute se apegou a ela. Por isso disse Noemi: Eis
que voltou tua cunhada ao seu povo e aos seus deuses; volta tu
também após tua cunhada. Disse, porém, Rute: Não me instes
para que te abandone, e deixe de seguir-te; porque aonde quer que tu
fores irei eu, e onde quer que pousares, ali pousarei eu; o teu povo
é o meu povo, o teu Deus é o meu Deus; onde quer que morreres
morrerei eu, e ali serei sepultada. Faça-me assim o Senhor, e outro
tanto, se outra coisa que não seja a morte me separar de ti.
Vendo Noemi, que de todo estava resolvida a ir com ela,
deixou de lhe falar.

Assim, pois, foram-se ambas, até que chegaram a Belém; e sucedeu
que, entrando elas em Belém, toda a cidade se comoveu por causa
delas, e diziam: Não é esta Noemi? Porém ela lhes dizia: Não
me chameis Noemi; chamai-me Mara; porque grande amargura me tem dado
o Todo-Poderoso. Cheia parti, porém vazia o Senhor me fez
tornar; por que pois me chamareis Noemi? O Senhor testifica contra
mim, e o Todo-Poderoso me tem feito mal. Assim Noemi voltou,
e com ela Rute a moabita, sua nora, que veio dos campos de Moabe; e
chegaram a Belém no princípio da colheita das cevadas.

\medskip

\lettrine{2} E tinha Noemi um parente de seu marido, homem
valente e poderoso, da família de Elimeleque; e era o seu nome Boaz.
E Rute, a moabita, disse a Noemi: Deixa-me ir ao campo, e
apanharei espigas atrás daquele em cujos olhos eu achar graça. E ela
disse: Vai, minha filha. Foi, pois, e chegou, e apanhava espigas
no campo após os segadores; e caiu-lhe em sorte uma parte do campo
de Boaz, que era da família de Elimeleque.

E eis que Boaz veio de Belém, e disse aos segadores: O Senhor seja
convosco. E disseram-lhe eles: O Senhor te abençoe. Depois disse
Boaz a seu moço, que estava posto sobre os segadores: De quem é esta
moça? E respondeu o moço, que estava posto sobre os segadores, e
disse: Esta é a moça moabita que voltou com Noemi dos campos de
Moabe. Disse-me ela: Deixa-me colher espigas, e ajuntá-las entre
as gavelas\footnote{Gavala: Feixe de espigas; paveia, fascículo. }
após os segadores. Assim ela veio, e desde pela manhã está aqui até
agora, a não ser um pouco que esteve sentada em casa. Então
disse Boaz a Rute: Ouves, filha minha; não vás colher em outro
campo, nem tampouco passes daqui; porém aqui ficarás com as minhas
moças. Os teus olhos estarão atentos no campo que segarem, e
irás após elas; não dei ordem aos moços, que não te molestem? Tendo
tu sede, vai aos vasos, e bebe do que os moços tirarem. Então
ela caiu sobre o seu rosto, e se inclinou à terra; e disse-lhe: Por
que achei graça em teus olhos, para que faças caso de mim, sendo eu
uma estrangeira? E respondeu Boaz, e disse-lhe: Bem se me
contou quanto fizeste à tua sogra, depois da morte de teu marido; e
deixaste a teu pai e a tua mãe, e a terra onde nasceste, e vieste
para um povo que antes não conheceste. O Senhor retribua o
teu feito; e te seja concedido pleno galardão da parte do Senhor
Deus de Israel, sob cujas asas te vieste abrigar. E disse
ela: Ache eu graça em teus olhos, senhor meu, pois me consolaste, e
falaste ao coração da tua serva, não sendo eu ainda como uma das
tuas criadas. E, sendo já hora de comer, disse-lhe Boaz:
Achega-te aqui, e come do pão, e molha o teu bocado no vinagre. E
ela se assentou ao lado dos segadores, e ele lhe deu do trigo
tostado, e comeu, e se fartou, e ainda lhe sobejou. E,
levantando-se ela a colher, Boaz deu ordem aos seus moços, dizendo:
Até entre as gavelas deixai-a colher, e não a censureis. E
deixai cair alguns punhados, e deixai-os ficar, para que os colha, e
não a repreendais.

E esteve ela apanhando naquele campo até à tarde; e debulhou o
que apanhou, e foi quase um efa de cevada. E tomou-a, e veio
à cidade; e viu sua sogra o que tinha apanhado; também tirou, e
deu-lhe o que sobejara depois de fartar-se. Então disse-lhe
sua sogra: Onde colheste hoje, e onde trabalhaste? Bendito seja
aquele que te reconheceu. E relatou à sua sogra com quem tinha
trabalhado, e disse: O nome do homem com quem hoje trabalhei é Boaz.
Então Noemi disse à sua nora: Bendito seja ele do Senhor, que
ainda não tem deixado a sua beneficência nem para com os vivos nem
para com os mortos. Disse-lhe mais Noemi: Este homem é nosso parente
chegado, e um dentre os nossos remidores. E disse Rute, a
moabita: Também ainda me disse: Com os moços que tenho te ajuntarás,
até que acabem toda a sega que tenho. E disse Noemi a sua
nora: Melhor é, filha minha, que saias com as suas moças, para que
noutro campo não te encontrem. Assim, ajuntou-se com as moças
de Boaz, para colher até que a sega das cevadas e dos trigos se
acabou; e ficou com a sua sogra.

\medskip

\lettrine{3} E disse-lhe Noemi, sua sogra: Minha filha, não
hei de buscar descanso, para que fiques bem? Ora, pois, não é
Boaz, com cujas moças estiveste, de nossa parentela? Eis que esta
noite padejará\footnote{Padejar: Revolver com a pá. Atirar (o pão)
ao ar com a pá, a fim de limpá-lo na eira.} a cevada na eira.
Lava-te, pois, e unge-te, e veste os teus vestidos, e desce à
eira; porém não te dês a conhecer ao homem, até que tenha acabado de
comer e beber. E há de ser que, quando ele se deitar, notarás o
lugar em que se deitar; então entrarás, e descobrir-lhe-ás os pés, e
te deitarás, e ele te fará saber o que deves fazer. E ela lhe
disse: Tudo quanto me disseres, farei.

Então foi para a eira, e fez conforme a tudo quanto sua sogra lhe
tinha ordenado. Havendo, pois, Boaz comido e bebido, e estando
já o seu coração alegre, veio deitar-se ao pé de um monte de grãos;
então veio ela de mansinho, e lhe descobriu os pés, e se deitou.
E sucedeu que, pela meia noite, o homem estremeceu, e se voltou;
e eis que uma mulher jazia a seus pés. E disse ele: Quem és tu?
E ela disse: Sou Rute, tua serva; estende pois tua capa sobre a tua
serva, porque tu és o remidor. E disse ele: Bendita sejas tu
do Senhor, minha filha; melhor fizeste esta tua última benevolência
do que a primeira, pois após nenhum dos jovens foste, quer pobre
quer rico. Agora, pois, minha filha, não temas; tudo quanto
disseste te farei, pois toda a cidade do meu povo sabe que és mulher
virtuosa. Porém agora é verdade que eu sou remidor, mas ainda
outro remidor há mais chegado do que eu. Fica-te aqui esta
noite, e será que, pela manhã, se ele te redimir, bem está, que te
redima; porém, se não quiser te redimir, vive o Senhor, que eu te
redimirei. Deita-te aqui até amanhã.

Ficou-se, pois, deitada a seus pés até pela manhã, e levantou-se
antes que pudesse um conhecer o outro, porquanto disse: Não se saiba
que alguma mulher veio à eira. Disse mais: Dá-me a capa que
tens sobre ti, e segura-a. E ela a segurou; e ele mediu seis medidas
de cevada, e lhas pôs em cima; então foi para a cidade. E foi
à sua sogra, que lhe disse: Como foi, minha filha? E ela lhe contou
tudo quanto aquele homem lhe fizera. Disse mais: Estas seis
medidas de cevada me deu, porque me disse: Não vás vazia à tua
sogra. Então disse ela: Espera, minha filha, até que saibas
como irá o caso, porque aquele homem não descansará até que conclua
hoje este negócio.

\medskip

\lettrine{4} E Boaz subiu à porta, e assentou-se ali; e eis
que o remidor de que Boaz tinha falado ia passando, e disse-lhe: Ó
fulano, vem cá, assenta-te aqui. E desviou-se para ali, e
assentou-se. Então tomou dez homens dos anciãos da cidade, e
disse: Assentai-vos aqui. E assentaram-se. Então disse ao
remidor: Aquela parte da terra que foi de Elimeleque, nosso irmão,
Noemi, que tornou da terra dos moabitas, está vendendo. E eu
resolvi informar-te disso e dizer-te: Compra-a diante dos
habitantes, e diante dos anciãos do meu povo; se a hás de redimir,
redime-a, e se não houver de redimir, declara-mo, para que o saiba,
pois outro não há senão tu que a redima, e eu depois de ti. Então
disse ele: Eu a redimirei. Disse porém Boaz: No dia em que
comprares a terra da mão de Noemi, também a comprarás da mão de
Rute, a moabita, mulher do falecido, para suscitar o nome do
falecido sobre a sua herança. Então disse o remidor: Para mim
não a poderei redimir, para que não prejudique a minha herança; toma
para ti o meu direito de remissão, porque eu não a poderei redimir.
Havia, pois, já de muito tempo este costume em Israel, quanto a
remissão e permuta, para confirmar todo o negócio; o homem
descalçava o sapato e o dava ao seu próximo; e isto era por
testemunho em Israel. Disse, pois, o remidor a Boaz: Toma-a para
ti. E descalçou o sapato.

Então Boaz disse aos anciãos e a todo o povo: Sois hoje
testemunhas de que tomei tudo quanto foi de Elimeleque, e de
Quiliom, e de Malom, da mão de Noemi, e de que também tomo
por mulher a Rute, a moabita, que foi mulher de Malom, para suscitar
o nome do falecido sobre a sua herança, para que o nome do falecido
não seja desarraigado dentre seus irmãos e da porta do seu lugar;
disto sois hoje testemunhas. E todo o povo que estava na
porta, e os anciãos, disseram: Somos testemunhas; o Senhor faça a
esta mulher, que entra na tua casa, como a Raquel e como a Lia, que
ambas edificaram a casa de Israel; e porta-te valorosamente em
Efrata, e faze-te nome afamado em Belém. E seja a tua casa
como a casa de Perez (que Tamar deu à luz a Judá), pela descendência
que o Senhor te der desta moça.

Assim tomou Boaz a Rute, e ela lhe foi por mulher; e ele a
possuiu, e o Senhor lhe fez conceber, e deu à luz um filho.
Então as mulheres disseram a Noemi: Bendito seja o Senhor,
que não deixou hoje de te dar remidor, e seja o seu nome afamado em
Israel. Ele te será por restaurador da alma, e nutrirá a tua
velhice, pois tua nora, que te ama, o deu à luz, e ela te é melhor
do que sete filhos. E Noemi tomou o filho, e o pôs no seu
colo, e foi sua ama. E as vizinhas lhe deram um nome,
dizendo: A Noemi nasceu um filho. E deram-lhe o nome de Obede. Este
é o pai de Jessé, pai de Davi. Estas são, pois, as gerações
de Perez: Perez gerou a Esrom, e Esrom gerou a Rão, e Rão
gerou a Aminadabe, e Aminadabe gerou a Naassom, e Naassom
gerou a Salmom, e Salmom gerou a Boaz, e Boaz gerou a Obede,
e Obede gerou a Jessé, e Jessé gerou a Davi.

