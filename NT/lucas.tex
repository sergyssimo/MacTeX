\chapter*{O Evangelho de Lucas}

\lettrine{1} Tendo, pois, muitos empreendido pôr em ordem a
narração dos fatos que entre nós se cumpriram, segundo nos
transmitiram os mesmos que os presenciaram desde o princípio, e
foram ministros da palavra, pareceu-me também a mim conveniente
descrevê-los a ti, ó excelente Teófilo, por sua ordem, havendo-me já
informado minuciosamente de tudo desde o princípio; para que
conheças a certeza das coisas de que já estás informado.

Existiu, no tempo de Herodes, rei da Judéia, um sacerdote chamado
Zacarias, da ordem de Abias, e cuja mulher era das filhas de Arão; e
o seu nome era Isabel. E eram ambos justos perante Deus, andando
sem repreensão em todos os mandamentos e preceitos do Senhor. E
não tinham filhos, porque Isabel era estéril, e ambos eram avançados
em idade. E aconteceu que, exercendo ele o sacerdócio diante de
Deus, na ordem da sua turma, segundo o costume sacerdotal,
coube-lhe em sorte entrar no templo do Senhor para oferecer o
incenso. E toda a multidão do povo estava fora, orando, à
hora do incenso. E um anjo do Senhor lhe apareceu, posto em
pé, à direita do altar do incenso. E Zacarias, vendo-o,
turbou-se, e caiu temor sobre ele. Mas o anjo lhe disse:
Zacarias, não temas, porque a tua oração foi ouvida, e Isabel, tua
mulher, dará à luz um filho, e lhe porás o nome de João. E
terás prazer e alegria, e muitos se alegrarão no seu nascimento,
porque será grande diante do Senhor, e não beberá vinho, nem
bebida forte, e será cheio do Espírito Santo, já desde o ventre de
sua mãe. E converterá muitos dos filhos de Israel ao Senhor
seu Deus, e irá adiante dele no espírito e virtude de Elias,
para converter os corações dos pais aos filhos, e os rebeldes à
prudência dos justos, com o fim de preparar ao Senhor um povo bem
disposto. Disse então Zacarias ao anjo: Como saberei isto?
pois eu já sou velho, e minha mulher avançada em idade. E,
respondendo o anjo, disse-lhe: Eu sou Gabriel, que assisto diante de
Deus, e fui enviado a falar-te e dar-te estas alegres novas.
E eis que ficarás mudo, e não poderás falar até ao dia em que
estas coisas aconteçam; porquanto não creste nas minhas palavras,
que a seu tempo se hão de cumprir. E o povo estava esperando
a Zacarias, e maravilhava-se de que tanto se demorasse no templo.
E, saindo ele, não lhes podia falar; e entenderam que tinha
tido uma visão no templo. E falava por acenos, e ficou mudo.
E sucedeu que, terminados os dias de seu ministério, voltou
para sua casa. E, depois daqueles dias, Isabel, sua mulher,
concebeu, e por cinco meses se ocultou, dizendo: Assim me fez
o Senhor, nos dias em que atentou em mim, para destruir o meu
opróbrio entre os homens.

E, no sexto mês, foi o anjo Gabriel enviado por Deus a uma cidade
da Galiléia, chamada Nazaré, a uma virgem desposada com um
homem, cujo nome era José, da casa de Davi; e o nome da virgem era
Maria. E, entrando o anjo aonde ela estava, disse: Salve,
agraciada; o Senhor é contigo; bendita és tu entre as mulheres.
E, vendo-o ela, turbou-se muito com aquelas palavras, e
considerava que saudação seria esta. Disse-lhe, então, o
anjo: Maria, não temas, porque achaste graça diante de Deus.
E eis que em teu ventre conceberás e darás à luz um filho, e
pôr-lhe-ás o nome de Jesus. Este será grande, e será chamado
filho do Altíssimo; e o Senhor Deus lhe dará o trono de Davi, seu
pai; e reinará eternamente na casa de Jacó, e o seu reino não
terá fim. E disse Maria ao anjo: Como se fará isto, visto que
não conheço homem algum? E, respondendo o anjo, disse-lhe:
Descerá sobre ti o Espírito Santo, e a virtude do Altíssimo te
cobrirá com a sua sombra; por isso também o Santo, que de ti há de
nascer, será chamado Filho de Deus. E eis que também Isabel,
tua prima, concebeu um filho em sua velhice; e é este o sexto mês
para aquela que era chamada estéril; porque para Deus nada é
impossível. Disse então Maria: Eis aqui a serva do Senhor;
cumpra-se em mim segundo a tua palavra. E o anjo ausentou-se dela.

E, naqueles dias, levantando-se Maria, foi apressada às
montanhas, a uma cidade de Judá, e entrou em casa de
Zacarias, e saudou a Isabel. E aconteceu que, ao ouvir Isabel
a saudação de Maria, a criancinha saltou no seu ventre; e Isabel foi
cheia do Espírito Santo. E exclamou com grande voz, e disse:
Bendita és tu entre as mulheres, e bendito o fruto do teu ventre.
E de onde me provém isto a mim, que venha visitar-me a mãe do
meu Senhor? Pois eis que, ao chegar aos meus ouvidos a voz da
tua saudação, a criancinha saltou de alegria no meu ventre.
Bem-aventurada a que creu, pois hão de cumprir-se as coisas
que da parte do Senhor lhe foram ditas. Disse então Maria: A
minha alma engrandece ao Senhor, e o meu espírito se alegra
em Deus meu Salvador; porque atentou na baixeza de sua serva;
pois eis que desde agora todas as gerações me chamarão
bem-aventurada, porque me fez grandes coisas o Poderoso; e
santo é seu nome. E a sua misericórdia é de geração em
geração sobre os que o temem. Com o seu braço agiu
valorosamente; dissipou os soberbos no pensamento de seus corações.
Depôs dos tronos os poderosos, e elevou os humildes.
Encheu de bens os famintos, e despediu vazios os ricos.
Auxiliou a Israel seu servo, recordando-se da sua
misericórdia; como falou a nossos pais, para com Abraão e a
sua posteridade, para sempre. E Maria ficou com ela quase
três meses, e depois voltou para sua casa.

E completou-se para Isabel o tempo de dar à luz, e teve um filho.
E os seus vizinhos e parentes ouviram que tinha Deus usado
para com ela de grande misericórdia, e alegraram-se com ela.
E aconteceu que, ao oitavo dia, vieram circuncidar o menino,
e lhe chamavam Zacarias, o nome de seu pai. E, respondendo
sua mãe, disse: Não, porém será chamado João. E disseram-lhe:
Ninguém há na tua parentela que se chame por este nome. E
perguntaram por acenos ao pai como queria que lhe chamassem.
E, pedindo ele uma tabuinha de escrever, escreveu, dizendo: O
seu nome é João. E todos se maravilharam. E logo a boca se
lhe abriu, e a língua se lhe soltou; e falava, louvando a Deus.
E veio temor sobre todos os seus vizinhos, e em todas as
montanhas da Judéia foram divulgadas todas estas coisas. E
todos os que as ouviam as conservavam em seus corações, dizendo:
Quem será, pois, este menino? E a mão do Senhor estava com ele.

E Zacarias, seu pai, foi cheio do Espírito Santo, e profetizou,
dizendo: Bendito o Senhor Deus de Israel, porque visitou e
remiu o seu povo, e nos levantou uma salvação poderosa na
casa de Davi seu servo. Como falou pela boca dos seus santos
profetas, desde o princípio do mundo; para nos livrar dos
nossos inimigos e da mão de todos os que nos odeiam; para
manifestar misericórdia a nossos pais, e lembrar-se da sua santa
aliança, e do juramento que jurou a Abraão nosso pai,
de conceder-nos que, libertados da mão de nossos inimigos, o
serviríamos sem temor, em santidade e justiça perante ele,
todos os dias da nossa vida. E tu, ó menino, serás chamado
profeta do Altíssimo, porque hás de ir ante a face do Senhor, a
preparar os seus caminhos; para dar ao seu povo conhecimento
da salvação, na remissão dos seus pecados; pelas entranhas da
misericórdia do nosso Deus, com que o oriente do alto nos visitou;
para iluminar aos que estão assentados em trevas e na sombra
da morte; a fim de dirigir os nossos pés pelo caminho da paz.
E o menino crescia, e se robustecia em espírito. E esteve nos
desertos até ao dia em que havia de mostrar-se a Israel.

\medskip

\lettrine{2} E aconteceu naqueles dias que saiu um decreto da
parte de César Augusto, para que todo o mundo se alistasse (Este
primeiro alistamento foi feito sendo Quirino presidente da Síria).
E todos iam alistar-se, cada um à sua própria cidade. E
subiu também José da Galiléia, da cidade de Nazaré, à Judéia, à
cidade de Davi, chamada Belém (porque era da casa e família de
Davi), a fim de alistar-se com Maria, sua esposa, que estava
grávida. E aconteceu que, estando eles ali, se cumpriram os dias
em que ela havia de dar à luz. E deu à luz a seu filho
primogênito, e envolveu-o em panos, e deitou-o numa manjedoura,
porque não havia lugar para eles na estalagem.

Ora, havia naquela mesma comarca pastores que estavam no campo, e
guardavam, durante as vigílias da noite, o seu rebanho. E eis
que o anjo do Senhor veio sobre eles, e a glória do Senhor os cercou
de resplendor, e tiveram grande temor. E o anjo lhes disse:
Não temais, porque eis aqui vos trago novas de grande alegria, que
será para todo o povo: Pois, na cidade de Davi, vos nasceu
hoje o Salvador, que é Cristo, o Senhor. E isto vos será por
sinal: Achareis o menino envolto em panos, e deitado numa
manjedoura. E, no mesmo instante, apareceu com o anjo uma
multidão dos exércitos celestiais, louvando a Deus, e dizendo:
Glória a Deus nas alturas, paz na terra, boa vontade para com
os homens. E aconteceu que, ausentando-se deles os anjos para
o céu, disseram os pastores uns aos outros: Vamos, pois, até Belém,
e vejamos isso que aconteceu, e que o Senhor nos fez saber. E
foram apressadamente, e acharam Maria, e José, e o menino deitado na
manjedoura. E, vendo-o, divulgaram a palavra que acerca do
menino lhes fora dita; e todos os que a ouviram se
maravilharam do que os pastores lhes diziam. Mas Maria
guardava todas estas coisas, conferindo-as em seu coração. E
voltaram os pastores, glorificando e louvando a Deus por tudo o que
tinham ouvido e visto, como lhes havia sido dito.

E, quando os oito dias foram cumpridos, para circuncidar o
menino, foi-lhe dado o nome de Jesus, que pelo anjo lhe fora posto
antes de ser concebido. E, cumprindo-se os dias da
purificação dela, segundo a lei de Moisés, o levaram a Jerusalém,
para o apresentarem ao Senhor

o que está escrito na lei do Senhor: Todo o macho
primogênito será consagrado ao Senhor); e para darem a oferta
segundo o disposto na lei do Senhor: Um par de rolas ou dois
pombinhos.

Havia em Jerusalém um homem cujo nome era Simeão; e este homem
era justo e temente a Deus, esperando a consolação de Israel; e o
Espírito Santo estava sobre ele. E fora-lhe revelado, pelo
Espírito Santo, que ele não morreria antes de ter visto o Cristo do
Senhor. E pelo Espírito foi ao templo e, quando os pais
trouxeram o menino Jesus, para com ele procederem segundo o uso da
lei, ele, então, o tomou em seus braços, e louvou a Deus, e
disse: Agora, Senhor, despedes em paz o teu servo, segundo a
tua palavra; pois já os meus olhos viram a tua salvação,
a qual tu preparaste perante a face de todos os povos;
luz para iluminar as nações, e para glória de teu povo
Israel. E José, e sua mãe, se maravilharam das coisas que
dele se diziam. E Simeão os abençoou, e disse a Maria, sua
mãe: Eis que este é posto para queda e elevação de muitos em Israel,
e para sinal que é contraditado

uma espada traspassará também a tua própria alma); para que se
manifestem os pensamentos de muitos corações. E estava ali a
profetisa Ana, filha de Fanuel, da tribo de Aser. Esta era já
avançada em idade, e tinha vivido com o marido sete anos, desde a
sua virgindade; e era viúva, de quase oitenta e quatro anos,
e não se afastava do templo, servindo a Deus em jejuns e orações, de
noite e de dia. E sobrevindo na mesma hora, ela dava graças a
Deus, e falava dele a todos os que esperavam a redenção em
Jerusalém. E, quando acabaram de cumprir tudo segundo a lei
do Senhor, voltaram à Galiléia, para a sua cidade de Nazaré.
E o menino crescia, e se fortalecia em espírito, cheio de
sabedoria; e a graça de Deus estava sobre ele.

Ora, todos os anos iam seus pais a Jerusalém à festa da páscoa;
e, tendo ele já doze anos, subiram a Jerusalém, segundo o
costume do dia da festa. E, regressando eles, terminados
aqueles dias, ficou o menino Jesus em Jerusalém, e não o soube José,
nem sua mãe. Pensando, porém, eles que viria de companhia
pelo caminho, andaram caminho de um dia, e procuravam-no entre os
parentes e conhecidos; e, como o não encontrassem, voltaram a
Jerusalém em busca dele. E aconteceu que, passados três dias,
o acharam no templo, assentado no meio dos doutores, ouvindo-os, e
interrogando-os. E todos os que o ouviam admiravam a sua
inteligência e respostas. E quando o viram, maravilharam-se,
e disse-lhe sua mãe: Filho, por que fizeste assim para conosco? Eis
que teu pai e eu ansiosos te procurávamos. E ele lhes disse:
Por que é que me procuráveis? Não sabeis que me convém tratar dos
negócios de meu Pai? E eles não compreenderam as palavras que
lhes dizia. E desceu com eles, e foi para Nazaré, e era-lhes
sujeito. E sua mãe guardava no seu coração todas estas coisas.
E crescia Jesus em sabedoria, e em estatura, e em graça para
com Deus e os homens.

\medskip

\lettrine{3} E no ano quinze do império de Tibério César,
sendo Pôncio Pilatos presidente da Judéia, e Herodes tetrarca da
Galiléia, e seu irmão Filipe tetrarca da Ituréia e da província de
Traconites, e Lisânias tetrarca de Abilene, sendo Anás e Caifás
sumos sacerdotes, veio no deserto a palavra de Deus a João, filho de
Zacarias. E percorreu toda a terra ao redor do Jordão, pregando
o batismo de arrependimento, para o perdão dos pecados; segundo
o que está escrito no livro das palavras do profeta Isaías, que diz:
Voz do que clama no deserto: Preparai o caminho do Senhor;
endireitai as suas veredas. Todo o vale se encherá, e se
abaixará todo o monte e outeiro; e o que é tortuoso se endireitará,
e os caminhos escabrosos se aplanarão; e toda a carne verá a
salvação de Deus. Dizia, pois, João à multidão que saía para ser
batizada por ele: Raça de víboras, quem vos ensinou a fugir da ira
que está para vir? Produzi, pois, frutos dignos de
arrependimento, e não comeceis a dizer em vós mesmos: Temos Abraão
por pai; porque eu vos digo que até destas pedras pode Deus suscitar
filhos a Abraão. E também já está posto o machado à raiz das
árvores; toda a árvore, pois, que não dá bom fruto, corta-se e
lança-se no fogo. E a multidão o interrogava, dizendo: Que
faremos, pois? E, respondendo ele, disse-lhes: Quem tiver
duas túnicas, reparta com o que não tem, e quem tiver alimentos,
faça da mesma maneira. E chegaram também uns publicanos, para
serem batizados, e disseram-lhe: Mestre, que devemos fazer? E
ele lhes disse: Não peçais mais do que o que vos está ordenado.
E uns soldados o interrogaram também, dizendo: E nós que
faremos? E ele lhes disse: A ninguém trateis mal nem defraudeis, e
contentai-vos com o vosso soldo.

E, estando o povo em expectação, e pensando todos de João, em
seus corações, se porventura seria o Cristo, respondeu João a
todos, dizendo: Eu, na verdade, batizo-vos com água, mas eis que vem
aquele que é mais poderoso do que eu, do qual não sou digno de
desatar a correia das alparcas; esse vos batizará com o Espírito
Santo e com fogo. Ele tem a pá na sua mão; e limpará a sua
eira, e ajuntará o trigo no seu celeiro, mas queimará a palha com
fogo que nunca se apaga. E assim, admoestando-os, muitas
outras coisas também anunciava ao povo. Sendo, porém, o
tetrarca Herodes repreendido por ele por causa de Herodias, mulher
de seu irmão Filipe, e por todas as maldades que Herodes tinha
feito, acrescentou a todas as outras ainda esta, a de
encerrar João num cárcere.

E aconteceu que, como todo o povo se batizava, sendo batizado
também Jesus, orando ele, o céu se abriu; e o Espírito Santo
desceu sobre ele em forma corpórea, como pomba; e ouviu-se uma voz
do céu, que dizia: Tu és o meu Filho amado, em ti me comprazo.
E o mesmo Jesus começava a ser de quase trinta anos, sendo
(como se cuidava) filho de José, e José de Heli, e Heli de
Matã, e Matã de Levi, e Levi de Melqui, e Melqui de Janai, e Janai
de José, e José de Matatias, e Matatias de Amós, e Amós de
Naum, e Naum de Esli, e Esli de Nagaí, e Nagaí de Máate, e
Máate de Matatias, e Matatias de Semei, e Semei de José, e José de
Jodá, e Jodá de Joanã, e Joanã de Resá, e Resá de Zorobabel,
e Zorobabel de Salatiel, e Salatiel de Neri, e Neri de
Melqui, e Melqui de Adi, e Adi de Cosã, e Cosã de Elmadã, e Elmadã
de Er, e Er de Josué, e Josué de Eliézer, e Eliézer de Jorim,
e Jorim de Matã, e Matã de Levi, e Levi de Simeão, e Simeão
de Judá, e Judá de José, e José de Jonã, e Jonã de Eliaquim,
e Eliaquim de Meleá, e Meleá de Mená, e Mená de Matatá, e
Matatá de Natã, e Natã de Davi, e Davi de Jessé, e Jessé de
Obede, e Obede de Boaz, e Boaz de Salá, e Salá de Naassom, e
Naassom de Aminadabe, e Aminadabe de Arão, e Arão de Esrom, e Esrom
de Perez, e Perez de Judá, e Judá de Jacó, e Jacó de Isaque,
e Isaque de Abraão, e Abraão de Terá, e Terá de Nacor, e
Nacor de Seruque, e Seruque de Ragaú, e Ragaú de Fáleque, e Fáleque
de Éber, e Éber de Salá, e Salá de Cainã, e Cainã de
Arfaxade, e Arfaxade de Sem, e Sem de Noé, e Noé de Lameque,
e Lameque de Matusalém, e Matusalém de Enoque, e Enoque de
Jarete, e Jarete de Maleleel, e Maleleel de Cainã, e Cainã de
Enos, e Enos de Sete, e Sete de Adão, e Adão de Deus.

\medskip

\lettrine{4} E Jesus, cheio do Espírito Santo, voltou do
Jordão e foi levado pelo Espírito ao deserto; e quarenta dias
foi tentado pelo diabo, e naqueles dias não comeu coisa alguma; e,
terminados eles, teve fome. E disse-lhe o diabo: Se tu és o
Filho de Deus, dize a esta pedra que se transforme em pão. E
Jesus lhe respondeu, dizendo: Está escrito que nem só de pão viverá
o homem, mas de toda a palavra de Deus. E o diabo, levando-o a
um alto monte, mostrou-lhe num momento de tempo todos os reinos do
mundo. E disse-lhe o diabo: Dar-te-ei a ti todo este poder e a
sua glória; porque a mim me foi entregue, e dou-o a quem quero.
Portanto, se tu me adorares, tudo será teu. E Jesus,
respondendo, disse-lhe: Vai-te para trás de mim, Satanás; porque
está escrito: Adorarás o Senhor teu Deus, e só a ele servirás.
Levou-o também a Jerusalém, e pô-lo sobre o pináculo do templo,
e disse-lhe: Se tu és o Filho de Deus, lança-te daqui abaixo;
porque está escrito: Mandará aos seus anjos, acerca de ti,
que te guardem, e que te sustenham nas mãos, para que nunca
tropeces com o teu pé em alguma pedra. E Jesus, respondendo,
disse-lhe: Dito está: Não tentarás ao Senhor teu Deus. E,
acabando o diabo toda a tentação, ausentou-se dele por algum tempo.

Então, pela virtude do Espírito, voltou Jesus para a Galiléia, e
a sua fama correu por todas as terras em derredor. E ensinava
nas suas sinagogas, e por todos era louvado. E, chegando a
Nazaré, onde fora criado, entrou num dia de sábado, segundo o seu
costume, na sinagoga, e levantou-se para ler. E foi-lhe dado
o livro do profeta Isaías; e, quando abriu o livro, achou o lugar em
que estava escrito: O Espírito do Senhor é sobre mim, pois
que me ungiu para evangelizar os pobres. Enviou-me a curar os
quebrantados do coração, a pregar liberdade aos cativos, e
restauração da vista aos cegos, a pôr em liberdade os oprimidos, a
anunciar o ano aceitável do Senhor. E, cerrando o livro, e
tornando-o a dar ao ministro, assentou-se; e os olhos de todos na
sinagoga estavam fitos nele. Então começou a dizer-lhes: Hoje
se cumpriu esta Escritura em vossos ouvidos. E todos lhe
davam testemunho, e se maravilhavam das palavras de graça que saíam
da sua boca; e diziam: Não é este o filho de José? E ele lhes
disse: Sem dúvida me direis este provérbio: Médico, cura-te a ti
mesmo; faze também aqui na tua pátria tudo que ouvimos ter sido
feito em Cafarnaum. E disse: Em verdade vos digo que nenhum
profeta é bem recebido na sua pátria. Em verdade vos digo que
muitas viúvas existiam em Israel nos dias de Elias, quando o céu se
cerrou por três anos e seis meses, de sorte que em toda a terra
houve grande fome; e a nenhuma delas foi enviado Elias, senão
a Sarepta de Sidom, a uma mulher viúva. E muitos leprosos
havia em Israel no tempo do profeta Eliseu, e nenhum deles foi
purificado, senão Naamã, o sírio. E todos, na sinagoga,
ouvindo estas coisas, se encheram de ira. E, levantando-se, o
expulsaram da cidade, e o levaram até ao cume do monte em que a
cidade deles estava edificada, para dali o precipitarem. Ele,
porém, passando pelo meio deles, retirou-se.

E desceu a Cafarnaum, cidade da Galiléia, e os ensinava nos
sábados. E admiravam a sua doutrina porque a sua palavra era
com autoridade. E estava na sinagoga um homem que tinha o
espírito de um demônio imundo, e exclamou em alta voz,
dizendo: Ah! que temos nós contigo, Jesus Nazareno? Vieste a
destruir-nos? Bem sei quem és: O Santo de Deus. E Jesus o
repreendeu, dizendo: Cala-te, e sai dele. E o demônio, lançando-o
por terra no meio do povo, saiu dele sem lhe fazer mal. E
veio espanto sobre todos, e falavam uns com os outros, dizendo: Que
palavra é esta, que até aos espíritos imundos manda com autoridade e
poder, e eles saem? E a sua fama divulgava-se por todos os
lugares, em redor daquela comarca. Ora, levantando-se Jesus
da sinagoga, entrou em casa de Simão; e a sogra de Simão estava
enferma com muita febre, e rogaram-lhe por ela. E,
inclinando-se para ela, repreendeu a febre, e esta a deixou. E ela,
levantando-se logo, servia-os. E, ao pôr do sol, todos os que
tinham enfermos de várias doenças lhos traziam; e, pondo as mãos
sobre cada um deles, os curava. E também de muitos saíam
demônios, clamando e dizendo: Tu és o Cristo, o Filho de Deus. E
ele, repreendendo-os, não os deixava falar, pois sabiam que ele era
o Cristo. E, sendo já dia, saiu, e foi para um lugar deserto;
e a multidão o procurava, e chegou junto dele; e o detinham, para
que não se ausentasse deles. Ele, porém, lhes disse: Também é
necessário que eu anuncie a outras cidades o evangelho do reino de
Deus; porque para isso fui enviado. E pregava nas sinagogas
da Galiléia.

\medskip

\lettrine{5} E aconteceu que, apertando-o a multidão, para
ouvir a palavra de Deus, estava ele junto ao lago de Genesaré; e
viu estar dois barcos junto à praia do lago; e os pescadores,
havendo descido deles, estavam lavando as redes. E, entrando num
dos barcos, que era o de Simão, pediu-lhe que o afastasse um pouco
da terra; e, assentando-se, ensinava do barco a multidão. E,
quando acabou de falar, disse a Simão: Faze-te ao mar alto, e lançai
as vossas redes para pescar. E, respondendo Simão, disse-lhe:
Mestre, havendo trabalhado toda a noite, nada apanhamos; mas, sobre
a tua palavra, lançarei a rede. E, fazendo assim, colheram uma
grande quantidade de peixes, e rompia-se-lhes a rede. E fizeram
sinal aos companheiros que estavam no outro barco, para que os
fossem ajudar. E foram, e encheram ambos os barcos, de maneira tal
que quase iam a pique. E vendo isto Simão Pedro, prostrou-se aos
pés de Jesus, dizendo: Senhor, ausenta-te de mim, que sou um homem
pecador. Pois que o espanto se apoderara dele, e de todos os que
com ele estavam, por causa da pesca de peixe que haviam feito.
E, de igual modo, também de Tiago e João, filhos de Zebedeu,
que eram companheiros de Simão. E disse Jesus a Simão: Não temas; de
agora em diante serás pescador de homens. E, levando os
barcos para terra, deixaram tudo, e o seguiram.

E aconteceu que, quando estava numa daquelas cidades, eis que um
homem cheio de lepra, vendo a Jesus, prostrou-se sobre o rosto, e
rogou-lhe, dizendo: Senhor, se quiseres, bem podes limpar-me.
E ele, estendendo a mão, tocou-lhe, dizendo: Quero, sê limpo.
E logo a lepra desapareceu dele. E ordenou-lhe que a ninguém
o dissesse. Mas vai, disse, mostra-te ao sacerdote, e oferece, pela
tua purificação, o que Moisés determinou, para que lhes sirva de
testemunho. A sua fama, porém, se propagava ainda mais, e
ajuntava-se muita gente para o ouvir e para ser por ele curada das
suas enfermidades. Ele, porém, retirava-se para os desertos,
e ali orava.

E aconteceu que, num daqueles dias, estava ensinando, e estavam
ali assentados fariseus e doutores da lei, que tinham vindo de todas
as aldeias da Galiléia, e da Judéia, e de Jerusalém. E a virtude do
Senhor estava com ele para curar. E eis que uns homens
transportaram numa cama um homem que estava paralítico, e procuravam
fazê-lo entrar e pô-lo diante dele. E, não achando por onde o
pudessem levar, por causa da multidão, subiram ao telhado, e por
entre as telhas o baixaram com a cama, até ao meio, diante de Jesus.
E, vendo ele a fé deles, disse-lhe: Homem, os teus pecados te
são perdoados. E os escribas e os fariseus começaram a
arrazoar, dizendo: Quem é este que diz blasfêmias? Quem pode perdoar
pecados, senão só Deus? Jesus, porém, conhecendo os seus
pensamentos, respondeu, e disse-lhes: Que arrazoais em vossos
corações? Qual é mais fácil? dizer: Os teus pecados te são
perdoados; ou dizer: Levanta-te, e anda? Ora, para que
saibais que o Filho do homem tem sobre a terra poder de perdoar
pecados (disse ao paralítico), a ti te digo: Levanta-te, toma a tua
cama, e vai para tua casa. E, levantando-se logo diante
deles, e tomando a cama em que estava deitado, foi para sua casa,
glorificando a Deus. E todos ficaram maravilhados, e
glorificaram a Deus; e ficaram cheios de temor, dizendo: Hoje vimos
prodígios.

E, depois disto, saiu, e viu um publicano, chamado Levi,
assentado na recebedoria, e disse-lhe: Segue-me. E ele,
deixando tudo, levantou-se e o seguiu. E fez-lhe Levi um
grande banquete em sua casa; e havia ali uma multidão de publicanos
e outros que estavam com eles à mesa. E os escribas deles, e
os fariseus, murmuravam contra os seus discípulos, dizendo: Por que
comeis e bebeis com publicanos e pecadores? E Jesus,
respondendo, disse-lhes: Não necessitam de médico os que estão sãos,
mas, sim, os que estão enfermos; eu não vim chamar os justos,
mas, sim, os pecadores, ao arrependimento. Disseram-lhe,
então, eles: Por que jejuam os discípulos de João muitas vezes, e
fazem orações, como também os dos fariseus, mas os teus comem e
bebem? E ele lhes disse: Podeis vós fazer jejuar os filhos
das bodas, enquanto o esposo está com eles? Dias virão,
porém, em que o esposo lhes será tirado, e então, naqueles dias,
jejuarão. E disse-lhes também uma parábola: Ninguém tira um
pedaço de uma roupa nova para a coser em roupa velha, pois romperá a
nova e o remendo não condiz com a velha. E ninguém deita
vinho novo em odres velhos; de outra sorte o vinho novo romperá os
odres, e entornar-se-á o vinho, e os odres se estragarão; mas
o vinho novo deve deitar-se em odres novos, e ambos juntamente se
conservarão. E ninguém tendo bebido o velho quer logo o novo,
porque diz: Melhor é o velho.

\medskip

\lettrine{6} E aconteceu que, no sábado
segundo-primeiro\footnote{KJ: ``And it came to pass on the second
sabbath after the first, that he went through the corn fields
\ldots''. Ou seja, ``no segundo sábado depois do primeiro''.},
passou pelas searas, e os seus discípulos iam arrancando espigas e,
esfregando-as com as mãos, as comiam. E alguns dos fariseus lhes
disseram: Por que fazeis o que não é lícito fazer nos sábados? E
Jesus, respondendo-lhes, disse: Nunca lestes o que fez Davi quando
teve fome, ele e os que com ele estavam? Como entrou na casa de
Deus, e tomou os pães da proposição, e os comeu, e deu também aos
que estavam com ele, os quais não é lícito comer senão só aos
sacerdotes? E dizia-lhes: O Filho do homem é Senhor até do
sábado. E aconteceu também noutro sábado, que entrou na
sinagoga, e estava ensinando; e havia ali um homem que tinha a mão
direita mirrada. E os escribas e fariseus observavam-no, se o
curaria no sábado, para acharem de que o acusar. Mas ele bem
conhecia os seus pensamentos; e disse ao homem que tinha a mão
mirrada: Levanta-te, e fica em pé no meio. E, levantando-se ele,
ficou em pé. Então Jesus lhes disse: Uma coisa vos hei de
perguntar: É lícito nos sábados fazer bem, ou fazer mal? salvar a
vida, ou matar? E, olhando para todos em redor, disse ao
homem: Estende a tua mão. E ele assim o fez, e a mão lhe foi
restituída sã como a outra. E ficaram cheios de furor, e uns
com os outros conferenciavam sobre o que fariam a Jesus.

E aconteceu que naqueles dias subiu ao monte a orar, e passou a
noite em oração a Deus. E, quando já era dia, chamou a si os
seus discípulos, e escolheu doze deles, a quem também deu o nome de
apóstolos: Simão, ao qual também chamou Pedro, e André, seu
irmão; Tiago e João; Filipe e Bartolomeu; Mateus e Tomé;
Tiago, filho de Alfeu, e Simão, chamado Zelote; e Judas,
irmão de Tiago, e Judas Iscariotes, que foi o traidor. E,
descendo com eles, parou num lugar plano, e também um grande número
de seus discípulos, e grande multidão de povo de toda a Judéia, e de
Jerusalém, e da costa marítima de Tiro e de Sidom; os quais tinham
vindo para o ouvir, e serem curados das suas enfermidades,
como também os atormentados dos espíritos imundos; e eram
curados. E toda a multidão procurava tocar-lhe, porque saía
dele virtude, e curava a todos.

E, levantando ele os olhos para os seus discípulos, dizia:
Bem-aventurados vós, os pobres, porque vosso é o reino de Deus.
Bem-aventurados vós, que agora tendes fome, porque sereis
fartos. Bem-aventurados vós, que agora chorais, porque haveis de
rir. Bem-aventurados sereis quando os homens vos odiarem e
quando vos separarem, e vos injuriarem, e rejeitarem o vosso nome
como mau, por causa do Filho do homem. Folgai nesse dia,
exultai; porque eis que é grande o vosso galardão no céu, pois assim
faziam os seus pais aos profetas. Mas ai de vós, ricos!
porque já tendes a vossa consolação. Ai de vós, os que estais
fartos, porque tereis fome. Ai de vós, os que agora rides, porque
vos lamentareis e chorareis. Ai de vós quando todos os homens
de vós disserem bem, porque assim faziam seus pais aos falsos
profetas.

Mas a vós, que isto ouvis, digo: Amai a vossos inimigos, fazei
bem aos que vos odeiam; bendizei os que vos maldizem, e orai
pelos que vos caluniam. Ao que te ferir numa face,
oferece-lhe também a outra; e ao que te houver tirado a capa, nem a
túnica recuses; e dá a qualquer que te pedir; e ao que tomar
o que é teu, não lho tornes a pedir. E como vós quereis que
os homens vos façam, da mesma maneira lhes fazei vós, também.
E se amardes aos que vos amam, que recompensa tereis? Também
os pecadores amam aos que os amam. E se fizerdes bem aos que
vos fazem bem, que recompensa tereis? Também os pecadores fazem o
mesmo. E se emprestardes àqueles de quem esperais tornar a
receber, que recompensa tereis? Também os pecadores emprestam aos
pecadores, para tornarem a receber outro tanto. Amai, pois, a
vossos inimigos, e fazei bem, e emprestai, sem nada esperardes, e
será grande o vosso galardão, e sereis filhos do Altíssimo; porque
ele é benigno até para com os ingratos e maus. Sede, pois,
misericordiosos, como também vosso Pai é misericordioso.

Não julgueis, e não sereis julgados; não condeneis, e não sereis
condenados; soltai, e soltar-vos-ão. Dai, e ser-vos-á dado;
boa medida, recalcada, sacudida e transbordando, vos deitarão no
vosso regaço; porque com a mesma medida com que medirdes também vos
medirão de novo. E dizia-lhes uma parábola: Pode porventura o
cego guiar o cego? Não cairão ambos na cova? O discípulo não
é superior a seu mestre, mas todo o que for perfeito será como o seu
mestre. E por que atentas tu no argueiro que está no olho de
teu irmão, e não reparas na trave que está no teu próprio olho?
Ou como podes dizer a teu irmão: Irmão, deixa-me tirar o
argueiro que está no teu olho, não atentando tu mesmo na trave que
está no teu olho? Hipócrita, tira primeiro a trave do teu olho, e
então verás bem para tirar o argueiro que está no olho de teu irmão.
Porque não há boa árvore que dê mau fruto, nem má árvore que
dê bom fruto. Porque cada árvore se conhece pelo seu próprio
fruto; pois não se colhem figos dos espinheiros, nem se vindimam
uvas dos abrolhos. O homem bom, do bom tesouro do seu coração
tira o bem, e o homem mau, do mau tesouro do seu coração tira o mal,
porque da abundância do seu coração fala a boca. E por que me
chamais, Senhor, Senhor, e não fazeis o que eu digo? Qualquer
que vem a mim e ouve as minhas palavras, e as observa, eu vos
mostrarei a quem é semelhante: É semelhante ao homem que
edificou uma casa, e cavou, e abriu bem fundo, e pôs os alicerces
sobre a rocha; e, vindo a enchente, bateu com ímpeto a corrente
naquela casa, e não a pôde abalar, porque estava fundada sobre a
rocha. Mas o que ouve e não pratica é semelhante ao homem que
edificou uma casa sobre terra, sem alicerces, na qual bateu com
ímpeto a corrente, e logo caiu; e foi grande a ruína daquela casa.

\medskip

\lettrine{7} E, depois de concluir todos estes discursos
perante o povo, entrou em Cafarnaum. E o servo de um certo
centurião, a quem muito estimava, estava doente, e moribundo. E,
quando ouviu falar de Jesus, enviou-lhe uns anciãos dos judeus,
rogando-lhe que viesse curar o seu servo. E, chegando eles junto
de Jesus, rogaram-lhe muito, dizendo: É digno de que lhe concedas
isto, porque ama a nossa nação, e ele mesmo nos edificou a
sinagoga. E foi Jesus com eles; mas, quando já estava perto da
casa, enviou-lhe o centurião uns amigos, dizendo-lhe: Senhor, não te
incomodes, porque não sou digno de que entres debaixo do meu
telhado. E por isso nem ainda me julguei digno de ir ter
contigo; dize, porém, uma palavra, e o meu criado sarará. Porque
também eu sou homem sujeito à autoridade, e tenho soldados sob o meu
poder, e digo a este: Vai, e ele vai; e a outro: Vem, e ele vem; e
ao meu servo: Faze isto, e ele o faz. E, ouvindo isto Jesus,
maravilhou-se dele, e voltando-se, disse à multidão que o seguia:
Digo-vos que nem ainda em Israel tenho achado tanta fé. E,
voltando para casa os que foram enviados, acharam são o servo
enfermo.

E aconteceu que, no dia seguinte, ele foi à cidade chamada Naim,
e com ele iam muitos dos seus discípulos, e uma grande multidão;
e, quando chegou perto da porta da cidade, eis que levavam um
defunto, filho único de sua mãe, que era viúva; e com ela ia uma
grande multidão da cidade. E, vendo-a, o Senhor moveu-se de
íntima compaixão por ela, e disse-lhe: Não chores. E,
chegando-se, tocou o esquife (e os que o levavam pararam), e disse:
Jovem, a ti te digo: Levanta-te. E o defunto assentou-se, e começou
a falar. E entregou-o a sua mãe. E de todos se
apoderou o temor, e glorificavam a Deus, dizendo: Um grande profeta
se levantou entre nós, e Deus visitou o seu povo. E correu
dele esta fama por toda a Judéia e por toda a terra circunvizinha.
E os discípulos de João anunciaram-lhe todas estas coisas.

E João, chamando dois dos seus discípulos, enviou-os a Jesus,
dizendo: És tu aquele que havia de vir, ou esperamos outro?
E, quando aqueles homens chegaram junto dele, disseram: João
o Batista enviou-nos a perguntar-te: És tu aquele que havia de vir,
ou esperamos outro? E, na mesma hora, curou muitos de
enfermidades, e males, e espíritos maus, e deu vista a muitos cegos.
Respondendo, então, Jesus, disse-lhes: Ide, e anunciai a João
o que tendes visto e ouvido: que os cegos vêem, os coxos andam, os
leprosos são purificados, os surdos ouvem, os mortos ressuscitam e
aos pobres anuncia-se o evangelho. E bem-aventurado é aquele
que em mim se não escandalizar. E, tendo-se retirado os
mensageiros de João, começou a dizer à multidão acerca de João: Que
saístes a ver no deserto? uma cana abalada pelo vento? Mas
que saístes a ver? um homem trajado de vestes delicadas? Eis que os
que andam com preciosas vestiduras, e em delícias, estão nos paços
reais. Mas que saístes a ver? um profeta? Sim, vos digo, e
muito mais do que profeta. Este é aquele de quem está
escrito: Eis que envio o meu anjo diante da tua face, o qual
preparará diante de ti o teu caminho. E eu vos digo que,
entre os nascidos de mulheres, não há maior profeta do que João o
Batista; mas o menor no reino de Deus é maior do que ele. E
todo o povo que o ouviu e os publicanos, tendo sido batizados com o
batismo de João, justificaram a Deus. Mas os fariseus e os
doutores da lei rejeitaram o conselho de Deus contra si mesmos, não
tendo sido batizados por ele. E disse o Senhor: A quem, pois,
compararei os homens desta geração, e a quem são semelhantes?
São semelhantes aos meninos que, assentados nas praças,
clamam uns aos outros, e dizem: Tocamo-vos flauta, e não dançastes;
cantamo-vos lamentações, e não chorastes. Porque veio João o
Batista, que não comia pão nem bebia vinho, e dizeis: Tem demônio;
veio o Filho do homem, que come e bebe, e dizeis: Eis aí um
homem comilão e bebedor de vinho, amigo dos publicanos e pecadores.
Mas a sabedoria é justificada por todos os seus filhos.

E rogou-lhe um dos fariseus que comesse com ele; e, entrando em
casa do fariseu, assentou-se à mesa. E eis que uma mulher da
cidade, uma pecadora, sabendo que ele estava à mesa em casa do
fariseu, levou um vaso de alabastro com ungüento; e, estando
por detrás, aos seus pés, chorando, começou a regar-lhe os pés com
lágrimas, e enxugava-lhos com os cabelos da sua cabeça; e
beijava-lhe os pés, e ungia-lhos com o ungüento. Quando isto
viu o fariseu que o tinha convidado, falava consigo, dizendo: Se
este fora profeta, bem saberia quem e qual é a mulher que lhe tocou,
pois é uma pecadora. E respondendo, Jesus disse-lhe: Simão,
uma coisa tenho a dizer-te. E ele disse: Dize-a, Mestre. Um
certo credor tinha dois devedores: um devia-lhe quinhentos
dinheiros, e outro cinqüenta. E, não tendo eles com que
pagar, perdoou-lhes a ambos. Dize, pois, qual deles o amará mais?
E Simão, respondendo, disse: Tenho para mim que é aquele a
quem mais perdoou. E ele lhe disse: Julgaste bem. E,
voltando-se para a mulher, disse a Simão: Vês tu esta mulher? Entrei
em tua casa, e não me deste água para os pés; mas esta regou-me os
pés com lágrimas, e mos enxugou com os seus cabelos. Não me
deste ósculo, mas esta, desde que entrou, não tem cessado de me
beijar os pés. Não me ungiste a cabeça com óleo, mas esta
ungiu-me os pés com ungüento. Por isso te digo que os seus
muitos pecados lhe são perdoados, porque muito amou; mas aquele a
quem pouco é perdoado pouco ama. E disse-lhe a ela: Os teus
pecados te são perdoados. E os que estavam à mesa começaram a
dizer entre si: Quem é este, que até perdoa pecados? E disse
à mulher: A tua fé te salvou; vai-te em paz.

\medskip

\lettrine{8} E aconteceu, depois disto, que andava de cidade
em cidade, e de aldeia em aldeia, pregando e anunciando o evangelho
do reino de Deus; e os doze iam com ele, e algumas mulheres que
haviam sido curadas de espíritos malignos e de enfermidades: Maria,
chamada Madalena, da qual saíram sete demônios; e Joana, mulher
de Cuza, procurador de Herodes, e Suzana, e muitas outras que o
serviam com seus bens.

E, ajuntando-se uma grande multidão, e vindo de todas as cidades
ter com ele, disse por parábola: Um semeador saiu a semear a sua
semente e, quando semeava, caiu alguma junto do caminho, e foi
pisada, e as aves do céu a comeram; e outra caiu sobre pedra e,
nascida, secou-se, pois que não tinha umidade; e outra caiu
entre espinhos e crescendo com ela os espinhos, a sufocaram; e
outra caiu em boa terra, e, nascida, produziu fruto, a cento por um.
Dizendo ele estas coisas, clamava: Quem tem ouvidos para ouvir,
ouça. E os seus discípulos o interrogaram, dizendo: Que parábola
é esta? E ele disse: A vós vos é dado conhecer os mistérios
do reino de Deus, mas aos outros por parábolas, para que vendo, não
vejam, e ouvindo, não entendam. Esta é, pois, a parábola: A
semente é a palavra de Deus; e os que estão junto do caminho,
estes são os que ouvem; depois vem o diabo, e tira-lhes do coração a
palavra, para que não se salvem, crendo; e os que estão sobre
pedra, estes são os que, ouvindo a palavra, a recebem com alegria,
mas, como não têm raiz, apenas crêem por algum tempo, e no tempo da
tentação se desviam; e a que caiu entre espinhos, esses são
os que ouviram e, indo por diante, são sufocados com os cuidados e
riquezas e deleites da vida, e não dão fruto com perfeição; e
a que caiu em boa terra, esses são os que, ouvindo a palavra, a
conservam num coração honesto e bom, e dão fruto com perseverança.
E ninguém, acendendo uma candeia, a cobre com algum vaso, ou
a põe debaixo da cama; mas põe-na no velador, para que os que entram
vejam a luz. Porque não há coisa oculta que não haja de
manifestar-se, nem escondida que não haja de saber-se e vir à luz.
Vede, pois, como ouvis; porque a qualquer que tiver lhe será
dado, e a qualquer que não tiver até o que parece ter lhe será
tirado. E foram ter com ele sua mãe e seus irmãos, e não
podiam aproximar-se dele, por causa da multidão. E foi-lhe
dito: Estão lá fora tua mãe e teus irmãos, que querem ver-te.
Mas, respondendo ele, disse-lhes: Minha mãe e meus irmãos são
aqueles que ouvem a palavra de Deus e a executam.

E aconteceu que, num daqueles dias, entrou num barco com seus
discípulos, e disse-lhes: Passemos para o outro lado do lago. E
partiram. E, navegando eles, adormeceu; e sobreveio uma
tempestade de vento no lago, e enchiam-se de água, estando em
perigo. E, chegando-se a ele, o despertaram, dizendo: Mestre,
Mestre, perecemos. E ele, levantando-se, repreendeu o vento e a
fúria da água; e cessaram, e fez-se bonança. E disse-lhes:
Onde está a vossa fé? E eles, temendo, maravilharam-se, dizendo uns
aos outros: Quem é este, que até aos ventos e à água manda, e lhe
obedecem? E navegaram para a terra dos gadarenos, que está
defronte da Galiléia. E, quando desceu para terra, saiu-lhe
ao encontro, vindo da cidade, um homem que desde muito tempo estava
possesso de demônios, e não andava vestido, nem habitava em qualquer
casa, mas nos sepulcros. E, quando viu a Jesus, prostrou-se
diante dele, exclamando, e dizendo com grande voz: Que tenho eu
contigo, Jesus, Filho do Deus Altíssimo? Peço-te que não me
atormentes. Porque tinha ordenado ao espírito imundo que
saísse daquele homem; pois já havia muito tempo que o arrebatava. E
guardavam-no preso, com grilhões e cadeias; mas, quebrando as
prisões, era impelido pelo demônio para os desertos. E
perguntou-lhe Jesus, dizendo: Qual é o teu nome? E ele disse:
Legião; porque tinham entrado nele muitos demônios. E
rogavam-lhe que os não mandasse para o abismo. E andava ali
pastando no monte uma vara de muitos porcos; e rogaram-lhe que lhes
concedesse entrar neles; e concedeu-lho. E, tendo saído os
demônios do homem, entraram nos porcos, e a manada precipitou-se de
um despenhadeiro no lago, e afogou-se. E aqueles que os
guardavam, vendo o que acontecera, fugiram, e foram anunciá-lo na
cidade e nos campos. E saíram a ver o que tinha acontecido, e
vieram ter com Jesus. Acharam então o homem, de quem haviam saído os
demônios, vestido, e em seu juízo, assentado aos pés de Jesus; e
temeram. E os que tinham visto contaram-lhes também como fora
salvo aquele endemoninhado. E toda a multidão da terra dos
gadarenos ao redor lhe rogou que se retirasse deles; porque estavam
possuídos de grande temor. E entrando ele no barco, voltou. E
aquele homem, de quem haviam saído os demônios, rogou-lhe que o
deixasse estar com ele; mas Jesus o despediu, dizendo: Torna
para tua casa, e conta quão grandes coisas te fez Deus. E ele foi
apregoando por toda a cidade quão grandes coisas Jesus lhe tinha
feito.

E aconteceu que, quando voltou Jesus, a multidão o recebeu,
porque todos o estavam esperando. E eis que chegou um homem
de nome Jairo, que era príncipe da sinagoga; e, prostrando-se aos
pés de Jesus, rogava-lhe que entrasse em sua casa; porque
tinha uma filha única, quase de doze anos, que estava à morte. E
indo ele, apertava-o a multidão. E uma mulher, que tinha um
fluxo de sangue, havia doze anos, e gastara com os médicos todos os
seus haveres, e por nenhum pudera ser curada, chegando por
detrás dele, tocou na orla do seu vestido, e logo estancou o fluxo
do seu sangue. E disse Jesus: Quem é que me tocou? E, negando
todos, disse Pedro e os que estavam com ele: Mestre, a multidão te
aperta e te oprime, e dizes: Quem é que me tocou? E disse
Jesus: Alguém me tocou, porque bem conheci que de mim saiu virtude.
Então, vendo a mulher que não podia ocultar-se, aproximou-se
tremendo e, prostrando-se ante ele, declarou-lhe diante de todo o
povo a causa por que lhe havia tocado, e como logo sarara. E
ele lhe disse: Tem bom ânimo, filha, a tua fé te salvou; vai em paz.
Estando ele ainda falando, chegou um dos do príncipe da
sinagoga, dizendo: A tua filha já está morta, não incomodes o
Mestre. Jesus, porém, ouvindo-o, respondeu-lhe, dizendo: Não
temas; crê somente, e será salva. E, entrando em casa, a
ninguém deixou entrar, senão a Pedro, e a Tiago, e a João, e ao pai
e a mãe da menina. E todos choravam, e a pranteavam; e ele
disse: Não choreis; não está morta, mas dorme. E riam-se
dele, sabendo que estava morta. Mas ele, pondo-os todos fora,
e pegando-lhe na mão, clamou, dizendo: Levanta-te, menina. E
o seu espírito voltou, e ela logo se levantou; e Jesus mandou que
lhe dessem de comer. E seus pais ficaram maravilhados; e ele
lhes mandou que a ninguém dissessem o que havia sucedido.

\medskip

\lettrine{9} E, convocando os seus doze discípulos, deu-lhes
virtude e poder sobre todos os demônios, para curarem enfermidades.
E enviou-os a pregar o reino de Deus, e a curar os enfermos.
E disse-lhes: Nada leveis convosco para o caminho, nem bordões,
nem alforje, nem pão, nem dinheiro; nem tenhais duas túnicas. E
em qualquer casa em que entrardes, ficai ali, e de lá saireis. E
se em qualquer cidade vos não receberem, saindo vós dali, sacudi o
pó dos vossos pés, em testemunho contra eles. E, saindo eles,
percorreram todas as aldeias, anunciando o evangelho, e fazendo
curas por toda a parte. E o tetrarca Herodes ouviu todas as
coisas que por ele foram feitas, e estava em dúvida, porque diziam
alguns que João ressuscitara dentre os mortos; e outros que Elias
tinha aparecido; e outros que um profeta dos antigos havia
ressuscitado. E disse Herodes: A João mandei eu degolar; quem é,
pois, este de quem ouço dizer tais coisas? E procurava vê-lo.

E, regressando os apóstolos, contaram-lhe tudo o que tinham
feito. E, tomando-os consigo, retirou-se para um lugar deserto de
uma cidade chamada Betsaida. E, sabendo-o a multidão, o
seguiu; e ele os recebeu, e falava-lhes do reino de Deus, e sarava
os que necessitavam de cura. E já o dia começava a declinar;
então, chegando-se a ele os doze, disseram-lhe: Despede a multidão,
para que, indo aos lugares e aldeias em redor, se agasalhem, e achem
que comer; porque aqui estamos em lugar deserto. Mas ele lhes
disse: Dai-lhes vós de comer. E eles disseram: Não temos senão cinco
pães e dois peixes, salvo se nós próprios formos comprar comida para
todo este povo. Porquanto estavam ali quase cinco mil homens.
Disse, então, aos seus discípulos: Fazei-os assentar, em ranchos de
cinqüenta em cinqüenta. E assim o fizeram, fazendo-os
assentar a todos. E, tomando os cinco pães e os dois peixes,
e olhando para o céu, abençoou-os, e partiu-os, e deu-os aos seus
discípulos para os porem diante da multidão. E comeram todos,
e saciaram-se; e levantaram, do que lhes sobejou, doze alcofas de
pedaços.

E aconteceu que, estando ele só, orando, estavam com ele os
discípulos; e perguntou-lhes, dizendo: Quem diz a multidão que eu
sou? E, respondendo eles, disseram: João o Batista; outros,
Elias, e outros que um dos antigos profetas ressuscitou. E
disse-lhes: E vós, quem dizeis que eu sou? E, respondendo Pedro,
disse: O Cristo de Deus. E, admoestando-os, mandou que a
ninguém referissem isso, dizendo: É necessário que o Filho do
homem padeça muitas coisas, e seja rejeitado dos anciãos e dos
escribas, e seja morto, e ressuscite ao terceiro dia. E dizia
a todos: Se alguém quer vir após mim, negue-se a si mesmo, e tome
cada dia a sua cruz, e siga-me. Porque, qualquer que quiser
salvar a sua vida, perdê-la-á; mas qualquer que, por amor de mim,
perder a sua vida, a salvará. Porque, que aproveita ao homem
granjear o mundo todo, perdendo-se ou prejudicando-se a si mesmo?
Porque, qualquer que de mim e das minhas palavras se
envergonhar, dele se envergonhará o Filho do homem, quando vier na
sua glória, e na do Pai e dos santos anjos. E em verdade vos
digo que, dos que aqui estão, alguns há que não provarão a morte até
que vejam o reino de Deus.

E aconteceu que, quase oito dias depois destas palavras, tomou
consigo a Pedro, a João e a Tiago, e subiu ao monte a orar.
E, estando ele orando, transfigurou-se a aparência do seu
rosto, e a sua roupa ficou branca e mui resplandecente. E eis
que estavam falando com ele dois homens, que eram Moisés e Elias,
os quais apareceram com glória, e falavam da sua morte, a
qual havia de cumprir-se em Jerusalém. E Pedro e os que
estavam com ele estavam carregados de sono; e, quando despertaram,
viram a sua glória e aqueles dois homens que estavam com ele.
E aconteceu que, quando aqueles se apartaram dele, disse
Pedro a Jesus: Mestre, bom é que nós estejamos aqui, e façamos três
tendas: uma para ti, uma para Moisés, e uma para Elias, não sabendo
o que dizia. E, dizendo ele isto, veio uma nuvem que os
cobriu com a sua sombra; e, entrando eles na nuvem, temeram.
E saiu da nuvem uma voz que dizia: Este é o meu amado Filho;
a ele ouvi. E, tendo soado aquela voz, Jesus foi achado só; e
eles calaram-se, e por aqueles dias não contaram a ninguém nada do
que tinham visto.

E aconteceu, no dia seguinte, que, descendo eles do monte, lhes
saiu ao encontro uma grande multidão; e eis que um homem da
multidão clamou, dizendo: Mestre, peço-te que olhes para meu filho,
porque é o único que eu tenho. Eis que um espírito o toma e
de repente clama, e o despedaça até espumar; e só o larga depois de
o ter quebrantado. E roguei aos teus discípulos que o
expulsassem, e não puderam. E Jesus, respondendo, disse: Ó
geração incrédula e perversa! até quando estarei ainda convosco e
vos sofrerei? Traze-me aqui o teu filho. E, quando vinha
chegando, o demônio o derrubou e convulsionou; porém, Jesus
repreendeu o espírito imundo, e curou o menino, e o entregou a seu
pai.

E todos pasmavam da majestade de Deus. E, maravilhando-se todos
de todas as coisas que Jesus fazia, disse aos seus discípulos:
Ponde vós estas palavras em vossos ouvidos, porque o Filho do
homem será entregue nas mãos dos homens. Mas eles não
entendiam esta palavra, que lhes era encoberta, para que a não
compreendessem; e temiam interrogá-lo acerca desta palavra. E
suscitou-se entre eles uma discussão sobre qual deles seria o maior.
Mas Jesus, vendo o pensamento de seus corações, tomou um
menino, pô-lo junto a si, e disse-lhes: Qualquer que receber
este menino em meu nome, recebe-me a mim; e qualquer que me receber
a mim, recebe o que me enviou; porque aquele que entre vós todos for
o menor, esse mesmo é grande. E, respondendo João, disse:
Mestre, vimos um que em teu nome expulsava os demônios, e lho
proibimos, porque não te segue conosco. E Jesus lhes disse:
Não o proibais, porque quem não é contra nós é por nós.

E aconteceu que, completando-se os dias para a sua assunção,
manifestou o firme propósito de ir a Jerusalém. E mandou
mensageiros adiante de si; e, indo eles, entraram numa aldeia de
samaritanos, para lhe prepararem pousada, mas não o
receberam, porque o seu aspecto era como de quem ia a Jerusalém.
E os seus discípulos, Tiago e João, vendo isto, disseram:
Senhor, queres que digamos que desça fogo do céu e os consuma, como
Elias também fez? Voltando-se, porém, repreendeu-os, e disse:
Vós não sabeis de que espírito sois. Porque o Filho do homem
não veio para destruir as almas dos homens, mas para salvá-las. E
foram para outra aldeia.

E aconteceu que, indo eles pelo caminho, lhe disse um: Senhor,
seguir-te-ei para onde quer que fores. E disse-lhe Jesus: As
raposas têm covis, e as aves do céu, ninhos, mas o Filho do homem
não tem onde reclinar a cabeça. E disse a outro: Segue-me.
Mas ele respondeu: Senhor, deixa que primeiro eu vá a enterrar meu
pai. Mas Jesus lhe observou: Deixa aos mortos o enterrar os
seus mortos; porém tu vai e anuncia o reino de Deus. Disse
também outro: Senhor, eu te seguirei, mas deixa-me despedir primeiro
dos que estão em minha casa. E Jesus lhe disse: Ninguém, que
lança mão do arado e olha para trás, é apto para o reino de Deus.

\medskip

\lettrine{10} E depois disto designou o Senhor ainda outros
setenta, e mandou-os adiante da sua face, de dois em dois, a todas
as cidades e lugares aonde ele havia de ir. E dizia-lhes: Grande
é, em verdade, a seara, mas os obreiros são poucos; rogai, pois, ao
Senhor da seara que envie obreiros para a sua seara. Ide; eis
que vos mando como cordeiros ao meio de lobos. Não leveis bolsa,
nem alforje, nem alparcas; e a ninguém saudeis pelo caminho. E,
em qualquer casa onde entrardes, dizei primeiro: Paz seja nesta
casa. E, se ali houver algum filho de paz, repousará sobre ele a
vossa paz; e, se não, voltará para vós. E ficai na mesma casa,
comendo e bebendo do que eles tiverem, pois digno é o obreiro de seu
salário. Não andeis de casa em casa. E, em qualquer cidade em
que entrardes, e vos receberem, comei do que vos for oferecido.
E curai os enfermos que nela houver, e dizei-lhes: É chegado a
vós o reino de Deus. Mas em qualquer cidade, em que entrardes
e vos não receberem, saindo por suas ruas, dizei: Até o pó,
que da vossa cidade se nos pegou, sacudimos sobre vós. Sabei,
contudo, isto, que já o reino de Deus é chegado a vós. E
digo-vos que mais tolerância haverá naquele dia para Sodoma do que
para aquela cidade. Ai de ti, Corazim, ai de ti, Betsaida!
Porque, se em Tiro e em Sidom se fizessem as maravilhas que em vós
foram feitas, já há muito, assentadas em saco e cinza, se teriam
arrependido. Portanto, para Tiro e Sidom haverá menos rigor,
no juízo, do que para vós. E tu, Cafarnaum, que te levantaste
até ao céu, até ao inferno serás abatida. Quem vos ouve a
vós, a mim me ouve; e quem vos rejeita a vós, a mim me rejeita; e
quem a mim me rejeita, rejeita aquele que me enviou.

E voltaram os setenta com alegria, dizendo: Senhor, pelo teu
nome, até os demônios se nos sujeitam. E disse-lhes: Eu via
Satanás, como raio, cair do céu. Eis que vos dou poder para
pisar serpentes e escorpiões, e toda a força do inimigo, e nada vos
fará dano algum. Mas, não vos alegreis porque se vos sujeitem
os espíritos; alegrai-vos antes por estarem os vossos nomes escritos
nos céus. Naquela mesma hora se alegrou Jesus no Espírito
Santo, e disse: Graças te dou, ó Pai, Senhor do céu e da terra, que
escondeste estas coisas aos sábios e inteligentes, e as revelaste às
criancinhas; assim é, ó Pai, porque assim te aprouve. Tudo
por meu Pai me foi entregue; e ninguém conhece quem é o Filho senão
o Pai, nem quem é o Pai senão o Filho, e aquele a quem o Filho o
quiser revelar. E, voltando-se para os discípulos, disse-lhes
em particular: Bem-aventurados os olhos que vêem o que vós vedes.
Pois vos digo que muitos profetas e reis desejaram ver o que
vós vedes, e não o viram; e ouvir o que ouvis, e não o ouviram.

E eis que se levantou um certo doutor da lei, tentando-o, e
dizendo: Mestre, que farei para herdar a vida eterna? E ele
lhe disse: Que está escrito na lei? Como lês? E, respondendo
ele, disse: Amarás ao Senhor teu Deus de todo o teu coração, e de
toda a tua alma, e de todas as tuas forças, e de todo o teu
entendimento, e ao teu próximo como a ti mesmo. E disse-lhe:
Respondeste bem; faze isso, e viverás. Ele, porém, querendo
justificar-se a si mesmo, disse a Jesus: E quem é o meu próximo?
E, respondendo Jesus, disse: Descia um homem de Jerusalém
para Jericó, e caiu nas mãos dos salteadores, os quais o despojaram,
e espancando-o, se retiraram, deixando-o meio morto. E,
ocasionalmente descia pelo mesmo caminho certo sacerdote; e,
vendo-o, passou de largo. E de igual modo também um levita,
chegando àquele lugar, e, vendo-o, passou de largo. Mas um
samaritano, que ia de viagem, chegou ao pé dele e, vendo-o, moveu-se
de íntima compaixão; e, aproximando-se, atou-lhe as feridas,
deitando-lhes azeite e vinho; e, pondo-o sobre a sua cavalgadura,
levou-o para uma estalagem, e cuidou dele; e, partindo no
outro dia, tirou dois denários\footnote{SBTB: dinheiros.}, e deu-os
ao hospedeiro, e disse-lhe: Cuida dele; e tudo o que de mais
gastares eu to pagarei quando voltar. Qual, pois, destes três
te parece que foi o próximo daquele que caiu nas mãos dos
salteadores? E ele disse: O que usou de misericórdia para com
ele. Disse, pois, Jesus: Vai, e faze da mesma maneira.

E aconteceu que, indo eles de caminho, entrou Jesus numa aldeia;
e certa mulher, por nome Marta, o recebeu em sua casa; e
tinha esta uma irmã chamada Maria, a qual, assentando-se também aos
pés de Jesus, ouvia a sua palavra. Marta, porém, andava
distraída em muitos serviços; e, aproximando-se, disse: Senhor, não
se te dá de que minha irmã me deixe servir só? Dize-lhe que me
ajude. E respondendo Jesus, disse-lhe: Marta, Marta, estás
ansiosa e afadigada com muitas coisas, mas uma só é necessária;
e Maria escolheu a boa parte, a qual não lhe será tirada.

\medskip

\lettrine{11} E aconteceu que, estando ele a orar num certo
lugar, quando acabou, lhe disse um dos seus discípulos: Senhor,
ensina-nos a orar, como também João ensinou aos seus discípulos.
E ele lhes disse: Quando orardes, dizei: Pai nosso, que estás
nos céus, santificado seja o teu nome; venha o teu reino; seja feita
a tua vontade, assim na terra, como no céu. Dá-nos cada dia o
nosso pão cotidiano; e perdoa-nos os nossos pecados, pois também
nós perdoamos a qualquer que nos deve, e não nos conduzas em
tentação, mas livra-nos do mal. Disse-lhes também: Qual de vós
terá um amigo, e, se for procurá-lo à meia-noite, e lhe disser:
Amigo, empresta-me três pães, pois que um amigo meu chegou a
minha casa, vindo de caminho, e não tenho que apresentar-lhe; se
ele, respondendo de dentro, disser: Não me importunes; já está a
porta fechada, e os meus filhos estão comigo na cama; não posso
levantar-me para tos dar; digo-vos que, ainda que não se levante
a dar-lhos, por ser seu amigo, levantar-se-á, todavia, por causa da
sua importunação, e lhe dará tudo o que houver
mister\footnote{Necessidade; urgência.}. E eu vos digo a vós:
Pedi, e dar-se-vos-á; buscai, e achareis; batei, e abrir-se-vos-á;
porque qualquer que pede recebe; e quem busca acha; e a quem
bate abrir-se-lhe-á. E qual o pai de entre vós que, se o
filho lhe pedir pão, lhe dará uma pedra? Ou, também, se lhe pedir
peixe, lhe dará por peixe uma serpente? Ou, também, se lhe
pedir um ovo, lhe dará um escorpião? Pois se vós, sendo maus,
sabeis dar boas dádivas aos vossos filhos, quanto mais dará o Pai
celestial o Espírito Santo àqueles que lho pedirem?

E estava ele expulsando um demônio, o qual era mudo. E aconteceu
que, saindo o demônio, o mudo falou; e maravilhou-se a multidão.
Mas alguns deles diziam: Ele expulsa os demônios por Belzebu,
príncipe dos demônios. E outros, tentando-o, pediam-lhe um
sinal do céu. Mas, conhecendo ele os seus pensamentos,
disse-lhes: Todo o reino, dividido contra si mesmo, será assolado; e
a casa, dividida contra si mesma, cairá. E, se também Satanás
está dividido contra si mesmo, como subsistirá o seu reino? Pois
dizeis que eu expulso os demônios por Belzebu. E, se eu
expulso os demônios por Belzebu, por quem os expulsam vossos filhos?
Eles, pois, serão os vossos juízes. Mas, se eu expulso os
demônios pelo dedo de Deus, certamente a vós é chegado o reino de
Deus. Quando o valente guarda, armado, a sua casa, em
segurança está tudo quanto tem; mas, sobrevindo outro mais
valente do que ele, e vencendo-o, tira-lhe toda a sua armadura em
que confiava, e reparte os seus despojos. Quem não é comigo é
contra mim; e quem comigo não ajunta, espalha. Quando o
espírito imundo tem saído do homem, anda por lugares secos, buscando
repouso; e, não o achando, diz: Tornarei para minha casa, de onde
saí. E, chegando, acha-a varrida e adornada. Então
vai, e leva consigo outros sete espíritos piores do que ele e,
entrando, habitam ali; e o último estado desse homem é pior do que o
primeiro.

E aconteceu que, dizendo ele estas coisas, uma mulher dentre a
multidão, levantando a voz, lhe disse: Bem-aventurado o ventre que
te trouxe e os peitos em que mamaste. Mas ele disse: Antes
bem-aventurados os que ouvem a palavra de Deus e a guardam.

E, ajuntando-se a multidão, começou a dizer: Maligna é esta
geração; ela pede um sinal; e não lhe será dado outro sinal, senão o
sinal do profeta Jonas; porquanto, assim como Jonas foi sinal
para os ninivitas, assim o Filho do homem o será também para esta
geração. A rainha do sul se levantará no juízo com os homens
desta geração, e os condenará; pois até dos confins da terra veio
ouvir a sabedoria de Salomão; e eis aqui está quem é maior do que
Salomão. Os homens de Nínive se levantarão no juízo com esta
geração, e a condenarão; pois se converteram com a pregação de
Jonas; e eis aqui está quem é maior do que Jonas. E ninguém,
acendendo uma candeia, a põe em oculto, nem debaixo do alqueire, mas
no velador, para que os que entram vejam a luz. A candeia do
corpo é o olho. Sendo, pois, o teu olho simples, também todo o teu
corpo será luminoso; mas, se for mau, também o teu corpo será
tenebroso. Vê, pois, que a luz que em ti há não sejam trevas.
Se, pois, todo o teu corpo é luminoso, não tendo em trevas
parte alguma, todo será luminoso, como quando a candeia te ilumina
com o seu resplendor.

E, estando ele ainda falando, rogou-lhe um fariseu que fosse
jantar com ele; e, entrando, assentou-se à mesa. Mas o
fariseu admirou-se, vendo que não se lavara antes de jantar.
E o Senhor lhe disse: Agora vós, os fariseus, limpais o
exterior do copo e do prato; mas o vosso interior está cheio de
rapina e maldade. Loucos! Quem fez o exterior não fez também
o interior? Antes dai esmola do que tiverdes, e eis que tudo
vos será limpo. Mas ai de vós, fariseus, que dizimais a
hortelã, e a arruda, e toda a hortaliça, e desprezais o juízo e o
amor de Deus. Importava fazer estas coisas, e não deixar as outras.
Ai de vós, fariseus, que amais os primeiros assentos nas
sinagogas, e as saudações nas praças. Ai de vós, escribas e
fariseus, hipócritas! que sois como as sepulturas que não aparecem,
e os homens que sobre elas andam não o sabem. E, respondendo
um dos doutores da lei, disse-lhe: Mestre, quando dizes isso, também
nos afrontas a nós. E ele lhe disse: Ai de vós também,
doutores da lei, que carregais os homens com cargas difíceis de
transportar, e vós mesmos nem ainda com um dos vossos dedos tocais
essas cargas. Ai de vós que edificais os sepulcros dos
profetas, e vossos pais os mataram. Bem testificais, pois,
que consentis nas obras de vossos pais; porque eles os mataram, e
vós edificais os seus sepulcros. Por isso diz também a
sabedoria de Deus: Profetas e apóstolos lhes mandarei; e eles
matarão uns, e perseguirão outros; para que desta geração
seja requerido o sangue de todos os profetas que, desde a fundação
do mundo, foi derramado; desde o sangue de Abel, até ao
sangue de Zacarias, que foi morto entre o altar e o templo; assim,
vos digo, será requerido desta geração. Ai de vós, doutores
da lei, que tirastes a chave da ciência; vós mesmos não entrastes, e
impedistes os que entravam. E, dizendo-lhes ele isto,
começaram os escribas e os fariseus a apertá-lo fortemente, e a
fazê-lo falar acerca de muitas coisas, armando-lhe ciladas, e
procurando apanhar da sua boca alguma coisa para o acusarem.

\medskip

\lettrine{12} Ajuntando-se entretanto muitos milhares de
pessoas, de sorte que se atropelavam uns aos outros, começou a dizer
aos seus discípulos: Acautelai-vos primeiramente do fermento dos
fariseus, que é a hipocrisia. Mas nada há encoberto que não haja
de ser descoberto; nem oculto, que não haja de ser sabido.
Porquanto tudo o que em trevas dissestes, à luz será ouvido; e o
que falastes ao ouvido no gabinete, sobre os telhados será
apregoado. E digo-vos, amigos meus: Não temais os que matam o
corpo e, depois, não têm mais que fazer. Mas eu vos mostrarei a
quem deveis temer; temei aquele que, depois de matar, tem poder para
lançar no inferno; sim, vos digo, a esse temei. Não se vendem
cinco passarinhos por dois asses\footnote{SBTB: ceitis.}? E nenhum
deles está esquecido diante de Deus. E até os cabelos da vossa
cabeça estão todos contados. Não temais pois; mais valeis vós do que
muitos passarinhos. E digo-vos que todo aquele que me confessar
diante dos homens também o Filho do homem o confessará diante dos
anjos de Deus. Mas quem me negar diante dos homens será negado
diante dos anjos de Deus. E a todo aquele que disser uma
palavra contra o Filho do homem ser-lhe-á perdoada, mas ao que
blasfemar contra o Espírito Santo não lhe será perdoado. E,
quando vos conduzirem às sinagogas, aos magistrados e potestades,
não estejais solícitos de como ou do que haveis de responder, nem do
que haveis de dizer. Porque na mesma hora vos ensinará o
Espírito Santo o que vos convenha falar.

E disse-lhe um da multidão: Mestre, dize a meu irmão que reparta
comigo a herança. Mas ele lhe disse: Homem, quem me pôs a mim
por juiz ou repartidor entre vós? E disse-lhes: Acautelai-vos
e guardai-vos da avareza; porque a vida de qualquer não consiste na
abundância do que possui. E propôs-lhe uma parábola, dizendo:
A herdade\footnote{Grande propriedade rural, composta, em geral, de
terras de semeadura, montados e casa de habitação; Quinta; Ant.
Herança.} de um homem rico tinha produzido com abundância; e
ele arrazoava consigo mesmo, dizendo: Que farei? Não tenho onde
recolher os meus frutos. E disse: Farei isto: Derrubarei os
meus celeiros, e edificarei outros maiores, e ali recolherei todas
as minhas novidades e os meus bens; e direi a minha alma:
Alma, tens em depósito muitos bens para muitos anos; descansa, come,
bebe e folga. Mas Deus lhe disse: Louco! esta noite te
pedirão a tua alma; e o que tens preparado, para quem será?
Assim é aquele que para si ajunta tesouros, e não é rico para
com Deus.

E disse aos seus discípulos: Portanto vos digo: Não estejais
apreensivos pela vossa vida, sobre o que comereis, nem pelo corpo,
sobre o que vestireis. Mais é a vida do que o sustento, e o
corpo mais do que as vestes. Considerai os corvos, que nem
semeiam, nem segam, nem têm despensa nem celeiro, e Deus os
alimenta; quanto mais valeis vós do que as aves? E qual de
vós, sendo solícito, pode acrescentar um côvado à sua estatura?
Pois, se nem ainda podeis as coisas mínimas, por que estais
ansiosos pelas outras? Considerai os lírios, como eles
crescem; não trabalham, nem fiam; e digo-vos que nem ainda Salomão,
em toda a sua glória, se vestiu como um deles. E, se Deus
assim veste a erva que hoje está no campo e amanhã é lançada no
forno, quanto mais a vós, homens de pouca fé? Não pergunteis,
pois, que haveis de comer, ou que haveis de beber, e não andeis
inquietos. Porque as nações do mundo buscam todas essas
coisas; mas vosso Pai sabe que precisais delas. Buscai antes
o reino de Deus, e todas estas coisas vos serão acrescentadas.
Não temais, ó pequeno rebanho, porque a vosso Pai agradou
dar-vos o reino. Vendei o que tendes, e dai esmolas. Fazei
para vós bolsas que não se envelheçam; tesouro nos céus que nunca
acabe, aonde não chega ladrão e a traça não rói. Porque, onde
estiver o vosso tesouro, ali estará também o vosso coração.
Estejam cingidos os vossos lombos, e acesas as vossas
candeias. E sede vós semelhantes aos homens que esperam o seu
senhor, quando houver de voltar das bodas, para que, quando vier, e
bater, logo possam abrir-lhe. Bem-aventurados aqueles servos,
os quais, quando o Senhor vier, achar vigiando! Em verdade vos digo
que se cingirá, e os fará assentar à mesa e, chegando-se, os
servirá. E, se vier na segunda vigília, e se vier na terceira
vigília, e os achar assim, bem-aventurados são os tais servos.
Sabei, porém, isto: que, se o pai de família soubesse a que
hora havia de vir o ladrão, vigiaria, e não deixaria minar a sua
casa. Portanto, estai vós também apercebidos; porque virá o
Filho do homem à hora que não imaginais.

E disse-lhe Pedro: Senhor, dizes essa parábola a nós, ou também a
todos? E disse o Senhor: Qual é, pois, o mordomo fiel e
prudente, a quem o senhor pôs sobre os seus servos, para lhes dar a
tempo a ração? Bem-aventurado aquele servo a quem o seu
senhor, quando vier, achar fazendo assim. Em verdade vos digo
que sobre todos os seus bens o porá. Mas, se aquele servo
disser em seu coração: O meu senhor tarda em vir; e começar a
espancar os criados e criadas, e a comer, e a beber, e a
embriagar-se, virá o senhor daquele servo no dia em que o não
espera, e numa hora que ele não sabe, e separá-lo-á, e lhe dará a
sua parte com os infiéis. E o servo que soube a vontade do
seu senhor, e não se aprontou, nem fez conforme a sua vontade, será
castigado com muitos açoites; mas o que a não soube, e fez
coisas dignas de açoites, com poucos açoites será castigado. E, a
qualquer que muito for dado, muito se lhe pedirá, e ao que muito se
lhe confiou, muito mais se lhe pedirá. Vim lançar fogo na
terra; e que mais quero, se já está aceso? Importa, porém,
que seja batizado com um certo batismo; e como me angustio até que
venha a cumprir-se! Cuidais vós que vim trazer paz à terra?
Não, vos digo, mas antes dissensão; porque daqui em diante
estarão cinco divididos numa casa: três contra dois, e dois contra
três. O pai estará dividido contra o filho, e o filho contra
o pai; a mãe contra a filha, e a filha contra a mãe; a sogra contra
sua nora, e a nora contra sua sogra.

E dizia também à multidão: Quando vedes a nuvem que vem do
ocidente, logo dizeis: Lá vem chuva, e assim sucede. E,
quando assopra o sul, dizeis: Haverá calma; e assim sucede.
Hipócritas, sabeis discernir a face da terra e do céu; como
não sabeis então discernir este tempo? E por que não julgais
também por vós mesmos o que é justo? Quando, pois, vais com o
teu adversário ao magistrado, procura livrar-te dele no caminho;
para que não suceda que te conduza ao juiz, e o juiz te entregue ao
meirinho\footnote{Funcionário da justiça.}, e o meirinho te encerre
na prisão. Digo-te que não sairás dali enquanto não pagares o
derradeiro lepto\footnote{SBTB: ceitil}.

\medskip

\lettrine{13} E, naquele mesmo tempo, estavam presentes ali
alguns que lhe falavam dos galileus, cujo sangue Pilatos misturara
com os seus sacrifícios. E, respondendo Jesus, disse-lhes:
Cuidais vós que esses galileus foram mais pecadores do que todos os
galileus, por terem padecido tais coisas? Não, vos digo; antes,
se não vos arrependerdes, todos de igual modo perecereis. E
aqueles dezoito, sobre os quais caiu a torre de Siloé e os matou,
cuidais que foram mais culpados do que todos quantos homens habitam
em Jerusalém? Não, vos digo; antes, se não vos arrependerdes,
todos de igual modo perecereis.

E dizia esta parábola: Um certo homem tinha uma figueira plantada
na sua vinha, e foi procurar nela fruto, não o achando; e disse
ao vinhateiro: Eis que há três anos venho procurar fruto nesta
figueira, e não o acho. Corta-a; por que ocupa ainda a terra
inutilmente? E, respondendo ele, disse-lhe: Senhor, deixa-a este
ano, até que eu a escave e a esterque; e, se der fruto, ficará
e, se não, depois a mandarás cortar.

E ensinava no sábado, numa das sinagogas. E eis que estava
ali uma mulher que tinha um espírito de enfermidade, havia já
dezoito anos; e andava curvada, e não podia de modo algum
endireitar-se. E, vendo-a Jesus, chamou-a a si, e disse-lhe:
Mulher, estás livre da tua enfermidade. E pôs as mãos sobre
ela, e logo se endireitou, e glorificava a Deus. E, tomando a
palavra o príncipe da sinagoga, indignado porque Jesus curava no
sábado, disse à multidão: Seis dias há em que é mister trabalhar;
nestes, pois, vinde para serdes curados, e não no dia de sábado.
Respondeu-lhe, porém, o Senhor, e disse: Hipócrita, no sábado
não desprende da manjedoura cada um de vós o seu boi, ou jumento, e
não o leva a beber? E não convinha soltar desta prisão, no
dia de sábado, esta filha de Abraão, a qual há dezoito anos Satanás
tinha presa? E, dizendo ele isto, todos os seus adversários
ficaram envergonhados, e todo o povo se alegrava por todas as coisas
gloriosas que eram feitas por ele.

E dizia: A que é semelhante o reino de Deus, e a que o
compararei? É semelhante ao grão de mostarda que um homem,
tomando-o, lançou na sua horta; e cresceu, e fez-se grande árvore, e
em seus ramos se aninharam as aves do céu. E disse outra vez:
A que compararei o reino de Deus? É semelhante ao fermento
que uma mulher, tomando-o, escondeu em três medidas de farinha, até
que tudo levedou. E percorria as cidades e as aldeias,
ensinando, e caminhando para Jerusalém.

E disse-lhe um: Senhor, são poucos os que se salvam? E ele lhe
respondeu: Porfiai\footnote{Porfiar: discutir acaloradamente;
altercar, contender. Competir ou lutar por (algo); disputar. Não
aceitar evidências contrárias ou a opinião de (outrem); obstinar-se,
insistir, teimar.}
 por entrar pela porta estreita; porque eu vos
digo que muitos procurarão entrar, e não poderão. Quando o
pai de família se levantar e cerrar a porta, e começardes, de fora,
a bater à porta, dizendo: Senhor, Senhor, abre-nos; e, respondendo
ele, vos disser: Não sei de onde vós sois; então começareis a
dizer: Temos comido e bebido na tua presença, e tu tens ensinado nas
nossas ruas. E ele vos responderá: Digo-vos que não sei de
onde vós sois; apartai-vos de mim, vós todos os que praticais a
iniqüidade. Ali haverá choro e ranger de dentes, quando
virdes Abraão, e Isaque, e Jacó, e todos os profetas no reino de
Deus, e vós lançados fora. E virão do oriente, e do ocidente,
e do norte, e do sul, e assentar-se-ão à mesa no reino de Deus.
E eis que derradeiros há que serão os primeiros; e primeiros
há que serão os derradeiros.

Naquele mesmo dia chegaram uns fariseus, dizendo-lhe: Sai, e
retira-te daqui, porque Herodes quer matar-te. E
respondeu-lhes: Ide, e dizei àquela raposa: Eis que eu expulso
demônios, e efetuo curas, hoje e amanhã, e no terceiro dia sou
consumado. Importa, porém, caminhar hoje, amanhã, e no dia
seguinte, para que não suceda que morra um profeta fora de
Jerusalém. Jerusalém, Jerusalém, que matas os profetas, e
apedrejas os que te são enviados! Quantas vezes quis eu ajuntar os
teus filhos, como a galinha os seus pintos debaixo das asas, e não
quiseste? Eis que a vossa casa se vos deixará deserta. E em
verdade vos digo que não me vereis até que venha o tempo em que
digais: Bendito aquele que vem em nome do Senhor.

\medskip

\lettrine{14} Aconteceu num sábado que, entrando ele em casa
de um dos principais dos fariseus para comer pão, eles o estavam
observando. E eis que estava ali diante dele um certo homem
hidrópico\footnote{Hidropisia: Acumulação anormal de líquido seroso
em tecidos ou em cavidade do corpo.}. E Jesus, tomando a
palavra, falou aos doutores da lei, e aos fariseus, dizendo: É
lícito curar no sábado? Eles, porém, calaram-se. E, tomando-o, o
curou e despediu. E disse-lhes: Qual será de vós o que,
caindo-lhe num poço, em dia de sábado, o jumento ou o boi, o não
tire logo? E nada lhe podiam replicar sobre isto.

E disse aos convidados uma parábola, reparando como escolhiam os
primeiros assentos, dizendo-lhes: Quando por alguém fores
convidado às bodas, não te assentes no primeiro lugar; não aconteça
que esteja convidado outro mais digno do que tu; e, vindo o que
te convidou a ti e a ele, te diga: Dá o lugar a este; e então, com
vergonha, tenhas de tomar o derradeiro lugar. Mas, quando
fores convidado, vai, e assenta-te no derradeiro lugar, para que,
quando vier o que te convidou, te diga: Amigo, sobe mais para o
alto\footnote{SBTB: ``sobe mais para cima'' - pleonasmo vicioso.
King James: ``Friend, go up higher''.}. Então terás honra diante dos
que estiverem contigo à mesa. Porquanto qualquer que a si
mesmo se exaltar será humilhado, e aquele que a si mesmo se humilhar
será exaltado. E dizia também ao que o tinha convidado:
Quando deres um jantar, ou uma ceia, não chames os teus amigos, nem
os teus irmãos, nem os teus parentes, nem vizinhos ricos, para que
não suceda que também eles te tornem a convidar, e te seja isso
recompensado. Mas, quando fizeres convite, chama os pobres,
aleijados, mancos e cegos, e serás bem-aventurado; porque
eles não têm com que to recompensar; mas recompensado te será na
ressurreição dos justos.

E, ouvindo isto, um dos que estavam com ele à mesa, disse-lhe:
Bem-aventurado o que comer pão no reino de Deus. Porém, ele
lhe disse: Um certo homem fez uma grande ceia, e convidou a muitos.
E à hora da ceia mandou o seu servo dizer aos convidados:
Vinde, que já tudo está preparado. E todos à uma começaram a
escusar-se. Disse-lhe o primeiro: Comprei um campo, e importa ir
vê-lo; rogo-te que me hajas por escusado. E outro disse:
Comprei cinco juntas de bois, e vou experimentá-los; rogo-te que me
hajas por escusado. E outro disse: Casei, e portanto não
posso ir. E, voltando aquele servo, anunciou estas coisas ao
seu senhor. Então o pai de família, indignado, disse ao seu servo:
Sai depressa pelas ruas e bairros da cidade, e traze aqui os pobres,
e aleijados, e mancos e cegos. E disse o servo: Senhor, feito
está como mandaste; e ainda há lugar. E disse o senhor ao
servo: Sai pelos caminhos e valados, e força-os a entrar, para que a
minha casa se encha. Porque eu vos digo que nenhum daqueles
homens que foram convidados provará a minha ceia.

Ora, ia com ele uma grande multidão; e, voltando-se, disse-lhe:
Se alguém vier a mim, e não aborrecer\footnote{Odiar. KJ: to
hate.} a seu pai, e mãe, e mulher, e filhos, e irmãos, e irmãs, e
ainda também a sua própria vida, não pode ser meu discípulo.
E qualquer que não levar a sua cruz, e não vier após mim, não
pode ser meu discípulo. Pois qual de vós, querendo edificar
uma torre, não se assenta primeiro a fazer as contas dos gastos,
para ver se tem com que a acabar? Para que não aconteça que,
depois de haver posto os alicerces, e não a podendo acabar, todos os
que a virem comecem a escarnecer dele, dizendo: Este homem
começou a edificar e não pôde acabar. Ou qual é o rei que,
indo à guerra a pelejar contra outro rei, não se assenta primeiro a
tomar conselho sobre se com dez mil pode sair ao encontro do que vem
contra ele com vinte mil? De outra maneira, estando o outro
ainda longe, manda embaixadores, e pede condições de paz.
Assim, pois, qualquer de vós, que não renuncia a tudo quanto
tem, não pode ser meu discípulo. Bom é o sal; mas, se o sal
degenerar, com que se há de salgar? Nem presta para a terra,
nem para o monturo; lançam-no fora. Quem tem ouvidos para ouvir,
ouça.

\medskip

\lettrine{15} E chegavam-se a ele todos os publicanos e
pecadores para o ouvir. E os fariseus e os escribas murmuravam,
dizendo: Este recebe pecadores, e come com eles. E ele lhes
propôs esta parábola, dizendo: Que homem dentre vós, tendo cem
ovelhas, e perdendo uma delas, não deixa no deserto as noventa e
nove, e não vai após a perdida até que venha a achá-la? E
achando-a, a põe sobre os seus ombros, cheio de
júbilo\footnote{SBTB: gostoso. KJ: And when he hath found it, he
layeth it on his shoulders, rejoicing. RA:  Achando-a, põe-na sobre
os ombros, cheio de júbilo. RC: E, achando-a, a põe sobre seus
ombros, cheio de júbilo.}; e, chegando a casa, convoca os amigos
e vizinhos, dizendo-lhes: Alegrai-vos comigo, porque já achei a
minha ovelha perdida. Digo-vos que assim haverá alegria no céu
por um pecador que se arrepende, mais do que por noventa e nove
justos que não necessitam de arrependimento. Ou qual a mulher
que, tendo dez dracmas, se perder uma dracma, não acende a candeia,
e varre a casa, e busca com diligência até a achar? E achando-a,
convoca as amigas e vizinhas, dizendo: Alegrai-vos comigo, porque já
achei a dracma perdida. Assim vos digo que há alegria diante
dos anjos de Deus por um pecador que se arrepende.

E disse: Um certo homem tinha dois filhos; e o mais moço
deles disse ao pai: Pai, dá-me a parte dos bens que me pertence. E
ele repartiu por eles a fazenda. E, poucos dias depois, o
filho mais novo, ajuntando tudo, partiu para uma terra longínqua, e
ali desperdiçou os seus bens, vivendo dissolutamente. E,
havendo ele gastado tudo, houve naquela terra uma grande fome, e
começou a padecer necessidades. E foi, e chegou-se a um dos
cidadãos daquela terra, o qual o mandou para os seus campos, a
apascentar porcos. E desejava encher o seu estômago com as
bolotas que os porcos comiam, e ninguém lhe dava nada. E,
tornando em si, disse: Quantos jornaleiros de meu pai têm abundância
de pão, e eu aqui pereço de fome! Levantar-me-ei, e irei ter
com meu pai, e dir-lhe-ei: Pai, pequei contra o céu e perante ti;
já não sou digno de ser chamado teu filho; faze-me como um
dos teus jornaleiros. E, levantando-se, foi para seu pai; e,
quando ainda estava longe, viu-o seu pai, e se moveu de íntima
compaixão e, correndo, lançou-se-lhe ao pescoço e o beijou. E
o filho lhe disse: Pai, pequei contra o céu e perante ti, e já não
sou digno de ser chamado teu filho. Mas o pai disse aos seus
servos: Trazei depressa a melhor roupa; e vesti-lho, e ponde-lhe um
anel na mão, e alparcas nos pés; e trazei o bezerro cevado, e
matai-o; e comamos, e alegremo-nos; porque este meu filho
estava morto, e reviveu, tinha-se perdido, e foi achado. E começaram
a alegrar-se. E o seu filho mais velho estava no campo; e
quando veio, e chegou perto de casa, ouviu a música e as danças.
E, chamando um dos servos, perguntou-lhe que era aquilo.
E ele lhe disse: Veio teu irmão; e teu pai matou o bezerro
cevado, porque o recebeu são e salvo. Mas ele se indignou, e
não queria entrar. E saindo o pai, instava com ele. Mas,
respondendo ele, disse ao pai: Eis que te sirvo há tantos anos, sem
nunca transgredir o teu mandamento, e nunca me deste um cabrito para
alegrar-me com os meus amigos; vindo, porém, este teu filho,
que desperdiçou os teus bens com as meretrizes, mataste-lhe o
bezerro cevado. E ele lhe disse: Filho, tu sempre estás
comigo, e todas as minhas coisas são tuas; mas era justo
alegrarmo-nos e folgarmos, porque este teu irmão estava morto, e
reviveu; e tinha-se perdido, e achou-se.

\medskip

\lettrine{16} E dizia também aos seus discípulos: Havia um
certo homem rico, o qual tinha um mordomo; e este foi acusado
perante ele de dissipar os seus bens. E ele, chamando-o,
disse-lhe: Que é isto que ouço de ti? Dá contas da tua mordomia,
porque já não poderás ser mais meu mordomo. E o mordomo disse
consigo: Que farei, pois que o meu senhor me tira a mordomia? Cavar,
não posso; de mendigar, tenho vergonha. Eu sei o que hei de
fazer, para que, quando for desapossado da mordomia, me recebam em
suas casas. E, chamando a si cada um dos devedores do seu
Senhor, disse ao primeiro: Quanto deves ao meu senhor? E ele
respondeu: Cem medidas de azeite. E disse-lhe: Toma a tua obrigação,
e assentando-te já, escreve cinqüenta. Disse depois a outro: E
tu, quanto deves? E ele respondeu: Cem alqueires de trigo. E
disse-lhe: Toma a tua obrigação, e escreve oitenta. E louvou
aquele senhor o injusto mordomo por haver procedido prudentemente,
porque os filhos deste mundo são mais prudentes na sua geração do
que os filhos da luz. E eu vos digo: Granjeai amigos com as
riquezas da injustiça; para que, quando estas vos faltarem, vos
recebam eles nos tabernáculos eternos. Quem é fiel no mínimo,
também é fiel no muito; quem é injusto no mínimo, também é injusto
no muito. Pois, se nas riquezas injustas não fostes fiéis,
quem vos confiará as verdadeiras? E, se no alheio não fostes
fiéis, quem vos dará o que é vosso? Nenhum servo pode servir
dois senhores; porque, ou há de odiar um e amar o outro, ou se há de
chegar a um e desprezar o outro. Não podeis servir a Deus e a
Mamom\footnote{Essa palavra vem do aramaico, que aparentemente
significa "riqueza" ou "propriedade". Jesus usou a palavra
personificada a fim de indicar o deus das riquezas carnais em
contraste com o Deus dos céus, que possui as verdadeiras riquezas e
que quer conferi-las a homens que vivam de conformidade com as suas
regras.}. E os fariseus, que eram avarentos, ouviam todas
estas coisas, e zombavam dele. E disse-lhes: Vós sois os que
vos justificais a vós mesmos diante dos homens, mas Deus conhece os
vossos corações, porque o que entre os homens é elevado, perante
Deus é abominação. A lei e os profetas duraram até João;
desde então é anunciado o reino de Deus, e todo o homem emprega
força para entrar nele. E é mais fácil passar o céu e a terra
do que cair um til da lei. Qualquer que deixa sua mulher, e
casa com outra, adultera; e aquele que casa com a repudiada pelo
marido, adultera também.

Ora, havia um homem rico, e vestia-se de púrpura e de linho
finíssimo, e vivia todos os dias regalada e esplendidamente.
Havia também um certo mendigo, chamado Lázaro, que jazia
cheio de chagas à porta daquele; e desejava alimentar-se com
as migalhas que caíam da mesa do rico; e os próprios cães vinham
lamber-lhe as chagas. E aconteceu que o mendigo morreu, e foi
levado pelos anjos para o seio de Abraão; e morreu também o rico, e
foi sepultado. E no inferno, ergueu os olhos, estando em
tormentos, e viu ao longe Abraão, e Lázaro no seu seio. E,
clamando, disse: Pai Abraão, tem misericórdia de mim, e manda a
Lázaro, que molhe na água a ponta do seu dedo e me refresque a
língua, porque estou atormentado nesta chama. Disse, porém,
Abraão: Filho, lembra-te de que recebeste os teus bens em tua vida,
e Lázaro somente males; e agora este é consolado e tu atormentado.
E, além disso, está posto um grande abismo entre nós e vós,
de sorte que os que quisessem passar daqui para vós não poderiam,
nem tampouco os de lá passar para cá. E disse ele: Rogo-te,
pois, ó pai, que o mandes à casa de meu pai, pois tenho cinco
irmãos; para que lhes dê testemunho, a fim de que não venham também
para este lugar de tormento. Disse-lhe Abraão: Têm Moisés e
os profetas; ouçam-nos. E disse ele: Não, pai Abraão; mas, se
algum dentre os mortos fosse ter com eles, arrepender-se-iam.
Porém, Abraão lhe disse: Se não ouvem a Moisés e aos
profetas, tampouco acreditarão, ainda que algum dos mortos
ressuscite.

\medskip

\lettrine{17} E disse aos discípulos: É impossível que não
venham escândalos, mas ai daquele por quem vierem! Melhor lhe
fora que lhe pusessem ao pescoço uma mó de atafona\footnote{Moinho
manual ou movido por cavalgaduras; azenha.}, e fosse lançado ao mar,
do que fazer tropeçar um destes pequenos. Olhai por vós mesmos.
E, se teu irmão pecar contra ti, repreende-o e, se ele se
arrepender, perdoa-lhe. E, se pecar contra ti sete vezes no dia,
e sete vezes no dia vier ter contigo, dizendo: Arrependo-me;
perdoa-lhe. Disseram então os apóstolos ao Senhor:
Acrescenta-nos a fé. E disse o Senhor: Se tivésseis fé como um
grão de mostarda, diríeis a esta amoreira: Desarraiga-te daqui, e
planta-te no mar; e ela vos obedeceria. E qual de vós terá um
servo a lavrar ou a apascentar gado, a quem, voltando ele do campo,
diga: Chega-te, e assenta-te à mesa? E não lhe diga antes:
Prepara-me a ceia, e cinge-te, e serve-me até que tenha comido e
bebido, e depois comerás e beberás tu? Porventura dá graças ao
tal servo, porque fez o que lhe foi mandado? Creio que não.
Assim também vós, quando fizerdes tudo o que vos for mandado,
dizei: Somos servos inúteis, porque fizemos somente o que devíamos
fazer.

E aconteceu que, indo ele a Jerusalém, passou pelo meio de
Samaria e da Galiléia; e, entrando numa certa aldeia,
saíram-lhe ao encontro dez homens leprosos, os quais pararam de
longe; e levantaram a voz, dizendo: Jesus, Mestre, tem
misericórdia de nós. E ele, vendo-os, disse-lhes: Ide, e
mostrai-vos aos sacerdotes. E aconteceu que, indo eles, ficaram
limpos. E um deles, vendo que estava são, voltou glorificando
a Deus em alta voz; e caiu aos seus pés, com o rosto em
terra, dando-lhe graças; e este era samaritano. E,
respondendo Jesus, disse: Não foram dez os limpos? E onde estão os
nove? Não houve quem voltasse para dar glória a Deus senão
este estrangeiro? E disse-lhe: Levanta-te, e vai; a tua fé te
salvou.

E, interrogado pelos fariseus sobre quando havia de vir o reino
de Deus, respondeu-lhes, e disse: O reino de Deus não vem com
aparência exterior. Nem dirão: Ei-lo aqui, ou: Ei-lo ali;
porque eis que o reino de Deus está entre vós. E disse aos
discípulos: Dias virão em que desejareis ver um dos dias do Filho do
homem, e não o vereis. E dir-vos-ão: Ei-lo aqui, ou: Ei-lo
ali. Não vades, nem os sigais; porque, como o relâmpago
ilumina desde uma extremidade inferior do céu até à outra
extremidade, assim será também o Filho do homem no seu dia.
Mas primeiro convém que ele padeça muito, e seja reprovado
por esta geração. E, como aconteceu nos dias de Noé, assim
será também nos dias do Filho do homem. Comiam, bebiam,
casavam, e davam-se em casamento, até ao dia em que Noé entrou na
arca, e veio o dilúvio, e os consumiu a todos. Como também da
mesma maneira aconteceu nos dias de Ló: Comiam, bebiam, compravam,
vendiam, plantavam e edificavam; mas no dia em que Ló saiu de
Sodoma choveu do céu fogo e enxofre, e os consumiu a todos.
Assim será no dia em que o Filho do homem se há de
manifestar. Naquele dia, quem estiver no telhado, tendo as
suas alfaias\footnote{Móvel ou utensílio de uso ou adorno doméstico;
enfeite, adorno, atavio; utensílio agrícola; paramento de igreja.}
em casa, não desça a tomá-las; e, da mesma sorte, o que estiver no
campo não volte para trás. Lembrai-vos da mulher de Ló.
Qualquer que procurar salvar a sua vida, perdê-la-á, e
qualquer que a perder, salvá-la-á. Digo-vos que naquela noite
estarão dois numa cama; um será tomado, e outro será deixado.
Duas estarão juntas, moendo; uma será tomada, e outra será
deixada. Dois estarão no campo; um será tomado, o outro será
deixado. E, respondendo, disseram-lhe: Onde, Senhor? E ele
lhes disse: Onde estiver o corpo, aí se ajuntarão as águias.

\medskip

\lettrine{18} E contou-lhes também uma parábola sobre o dever
de orar sempre, e nunca desfalecer, dizendo: Havia numa cidade
um certo juiz, que nem a Deus temia, nem respeitava o homem.
Havia também, naquela mesma cidade, uma certa viúva, que ia ter
com ele, dizendo: Faze-me justiça contra o meu adversário. E por
algum tempo não quis atendê-la; mas depois disse consigo: Ainda que
não temo a Deus, nem respeito os homens, todavia, como esta
viúva me molesta, hei de fazer-lhe justiça, para que enfim não
volte, e me importune muito. E disse o Senhor: Ouvi o que diz o
injusto juiz. E Deus não fará justiça aos seus escolhidos, que
clamam a ele de dia e de noite, ainda que tardio para com eles?
Digo-vos que depressa lhes fará justiça. Quando porém vier o
Filho do homem, porventura achará fé na terra?

E disse também esta parábola a uns que confiavam em si mesmos,
crendo que eram justos, e desprezavam os outros: Dois homens
subiram ao templo, para orar; um, fariseu, e o outro, publicano.
O fariseu, estando em pé, orava consigo desta maneira: Ó
Deus, graças te dou porque não sou como os demais homens,
roubadores, injustos e adúlteros; nem ainda como este publicano.
Jejuo duas vezes na semana, e dou os dízimos de tudo quanto
possuo. O publicano, porém, estando em pé, de longe, nem
ainda queria levantar os olhos ao céu, mas batia no peito, dizendo:
Ó Deus, tem misericórdia de mim, pecador! Digo-vos que este
desceu justificado para sua casa, e não aquele; porque qualquer que
a si mesmo se exalta será humilhado, e qualquer que a si mesmo se
humilha será exaltado.

E traziam-lhe também meninos, para que ele lhes tocasse; e os
discípulos, vendo isto, repreendiam-nos. Mas Jesus,
chamando-os para si, disse: Deixai vir a mim os meninos, e não os
impeçais, porque dos tais é o reino de Deus. Em verdade vos
digo que, qualquer que não receber o reino de Deus como menino, não
entrará nele.

E perguntou-lhe um certo príncipe, dizendo: Bom Mestre, que hei
de fazer para herdar a vida eterna? Jesus lhe disse: Por que
me chamas bom? Ninguém há bom, senão um, que é Deus. Sabes os
mandamentos: Não adulterarás, não matarás, não furtarás, não dirás
falso testemunho, honra a teu pai e a tua mãe. E disse ele:
Todas essas coisas tenho observado desde a minha mocidade. E
quando Jesus ouviu isto, disse-lhe: Ainda te falta uma coisa; vende
tudo quanto tens, reparte-o pelos pobres, e terás um tesouro no céu;
vem, e segue-me. Mas, ouvindo ele isto, ficou muito triste,
porque era muito rico. E, vendo Jesus que ele ficara muito
triste, disse: Quão dificilmente entrarão no reino de Deus os que
têm riquezas! Porque é mais fácil entrar um camelo pelo fundo
de uma agulha do que entrar um rico no reino de Deus. E os
que ouviram isto disseram: Logo quem pode salvar-se? Mas ele
respondeu: As coisas que são impossíveis aos homens são possíveis a
Deus. E disse Pedro: Eis que nós deixamos tudo e te seguimos.
E ele lhes disse: Na verdade vos digo que ninguém há, que
tenha deixado casa, ou pais, ou irmãos, ou mulher, ou filhos, pelo
reino de Deus, que não haja de receber muito mais neste
mundo, e na idade vindoura a vida eterna.

E, tomando consigo os doze, disse-lhes: Eis que subimos a
Jerusalém, e se cumprirá no Filho do homem tudo o que pelos profetas
foi escrito; pois há de ser entregue aos gentios, e
escarnecido, injuriado e cuspido; e, havendo-o açoitado, o
matarão; e ao terceiro dia ressuscitará. E eles nada disto
entendiam, e esta palavra lhes era encoberta, não percebendo o que
se lhes dizia.

E aconteceu que chegando ele perto de Jericó, estava um cego
assentado junto do caminho, mendigando. E, ouvindo passar a
multidão, perguntou que era aquilo. E disseram-lhe que Jesus
Nazareno passava. Então clamou, dizendo: Jesus, Filho de
Davi, tem misericórdia de mim. E os que iam passando
repreendiam-no para que se calasse; mas ele clamava ainda mais:
Filho de Davi, tem misericórdia de mim! Então Jesus, parando,
mandou que lho trouxessem; e, chegando ele, perguntou-lhe,
dizendo: Que queres que te faça? E ele disse: Senhor, que eu
veja. E Jesus lhe disse: Vê; a tua fé te salvou. E
logo viu, e seguia-o, glorificando a Deus. E todo o povo, vendo
isto, dava louvores a Deus.

\medskip

\lettrine{19} E, tendo Jesus entrado em Jericó, ia passando.
E eis que havia ali um homem chamado Zaqueu; e era este um chefe
dos publicanos, e era rico. E procurava ver quem era Jesus, e
não podia, por causa da multidão, pois era de pequena estatura.
E, correndo adiante, subiu a um sicômoro para o ver; porque
havia de passar por ali. E quando Jesus chegou àquele lugar,
olhando para cima, viu-o e disse-lhe: Zaqueu, desce depressa, porque
hoje me convém pousar em tua casa. E, apressando-se, desceu, e
recebeu-o alegremente. E, vendo todos isto, murmuravam, dizendo
que entrara para ser hóspede de um homem pecador. E,
levantando-se Zaqueu, disse ao Senhor: Senhor, eis que eu dou aos
pobres metade dos meus bens; e, se nalguma coisa tenho defraudado
alguém, o restituo quadruplicado. E disse-lhe Jesus: Hoje veio a
salvação a esta casa, pois também este é filho de Abraão.
Porque o Filho do homem veio buscar e salvar o que se havia
perdido.

E, ouvindo eles estas coisas, ele prosseguiu, e contou uma
parábola; porquanto estava perto de Jerusalém, e cuidavam que logo
se havia de manifestar o reino de Deus. Disse pois: Certo
homem nobre partiu para uma terra remota, a fim de tomar para si um
reino e voltar depois. E, chamando dez servos seus, deu-lhes
dez minas, e disse-lhes: Negociai até que eu venha. Mas os
seus concidadãos odiavam-no, e mandaram após ele embaixadores,
dizendo: Não queremos que este reine sobre nós. E aconteceu
que, voltando ele, depois de ter tomado o reino, disse que lhe
chamassem aqueles servos, a quem tinha dado o dinheiro, para saber o
que cada um tinha ganhado, negociando. E veio o primeiro,
dizendo: Senhor, a tua mina rendeu dez minas. E ele lhe
disse: Bem está, servo bom, porque no mínimo foste fiel, sobre dez
cidades terás autoridade. E veio o segundo, dizendo: Senhor,
a tua mina rendeu cinco minas. E a este disse também: Sê tu
também sobre cinco cidades. E veio outro, dizendo: Senhor,
aqui está a tua mina, que guardei num lenço; porque tive medo
de ti, que és homem rigoroso, que tomas o que não puseste, e segas o
que não semeaste. Porém, ele lhe disse: Mau servo, pela tua
boca te julgarei. Sabias que eu sou homem rigoroso, que tomo o que
não pus, e sego o que não semeei; por que não puseste, pois,
o meu dinheiro no banco, para que eu, vindo, o exigisse com os
juros? E disse aos que estavam com ele: Tirai-lhe a mina, e
dai-a ao que tem dez minas.

disseram-lhe eles: Senhor, ele tem dez minas.) Pois eu
vos digo que a qualquer que tiver ser-lhe-á dado, mas ao que não
tiver, até o que tem lhe será tirado. E quanto àqueles meus
inimigos que não quiseram que eu reinasse sobre eles, trazei-os
aqui, e matai-os diante de mim.

E, dito isto, ia caminhando adiante, subindo para Jerusalém.
E aconteceu que, chegando perto de Betfagé, e de Betânia, ao
monte chamado das Oliveiras, mandou dois dos seus discípulos,
dizendo: Ide à aldeia que está defronte, e aí, ao entrar,
achareis preso um jumentinho em que nenhum homem ainda montou;
soltai-o e trazei-o. E, se alguém vos perguntar: Por que o
soltais? assim lhe direis: Porque o Senhor o há de
mister\footnote{Precisão, necessidade; urgência; aquilo que é
necessário ou forçoso.}. E, indo os que haviam sido mandados,
acharam como lhes dissera. E, quando soltaram o jumentinho,
seus donos lhes disseram: Por que soltais o jumentinho? E
eles responderam: O Senhor o há de mister. E trouxeram-no a
Jesus; e, lançando sobre o jumentinho as suas vestes, puseram Jesus
em cima. E, indo ele, estendiam no caminho as suas vestes.
E, quando já chegava perto da descida do Monte das Oliveiras,
toda a multidão dos discípulos, regozijando-se, começou a dar
louvores a Deus em alta voz, por todas as maravilhas que tinham
visto, dizendo: Bendito o Rei que vem em nome do Senhor; paz
no céu, e glória nas alturas. E disseram-lhe de entre a
multidão alguns dos fariseus: Mestre, repreende os teus discípulos.
E, respondendo ele, disse-lhes: Digo-vos que, se estes se
calarem, as próprias pedras clamarão.

E, quando ia chegando, vendo a cidade, chorou sobre ela,
dizendo: Ah! se tu conhecesses também, ao menos neste teu
dia, o que à tua paz pertence! Mas agora isto está encoberto aos
teus olhos. Porque dias virão sobre ti, em que os teus
inimigos te cercarão de trincheiras, e te sitiarão, e te estreitarão
de todos os lados; e te derrubarão, a ti e aos teus filhos
que dentro de ti estiverem, e não deixarão em ti pedra sobre pedra,
pois que não conheceste o tempo da tua visitação. E, entrando
no templo, começou a expulsar todos os que nele vendiam e compravam,
dizendo-lhes: Está escrito: A minha casa é casa de oração;
mas vós fizestes dela covil de salteadores. E todos os dias
ensinava no templo; mas os principais dos sacerdotes, e os escribas,
e os principais do povo procuravam matá-lo. E não achavam
meio de o fazer, porque todo o povo pendia para ele, escutando-o.

\medskip

\lettrine{20} E aconteceu num daqueles dias que, estando ele
ensinando o povo no templo, e anunciando o evangelho, sobrevieram os
principais dos sacerdotes e os escribas com os anciãos, e
falaram-lhe, dizendo: Dize-nos, com que autoridade fazes estas
coisas? Ou, quem é que te deu esta autoridade? E, respondendo
ele, disse-lhes: Também eu vos farei uma pergunta: Dizei-me pois:
O batismo de João era do céu ou dos homens? E eles
arrazoavam entre si, dizendo: Se dissermos: Do céu, ele nos dirá:
Então por que o não crestes? E se dissermos: Dos homens; todo o
povo nos apedrejará, pois têm por certo que João era profeta. E
responderam que não sabiam de onde era. E Jesus lhes disse:
Tampouco vos direi com que autoridade faço isto.

E começou a dizer ao povo esta parábola: Certo homem plantou uma
vinha, e arrendou-a a uns lavradores, e partiu para fora da terra
por muito tempo; e no tempo próprio mandou um servo aos
lavradores, para que lhe dessem dos frutos da vinha; mas os
lavradores, espancando-o, mandaram-no vazio. E tornou ainda a
mandar outro servo; mas eles, espancando também a este, e
afrontando-o, mandaram-no vazio. E tornou ainda a mandar um
terceiro; mas eles, ferindo também a este, o expulsaram. E
disse o senhor da vinha: Que farei? Mandarei o meu filho amado;
talvez, vendo-o, seja respeitado. Mas, vendo-o os lavradores,
arrazoaram entre si, dizendo: Este é o herdeiro; vinde, matemo-lo,
para que a herança seja nossa. E, lançando-o fora da vinha, o
mataram. Que lhes fará, pois, o Senhor da vinha? Irá, e
destruirá estes lavradores, e dará a outros a vinha. E, ouvindo eles
isto, disseram: Não seja assim! Mas ele, olhando para eles,
disse: Que é isto, pois, que está escrito? A pedra, que os
edificadores reprovaram, essa foi feita cabeça da esquina.
Qualquer que cair sobre aquela pedra ficará em pedaços, e
aquele sobre quem ela cair será feito em pó. E os principais
dos sacerdotes e os escribas procuravam lançar mão dele naquela
mesma hora; mas temeram o povo; porque entenderam que contra eles
dissera esta parábola.

E, observando-o, mandaram espias, que se fingissem justos, para o
apanharem nalguma palavra, e o entregarem à jurisdição e poder do
presidente. E perguntaram-lhe, dizendo: Mestre, nós sabemos
que falas e ensinas bem e retamente, e que não consideras a
aparência da pessoa, mas ensinas com verdade o caminho de Deus.
É-nos lícito dar tributo a César ou não? E, entendendo
ele a sua astúcia, disse-lhes: Por que me tentais? Mostrai-me
uma moeda. De quem tem a imagem e a inscrição? E, respondendo eles,
disseram: De César. Disse-lhes então: Dai, pois, a César o
que é de César, e a Deus o que é de Deus. E não puderam
apanhá-lo em palavra alguma diante do povo; e, maravilhados da sua
resposta, calaram-se.

E, chegando-se alguns dos saduceus, que dizem não haver
ressurreição, perguntaram-lhe, dizendo: Mestre, Moisés nos
deixou escrito que, se o irmão de algum falecer, tendo mulher, e não
deixar filhos, o irmão dele tome a mulher, e suscite posteridade a
seu irmão. Houve, pois, sete irmãos, e o primeiro tomou
mulher, e morreu sem filhos; e tomou-a o segundo por mulher,
e ele morreu sem filhos. E tomou-a o terceiro, e igualmente
também os sete; e morreram, e não deixaram filhos. E por
último, depois de todos, morreu também a mulher. Portanto, na
ressurreição, de qual deles será a mulher, pois que os sete por
mulher a tiveram? E, respondendo Jesus, disse-lhes: Os filhos
deste mundo casam-se, e dão-se em casamento; mas os que forem
havidos por dignos de alcançar o mundo vindouro, e a ressurreição
dentre os mortos, nem hão de casar, nem ser dados em casamento;
porque já não podem mais morrer; pois são iguais aos anjos, e
são filhos de Deus, sendo filhos da ressurreição. E que os
mortos hão de ressuscitar também o mostrou Moisés junto da sarça,
quando chama ao Senhor Deus de Abraão, e Deus de Isaque, e Deus de
Jacó. Ora, Deus não é Deus de mortos, mas de vivos; porque
para ele vivem todos.

E, respondendo alguns dos escribas, disseram: Mestre, disseste
bem. E não ousavam perguntar-lhe mais coisa alguma. E
ele lhes disse: Como dizem que o Cristo é filho de Davi?
Visto como o mesmo Davi diz no livro dos Salmos: Disse o
Senhor ao meu Senhor: Assenta-te à minha direita, até que eu
ponha os teus inimigos por escabelo de teus pés. Se Davi lhe
chama Senhor, como é ele seu filho? E, ouvindo-o todo o povo,
disse Jesus aos seus discípulos: Guardai-vos dos escribas,
que querem andar com vestes compridas; e amam as saudações nas
praças, e as principais cadeiras nas sinagogas, e os primeiros
lugares nos banquetes; que devoram as casas das viúvas,
fazendo, por pretexto, longas orações. Estes receberão maior
condenação.

\medskip

\lettrine{21} E, olhando ele, viu os ricos lançarem as suas
ofertas na arca do tesouro; e viu também uma pobre viúva lançar
ali dois leptos\footnote{SBTB: duas pequenas moedas.}; e disse:
Em verdade vos digo que lançou mais do que todos, esta pobre viúva;
porque todos aqueles deitaram para as ofertas de Deus do que
lhes sobeja; mas esta, da sua pobreza, deitou todo o sustento que
tinha.

E, dizendo alguns a respeito do templo, que estava ornado de
formosas pedras e dádivas, disse: Quanto a estas coisas que
vedes, dias virão em que não se deixará pedra sobre pedra, que não
seja derrubada. E perguntaram-lhe, dizendo: Mestre, quando
serão, pois, estas coisas? E que sinal haverá quando isto estiver
para acontecer? Disse então ele: Vede não vos enganem, porque
virão muitos em meu nome, dizendo: Sou eu, e o tempo está próximo.
Não vades, portanto, após eles. E, quando ouvirdes de guerras e
sedições, não vos assusteis. Porque é necessário que isto aconteça
primeiro, mas o fim não será logo. Então lhes disse:
Levantar-se-á nação contra nação, e reino contra reino; e
haverá em vários lugares grandes terremotos, e fomes e pestilências;
haverá também coisas espantosas, e grandes sinais do céu. Mas
antes de todas estas coisas lançarão mão de vós, e vos perseguirão,
entregando-vos às sinagogas e às prisões, e conduzindo-vos à
presença de reis e presidentes, por amor do meu nome. E vos
acontecerá isto para testemunho. Proponde, pois, em vossos
corações não premeditar como haveis de responder; porque eu
vos darei boca e sabedoria a que não poderão resistir nem
contradizer todos quantos se vos opuserem. E até pelos pais,
e irmãos, e parentes, e amigos sereis entregues; e matarão alguns de
vós. E de todos sereis odiados por causa do meu nome.
Mas não perecerá um único cabelo da vossa cabeça. Na
vossa paciência possuí as vossas almas.

Mas, quando virdes Jerusalém cercada de exércitos, sabei então
que é chegada a sua desolação. Então, os que estiverem na
Judéia, fujam para os montes; os que estiverem no meio da cidade,
saiam; e os que nos campos não entrem nela. Porque dias de
vingança são estes, para que se cumpram todas as coisas que estão
escritas. Mas ai das grávidas, e das que criarem naqueles
dias! porque haverá grande aperto na terra, e ira sobre este povo.
E cairão ao fio da espada, e para todas as nações serão
levados cativos; e Jerusalém será pisada pelos gentios, até que os
tempos dos gentios se completem. E haverá sinais no sol e na
lua e nas estrelas; e na terra angústia das nações, em perplexidade
pelo bramido do mar e das ondas. Homens desmaiando de terror,
na expectação das coisas que sobrevirão ao mundo; porquanto as
virtudes do céu serão abaladas. E então verão vir o Filho do
homem numa nuvem, com poder e grande glória. Ora, quando
estas coisas começarem a acontecer, olhai para cima e levantai as
vossas cabeças, porque a vossa redenção está próxima.

E disse-lhes uma parábola: Olhai para a figueira, e para todas as
árvores; quando já têm rebentado, vós sabeis por vós mesmos,
vendo-as, que perto está já o verão. Assim também vós, quando
virdes acontecer estas coisas, sabei que o reino de Deus está perto.
Em verdade vos digo que não passará esta geração até que tudo
aconteça. Passará o céu e a terra, mas as minhas palavras não
hão de passar. E olhai por vós, não aconteça que os vossos
corações se carreguem de glutonaria, de embriaguez, e dos cuidados
da vida, e venha sobre vós de improviso aquele dia. Porque
virá como um laço sobre todos os que habitam na face de toda a
terra. Vigiai, pois, em todo o tempo, orando, para que sejais
havidos por dignos de evitar todas estas coisas que hão de
acontecer, e de estar em pé diante do Filho do homem. E de
dia ensinava no templo, e à noite, saindo, ficava no monte chamado
das Oliveiras. E todo o povo ia ter com ele ao templo, de
manhã cedo, para o ouvir.

\medskip

\lettrine{22} Estava, pois, perto a festa dos ázimos, chamada
a páscoa. E os principais dos sacerdotes, e os escribas, andavam
procurando como o matariam; porque temiam o povo. Entrou, porém,
Satanás em Judas, que tinha por sobrenome Iscariotes, o qual era do
número dos doze. E foi, e falou com os principais dos
sacerdotes, e com os capitães, de como lho entregaria; os quais
se alegraram, e convieram em lhe dar dinheiro. E ele concordou;
e buscava oportunidade para lho entregar sem alvoroço.

Chegou, porém, o dia dos ázimos, em que importava sacrificar a
páscoa. E mandou a Pedro e a João, dizendo: Ide, preparai-nos a
páscoa, para que a comamos. E eles lhe perguntaram: Onde queres
que a preparemos? E ele lhes disse: Eis que, quando entrardes
na cidade, encontrareis um homem, levando um cântaro de água;
segui-o até à casa em que ele entrar. E direis ao pai de
família da casa: O Mestre te diz: Onde está o aposento em que hei de
comer a páscoa com os meus discípulos? Então ele vos mostrará
um grande cenáculo\footnote{Sala em que se comia a ceia ou o jantar;
p. ext. refeitório.} mobiliado\footnote{SBTB: mobilado.}; aí fazei
preparativos. E, indo eles, acharam como lhes havia sido
dito; e prepararam a páscoa. E, chegada a hora, pôs-se à
mesa, e com ele os doze apóstolos. E disse-lhes: Desejei
muito comer convosco esta páscoa, antes que padeça; porque
vos digo que não a comerei mais até que ela se cumpra no reino de
Deus. E, tomando o cálice, e havendo dado graças, disse:
Tomai-o, e reparti-o entre vós; porque vos digo que já não
beberei do fruto da vide, até que venha o reino de Deus. E,
tomando o pão, e havendo dado graças, partiu-o, e deu-lho, dizendo:
Isto é o meu corpo, que por vós é dado; fazei isto em memória de
mim. Semelhantemente, tomou o cálice, depois da ceia,
dizendo: Este cálice é o novo testamento no meu sangue, que é
derramado por vós.

Mas eis que a mão do que me trai está comigo à mesa. E, na
verdade, o Filho do homem vai segundo o que está determinado; mas ai
daquele homem por quem é traído! E começaram a perguntar
entre si qual deles seria o que havia de fazer isto. E houve
também entre eles contenda, sobre qual deles parecia ser o maior.
E ele lhes disse: Os reis dos gentios dominam sobre eles, e
os que têm autoridade sobre eles são chamados benfeitores.
Mas não sereis vós assim; antes o maior entre vós seja como o
menor; e quem governa como quem serve. Pois qual é maior:
quem está à mesa, ou quem serve? Porventura não é quem está à mesa?
Eu, porém, entre vós sou como aquele que serve. E vós sois os
que tendes permanecido comigo nas minhas tentações. E eu vos
destino o reino, como meu Pai mo destinou, para que comais e
bebais à minha mesa no meu reino, e vos assenteis sobre tronos,
julgando as doze tribos de Israel. Disse também o Senhor:
Simão, Simão, eis que Satanás vos pediu para vos
cirandar\footnote{Passar pela ciranda (peneira grossa com que se
joeiram grãos de areia, etc.); joeirar, peneirar.} como trigo;
mas eu roguei por ti, para que a tua fé não desfaleça; e tu,
quando te converteres, confirma teus irmãos. E ele lhe disse:
Senhor, estou pronto a ir contigo até à prisão e à morte. Mas
ele disse: Digo-te, Pedro, que não cantará hoje o galo antes que
três vezes negues que me conheces. E disse-lhes: Quando vos
mandei sem bolsa, alforje, ou alparcas, faltou-vos porventura alguma
coisa? Eles responderam: Nada. Disse-lhes pois: Mas agora,
aquele que tiver bolsa, tome-a, como também o alforje; e, o que não
tem espada, venda a sua capa e compre-a; porquanto vos digo
que importa que em mim se cumpra aquilo que está escrito: E com os
malfeitores foi contado. Porque o que está escrito de mim terá
cumprimento. E eles disseram: Senhor, eis aqui duas espadas.
E ele lhes disse: Basta.

E, saindo, foi, como costumava, para o Monte das Oliveiras; e
também os seus discípulos o seguiram. E quando chegou àquele
lugar, disse-lhes: Orai, para que não entreis em tentação. E
apartou-se deles cerca de um tiro de pedra; e, pondo-se de joelhos,
orava, dizendo: Pai, se queres, passa de mim este cálice;
todavia não se faça a minha vontade, mas a tua. E
apareceu-lhe um anjo do céu, que o fortalecia. E, posto em
agonia, orava mais intensamente. E o seu suor tornou-se em grandes
gotas de sangue, que corriam até ao chão. E, levantando-se da
oração, veio para os seus discípulos, e achou-os dormindo de
tristeza. E disse-lhes: Por que estais dormindo?
Levantai-vos, e orai, para que não entreis em tentação.

E, estando ele ainda a falar, surgiu uma multidão; e um dos doze,
que se chamava Judas, ia adiante dela, e chegou-se a Jesus para o
beijar. E Jesus lhe disse: Judas, com um beijo trais o Filho
do homem? E, vendo os que estavam com ele o que ia suceder,
disseram-lhe: Senhor, feriremos à espada? E um deles feriu o
servo do sumo sacerdote, e cortou-lhe a orelha direita. E,
respondendo Jesus, disse: Deixai-os; basta. E, tocando-lhe a orelha,
o curou. E disse Jesus aos principais dos sacerdotes, e
capitães do templo, e anciãos, que tinham ido contra ele: Saístes,
como a um salteador, com espadas e varapaus? Tenho estado
todos os dias convosco no templo, e não estendestes as mãos contra
mim, mas esta é a vossa hora e o poder das trevas.

Então, prendendo-o, o levaram, e o puseram em casa do sumo
sacerdote. E Pedro seguia-o de longe. E, havendo-se acendido
fogo no meio do pátio, estando todos sentados, assentou-se Pedro
entre eles. E como certa criada, vendo-o estar assentado ao
fogo, pusesse os olhos nele, disse: Este também estava com ele.
Porém, ele negou-o, dizendo: Mulher, não o conheço. E,
um pouco depois, vendo-o outro, disse: Tu és também deles. Mas Pedro
disse: Homem, não sou. E, passada quase uma hora, um outro
afirmava, dizendo: Também este verdadeiramente estava com ele, pois
também é galileu. E Pedro disse: Homem, não sei o que dizes.
E logo, estando ele ainda a falar, cantou o galo. E,
virando-se o Senhor, olhou para Pedro, e Pedro lembrou-se da palavra
do Senhor, como lhe havia dito: Antes que o galo cante hoje, me
negarás três vezes. E, saindo Pedro\footnote{SBTB: saindo
Pedro para fora. KJ: And Peter went out, and wept bitterly. RA:
Então, Pedro, saindo dali, chorou amargamente.}, chorou amargamente.

E os homens que detinham Jesus zombavam dele, ferindo-o.
E, vendando-lhe os olhos, feriam-no no rosto, e
perguntavam-lhe, dizendo: Profetiza, quem é que te feriu? E
outras muitas coisas diziam contra ele, blasfemando. E logo
que foi dia ajuntaram-se os anciãos do povo, e os principais dos
sacerdotes e os escribas, e o conduziram ao seu concílio, e lhe
perguntaram: És tu o Cristo? Dize-no-lo. Ele replicou: Se
vo-lo disser, não o crereis; e também, se vos perguntar, não
me respondereis, nem me soltareis. Desde agora o Filho do
homem se assentará à direita do poder de Deus. E disseram
todos: Logo, és tu o Filho de Deus? E ele lhes disse: Vós dizeis que
eu sou. Então disseram: De que mais testemunho necessitamos?
pois nós mesmos o ouvimos da sua boca.

\medskip

\lettrine{23} E, levantando-se toda a multidão deles, o
levaram a Pilatos. E começaram a acusá-lo, dizendo: Havemos
achado este pervertendo a nossa nação, proibindo dar o tributo a
César, e dizendo que ele mesmo é Cristo, o rei. E Pilatos
perguntou-lhe, dizendo: Tu és o Rei dos Judeus? E ele, respondendo,
disse-lhe: Tu o dizes. E disse Pilatos aos principais dos
sacerdotes, e à multidão: Não acho culpa alguma neste homem. Mas
eles insistiam cada vez mais, dizendo: Alvoroça o povo ensinando por
toda a Judéia, começando desde a Galiléia até aqui. Então
Pilatos, ouvindo falar da Galiléia perguntou se aquele homem era
galileu. E, sabendo que era da jurisdição de Herodes, remeteu-o
a Herodes, que também naqueles dias estava em Jerusalém. E
Herodes, quando viu a Jesus, alegrou-se muito; porque havia muito
que desejava vê-lo, por ter ouvido dele muitas coisas; e esperava
que lhe veria fazer algum sinal. E interrogava-o com muitas
palavras, mas ele nada lhe respondia. E estavam os principais
dos sacerdotes, e os escribas, acusando-o com grande veemência.
E Herodes, com os seus soldados, desprezou-o e, escarnecendo
dele, vestiu-o de uma roupa resplandecente e tornou a enviá-lo a
Pilatos. E no mesmo dia, Pilatos e Herodes entre si se
fizeram amigos; pois dantes andavam em inimizade um com o outro.

E, convocando Pilatos os principais dos sacerdotes, e os
magistrados, e o povo, disse-lhes: Haveis-me apresentado este
homem como pervertedor do povo; e eis que, examinando-o na vossa
presença, nenhuma culpa, das de que o acusais, acho neste homem.
Nem mesmo Herodes, porque a ele vos remeti, e eis que não tem
feito coisa alguma digna de morte. Castigá-lo-ei, pois, e
soltá-lo-ei. E era-lhe necessário soltar-lhes um pela festa.
Mas toda a multidão clamou a uma, dizendo: Fora daqui com
este, e solta-nos Barrabás. O qual fora lançado na prisão por
causa de uma sedição\footnote{Perturbação da ordem pública;
agitação, sublevação, revolta, motim.} feita na cidade, e de um
homicídio. Falou, pois, outra vez Pilatos, querendo soltar a
Jesus. Mas eles clamavam em contrário, dizendo: Crucifica-o,
crucifica-o. Então ele, pela terceira vez, lhes disse: Mas
que mal fez este? Não acho nele culpa alguma de morte. Castigá-lo-ei
pois, e soltá-lo-ei. Mas eles instavam com grandes gritos,
pedindo que fosse crucificado. E os seus gritos, e os dos principais
dos sacerdotes, redobravam. Então Pilatos julgou que devia
fazer o que eles pediam. E soltou-lhes o que fora lançado na
prisão por uma sedição e homicídio, que era o que pediam; mas
entregou Jesus à vontade deles.

E quando o iam levando, tomaram um certo Simão, cireneu, que
vinha do campo, e puseram-lhe a cruz às costas, para que a levasse
após Jesus. E seguia-o grande multidão de povo e de mulheres,
as quais batiam nos peitos, e o lamentavam. Jesus, porém,
voltando-se para elas, disse: Filhas de Jerusalém, não choreis por
mim; chorai antes por vós mesmas, e por vossos filhos. Porque
eis que hão de vir dias em que dirão: Bem-aventuradas as estéreis, e
os ventres que não geraram, e os peitos que não amamentaram!
Então começarão a dizer aos montes: Caí sobre nós, e aos
outeiros: Cobri-nos. Porque, se ao madeiro verde fazem isto,
que se fará ao seco?

E também conduziram outros dois, que eram malfeitores, para com
ele serem mortos. E, quando chegaram ao lugar chamado a
Caveira, ali o crucificaram, e aos malfeitores, um à direita e outro
à esquerda. E dizia Jesus: Pai, perdoa-lhes, porque não sabem
o que fazem. E, repartindo as suas vestes, lançaram sortes. E
o povo estava olhando. E também os príncipes zombavam dele, dizendo:
Aos outros salvou, salve-se a si mesmo, se este é o Cristo, o
escolhido de Deus. E também os soldados o escarneciam,
chegando-se a ele, e apresentando-lhe vinagre. E dizendo: Se
tu és o Rei dos Judeus, salva-te a ti mesmo. E também por
cima dele, estava um título, escrito em letras gregas, romanas, e
hebraicas: ESTE É O REI DOS JUDEUS. E um dos malfeitores que
estavam pendurados blasfemava dele, dizendo: Se tu és o Cristo,
salva-te a ti mesmo, e a nós. Respondendo, porém, o outro,
repreendia-o, dizendo: Tu nem ainda temes a Deus, estando na mesma
condenação? E nós, na verdade, com justiça, porque recebemos
o que os nossos feitos mereciam; mas este nenhum mal fez. E
disse a Jesus: Senhor, lembra-te de mim, quando entrares no teu
reino. E disse-lhe Jesus: Em verdade te digo que hoje estarás
comigo no Paraíso.

E era já quase a hora sexta, e houve trevas em toda a terra até à
hora nona, escurecendo-se o sol; e rasgou-se ao meio o véu do
templo. E, clamando Jesus com grande voz, disse: Pai, nas
tuas mãos entrego o meu espírito. E, havendo dito isto, expirou.
E o centurião, vendo o que tinha acontecido, deu glória a
Deus, dizendo: Na verdade, este homem era justo. E toda a
multidão que se ajuntara a este espetáculo, vendo o que havia
acontecido, voltava batendo nos peitos. E todos os seus
conhecidos, e as mulheres que juntamente o haviam seguido desde a
Galiléia, estavam de longe vendo estas coisas.

E eis que um homem por nome José, senador, homem de bem e justo,
que não tinha consentido no conselho e nos atos dos outros,
de Arimatéia, cidade dos judeus, e que também esperava o reino de
Deus; esse, chegando a Pilatos, pediu o corpo de Jesus.
E, havendo-o tirado, envolveu-o num lençol, e pô-lo num
sepulcro escavado numa penha, onde ninguém ainda havia sido posto.
E era o dia da preparação, e amanhecia o sábado. E as
mulheres, que tinham vindo com ele da Galiléia, seguiram também e
viram o sepulcro, e como foi posto o seu corpo. E, voltando
elas, prepararam especiarias e ungüentos; e no sábado repousaram,
conforme o mandamento.

\medskip

\lettrine{24} E no primeiro dia da semana, muito de madrugada,
foram elas ao sepulcro, levando as especiarias que tinham preparado,
e algumas outras com elas. E acharam a pedra revolvida do
sepulcro. E, entrando, não acharam o corpo do Senhor Jesus.
E aconteceu que, estando elas muito perplexas a esse respeito,
eis que pararam junto delas dois homens, com vestes resplandecentes.
E, estando elas muito atemorizadas, e abaixando o rosto para o
chão, eles lhes disseram: Por que buscais o vivente entre os mortos?
Não está aqui, mas ressuscitou. Lembrai-vos como vos falou,
estando ainda na Galiléia, dizendo: Convém que o Filho do homem
seja entregue nas mãos de homens pecadores, e seja crucificado, e ao
terceiro dia ressuscite. E lembraram-se das suas palavras.
E, voltando do sepulcro, anunciaram todas estas coisas aos onze
e a todos os demais. E eram Maria Madalena, e Joana, e Maria,
mãe de Tiago, e as outras que com elas estavam, as que diziam estas
coisas aos apóstolos. E as suas palavras lhes pareciam como
desvario, e não as creram. Pedro, porém, levantando-se,
correu ao sepulcro e, abaixando-se, viu só os lençóis ali postos; e
retirou-se, admirando consigo aquele caso.

E eis que no mesmo dia iam dois deles para uma aldeia, que
distava de Jerusalém sessenta estádios, cujo nome era Emaús.
E iam falando entre si de tudo aquilo que havia sucedido.
E aconteceu que, indo eles falando entre si, e fazendo
perguntas um ao outro, o mesmo Jesus se aproximou, e ia com eles.
Mas os olhos deles estavam como que fechados, para que o não
conhecessem. E ele lhes disse: Que palavras são essas que,
caminhando, trocais entre vós, e por que estais tristes? E,
respondendo um, cujo nome era Cléopas, disse-lhe: És tu só peregrino
em Jerusalém, e não sabes as coisas que nela têm sucedido nestes
dias? E ele lhes perguntou: Quais? E eles lhe disseram: As
que dizem respeito a Jesus Nazareno, que foi homem profeta, poderoso
em obras e palavras diante de Deus e de todo o povo; e como
os principais dos sacerdotes e os nossos príncipes o entregaram à
condenação de morte, e o crucificaram. E nós esperávamos que
fosse ele o que remisse Israel; mas agora, sobre tudo isso, é já
hoje o terceiro dia desde que essas coisas aconteceram. É
verdade que também algumas mulheres dentre nós nos maravilharam, as
quais de madrugada foram ao sepulcro; e, não achando o seu
corpo, voltaram, dizendo que também tinham visto uma visão de anjos,
que dizem que ele vive. E alguns dos que estavam conosco
foram ao sepulcro, e acharam ser assim como as mulheres haviam dito;
porém, a ele não o viram. E ele lhes disse: Ó néscios, e
tardos de coração para crer tudo o que os profetas disseram!
Porventura não convinha que o Cristo padecesse estas coisas e
entrasse na sua glória? E, começando por Moisés, e por todos
os profetas, explicava-lhes o que dele se achava em todas as
Escrituras. E chegaram à aldeia para onde iam, e ele fez como
quem ia para mais longe. E eles o constrangeram, dizendo:
Fica conosco, porque já é tarde, e já declinou o dia. E entrou para
ficar com eles. E aconteceu que, estando com eles à mesa,
tomando o pão, o abençoou e partiu-o, e lho deu.
Abriram-se-lhes então os olhos, e o conheceram, e ele
desapareceu-lhes. E disseram um para o outro: Porventura não
ardia em nós o nosso coração quando, pelo caminho, nos falava, e
quando nos abria as Escrituras? E na mesma hora,
levantando-se, tornaram para Jerusalém, e acharam congregados os
onze, e os que estavam com eles, os quais diziam: Ressuscitou
verdadeiramente o Senhor, e já apareceu a Simão. E eles lhes
contaram o que lhes acontecera no caminho, e como deles fora
conhecido no partir do pão.

E falando eles destas coisas, o mesmo Jesus se apresentou no meio
deles, e disse-lhes: Paz seja convosco. E eles, espantados e
atemorizados, pensavam que viam algum espírito. E ele lhes
disse: Por que estais perturbados, e por que sobem tais pensamentos
aos vossos corações? Vede as minhas mãos e os meus pés, que
sou eu mesmo; apalpai-me e vede, pois um espírito não tem carne nem
ossos, como vedes que eu tenho. E, dizendo isto, mostrou-lhes
as mãos e os pés. E, não o crendo eles ainda por causa da
alegria, e estando maravilhados, disse-lhes: Tendes aqui alguma
coisa que comer? Então eles apresentaram-lhe parte de um
peixe assado, e um favo de mel; o que ele tomou, e comeu
diante deles. E disse-lhes: São estas as palavras que vos
disse estando ainda convosco: Que convinha que se cumprisse tudo o
que de mim estava escrito na lei de Moisés, e nos profetas e nos
Salmos. Então abriu-lhes o entendimento para compreenderem as
Escrituras. E disse-lhes: Assim está escrito, e assim
convinha que o Cristo padecesse, e ao terceiro dia ressuscitasse
dentre os mortos, e em seu nome se pregasse o arrependimento
e a remissão dos pecados, em todas as nações, começando por
Jerusalém. E destas coisas sois vós testemunhas. E eis
que sobre vós envio a promessa de meu Pai; ficai, porém, na cidade
de Jerusalém, até que do alto sejais revestidos de poder.

E levou-os fora, até Betânia; e, levantando as suas mãos, os
abençoou. E aconteceu que, abençoando-os ele, se apartou
deles e foi elevado ao céu. E, adorando-o eles, tornaram com
grande júbilo para Jerusalém. E estavam sempre no templo,
louvando e bendizendo a Deus. Amém.

