\thispagestyle{empty}
\chapter*{Primeira Epístola de Paulo aos Coríntios}

\lettrine{1} Paulo (chamado apóstolo de Jesus Cristo, pela
vontade de Deus), e o irmão Sóstenes, à igreja de Deus que está
em Corinto, aos santificados em Cristo Jesus, chamados santos, com
todos os que em todo o lugar invocam o nome de nosso Senhor Jesus
Cristo, Senhor deles e nosso: Graça e paz da parte de Deus nosso
Pai, e do Senhor Jesus Cristo. Sempre dou graças ao meu Deus por
vós pela graça de Deus que vos foi dada em Jesus Cristo. Porque
em tudo fostes enriquecidos nele, em toda a palavra e em todo o
conhecimento

 (como o testemunho de Cristo foi mesmo confirmado entre vós).
De maneira que nenhum dom vos falta, esperando a manifestação de
nosso Senhor Jesus Cristo, o qual vos confirmará também até ao
fim, para serdes irrepreensíveis no dia de nosso Senhor Jesus
Cristo. Fiel é Deus, pelo qual fostes chamados para a comunhão
de seu Filho Jesus Cristo nosso Senhor.

Rogo-vos, porém, irmãos, pelo nome de nosso Senhor Jesus Cristo,
que digais todos uma mesma coisa, e que não haja entre vós
dissensões; antes sejais unidos em um mesmo pensamento e em um mesmo
parecer. Porque a respeito de vós, irmãos meus, me foi
comunicado pelos da família de Cloé que há contendas entre vós.
Quero dizer com isto, que cada um de vós diz: Eu sou de
Paulo, e eu de Apolo, e eu de Cefas, e eu de Cristo. Está
Cristo dividido? foi Paulo crucificado por vós? ou fostes vós
batizados em nome de Paulo?

Dou graças a Deus, porque a nenhum de vós batizei, senão a Crispo
e a Gaio, para que ninguém diga que fostes batizados em meu
nome. E batizei também a família de Estéfanas; além destes,
não sei se batizei algum outro.

Porque Cristo enviou-me, não para batizar, mas para evangelizar;
não em sabedoria de palavras, para que a cruz de Cristo se não faça
vã. Porque a palavra da cruz é loucura para os que perecem;
mas para nós, que somos salvos, é o poder de Deus. Porque
está escrito: Destruirei a sabedoria dos sábios, e aniquilarei a
inteligência dos inteligentes. Onde está o sábio? Onde está o
escriba? Onde está o inquiridor deste século? Porventura não tornou
Deus louca a sabedoria deste mundo? Visto como na sabedoria
de Deus o mundo não conheceu a Deus pela sua sabedoria, aprouve a
Deus salvar os crentes pela loucura da pregação. Porque os
judeus pedem sinal, e os gregos buscam sabedoria; mas nós
pregamos a Cristo crucificado, que é escândalo para os judeus, e
loucura para os gregos. Mas para os que são chamados, tanto
judeus como gregos, lhes pregamos a Cristo, poder de Deus, e
sabedoria de Deus. Porque a loucura de Deus é mais sábia do
que os homens; e a fraqueza de Deus é mais forte do que os homens.
Porque, vede, irmãos, a vossa vocação, que não são muitos os
sábios segundo a carne, nem muitos os poderosos, nem muitos os
nobres que são chamados. Mas Deus escolheu as coisas loucas
deste mundo para confundir as sábias; e Deus escolheu as coisas
fracas deste mundo para confundir as fortes; e Deus escolheu
as coisas vis deste mundo, e as desprezíveis, e as que não são, para
aniquilar as que são; para que nenhuma carne se glorie
perante ele. Mas vós sois dele, em Jesus Cristo, o qual para
nós foi feito por Deus sabedoria, e justiça, e santificação, e
redenção; para que, como está escrito: Aquele que se gloria
glorie-se no Senhor.

\medskip

\lettrine{2} E eu, irmãos, quando fui ter convosco,
anunciando-vos o testemunho de Deus, não fui com sublimidade de
palavras ou de sabedoria. Porque nada me propus saber entre vós,
senão a Jesus Cristo, e este crucificado. E eu estive convosco
em fraqueza, e em temor, e em grande tremor. A minha palavra, e
a minha pregação, não consistiram em palavras persuasivas de
sabedoria humana, mas em demonstração de Espírito e de poder;
para que a vossa fé não se apoiasse em sabedoria dos homens, mas
no poder de Deus.

Todavia falamos sabedoria entre os perfeitos; não, porém, a
sabedoria deste mundo, nem dos príncipes deste mundo, que se
aniquilam; mas falamos a sabedoria de Deus, oculta em mistério,
a qual Deus ordenou antes dos séculos para nossa glória; a qual
nenhum dos príncipes deste mundo conheceu; porque, se a conhecessem,
nunca crucificariam ao Senhor da glória. Mas, como está escrito:
As coisas que o olho não viu, e o ouvido não ouviu, e não subiram ao
coração do homem, são as que Deus preparou para os que o amam.
Mas Deus no-las revelou pelo seu Espírito; porque o Espírito
penetra todas as coisas, ainda as profundezas de Deus.
Porque, qual dos homens sabe as coisas do homem, senão o
espírito do homem, que nele está? Assim também ninguém sabe as
coisas de Deus, senão o Espírito de Deus. Mas nós não
recebemos o espírito do mundo, mas o Espírito que provém de Deus,
para que pudéssemos conhecer o que nos é dado gratuitamente por
Deus. As quais também falamos, não com palavras de sabedoria
humana, mas com as que o Espírito Santo ensina, comparando as coisas
espirituais com as espirituais. Ora, o homem natural não
compreende as coisas do Espírito de Deus, porque lhe parecem
loucura; e não pode entendê-las, porque elas se discernem
espiritualmente. Mas o que é espiritual discerne bem tudo, e
ele de ninguém é discernido. Porque, quem conheceu a mente do
Senhor, para que possa instruí-lo? Mas nós temos a mente de Cristo.

\medskip

\lettrine{3} E eu, irmãos, não vos pude falar como a
espirituais, mas como a carnais, como a meninos em Cristo. Com
leite vos criei, e não com carne, porque ainda não podíeis, nem
tampouco ainda agora podeis, porque ainda sois carnais; pois,
havendo entre vós inveja, contendas e dissensões, não sois
porventura carnais, e não andais segundo os homens? Porque,
dizendo um: Eu sou de Paulo; e outro: Eu de Apolo; porventura não
sois carnais?

Pois, quem é Paulo, e quem é Apolo, senão ministros pelos quais
crestes, e conforme o que o Senhor deu a cada um? Eu plantei,
Apolo regou; mas Deus deu o crescimento. Por isso, nem o que
planta é alguma coisa, nem o que rega, mas Deus, que dá o
crescimento. Ora, o que planta e o que rega são um; mas cada um
receberá o seu galardão segundo o seu trabalho. Porque nós somos
cooperadores de Deus; vós sois lavoura de Deus e edifício de Deus.
Segundo a graça de Deus que me foi dada, pus eu, como sábio
arquiteto, o fundamento, e outro edifica sobre ele; mas veja cada um
como edifica sobre ele.

Porque ninguém pode pôr outro fundamento além do que já está
posto, o qual é Jesus Cristo. E, se alguém sobre este
fundamento formar um edifício de ouro, prata, pedras preciosas,
madeira, feno, palha, a obra de cada um se manifestará; na
verdade o dia a declarará, porque pelo fogo será descoberta; e o
fogo provará qual seja a obra de cada um. Se a obra que
alguém edificou nessa parte permanecer, esse receberá galardão.
Se a obra de alguém se queimar, sofrerá detrimento; mas o tal
será salvo, todavia como pelo fogo.

Não sabeis vós que sois o templo de Deus e que o Espírito de Deus
habita em vós? Se alguém destruir o templo de Deus, Deus o
destruirá; porque o templo de Deus, que sois vós, é santo.

Ninguém se engane a si mesmo. Se alguém dentre vós se tem por
sábio neste mundo, faça-se louco para ser sábio. Porque a
sabedoria deste mundo é loucura diante de Deus; pois está escrito:
Ele apanha os sábios na sua própria astúcia. E outra vez: O
Senhor conhece os pensamentos dos sábios, que são vãos.

Portanto, ninguém se glorie nos homens; porque tudo é vosso;
seja Paulo, seja Apolo, seja Cefas, seja o mundo, seja a
vida, seja a morte, seja o presente, seja o futuro; tudo é vosso,
e vós de Cristo, e Cristo de Deus.

\medskip

\lettrine{4} Que os homens nos considerem como ministros de
Cristo, e despenseiros dos mistérios de Deus. Além disso
requer-se dos despenseiros que cada um se ache fiel. Todavia, a
mim mui pouco se me dá de ser julgado por vós, ou por algum juízo
humano; nem eu tampouco a mim mesmo me julgo. Porque em nada me
sinto culpado; mas nem por isso me considero justificado, pois quem
me julga é o Senhor. Portanto, nada julgueis antes de tempo, até
que o Senhor venha, o qual também trará à luz as coisas ocultas das
trevas, e manifestará os desígnios dos corações; e então cada um
receberá de Deus o louvor. E eu, irmãos, apliquei estas coisas,
por semelhança, a mim e a Apolo, por amor de vós; para que em nós
aprendais a não ir além do que está escrito, não vos ensoberbecendo
a favor de um contra outro.

Porque, quem te faz diferente? E que tens tu que não tenhas
recebido? E, se o recebeste, por que te glorias, como se não o
houveras recebido? Já estais fartos! já estais ricos! sem nós
reinais! e quisera reinásseis para que também nós viéssemos a reinar
convosco! Porque tenho para mim, que Deus a nós, apóstolos, nos
pôs por últimos, como condenados à morte; pois somos feitos
espetáculo ao mundo, aos anjos, e aos homens. Nós somos
loucos por amor de Cristo, e vós sábios em Cristo; nós fracos, e vós
fortes; vós ilustres, e nós vis. Até esta presente hora
sofremos fome, e sede, e estamos nus, e recebemos bofetadas, e não
temos pousada certa, e nos afadigamos, trabalhando com nossas
próprias mãos. Somos injuriados, e bendizemos; somos perseguidos, e
sofremos; somos blasfemados, e rogamos; até ao presente temos
chegado a ser como o lixo deste mundo, e como a escória de todos.

Não escrevo estas coisas para vos envergonhar; mas admoesto-vos
como meus filhos amados. Porque ainda que tivésseis dez mil
aios em Cristo, não teríeis, contudo, muitos pais; porque eu pelo
evangelho vos gerei em Jesus Cristo. Admoesto-vos, portanto,
a que sejais meus imitadores.

Por esta causa vos mandei Timóteo, que é meu filho amado, e fiel
no Senhor, o qual vos lembrará os meus caminhos em Cristo, como por
toda a parte ensino em cada igreja. Mas alguns andam
ensoberbecidos, como se eu não houvesse de ir ter convosco.
Mas em breve irei ter convosco, se o Senhor quiser, e então
conhecerei, não as palavras dos que andam ensoberbecidos, mas o
poder. Porque o reino de Deus não consiste em palavras, mas
em poder. Que quereis? Irei ter convosco com vara ou com amor
e espírito de mansidão?

\medskip

\lettrine{5} Geralmente se ouve que há entre vós fornicação, e
fornicação tal, que nem ainda entre os gentios se nomeia, como é
haver quem abuse da mulher de seu pai. Estais ensoberbecidos, e
nem ao menos vos entristecestes por não ter sido dentre vós tirado
quem cometeu tal ação. Eu, na verdade, ainda que ausente no
corpo, mas presente no espírito, já determinei, como se estivesse
presente, que o que tal ato praticou, em nome de nosso Senhor
Jesus Cristo, juntos vós e o meu espírito, pelo poder de nosso
Senhor Jesus Cristo, seja entregue a Satanás para destruição da
carne, para que o espírito seja salvo no dia do Senhor Jesus.
Não é boa a vossa jactância. Não sabeis que um pouco de fermento
faz levedar toda a massa?

Alimpai-vos, pois, do fermento velho, para que sejais uma nova
massa, assim como estais sem fermento. Porque Cristo, nossa páscoa,
foi sacrificado por nós. Por isso façamos a festa, não com o
fermento velho, nem com o fermento da maldade e da malícia, mas com
os ázimos da sinceridade e da verdade.

Já por carta vos tenho escrito, que não vos associeis com os que
se prostituem; isto não quer dizer absolutamente com os
devassos deste mundo, ou com os avarentos, ou com os roubadores, ou
com os idólatras; porque então vos seria necessário sair do mundo.
Mas agora vos escrevi que não vos associeis com aquele que,
dizendo-se irmão, for devasso, ou avarento, ou idólatra, ou
maldizente, ou beberrão, ou roubador; com o tal nem ainda comais.
Porque, que tenho eu em julgar também os que estão de fora?
Não julgais vós os que estão dentro? Mas Deus julga os que
estão de fora. Tirai pois dentre vós a esse iníquo.

\medskip

\lettrine{6} Ousa algum de vós, tendo algum negócio contra
outro, ir a juízo perante os injustos, e não perante os santos?
Não sabeis vós que os santos hão de julgar o mundo? Ora, se o
mundo deve ser julgado por vós, sois porventura indignos de julgar
as coisas mínimas? Não sabeis vós que havemos de julgar os
anjos? Quanto mais as coisas pertencentes a esta vida? Então, se
tiverdes negócios em juízo, pertencentes a esta vida, pondes para
julgá-los os que são de menos estima na igreja? Para vos
envergonhar o digo. Não há, pois, entre vós sábios, nem mesmo um,
que possa julgar entre seus irmãos? Mas o irmão vai a juízo com
o irmão, e isto perante infiéis. Na verdade é já realmente uma
falta entre vós, terdes demandas uns contra os outros. Por que não
sofreis antes a injustiça? Por que não sofreis antes o dano? Mas
vós mesmos fazeis a injustiça e fazeis o dano, e isto aos irmãos.

Não sabeis que os injustos não hão de herdar o reino de Deus?
Não erreis: nem os devassos, nem os idólatras, nem os
adúlteros, nem os efeminados, nem os sodomitas, nem os ladrões, nem
os avarentos, nem os bêbados, nem os maldizentes, nem os roubadores
herdarão o reino de Deus. E é o que alguns têm sido; mas
haveis sido lavados, mas haveis sido santificados, mas haveis sido
justificados em nome do Senhor Jesus, e pelo Espírito do nosso Deus.

Todas as coisas me são lícitas, mas nem todas as coisas convêm.
Todas as coisas me são lícitas, mas eu não me deixarei dominar por
nenhuma. Os alimentos são para o estômago e o estômago para
os alimentos; Deus, porém, aniquilará tanto um como os outros. Mas o
corpo não é para a prostituição, senão para o Senhor, e o Senhor
para o corpo. Ora, Deus, que também ressuscitou o Senhor, nos
ressuscitará a nós pelo seu poder. Não sabeis vós que os
vossos corpos são membros de Cristo? Tomarei, pois, os membros de
Cristo, e fá-los-ei membros de uma meretriz? Não, por certo.
Ou não sabeis que o que se ajunta com a meretriz, faz-se um
corpo com ela? Porque serão, disse, dois numa só carne. Mas o
que se ajunta com o Senhor é um mesmo espírito. Fugi da
prostituição. Todo o pecado que o homem comete é fora do corpo; mas
o que se prostitui peca contra o seu próprio corpo. Ou não
sabeis que o vosso corpo é o templo do Espírito Santo, que habita em
vós, proveniente de Deus, e que não sois de vós mesmos?
Porque fostes comprados por bom preço; glorificai, pois, a
Deus no vosso corpo, e no vosso espírito, os quais pertencem a Deus.

\medskip

\lettrine{7} Ora, quanto às coisas que me escrevestes, bom
seria que o homem não tocasse em mulher; mas, por causa da
prostituição, cada um tenha a sua própria mulher, e cada uma tenha o
seu próprio marido. O marido pague à mulher a devida
benevolência, e da mesma sorte a mulher ao marido. A mulher não
tem poder sobre o seu próprio corpo, mas tem-no o marido; e também
da mesma maneira o marido não tem poder sobre o seu próprio corpo,
mas tem-no a mulher. Não vos priveis um ao outro, senão por
consentimento mútuo por algum tempo, para vos aplicardes ao jejum e
à oração; e depois ajuntai-vos outra vez, para que Satanás não vos
tente pela vossa incontinência. Digo, porém, isto como que por
permissão e não por mandamento. Porque quereria que todos os
homens fossem como eu mesmo; mas cada um tem de Deus o seu próprio
dom, um de uma maneira e outro de outra. Digo, porém, aos
solteiros e às viúvas, que lhes é bom se ficarem como eu. Mas,
se não podem conter-se, casem-se. Porque é melhor casar do que
abrasar-se.

Todavia, aos casados mando, não eu mas o Senhor, que a mulher não
se aparte do marido. Se, porém, se apartar, que fique sem
casar, ou que se reconcilie com o marido; e que o marido não deixe a
mulher. Mas aos outros digo eu, não o Senhor: Se algum irmão
tem mulher descrente, e ela consente em habitar com ele, não a
deixe. E se alguma mulher tem marido descrente, e ele
consente em habitar com ela, não o deixe. Porque o marido
descrente é santificado pela mulher; e a mulher descrente é
santificada pelo marido; de outra sorte os vossos filhos seriam
imundos; mas agora são santos. Mas, se o descrente se
apartar, aparte-se; porque neste caso o irmão, ou irmã, não esta
sujeito à servidão; mas Deus chamou-nos para a paz. Porque,
de onde sabes, ó mulher, se salvarás teu marido? ou, de onde sabes,
ó marido, se salvarás tua mulher?

E assim cada um ande como Deus lhe repartiu, cada um como o
Senhor o chamou. É o que ordeno em todas as igrejas. É alguém
chamado, estando circuncidado? fique circuncidado. É alguém chamado
estando incircuncidado? não se circuncide. A circuncisão é
nada e a incircuncisão nada é, mas, sim, a observância dos
mandamentos de Deus. Cada um fique na vocação em que foi
chamado. Foste chamado sendo servo? não te dê cuidado; e, se
ainda podes ser livre, aproveita a ocasião. Porque o que é
chamado pelo Senhor, sendo servo, é liberto do Senhor; e da mesma
maneira também o que é chamado sendo livre, servo é de Cristo.
Fostes comprados por bom preço; não vos façais servos dos
homens. Irmãos, cada um fique diante de Deus no estado em que
foi chamado.

Ora, quanto às virgens, não tenho mandamento do Senhor; dou,
porém, o meu parecer, como quem tem alcançado misericórdia do Senhor
para ser fiel. Tenho, pois, por bom, por causa da instante
necessidade, que é bom para o homem o estar assim. Estás
ligado à mulher? não busques separar-te. Estás livre de mulher? não
busques mulher. Mas, se te casares, não pecas; e, se a virgem
se casar, não peca. Todavia os tais terão tribulações na carne, e eu
quereria poupar-vos. Isto, porém, vos digo, irmãos, que o
tempo se abrevia; o que resta é que também os que têm mulheres sejam
como se não as tivessem; e os que choram, como se não
chorassem; e os que folgam, como se não folgassem; e os que compram,
como se não possuíssem; e os que usam deste mundo, como se
dele não abusassem, porque a aparência deste mundo passa. E
bem quisera eu que estivésseis sem cuidado. O solteiro cuida das
coisas do Senhor, em como há de agradar ao Senhor; mas o que
é casado cuida das coisas do mundo, em como há de agradar à mulher.
Há diferença entre a mulher casada e a virgem. A solteira
cuida das coisas do Senhor para ser santa, tanto no corpo como no
espírito; porém, a casada cuida das coisas do mundo, em como há de
agradar ao marido. E digo isto para proveito vosso; não para
vos enlaçar, mas para o que é decente e conveniente, para vos
unirdes ao Senhor sem distração alguma.

Mas, se alguém julga que trata indignamente a sua virgem, se
tiver passado a flor da idade, e se for necessário, que faça o tal o
que quiser; não peca; casem-se. Todavia o que está firme em
seu coração, não tendo necessidade, mas com poder sobre a sua
própria vontade, se resolveu no seu coração guardar a sua virgem,
faz bem. De sorte que, o que a dá em casamento faz bem; mas o
que não a dá em casamento faz melhor.

A mulher casada está ligada pela lei todo o tempo que o seu
marido vive; mas, se falecer o seu marido fica livre para casar com
quem quiser, contanto que seja no Senhor. Será, porém, mais
bem-aventurada se ficar assim, segundo o meu parecer, e também eu
cuido que tenho o Espírito de Deus.

\medskip

\lettrine{8} Ora, no tocante às coisas sacrificadas aos
ídolos, sabemos que todos temos ciência. A ciência incha, mas o amor
edifica. E, se alguém cuida saber alguma coisa, ainda não sabe
como convém saber. Mas, se alguém ama a Deus, esse é conhecido
dele.

Assim que, quanto ao comer das coisas sacrificadas aos ídolos,
sabemos que o ídolo nada é no mundo, e que não há outro Deus, senão
um só. Porque, ainda que haja também alguns que se chamem
deuses, quer no céu quer na terra (como há muitos deuses e muitos
senhores), todavia para nós há um só Deus, o Pai, de quem é tudo
e para quem nós vivemos; e um só Senhor, Jesus Cristo, pelo qual são
todas as coisas, e nós por ele.

Mas nem em todos há conhecimento; porque alguns até agora comem,
no seu costume para com o ídolo, coisas sacrificadas ao ídolo; e a
sua consciência, sendo fraca, fica contaminada. Ora a comida não
nos faz agradáveis a Deus, porque, se comemos, nada temos de mais e,
se não comemos, nada nos falta. Mas vede que essa liberdade não
seja de alguma maneira escândalo para os fracos. Porque, se
alguém te vir a ti, que tens ciência, sentado à mesa no templo dos
ídolos, não será a consciência do que é fraco induzida a comer das
coisas sacrificadas aos ídolos? E pela tua ciência perecerá o
irmão fraco, pelo qual Cristo morreu. Ora, pecando assim
contra os irmãos, e ferindo a sua fraca consciência, pecais contra
Cristo. Por isso, se a comida escandalizar a meu irmão, nunca
mais comerei carne, para que meu irmão não se escandalize.

\medskip

\lettrine{9} Não sou eu apóstolo? Não sou livre? Não vi eu a
Jesus Cristo Senhor nosso? Não sois vós a minha obra no Senhor?
Se eu não sou apóstolo para os outros, ao menos o sou para vós;
porque vós sois o selo do meu apostolado no Senhor.

Esta é minha defesa para com os que me condenam. Não temos nós
direito de comer e beber? Não temos nós direito de levar conosco
uma esposa crente, como também os demais apóstolos, e os irmãos do
Senhor, e Cefas? Ou só eu e Barnabé não temos direito de deixar
de trabalhar? Quem jamais milita à sua própria custa? Quem
planta a vinha e não come do seu fruto? Ou quem apascenta o gado e
não se alimenta do leite do gado? Digo eu isto segundo os
homens? Ou não diz a lei também o mesmo? Porque na lei de Moisés
está escrito: Não atarás a boca ao boi que trilha o grão. Porventura
tem Deus cuidado dos bois? Ou não o diz certamente por nós?
Certamente que por nós está escrito; porque o que lavra deve lavrar
com esperança e o que debulha deve debulhar com esperança de ser
participante. Se nós vos semeamos as coisas espirituais, será
muito que de vós recolhamos as carnais? Se outros participam
deste poder sobre vós, por que não, e mais justamente, nós? Mas nós
não usamos deste direito; antes suportamos tudo, para não pormos
impedimento algum ao evangelho de Cristo. Não sabeis vós que
os que administram o que é sagrado comem do que é do templo? E que
os que de contínuo estão junto ao altar, participam do altar?
Assim ordenou também o Senhor aos que anunciam o evangelho,
que vivam do evangelho.

Mas eu de nenhuma destas coisas usei, e não escrevi isto para que
assim se faça comigo; porque melhor me fora morrer, do que alguém
fazer vã esta minha glória. Porque, se anuncio o evangelho,
não tenho de que me gloriar, pois me é imposta essa obrigação; e ai
de mim, se não anunciar o evangelho! E por isso, se o faço de
boa mente, terei prêmio; mas, se de má vontade, apenas uma
dispensação me é confiada. Logo, que prêmio tenho? Que,
evangelizando, proponha de graça o evangelho de Cristo para não
abusar do meu poder no evangelho.

Porque, sendo livre para com todos, fiz-me servo de todos para
ganhar ainda mais. E fiz-me como judeu para os judeus, para
ganhar os judeus; para os que estão debaixo da lei, como se
estivesse debaixo da lei, para ganhar os que estão debaixo da lei.
Para os que estão sem lei, como se estivesse sem lei (não
estando sem lei para com Deus, mas debaixo da lei de Cristo), para
ganhar os que estão sem lei. Fiz-me como fraco para os
fracos, para ganhar os fracos. Fiz-me tudo para todos, para por
todos os meios chegar a salvar alguns. E eu faço isto por
causa do evangelho, para ser também participante dele.

Não sabeis vós que os que correm no estádio, todos, na verdade,
correm, mas um só leva o prêmio? Correi de tal maneira que o
alcanceis. E todo aquele que luta de tudo se abstém; eles o
fazem para alcançar uma coroa corruptível; nós, porém, uma
incorruptível. Pois eu assim corro, não como a coisa incerta;
assim combato, não como batendo no ar. Antes subjugo o meu
corpo, e o reduzo à servidão, para que, pregando aos outros, eu
mesmo não venha de alguma maneira a ficar reprovado.

\medskip

\lettrine{10} Ora, irmãos, não quero que ignoreis que nossos
pais estiveram todos debaixo da nuvem, e todos passaram pelo mar.
E todos foram batizados em Moisés, na nuvem e no mar, e
todos comeram de uma mesma comida espiritual, e beberam todos de
uma mesma bebida espiritual, porque bebiam da pedra espiritual que
os seguia; e a pedra era Cristo. Mas Deus não se agradou da
maior parte deles, por isso foram prostrados no deserto.

E estas coisas foram-nos feitas em figura, para que não cobicemos
as coisas más, como eles cobiçaram. Não vos façais, pois,
idólatras, como alguns deles, conforme está escrito: O povo
assentou-se a comer e a beber, e levantou-se para folgar. E não
nos prostituamos, como alguns deles fizeram; e caíram num dia vinte
e três mil. E não tentemos a Cristo, como alguns deles também
tentaram, e pereceram pelas serpentes. E não murmureis, como
também alguns deles murmuraram, e pereceram pelo destruidor.
Ora, tudo isto lhes sobreveio como figuras, e estão escritas
para aviso nosso, para quem já são chegados os fins dos séculos.
Aquele, pois, que cuida estar em pé, olhe não caia.
Não veio sobre vós tentação, senão humana; mas fiel é Deus,
que não vos deixará tentar acima do que podeis, antes com a tentação
dará também o escape, para que a possais suportar. Portanto,
meus amados, fugi da idolatria.

Falo como a entendidos; julgai vós mesmos o que digo.
Porventura o cálice de bênção, que abençoamos, não é a
comunhão do sangue de Cristo? O pão que partimos não é porventura a
comunhão do corpo de Cristo? Porque nós, sendo muitos, somos
um só pão e um só corpo, porque todos participamos do mesmo pão.
Vede a Israel segundo a carne; os que comem os sacrifícios
não são porventura participantes do altar? Mas que digo? Que
o ídolo é alguma coisa? Ou que o sacrificado ao ídolo é alguma
coisa? Antes digo que as coisas que os gentios sacrificam, as
sacrificam aos demônios, e não a Deus. E não quero que sejais
participantes com os demônios. Não podeis beber o cálice do
Senhor e o cálice dos demônios; não podeis ser participantes da mesa
do Senhor e da mesa dos demônios. Ou irritaremos o Senhor?
Somos nós mais fortes do que ele?

Todas as coisas me são lícitas, mas nem todas as coisas convêm;
todas as coisas me são lícitas, mas nem todas as coisas edificam.
Ninguém busque o proveito próprio; antes cada um o que é de
outrem. Comei de tudo quanto se vende no açougue, sem
perguntar nada, por causa da consciência. Porque a terra é do
Senhor e toda a sua plenitude. E, se algum dos infiéis vos
convidar, e quiserdes ir, comei de tudo o que se puser diante de
vós, sem nada perguntar, por causa da consciência. Mas, se
alguém vos disser: Isto foi sacrificado aos ídolos, não comais, por
causa daquele que vos advertiu e por causa da consciência; porque a
terra é do Senhor, e toda a sua plenitude. Digo, porém, a
consciência, não a tua, mas a do outro. Pois por que há de a minha
liberdade ser julgada pela consciência de outrem? E, se eu
com graça participo, por que sou blasfemado naquilo por que dou
graças? Portanto, quer comais quer bebais, ou façais outra
qualquer coisa, fazei tudo para glória de Deus. Portai-vos de
modo que não deis escândalo nem aos judeus, nem aos gregos, nem à
igreja de Deus. Como também eu em tudo agrado a todos, não
buscando o meu próprio proveito, mas o de muitos, para que assim se
possam salvar.

\medskip

\lettrine{11} Sede meus imitadores, como também eu de Cristo.
E louvo-vos, irmãos, porque em tudo vos lembrais de mim, e
retendes os preceitos como vo-los entreguei. Mas quero que
saibais que Cristo é a cabeça de todo o homem, e o homem a cabeça da
mulher; e Deus a cabeça de Cristo. Todo o homem que ora ou
profetiza, tendo a cabeça coberta, desonra a sua própria cabeça.
Mas toda a mulher que ora ou profetiza com a cabeça descoberta,
desonra a sua própria cabeça, porque é como se estivesse rapada.
Portanto, se a mulher não se cobre com véu, tosquie-se também.
Mas, se para a mulher é coisa indecente tosquiar-se ou rapar-se, que
ponha o véu. O homem, pois, não deve cobrir a cabeça, porque é a
imagem e glória de Deus, mas a mulher é a glória do homem.
Porque o homem não provém da mulher, mas a mulher do homem.
Porque também o homem não foi criado por causa da mulher, mas a
mulher por causa do homem. Portanto, a mulher deve ter sobre
a cabeça sinal de poderio, por causa dos anjos. Todavia, nem
o homem é sem a mulher, nem a mulher sem o homem, no Senhor.
Porque, como a mulher provém do homem, assim também o homem
provém da mulher, mas tudo vem de Deus. Julgai entre vós
mesmos: é decente que a mulher ore a Deus descoberta? Ou não
vos ensina a mesma natureza que é desonra para o homem ter cabelo
crescido? Mas ter a mulher cabelo crescido lhe é honroso,
porque o cabelo lhe foi dado em lugar de véu. Mas, se alguém
quiser ser contencioso, nós não temos tal costume, nem as igrejas de
Deus.

Nisto, porém, que vou dizer-vos não vos louvo; porquanto vos
ajuntais, não para melhor, senão para pior. Porque antes de
tudo ouço que, quando vos ajuntais na igreja, há entre vós
dissensões; e em parte o creio. E até importa que haja entre
vós heresias, para que os que são sinceros se manifestem entre vós.
De sorte que, quando vos ajuntais num lugar, não é para comer
a ceia do Senhor. Porque, comendo, cada um toma
antecipadamente a sua própria ceia; e assim um tem fome e outro
embriaga-se. Não tendes porventura casas para comer e para
beber? Ou desprezais a igreja de Deus, e envergonhais os que nada
têm? Que vos direi? Louvar-vos-ei? Nisto não vos louvo.

Porque eu recebi do Senhor o que também vos ensinei: que o Senhor
Jesus, na noite em que foi traído, tomou o pão; e, tendo dado
graças, o partiu e disse: Tomai, comei; isto é o meu corpo que é
partido por vós; fazei isto em memória de mim.
Semelhantemente também, depois de cear, tomou o cálice,
dizendo: Este cálice é o novo testamento no meu sangue; fazei isto,
todas as vezes que beberdes, em memória de mim. Porque todas
as vezes que comerdes este pão e beberdes este cálice anunciais a
morte do Senhor, até que venha. Portanto, qualquer que comer
este pão, ou beber o cálice do Senhor indignamente, será culpado do
corpo e do sangue do Senhor. Examine-se, pois, o homem a si
mesmo, e assim coma deste pão e beba deste cálice. Porque o
que come e bebe indignamente, come e bebe para sua própria
condenação, não discernindo o corpo do Senhor. Por causa
disto há entre vós muitos fracos e doentes, e muitos que dormem.
Porque, se nós nos julgássemos a nós mesmos, não seríamos
julgados. Mas, quando somos julgados, somos repreendidos pelo
Senhor, para não sermos condenados com o mundo. Portanto,
meus irmãos, quando vos ajuntais para comer, esperai uns pelos
outros. Mas, se algum tiver fome, coma em casa, para que não
vos ajunteis para condenação. Quanto às demais coisas, ordená-las-ei
quando for.

\medskip

\lettrine{12} Acerca dos dons espirituais, não quero, irmãos,
que sejais ignorantes. Vós bem sabeis que éreis gentios, levados
aos ídolos mudos, conforme éreis guiados. Portanto, vos quero
fazer compreender que ninguém que fala pelo Espírito de Deus diz:
Jesus é anátema, e ninguém pode dizer que Jesus é o Senhor, senão
pelo Espírito Santo. Ora, há diversidade de dons, mas o Espírito
é o mesmo. E há diversidade de ministérios, mas o Senhor é o
mesmo. E há diversidade de operações, mas é o mesmo Deus que
opera tudo em todos. Mas a manifestação do Espírito é dada a
cada um, para o que for útil. Porque a um pelo Espírito é dada a
palavra da sabedoria; e a outro, pelo mesmo Espírito, a palavra da
ciência; e a outro, pelo mesmo Espírito, a fé; e a outro, pelo
mesmo Espírito, os dons de curar; e a outro a operação de
maravilhas; e a outro a profecia; e a outro o dom de discernir os
espíritos; e a outro a variedade de línguas; e a outro a
interpretação das línguas. Mas um só e o mesmo Espírito opera
todas estas coisas, repartindo particularmente a cada um como quer.

Porque, assim como o corpo é um, e tem muitos membros, e todos os
membros, sendo muitos, são um só corpo, assim é Cristo também.
Pois todos nós fomos batizados em um Espírito, formando um
corpo, quer judeus, quer gregos, quer servos, quer livres, e todos
temos bebido de um Espírito. Porque também o corpo não é um
só membro, mas muitos. Se o pé disser: Porque não sou mão,
não sou do corpo; não será por isso do corpo? E se a orelha
disser: Porque não sou olho não sou do corpo; não será por isso do
corpo? Se todo o corpo fosse olho, onde estaria o ouvido? Se
todo fosse ouvido, onde estaria o olfato? Mas agora Deus
colocou os membros no corpo, cada um deles como quis. E, se
todos fossem um só membro, onde estaria o corpo? Assim, pois,
há muitos membros, mas um corpo. E o olho não pode dizer à
mão: Não tenho necessidade de ti; nem ainda a cabeça aos pés: Não
tenho necessidade de vós. Antes, os membros do corpo que
parecem ser os mais fracos são necessários; e os que
reputamos serem menos honrosos no corpo, a esses honramos muito
mais; e aos que em nós são menos decorosos damos muito mais honra.
Porque os que em nós são mais nobres não têm necessidade
disso, mas Deus assim formou o corpo, dando muito mais honra ao que
tinha falta dela; para que não haja divisão no corpo, mas
antes tenham os membros igual cuidado uns dos outros. De
maneira que, se um membro padece, todos os membros padecem com ele;
e, se um membro é honrado, todos os membros se regozijam com ele.

Ora, vós sois o corpo de Cristo, e seus membros em particular.
E a uns pôs Deus na igreja, primeiramente apóstolos, em
segundo lugar profetas, em terceiro doutores, depois milagres,
depois dons de curar, socorros, governos, variedades de línguas.
Porventura são todos apóstolos? são todos profetas? são todos
doutores? são todos operadores de milagres? Têm todos o dom
de curar? falam todos diversas línguas? interpretam todos?
Portanto, procurai com zelo os melhores dons; e eu vos
mostrarei um caminho mais excelente.

\medskip

\lettrine{13} Ainda que eu falasse as línguas dos homens e dos
anjos, e não tivesse amor, seria como o metal que soa ou como o sino
que tine. E ainda que tivesse o dom de profecia, e conhecesse
todos os mistérios e toda a ciência, e ainda que tivesse toda a fé,
de maneira tal que transportasse os montes, e não tivesse amor, nada
seria. E ainda que distribuísse toda a minha fortuna para
sustento dos pobres, e ainda que entregasse o meu corpo para ser
queimado, e não tivesse amor, nada disso me aproveitaria.

O amor é sofredor, é benigno; o amor não é invejoso; o amor não
trata com leviandade, não se ensoberbece. Não se porta com
indecência, não busca os seus interesses, não se irrita, não
suspeita mal; não folga com a injustiça, mas folga com a
verdade; tudo sofre, tudo crê, tudo espera, tudo suporta.

O amor nunca falha; mas havendo profecias, serão aniquiladas;
havendo línguas, cessarão; havendo ciência, desaparecerá;
porque, em parte, conhecemos, e em parte profetizamos;
mas, quando vier o que é perfeito, então o que o é em parte
será aniquilado. Quando eu era menino, falava como menino,
sentia como menino, discorria como menino, mas, logo que cheguei a
ser homem, acabei com as coisas de menino. Porque agora vemos
por espelho em enigma, mas então veremos face a face; agora conheço
em parte, mas então conhecerei como também sou conhecido.
Agora, pois, permanecem a fé, a esperança e o amor, estes
três, mas o maior destes é o amor.

\medskip

\lettrine{14} Segui o amor, e procurai com zelo os dons
espirituais, mas principalmente o de profetizar. Porque o que
fala em língua desconhecida não fala aos homens, senão a Deus;
porque ninguém o entende, e em espírito fala mistérios. Mas o
que profetiza fala aos homens, para edificação, exortação e
consolação. O que fala em língua desconhecida edifica-se a si
mesmo, mas o que profetiza edifica a igreja. E eu quero que
todos vós faleis em línguas, mas muito mais que profetizeis; porque
o que profetiza é maior do que o que fala em línguas, a não ser que
também interprete para que a igreja receba edificação.

E agora, irmãos, se eu for ter convosco falando em línguas, que
vos aproveitaria, se não vos falasse ou por meio da revelação, ou da
ciência, ou da profecia, ou da doutrina? Da mesma sorte, se as
coisas inanimadas, que fazem som, seja flauta, seja cítara, não
formarem sons distintos, como se conhecerá o que se toca com a
flauta ou com a cítara? Porque, se a trombeta der sonido
incerto, quem se preparará para a batalha? Assim também vós, se
com a língua não pronunciardes palavras bem inteligíveis, como se
entenderá o que se diz? porque estareis como que falando ao ar.
Há, por exemplo, tanta espécie de vozes no mundo, e nenhuma
delas é sem significação. Mas, se eu ignorar o sentido da
voz, serei bárbaro para aquele a quem falo, e o que fala será
bárbaro para mim. Assim também vós, como desejais dons
espirituais, procurai abundar neles, para edificação da igreja.
Por isso, o que fala em língua desconhecida, ore para que a
possa interpretar. Porque, se eu orar em língua desconhecida,
o meu espírito ora bem, mas o meu entendimento fica sem fruto.

Que farei, pois? Orarei com o espírito, mas também orarei com o
entendimento; cantarei com o espírito, mas também cantarei com o
entendimento. De outra maneira, se tu bendisseres com o
espírito, como dirá o que ocupa o lugar de indouto, o Amém, sobre a
tua ação de graças, visto que não sabe o que dizes? Porque
realmente tu dás bem as graças, mas o outro não é edificado.
Dou graças ao meu Deus, porque falo mais línguas do que vós
todos. Todavia eu antes quero falar na igreja cinco palavras
na minha própria inteligência, para que possa também instruir os
outros, do que dez mil palavras em língua desconhecida.
Irmãos, não sejais meninos no entendimento, mas sede meninos
na malícia, e adultos no entendimento.

Está escrito na lei: Por gente de outras línguas, e por outros
lábios, falarei a este povo; e ainda assim me não ouvirão, diz o
Senhor. De sorte que as línguas são um sinal, não para os
fiéis, mas para os infiéis; e a profecia não é sinal para os
infiéis, mas para os fiéis. Se, pois, toda a igreja se
congregar num lugar, e todos falarem em línguas, e entrarem indoutos
ou infiéis, não dirão porventura que estais loucos? Mas, se
todos profetizarem, e algum indouto ou infiel entrar, de todos é
convencido, de todos é julgado. Portanto, os segredos do seu
coração ficarão manifestos, e assim, lançando-se sobre o seu rosto,
adorará a Deus, publicando que Deus está verdadeiramente entre vós.

Que fareis pois, irmãos? Quando vos ajuntais, cada um de vós tem
salmo, tem doutrina, tem revelação, tem língua, tem interpretação.
Faça-se tudo para edificação. E, se alguém falar em língua
desconhecida, faça-se isso por dois, ou quando muito três, e por sua
vez, e haja intérprete. Mas, se não houver intérprete, esteja
calado na igreja, e fale consigo mesmo, e com Deus. E falem
dois ou três profetas, e os outros julguem. Mas, se a outro,
que estiver assentado, for revelada alguma coisa, cale-se o
primeiro. Porque todos podereis profetizar, uns depois dos
outros; para que todos aprendam, e todos sejam consolados. E
os espíritos dos profetas estão sujeitos aos profetas. Porque
Deus não é Deus de confusão, senão de paz, como em todas as igrejas
dos santos.

As vossas mulheres estejam caladas nas igrejas; porque não lhes é
permitido falar; mas estejam sujeitas, como também ordena a lei.
E, se querem aprender alguma coisa, interroguem em casa a
seus próprios maridos; porque é vergonhoso que as mulheres falem na
igreja.

Porventura saiu dentre vós a palavra de Deus? Ou veio ela somente
para vós? Se alguém cuida ser profeta, ou espiritual,
reconheça que as coisas que vos escrevo são mandamentos do Senhor.
Mas, se alguém ignora isto, que ignore. Portanto,
irmãos, procurai, com zelo, profetizar, e não proibais falar
línguas. Mas faça-se tudo decentemente e com ordem.

\medskip

\lettrine{15} Também vos notifico, irmãos, o evangelho que já
vos tenho anunciado; o qual também recebestes, e no qual também
permaneceis. Pelo qual também sois salvos se o retiverdes tal
como vo-lo tenho anunciado; se não é que crestes em vão. Porque
primeiramente vos entreguei o que também recebi: que Cristo morreu
por nossos pecados, segundo as Escrituras, e que foi sepultado,
e que ressuscitou ao terceiro dia, segundo as Escrituras. E que
foi visto por Cefas, e depois pelos doze. Depois foi visto, uma
vez, por mais de quinhentos irmãos, dos quais vive ainda a maior
parte, mas alguns já dormem também. Depois foi visto por Tiago,
depois por todos os apóstolos. E por derradeiro de todos me
apareceu também a mim, como a um abortivo. Porque eu sou o menor
dos apóstolos, que não sou digno de ser chamado apóstolo, pois que
persegui a igreja de Deus. Mas pela graça de Deus sou o que
sou; e a sua graça para comigo não foi vã, antes trabalhei muito
mais do que todos eles; todavia não eu, mas a graça de Deus, que
está comigo. Então, ou seja eu ou sejam eles, assim pregamos
e assim haveis crido.

Ora, se se prega que Cristo ressuscitou dentre os mortos, como
dizem alguns dentre vós que não há ressurreição de mortos? E,
se não há ressurreição de mortos, também Cristo não ressuscitou.
E, se Cristo não ressuscitou, logo é vã a nossa pregação, e
também é vã a vossa fé. E assim somos também considerados
como falsas testemunhas de Deus, pois testificamos de Deus, que
ressuscitou a Cristo, ao qual, porém, não ressuscitou, se, na
verdade, os mortos não ressuscitam. Porque, se os mortos não
ressuscitam, também Cristo não ressuscitou. E, se Cristo não
ressuscitou, é vã a vossa fé, e ainda permaneceis nos vossos
pecados. E também os que dormiram em Cristo estão perdidos.
Se esperamos em Cristo só nesta vida, somos os mais
miseráveis de todos os homens.

Mas de fato Cristo ressuscitou dentre os mortos, e foi feito as
primícias dos que dormem. Porque assim como a morte veio por
um homem, também a ressurreição dos mortos veio por um homem.
Porque, assim como todos morrem em Adão, assim também todos
serão vivificados em Cristo. Mas cada um por sua ordem:
Cristo as primícias, depois os que são de Cristo, na sua vinda.
Depois virá o fim, quando tiver entregado o reino a Deus, ao
Pai, e quando houver aniquilado todo o império, e toda a potestade e
força. Porque convém que reine até que haja posto a todos os
inimigos debaixo de seus pés. Ora, o último inimigo que há de
ser aniquilado é a morte. Porque todas as coisas sujeitou
debaixo de seus pés. Mas, quando diz que todas as coisas lhe estão
sujeitas, claro está que se excetua aquele que lhe sujeitou todas as
coisas. E, quando todas as coisas lhe estiverem sujeitas,
então também o mesmo Filho se sujeitará àquele que todas as coisas
lhe sujeitou, para que Deus seja tudo em todos. Doutra
maneira, que farão os que se batizam pelos mortos, se absolutamente
os mortos não ressuscitam? Por que se batizam eles então pelos
mortos? Por que estamos nós também a toda a hora em perigo?
Eu protesto que cada dia morro, gloriando-me em vós, irmãos,
por Cristo Jesus nosso Senhor. Se, como homem, combati em
Éfeso contra as bestas, que me aproveita isso, se os mortos não
ressuscitam? Comamos e bebamos, que amanhã morreremos. Não
vos enganeis: as más conversações corrompem os bons costumes.
Vigiai justamente e não pequeis; porque alguns ainda não têm
o conhecimento de Deus; digo-o para vergonha vossa.

Mas alguém dirá: Como ressuscitarão os mortos? E com que corpo
virão? Insensato! o que tu semeias não é vivificado, se
primeiro não morrer. E, quando semeias, não semeias o corpo
que há de nascer, mas o simples grão, como de trigo, ou de outra
qualquer semente. Mas Deus dá-lhe o corpo como quer, e a cada
semente o seu próprio corpo. Nem toda a carne é uma mesma
carne, mas uma é a carne dos homens, e outra a carne dos animais, e
outra a dos peixes e outra a das aves. E há corpos celestes e
corpos terrestres, mas uma é a glória dos celestes e outra a dos
terrestres. Uma é a glória do sol, e outra a glória da lua, e
outra a glória das estrelas; porque uma estrela difere em glória de
outra estrela. Assim também a ressurreição dentre os mortos.
Semeia-se o corpo em corrupção; ressuscitará em incorrupção.
Semeia-se em ignomínia, ressuscitará em glória. Semeia-se em
fraqueza, ressuscitará com vigor. Semeia-se corpo natural,
ressuscitará corpo espiritual. Se há corpo natural, há também corpo
espiritual. Assim está também escrito: O primeiro homem,
Adão, foi feito em alma vivente; o último Adão em espírito
vivificante. Mas não é primeiro o espiritual, senão o
natural; depois o espiritual. O primeiro homem, da terra, é
terreno; o segundo homem, o Senhor, é do céu. Qual o terreno,
tais são também os terrestres; e, qual o celestial, tais também os
celestiais. E, assim como trouxemos a imagem do terreno,
assim traremos também a imagem do celestial. E agora digo
isto, irmãos: que a carne e o sangue não podem herdar o reino de
Deus, nem a corrupção herdar a incorrupção.

Eis aqui vos digo um mistério: Na verdade, nem todos dormiremos,
mas todos seremos transformados; num momento, num abrir e
fechar de olhos, ante a última trombeta; porque a trombeta soará, e
os mortos ressuscitarão incorruptíveis, e nós seremos transformados.
Porque convém que isto que é corruptível se revista da
incorruptibilidade, e que isto que é mortal se revista da
imortalidade. E, quando isto que é corruptível se revestir da
incorruptibilidade, e isto que é mortal se revestir da imortalidade,
então cumprir-se-á a palavra que está escrita: Tragada foi a morte
na vitória. Onde está, ó morte, o teu aguilhão?\footnote{A
ponta de ferro da aguilhada; ferrão. Ponta aguçada; bico. Estímulo,
incitamento, incentivo. Sofrimento pungente.} Onde está, ó inferno,
a tua vitória? Ora, o aguilhão da morte é o pecado, e a força
do pecado é a lei. Mas graças a Deus que nos dá a vitória por
nosso Senhor Jesus Cristo.

Portanto, meus amados irmãos, sede firmes e constantes, sempre
abundantes na obra do Senhor, sabendo que o vosso trabalho não é vão
no Senhor.

\medskip

\lettrine{16} Ora, quanto à coleta que se faz para os santos,
fazei vós também o mesmo que ordenei às igrejas da Galácia. No
primeiro dia da semana cada um de vós ponha de parte o que puder
ajuntar, conforme a sua prosperidade, para que não se façam as
coletas quando eu chegar. E, quando tiver chegado, mandarei os
que por cartas aprovardes, para levar a vossa dádiva a Jerusalém.
E, se valer a pena que eu também vá, irão comigo.

Irei, porém, ter convosco depois de ter passado pela Macedônia
(porque tenho de passar pela Macedônia). E bem pode ser que
fique convosco, e passe também o inverno, para que me acompanheis
aonde quer que eu for. Porque não vos quero agora ver de
passagem, mas espero ficar convosco algum tempo, se o Senhor o
permitir. Ficarei, porém, em Éfeso até ao Pentecostes;
porque uma porta grande e eficaz se me abriu; e há muitos
adversários.

E, se Timóteo for, vede que esteja sem temor convosco; porque
trabalha na obra do Senhor, como eu também. Portanto, ninguém
o despreze, mas acompanhai-o em paz, para que venha ter comigo; pois
o espero com os irmãos. E, acerca do irmão Apolo, roguei-lhe
muito que fosse com os irmãos ter convosco, mas, na verdade, não
teve vontade de ir agora; irá, porém, quando se lhe oferecer boa
ocasião.

Vigiai, estai firmes na fé; portai-vos varonilmente, e
fortalecei-vos. Todas as vossas coisas sejam feitas com amor.
Agora vos rogo, irmãos (sabeis que a família de Estéfanas é
as primícias da Acaia, e que se tem dedicado ao ministério dos
santos), que também vos sujeiteis aos tais, e a todo aquele
que auxilia na obra e trabalha. Folgo, porém, com a vinda de
Estéfanas, de Fortunato e de Acaico; porque estes supriram o que da
vossa parte me faltava. Porque recrearam o meu espírito e o
vosso. Reconhecei, pois, aos tais.

As igrejas da Ásia vos saúdam. Saúdam-vos afetuosamente no Senhor
Áqüila e Priscila, com a igreja que está em sua casa. Todos
os irmãos vos saúdam. Saudai-vos uns aos outros com ósculo santo.
Saudação da minha própria mão, de Paulo. Se alguém não
ama ao Senhor Jesus Cristo, seja anátema. Maranata! A graça
do Senhor Jesus Cristo seja convosco. O meu amor seja com
todos vós em Cristo Jesus. Amém.

