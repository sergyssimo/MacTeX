\thispagestyle{empty}
\chapter*{Atos dos Apóstolos}

\lettrine{1} Fiz o primeiro tratado, ó Teófilo, acerca de tudo
que Jesus começou, não só a fazer, mas a ensinar, até ao dia em
que foi recebido em cima, depois de ter dado mandamentos, pelo
Espírito Santo, aos apóstolos que escolhera; aos quais também,
depois de ter padecido, se apresentou vivo, com muitas e infalíveis
provas, sendo visto por eles por espaço de quarenta dias, e falando
das coisas concernentes ao reino de Deus. E, estando com eles,
determinou-lhes que não se ausentassem de Jerusalém, mas que
esperassem a promessa do Pai, que (disse ele) de mim ouvistes.
Porque, na verdade, João batizou com água, mas vós sereis
batizados com o Espírito Santo, não muito depois destes dias.

Aqueles, pois, que se haviam reunido perguntaram-lhe, dizendo:
Senhor, restaurarás tu neste tempo o reino a Israel? E
disse-lhes: Não vos pertence saber os tempos ou as estações que o
Pai estabeleceu pelo seu próprio poder. Mas recebereis a virtude
do Espírito Santo, que há de vir sobre vós; e ser-me-eis
testemunhas, tanto em Jerusalém como em toda a Judéia e Samaria, e
até aos confins da terra. E, quando dizia isto, vendo-o eles,
foi elevado às alturas, e uma nuvem o recebeu, ocultando-o a seus
olhos. E, estando com os olhos fitos no céu, enquanto ele
subia, eis que junto deles se puseram dois homens vestidos de
branco. Os quais lhes disseram: Homens galileus, por que
estais olhando para o céu? Esse Jesus, que dentre vós foi recebido
em cima no céu, há de vir assim como para o céu o vistes ir.

Então voltaram para Jerusalém, do monte chamado das Oliveiras, o
qual está perto de Jerusalém, à distância do caminho de um sábado.
E, entrando, subiram ao cenáculo, onde habitavam Pedro e
Tiago, João e André, Filipe e Tomé, Bartolomeu e Mateus, Tiago,
filho de Alfeu, Simão, o Zelote, e Judas, irmão de Tiago.
Todos estes perseveravam unanimemente em oração e súplicas,
com as mulheres, e Maria mãe de Jesus, e com seus irmãos.

E naqueles dias, levantando-se Pedro no meio dos discípulos (ora
a multidão junta era de quase cento e vinte pessoas) disse:
Homens irmãos, convinha que se cumprisse a Escritura que o
Espírito Santo predisse pela boca de Davi, acerca de Judas, que foi
o guia daqueles que prenderam a Jesus; porque foi contado
conosco e alcançou sorte neste ministério. Ora, este adquiriu
um campo com o galardão da iniqüidade; e, precipitando-se, rebentou
pelo meio, e todas as suas entranhas se derramaram. E foi
notório a todos os que habitam em Jerusalém; de maneira que na sua
própria língua esse campo se chama Aceldama, isto é, Campo de
Sangue. Porque no livro dos Salmos está escrito: Fique
deserta a sua habitação, e não haja quem nela habite, tome outro o
seu bispado. É necessário, pois, que, dos homens que
conviveram conosco todo o tempo em que o Senhor Jesus entrou e saiu
dentre nós, começando desde o batismo de João até ao dia em
que de entre nós foi recebido em cima, um deles se faça conosco
testemunha da sua ressurreição. E apresentaram dois: José,
chamado Barsabás, que tinha por sobrenome o Justo, e Matias.
E, orando, disseram: Tu, Senhor, conhecedor dos corações de
todos, mostra qual destes dois tens escolhido, para que tome
parte neste ministério e apostolado, de que Judas se desviou, para
ir para o seu próprio lugar. E, lançando-lhes sortes, caiu a
sorte sobre Matias. E por voto comum foi contado com os onze
apóstolos.

\medskip

\lettrine{2} E, cumprindo-se o dia de Pentecostes, estavam
todos concordemente no mesmo lugar; e de repente veio do céu um
som, como de um vento veemente e impetuoso, e encheu toda a casa em
que estavam assentados. E foram vistas por eles línguas
repartidas, como que de fogo, as quais pousaram sobre cada um deles.
E todos foram cheios do Espírito Santo, e começaram a falar
noutras línguas, conforme o Espírito Santo lhes concedia que
falassem.

E em Jerusalém estavam habitando judeus, homens religiosos, de
todas as nações que estão debaixo do céu. E, quando aquele som
ocorreu, ajuntou-se uma multidão, e estava confusa, porque cada um
os ouvia falar na sua própria língua. E todos pasmavam e se
maravilhavam, dizendo uns aos outros: Pois quê! não são galileus
todos esses homens que estão falando? Como, pois, os ouvimos,
cada um, na nossa própria língua em que somos nascidos? Partos e
medos, elamitas e os que habitam na Mesopotâmia, Judéia, Capadócia,
Ponto e Ásia, e Frígia e Panfília, Egito e partes da Líbia,
junto a Cirene, e forasteiros romanos, tanto judeus como prosélitos,
cretenses e árabes, todos nós temos ouvido em nossas próprias
línguas falar das grandezas de Deus. E todos se maravilhavam
e estavam suspensos, dizendo uns para os outros: Que quer isto
dizer? E outros, zombando, diziam: Estão cheios de mosto.

Pedro, porém, pondo-se em pé com os onze, levantou a sua voz, e
disse-lhes: Homens judeus, e todos os que habitais em Jerusalém,
seja-vos isto notório, e escutai as minhas palavras. Estes
homens não estão embriagados, como vós pensais, sendo a terceira
hora do dia. Mas isto é o que foi dito pelo profeta Joel:
E nos últimos dias acontecerá, diz Deus, que do meu Espírito
derramarei sobre toda a carne; e os vossos filhos e as vossas filhas
profetizarão, os vossos jovens terão visões, e os vossos velhos
terão sonhos; e também do meu Espírito derramarei sobre os
meus servos e as minhas servas naqueles dias, e profetizarão;
e farei aparecer prodígios em cima, no céu; e sinais em baixo
na terra, sangue, fogo e vapor de fumo. O sol se converterá
em trevas, e a lua em sangue, antes de chegar o grande e glorioso
dia do Senhor; e acontecerá que todo aquele que invocar o
nome do Senhor será salvo. Homens israelitas, escutai estas
palavras: A Jesus Nazareno, homem aprovado por Deus entre vós com
maravilhas, prodígios e sinais, que Deus por ele fez no meio de vós,
como vós mesmos bem sabeis; a este que vos foi entregue pelo
determinado conselho e presciência de Deus, prendestes,
crucificastes e matastes pelas mãos de injustos; ao qual Deus
ressuscitou, soltas as ânsias da morte, pois não era possível que
fosse retido por ela; porque dele disse Davi: Sempre via
diante de mim o Senhor, porque está à minha direita, para que eu não
seja comovido; por isso se alegrou o meu coração, e a minha
língua exultou; e ainda a minha carne há de repousar em esperança;
pois não deixarás a minha alma no inferno, nem permitirás que
o teu Santo veja a corrupção; fizeste-me conhecidos os
caminhos da vida; com a tua face me encherás de júbilo.
Homens irmãos, seja-me lícito dizer-vos livremente acerca do
patriarca Davi, que ele morreu e foi sepultado, e entre nós está até
hoje a sua sepultura. Sendo, pois, ele profeta, e sabendo que
Deus lhe havia prometido com juramento que do fruto de seus lombos,
segundo a carne, levantaria o Cristo, para o assentar sobre o seu
trono, nesta previsão, disse da ressurreição de Cristo, que a
sua alma não foi deixada no inferno, nem a sua carne viu a
corrupção. Deus ressuscitou a este Jesus, do que todos nós
somos testemunhas. De sorte que, exaltado pela destra de
Deus, e tendo recebido do Pai a promessa do Espírito Santo, derramou
isto que vós agora vedes e ouvis. Porque Davi não subiu aos
céus, mas ele próprio diz: Disse o Senhor ao meu Senhor: Assenta-te
à minha direita, até que ponha os teus inimigos por escabelo
de teus pés. Saiba pois com certeza toda a casa de Israel que
a esse Jesus, a quem vós crucificastes, Deus o fez Senhor e Cristo.

E, ouvindo eles isto, compungiram-se em seu coração, e
perguntaram a Pedro e aos demais apóstolos: Que faremos, homens
irmãos? E disse-lhes Pedro: Arrependei-vos, e cada um de vós
seja batizado em nome de Jesus Cristo, para perdão dos pecados; e
recebereis o dom do Espírito Santo; porque a promessa vos diz
respeito a vós, a vossos filhos, e a todos os que estão longe, a
tantos quantos Deus nosso Senhor chamar. E com muitas outras
palavras isto testificava, e os exortava, dizendo: Salvai-vos desta
geração perversa. De sorte que foram batizados os que de bom
grado receberam a sua palavra; e naquele dia agregaram-se quase três
mil almas.

E perseveravam na doutrina dos apóstolos, e na comunhão, e no
partir do pão, e nas orações. E em toda a alma havia temor, e
muitas maravilhas e sinais se faziam pelos apóstolos. E todos
os que criam estavam juntos, e tinham tudo em comum. E
vendiam suas propriedades e bens, e repartiam com todos, segundo
cada um havia de mister. E, perseverando unânimes todos os
dias no templo, e partindo o pão em casa, comiam juntos com alegria
e singeleza de coração, louvando a Deus, e caindo na graça de
todo o povo. E todos os dias acrescentava o Senhor à igreja aqueles
que se haviam de salvar.

\medskip

\lettrine{3} E Pedro e João subiam juntos ao templo à hora da
oração, a nona. E era trazido um homem que desde o ventre de sua
mãe era coxo, o qual todos os dias punham à porta do templo, chamada
Formosa, para pedir esmola aos que entravam. O qual, vendo a
Pedro e a João que iam entrando no templo, pediu que lhe dessem uma
esmola. E Pedro, com João, fitando os olhos nele, disse: Olha
para nós. E olhou para eles, esperando receber deles alguma
coisa. E disse Pedro: Não tenho prata nem ouro; mas o que tenho
isso te dou. Em nome de Jesus Cristo, o Nazareno, levanta-te e anda.
E, tomando-o pela mão direita, o levantou, e logo os seus pés e
artelhos se firmaram. E, saltando ele, pôs-se em pé, e andou, e
entrou com eles no templo, andando, e saltando, e louvando a Deus.
E todo o povo o viu andar e louvar a Deus; e
conheciam-no, pois era ele o que se assentava a pedir esmola à porta
Formosa do templo; e ficaram cheios de pasmo e assombro, pelo que
lhe acontecera. E, apegando-se o coxo, que fora curado, a
Pedro e João, todo o povo correu atônito para junto deles, ao
alpendre chamado de Salomão.

E quando Pedro viu isto, disse ao povo: Homens israelitas, por
que vos maravilhais disto? Ou, por que olhais tanto para nós, como
se por nossa própria virtude ou santidade fizéssemos andar este
homem? O Deus de Abraão, de Isaque e de Jacó, o Deus de
nossos pais, glorificou a seu filho Jesus, a quem vós entregastes e
perante a face de Pilatos negastes, tendo ele determinado que fosse
solto. Mas vós negastes o Santo e o Justo, e pedistes que se
vos desse um homem homicida. E matastes o Príncipe da vida,
ao qual Deus ressuscitou dentre os mortos, do que nós somos
testemunhas. E pela fé no seu nome fez o seu nome fortalecer
a este que vedes e conheceis; sim, a fé que vem por ele, deu a este,
na presença de todos vós, esta perfeita saúde. E agora,
irmãos, eu sei que o fizestes por ignorância, como também os vossos
príncipes. Mas Deus assim cumpriu o que já dantes pela boca
de todos os seus profetas havia anunciado; que o Cristo havia de
padecer. Arrependei-vos, pois, e convertei-vos, para que
sejam apagados os vossos pecados, e venham assim os tempos do
refrigério pela presença do Senhor, e envie ele a Jesus
Cristo, que já dantes vos foi pregado. O qual convém que o
céu contenha até aos tempos da restauração de tudo, dos quais Deus
falou pela boca de todos os seus santos profetas, desde o princípio.
Porque Moisés disse aos pais: O Senhor vosso Deus levantará
de entre vossos irmãos um profeta semelhante a mim; a ele ouvireis
em tudo quanto vos disser. E acontecerá que toda a alma que
não escutar esse profeta será exterminada dentre o povo. Sim,
e todos os profetas, desde Samuel, todos quantos depois falaram,
também predisseram estes dias. Vós sois os filhos dos
profetas e da aliança que Deus fez com nossos pais, dizendo a
Abraão: Na tua descendência serão benditas todas as famílias da
terra. Ressuscitando Deus a seu Filho Jesus, primeiro o
enviou a vós, para que nisso vos abençoasse, no apartar, a cada um
de vós, das vossas maldades.

\medskip

\lettrine{4} E, estando eles falando ao povo, sobrevieram os
sacerdotes, e o capitão do templo, e os saduceus, doendo-se
muito de que ensinassem o povo, e anunciassem em Jesus a
ressurreição dentre os mortos. E lançaram mão deles, e os
encerraram na prisão até ao dia seguinte, pois já era tarde.
Muitos, porém, dos que ouviram a palavra creram, e chegou o
número desses homens a quase cinco mil.

E aconteceu, no dia seguinte, reunirem-se em Jerusalém os seus
principais, os anciãos, os escribas, e Anás, o sumo sacerdote, e
Caifás, e João, e Alexandre, e todos quantos havia da linhagem do
sumo sacerdote. E, pondo-os no meio, perguntaram: Com que poder
ou em nome de quem fizestes isto? Então Pedro, cheio do Espírito
Santo, lhes disse: Principais do povo, e vós, anciãos de Israel,
visto que hoje somos interrogados acerca do benefício feito a um
homem enfermo, e do modo como foi curado, seja conhecido de
vós todos, e de todo o povo de Israel, que em nome de Jesus Cristo,
o Nazareno, aquele a quem vós crucificastes e a quem Deus
ressuscitou dentre os mortos, em nome desse é que este está são
diante de vós. Ele é a pedra que foi rejeitada por vós, os
edificadores, a qual foi posta por cabeça de esquina. E em
nenhum outro há salvação, porque também debaixo do céu nenhum outro
nome há, dado entre os homens, pelo qual devamos ser salvos.
Então eles, vendo a ousadia de Pedro e João, e informados de
que eram homens sem letras e indoutos, maravilharam-se e
reconheceram que eles haviam estado com Jesus. E, vendo estar
com eles o homem que fora curado, nada tinham que dizer em
contrário.

Todavia, mandando-os sair fora do conselho, conferenciaram entre
si, dizendo: Que havemos de fazer a estes homens? porque a
todos os que habitam em Jerusalém é manifesto que por eles foi feito
um sinal notório, e não o podemos negar; Mas, para que não se
divulgue mais entre o povo, ameacemo-los para que não falem mais
nesse nome a homem algum. E, chamando-os, disseram-lhes que
absolutamente não falassem, nem ensinassem, no nome de Jesus.
Respondendo, porém, Pedro e João, lhes disseram: Julgai vós
se é justo, diante de Deus, ouvir-vos antes a vós do que a Deus;
porque não podemos deixar de falar do que temos visto e
ouvido. Mas eles ainda os ameaçaram mais e, não achando
motivo para os castigar, deixaram-nos ir, por causa do povo; porque
todos glorificavam a Deus pelo que acontecera; pois tinha
mais de quarenta anos o homem em quem se operara aquele milagre de
saúde.

E, soltos eles, foram para os seus, e contaram tudo o que lhes
disseram os principais dos sacerdotes e os anciãos. E,
ouvindo eles isto, unânimes levantaram a voz a Deus, e disseram:
Senhor, tu és o Deus que fizeste o céu, e a terra, e o mar e tudo o
que neles há; que disseste pela boca de Davi, teu servo: Por
que bramaram os gentios, e os povos pensaram coisas vãs?
Levantaram-se os reis da terra, e os príncipes se ajuntaram à
uma, contra o Senhor e contra o seu Ungido. Porque
verdadeiramente contra o teu Santo Filho Jesus, que tu ungiste, se
ajuntaram, não só Herodes, mas Pôncio Pilatos, com os gentios e os
povos de Israel; para fazerem tudo o que a tua mão e o teu
conselho tinham anteriormente determinado que se havia de fazer.
Agora, pois, ó Senhor, olha para as suas ameaças, e concede
aos teus servos que falem com toda a ousadia a tua palavra;
enquanto estendes a tua mão para curar, e para que se façam
sinais e prodígios pelo nome de teu santo Filho Jesus. E,
tendo orado, moveu-se o lugar em que estavam reunidos; e todos foram
cheios do Espírito Santo, e anunciavam com ousadia a palavra de
Deus.

E era um o coração e a alma da multidão dos que criam, e ninguém
dizia que coisa alguma do que possuía era sua própria, mas todas as
coisas lhes eram comuns. E os apóstolos davam, com grande
poder, testemunho da ressurreição do Senhor Jesus, e em todos eles
havia abundante graça. Não havia, pois, entre eles
necessitado algum; porque todos os que possuíam herdades ou casas,
vendendo-as, traziam o preço do que fora vendido, e o depositavam
aos pés dos apóstolos. E repartia-se a cada um, segundo a
necessidade que cada um tinha. Então José, cognominado pelos
apóstolos Barnabé (que, traduzido, é Filho da consolação), levita,
natural de Chipre, possuindo uma herdade, vendeu-a, e trouxe
o preço, e o depositou aos pés dos apóstolos.

\medskip

\lettrine{5} Mas um certo homem chamado Ananias, com Safira,
sua mulher, vendeu uma propriedade, e reteve parte do preço,
sabendo-o também sua mulher; e, levando uma parte, a depositou aos
pés dos apóstolos. Disse então Pedro: Ananias, por que encheu
Satanás o teu coração, para que mentisses ao Espírito Santo, e
retivesses parte do preço da herdade? Guardando-a não ficava
para ti? E, vendida, não estava em teu poder? Por que formaste este
desígnio em teu coração? Não mentiste aos homens, mas a Deus. E
Ananias, ouvindo estas palavras, caiu e expirou. E um grande temor
veio sobre todos os que isto ouviram. E, levantando-se os moços,
cobriram o morto e, transportando-o para fora, o sepultaram. E,
passando um espaço quase de três horas, entrou também sua mulher,
não sabendo o que havia acontecido. E disse-lhe Pedro: Dize-me,
vendestes por tanto aquela herdade? E ela disse: Sim, por tanto.
Então Pedro lhe disse: Por que é que entre vós vos concertastes
para tentar o Espírito do Senhor? Eis aí à porta os pés dos que
sepultaram o teu marido, e também te levarão a ti. E logo
caiu aos seus pés, e expirou. E, entrando os moços, acharam-na
morta, e a sepultaram junto de seu marido. E houve um grande
temor em toda a igreja, e em todos os que ouviram estas coisas.

E muitos sinais e prodígios eram feitos entre o povo pelas mãos
dos apóstolos. E estavam todos unanimemente no alpendre de Salomão.
Dos outros, porém, ninguém ousava ajuntar-se a eles; mas o
povo tinha-os em grande estima. E a multidão dos que criam no
Senhor, tanto homens como mulheres, crescia cada vez mais. De
sorte que transportavam os enfermos para as ruas, e os punham em
leitos e em camilhas\footnote{Canapé (espécie de sofá, ger. com a
estrutura de madeira visível) ou encosto para se repousar ou dormir
a sesta.} para que ao menos a sombra de Pedro, quando este passasse,
cobrisse alguns deles. E até das cidades circunvizinhas
concorria muita gente a Jerusalém, conduzindo enfermos e
atormentados de espíritos imundos; os quais eram todos curados.

E, levantando-se o sumo sacerdote, e todos os que estavam com ele
(e eram eles da seita dos saduceus), encheram-se de inveja, e
lançaram mão dos apóstolos, e os puseram na prisão pública.
Mas de noite um anjo do Senhor abriu as portas da prisão e,
tirando-os para fora, disse: Ide e apresentai-vos no templo,
e dizei ao povo todas as palavras desta vida. E, ouvindo eles
isto, entraram de manhã cedo no templo, e ensinavam. Chegando,
porém, o sumo sacerdote e os que estavam com ele, convocaram o
conselho, e a todos os anciãos dos filhos de Israel, e enviaram ao
cárcere, para que de lá os trouxessem. Mas, tendo lá ido os
servidores, não os acharam na prisão e, voltando, lho anunciaram,
dizendo: Achamos realmente o cárcere fechado, com toda a
segurança, e os guardas, que estavam fora, diante das portas; mas,
quando abrimos, ninguém achamos dentro. Então o sumo
sacerdote, o capitão do templo e os chefes dos sacerdotes, ouvindo
estas palavras, estavam perplexos acerca deles e do que viria a ser
aquilo. E, chegando um, anunciou-lhes, dizendo: Eis que os
homens que encerrastes na prisão estão no templo e ensinam ao povo.

Então foi o capitão com os servidores, e os trouxe, não com
violência (porque temiam ser apedrejados pelo povo). E,
trazendo-os, os apresentaram ao conselho. E o sumo sacerdote os
interrogou, dizendo: Não vos admoestamos nós expressamente
que não ensinásseis nesse nome? E eis que enchestes Jerusalém dessa
vossa doutrina, e quereis lançar sobre nós o sangue desse homem.
Porém, respondendo Pedro e os apóstolos, disseram: Mais
importa obedecer a Deus do que aos homens. O Deus de nossos
pais ressuscitou a Jesus, ao qual vós matastes, suspendendo-o no
madeiro. Deus com a sua destra o elevou a Príncipe e
Salvador, para dar a Israel o arrependimento e a remissão dos
pecados. E nós somos testemunhas acerca destas palavras, nós
e também o Espírito Santo, que Deus deu àqueles que lhe obedecem.
E, ouvindo eles isto, se enfureciam, e deliberaram matá-los.
Mas, levantando-se no conselho um certo fariseu, chamado
Gamaliel, doutor da lei, venerado por todo o povo, mandou que por um
pouco levassem para fora os apóstolos; e disse-lhes: Homens
israelitas, acautelai-vos a respeito do que haveis de fazer a estes
homens, porque antes destes dias levantou-se Teudas, dizendo
ser alguém; a este se ajuntou o número de uns quatrocentos homens; o
qual foi morto, e todos os que lhe deram ouvidos foram dispersos e
reduzidos a nada. Depois deste levantou-se Judas, o galileu,
nos dias do alistamento, e levou muito povo após si; mas também este
pereceu, e todos os que lhe deram ouvidos foram dispersos. E
agora digo-vos: Dai de mão a estes homens, e deixai-os, porque, se
este conselho ou esta obra é de homens, se desfará, mas, se é
de Deus, não podereis desfazê-la; para que não aconteça serdes
também achados combatendo contra Deus. E concordaram com ele.
E, chamando os apóstolos, e tendo-os açoitado, mandaram que não
falassem no nome de Jesus, e os deixaram ir. Retiraram-se,
pois, da presença do conselho, regozijando-se de terem sido julgados
dignos de padecer afronta pelo nome de Jesus. E todos os
dias, no templo e nas casas, não cessavam de ensinar, e de anunciar
a Jesus Cristo.

\medskip

\lettrine{6} Ora, naqueles dias, crescendo o número dos
discípulos, houve uma murmuração dos gregos contra os hebreus,
porque as suas viúvas eram desprezadas no ministério cotidiano.
E os doze, convocando a multidão dos discípulos, disseram: Não é
razoável que nós deixemos a palavra de Deus e sirvamos às mesas.
Escolhei, pois, irmãos, dentre vós, sete homens de boa
reputação, cheios do Espírito Santo e de sabedoria, aos quais
constituamos sobre este importante negócio. Mas nós
perseveraremos na oração e no ministério da palavra. E este
parecer contentou a toda a multidão, e elegeram Estêvão, homem cheio
de fé e do Espírito Santo, e Filipe, e Prócoro, e Nicanor, e Timão,
e Parmenas e Nicolau, prosélito de Antioquia; e os apresentaram
ante os apóstolos, e estes, orando, lhes impuseram as mãos. E
crescia a palavra de Deus, e em Jerusalém se multiplicava muito o
número dos discípulos, e grande parte dos sacerdotes obedecia à fé.

E Estêvão, cheio de fé e de poder, fazia prodígios e grandes
sinais entre o povo. E levantaram-se alguns que eram da sinagoga
chamada dos libertinos, e dos cireneus e dos alexandrinos, e dos que
eram da Cilícia e da Ásia, e disputavam com Estêvão. E não
podiam resistir à sabedoria, e ao Espírito com que falava.
Então subornaram uns homens, para que dissessem: Ouvimos-lhe
proferir palavras blasfemas contra Moisés e contra Deus. E
excitaram o povo, os anciãos e os escribas; e, investindo contra
ele, o arrebataram e o levaram ao conselho. E apresentaram
falsas testemunhas, que diziam: Este homem não cessa de proferir
palavras blasfemas contra este santo lugar e a lei; porque
nós lhe ouvimos dizer que esse Jesus Nazareno há de destruir este
lugar e mudar os costumes que Moisés nos deu. Então todos os
que estavam assentados no conselho, fixando os olhos nele, viram o
seu rosto como o rosto de um anjo.

\medskip

\lettrine{7} E disse o sumo sacerdote: Porventura é isto
assim? E ele disse: Homens, irmãos, e pais, ouvi. O Deus da
glória apareceu a nosso pai Abraão, estando na Mesopotâmia, antes de
habitar em Harã, e disse-lhe: Sai da tua terra e dentre a tua
parentela, e dirige-te à terra que eu te mostrar. Então saiu da
terra dos caldeus, e habitou em Harã. E dali, depois que seu pai
faleceu, Deus o trouxe para esta terra em que habitais agora. E
não lhe deu nela herança, nem ainda o espaço de um pé; mas prometeu
que lhe daria a posse dela, e depois dele, à sua descendência, não
tendo ele ainda filho. E falou Deus assim: Que a sua
descendência seria peregrina em terra alheia, e a sujeitariam à
escravidão, e a maltratariam por quatrocentos anos. E eu
julgarei a nação que os tiver escravizado, disse Deus. E depois
disto sairão e me servirão neste lugar. E deu-lhe a aliança da
circuncisão; e assim gerou a Isaque, e o circuncidou ao oitavo dia;
e Isaque a Jacó; e Jacó aos doze patriarcas. E os patriarcas,
movidos de inveja, venderam José para o Egito; mas Deus era com ele.
E livrou-o de todas as suas tribulações, e lhe deu graça e
sabedoria ante Faraó, rei do Egito, que o constituiu governador
sobre o Egito e toda a sua casa. Sobreveio então a todo o
país do Egito e de Canaã fome e grande tribulação; e nossos pais não
achavam alimentos. Mas tendo ouvido Jacó que no Egito havia
trigo, enviou ali nossos pais, a primeira vez. E na segunda
vez foi José conhecido por seus irmãos, e a sua linhagem foi
manifesta a Faraó. E José mandou chamar a seu pai Jacó, e a
toda a sua parentela, que era de setenta e cinco almas. E
Jacó desceu ao Egito, e morreu, ele e nossos pais; e foram
transportados para Siquém, e depositados na sepultura que Abraão
comprara por certa soma de dinheiro aos filhos de Emor, pai de
Siquém.

Aproximando-se, porém, o tempo da promessa que Deus tinha feito a
Abraão, o povo cresceu e se multiplicou no Egito; até que se
levantou outro rei, que não conhecia a José. Esse, usando de
astúcia contra a nossa linhagem, maltratou nossos pais, ao ponto de
os fazer enjeitar as suas crianças, para que não se multiplicassem.
Nesse tempo nasceu Moisés, e era mui formoso, e foi criado
três meses em casa de seu pai. E, sendo enjeitado, tomou-o a
filha de Faraó, e o criou como seu filho. E Moisés foi
instruído em toda a ciência dos egípcios; e era poderoso em suas
palavras e obras. E, quando completou a idade de quarenta
anos, veio-lhe ao coração ir visitar seus irmãos, os filhos de
Israel. E, vendo maltratado um deles, o defendeu, e vingou o
ofendido, matando o egípcio. E ele cuidava que seus irmãos
entenderiam que Deus lhes havia de dar a liberdade pela sua mão; mas
eles não entenderam. E no dia seguinte, pelejando eles, foi
por eles visto, e quis levá-los à paz, dizendo: Homens, sois irmãos;
por que vos agravais um ao outro? E o que ofendia o seu
próximo o repeliu, dizendo: Quem te constituiu príncipe e juiz sobre
nós? Queres tu matar-me, como ontem mataste o egípcio?
E a esta palavra fugiu Moisés, e esteve como estrangeiro na
terra de Midiã, onde gerou dois filhos.

E, completados quarenta anos, apareceu-lhe o anjo do Senhor no
deserto do monte Sinai, numa chama de fogo no meio de uma sarça.
Então Moisés, quando viu isto, se maravilhou da visão; e,
aproximando-se para observar, foi-lhe dirigida a voz do Senhor,
dizendo: Eu sou o Deus de teus pais, o Deus de Abraão, e o
Deus de Isaque, e o Deus de Jacó. E Moisés, todo trêmulo, não ousava
olhar. E disse-lhe o Senhor: Tira as alparcas dos teus pés,
porque o lugar em que estás é terra santa. Tenho visto
atentamente a aflição do meu povo que está no Egito, e ouvi os seus
gemidos, e desci a livrá-los. Agora, pois, vem, e enviar-te-ei ao
Egito. A este Moisés, ao qual haviam negado, dizendo: Quem te
constituiu príncipe e juiz? a este enviou Deus como príncipe e
libertador, pela mão do anjo que lhe aparecera na sarça. Foi
este que os conduziu para fora, fazendo prodígios e sinais na terra
do Egito, e no Mar Vermelho, e no deserto, por quarenta anos.
Este é aquele Moisés que disse aos filhos de Israel: O Senhor
vosso Deus vos levantará dentre vossos irmãos um profeta como eu; a
ele ouvireis. Este é o que esteve entre a congregação no
deserto, com o anjo que lhe falava no monte Sinai, e com nossos
pais, o qual recebeu as palavras de vida para no-las dar. Ao
qual nossos pais não quiseram obedecer, antes o rejeitaram e em seu
coração se tornaram ao Egito, dizendo a Arão: Faze-nos deuses
que vão adiante de nós; porque a esse Moisés, que nos tirou da terra
do Egito, não sabemos o que lhe aconteceu. E naqueles dias
fizeram o bezerro, e ofereceram sacrifícios ao ídolo, e se alegraram
nas obras das suas mãos.

Mas Deus se afastou, e os abandonou a que servissem ao exército
do céu, como está escrito no livro dos profetas: Porventura me
oferecestes vítimas e sacrifícios no deserto por quarenta anos, ó
casa de Israel? Antes tomastes o tabernáculo de Moloque, e a
estrela do vosso deus Renfã, figuras que vós fizestes para as
adorar. Transportar-vos-ei, pois, para além da Babilônia.
Estava entre nossos pais no deserto o tabernáculo do
testemunho, como ordenara aquele que disse a Moisés que o fizesse
segundo o modelo que tinha visto. O qual, nossos pais,
recebendo-o também, o levaram com Josué quando entraram na posse das
nações que Deus lançou para fora da presença de nossos pais, até aos
dias de Davi, que achou graça diante de Deus, e pediu que
pudesse achar tabernáculo para o Deus de Jacó. E Salomão lhe
edificou casa; mas o Altíssimo não habita em templos feitos
por mãos de homens, como diz o profeta: O céu é o meu trono,
e a terra o estrado dos meus pés. Que casa me edificareis? diz o
Senhor, Ou qual é o lugar do meu repouso? Porventura não fez
a minha mão todas estas coisas?

Homens de dura cerviz, e incircuncisos de coração e ouvido, vós
sempre resistis ao Espírito Santo; assim vós sois como vossos pais.
A qual dos profetas não perseguiram vossos pais? Até mataram
os que anteriormente anunciaram a vinda do Justo, do qual vós agora
fostes traidores e homicidas; vós, que recebestes a lei por
ordenação dos anjos, e não a guardastes.

E, ouvindo eles isto, enfureciam-se em seus corações, e rangiam
os dentes contra ele. Mas ele, estando cheio do Espírito
Santo, fixando os olhos no céu, viu a glória de Deus, e Jesus, que
estava à direita de Deus; e disse: Eis que vejo os céus
abertos, e o Filho do homem, que está em pé à mão direita de Deus.
Mas eles gritaram com grande voz, taparam os seus ouvidos, e
arremeteram unânimes contra ele. E, expulsando-o da cidade, o
apedrejavam. E as testemunhas depuseram as suas capas aos pés de um
jovem chamado Saulo. E apedrejaram a Estêvão que em invocação
dizia: Senhor Jesus, recebe o meu espírito. E, pondo-se de
joelhos, clamou com grande voz: Senhor, não lhes imputes este
pecado. E, tendo dito isto, adormeceu.

\medskip

\lettrine{8} E também Saulo consentiu na morte dele. E fez-se
naquele dia uma grande perseguição contra a igreja que estava em
Jerusalém; e todos foram dispersos pelas terras da Judéia e de
Samaria, exceto os apóstolos. E uns homens piedosos foram
enterrar Estêvão, e fizeram sobre ele grande pranto. E Saulo
assolava a igreja, entrando pelas casas; e, arrastando homens e
mulheres, os encerrava na prisão.

Mas os que andavam dispersos iam por toda a parte, anunciando a
palavra. E, descendo Filipe à cidade de Samaria lhes pregava a
Cristo. E as multidões unanimemente prestavam atenção ao que
Filipe dizia, porque ouviam e viam os sinais que ele fazia; pois
que os espíritos imundos saíam de muitos que os tinham, clamando em
alta voz; e muitos paralíticos e coxos eram curados. E havia
grande alegria naquela cidade. E estava ali um certo homem,
chamado Simão, que anteriormente exercera naquela cidade a arte
mágica, e tinha iludido o povo de Samaria, dizendo que era uma
grande personagem; ao qual todos atendiam, desde o menor até
ao maior, dizendo: Este é a grande virtude de Deus. E
atendiam-no, porque já desde muito tempo os havia iludido com artes
mágicas. Mas, como cressem em Filipe, que lhes pregava acerca
do reino de Deus, e do nome de Jesus Cristo, se batizavam, tanto
homens como mulheres. E creu até o próprio Simão; e, sendo
batizado, ficou de contínuo com Filipe; e, vendo os sinais e as
grandes maravilhas que se faziam, estava atônito.

Os apóstolos, pois, que estavam em Jerusalém, ouvindo que Samaria
recebera a palavra de Deus, enviaram para lá Pedro e João. Os
quais, tendo descido, oraram por eles para que recebessem o Espírito
Santo

sobre nenhum deles tinha ainda descido; mas somente eram
batizados em nome do Senhor Jesus). Então lhes impuseram as
mãos, e receberam o Espírito Santo. E Simão, vendo que pela
imposição das mãos dos apóstolos era dado o Espírito Santo, lhes
ofereceu dinheiro, dizendo: Dai-me também a mim esse poder,
para que aquele sobre quem eu puser as mãos receba o Espírito Santo.
Mas disse-lhe Pedro: O teu dinheiro seja contigo para
perdição, pois cuidaste que o dom de Deus se alcança por dinheiro.
Tu não tens parte nem sorte nesta palavra, porque o teu
coração não é reto diante de Deus. Arrepende-te, pois, dessa
tua iniqüidade, e ora a Deus, para que porventura te seja perdoado o
pensamento do teu coração; pois vejo que estás em fel de
amargura, e em laço de iniqüidade. Respondendo, porém, Simão,
disse: Orai vós por mim ao Senhor, para que nada do que dissestes
venha sobre mim. Tendo eles, pois, testificado e falado a
palavra do Senhor, voltaram para Jerusalém e em muitas aldeias dos
samaritanos anunciaram o evangelho.

E o anjo do Senhor falou a Filipe, dizendo: Levanta-te, e vai
para o lado do sul, ao caminho que desce de Jerusalém para Gaza, que
está deserta. E levantou-se, e foi; e eis que um homem
etíope, eunuco, mordomo-mor de Candace, rainha dos etíopes, o qual
era superintendente de todos os seus tesouros, e tinha ido a
Jerusalém para adoração, regressava e, assentado no seu
carro, lia o profeta Isaías. E disse o Espírito a Filipe:
Chega-te, e ajunta-te a esse carro. E, correndo Filipe, ouviu
que lia o profeta Isaías, e disse: Entendes tu o que lês? E
ele disse: Como poderei entender, se alguém não me ensinar? E rogou
a Filipe que subisse e com ele se assentasse. E o lugar da
Escritura que lia era este: Foi levado como a ovelha para o
matadouro; e, como está mudo o cordeiro diante do que o tosquia,
assim não abriu a sua boca. Na sua humilhação foi tirado o
seu julgamento; e quem contará a sua geração? Porque a sua vida é
tirada da terra. E, respondendo o eunuco a Filipe, disse:
Rogo-te, de quem diz isto o profeta? De si mesmo, ou de algum outro?
Então Filipe, abrindo a sua boca, e começando nesta
Escritura, lhe anunciou a Jesus. E, indo eles caminhando,
chegaram ao pé de alguma água, e disse o eunuco: Eis aqui água; que
impede que eu seja batizado? E disse Filipe: É lícito, se
crês de todo o coração. E, respondendo ele, disse: Creio que Jesus
Cristo é o Filho de Deus. E mandou parar o carro, e desceram
ambos à água, tanto Filipe como o eunuco, e o batizou. E,
quando saíram da água, o Espírito do Senhor arrebatou a Filipe, e
não o viu mais o eunuco; e, jubiloso, continuou o seu caminho.
E Filipe se achou em Azoto e, indo passando, anunciava o
evangelho em todas as cidades, até que chegou a Cesaréia.

\medskip

\lettrine{9} E Saulo, respirando ainda ameaças e mortes contra
os discípulos do Senhor, dirigiu-se ao sumo sacerdote. E
pediu-lhe cartas para Damasco, para as sinagogas, a fim de que, se
encontrasse alguns daquela seita, quer homens quer mulheres, os
conduzisse presos a Jerusalém. E, indo no caminho, aconteceu
que, chegando perto de Damasco, subitamente o cercou um resplendor
de luz do céu. E, caindo em terra, ouviu uma voz que lhe dizia:
Saulo, Saulo, por que me persegues? E ele disse: Quem és,
Senhor? E disse o Senhor: Eu sou Jesus, a quem tu persegues. Duro é
para ti recalcitrar contra os aguilhões. E ele, tremendo e
atônito, disse: Senhor, que queres que eu faça? E disse-lhe o
Senhor: Levanta-te, e entra na cidade, e lá te será dito o que te
convém fazer. E os homens, que iam com ele, pararam espantados,
ouvindo a voz, mas não vendo ninguém. E Saulo levantou-se da
terra, e, abrindo os olhos, não via a ninguém. E, guiando-o pela
mão, o conduziram a Damasco. E esteve três dias sem ver, e não
comeu nem bebeu.

E havia em Damasco um certo discípulo chamado Ananias; e
disse-lhe o Senhor em visão: Ananias! E ele respondeu: Eis-me aqui,
Senhor. E disse-lhe o Senhor: Levanta-te, e vai à rua chamada
Direita, e pergunta em casa de Judas por um homem de Tarso chamado
Saulo; pois eis que ele está orando; e numa visão ele viu que
entrava um homem chamado Ananias, e punha sobre ele a mão, para que
tornasse a ver. E respondeu Ananias: Senhor, a muitos ouvi
acerca deste homem, quantos males tem feito aos teus santos em
Jerusalém; e aqui tem poder dos principais dos sacerdotes
para prender a todos os que invocam o teu nome. Disse-lhe,
porém, o Senhor: Vai, porque este é para mim um vaso escolhido, para
levar o meu nome diante dos gentios, e dos reis e dos filhos de
Israel. E eu lhe mostrarei quanto deve padecer pelo meu nome.
E Ananias foi, e entrou na casa e, impondo-lhe as mãos,
disse: Irmão Saulo, o SENHOR Jesus, que te apareceu no caminho por
onde vinhas, me enviou, para que tornes a ver e sejas cheio do
Espírito Santo. E logo lhe caíram dos olhos como que umas
escamas, e recuperou a vista; e, levantando-se, foi batizado.
E, tendo comido, ficou confortado. E esteve Saulo alguns dias
com os discípulos que estavam em Damasco. E logo nas
sinagogas pregava a Cristo, que este é o Filho de Deus. E
todos os que o ouviam estavam atônitos, e diziam: Não é este o que
em Jerusalém perseguia os que invocavam este nome, e para isso veio
aqui, para os levar presos aos principais dos sacerdotes?
Saulo, porém, se esforçava muito mais, e confundia os judeus
que habitavam em Damasco, provando que aquele era o Cristo.

E, tendo passado muitos dias, os judeus tomaram conselho entre si
para o matar. Mas as suas ciladas vieram ao conhecimento de
Saulo; e como eles guardavam as portas, tanto de dia como de noite,
para poderem tirar-lhe a vida, tomando-o de noite os
discípulos o desceram, dentro de um cesto, pelo muro. E,
quando Saulo chegou a Jerusalém, procurava ajuntar-se aos
discípulos, mas todos o temiam, não crendo que fosse discípulo.
Então Barnabé, tomando-o consigo, o trouxe aos apóstolos, e
lhes contou como no caminho ele vira ao Senhor e lhe falara, e como
em Damasco falara ousadamente no nome de Jesus. E andava com
eles em Jerusalém, entrando e saindo, e falava ousadamente no
nome do Senhor Jesus. Falava e disputava também contra os gregos,
mas eles procuravam matá-lo. Sabendo-o, porém, os irmãos, o
acompanharam até Cesaréia, e o enviaram a Tarso. Assim, pois,
as igrejas em toda a Judéia, e Galiléia e Samaria tinham paz, e eram
edificadas; e se multiplicavam, andando no temor do Senhor e
consolação do Espírito Santo.

E aconteceu que, passando Pedro por toda a parte, veio também aos
santos que habitavam em Lida. E achou ali certo homem,
chamado Enéias, jazendo numa cama havia oito anos, o qual era
paralítico. E disse-lhe Pedro: Enéias, Jesus Cristo te dá
saúde; levanta-te e faze a tua cama. E logo se levantou. E
viram-no todos os que habitavam em Lida e Sarona, os quais se
converteram ao Senhor.

E havia em Jope uma discípula chamada Tabita, que traduzido se
diz Dorcas. Esta estava cheia de boas obras e esmolas que fazia.
E aconteceu naqueles dias que, enfermando ela, morreu; e,
tendo-a lavado, a depositaram num quarto alto. E, como Lida
era perto de Jope, ouvindo os discípulos que Pedro estava ali, lhe
mandaram dois homens, rogando-lhe que não se demorasse em vir ter
com eles. E, levantando-se Pedro, foi com eles; e quando
chegou o levaram ao quarto alto, e todas as viúvas o rodearam,
chorando e mostrando as túnicas e roupas que Dorcas fizera quando
estava com elas. Mas Pedro, fazendo sair a todos, pôs-se de
joelhos e orou; e, voltando-se para o corpo, disse: Tabita,
levanta-te. E ela abriu os olhos, e, vendo a Pedro, assentou-se.
E ele, dando-lhe a mão, a levantou e, chamando os santos e as
viúvas, apresentou-lha viva. E foi isto notório por toda a
Jope, e muitos creram no Senhor. E ficou muitos dias em Jope,
com um certo Simão curtidor.

\medskip

\lettrine{10} E havia em Cesaréia um homem por nome Cornélio,
centurião da coorte chamada italiana, piedoso e temente a Deus,
com toda a sua casa, o qual fazia muitas esmolas ao povo, e de
contínuo orava a Deus. Este, quase à hora nona do dia, viu
claramente numa visão um anjo de Deus, que se dirigia para ele e
dizia: Cornélio. O qual, fixando os olhos nele, e muito
atemorizado, disse: Que é, Senhor? E disse-lhe: As tuas orações e as
tuas esmolas têm subido para memória diante de Deus; agora,
pois, envia homens a Jope, e manda chamar a Simão, que tem por
sobrenome Pedro. Este está com um certo Simão curtidor, que tem
a sua casa junto do mar. Ele te dirá o que deves fazer. E,
retirando-se o anjo que lhe falava, chamou dois dos seus criados, e
a um piedoso soldado dos que estavam ao seu serviço. E,
havendo-lhes contado tudo, os enviou a Jope.

E no dia seguinte, indo eles seu caminho, e estando já perto da
cidade, subiu Pedro ao terraço para orar, quase à hora sexta.
E tendo fome, quis comer; e, enquanto lho preparavam,
sobreveio-lhe um arrebatamento de sentidos, e viu o céu
aberto, e que descia um vaso, como se fosse um grande lençol atado
pelas quatro pontas, e vindo para a terra. No qual havia de
todos os animais quadrúpedes e répteis da terra, e aves do céu.
E foi-lhe dirigida uma voz: Levanta-te, Pedro, mata e come.
Mas Pedro disse: De modo nenhum, Senhor, porque nunca comi
coisa alguma comum e imunda. E segunda vez lhe disse a voz:
Não faças tu comum ao que Deus purificou. E aconteceu isto
por três vezes; e o vaso tornou a recolher-se ao céu. E
estando Pedro duvidando entre si acerca do que seria aquela visão
que tinha visto, eis que os homens que foram enviados por Cornélio
pararam à porta, perguntando pela casa de Simão. E, chamando,
perguntaram se Simão, que tinha por sobrenome Pedro, morava ali.

E, pensando Pedro naquela visão, disse-lhe o Espírito: Eis que
três homens te buscam. Levanta-te pois, desce, e vai com
eles, não duvidando; porque eu os enviei. E, descendo Pedro
para junto dos homens que lhe foram enviados por Cornélio, disse:
Sou eu a quem procurais; qual é a causa por que estais aqui?
E eles disseram: Cornélio, o centurião, homem justo e temente
a Deus, e que tem bom testemunho de toda a nação dos judeus, foi
avisado por um santo anjo para que te chamasse a sua casa, e ouvisse
as tuas palavras. Então, chamando-os para dentro, os recebeu
em casa. E no dia seguinte foi Pedro com eles, e foram com ele
alguns irmãos de Jope. E no dia imediato chegaram a Cesaréia.
E Cornélio os estava esperando, tendo já convidado os seus parentes
e amigos mais íntimos. E aconteceu que, entrando Pedro, saiu
Cornélio a recebê-lo, e, prostrando-se a seus pés o adorou.
Mas Pedro o levantou, dizendo: Levanta-te, que eu também sou
homem. E, falando com ele, entrou, e achou muitos que ali se
haviam ajuntado. E disse-lhes: Vós bem sabeis que não é
lícito a um homem judeu ajuntar-se ou chegar-se a estrangeiros; mas
Deus mostrou-me que a nenhum homem chame comum ou imundo. Por
isso, sendo chamado, vim sem contradizer. Pergunto, pois, por que
razão mandastes chamar-me? E disse Cornélio: Há quatro dias
estava eu em jejum até esta hora, orando em minha casa à hora nona.
E eis que diante de mim se apresentou um homem com vestes
resplandecentes, e disse: Cornélio, a tua oração foi ouvida, e as
tuas esmolas estão em memória diante de Deus. Envia, pois, a
Jope, e manda chamar Simão, o que tem por sobrenome Pedro; este está
em casa de Simão o curtidor, junto do mar, e ele, vindo, te falará.
E logo mandei chamar-te, e bem fizeste em vir. Agora, pois,
estamos todos presentes diante de Deus, para ouvir tudo quanto por
Deus te é mandado.

E, abrindo Pedro a boca, disse: Reconheço por verdade que Deus
não faz acepção de pessoas; mas que lhe é agradável aquele
que, em qualquer nação, o teme e faz o que é justo. A palavra
que ele enviou aos filhos de Israel, anunciando a paz por Jesus
Cristo (este é o Senhor de todos); esta palavra, vós bem
sabeis, veio por toda a Judéia, começando pela Galiléia, depois do
batismo que João pregou; como Deus ungiu a Jesus de Nazaré
com o Espírito Santo e com virtude; o qual andou fazendo bem, e
curando a todos os oprimidos do diabo, porque Deus era com ele.
E nós somos testemunhas de todas as coisas que fez, tanto na
terra da Judéia como em Jerusalém; ao qual mataram, pendurando-o num
madeiro. A este ressuscitou Deus ao terceiro dia, e fez que
se manifestasse, não a todo o povo, mas às testemunhas que
Deus antes ordenara; a nós, que comemos e bebemos juntamente com
ele, depois que ressuscitou dentre os mortos. E nos mandou
pregar ao povo, e testificar que ele é o que por Deus foi
constituído juiz dos vivos e dos mortos. A este dão
testemunho todos os profetas, de que todos os que nele crêem
receberão o perdão dos pecados pelo seu nome.

E, dizendo Pedro ainda estas palavras, caiu o Espírito Santo
sobre todos os que ouviam a palavra. E os fiéis que eram da
circuncisão, todos quantos tinham vindo com Pedro, maravilharam-se
de que o dom do Espírito Santo se derramasse também sobre os
gentios. Porque os ouviam falar línguas, e magnificar a Deus.
Respondeu, então, Pedro: Pode alguém porventura recusar a
água, para que não sejam batizados estes, que também receberam como
nós o Espírito Santo? E mandou que fossem batizados em nome
do Senhor. Então rogaram-lhe que ficasse com eles por alguns dias.

\medskip

\lettrine{11} E ouviram os apóstolos, e os irmãos que estavam
na Judéia, que também os gentios tinham recebido a palavra de Deus.
E, subindo Pedro a Jerusalém, disputavam com ele os que eram da
circuncisão, dizendo: Entraste em casa de homens incircuncisos,
e comeste com eles. Mas Pedro começou a fazer-lhes uma exposição
por ordem, dizendo: Estando eu orando na cidade de Jope, tive,
num arrebatamento dos sentidos, uma visão; via um vaso, como um
grande lençol que descia do céu e vinha até junto de mim. E,
pondo nele os olhos, considerei, e vi animais da terra, quadrúpedes,
e feras, e répteis e aves do céu. E ouvi uma voz que me dizia:
Levanta-te, Pedro; mata e come. Mas eu disse: De maneira
nenhuma, Senhor; pois, nunca em minha boca entrou coisa alguma comum
ou imunda. Mas a voz respondeu-me do céu segunda vez: Não chames
tu comum ao que Deus purificou. E sucedeu isto por três
vezes; e tudo tornou a recolher-se ao céu. E eis que, na
mesma hora, pararam, junto da casa em que eu estava, três homens que
me foram enviados de Cesaréia. E disse-me o Espírito que
fosse com eles, nada duvidando; e também estes seis irmãos foram
comigo, e entramos em casa daquele homem; e contou-nos como
vira em pé um anjo em sua casa, e lhe dissera: Envia homens a Jope,
e manda chamar a Simão, que tem por sobrenome Pedro, o qual
te dirá palavras com que te salves, tu e toda a tua casa. E,
quando comecei a falar, caiu sobre eles o Espírito Santo, como
também sobre nós ao princípio. E lembrei-me do dito do
Senhor, quando disse: João certamente batizou com água; mas vós
sereis batizados com o Espírito Santo. Portanto, se Deus lhes
deu o mesmo dom que a nós, quando havemos crido no Senhor Jesus
Cristo, quem era então eu, para que pudesse resistir a Deus?
E, ouvindo estas coisas, apaziguaram-se, e glorificaram a
Deus, dizendo: Na verdade até aos gentios deu Deus o arrependimento
para a vida.

E os que foram dispersos pela perseguição que sucedeu por causa
de Estêvão caminharam até à Fenícia, Chipre e Antioquia, não
anunciando a ninguém a palavra, senão somente aos judeus. E
havia entre eles alguns homens chíprios e cirenenses, os quais
entrando em Antioquia falaram aos gregos, anunciando o Senhor Jesus.
E a mão do Senhor era com eles; e grande número creu e se
converteu ao Senhor. E chegou a fama destas coisas aos
ouvidos da igreja que estava em Jerusalém; e enviaram Barnabé a
Antioquia. O qual, quando chegou, e viu a graça de Deus, se
alegrou, e exortou a todos a que permanecessem no Senhor, com
propósito de coração; porque era homem de bem e cheio do
Espírito Santo e de fé. E muita gente se uniu ao Senhor. E
partiu Barnabé para Tarso, a buscar Saulo; e, achando-o, o conduziu
para Antioquia. E sucedeu que todo um ano se reuniram naquela
igreja, e ensinaram muita gente; e em Antioquia foram os discípulos,
pela primeira vez, chamados cristãos.

E naqueles dias desceram profetas de Jerusalém para Antioquia.
E, levantando-se um deles, por nome Ágabo, dava a entender
pelo Espírito, que haveria uma grande fome em todo o mundo, e isso
aconteceu no tempo de Cláudio César. E os discípulos
determinaram mandar, cada um conforme o que pudesse, socorro aos
irmãos que habitavam na Judéia. O que eles com efeito
fizeram, enviando-o aos anciãos por mão de Barnabé e de Saulo.

\medskip

\lettrine{12} E por aquele mesmo tempo o rei Herodes estendeu
as mãos sobre alguns da igreja, para os maltratar; e matou à
espada Tiago, irmão de João. E, vendo que isso agradara aos
judeus, continuou, mandando prender também a Pedro. E eram os dias
dos ázimos. E, havendo-o prendido, o encerrou na prisão,
entregando-o a quatro quaternos\footnote{Do lat. quaternu, `de
quatro em quatro'. Adj.: Composto de quatro coisas, modos,
elementos, etc.} de soldados, para que o guardassem, querendo
apresentá-lo ao povo depois da páscoa\footnote{KJ:And when he had
apprehended him, he put him in prison, and delivered him to four
quaternions of soldiers to keep him; intending after \textbf{Easter}
to bring him forth to the people. ``Easter'' --- páscoa pagã.}.

Pedro, pois, era guardado na prisão; mas a igreja fazia contínua
oração por ele a Deus. E quando Herodes estava para o fazer
comparecer, nessa mesma noite estava Pedro dormindo entre dois
soldados, ligado com duas cadeias, e os guardas diante da porta
guardavam a prisão. E eis que sobreveio o anjo do Senhor, e
resplandeceu uma luz na prisão; e, tocando a Pedro na
ilharga\footnote{Cada uma das partes laterais e inferiores do
baixo-ventre; flanco.}, o despertou, dizendo: Levanta-te depressa. E
caíram-lhe das mãos as cadeias. E disse-lhe o anjo: Cinge-te, e
ata as tuas alparcas. E ele assim o fez. Disse-lhe mais: Lança às
costas a tua capa, e segue-me. E, saindo, o seguia. E não sabia
que era real o que estava sendo feito pelo anjo, mas cuidava que via
alguma visão. E, quando passaram a primeira e segunda guarda,
chegaram à porta de ferro, que dá para a cidade, a qual se lhes
abriu por si mesma; e, tendo saído, percorreram uma rua, e logo o
anjo se apartou dele. E Pedro, tornando a si, disse: Agora
sei verdadeiramente que o Senhor enviou o seu anjo, e me livrou da
mão de Herodes, e de tudo o que o povo dos judeus esperava.
E, considerando ele nisto, foi à casa de Maria, mãe de João,
que tinha por sobrenome Marcos, onde muitos estavam reunidos e
oravam. E, batendo Pedro à porta do pátio, uma menina chamada
Rode saiu a escutar; e, conhecendo a voz de Pedro, de gozo
não abriu a porta, mas, correndo para dentro, anunciou que Pedro
estava à porta. E disseram-lhe: Estás fora de ti. Mas ela
afirmava que assim era. E diziam: É o seu anjo. Mas Pedro
perseverava em bater e, quando abriram, viram-no, e se espantaram.
E acenando-lhes ele com a mão para que se calassem,
contou-lhes como o Senhor o tirara da prisão, e disse: Anunciai isto
a Tiago e aos irmãos. E, saindo, partiu para outro lugar. E,
sendo já dia, houve não pouco alvoroço entre os soldados sobre o que
seria feito de Pedro. E, quando Herodes o procurou e o não
achou, feita inquirição aos guardas, mandou-os justiçar. E, partindo
da Judéia para Cesaréia, ficou ali.

E ele estava irritado com os de Tiro e de Sidom; mas estes, vindo
de comum acordo ter com ele, e obtendo a amizade de Blasto, que era
o camarista\footnote{Fidalgo ou fidalga a serviço de pessoas reais;
camareiro.} do rei, pediam paz; porquanto o seu país se abastecia do
país do rei. E num dia designado, vestindo Herodes as vestes
reais, estava assentado no tribunal e lhes fez uma
prática\footnote{Discurso rápido; conversação; conferência.}.
E o povo exclamava: Voz de Deus, e não de homem. E no
mesmo instante feriu-o o anjo do Senhor, porque não deu glória a
Deus e, comido de bichos, expirou. E a palavra de Deus
crescia e se multiplicava. E Barnabé e Saulo, havendo
terminado aquele serviço, voltaram de Jerusalém, levando também
consigo a João, que tinha por sobrenome Marcos.

\medskip

\lettrine{13} E na igreja que estava em Antioquia havia alguns
profetas e doutores, a saber: Barnabé e Simeão chamado Níger, e
Lúcio, cireneu, e Manaém, que fora criado com Herodes o tetrarca, e
Saulo. E, servindo eles ao Senhor, e jejuando, disse o Espírito
Santo: Apartai-me a Barnabé e a Saulo para a obra a que os tenho
chamado. Então, jejuando e orando, e pondo sobre eles as mãos,
os despediram.

E assim estes, enviados pelo Espírito Santo, desceram a Selêucia e
dali navegaram para Chipre. E, chegados a Salamina, anunciavam a
palavra de Deus nas sinagogas dos judeus; e tinham também a João
como cooperador. E, havendo atravessado a ilha até Pafos,
acharam um certo judeu mágico, falso profeta, chamado Barjesus,
o qual estava com o procônsul Sérgio Paulo, homem prudente.
Este, chamando a si Barnabé e Saulo, procurava muito ouvir a palavra
de Deus. Mas resistia-lhes Elimas, o encantador (porque assim se
interpreta o seu nome), procurando apartar da fé o procônsul.
Todavia Saulo, que também se chama Paulo, cheio do Espírito
Santo, e fixando os olhos nele, disse: Ó filho do diabo,
cheio de todo o engano e de toda a malícia, inimigo de toda a
justiça, não cessarás de perturbar os retos caminhos do Senhor?
Eis aí, pois, agora contra ti a mão do Senhor, e ficarás
cego, sem ver o sol por algum tempo. E no mesmo instante a escuridão
e as trevas caíram sobre ele e, andando à roda, buscava a quem o
guiasse pela mão. Então o procônsul, vendo o que havia
acontecido, creu, maravilhado da doutrina do Senhor. E,
partindo de Pafos, Paulo e os que estavam com ele chegaram a Perge,
da Panfília. Mas João, apartando-se deles, voltou para Jerusalém.

E eles, saindo de Perge, chegaram a Antioquia, da Pisídia, e,
entrando na sinagoga, num dia de sábado, assentaram-se; e,
depois da lição da lei e dos profetas, lhes mandaram dizer os
principais da sinagoga: Homens irmãos, se tendes alguma palavra de
consolação para o povo, falai. E, levantando-se Paulo, e
pedindo silêncio com a mão, disse: Homens israelitas, e os que
temeis a Deus, ouvi: O Deus deste povo de Israel escolheu a
nossos pais, e exaltou o povo, sendo eles estrangeiros na terra do
Egito; e com braço poderoso os tirou dela; e suportou os seus
costumes no deserto por espaço de quase quarenta anos. E,
destruindo a sete nações na terra de Canaã, deu-lhes por sorte a
terra deles. E, depois disto, por quase quatrocentos e
cinqüenta anos, lhes deu juízes, até ao profeta Samuel. E
depois pediram um rei, e Deus lhes deu por quarenta anos, a Saul
filho de Cis, homem da tribo de Benjamim. E, quando este foi
retirado, levantou-lhes como rei a Davi, ao qual também deu
testemunho, e disse: Achei a Davi, filho de Jessé, homem conforme o
meu coração, que executará toda a minha vontade. Da
descendência deste, conforme a promessa, levantou Deus a Jesus para
Salvador de Israel; tendo primeiramente João, antes da vinda
dele, pregado a todo o povo de Israel o batismo do arrependimento.
Mas João, quando completava a carreira, disse: Quem pensais
vós que eu sou? Eu não sou o Cristo; mas eis que após mim vem aquele
a quem não sou digno de desatar as alparcas dos pés. Homens
irmãos, filhos da geração de Abraão, e os que dentre vós temem a
Deus, a vós vos é enviada a palavra desta salvação. Por não
terem conhecido a este, os que habitavam em Jerusalém, e os seus
príncipes, condenaram-no, cumprindo assim as vozes dos profetas que
se lêem todos os sábados. E, embora não achassem alguma causa
de morte, pediram a Pilatos que ele fosse morto. E, havendo
eles cumprido todas as coisas que dele estavam escritas, tirando-o
do madeiro, o puseram na sepultura; mas Deus o ressuscitou
dentre os mortos. E ele por muitos dias foi visto pelos que
subiram com ele da Galiléia a Jerusalém, e são suas testemunhas para
com o povo. E nós vos anunciamos que a promessa que foi feita
aos pais, Deus a cumpriu a nós, seus filhos, ressuscitando a Jesus;
como também está escrito no salmo segundo: Meu filho és tu,
hoje te gerei. E que o ressuscitaria dentre os mortos, para
nunca mais tornar à corrupção, disse-o assim: As santas e fiéis
bênçãos de Davi vos darei. Por isso também em outro salmo
diz: Não permitirás que o teu santo veja corrupção. Porque,
na verdade, tendo Davi no seu tempo servido conforme a vontade de
Deus, dormiu, foi posto junto de seus pais e viu a corrupção.
Mas aquele a quem Deus ressuscitou nenhuma corrupção viu.
Seja-vos, pois, notório, homens irmãos, que por este se vos
anuncia a remissão dos pecados. E de tudo o que, pela lei de
Moisés, não pudestes ser justificados, por ele é justificado todo
aquele que crê. Vede, pois, que não venha sobre vós o que
está dito nos profetas: Vede, ó desprezadores, e espantai-vos
e desaparecei; porque opero uma obra em vossos dias, obra tal que
não crereis, se alguém vo-la contar.

E, saídos os judeus da sinagoga, os gentios rogaram que no sábado
seguinte lhes fossem ditas as mesmas coisas. E, despedida a
sinagoga, muitos dos judeus e dos prosélitos religiosos seguiram
Paulo e Barnabé; os quais, falando-lhes, os exortavam a que
permanecessem na graça de Deus. E no sábado seguinte
ajuntou-se quase toda a cidade para ouvir a palavra de Deus.
Então os judeus, vendo a multidão, encheram-se de inveja e,
blasfemando, contradiziam o que Paulo falava. Mas Paulo e
Barnabé, usando de ousadia, disseram: Era mister que a vós se vos
pregasse primeiro a palavra de Deus; mas, visto que a rejeitais, e
não vos julgais dignos da vida eterna, eis que nos voltamos para os
gentios; porque o Senhor assim no-lo mandou: Eu te pus para
luz dos gentios, a fim de que sejas para salvação até os confins da
terra. E os gentios, ouvindo isto, alegraram-se, e
glorificavam a palavra do Senhor; e creram todos quantos estavam
ordenados para a vida eterna. E a palavra do Senhor se
divulgava por toda aquela província. Mas os judeus incitaram
algumas mulheres religiosas e honestas, e os principais da cidade, e
levantaram perseguição contra Paulo e Barnabé, e os lançaram fora
dos seus termos. Sacudindo, porém, contra eles o pó dos seus
pés, partiram para Icônio. E os discípulos estavam cheios de
alegria e do Espírito Santo.

\medskip

\lettrine{14} E aconteceu que em Icônio entraram juntos na
sinagoga dos judeus, e falaram de tal modo que creu uma grande
multidão, não só de judeus mas de gregos. Mas os judeus
incrédulos incitaram e irritaram, contra os irmãos, os ânimos dos
gentios. Detiveram-se, pois, muito tempo, falando ousadamente
acerca do Senhor, o qual dava testemunho à palavra da sua graça,
permitindo que por suas mãos se fizessem sinais e prodígios. E
dividiu-se a multidão da cidade; e uns eram pelos judeus, e outros
pelos apóstolos. E havendo um motim, tanto dos judeus como dos
gentios, com os seus principais, para os insultarem e apedrejarem,
sabendo-o eles, fugiram para Listra e Derbe, cidades de
Licaônia, e para a província circunvizinha; e ali pregavam o
evangelho.

E estava assentado em Listra certo homem leso dos pés, coxo desde
o ventre de sua mãe, o qual nunca tinha andado. Este ouviu falar
Paulo, que, fixando nele os olhos, e vendo que tinha fé para ser
curado, disse em voz alta: Levanta-te direito sobre teus pés.
E ele saltou e andou. E as multidões, vendo o que Paulo
fizera, levantaram a sua voz, dizendo em língua licaônica:
Fizeram-se os deuses semelhantes aos homens, e desceram até nós.
E chamavam Júpiter a Barnabé, e Mercúrio a Paulo; porque este
era o que falava. E o sacerdote de Júpiter, cujo templo
estava em frente da cidade, trazendo para a entrada da porta touros
e grinaldas, queria com a multidão sacrificar-lhes. Ouvindo,
porém, isto os apóstolos Barnabé e Paulo, rasgaram as suas vestes, e
saltaram para o meio da multidão, clamando, e dizendo:
Senhores, por que fazeis essas coisas? Nós também somos homens como
vós, sujeitos às mesmas paixões, e vos anunciamos que vos convertais
dessas vaidades ao Deus vivo, que fez o céu, e a terra, o mar, e
tudo quanto há neles; o qual nos tempos passados deixou andar
todas as nações em seus próprios caminhos. E contudo, não se
deixou a si mesmo sem testemunho, beneficiando-vos lá do céu,
dando-vos chuvas e tempos frutíferos, enchendo de mantimento e de
alegria os vossos corações. E, dizendo isto, com dificuldade
impediram que as multidões lhes sacrificassem.

Sobrevieram, porém, uns judeus de Antioquia e de Icônio que,
tendo convencido a multidão, apedrejaram a Paulo e o arrastaram para
fora da cidade, cuidando que estava morto. Mas, rodeando-o os
discípulos, levantou-se, e entrou na cidade, e no dia seguinte saiu
com Barnabé para Derbe. E, tendo anunciado o evangelho
naquela cidade e feito muitos discípulos, voltaram para Listra, e
Icônio e Antioquia, confirmando os ânimos dos discípulos,
exortando-os a permanecer na fé, pois que por muitas tribulações nos
importa entrar no reino de Deus. E, havendo-lhes, por comum
consentimento, eleito anciãos em cada igreja, orando com jejuns, os
encomendaram ao Senhor em quem haviam crido. Passando depois
por Pisídia, dirigiram-se a Panfília. E, tendo anunciado a
palavra em Perge, desceram a Atália. E dali navegaram para
Antioquia, de onde tinham sido encomendados à graça de Deus para a
obra que já haviam cumprido. E, quando chegaram e reuniram a
igreja, relataram quão grandes coisas Deus fizera por eles, e como
abrira aos gentios a porta da fé. E ficaram ali não pouco
tempo com os discípulos.

\medskip

\lettrine{15} Então alguns que tinham descido da Judéia
ensinavam assim os irmãos: Se não vos circuncidardes conforme o uso
de Moisés, não podeis salvar-vos. Tendo tido Paulo e Barnabé não
pequena discussão e contenda contra eles, resolveu-se que Paulo e
Barnabé, e alguns dentre eles, subissem a Jerusalém, aos apóstolos e
aos anciãos, sobre aquela questão. E eles, sendo acompanhados
pela igreja, passavam pela Fenícia e por Samaria, contando a
conversão dos gentios; e davam grande alegria a todos os irmãos.
E, quando chegaram a Jerusalém, foram recebidos pela igreja e
pelos apóstolos e anciãos, e lhes anunciaram quão grandes coisas
Deus tinha feito com eles. Alguns, porém, da seita dos fariseus,
que tinham crido, se levantaram, dizendo que era mister
circuncidá-los e mandar-lhes que guardassem a lei de Moisés.

Congregaram-se, pois, os apóstolos e os anciãos para considerar
este assunto. E, havendo grande contenda, levantou-se Pedro e
disse-lhes: Homens irmãos, bem sabeis que já há muito tempo Deus me
elegeu dentre nós, para que os gentios ouvissem da minha boca a
palavra do evangelho, e cressem. E Deus, que conhece os
corações, lhes deu testemunho, dando-lhes o Espírito Santo, assim
como também a nós; e não fez diferença alguma entre eles e nós,
purificando os seus corações pela fé. Agora, pois, por que
tentais a Deus, pondo sobre a cerviz dos discípulos um jugo que nem
nossos pais nem nós pudemos suportar? Mas cremos que seremos
salvos pela graça do Senhor Jesus Cristo, como eles também.
Então toda a multidão se calou e escutava a Barnabé e a
Paulo, que contavam quão grandes sinais e prodígios Deus havia feito
por meio deles entre os gentios. E, havendo-se eles calado,
tomou Tiago a palavra, dizendo: Homens irmãos, ouvi-me: Simão
relatou como primeiramente Deus visitou os gentios, para tomar deles
um povo para o seu nome. E com isto concordam as palavras dos
profetas; como está escrito: Depois disto voltarei, e
reedificarei o tabernáculo de Davi, que está caído, levantá-lo-ei
das suas ruínas, e tornarei a edificá-lo. Para que o restante
dos homens busque ao Senhor, e todos os gentios, sobre os quais o
meu nome é invocado, diz o Senhor, que faz todas estas coisas,
conhecidas são a Deus, desde o princípio do mundo, todas as
suas obras. Por isso julgo que não se deve perturbar aqueles,
dentre os gentios, que se convertem a Deus. Mas escrever-lhes
que se abstenham das contaminações dos ídolos, da prostituição, do
que é sufocado e do sangue. Porque Moisés, desde os tempos
antigos, tem em cada cidade quem o pregue, e cada sábado é lido nas
sinagogas.

Então pareceu bem aos apóstolos e aos anciãos, com toda a igreja,
eleger homens dentre eles e enviá-los com Paulo e Barnabé a
Antioquia, a saber: Judas, chamado Barsabás, e Silas, homens
distintos entre os irmãos. E por intermédio deles escreveram
o seguinte: Os apóstolos, e os anciãos e os irmãos, aos irmãos
dentre os gentios que estão em Antioquia, e Síria e Cilícia, saúde.
Porquanto ouvimos que alguns que saíram dentre nós vos
perturbaram com palavras, e transtornaram as vossas almas, dizendo
que deveis circuncidar-vos e guardar a lei, não lhes tendo nós dado
mandamento, pareceu-nos bem, reunidos concordemente, eleger
alguns homens e enviá-los com os nossos amados Barnabé e Paulo,
homens que já expuseram as suas vidas pelo nome de nosso
Senhor Jesus Cristo. Enviamos, portanto, Judas e Silas, os
quais por palavra vos anunciarão também as mesmas coisas. Na
verdade pareceu bem ao Espírito Santo e a nós, não vos impor mais
encargo algum, senão estas coisas necessárias: Que vos
abstenhais das coisas sacrificadas aos ídolos, e do sangue, e da
carne sufocada, e da prostituição, das quais coisas bem fazeis se
vos guardardes. Bem vos vá. Tendo eles então se despedido,
partiram para Antioquia e, ajuntando a multidão, entregaram a carta.
E, quando a leram, alegraram-se pela exortação. Depois
Judas e Silas, que também eram profetas, exortaram e confirmaram os
irmãos com muitas palavras. E, detendo-se ali algum tempo, os
irmãos os deixaram voltar em paz para os apóstolos; Mas
pareceu bem a Silas ficar ali. E Paulo e Barnabé ficaram em
Antioquia, ensinando e pregando, com muitos outros, a palavra do
Senhor.

E alguns dias depois, disse Paulo a Barnabé: Tornemos a visitar
nossos irmãos por todas as cidades em que já anunciamos a palavra do
Senhor, para ver como estão. E Barnabé aconselhava que
tomassem consigo a João, chamado Marcos. Mas a Paulo parecia
razoável que não tomassem consigo aquele que desde a Panfília se
tinha apartado deles e não os acompanhou naquela obra. E tal
contenda houve entre eles, que se apartaram um do outro. Barnabé,
levando consigo a Marcos, navegou para Chipre. E Paulo, tendo
escolhido a Silas, partiu, encomendado pelos irmãos à graça de Deus.
E passou pela Síria e Cilícia, confirmando as igrejas.

\medskip

\lettrine{16} E chegou a Derbe e Listra. E eis que estava ali
um certo discípulo por nome Timóteo, filho de uma judia que era
crente, mas de pai grego; do qual davam bom testemunho os irmãos
que estavam em Listra e em Icônio. Paulo quis que este fosse com
ele; e tomando-o, o circuncidou, por causa dos judeus que estavam
naqueles lugares; porque todos sabiam que seu pai era grego. E,
quando iam passando pelas cidades, lhes entregavam, para serem
observados, os decretos que haviam sido estabelecidos pelos
apóstolos e anciãos em Jerusalém. De sorte que as igrejas eram
confirmadas na fé, e cada dia cresciam em número.

E, passando pela Frígia e pela província da Galácia, foram
impedidos pelo Espírito Santo de anunciar a palavra na Ásia. E,
quando chegaram a Mísia, intentavam ir para Bitínia, mas o Espírito
não lho permitiu. E, tendo passado por Mísia, desceram a Trôade.
E Paulo teve de noite uma visão, em que se apresentou um homem
da Macedônia, e lhe rogou, dizendo: Passa à Macedônia, e ajuda-nos.
E, logo depois desta visão, procuramos partir para a
Macedônia, concluindo que o Senhor nos chamava para lhes anunciarmos
o evangelho. E, navegando de Trôade, fomos correndo em
caminho direito para a Samotrácia e, no dia seguinte, para Neápolis;
e dali para Filipos, que é a primeira cidade desta parte da
Macedônia, e é uma colônia; e estivemos alguns dias nesta cidade.
E no dia de sábado saímos fora das portas\footnote{KJ: And on
the sabbath we went out of the city by a river side \ldots{}. RA: No
sábado, saímos da cidade \ldots{}.}, para a beira do rio, onde se
costumava fazer oração; e, assentando-nos, falamos às mulheres que
ali se ajuntaram. E uma certa mulher, chamada Lídia,
vendedora de púrpura, da cidade de Tiatira, e que servia a Deus, nos
ouvia, e o Senhor lhe abriu o coração para que estivesse atenta ao
que Paulo dizia. E, depois que foi batizada, ela e a sua
casa, nos rogou, dizendo: Se haveis julgado que eu seja fiel ao
Senhor, entrai em minha casa, e ficai ali. E nos constrangeu a isso.

E aconteceu que, indo nós à oração, nos saiu ao encontro uma
jovem, que tinha espírito de adivinhação, a qual, adivinhando, dava
grande lucro aos seus senhores. Esta, seguindo a Paulo e a
nós, clamava, dizendo: Estes homens, que nos anunciam o caminho da
salvação, são servos do Deus Altíssimo. E isto fez ela por
muitos dias. Mas Paulo, perturbado, voltou-se e disse ao espírito:
Em nome de Jesus Cristo, te mando que saias dela. E na mesma hora
saiu. E, vendo seus senhores que a esperança do seu lucro
estava perdida, prenderam Paulo e Silas, e os levaram à praça, à
presença dos magistrados. E, apresentando-os aos magistrados,
disseram: Estes homens, sendo judeus, perturbaram a nossa cidade,
e nos expõem costumes que não nos é lícito receber nem
praticar, visto que somos romanos. E a multidão se levantou
unida contra eles, e os magistrados, rasgando-lhes as vestes,
mandaram açoitá-los com varas. E, havendo-lhes dado muitos
açoites, os lançaram na prisão, mandando ao carcereiro que os
guardasse com segurança. O qual, tendo recebido tal ordem, os
lançou no cárcere interior, e lhes segurou os pés no tronco.

E, perto da meia-noite, Paulo e Silas oravam e cantavam hinos a
Deus, e os outros presos os escutavam. E de repente sobreveio
um tão grande terremoto, que os alicerces do cárcere se moveram, e
logo se abriram todas as portas, e foram soltas as prisões de todos.
E, acordando o carcereiro, e vendo abertas as portas da
prisão, tirou a espada, e quis matar-se, cuidando que os presos já
tinham fugido. Mas Paulo clamou com grande voz, dizendo: Não
te faças nenhum mal, que todos aqui estamos. E, pedindo luz,
saltou dentro e, todo trêmulo, se prostrou ante Paulo e Silas.
E, tirando-os para fora, disse: Senhores, que é necessário
que eu faça para me salvar? E eles disseram: Crê no Senhor
Jesus Cristo e serás salvo, tu e a tua casa. E lhe pregavam a
palavra do Senhor, e a todos os que estavam em sua casa. E,
tomando-os ele consigo naquela mesma hora da noite, lavou-lhes os
vergões\footnote{Vinco ou marca na pele, produzido por pancada,
sobretudo de vergasta ou azorrague, ou por outra causa.}; e logo foi
batizado, ele e todos os seus. E, levando-os à sua casa, lhes
pôs a mesa; e, na sua crença em Deus, alegrou-se com toda a sua
casa.

E, sendo já dia, os magistrados mandaram
quadrilheiros\footnote{Indivíduo que faz parte de quadrilha. Membro
de quadrilha de guerreiros ou de jogadores das canas.}, dizendo:
Soltai aqueles homens. E o carcereiro anunciou a Paulo estas
palavras, dizendo: Os magistrados mandaram que vos soltasse; agora,
pois, saí e ide em paz. Mas Paulo replicou: Açoitaram-nos
publicamente e, sem sermos condenados, sendo homens romanos, nos
lançaram na prisão, e agora encobertamente nos lançam fora? Não será
assim; mas venham eles mesmos e tirem-nos para fora. E os
quadrilheiros foram dizer aos magistrados estas palavras; e eles
temeram, ouvindo que eram romanos. E, vindo, lhes dirigiram
súplicas; e, tirando-os para fora, lhes pediram que saíssem da
cidade. E, saindo da prisão, entraram em casa de Lídia e,
vendo os irmãos, os confortaram, e depois partiram.

\medskip

\lettrine{17} E passando por Anfípolis e Apolônia, chegaram a
Tessalônica, onde havia uma sinagoga de judeus. E Paulo, como
tinha por costume, foi ter com eles; e por três sábados disputou com
eles sobre as Escrituras, expondo e demonstrando que convinha
que o Cristo padecesse e ressuscitasse dentre os mortos. E este
Jesus, que vos anuncio, dizia ele, é o Cristo. E alguns deles
creram, e ajuntaram-se com Paulo e Silas; e também uma grande
multidão de gregos religiosos, e não poucas mulheres principais.
Mas os judeus desobedientes, movidos de inveja, tomaram consigo
alguns homens perversos, dentre os vadios e, ajuntando o povo,
alvoroçaram a cidade, e assaltando a casa de Jasom, procuravam
trazê-los para junto do povo. E, não os achando, trouxeram Jasom
e alguns irmãos à presença dos magistrados da cidade, clamando:
Estes que têm alvoroçado o mundo, chegaram também aqui; os quais
Jasom recolheu; e todos estes procedem contra os decretos de César,
dizendo que há outro rei, Jesus. E alvoroçaram a multidão e os
principais da cidade, que ouviram estas coisas. Tendo, porém,
recebido satisfação de Jasom e dos demais, os soltaram.

E logo os irmãos enviaram de noite Paulo e Silas a Beréia; e
eles, chegando lá, foram à sinagoga dos judeus. Ora, estes
foram mais nobres do que os que estavam em Tessalônica, porque de
bom grado receberam a palavra, examinando cada dia nas Escrituras se
estas coisas eram assim. De sorte que creram muitos deles, e
também mulheres gregas da classe nobre, e não poucos homens.
Mas, logo que os judeus de Tessalônica souberam que a palavra
de Deus também era anunciada por Paulo em Beréia, foram lá, e
excitaram as multidões. No mesmo instante os irmãos mandaram
a Paulo que fosse até ao mar, mas Silas e Timóteo ficaram ali.
E os que acompanhavam Paulo o levaram até Atenas, e,
recebendo ordem para que Silas e Timóteo fossem ter com ele o mais
depressa possível, partiram.

E, enquanto Paulo os esperava em Atenas, o seu espírito se
comovia em si mesmo, vendo a cidade tão entregue à idolatria.
De sorte que disputava na sinagoga com os judeus e
religiosos, e todos os dias na praça com os que se apresentavam.
E alguns dos filósofos epicureus e estóicos contendiam com
ele; e uns diziam: Que quer dizer este paroleiro\footnote{Que ou
aquele que gosta de parolas ou de estar à parola (conversa sem
conseqüência ou compromisso; conversa fiada; seqüência de palavras
ocas; palavreado, parlenda, palanfrório, paleio; tagarelice, trela);
parlapatão, fanfarrão, embusteiro, mentiroso.}? E outros: Parece que
é pregador de deuses estranhos; porque lhes anunciava a Jesus e a
ressurreição. E tomando-o, o levaram ao
Areópago\footnote{Tribunal ateniense, assembléia de magistrados,
sábios, literatos, etc.}, dizendo: Poderemos nós saber que nova
doutrina é essa de que falas? Pois coisas estranhas nos
trazes aos ouvidos; queremos pois saber o que vem a ser isto

todos os atenienses e estrangeiros residentes, de nenhuma
outra coisa se ocupavam, senão de dizer e ouvir alguma novidade).

E, estando Paulo no meio do Areópago, disse: Homens atenienses,
em tudo vos vejo um tanto supersticiosos; porque, passando eu
e vendo os vossos santuários, achei também um altar em que estava
escrito: ao deus desconhecido. Esse, pois, que vós honrais, não o
conhecendo, é o que eu vos anuncio. O Deus que fez o mundo e
tudo que nele há, sendo Senhor do céu e da terra, não habita em
templos feitos por mãos de homens; nem tampouco é servido por
mãos de homens, como que necessitando de alguma coisa; pois ele
mesmo é quem dá a todos a vida, e a respiração, e todas as coisas;
e de um só sangue fez toda a geração dos homens, para habitar
sobre toda a face da terra, determinando os tempos já dantes
ordenados, e os limites da sua habitação; para que buscassem
ao Senhor, se porventura, tateando, o pudessem achar; ainda que não
está longe de cada um de nós; porque nele vivemos, e nos
movemos, e existimos; como também alguns dos vossos poetas disseram:
Pois somos também sua geração. Sendo nós, pois, geração de
Deus, não havemos de cuidar que a divindade seja semelhante ao ouro,
ou à prata, ou à pedra esculpida por artifício e imaginação dos
homens. Mas Deus, não tendo em conta os tempos da ignorância,
anuncia agora a todos os homens, e em todo o lugar, que se
arrependam; porquanto tem determinado um dia em que com
justiça há de julgar o mundo, por meio do homem que destinou; e
disso deu certeza a todos, ressuscitando-o dentre os mortos.

E, como ouviram falar da ressurreição dos mortos, uns
escarneciam, e outros diziam: Acerca disso te ouviremos outra vez.
E assim Paulo saiu do meio deles. Todavia, chegando
alguns homens a ele, creram; entre os quais foi Dionísio,
areopagita, uma mulher por nome Dâmaris, e com eles outros.

\medskip

\lettrine{18} E depois disto partiu Paulo de Atenas, e chegou
a Corinto. E, achando um certo judeu por nome Áqüila, natural do
Ponto, que havia pouco tinha vindo da Itália, e Priscila, sua mulher
(pois Cláudio tinha mandado que todos os judeus saíssem de Roma),
ajuntou-se com eles, e, como era do mesmo ofício, ficou com
eles, e trabalhava; pois tinham por ofício fazer tendas. E todos
os sábados disputava na sinagoga, e convencia a judeus e gregos.
E, quando Silas e Timóteo desceram da Macedônia, foi Paulo
impulsionado no espírito, testificando aos judeus que Jesus era o
Cristo. Mas, resistindo e blasfemando eles, sacudiu as vestes, e
disse-lhes: O vosso sangue seja sobre a vossa cabeça; eu estou
limpo, e desde agora parto para os gentios.

E, saindo dali, entrou em casa de um homem chamado Tício Justo,
que servia a Deus, e cuja casa estava junto da sinagoga. E
Crispo, principal da sinagoga, creu no Senhor com toda a sua casa; e
muitos dos coríntios, ouvindo-o, creram e foram batizados. E
disse o Senhor em visão a Paulo: Não temas, mas fala, e não te
cales; porque eu sou contigo, e ninguém lançará mão de ti
para te fazer mal, pois tenho muito povo nesta cidade. E
ficou ali um ano e seis meses, ensinando entre eles a palavra de
Deus.

Mas, sendo Gálio procônsul da Acaia, levantaram-se os judeus
concordemente contra Paulo, e o levaram ao tribunal, dizendo:
Este persuade os homens a servir a Deus contra a lei. E,
querendo Paulo abrir a boca, disse Gálio aos judeus: Se houvesse, ó
judeus, algum agravo ou crime enorme, com razão vos sofreria,
mas, se a questão é de palavras, e de nomes, e da lei que
entre vós há, vede-o vós mesmos; porque eu não quero ser juiz dessas
coisas. E expulsou-os do tribunal. Então todos os
gregos agarraram Sóstenes, principal da sinagoga, e o feriram diante
do tribunal; e a Gálio nada destas coisas o incomodava.

E Paulo, ficando ainda ali muitos dias, despediu-se dos irmãos, e
dali navegou para a Síria, e com ele Priscila e Áqüila, tendo rapado
a cabeça em Cencréia, porque tinha voto. E chegou a Éfeso, e
deixou-os ali; mas ele, entrando na sinagoga, disputava com os
judeus. E, rogando-lhe eles que ficasse por mais algum tempo,
não conveio nisso. Antes se despediu deles, dizendo: É-me de
todo preciso celebrar a solenidade que vem em Jerusalém; mas
querendo Deus, outra vez voltarei a vós. E partiu de Éfeso.
E, chegando a Cesaréia, subiu a Jerusalém e, saudando a
igreja, desceu a Antioquia. E, estando ali algum tempo,
partiu, passando sucessivamente pela província da Galácia e da
Frígia, confirmando a todos os discípulos.

E chegou a Éfeso um certo judeu chamado Apolo, natural de
Alexandria, homem eloqüente e poderoso nas Escrituras. Este
era instruído no caminho do Senhor e, fervoroso de espírito, falava
e ensinava diligentemente as coisas do Senhor, conhecendo somente o
batismo de João. Ele começou a falar ousadamente na sinagoga;
e, quando o ouviram Priscila e Áqüila, o levaram consigo e lhe
declararam mais precisamente o caminho de Deus. Querendo ele
passar à Acaia, o animaram os irmãos, e escreveram aos discípulos
que o recebessem; o qual, tendo chegado, aproveitou muito aos que
pela graça criam. Porque com grande veemência, convencia
publicamente os judeus, mostrando pelas Escrituras que Jesus era o
Cristo.

\medskip

\lettrine{19} E sucedeu que, enquanto Apolo estava em Corinto,
Paulo, tendo passado por todas as regiões superiores, chegou a
Éfeso; e achando ali alguns discípulos, disse-lhes: Recebestes
vós já o Espírito Santo quando crestes? E eles disseram-lhe: Nós nem
ainda ouvimos que haja Espírito Santo. Perguntou-lhes, então: Em
que sois batizados então? E eles disseram: No batismo de João.
Mas Paulo disse: Certamente João batizou com o batismo do
arrependimento, dizendo ao povo que cresse no que após ele havia de
vir, isto é, em Jesus Cristo. E os que ouviram foram batizados
em nome do Senhor Jesus. E, impondo-lhes Paulo as mãos, veio
sobre eles o Espírito Santo; e falavam línguas, e profetizavam.
E estes eram, ao todo, uns doze homens.

E, entrando na sinagoga, falou ousadamente por espaço de três
meses, disputando e persuadindo-os acerca do reino de Deus. Mas,
como alguns deles se endurecessem e não obedecessem, falando mal do
Caminho perante a multidão, retirou-se deles, e separou os
discípulos, disputando todos os dias na escola de um certo Tirano.
E durou isto por espaço de dois anos; de tal maneira que
todos os que habitavam na Ásia ouviram a palavra do Senhor Jesus,
assim judeus como gregos. E Deus pelas mãos de Paulo fazia
maravilhas extraordinárias. De sorte que até os lenços e
aventais se levavam do seu corpo aos enfermos, e as enfermidades
fugiam deles, e os espíritos malignos saíam.

E alguns dos exorcistas judeus ambulantes tentavam invocar o nome
do Senhor Jesus sobre os que tinham espíritos malignos, dizendo:
Esconjuro-vos por Jesus a quem Paulo prega. E os que faziam
isto eram sete filhos de Ceva, judeu, principal dos sacerdotes.
Respondendo, porém, o espírito maligno, disse: Conheço a
Jesus, e bem sei quem é Paulo; mas vós quem sois? E, saltando
neles o homem que tinha o espírito maligno, e assenhoreando-se de
todos, pôde mais do que eles; de tal maneira que, nus e feridos,
fugiram daquela casa. E foi isto notório a todos os que
habitavam em Éfeso, tanto judeus como gregos; e caiu temor sobre
todos eles, e o nome do Senhor Jesus era engrandecido. E
muitos dos que tinham crido vinham, confessando e publicando os seus
feitos. Também muitos dos que seguiam artes mágicas trouxeram
os seus livros, e os queimaram na presença de todos e, feita a conta
do seu preço, acharam que montava a cinqüenta mil peças de prata.
Assim a palavra do Senhor crescia poderosamente e prevalecia.

E, cumpridas estas coisas, Paulo propôs, em espírito, ir a
Jerusalém, passando pela Macedônia e pela Acaia, dizendo: Depois que
houver estado ali, importa-me ver também Roma. E, enviando à
Macedônia dois daqueles que o serviam, Timóteo e Erasto, ficou ele
por algum tempo na Ásia. E, naquele mesmo tempo, houve um não
pequeno alvoroço acerca do Caminho. Porque um certo ourives
da prata, por nome Demétrio, que fazia de prata nichos de Diana,
dava não pouco lucro aos artífices, aos quais, havendo-os
ajuntado com os oficiais de obras semelhantes, disse: Senhores, vós
bem sabeis que deste ofício temos a nossa prosperidade; e bem
vedes e ouvis que não só em Éfeso, mas até quase em toda a Ásia,
este Paulo tem convencido e afastado uma grande multidão, dizendo
que não são deuses os que se fazem com as mãos. E não somente
há o perigo de que a nossa profissão caia em descrédito, mas também
de que o próprio templo da grande deusa Diana seja estimado em nada,
vindo a ser destruída a majestade daquela que toda a Ásia e o mundo
veneram. E, ouvindo-o, encheram-se de ira, e clamaram,
dizendo: Grande é a Diana dos efésios. E encheu-se de
confusão toda a cidade e, unânimes, correram ao teatro, arrebatando
a Gaio e a Aristarco, macedônios, companheiros de Paulo na viagem.
E, querendo Paulo apresentar-se ao povo, não lho permitiram
os discípulos. E também alguns dos principais da Ásia, que
eram seus amigos, lhe rogaram que não se apresentasse no teatro.
Uns, pois, clamavam de uma maneira, outros de outra, porque o
ajuntamento era confuso; e os mais deles não sabiam por que causa se
tinham ajuntado. Então tiraram Alexandre dentre a multidão,
impelindo-o os judeus para diante; e Alexandre, acenando com a mão,
queria dar razão disto ao povo. Mas quando conheceram que era
judeu, todos unanimemente levantaram a voz, clamando por espaço de
quase duas horas: Grande é a Diana dos efésios. Então o
escrivão da cidade, tendo apaziguado a multidão, disse: Homens
efésios, qual é o homem que não sabe que a cidade dos efésios é a
guardadora do templo da grande deusa Diana, e da imagem que desceu
de Júpiter? Ora, não podendo isto ser contraditado, convém
que vos aplaqueis e nada façais temerariamente; porque estes
homens que aqui trouxestes nem são sacrílegos nem blasfemam da vossa
deusa. Mas, se Demétrio e os artífices que estão com ele têm
alguma coisa contra alguém, há audiências e há procônsules; que se
acusem uns aos outros; e, se alguma outra coisa demandais,
averiguar-se-á em legítima assembléia. Na verdade até
corremos perigo de que, por hoje, sejamos acusados de sedição, não
havendo causa alguma com que possamos justificar este concurso.
E, tendo dito isto, despediu a assembléia.

\medskip

\lettrine{20} E, depois que cessou o alvoroço, Paulo chamou a
si os discípulos e, abraçando-os, saiu para a Macedônia. E,
havendo andado por aquelas terras, exortando-os com muitas palavras,
veio à Grécia. E, passando ali três meses, e sendo-lhe pelos
judeus postas ciladas, como tivesse de navegar para a Síria,
determinou voltar pela Macedônia. E acompanhou-o, até à Ásia,
Sópater, de Beréia, e, dos de Tessalônica, Aristarco, e Segundo, e
Gaio de Derbe, e Timóteo, e, dos da Ásia, Tíquico e Trófimo.
Estes, indo adiante, nos esperaram em Trôade. E, depois dos
dias dos pães ázimos, navegamos de Filipos, e em cinco dias fomos
ter com eles a Trôade, onde estivemos sete dias.

E no primeiro dia da semana, ajuntando-se os discípulos para
partir o pão, Paulo, que havia de partir no dia seguinte, falava com
eles; e prolongou a prática\footnote{Palestra, conferência, fala.
pequeno discurso feito por um eclesiástico aos fiéis.} até à
meia-noite. E havia muitas luzes no cenáculo onde estavam
juntos. E, estando um certo jovem, por nome Êutico, assentado
numa janela, caiu do terceiro andar, tomado de um sono profundo que
lhe sobreveio durante o extenso discurso de Paulo; e foi levantado
morto. Paulo, porém, descendo, inclinou-se sobre ele e,
abraçando-o, disse: Não vos perturbeis, que a sua alma nele está.
E subindo, e partindo o pão, e comendo, ainda lhes falou
largamente até à alvorada; e assim partiu. E levaram vivo o
jovem, e ficaram não pouco consolados.

Nós, porém, subindo ao navio, navegamos até Assôs, onde devíamos
receber a Paulo, porque assim o ordenara, indo ele por terra.
E, logo que se ajuntou conosco em Assôs, o recebemos, e fomos
a Mitilene. E, navegando dali, chegamos no dia seguinte
defronte de Quios, e no outro aportamos a Samos e, ficando em
Trogílio, chegamos no dia seguinte a Mileto. Porque já Paulo
tinha determinado passar ao largo de Éfeso, para não gastar tempo na
Ásia. Apressava-se, pois, para estar, se lhe fosse possível, em
Jerusalém no dia de Pentecostes.

E de Mileto mandou a Éfeso, a chamar os anciãos da igreja.
E, logo que chegaram junto dele, disse-lhes: Vós bem sabeis,
desde o primeiro dia em que entrei na Ásia, como em todo esse tempo
me portei no meio de vós, servindo ao Senhor com toda a
humildade, e com muitas lágrimas e tentações, que pelas ciladas dos
judeus me sobrevieram; como nada, que útil seja, deixei de
vos anunciar, e ensinar publicamente e pelas casas,
testificando, tanto aos judeus como aos gregos, a conversão a
Deus, e a fé em nosso Senhor Jesus Cristo. E agora, eis que,
ligado eu pelo espírito, vou para Jerusalém, não sabendo o que lá me
há de acontecer, senão o que o Espírito Santo de cidade em
cidade me revela, dizendo que me esperam prisões e tribulações.
Mas em nada tenho a minha vida por preciosa, contanto que
cumpra com alegria a minha carreira, e o ministério que recebi do
Senhor Jesus, para dar testemunho do evangelho da graça de Deus.
E agora, na verdade, sei que todos vós, por quem passei
pregando o reino de Deus, não vereis mais o meu rosto.
Portanto, no dia de hoje, vos protesto que estou limpo do
sangue de todos. Porque nunca deixei de vos anunciar todo o
conselho de Deus. Olhai, pois, por vós, e por todo o rebanho
sobre que o Espírito Santo vos constituiu bispos, para apascentardes
a igreja de Deus, que ele resgatou com seu próprio sangue.
Porque eu sei isto que, depois da minha partida, entrarão no
meio de vós lobos cruéis, que não pouparão ao rebanho; e que
de entre vós mesmos se levantarão homens que falarão coisas
perversas, para atraírem os discípulos após si. Portanto,
vigiai, lembrando-vos de que durante três anos, não cessei, noite e
dia, de admoestar com lágrimas a cada um de vós. Agora, pois,
irmãos, encomendo-vos a Deus e à palavra da sua graça; a ele que é
poderoso para vos edificar e dar herança entre todos os
santificados. De ninguém cobicei a prata, nem o ouro, nem o
vestuário. Sim, vós mesmos sabeis que para o que me era
necessário a mim, e aos que estão comigo, estas mãos me serviram.
Tenho-vos mostrado em tudo que, trabalhando assim, é
necessário auxiliar os enfermos, e recordar as palavras do Senhor
Jesus, que disse: Mais bem-aventurada coisa é dar do que receber.

E, havendo dito isto, pôs-se de joelhos, e orou com todos eles.
E levantou-se um grande pranto entre todos e, lançando-se ao
pescoço de Paulo, o beijavam, entristecendo-se muito,
principalmente pela palavra que dissera, que não veriam mais o seu
rosto. E acompanharam-no até o navio.

\medskip

\lettrine{21} E aconteceu que, separando-nos deles, navegamos
e fomos correndo caminho direito, e chegamos a Cós, e no dia
seguinte a Rodes, de onde passamos a Pátara. E, achando um
navio, que ia para a Fenícia, embarcamos nele, e partimos. E,
indo já à vista de Chipre, deixando-a à esquerda, navegamos para a
Síria e chegamos a Tiro; porque o navio havia de ser descarregado
ali. E, achando discípulos, ficamos ali sete dias; e eles pelo
Espírito diziam a Paulo que não subisse a Jerusalém. E, havendo
passado ali aqueles dias, saímos, e seguimos nosso caminho,
acompanhando-nos todos, com suas mulheres e filhos até fora da
cidade; e, postos de joelhos na praia, oramos. E, despedindo-nos
uns dos outros, subimos ao navio; e eles voltaram para suas casas.
E nós, concluída a navegação de Tiro, viemos a Ptolemaida; e,
havendo saudado os irmãos, ficamos com eles um dia.

E no dia seguinte, partindo dali Paulo, e nós que com ele
estávamos, chegamos a Cesaréia; e, entrando em casa de Filipe, o
evangelista, que era um dos sete, ficamos com ele. E tinha este
quatro filhas virgens, que profetizavam. E, demorando-nos ali
por muitos dias, chegou da Judéia um profeta, por nome Ágabo;
e, vindo ter conosco, tomou a cinta de Paulo, e ligando-se os
seus próprios pés e mãos, disse: Isto diz o Espírito Santo: Assim
ligarão os judeus em Jerusalém o homem de quem é esta cinta, e o
entregarão nas mãos dos gentios. E, ouvindo nós isto,
rogamos-lhe, tanto nós como os que eram daquele lugar, que não
subisse a Jerusalém. Mas Paulo respondeu: Que fazeis vós,
chorando e magoando-me o coração? Porque eu estou pronto não só a
ser ligado, mas ainda a morrer em Jerusalém pelo nome do Senhor
Jesus. E, como não podíamos convencê-lo, nos aquietamos,
dizendo: Faça-se a vontade do Senhor.

E depois daqueles dias, havendo feito os nossos preparativos,
subimos a Jerusalém. E foram também conosco alguns discípulos
de Cesaréia, levando consigo um certo Mnasom, chíprio, discípulo
antigo, com quem havíamos de hospedar-nos. E, logo que
chegamos a Jerusalém, os irmãos nos receberam de muito boa vontade.
E no dia seguinte, Paulo entrou conosco em casa de Tiago, e
todos os anciãos vieram ali. E, havendo-os saudado,
contou-lhes por miúdo o que por seu ministério Deus fizera entre os
gentios. E, ouvindo-o eles, glorificaram ao Senhor, e
disseram-lhe: Bem vês, irmão, quantos milhares de judeus há que
crêem, e todos são zeladores da lei. E já acerca de ti foram
informados de que ensinas todos os judeus que estão entre os gentios
a apartarem-se de Moisés, dizendo que não devem circuncidar seus
filhos, nem andar segundo o costume da lei. Que faremos pois?
em todo o caso é necessário que a multidão se ajunte; porque terão
ouvido que já és vindo. Faze, pois, isto que te dizemos:
Temos quatro homens que fizeram voto. Toma estes contigo, e
santifica-te com eles, e faze por eles os gastos para que rapem a
cabeça, e todos ficarão sabendo que nada há daquilo de que foram
informados acerca de ti, mas que também tu mesmo andas guardando a
lei. Todavia, quanto aos que crêem dos gentios, já nós
havemos escrito, e achado por bem, que nada disto observem; mas que
só se guardem do que se sacrifica aos ídolos, e do sangue, e do
sufocado e da prostituição. Então Paulo, tomando consigo
aqueles homens, entrou no dia seguinte no templo, já santificado com
eles, anunciando serem já cumpridos os dias da purificação; e ficou
ali até se oferecer por cada um deles a oferta.

E quando os sete dias estavam quase a terminar, os judeus da
Ásia, vendo-o no templo, alvoroçaram todo o povo e lançaram mão
dele, clamando: Homens israelitas, acudi; este é o homem que
por todas as partes ensina a todos contra o povo e contra a lei, e
contra este lugar; e, demais disto, introduziu também no templo os
gregos, e profanou este santo lugar. Porque tinham visto com
ele na cidade a Trófimo de Éfeso, o qual pensavam que Paulo
introduzira no templo. E alvoroçou-se toda a cidade, e houve
grande concurso de povo; e, pegando Paulo, o arrastaram para fora do
templo, e logo as portas se fecharam. E, procurando eles
matá-lo, chegou ao tribuno da coorte o aviso de que Jerusalém estava
toda em confusão; o qual, tomando logo consigo soldados e
centuriões, correu para eles. E, quando viram o tribuno e os
soldados, cessaram de ferir a Paulo. Então, aproximando-se o
tribuno, o prendeu e o mandou atar com duas cadeias, e lhe perguntou
quem era e o que tinha feito. E na multidão uns clamavam de
uma maneira, outros de outra; mas, como nada podia saber ao certo,
por causa do alvoroço, mandou conduzi-lo para a fortaleza. E
sucedeu que, chegando às escadas, os soldados tiveram de lhe pegar
por causa da violência da multidão. Porque a multidão do povo
o seguia, clamando: Mata-o! E, quando iam a introduzir Paulo
na fortaleza, disse Paulo ao tribuno: É-me permitido dizer-te alguma
coisa? E ele disse: Sabes o grego? Não és tu porventura
aquele egípcio que antes destes dias fez uma sedição e levou ao
deserto quatro mil salteadores? Mas Paulo lhe disse: Na
verdade que sou um homem judeu, cidadão de Tarso, cidade não pouco
célebre na Cilícia; rogo-te, porém, que me permitas falar ao povo.
E, havendo-lho permitido, Paulo, pondo-se em pé nas escadas,
fez sinal com a mão ao povo; e, feito grande silêncio, falou-lhes em
língua hebraica, dizendo:

\medskip

\lettrine{22} Homens, irmãos e pais, ouvi agora a minha defesa
perante vós (e, quando ouviram falar-lhes em língua hebraica,
maior silêncio guardaram). E disse:

Quanto a mim, sou judeu, nascido em Tarso da Cilícia, e nesta
cidade criado aos pés de Gamaliel, instruído conforme a verdade da
lei de nossos pais, zelador de Deus, como todos vós hoje sois. E
persegui este caminho até à morte, prendendo, e pondo em prisões,
tanto homens como mulheres, como também o sumo sacerdote me é
testemunha, e todo o conselho dos anciãos. E, recebendo destes
cartas para os irmãos, fui a Damasco, para trazer maniatados para
Jerusalém aqueles que ali estivessem, a fim de que fossem
castigados. Ora, aconteceu que, indo eu já de caminho, e
chegando perto de Damasco, quase ao meio-dia, de repente me rodeou
uma grande luz do céu. E caí por terra, e ouvi uma voz que me
dizia: Saulo, Saulo, por que me persegues? E eu respondi: Quem
és, Senhor? E disse-me: Eu sou Jesus Nazareno, a quem tu persegues.
E os que estavam comigo viram, em verdade, a luz, e se
atemorizaram muito, mas não ouviram a voz daquele que falava comigo.
Então disse eu: Senhor, que farei? E o Senhor disse-me:
Levanta-te, e vai a Damasco, e ali se te dirá tudo o que te é
ordenado fazer. E, como eu não via, por causa do esplendor
daquela luz, fui levado pela mão dos que estavam comigo, e cheguei a
Damasco. E um certo Ananias, homem piedoso conforme a lei,
que tinha bom testemunho de todos os judeus que ali moravam,
vindo ter comigo, e apresentando-se, disse-me: Saulo, irmão,
recobra a vista. E naquela mesma hora o vi. E ele disse: O
Deus de nossos pais de antemão te designou para que conheças a sua
vontade, e vejas aquele Justo e ouças a voz da sua boca.
Porque hás de ser sua testemunha para com todos os homens do
que tens visto e ouvido. E agora por que te deténs?
Levanta-te, e batiza-te, e lava os teus pecados, invocando o nome do
Senhor. E aconteceu que, tornando eu para Jerusalém, quando
orava no templo, fui arrebatado para fora de mim. E vi aquele
que me dizia: Dá-te pressa e sai apressadamente de Jerusalém; porque
não receberão o teu testemunho acerca de mim. E eu disse:
Senhor, eles bem sabem que eu lançava na prisão e açoitava nas
sinagogas os que criam em ti. E quando o sangue de Estêvão,
tua testemunha, se derramava, também eu estava presente, e consentia
na sua morte, e guardava as capas dos que o matavam. E
disse-me: Vai, porque hei de enviar-te aos gentios de longe.

E ouviram-no até esta palavra, e levantaram a voz, dizendo: Tira
da terra um tal homem, porque não convém que viva. E,
clamando eles, e arrojando de si as vestes, e lançando pó para o ar,
o tribuno mandou que o levassem para a fortaleza, dizendo que
o examinassem com açoites, para saber por que causa assim clamavam
contra ele. E, quando o estavam atando com correias, disse
Paulo ao centurião que ali estava: É-vos lícito açoitar um romano,
sem ser condenado? E, ouvindo isto, o centurião foi, e
anunciou ao tribuno, dizendo: Vê o que vais fazer, porque este homem
é romano. E, vindo o tribuno, disse-lhe: Dize-me, és tu
romano? E ele disse: Sim. E respondeu o tribuno: Eu com
grande soma de dinheiro alcancei este direito de cidadão. Paulo
disse: Mas eu o sou de nascimento. E logo dele se apartaram
os que o haviam de examinar; e até o tribuno teve temor, quando
soube que era romano, visto que o tinha ligado. E no dia
seguinte, querendo saber ao certo a causa por que era acusado pelos
judeus, soltou-o das prisões, e mandou vir o principais dos
sacerdotes, e todo o seu conselho; e, trazendo Paulo, o apresentou
diante deles.

\medskip

\lettrine{23} E, pondo Paulo os olhos no conselho, disse:
Homens irmãos, até ao dia de hoje tenho andado diante de Deus com
toda a boa consciência. Mas o sumo sacerdote, Ananias, mandou
aos que estavam junto dele que o ferissem na boca. Então Paulo
lhe disse: Deus te ferirá, parede branqueada; tu estás aqui
assentado para julgar-me conforme a lei, e contra a lei me mandas
ferir? E os que ali estavam disseram: Injurias o sumo sacerdote
de Deus? E Paulo disse: Não sabia, irmãos, que era o sumo
sacerdote; porque está escrito: Não dirás mal do príncipe do teu
povo.

E Paulo, sabendo que uma parte era de saduceus e outra de
fariseus, clamou no conselho: Homens irmãos, eu sou fariseu, filho
de fariseu; no tocante à esperança e ressurreição dos mortos sou
julgado. E, havendo dito isto, houve dissensão entre os fariseus
e saduceus; e a multidão se dividiu. Porque os saduceus dizem
que não há ressurreição, nem anjo, nem espírito; mas os fariseus
reconhecem uma e outra coisa. E originou-se um grande clamor; e,
levantando-se os escribas da parte dos fariseus, contendiam,
dizendo: Nenhum mal achamos neste homem, e, se algum espírito ou
anjo lhe falou, não lutemos contra Deus. E, havendo grande
dissensão, o tribuno, temendo que Paulo fosse despedaçado por eles,
mandou descer a soldadesca\footnote{A classe militar; a gente de
guerra; a tropa. Bando de soldados indisciplinados.}, para que o
tirassem do meio deles, e o levassem para a fortaleza. E na
noite seguinte, apresentando-se-lhe o Senhor, disse: Paulo, tem
ânimo; porque, como de mim testificaste em Jerusalém, assim importa
que testifiques também em Roma.

E, quando já era dia, alguns dos judeus fizeram uma conspiração,
e juraram, dizendo que não comeriam nem beberiam enquanto não
matassem a Paulo. E eram mais de quarenta os que fizeram esta
conjuração. E estes foram ter com os principais dos
sacerdotes e anciãos, e disseram: Conjuramo-nos, sob pena de
maldição, a nada provarmos até que matemos a Paulo. Agora,
pois, vós, com o conselho, rogai ao tribuno que vo-lo traga amanhã,
como que querendo saber mais alguma coisa de seus negócios, e, antes
que chegue, estaremos prontos para o matar. E o filho da irmã
de Paulo, tendo ouvido acerca desta cilada, foi, e entrou na
fortaleza, e o anunciou a Paulo. E Paulo, chamando a si um
dos centuriões, disse: Leva este jovem ao tribuno, porque tem alguma
coisa que lhe comunicar. Tomando-o ele, pois, o levou ao
tribuno, e disse: O preso Paulo, chamando-me a si, rogou-me que
trouxesse este jovem, que tem alguma coisa para dizer-te. E o
tribuno, tomando-o pela mão, e pondo-se à parte, perguntou-lhe em
particular: Que tens que me contar? E disse ele: Os judeus se
concertaram rogar-te que amanhã leves Paulo ao conselho, como que
tendo de inquirir dele mais alguma coisa ao certo. Mas tu não
os creias; porque mais de quarenta homens de entre eles lhe andam
armando ciladas; os quais se obrigaram, sob pena de maldição, a não
comer nem beber até que o tenham morto; e já estão apercebidos,
esperando de ti promessa. Então o tribuno despediu o jovem,
mandando-lhe que a ninguém dissesse que lhe havia contado aquilo.
E, chamando dois centuriões, lhes disse: Aprontai para as
três horas da noite duzentos soldados, e setenta de cavalaria, e
duzentos archeiros\footnote{Guarda do paço; alabardeiro (alabarda =
arma antiga, constituída de uma longa haste de madeira rematada em
ferro largo e pontiagudo, atravessado por outro em forma de
meia-lua).} para irem até Cesaréia; e aparelhai cavalgaduras,
para que, pondo nelas a Paulo, o levem salvo ao governador
Félix\footnote{SBTB: presidente. KJ: Felix the governor. RA e RC:
governador.}. E escreveu uma carta, que continha isto:
Cláudio Lísias, a Félix, potentíssimo
governador\footnote{SBTB: presidente. KJ, RA e RC: governador.},
saúde. Esse homem foi preso pelos judeus; e, estando já a
ponto de ser morto por eles, sobrevim eu com a soldadesca, e o
livrei, informado de que era romano. E, querendo saber a
causa por que o acusavam, o levei ao seu conselho. E achei
que o acusavam de algumas questões da sua lei; mas que nenhum crime
havia nele digno de morte ou de prisão. E, sendo-me
notificado que os judeus haviam de armar ciladas a esse homem, logo
to enviei, mandando também aos acusadores que perante ti digam o que
tiverem contra ele. Passa bem. Tomando, pois, os soldados a
Paulo, como lhe fora mandado, o trouxeram de noite a Antipátride.
E no dia seguinte, deixando aos de cavalo irem com ele,
tornaram à fortaleza. Os quais, logo que chegaram a Cesaréia,
e entregaram a carta ao governador\footnote{SBTB: presidente.}, lhe
apresentaram Paulo. E o governador\footnote{SBTB:
presidente.}, lida a carta, perguntou de que província era; e,
sabendo que era da Cilícia, disse: Ouvir-te-ei, quando também
aqui vierem os teus acusadores. E mandou que o guardassem no
pretório de Herodes.

\medskip

\lettrine{24} E, cinco dias depois, o sumo sacerdote Ananias
desceu com os anciãos, e um certo Tértulo, orador, os quais
compareceram perante o governador\footnote{SBTB: presidente.} contra
Paulo. E, sendo chamado, Tértulo começou a acusá-lo, dizendo:
Visto como por ti temos tanta paz e por tua prudência se fazem a
este povo muitos e louváveis serviços, sempre e em todo o lugar,
ó potentíssimo Félix, com todo o agradecimento o queremos
reconhecer. Mas, para que não te detenha muito, rogo-te que,
conforme a tua eqüidade, nos ouças por pouco tempo. Temos achado
que este homem é uma peste, e promotor de sedições entre todos os
judeus, por todo o mundo; e o principal defensor da seita dos
nazarenos; o qual intentou também profanar o templo; e nós o
prendemos, e conforme a nossa lei o quisemos julgar. Mas,
sobrevindo o tribuno Lísias, no-lo tirou de entre as mãos com grande
violência, mandando aos seus acusadores que viessem a ti; e dele
tu mesmo, examinando-o, poderás entender tudo o de que o acusamos.
E também os judeus o acusavam, dizendo serem estas coisas assim.

Paulo, porém, fazendo-lhe o governador\footnote{SBTB:
presidente.} sinal que falasse, respondeu: Porque sei que já vai
para muitos anos que desta nação és juiz, com tanto melhor ânimo
respondo por mim. Pois bem podes saber que não há mais de
doze dias que subi a Jerusalém a adorar; e não me acharam no
templo falando com alguém, nem amotinando o povo nas sinagogas, nem
na cidade. Nem tampouco podem provar as coisas de que agora
me acusam. Mas confesso-te isto que, conforme aquele caminho
que chamam seita, assim sirvo ao Deus de nossos pais, crendo tudo
quanto está escrito na lei e nos profetas. Tendo esperança em
Deus, como estes mesmos também esperam, de que há de haver
ressurreição de mortos, assim dos justos como dos injustos. E
por isso procuro sempre ter uma consciência sem ofensa, tanto para
com Deus como para com os homens. Ora, muitos anos depois,
vim trazer à minha nação esmolas e ofertas. Nisto me acharam
já santificado no templo, não em ajuntamentos, nem com alvoroços,
uns certos judeus da Ásia, os quais convinha que estivessem
presentes perante ti, e me acusassem, se alguma coisa contra mim
tivessem. Ou digam estes mesmos, se acharam em mim alguma
iniqüidade, quando compareci perante o conselho, a não ser
estas palavras que, estando entre eles, clamei: Hoje sou julgado por
vós acerca da ressurreição dos mortos.

Então Félix, havendo ouvido estas coisas, lhes pôs
dilação\footnote{Adiamento, prorrogação; demora, tardança, delonga;
prazo.}, dizendo: Havendo-me informado melhor deste Caminho, quando
o tribuno Lísias tiver descido, então tomarei inteiro conhecimento
dos vossos negócios. E mandou ao centurião que o guardasse em
prisão, tratando-o com brandura, e que a ninguém dos seus proibisse
servi-lo ou vir ter com ele. E alguns dias depois, vindo
Félix com sua mulher Drusila, que era judia, mandou chamar a Paulo,
e ouviu-o acerca da fé em Cristo. E, tratando ele da justiça,
e da temperança, e do juízo vindouro, Félix, espavorido, respondeu:
Por agora vai-te, e em tendo oportunidade te chamarei.
Esperando ao mesmo tempo que Paulo lhe desse dinheiro, para
que o soltasse; pelo que também muitas vezes o mandava chamar, e
falava com ele. Mas, passados dois anos, Félix teve por
sucessor a Pórcio Festo; e, querendo Félix comprazer aos judeus,
deixou a Paulo preso.

\medskip

\lettrine{25} Entrando, pois, Festo na província, subiu dali a
três dias de Cesaréia a Jerusalém. E o sumo sacerdote e os
principais dos judeus compareceram perante ele contra Paulo, e lhe
rogaram, pedindo como favor contra ele que o fizesse vir a
Jerusalém, armando ciladas para o matarem no caminho. Mas Festo
respondeu que Paulo estava guardado em Cesaréia, e que ele
brevemente partiria para lá. Os que, pois, disse, dentre vós,
têm poder, desçam comigo e, se neste homem houver algum crime,
acusem-no. E, havendo-se demorado entre eles mais de dez dias,
desceu a Cesaréia; e no dia seguinte, assentando-se no tribunal,
mandou que trouxessem Paulo. E, chegando ele, rodearam-no os
judeus que haviam descido de Jerusalém, trazendo contra Paulo muitas
e graves acusações, que não podiam provar. Mas ele, em sua
defesa, disse: Eu não pequei em coisa alguma contra a lei dos
judeus, nem contra o templo, nem contra César. Todavia Festo,
querendo comprazer aos judeus, respondendo a Paulo, disse: Queres tu
subir a Jerusalém, e ser lá perante mim julgado acerca destas
coisas? Mas Paulo disse: Estou perante o tribunal de César,
onde convém que seja julgado; não fiz agravo algum aos judeus, como
tu muito bem sabes. Se fiz algum agravo, ou cometi alguma
coisa digna de morte, não recuso morrer; mas, se nada há das coisas
de que estes me acusam, ninguém me pode entregar a eles; apelo para
César. Então Festo, tendo falado com o conselho, respondeu:
Apelaste para César? para César irás.

E, passados alguns dias, o rei Agripa e Berenice vieram a
Cesaréia, a saudar Festo. E, como ali ficassem muitos dias,
Festo contou ao rei os negócios de Paulo, dizendo: Um certo homem
foi deixado por Félix aqui preso, por cujo respeito os
principais dos sacerdotes e os anciãos dos judeus, estando eu em
Jerusalém, compareceram perante mim, pedindo sentença contra ele.
Aos quais respondi não ser costume dos romanos entregar algum
homem à morte, sem que o acusado tenha presentes os seus acusadores,
e possa defender-se da acusação. De sorte que, chegando eles
aqui juntos, no dia seguinte, sem fazer dilação alguma, assentado no
tribunal, mandei que trouxessem o homem. Acerca do qual,
estando presentes os acusadores, nenhuma coisa apontaram daquelas
que eu suspeitava. Tinham, porém, contra ele algumas questões
acerca da sua superstição, e de um tal Jesus, morto, que Paulo
afirmava viver. E, estando eu perplexo acerca da inquirição
desta causa, disse se queria ir a Jerusalém, e lá ser julgado acerca
destas coisas. E, apelando Paulo para que fosse reservado ao
conhecimento de Augusto, mandei que o guardassem até que o envie a
César. Então Agripa disse a Festo: Bem quisera eu também
ouvir esse homem. E ele disse: Amanhã o ouvirás. E, no dia
seguinte, vindo Agripa e Berenice, com muito aparato, entraram no
auditório com os tribunos e homens principais da cidade, sendo
trazido Paulo por mandado de Festo. E Festo disse: Rei
Agripa, e todos os senhores que estais presentes conosco; aqui vedes
um homem de quem toda a multidão dos judeus me tem falado, tanto em
Jerusalém como aqui, clamando que não convém que viva mais.
Mas, achando eu que nenhuma coisa digna de morte fizera, e
apelando ele mesmo também para Augusto, tenho determinado
enviar-lho. Do qual não tenho coisa alguma certa que escreva
ao meu senhor, e por isso perante vós o trouxe, principalmente
perante ti, ó rei Agripa, para que, depois de interrogado, tenha
alguma coisa que escrever. Porque me parece contra a razão
enviar um preso, e não notificar contra ele as acusações.

\medskip

\lettrine{26} Depois Agripa disse a Paulo: É permitido que te
defendas. Então Paulo, estendendo a mão em sua defesa, respondeu:
Tenho-me por feliz, ó rei Agripa, de que perante ti me haja hoje
de defender de todas as coisas de que sou acusado pelos judeus;
mormente sabendo eu que tens conhecimento de todos os costumes e
questões que há entre os judeus; por isso te rogo que me ouças com
paciência. Quanto à minha vida, desde a mocidade, como decorreu
desde o princípio entre os da minha nação, em Jerusalém, todos os
judeus a conhecem, sabendo de mim desde o princípio (se o
quiserem testificar), que, conforme a mais severa seita da nossa
religião, vivi fariseu. E agora pela esperança da promessa que
por Deus foi feita a nossos pais estou aqui e sou julgado. À
qual as nossas doze tribos esperam chegar, servindo a Deus
continuamente, noite e dia. Por esta esperança, ó rei Agripa, eu sou
acusado pelos judeus. Pois quê? julga-se coisa incrível entre
vós que Deus ressuscite os mortos? Bem tinha eu imaginado que
contra o nome de Jesus Nazareno devia eu praticar muitos atos;
o que também fiz em Jerusalém. E, havendo recebido
autorização dos principais dos sacerdotes, encerrei muitos dos
santos nas prisões; e quando os matavam eu dava o meu voto contra
eles. E, castigando-os muitas vezes por todas as sinagogas,
os obriguei a blasfemar. E, enfurecido demasiadamente contra eles,
até nas cidades estranhas os persegui.

Sobre o que, indo então a Damasco, com poder e comissão dos
principais dos sacerdotes, ao meio-dia, ó rei, vi no caminho
uma luz do céu, que excedia o esplendor do sol, cuja claridade me
envolveu a mim e aos que iam comigo. E, caindo nós todos por
terra, ouvi uma voz que me falava, e em língua hebraica dizia:
Saulo, Saulo, por que me persegues? Dura coisa te é recalcitrar
contra os aguilhões. E disse eu: Quem és, Senhor? E ele
respondeu: Eu sou Jesus, a quem tu persegues; mas levanta-te
e põe-te sobre teus pés, porque te apareci por isto, para te pôr por
ministro e testemunha tanto das coisas que tens visto como daquelas
pelas quais te aparecerei ainda; livrando-te deste povo, e
dos gentios, a quem agora te envio, para lhes abrires os
olhos, e das trevas os converteres à luz, e do poder de Satanás a
Deus; a fim de que recebam a remissão de pecados, e herança entre os
que são santificados pela fé em mim. Por isso, ó rei Agripa,
não fui desobediente à visão celestial. Antes anunciei
primeiramente aos que estão em Damasco e em Jerusalém, e por toda a
terra da Judéia, e aos gentios, que se emendassem e se convertessem
a Deus, fazendo obras dignas de arrependimento. Por causa
disto os judeus lançaram mão de mim no templo, e procuraram
matar-me. Mas, alcançando socorro de Deus, ainda até ao dia
de hoje permaneço dando testemunho tanto a pequenos como a grandes,
não dizendo nada mais do que o que os profetas e Moisés disseram que
devia acontecer, isto é, que o Cristo devia padecer, e sendo
o primeiro da ressurreição dentre os mortos, devia anunciar a luz a
este povo e aos gentios.

E, dizendo ele isto em sua defesa, disse Festo em alta voz: Estás
louco, Paulo; as muitas letras te fazem delirar. Mas ele
disse: Não deliro, ó potentíssimo Festo; antes digo palavras de
verdade e de um são juízo. Porque o rei, diante de quem falo
com ousadia, sabe estas coisas, pois não creio que nada disto lhe é
oculto; porque isto não se fez em qualquer canto. Crês tu nos
profetas, ó rei Agripa? Bem sei que crês. E disse Agripa a
Paulo: Por pouco me queres persuadir a que me faça cristão! E
disse Paulo: Prouvera a Deus que, ou por pouco ou por muito, não
somente tu, mas também todos quantos hoje me estão ouvindo, se
tornassem tais qual eu sou, exceto estas cadeias. E, dizendo
ele isto, levantou-se o rei, o governador\footnote{SBTB:
presidente.}, e Berenice, e os que com eles estavam assentados.
E, apartando-se dali falavam uns com os outros, dizendo: Este
homem nada fez digno de morte ou de prisões. E Agripa disse a
Festo: Bem podia soltar-se este homem, se não houvera apelado para
César.

\medskip

\lettrine{27} E, como se determinou que havíamos de navegar
para a Itália, entregaram Paulo, e alguns outros presos, a um
centurião por nome Júlio, da coorte augusta. E, embarcando nós
em um navio adramitino, partimos navegando pelos lugares da costa da
Ásia, estando conosco Aristarco, macedônio, de Tessalônica. E
chegamos no dia seguinte a Sidom, e Júlio, tratando Paulo
humanamente, lhe permitiu ir ver os amigos, para que cuidassem dele.
E, partindo dali, fomos navegando abaixo de Chipre, porque os
ventos eram contrários. E, tendo atravessado o mar, ao longo da
Cilícia e Panfília, chegamos a Mirra, na Lícia. E, achando ali o
centurião um navio de Alexandria, que navegava para a Itália, nos
fez embarcar nele. E, como por muitos dias navegássemos
vagarosamente, havendo chegado apenas defronte de Cnido, não nos
permitindo o vento ir mais adiante, navegamos abaixo de Creta, junto
de Salmone. E, costeando-a dificilmente, chegamos a um lugar
chamando Bons Portos, perto do qual estava a cidade de Laséia.
E, passado muito tempo, e sendo já perigosa a navegação, pois,
também o jejum já tinha passado, Paulo os admoestava,
dizendo-lhes: Senhores, vejo que a navegação há de ser
incômoda, e com muito dano, não só para o navio e carga, mas também
para as nossas vidas. Mas o centurião cria mais no piloto e
no mestre, do que no que dizia Paulo.

E, como aquele porto não era cômodo para invernar, os mais deles
foram de parecer que se partisse dali para ver se podiam chegar a
Fenice, que é um porto de Creta que olha para o lado do vento da
África e do Coro, e invernar ali. E, soprando o sul
brandamente, lhes pareceu terem já o que desejavam e, fazendo-se de
vela, foram de muito perto costeando Creta. Mas não muito
depois deu nela um pé de vento, chamado Euro-aquilão. E,
sendo o navio arrebatado, e não podendo navegar contra o vento,
dando de mão a tudo, nos deixamos ir à toa. E, correndo
abaixo de uma pequena ilha chamada Clauda, apenas pudemos ganhar o
batel\footnote{Pequeno barco; embarcação miúda, usada nas naus e
galeões.}. E, levado este para cima, usaram de todos os
meios, cingindo o navio; e, temendo darem à costa na
Sirte\footnote{Recife ou banco movediço de areia.}, amainadas as
velas, assim foram à toa. E, andando nós agitados por uma
veemente tempestade, no dia seguinte aliviaram o navio. E ao
terceiro dia nós mesmos, com as nossas próprias mãos, lançamos ao
mar a armação do navio. E, não aparecendo, havia já muitos
dias, nem sol nem estrelas, e caindo sobre nós uma não pequena
tempestade, fugiu-nos toda a esperança de nos salvarmos.

E, havendo já muito que não se comia, então Paulo, pondo-se em pé
no meio deles, disse: Fora, na verdade, razoável, ó senhores, ter-me
ouvido a mim e não partir de Creta, e assim evitariam este incômodo
e esta perda. Mas agora vos admoesto a que tenhais bom ânimo,
porque não se perderá a vida de nenhum de vós, mas somente o navio.
Porque esta mesma noite o anjo de Deus, de quem eu sou, e a
quem sirvo, esteve comigo, dizendo: Paulo, não temas; importa
que sejas apresentado a César, e eis que Deus te deu todos quantos
navegam contigo. Portanto, ó senhores, tende bom ânimo;
porque creio em Deus, que há de acontecer assim como a mim me foi
dito. É contudo necessário irmos dar numa ilha. E,
quando chegou a décima quarta noite, sendo impelidos de um e outro
lado no mar Adriático, lá pela meia-noite suspeitaram os marinheiros
que estavam próximos de alguma terra. E, lançando o prumo,
acharam vinte braças; e, passando um pouco mais adiante, tornando a
lançar o prumo, acharam quinze braças. E, temendo ir dar em
alguns rochedos, lançaram da popa quatro âncoras, desejando que
viesse o dia. Procurando, porém, os marinheiros fugir do
navio, e tendo já deitado o batel ao mar, como que querendo lançar
as âncoras pela proa, disse Paulo ao centurião e aos
soldados: Se estes não ficarem no navio, não podereis salvar-vos.
Então os soldados cortaram os cabos do batel, e o deixaram
cair. E, entretanto que o dia vinha, Paulo exortava a todos a
que comessem alguma coisa, dizendo: É já hoje o décimo quarto dia
que esperais, e permaneceis sem comer, não havendo provado nada.
Portanto, exorto-vos a que comais alguma coisa, pois é para a
vossa saúde; porque nem um cabelo cairá da cabeça de qualquer de
vós. E, havendo dito isto, tomando o pão, deu graças a Deus
na presença de todos; e, partindo-o, começou a comer. E,
tendo já todos bom ânimo, puseram-se também a comer. E éramos
ao todo, no navio, duzentas e setenta e seis almas. E,
refeitos com a comida, aliviaram o navio, lançando o trigo ao mar.
E, sendo já dia, não conheceram a terra; enxergaram, porém,
uma enseada que tinha praia, e consultaram-se sobre se deveriam
encalhar nela o navio. E, levantando as âncoras, deixaram-no
ir ao mar, largando também as amarras do leme; e, alçando a vela
maior ao vento, dirigiram-se para a praia. Dando, porém, num
lugar de dois mares, encalharam ali o navio; e, fixa a proa, ficou
imóvel, mas a popa abria-se com a força das ondas. Então a
idéia dos soldados foi que matassem os presos para que nenhum
fugisse, escapando a nado. Mas o centurião, querendo salvar a
Paulo, lhes estorvou este intento; e mandou que os que pudessem
nadar se lançassem primeiro ao mar, e se salvassem em terra;
e os demais, uns em tábuas e outros em coisas do navio. E
assim aconteceu que todos chegaram à terra a salvo.

\medskip

\lettrine{28} E, havendo escapado, então souberam que a ilha
se chamava Malta. E os bárbaros usaram conosco de não pouca
humanidade; porque, acendendo uma grande fogueira, nos recolheram a
todos por causa da chuva que caía, e por causa do frio. E,
havendo Paulo ajuntado uma quantidade de vides, e pondo-as no fogo,
uma víbora, fugindo do calor, lhe acometeu a mão. E os bárbaros,
vendo-lhe a víbora pendurada na mão, diziam uns aos outros:
Certamente este homem é homicida, visto como, escapando do mar, a
justiça não o deixa viver. Mas, sacudindo ele a víbora no fogo,
não sofreu nenhum mal. E eles esperavam que viesse a inchar ou a
cair morto de repente; mas tendo esperado já muito, e vendo que
nenhum incômodo lhe sobrevinha, mudando de parecer, diziam que era
um deus. E ali, próximo daquele lugar, havia umas herdades que
pertenciam ao principal da ilha, por nome Públio, o qual nos recebeu
e hospedou benignamente por três dias. E aconteceu estar de cama
enfermo de febre e disenteria o pai de Públio, que Paulo foi ver, e,
havendo orado, pôs as mãos sobre ele, e o curou. Feito, pois,
isto, vieram também ter com ele os demais que na ilha tinham
enfermidades, e sararam. Os quais nos distinguiram também com
muitas honras; e, havendo de navegar, nos proveram das coisas
necessárias.

E três meses depois partimos num navio de Alexandria que
invernara na ilha, o qual tinha por insígnia Castor e Pólux.
E, chegando a Siracusa, ficamos ali três dias. De
onde, indo costeando, viemos a Régio; e soprando, um dia depois, um
vento do sul, chegamos no segundo dia a Potéoli. Onde,
achando alguns irmãos, nos rogaram que por sete dias ficássemos com
eles; e depois nos dirigimos a Roma. E de lá, ouvindo os
irmãos novas de nós, nos saíram ao encontro à Praça de Ápio e às
Três Vendas, e Paulo, vendo-os, deu graças a Deus e tomou ânimo.
E, logo que chegamos a Roma, o centurião entregou os presos
ao capitão da guarda; mas a Paulo se lhe permitiu morar por sua
conta à parte, com o soldado que o guardava.

E aconteceu que, três dias depois, Paulo convocou os principais
dos judeus, e, juntos eles, lhes disse: Homens irmãos, não havendo
eu feito nada contra o povo, ou contra os ritos paternos, vim
contudo preso desde Jerusalém, entregue nas mãos dos romanos;
os quais, havendo-me examinado, queriam soltar-me, por não
haver em mim crime algum de morte. Mas, opondo-se os judeus,
foi-me forçoso apelar para César, não tendo, contudo, de que acusar
a minha nação. Por esta causa vos chamei, para vos ver e
falar; porque pela esperança de Israel estou com esta cadeia.
Então eles lhe disseram: Nós não recebemos acerca de ti carta
alguma da Judéia, nem veio aqui algum dos irmãos, que nos anunciasse
ou dissesse de ti mal algum. No entanto bem quiséramos ouvir
de ti o que sentes; porque, quanto a esta seita, notório nos é que
em toda a parte se fala contra ela.

E, havendo-lhe eles assinalado um dia, muitos foram ter com ele à
pousada, aos quais declarava com bom testemunho o reino de Deus, e
procurava persuadi-los à fé em Jesus, tanto pela lei de Moisés como
pelos profetas, desde a manhã até à tarde. E alguns criam no
que se dizia; mas outros não criam. E, como ficaram entre si
discordes, despediram-se, dizendo Paulo esta palavra: Bem falou o
Espírito Santo a nossos pais pelo profeta Isaías, dizendo:
Vai a este povo, e dize: De ouvido ouvireis, e de maneira nenhuma
entendereis; e, vendo vereis, e de maneira nenhuma percebereis.
Porquanto o coração deste povo está endurecido, e com os
ouvidos ouviram pesadamente, e fecharam os olhos, para que nunca com
os olhos vejam, nem com os ouvidos ouçam, nem do coração entendam, e
se convertam, e eu os cure. Seja-vos, pois, notório que esta
salvação de Deus é enviada aos gentios, e eles a ouvirão. E,
havendo ele dito estas palavras, partiram os judeus, tendo entre si
grande contenda.

E Paulo ficou dois anos inteiros na sua própria habitação que
alugara, e recebia todos quantos vinham vê-lo; pregando o
reino de Deus, e ensinando com toda a liberdade as coisas
pertencentes ao Senhor Jesus Cristo, sem impedimento algum.

