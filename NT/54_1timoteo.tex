\thispagestyle{empty}
\chapter*{Primeira Epístola de Paulo a Timóteo}

\lettrine{1} Paulo, apóstolo de Jesus Cristo, segundo o
mandado de Deus, nosso Salvador, e do Senhor Jesus Cristo, esperança
nossa, a Timóteo, meu verdadeiro filho na fé: Graça,
misericórdia e paz da parte de Deus nosso Pai, e da de Cristo Jesus,
nosso Senhor. Como te roguei, quando parti para a Macedônia, que
ficasses em Éfeso, para advertires a alguns, que não ensinem outra
doutrina, nem se dêem a fábulas ou a genealogias intermináveis,
que mais produzem questões do que edificação de Deus, que consiste
na fé; assim o faço agora.

Ora, o fim do mandamento é o amor de um coração puro, e de uma boa
consciência, e de uma fé não fingida. Do que, desviando-se
alguns, se entregaram a vãs contendas; querendo ser mestres da
lei, e não entendendo nem o que dizem nem o que afirmam.
Sabemos, porém, que a lei é boa, se alguém dela usa
legitimamente; sabendo isto, que a lei não é feita para o justo,
mas para os injustos e obstinados, para os ímpios e pecadores, para
os profanos e irreligiosos\footnote{Não religioso. Ateu, ímpio.},
para os parricidas\footnote{Pessoa que matou o pai, mãe ou qualquer
dos ascendentes.} e matricidas, para os homicidas, para os
devassos, para os sodomitas, para os roubadores de homens, para os
mentirosos, para os perjuros\footnote{Que ou aquele que perjura, que
falta à fé jurada.}, e para o que for contrário à sã doutrina,
conforme o evangelho da glória de Deus bem-aventurado, que me
foi confiado.

E dou graças ao que me tem confortado, a Cristo Jesus Senhor
nosso, porque me teve por fiel, pondo-me no ministério; a
mim, que dantes fui blasfemo, e perseguidor, e injurioso; mas
alcancei misericórdia, porque o fiz ignorantemente, na
incredulidade. E a graça de nosso Senhor superabundou com a
fé e amor que há em Jesus Cristo. Esta é uma palavra fiel, e
digna de toda a aceitação, que Cristo Jesus veio ao mundo, para
salvar os pecadores, dos quais eu sou o principal. Mas por
isso alcancei misericórdia, para que em mim, que sou o principal,
Jesus Cristo mostrasse toda a sua longanimidade, para exemplo dos
que haviam de crer nele para a vida eterna. Ora, ao Rei dos
séculos, imortal, invisível, ao único Deus sábio, seja honra e
glória para todo o sempre. Amém.

Este mandamento te dou, meu filho Timóteo, que, segundo as
profecias que houve acerca de ti, milites por elas boa milícia;
conservando a fé, e a boa consciência, a qual alguns,
rejeitando, fizeram naufrágio na fé. E entre esses foram
Himeneu e Alexandre, os quais entreguei a Satanás, para que aprendam
a não blasfemar.

\medskip

\lettrine{2} Admoesto-te, pois, antes de tudo, que se façam
deprecações\footnote{Deprecar: Pedir com instância e submissão;
rogar, suplicar, implorar.}, orações, intercessões, e ações de
graças, por todos os homens; pelos reis, e por todos os que
estão em eminência, para que tenhamos uma vida quieta e sossegada,
em toda a piedade e honestidade; porque isto é bom e agradável
diante de Deus nosso Salvador, que quer que todos os homens se
salvem, e venham ao conhecimento da verdade. Porque há um só
Deus, e um só Mediador entre Deus e os homens, Jesus Cristo homem.
O qual se deu a si mesmo em preço de redenção por todos, para
servir de testemunho a seu tempo. Para o que (digo a verdade em
Cristo, não minto) fui constituído pregador, e apóstolo, e doutor
dos gentios na fé e na verdade. Quero, pois, que os homens orem
em todo o lugar, levantando mãos santas, sem ira nem contenda.

Que do mesmo modo as mulheres se ataviem em traje honesto, com
pudor e modéstia, não com tranças, ou com ouro, ou pérolas, ou
vestidos preciosos, mas (como convém a mulheres que fazem
profissão de servir a Deus) com boas obras. A mulher aprenda
em silêncio, com toda a sujeição. Não permito, porém, que a
mulher ensine, nem use de autoridade sobre o marido, mas que esteja
em silêncio. Porque primeiro foi formado Adão, depois Eva.
E Adão não foi enganado, mas a mulher, sendo enganada, caiu
em transgressão. Salvar-se-á, porém, dando à luz filhos, se
permanecer com modéstia na fé, no amor e na santificação.

\medskip

\lettrine{3} Esta é uma palavra fiel: se alguém deseja o
episcopado, excelente obra deseja. Convém, pois, que o bispo
seja irrepreensível, marido de uma mulher, vigilante, sóbrio,
honesto, hospitaleiro, apto para ensinar; não dado ao vinho, não
espancador, não cobiçoso de torpe ganância, mas moderado, não
contencioso, não avarento; que governe bem a sua própria casa,
tendo seus filhos em sujeição, com toda a modéstia (porque, se
alguém não sabe governar a sua própria casa, terá cuidado da igreja
de Deus?); não neófito, para que, ensoberbecendo-se, não caia na
condenação do diabo. Convém também que tenha bom testemunho dos
que estão de fora, para que não caia em afronta, e no laço do diabo.

Da mesma sorte os diáconos sejam honestos, não de língua dobre,
não dados a muito vinho, não cobiçosos de torpe ganância;
guardando o mistério da fé numa consciência pura. E
também estes sejam primeiro provados, depois sirvam, se forem
irrepreensíveis. Da mesma sorte as esposas sejam honestas,
não maldizentes, sóbrias e fiéis em tudo. Os diáconos sejam
maridos de uma só mulher, e governem bem a seus filhos e suas
próprias casas. Porque os que servirem bem como diáconos,
adquirirão para si uma boa posição e muita confiança na fé que há em
Cristo Jesus.

Escrevo-te estas coisas, esperando ir ver-te bem depressa;
mas, se tardar, para que saibas como convém andar na casa de
Deus, que é a igreja do Deus vivo, a coluna e firmeza da verdade.
E, sem dúvida alguma, grande é o mistério da piedade: Deus se
manifestou em carne, foi justificado no Espírito, visto dos anjos,
pregado aos gentios, crido no mundo, recebido acima na glória.

\medskip

\lettrine{4} Mas o Espírito expressamente diz que nos últimos
tempos apostatarão alguns da fé, dando ouvidos a espíritos
enganadores, e a doutrinas de demônios; pela hipocrisia de
homens que falam mentiras, tendo cauterizada a sua própria
consciência; proibindo o casamento, e ordenando a abstinência
dos alimentos que Deus criou para os fiéis, e para os que conhecem a
verdade, a fim de usarem deles com ações de graças; porque toda
a criatura de Deus é boa, e não há nada que rejeitar, sendo recebido
com ações de graças. Porque pela palavra de Deus e pela oração é
santificada.

Propondo estas coisas aos irmãos, serás bom ministro de Jesus
Cristo, criado com as palavras da fé e da boa doutrina que tens
seguido. Mas rejeita as fábulas profanas e de velhas, e
exercita-te a ti mesmo em piedade; porque o exercício corporal
para pouco aproveita, mas a piedade para tudo é proveitosa, tendo a
promessa da vida presente e da que há de vir. Esta palavra é
fiel e digna de toda a aceitação; porque para isto
trabalhamos e lutamos, pois esperamos no Deus vivo, que é o Salvador
de todos os homens, principalmente dos fiéis. Manda estas
coisas e ensina-as. Ninguém despreze a tua mocidade; mas sê o
exemplo dos fiéis, na palavra, no trato, no amor, no espírito, na
fé, na pureza. Persiste em ler, exortar e ensinar, até que eu
vá. Não desprezes o dom que há em ti, o qual te foi dado por
profecia, com a imposição das mãos do presbitério. Medita
estas coisas; ocupa-te nelas, para que o teu aproveitamento seja
manifesto a todos. Tem cuidado de ti mesmo e da doutrina.
Persevera nestas coisas; porque, fazendo isto, te salvarás, tanto a
ti mesmo como aos que te ouvem.

\medskip

\lettrine{5} Não repreendas asperamente os anciãos, mas
admoesta-os como a pais; aos moços como a irmãos; as mulheres
idosas, como a mães, às moças, como a irmãs, em toda a pureza.

Honra as viúvas que verdadeiramente são viúvas. Mas, se alguma
viúva tiver filhos, ou netos, aprendam primeiro a exercer piedade
para com a sua própria família, e a recompensar seus pais; porque
isto é bom e agradável diante de Deus. Ora, a que é
verdadeiramente viúva e desamparada espera em Deus, e persevera de
noite e de dia em rogos e orações; mas a que vive em deleites,
vivendo está morta. Manda, pois, estas coisas, para que elas
sejam irrepreensíveis. Mas, se alguém não tem cuidado dos seus,
e principalmente dos da sua família, negou a fé, e é pior do que o
infiel. Nunca seja inscrita viúva com menos de sessenta anos, e
só a que tenha sido mulher de um só marido; tendo testemunho
de boas obras: Se criou os filhos, se exercitou hospitalidade, se
lavou os pés aos santos, se socorreu os aflitos, se praticou toda a
boa obra. Mas não admitas as viúvas mais novas, porque,
quando se tornam levianas contra Cristo, querem casar-se;
tendo já a sua condenação por haverem aniquilado a primeira
fé. E, além disto, aprendem também a andar ociosas de casa em
casa; e não só ociosas, mas também paroleiras\footnote{Parolar:
Falar muito; tagarelar. Paroleiro: Que ou aquele que gosta de
parolas ou de estar à parola. Parlapatão, fanfarrão, embusteiro,
mentiroso.} e curiosas, falando o que não convém. Quero,
pois, que as que são moças se casem, gerem filhos, governem a casa,
e não dêem ocasião ao adversário de maldizer; porque já
algumas se desviaram, indo após Satanás. Se algum crente ou
alguma crente tem viúvas, socorra-as, e não se sobrecarregue a
igreja, para que se possam sustentar as que deveras são viúvas.

Os presbíteros que governam bem sejam estimados por dignos de
duplicada honra, principalmente os que trabalham na palavra e na
doutrina; porque diz a Escritura: Não ligarás a boca ao boi
que debulha. E: Digno é o obreiro do seu salário. Não aceites
acusação contra o presbítero, senão com duas ou três testemunhas.
Aos que pecarem, repreende-os na presença de todos, para que
também os outros tenham temor. Conjuro-te diante de Deus, e
do Senhor Jesus Cristo, e dos anjos eleitos, que sem prevenção
guardes estas coisas, nada fazendo por parcialidade. A
ninguém imponhas precipitadamente as mãos, nem participes dos
pecados alheios; conserva-te a ti mesmo puro. Não bebas mais
água só, mas usa de um pouco de vinho, por causa do teu estômago e
das tuas freqüentes enfermidades. Os pecados de alguns homens
são manifestos, precedendo o juízo; e em alguns manifestam-se
depois. Assim mesmo também as boas obras são manifestas, e as
que são de outra maneira não podem ocultar-se.

\medskip

\lettrine{6} Todos os servos que estão debaixo do jugo estimem
a seus senhores por dignos de toda a honra, para que o nome de Deus
e a doutrina não sejam blasfemados. E os que têm senhores
crentes não os desprezem, por serem irmãos; antes os sirvam melhor,
porque eles, que participam do benefício, são crentes e amados. Isto
ensina e exorta. Se alguém ensina alguma outra doutrina, e se
não conforma com as sãs palavras de nosso Senhor Jesus Cristo, e com
a doutrina que é segundo a piedade, é soberbo, e nada sabe, mas
delira acerca de questões e contendas de palavras, das quais nascem
invejas, porfias, blasfêmias, ruins suspeitas, contendas de
homens corruptos de entendimento, e privados da verdade, cuidando
que a piedade seja causa de ganho; aparta-te dos tais.

Mas é grande ganho a piedade com contentamento. Porque nada
trouxemos para este mundo, e manifesto é que nada podemos levar
dele. Tendo, porém, sustento, e com que nos cobrirmos, estejamos
com isso contentes. Mas os que querem ser ricos caem em
tentação, e em laço, e em muitas concupiscências loucas e nocivas,
que submergem os homens na perdição e ruína. Porque o amor ao
dinheiro é a raiz de toda a espécie de males; e nessa cobiça alguns
se desviaram da fé, e se traspassaram a si mesmos com muitas dores.
Mas tu, ó homem de Deus, foge destas coisas, e segue a
justiça, a piedade, a fé, o amor, a paciência, a mansidão.
Milita a boa milícia da fé, toma posse da vida eterna, para a
qual também foste chamado, tendo já feito boa confissão diante de
muitas testemunhas.

Mando-te diante de Deus, que todas as coisas vivifica, e de
Cristo Jesus, que diante de Pôncio Pilatos deu o testemunho de boa
confissão, que guardes este mandamento sem mácula e
repreensão, até à aparição de nosso Senhor Jesus Cristo; a
qual a seu tempo mostrará o bem-aventurado, e único poderoso Senhor,
Rei dos reis e Senhor dos senhores; aquele que tem, ele só, a
imortalidade, e habita na luz inacessível; a quem nenhum dos homens
viu nem pode ver, ao qual seja honra e poder sempiterno\footnote{Que
não teve princípio nem há de ter fim; eterno. Que dura sempre;
perpétuo, contínuo.}. Amém. Manda aos ricos deste mundo que
não sejam altivos, nem ponham a esperança na incerteza das riquezas,
mas em Deus, que abundantemente nos dá todas as coisas para delas
gozarmos; que façam bem, enriqueçam em boas obras, repartam
de boa mente, e sejam comunicáveis; que entesourem para si
mesmos um bom fundamento para o futuro, para que possam alcançar a
vida eterna. Ó Timóteo, guarda o depósito que te foi
confiado, tendo horror aos clamores vãos e profanos e às oposições
da falsamente chamada ciência, a qual professando-a alguns,
se desviaram da fé. A graça seja contigo. Amém.

