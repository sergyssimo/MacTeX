\thispagestyle{empty}
\chapter*{Segunda Epístola de João}

O presbítero à senhora eleita, e a seus filhos, aos quais amo na
verdade, e não somente eu, mas também todos os que têm conhecido a
verdade, por amor da verdade que está em nós, e para sempre
estará conosco: Graça, misericórdia e paz, da parte de Deus Pai
e da do Senhor Jesus Cristo, o Filho do Pai, seja convosco na
verdade e amor. Muito me alegro por achar que alguns de teus
filhos andam na verdade, assim como temos recebido o mandamento do
Pai.

E agora, senhora, rogo-te, não como se escrevesse um novo
mandamento, mas aquele mesmo que desde o princípio tivemos: que nos
amemos uns aos outros. E o amor é este: que andemos segundo os
seus mandamentos. Este é o mandamento, como já desde o princípio
ouvistes, que andeis nele.

Porque já muitos enganadores entraram no mundo, os quais não
confessam que Jesus Cristo veio em carne. Este tal é o enganador e o
anticristo. Olhai por vós mesmos, para que não percamos o que
temos ganho, antes recebamos o inteiro galardão. Todo aquele que
prevarica, e não persevera na doutrina de Cristo, não tem a Deus.
Quem persevera na doutrina de Cristo, esse tem tanto ao Pai como ao
Filho.

Se alguém vem ter convosco, e não traz esta doutrina, não o
recebais em casa, nem tampouco o saudeis. Porque quem o saúda
tem parte nas suas más obras.

Tendo muito que escrever-vos, não quis fazê-lo com papel e tinta;
mas espero ir ter convosco e falar face a face, para que o nosso
gozo seja cumprido. Saúdam-te os filhos de tua irmã, a
eleita. Amém.
