\addchap{O Evangelho de Mateus}
\lettrine{1}\ Livro da geração de Jesus Cristo, filho de Davi,
filho de Abraão. Abraão gerou a Isaque; e Isaque gerou a Jacó; e
Jacó gerou a Judá e a seus irmãos; e Judá gerou, de Tamar, a
Perez e a Zerá; e Perez gerou a Esrom; e Esrom gerou a Arão; e
Arão gerou a Aminadabe; e Aminadabe gerou a Naassom; e Naassom gerou
a Salmom; e Salmom gerou, de Raabe, a Boaz; e Boaz gerou de Rute
a Obede; e Obede gerou a Jessé; e Jessé gerou ao rei Davi; e o
rei Davi gerou a Salomão da que foi mulher de Urias. E Salomão
gerou a Roboão; e Roboão gerou a Abias; e Abias gerou a Asa; e
Asa gerou a Josafá; e Josafá gerou a Jorão; e Jorão gerou a Uzias;
e Uzias gerou a Jotão; e Jotão gerou a Acaz; e Acaz gerou a
Ezequias; e Ezequias gerou a Manassés; e Manassés gerou a
Amom; e Amom gerou a Josias; e Josias gerou a Jeconias e a
seus irmãos na deportação para Babilônia. E, depois da
deportação para a Babilônia, Jeconias gerou a Salatiel; e Salatiel
gerou a Zorobabel; e Zorobabel gerou a Abiúde; e Abiúde gerou
a Eliaquim; e Eliaquim gerou a Azor; e Azor gerou a Sadoque;
e Sadoque gerou a Aquim; e Aquim gerou a Eliúde; e Eliúde
gerou a Eleázar; e Eleázar gerou a Matã; e Matã gerou a Jacó;
e Jacó gerou a José, marido de Maria, da qual nasceu JESUS,
que se chama o Cristo. De sorte que todas as gerações, desde
Abraão até Davi, são catorze gerações; e desde Davi até a deportação
para a Babilônia, catorze gerações; e desde a deportação para a
Babilônia até Cristo, catorze gerações.

Ora, o nascimento de Jesus Cristo foi assim: Estando Maria, sua
mãe, desposada com José, antes de se ajuntarem, achou-se ter
concebido do Espírito Santo. Então José, seu marido, como era
justo, e a não queria infamar, intentou deixá-la secretamente.
E, projetando ele isto, eis que em sonho lhe apareceu um
\textcolor{red}{anjo do Senhor}, dizendo: José, filho de Davi, não temas
receber a Maria, tua mulher, porque o que nela está gerado é do
Espírito Santo; e dará à luz um filho e chamarás o seu nome
\textcolor{red}{JESUS}; porque ele \textcolor{red}{salvará o seu povo dos seus
pecados}. Tudo isto aconteceu para que se cumprisse o que foi
dito da parte do Senhor, pelo profeta, que diz: Eis que a
virgem conceberá, e dará à luz um filho, e chamá-lo-ão pelo nome de
\textcolor{red}{EMANUEL}\footnote{Isaías 7.14.}, que traduzido é: Deus
conosco. E José, despertando do sono, fez como o anjo do
Senhor lhe ordenara, e recebeu a sua mulher; e não a conheceu
até que deu à luz seu filho, o primogênito; e pôs-lhe por nome
Jesus.

\medskip

\lettrine{2}\ E, tendo nascido Jesus em Belém de Judéia, no
tempo do rei Herodes, eis que uns magos vieram do oriente a
Jerusalém, dizendo: Onde está aquele que é nascido rei dos
judeus? porque vimos a sua estrela no oriente, e viemos a adorá-lo.
E o rei Herodes, ouvindo isto, perturbou-se, e toda Jerusalém
com ele. E, congregados todos os príncipes dos sacerdotes, e os
escribas do povo, perguntou-lhes onde havia de nascer o Cristo.
E eles lhe disseram: Em Belém de Judéia; porque assim está
escrito pelo profeta: E tu, Belém, terra de Judá, de modo nenhum
és a menor entre as capitais de Judá; porque de ti sairá o Guia que
há de apascentar o meu povo de Israel\footnote{Miquéias 5.2.}.
Então Herodes, chamando secretamente os magos, inquiriu
exatamente deles acerca do tempo em que a estrela lhes aparecera.
E, enviando-os a Belém, disse: Ide, e perguntai diligentemente
pelo menino e, quando o achardes, participai-mo, para que também eu
vá e o adore.

E, tendo eles ouvido o rei, partiram; e eis que a estrela, que
tinham visto no oriente, ia adiante deles, até que, chegando, se
deteve sobre o lugar onde estava o menino. E, vendo eles a
estrela, regozijaram-se muito com grande alegria. E, entrando
na casa, acharam o menino com Maria sua mãe e, prostrando-se, o
adoraram; e abrindo os seus tesouros, ofertaram-lhe dádivas: ouro,
incenso e mirra. E, sendo por divina revelação avisados em
sonhos para que não voltassem para junto de Herodes, partiram para a
sua terra por outro caminho.

E, tendo eles se retirado, eis que o \textcolor{red}{anjo do Senhor}
apareceu a José em sonhos, dizendo: Levanta-te, e toma o menino e
sua mãe, e foge para o Egito, e demora-te lá até que eu te diga;
porque Herodes há de procurar o menino para o matar. E,
levantando-se ele, tomou o menino e sua mãe, de noite, e foi para o
Egito. E esteve lá, até à morte de Herodes, para que se
cumprisse o que foi dito da parte do Senhor pelo profeta, que diz:
\textcolor{red}{Do Egito chamei o meu Filho}\footnote{Os 11.1}.

Então Herodes, vendo que tinha sido iludido pelos magos,
irritou-se muito, e mandou matar todos os meninos que havia em
Belém, e em todos os seus contornos, de dois anos para baixo,
segundo o tempo que diligentemente inquirira dos magos. Então
se cumpriu o que foi dito pelo profeta Jeremias, que diz: Em
Ramá se ouviu uma voz, lamentação, choro e grande pranto: Raquel
chorando os seus filhos, e não querendo ser consolada, porque já não
existem\footnote{Jr 31.15}.

Morto, porém, Herodes, eis que o \textcolor{red}{anjo do Senhor} apareceu
num sonho a José no Egito, dizendo: Levanta-te, e toma o
menino e sua mãe, e vai para a terra de Israel; porque já estão
mortos os que procuravam a morte do menino. Então ele se
levantou, e tomou o menino e sua mãe, e foi para a terra de Israel.
E, ouvindo que Arquelau reinava na Judéia em lugar de
Herodes, seu pai, receou ir para lá; mas \textcolor{red}{avisado em sonhos},
\textcolor{red}{por divina revelação}, foi para as partes da Galiléia.
E chegou, e habitou numa cidade chamada Nazaré, para que se
cumprisse o que fora dito pelos profetas: \textcolor{red}{Ele será chamado
Nazareno}.

\medskip

\lettrine{3}\ E, naqueles dias, apareceu João o Batista
pregando no deserto da Judéia, e dizendo: Arrependei-vos, porque
é chegado o reino dos céus. Porque este é o anunciado pelo
profeta Isaías, que disse: Voz do que clama no deserto: Preparai o
caminho do Senhor, endireitai as suas veredas\footnote{Is 40.3}.
E este João tinha as suas vestes de pelos de camelo, e um cinto
de couro em torno de seus lombos; e alimentava-se de gafanhotos e de
mel silvestre. Então ia ter com ele Jerusalém, e toda a Judéia,
e toda a província adjacente ao Jordão; e eram por ele batizados
no rio Jordão, confessando os seus pecados.

E, vendo ele muitos dos fariseus e dos saduceus, que vinham ao seu
batismo, dizia-lhes: Raça de víboras, quem vos ensinou a fugir da
ira futura? Produzi, pois, frutos dignos de arrependimento;
e não presumais, de vós mesmos, dizendo: Temos por pai a Abraão;
porque eu vos digo que, mesmo destas pedras, Deus pode suscitar
filhos a Abraão. E também agora está posto o machado à raiz
das árvores; toda a árvore, pois, que não produz bom fruto, é
cortada e lançada no fogo. E eu, em verdade, vos batizo com
água, para o arrependimento; mas aquele que vem após mim é mais
poderoso do que eu; cujas alparcas\footnote{Ou alpercata: sandália
que se prende ao pé por tiras de couro ou de pano.} não sou digno de
levar; \textcolor{red}{ele vos batizará com o Espírito Santo, e com fogo}.
Em sua mão tem a pá, e limpará a sua eira\footnote{Local de
terra batida, cimentado ou lajeado, próprio para debulhar, trilhar,
secar e limpar cereais e legumes }, e recolherá no celeiro o seu
trigo, e queimará a palha com fogo que nunca se apagará.

Então veio Jesus da Galiléia ter com João, junto do Jordão, para
ser batizado por ele. Mas João opunha-se-lhe, dizendo: Eu
careço de ser batizado por ti, e vens tu a mim? Jesus, porém,
respondendo, disse-lhe: Deixa por agora, porque assim nos convém
cumprir toda a justiça. Então ele o permitiu. E, sendo Jesus
batizado, saiu logo da água, e eis que se lhe abriram os céus, e viu
o Espírito de Deus descendo como pomba e vindo sobre ele. E
eis que uma voz dos céus dizia: \textcolor{red}{Este é o meu Filho amado, em
quem me comprazo}.

\medskip

\lettrine{4}\ Então foi conduzido Jesus pelo Espírito ao
deserto, para ser tentado pelo diabo. E, tendo jejuado quarenta
dias e quarenta noites, depois teve fome; e, chegando-se a ele o
tentador, disse: Se tu és o Filho de Deus, manda que estas pedras se
tornem em pães. Ele, porém, respondendo, disse: Está escrito:
Nem só de pão viverá o homem, mas de toda a palavra que sai da boca
de Deus. Então o diabo o transportou à cidade santa, e colocou-o
sobre o pináculo\footnote{O ponto mais alto de um lugar (edifício,
torre etc.). O ponto mais alto de um monte, montanha etc.; cume.
Derivação, sentido figurado: o grau mais alto; auge.} do templo,
e disse-lhe: Se tu és o Filho de Deus, lança-te de aqui abaixo;
porque está escrito: Que aos seus anjos dará ordens a teu respeito,
e tomar-te-ão nas mãos, para que nunca tropeces em alguma pedra.
Disse-lhe Jesus: Também está escrito: Não tentarás o Senhor teu
Deus. Novamente o transportou o diabo a um monte muito alto; e
mostrou-lhe todos os reinos do mundo, e a glória deles. E
disse-lhe: Tudo isto te darei se, prostrado, me adorares.
Então disse-lhe Jesus: Vai-te, Satanás, porque está escrito:
Ao Senhor teu Deus adorarás, e só a ele servirás. Então o
diabo o deixou; e, eis que chegaram os anjos, e o serviam.

Jesus, porém, ouvindo que João estava preso, voltou para a
Galiléia; e, deixando Nazaré, foi habitar em Cafarnaum,
cidade marítima, nos confins de Zebulom e Naftali; para que
se cumprisse o que foi dito pelo profeta Isaías, que diz: A
terra de Zebulom, e a terra de Naftali, junto ao caminho do mar,
além do Jordão, a Galiléia das nações; o povo, que estava
assentado em trevas, viu uma grande luz; e, aos que estavam
assentados na região e sombra da morte, a luz raiou\footnote{Is 9.
1,2.}. Desde então começou Jesus a pregar, e a dizer:
\textcolor{red}{Arrependei-vos, porque é chegado o reino dos céus}.

E Jesus, andando junto ao mar da Galiléia, viu a dois irmãos,
Simão, chamado Pedro, e André, os quais lançavam as redes ao mar,
porque eram pescadores; e disse-lhes: Vinde após mim, e eu
vos farei pescadores de homens. Então eles, deixando logo as
redes, seguiram-no. E, adiantando-se dali, viu outros dois
irmãos, Tiago, filho de Zebedeu, e João, seu irmão, num barco com
seu pai, Zebedeu, consertando as redes; e chamou-os; eles,
deixando imediatamente o barco e seu pai, seguiram-no.

E percorria Jesus toda a Galiléia, ensinando nas suas sinagogas e
pregando o evangelho do reino, e curando todas as enfermidades e
moléstias entre o povo. E a sua fama correu por toda a Síria,
e traziam-lhe todos os que padeciam, acometidos de várias
enfermidades e tormentos, os endemoninhados, os
lunáticos\footnote{Diz-se de indivíduo de humor inconstante, ou que
é dado a divagações, que vive no mundo da Lua. Que ou aquele que
procede de maneira incoerente, com excesso de excentricidade;
maluco.}, e os paralíticos, e ele os curava. E seguia-o uma
grande multidão da Galiléia, de Decápolis, de Jerusalém, da Judéia,
e de além do Jordão.

\medskip

\lettrine{5}\ E Jesus, vendo a multidão, subiu a um monte, e,
assentando-se, aproximaram-se dele os seus discípulos; e,
abrindo a sua boca, os ensinava, dizendo:

Bem-aventurados os pobres de espírito, porque deles é o reino dos
céus; bem-aventurados os que choram, porque eles serão
consolados; bem-aventurados os mansos, porque eles herdarão a
terra; bem-aventurados os que têm fome e sede de justiça, porque
eles serão fartos; bem-aventurados os misericordiosos, porque
eles alcançarão misericórdia; bem-aventurados os limpos de
coração, porque eles verão a Deus; bem-aventurados os
pacificadores, porque eles serão chamados filhos de Deus;
bem-aventurados os que sofrem perseguição por causa da
justiça, porque deles é o reino dos céus; bem-aventurados
sois vós, quando vos injuriarem e perseguirem e, mentindo, disserem
todo o mal contra vós por minha causa. Exultai e alegrai-vos,
porque é grande o vosso galardão nos céus; porque assim perseguiram
os profetas que foram antes de vós.

Vós sois o sal da terra; e se o sal for insípido, com que se há
de salgar? Para nada mais presta senão para se lançar fora, e ser
pisado pelos homens. Vós sois a luz do mundo; não se pode
esconder uma cidade edificada sobre um monte; nem se acende a
candeia e se coloca debaixo do alqueire\footnote{Antiga medida de
capacidade us. sobretudo para cereais, mas de volume variável (na
região de Lisboa equivalia a 13,8 litros). Brasil: unidade de medida
de superfície agrária. Área de terreno que comporta um alqueire de
semeadura (ger. 100 braças de 2,20m lineares). Recipiente quadrado,
ger. de madeira e com duas asas, utilizado para medir um alqueire de
cereais Regionalismo (Portugal): o peso representado pelo conteúdo
desse recipiente; antiga medida para líquidos equivalente a seis
canadas (cerca de 8 litros). Em algumas províncias portuguesas,
antiga medida para líquidos equivalente a meio almude. Regionalismo
(Portugal): antiga medida de superfície correspondente a 15.625
palmos quadrados.  KJ: Neither do men light a candle, and put it
under a bushel, but on a candlestick; and it giveth light unto all
that are in the house.}, mas no velador\footnote{Utensílio formado
de uma haste de madeira apoiada numa base, tendo na parte superior
uma espécie de disco onde se coloca um candeeiro ou uma vela.}, e dá
luz a todos que estão na casa. Assim resplandeça a vossa luz
diante dos homens, para que vejam as vossas boas obras e glorifiquem
a vosso Pai, que está nos céus.

Não cuideis que vim destruir a lei ou os profetas: não vim
ab-rogar\footnote{Fazer cessar a existência ou a obrigatoriedade de
uma lei, em sua totalidade. Fazer cair em desuso (hábito, costume
etc.); suprimir.}, mas cumprir. Porque em verdade vos digo
que, até que o céu e a terra passem, nem um jota ou um til se
omitirá da lei, sem que tudo seja cumprido. Qualquer, pois,
que violar um destes mandamentos, por menor que seja, e assim
ensinar aos homens, será chamado o menor no reino dos céus; aquele,
porém, que os cumprir e ensinar será chamado grande no reino dos
céus. Porque vos digo que, se a vossa justiça não exceder a
dos escribas e fariseus, de modo nenhum entrareis no reino dos céus.

Ouvistes que foi dito aos antigos: Não matarás; mas qualquer que
matar será réu de juízo. Eu, porém, vos digo que qualquer
que, sem motivo, se encolerizar contra seu irmão, será réu de juízo;
e qualquer que disser a seu irmão: Raca\footnote{Uso pejorativo.
Diz-se de ou pessoa sem importância, insignificante, idiota.}, será
réu do sinédrio; e qualquer que lhe disser: Louco, será réu do fogo
do inferno. Portanto, se trouxeres a tua oferta ao altar, e
aí te lembrares de que teu irmão tem alguma coisa contra ti,
deixa ali diante do altar a tua oferta, e vai reconciliar-te
primeiro com teu irmão e, depois, vem e apresenta a tua oferta.
Concilia-te depressa com o teu adversário, enquanto estás no
caminho com ele, para que não aconteça que o adversário te entregue
ao juiz, e o juiz te entregue ao oficial, e te encerrem na prisão.
Em verdade te digo que de maneira nenhuma sairás dali
enquanto não pagares o último lepto\footnote{SBTB: ceitil. Como
veremos, mais dois versículos traduzem por ceitil o nome de duas
moedas utilizadas na época de Jesus: o \emph{lepton} (do grego
\emph{leptos}, ``pequeno, fino''), a única moeda judaica mencionada
no Novo Testamento e o \emph{assarion} (ou, como é traduzido para o
português, asse) romano. Ora, o ceitil era uma antiga moeda
portuguêsa, que valia um sexto do real português de então. Não faz
nenhum sentido usar esse termo. A tradução deve ser literal nestes
casos e não associativa: havia uma moeda, judaica, de bronze, que se
chamava lepto e que correspondia a $1/8$ do \emph{assarion} (moeda
de cobre romana).}.

Ouvistes que foi dito aos antigos: Não cometerás adultério.
Eu, porém, vos digo, que qualquer que atentar numa mulher
para a cobiçar, já em seu coração cometeu adultério com ela.
Portanto, se o teu olho direito te escandalizar, arranca-o e
atira-o para longe de ti; pois te é melhor que se perca um dos teus
membros do que seja todo o teu corpo lançado no inferno. E,
se a tua mão direita te escandalizar, corta-a e atira-a para longe
de ti, porque te é melhor que um dos teus membros se perca do que
seja todo o teu corpo lançado no inferno. Também foi dito:
Qualquer que deixar sua mulher, dê-lhe carta de
divórcio\footnote{SBTB: desquite. A palavra desquite está
completamente desatualizada do contexto atual. Não se usa mais tal
termo. O mais correto aqui seria \texttt{divórcio}. A King James
corrobora essa opinião: ``writing of divorcement''.}. Eu,
porém, vos digo que qualquer que repudiar sua mulher, a não ser por
causa de prostituição, faz que ela cometa adultério, e qualquer que
casar com a repudiada comete adultério.

Outrossim, ouvistes que foi dito aos antigos: Não perjurarás, mas
cumprirás os teus juramentos ao Senhor. Eu, porém, vos digo
que de maneira nenhuma jureis; nem pelo céu, porque é o trono de
Deus; nem pela terra, porque é o escabelo de seus pés; nem
por Jerusalém, porque é a cidade do grande Rei; nem jurarás
pela tua cabeça, porque não podes tornar um cabelo branco ou preto.
Seja, porém, o vosso falar: Sim, sim; Não, não; porque o que
passa disto é de procedência maligna.

Ouvistes que foi dito: Olho por olho, e dente por dente.
Eu, porém, vos digo que não resistais ao mal; mas, se
qualquer te bater na face direita, oferece-lhe também a outra;
e, ao que quiser pleitear contigo, e tirar-te a túnica,
larga-lhe também a capa; e, se qualquer te obrigar a caminhar
uma milha, vai com ele duas. Dá a quem te pedir, e não te
desvies daquele que quiser que lhe emprestes.

Ouvistes que foi dito: Amarás o teu próximo, e odiarás o teu
inimigo. Eu, porém, vos digo: Amai a vossos inimigos,
bendizei os que vos maldizem, fazei bem aos que vos odeiam, e orai
pelos que vos maltratam e vos perseguem; para que sejais filhos do
vosso Pai que está nos céus; porque faz que o seu sol se
levante sobre maus e bons, e a chuva desça sobre justos e injustos.
Pois, se amardes os que vos amam, que galardão tereis? Não
fazem os publicanos também o mesmo? E, se saudardes
unicamente os vossos irmãos, que fazeis de mais? Não fazem os
publicanos também assim? Sede vós pois perfeitos, como é
perfeito o vosso Pai que está nos céus.

\medskip

\lettrine{6}\ Guardai-vos de fazer a vossa esmola diante dos
homens, para serdes vistos por eles; aliás, não tereis galardão
junto de vosso Pai, que está nos céus. Quando, pois, deres
esmola, não faças tocar trombeta diante de ti, como fazem os
hipócritas nas sinagogas e nas ruas, para serem glorificados pelos
homens. Em verdade vos digo que já receberam o seu galardão.
Mas, quando tu deres esmola, não saiba a tua mão esquerda o que
faz a tua direita; para que a tua esmola seja dada em secreto; e
teu Pai, que vê em secreto, ele mesmo te recompensará publicamente.

E, quando orares, não sejas como os hipócritas; pois se comprazem
em orar em pé nas sinagogas, e às esquinas das ruas, para serem
vistos pelos homens. Em verdade vos digo que já receberam o seu
galardão. Mas tu, quando orares, entra no teu aposento e,
fechando a tua porta, ora a teu Pai que está em secreto; e teu Pai,
que vê em secreto, te recompensará publicamente. E, orando, não
useis de vãs repetições, como os gentios, que pensam que por muito
falarem serão ouvidos. Não vos assemelheis, pois, a eles; porque
vosso Pai sabe o que vos é necessário, antes de vós lho pedirdes.

Portanto, vós orareis assim: Pai nosso, que estás nos céus,
santificado seja o teu nome; venha o teu reino, seja feita a
tua vontade, assim na terra como no céu; o pão nosso de cada
dia nos dá hoje; e perdoa-nos as nossas dívidas, assim como
nós perdoamos aos nossos devedores; e não nos
induzas\footnote{O irmão Hélio de Menezes já havia alertado sobre
essa tradução, sugerindo ``conduzas'' em vez de ``induzas''. King
James: ``And \emph{lead us} not into temptation, but deliver us from
evil''.} à tentação; mas livra-nos do mal; porque teu é o reino, e o
poder, e a glória, para sempre. Amém. Porque, se perdoardes
aos homens as suas ofensas, também vosso Pai celestial vos perdoará
a vós; se, porém, não perdoardes aos homens as suas ofensas,
também vosso Pai vos não perdoará as vossas ofensas.

E, quando jejuardes, não vos mostreis contristados como os
hipócritas; porque desfiguram os seus rostos, para que aos homens
pareça que jejuam. Em verdade vos digo que já receberam o seu
galardão. Tu, porém, quando jejuares, unge a tua cabeça, e
lava o teu rosto, para não pareceres aos homens que jejuas,
mas a teu Pai, que está em secreto; e teu Pai, que vê em secreto, te
recompensará publicamente.

Não ajunteis tesouros na terra, onde a traça e a ferrugem tudo
consomem, e onde os ladrões minam e roubam; mas ajuntai
tesouros no céu, onde nem a traça nem a ferrugem consomem, e onde os
ladrões não minam nem roubam. Porque onde estiver o vosso
tesouro, aí estará também o vosso coração. A candeia do corpo
são os olhos; de sorte que, se os teus olhos forem bons, todo o teu
corpo terá luz; se, porém, os teus olhos forem maus, o teu
corpo será tenebroso. Se, portanto, a luz que em ti há são trevas,
quão grandes serão tais trevas! Ninguém pode servir a dois
senhores; porque ou há de odiar um e amar o outro, ou se dedicará a
um e desprezará o outro. Não podeis servir a Deus e a
Mamom\footnote{Essa palavra vem do aramaico, que aparentemente
significa ``riqueza'' ou ``propriedade''. Jesus usou a palavra
personificada a fim de indicar o deus das riquezas carnais em
contraste com o Deus dos céus, que possui as verdadeiras riquezas e
que quer conferi-las a homens que vivam de conformidade com as suas
regras}.

Por isso vos digo: Não andeis cuidadosos quanto à vossa vida,
pelo que haveis de comer ou pelo que haveis de beber; nem quanto ao
vosso corpo, pelo que haveis de vestir. Não é a vida mais do que o
mantimento, e o corpo mais do que o vestuário? Olhai para as
aves do céu, que nem semeiam, nem segam, nem ajuntam em celeiros; e
vosso Pai celestial as alimenta. Não tendes vós muito mais valor do
que elas? E qual de vós poderá, com todos os seus cuidados,
acrescentar um côvado à sua estatura? E, quanto ao vestuário,
por que andais solícitos? Olhai para os lírios do campo, como eles
crescem; não trabalham nem fiam; e eu vos digo que nem mesmo
Salomão, em toda a sua glória, se vestiu como qualquer deles.
Pois, se Deus assim veste a erva do campo, que hoje existe, e
amanhã é lançada no forno, não vos vestirá muito mais a vós, homens
de pouca fé? Não andeis, pois, inquietos, dizendo: Que
comeremos, ou que beberemos, ou com que nos vestiremos?
 (Porque todas estas coisas os gentios procuram). De certo
vosso Pai celestial bem sabe que necessitais de todas estas coisas;
mas, \textcolor{red}{buscai primeiro o reino de Deus, e a sua
justiça, e todas estas coisas vos serão acrescentadas}. Não
vos inquieteis, pois, pelo dia de amanhã, porque o dia de amanhã
cuidará de si mesmo. Basta a cada dia o seu mal.

\medskip

\lettrine{7}\ Não julgueis, para que não sejais julgados.
Porque com o juízo com que julgardes sereis julgados, e com a
medida com que tiverdes medido vos hão de medir a vós. E por que
reparas tu no argueiro\footnote{Partícula pequeníssima, destacada de
qualquer corpo; grânulo, cisco. Coisa mínima, sem qualquer
importância; ninharia, nonada.} que está no olho do teu irmão, e não
vês a trave que está no teu olho? Ou como dirás a teu irmão:
Deixa-me tirar o argueiro do teu olho, estando uma trave no teu?
Hipócrita, tira primeiro a trave do teu olho, e então cuidarás
em tirar o argueiro do olho do teu irmão. Não deis aos cães as
coisas santas, nem deiteis aos porcos as vossas pérolas, não
aconteça que as pisem com os pés e, voltando-se, vos despedacem.

Pedi, e dar-se-vos-á; buscai, e encontrareis; batei, e
abrir-se-vos-á. Porque, aquele que pede, recebe; e, o que busca,
encontra; e, ao que bate, abrir-se-lhe-á. E qual de entre vós é
o homem que, pedindo-lhe pão o seu filho, lhe dará uma pedra?
E, pedindo-lhe peixe, lhe dará uma serpente? Se vós,
pois, sendo maus, sabeis dar boas coisas aos vossos filhos, quanto
mais vosso Pai, que está nos céus, dará bens aos que lhe pedirem?

Portanto, tudo o que vós quereis que os homens vos façam,
fazei-lho também vós, porque \textcolor{red}{esta é a lei e os profetas}.
Entrai pela porta estreita; porque larga é a porta, e
espaçoso o caminho que conduz à perdição, e muitos são os que entram
por ela; e porque estreita é a porta, e apertado o caminho
que leva à vida, e poucos há que a encontrem.

A\-cau\-te\-lai-vos, porém, dos falsos profetas, que vêm até vós
vestidos como ovelhas, mas, interiormente, são lobos devoradores.
Por seus frutos os conhecereis. Porventura colhem-se uvas dos
espinheiros, ou figos dos abrolhos? Assim, toda a árvore boa
produz bons frutos, e toda a árvore má produz frutos maus.
Não pode a árvore boa dar maus frutos; nem a árvore má dar
frutos bons. Toda a árvore que não dá bom fruto corta-se e
lança-se no fogo. Portanto, pelos seus frutos os conhecereis.

Nem todo o que me diz: Senhor, Senhor! entrará no reino dos céus,
mas aquele que faz a vontade de meu Pai, que está nos céus.
Muitos me dirão naquele dia: Senhor, Senhor, não profetizamos
nós em teu nome? E em teu nome não expulsamos demônios? E em teu
nome não fizemos muitas maravilhas? E então lhes direi
abertamente: Nunca vos conheci; apartai-vos de mim, vós que
praticais a iniqüidade. Todo aquele, pois, que escuta estas
minhas palavras, e as pratica, assemelhá-lo-ei ao homem prudente,
que edificou a sua casa sobre a rocha; e desceu a chuva, e
correram rios, e assopraram ventos, e combateram aquela casa, e não
caiu, porque estava edificada sobre a rocha. E aquele que
ouve estas minhas palavras, e não as cumpre, compará-lo-ei ao homem
insensato, que edificou a sua casa sobre a areia; e desceu a
chuva, e correram rios, e assopraram ventos, e combateram aquela
casa, e caiu, e foi grande a sua queda. E aconteceu que,
concluindo Jesus este discurso, a multidão se admirou da sua
doutrina; porquanto os ensinava como tendo autoridade; e não
como os escribas.

\medskip

\lettrine{8}\ E, descendo ele do monte, seguiu-o uma grande
multidão. E, eis que veio um leproso, e o adorou, dizendo:
Senhor, se quiseres, podes tornar-me limpo. E Jesus, estendendo
a mão, tocou-o, dizendo: Quero; sê limpo. E logo ficou purificado da
lepra. Disse-lhe então Jesus: Olha, não o digas a alguém, mas
vai, mostra-te ao sacerdote, e apresenta a oferta que Moisés
determinou, para lhes servir de testemunho.

E, entrando Jesus em Cafarnaum, chegou junto dele um centurião,
rogando-lhe, e dizendo: Senhor, o meu criado jaz em casa,
paralítico, e violentamente atormentado. E Jesus lhe disse: Eu
irei, e lhe darei saúde. E o centurião, respondendo, disse:
Senhor, não sou digno de que entres debaixo do meu telhado, mas dize
somente uma palavra, e o meu criado há de sarar. Pois também eu
sou homem sob autoridade, e tenho soldados às minhas ordens; e digo
a este: Vai, e ele vai; e a outro: Vem, e ele vem; e ao meu criado:
Faze isto, e ele o faz. E maravilhou-se Jesus, ouvindo isto,
e disse aos que o seguiam: Em verdade vos digo que nem mesmo em
Israel encontrei tanta fé. Mas eu vos digo que muitos virão
do oriente e do ocidente, e assentar-se-ão à mesa com Abraão, e
Isaque, e Jacó, no reino dos céus; e os filhos do reino serão
lançados nas trevas exteriores; ali haverá pranto e ranger de
dentes. Então disse Jesus ao centurião: Vai, e como creste te
seja feito. E naquela mesma hora o seu criado sarou.

E Jesus, entrando em casa de Pedro, viu a sogra deste acamada, e
com febre. E tocou-lhe na mão, e a febre a deixou; e
levantou-se, e serviu-os. E, chegada a tarde, trouxeram-lhe
muitos endemoninhados, e ele com a sua palavra expulsou deles os
espíritos, e curou todos os que estavam enfermos; para que se
cumprisse o que fora dito pelo profeta Isaías, que diz: \textcolor{red}{Ele
tomou sobre si as nossas enfermidades, e levou as nossas
doenças}\footnote{Is 53.4.}.

E Jesus, vendo em torno de si uma grande multidão, ordenou que
passassem para o outro lado; e, aproximando-se dele um
escriba, disse-lhe: Mestre, aonde quer que fores, eu te seguirei.
E disse Jesus: As raposas têm covis, e as aves do céu têm
ninhos, mas o Filho do homem não tem onde reclinar a cabeça.
E outro de seus discípulos lhe disse: Senhor, permite-me que
primeiramente vá sepultar meu pai. Jesus, porém, disse-lhe:
Segue-me, e deixa os mortos sepultar os seus mortos.

E, entrando ele no barco, seus discípulos o seguiram; e
eis que no mar se levantou uma tempestade, tão grande que o barco
era coberto pelas ondas; ele, porém, estava dormindo. E os
seus discípulos, aproximando-se, o despertaram, dizendo: Senhor,
salva-nos! que perecemos. E ele disse-lhes: Por que temeis,
homens de pouca fé? Então, levantando-se, repreendeu os ventos e o
mar, e seguiu-se uma grande bonança. E aqueles homens se
maravilharam, dizendo: Que homem é este, que até os ventos e o mar
lhe obedecem?

E, tendo chegado ao outro lado, à província dos gadarenos,
saíram-lhe ao encontro dois endemoninhados, vindos dos sepulcros;
tão ferozes eram que ninguém podia passar por aquele caminho.
E eis que clamaram, dizendo: Que temos nós contigo, Jesus,
Filho de Deus? Vieste aqui atormentar-nos antes do tempo? E
andava pastando distante deles uma manada de muitos porcos. E
os demônios rogaram-lhe, dizendo: Se nos expulsas, permite-nos que
entremos naquela manada de porcos. E ele lhes disse: Ide. E,
saindo eles, se introduziram na manada dos porcos; e eis que toda
aquela manada de porcos se precipitou no mar por um despenhadeiro, e
morreram nas águas. Os porqueiros fugiram e, chegando à
cidade, divulgaram tudo o que acontecera aos endemoninhados.
E eis que toda aquela cidade saiu ao encontro de Jesus e,
vendo-o, rogaram-lhe que se retirasse dos seus termos.

\medskip

\lettrine{9}\ E, entrando no barco, passou para o outro lado, e
chegou à sua cidade. E eis que lhe trouxeram um paralítico, deitado
numa cama. E Jesus, vendo a fé deles, disse ao paralítico:
Filho, tem bom ânimo, perdoados te são os teus pecados. E eis
que alguns dos escribas diziam entre si: Ele blasfema. Mas
Jesus, conhecendo os seus pensamentos, disse: Por que pensais mal em
vossos corações? Pois, qual é mais fácil? dizer: Perdoados te
são os teus pecados; ou dizer: Levanta-te e anda? Ora, para que
saibais que o Filho do homem tem na terra autoridade para perdoar
pecados (disse então ao paralítico): Levanta-te, toma a tua cama, e
vai para tua casa. E, levantando-se, foi para sua casa. E a
multidão, vendo isto, maravilhou-se, e glorificou a Deus, que dera
tal poder aos homens.

E Jesus, passando adiante dali, viu assentado na recebedoria um
homem, chamado Mateus, e disse-lhe: Segue-me. E ele, levantando-se,
o seguiu. E aconteceu que, estando ele em casa sentado à
mesa, chegaram muitos publicanos e pecadores, e sen\-ta\-ram-se
juntamente com Jesus e seus discípulos. E os fariseus, vendo
isto, disseram aos seus discípulos: Por que come o vosso Mestre com
os publicanos e pecadores? Jesus, porém, ouvindo, disse-lhes:
Não necessitam de médico os sãos, mas, sim, os doentes. Ide,
porém, e aprendei o que significa: Misericórdia quero, e não
sacrifício\footnote{Os 6.6.}. Porque eu não vim a chamar os justos,
mas os pecadores, ao arrependimento.

Então, chegaram ao pé dele\footnote{KJ: Then came to him the
disciples of John, saying, Why do we and the Pharisees fast oft, but
thy disciples fast not?} os discípulos de João, dizendo: Por que
jejuamos nós e os fariseus muitas vezes, e os teus discípulos não
jejuam? E disse-lhes Jesus: Podem porventura andar tristes os
filhos das bodas, enquanto o esposo está com eles? Dias, porém,
virão, em que lhes será tirado o esposo, e então jejuarão.
Ninguém deita remendo de pano novo em roupa velha, porque
semelhante remendo rompe a roupa, e faz-se maior a rotura.
Nem se deita vinho novo em odres velhos; aliás rompem-se os
odres, e entorna-se o vinho, e os odres estragam-se; mas deita-se
vinho novo em odres novos, e assim ambos se conservam.

Dizendo-lhes ele estas coisas, eis que chegou um chefe, e o
adorou, dizendo: Minha filha faleceu agora mesmo; mas vem, impõe-lhe
a tua mão, e ela viverá. E Jesus, levantando-se, seguiu-o,
ele e os seus discípulos. E eis que uma mulher que havia já
doze anos padecia de um fluxo de sangue, chegando por detrás dele,
tocou a orla de sua roupa; porque dizia consigo: Se eu
tão-somente tocar a sua roupa, ficarei sã. E Jesus,
voltando-se, e vendo-a, disse: Tem ânimo, filha, a tua fé te salvou.
E imediatamente a mulher ficou sã. E Jesus, chegando à casa
daquele chefe, e vendo os instrumentistas, e o povo em alvoroço,
disse-lhes: Retirai-vos, que a menina não está morta, mas
dorme. E riam-se dele. E, logo que o povo foi posto fora,
entrou Jesus, e pegou-lhe na mão, e a menina levantou-se. E
espalhou-se aquela notícia por todo aquele país.

E, partindo Jesus dali, seguiram-no dois cegos, clamando, e
dizendo: Tem compaixão de nós, filho de Davi. E, quando
chegou à casa, os cegos se aproximaram dele; e Jesus disse-lhes:
Credes vós que eu possa fazer isto? Disseram-lhe eles: Sim, Senhor.
Tocou então os olhos deles, dizendo: Seja-vos feito segundo a
vossa fé. E os olhos se lhes abriram. E Jesus ameaçou-os,
dizendo: Olhai que ninguém o saiba. Mas, tendo eles saído,
divulgaram a sua fama por toda aquela terra. E, havendo-se
eles retirado, trouxeram-lhe um homem mudo e endemoninhado.
E, expulso o demônio, falou o mudo; e a multidão se
maravilhou, dizendo: Nunca tal se viu em Israel. Mas os
fariseus diziam: Ele expulsa os demônios pelo príncipe dos demônios.

E percorria Jesus todas as cidades e aldeias, ensinando nas
sinagogas deles, e pregando o evangelho do reino, e curando todas as
enfermidades e moléstias entre o povo. E, vendo as multidões,
teve grande compaixão delas, porque andavam cansadas e desgarradas,
como ovelhas que não têm pastor. Então, disse aos seus
discípulos: A seara é realmente grande, mas poucos os ceifeiros.
Rogai, pois, ao Senhor da seara, que mande ceifeiros para a
sua seara.

\medskip

\lettrine{10}\ E, chamando os seus doze discípulos, deu-lhes
poder sobre os espíritos imundos, para os expulsarem, e para curarem
toda a enfermidade e todo o mal. Ora, os nomes dos doze
apóstolos são estes: O primeiro, Simão, chamado Pedro, e André, seu
irmão; Tiago, filho de Zebedeu, e João, seu irmão; Filipe e
Bartolomeu; Tomé e Mateus, o publicano; Tiago, filho de Alfeu, e
Lebeu, apelidado Tadeu; Simão o Zelote, e Judas Iscariotes,
aquele que o traiu.

Jesus enviou estes doze, e lhes ordenou, dizendo: Não ireis pelo
caminho dos gentios, nem entrareis em cidade de samaritanos; mas
ide antes às ovelhas perdidas da casa de Israel; e, indo,
pregai, dizendo: É chegado o reino dos céus. Curai os enfermos,
limpai os leprosos, ressuscitai os mortos, expulsai os demônios; de
graça recebestes, de graça dai. Não possuais ouro, nem prata,
nem cobre, em vossos cintos, nem alforjes para o caminho, nem
duas túnicas, nem alparcas, nem bordão\footnote{Cajado grosso ou
vara, por vezes arqueada na parte superior, us. como apoio para
tornar mais seguro o andar.}; porque digno é o operário do seu
alimento. E, em qualquer cidade ou aldeia em que entrardes,
procurai saber quem nela seja digno, e hospedai-vos aí, até que vos
retireis. E, quando entrardes nalguma casa, saudai-a;
e, se a casa for digna, desça sobre ela a vossa paz; mas, se
não for digna, torne para vós a vossa paz. E, se ninguém vos
receber, nem escutar as vossas palavras, saindo daquela casa ou
cidade, sacudi o pó dos vossos pés. Em verdade vos digo que,
no dia do juízo, haverá menos rigor para o país de Sodoma e Gomorra
do que para aquela cidade.

Eis que vos envio como ovelhas ao meio de lobos; portanto, sede
prudentes como as serpentes e inofensivos como as pombas.
A\-cau\-te\-lai-vos, porém, dos homens; porque eles vos entregarão
aos sinédrios, e vos açoitarão nas suas sinagogas; e sereis
até conduzidos à presença dos governadores, e dos reis, por causa de
mim, para lhes servir de testemunho a eles, e aos gentios.
Mas, quando vos entregarem, não vos dê cuidado como, ou o que
haveis de falar, porque naquela mesma hora vos será ministrado o que
haveis de dizer. Porque não sois vós quem falará, mas o
Espírito de vosso Pai é que fala em vós. E o irmão entregará
à morte o irmão, e o pai o filho; e os filhos se levantarão contra
os pais, e os matarão. E odiados de todos sereis por causa do
meu nome; mas aquele que perseverar até ao fim será salvo.
Quando pois vos perseguirem nesta cidade, fugi para outra;
porque em verdade vos digo que não acabareis de percorrer as cidades
de Israel sem que venha o Filho do homem. Não é o discípulo
mais do que o mestre, nem o servo mais do que o seu senhor.
Basta ao discípulo ser como seu mestre, e ao servo como seu
senhor. Se chamaram Belzebu ao pai de família, quanto mais aos seus
domésticos? Portanto, não os temais; porque nada há encoberto
que não haja de revelar-se, nem oculto que não haja de saber-se.
O que vos digo em trevas dizei-o em luz; e o que escutais ao
ouvido pregai-o sobre os telhados. E não temais os que matam
o corpo e não podem matar a alma; temei antes aquele que pode fazer
perecer no inferno a alma e o corpo. Não se vendem dois
passarinhos por um asse\footnote{SBTB: ceitil. Neste caso, o grego
registra outra moeda: o \emph{assarion} romano, ou, em nosso
português, asse. Cf. Lc 12.6.}? e nenhum deles cairá em terra sem a
vontade de vosso Pai. E até mesmo os cabelos da vossa cabeça
estão todos contados. Não temais, pois; mais valeis vós do
que muitos passarinhos. Portanto, qualquer que me confessar
diante dos homens, eu o confessarei diante de meu Pai, que está nos
céus. Mas qualquer que me negar diante dos homens, eu o
negarei também diante de meu Pai, que está nos céus. Não
cuideis que vim trazer a paz à terra; não vim trazer paz, mas
espada; porque eu vim pôr em dissensão o homem contra seu
pai, e a filha contra sua mãe, e a nora contra sua sogra; e
assim os inimigos do homem serão os seus familiares. Quem ama
o pai ou a mãe mais do que a mim não é digno de mim; e quem ama o
filho ou a filha mais do que a mim não é digno de mim. E quem
não toma a sua cruz, e não segue após mim, não é digno de mim.
Quem achar a sua vida perdê-la-á; e quem perder a sua vida,
por amor de mim, achá-la-á. Quem vos recebe, a mim me recebe;
e quem me recebe a mim, recebe aquele que me enviou. Quem
recebe um profeta em qualidade de profeta, receberá galardão de
profeta; e quem recebe um justo na qualidade de justo, receberá
galardão de justo. E qualquer que tiver dado só que seja um
copo de água fria a um destes pequenos, em nome de discípulo, em
verdade vos digo que de modo algum perderá o seu galardão.

\medskip

\lettrine{11}\ E aconteceu que, acabando Jesus de dar
instruções aos seus doze discípulos, partiu dali a ensinar e a
pregar nas cidades deles. E João, ouvindo no cárcere falar dos
feitos de Cristo, enviou dois dos seus discípulos, a dizer-lhe:
És tu aquele que havia de vir, ou esperamos outro? E Jesus,
respondendo, disse-lhes: Ide, e anunciai a João as coisas que ouvis
e vedes: Os cegos vêem, e os coxos andam; os leprosos são
limpos, e os surdos ouvem; os mortos são ressuscitados, e aos pobres
é anunciado o evangelho. E bem-aventurado é aquele que não se
escandalizar em mim.

E, partindo eles, começou Jesus a dizer às turbas, a respeito de
João: Que fostes ver no deserto? uma cana agitada pelo vento?
Sim, que fostes ver? um homem ricamente vestido? Os que trajam
ricamente estão nas casas dos reis. Mas, então que fostes ver?
um profeta? Sim, vos digo eu, e muito mais do que profeta;
porque é este de quem está escrito: Eis que diante da tua
face envio o meu anjo\footnote{RA e BJ:``mensageiro''. King James:
``messenger''.}, que preparará diante de ti o teu
caminho\footnote{Ml 3.1.}. Em verdade vos digo que, entre os
que de mulher têm nascido, não apareceu alguém maior do que João o
Batista; mas aquele que é o menor no reino dos céus é maior do que
ele. E, desde os dias de João o Batista até agora, se faz
violência ao reino dos céus, e pela força se apoderam dele.
Porque todos os profetas e a lei profetizaram até João.
E, se quereis dar crédito, é este o Elias que havia de vir.
Quem tem ouvidos para ouvir, ouça.

Mas, a quem assemelharei esta geração? É semelhante aos meninos
que se assentam nas praças, e clamam aos seus companheiros, e
dizem: Tocamo-vos flauta, e não dançastes; cantamo-vos lamentações,
e não chorastes. Porquanto veio João, não comendo nem
bebendo, e dizem: Tem demônio. Veio o Filho do homem, comendo
e bebendo, e dizem: Eis aí um homem comilão e beberrão, amigo dos
publicanos e pecadores. Mas a sabedoria é justificada por seus
filhos. Então começou ele a lançar em rosto às cidades onde
se operou a maior parte dos seus prodígios o não se haverem
arrependido, dizendo: Ai de ti, Corazim! Ai de ti, Betsaida!
Porque, se em Tiro e em Sidom fossem feitos os prodígios que em vós
se fizeram, há muito que se teriam arrependido, com saco e com
cinza. Por isso eu vos digo que haverá menos rigor para Tiro
e Sidom, no dia do juízo, do que para vós. E tu, Cafarnaum,
que te ergues até aos céus, serás abatida até aos infernos; porque,
se em Sodoma tivessem sido feitos os prodígios que em ti se
operaram, teria ela permanecido até hoje. Eu vos digo, porém,
que haverá menos rigor para os de Sodoma, no dia do juízo, do que
para ti.

Naquele tempo, respondendo Jesus, disse: Graças te dou, ó Pai,
Senhor do céu e da terra, que ocultaste estas coisas aos sábios e
entendidos, e as revelaste aos pequeninos. Sim, ó Pai, porque
assim te aprouve. Todas as coisas me foram entregues por meu
Pai, e ninguém conhece o Filho, senão o Pai; e ninguém conhece o
Pai, senão o Filho, e aquele a quem o Filho o quiser revelar.
Vinde a mim, todos os que estais cansados e oprimidos, e eu
vos aliviarei. Tomai sobre vós o meu jugo, e aprendei de mim,
que sou manso e humilde de coração; e encontrareis descanso para as
vossas almas. Porque o meu jugo é suave e o meu fardo é leve.

\medskip

\lettrine{12}\ Naquele tempo passou Jesus pelas searas, em um
sábado; e os seus discípulos, tendo fome, começaram a colher
espigas, e a comer. E os fariseus, vendo isto, disseram-lhe: Eis
que os teus discípulos fazem o que não é lícito fazer num sábado.
Ele, porém, lhes disse: Não tendes lido o que fez Davi, quando
teve fome, ele e os que com ele estavam? Como entrou na casa de
Deus, e comeu os pães da proposição, que não lhe era lícito comer,
nem aos que com ele estavam, mas só aos sacerdotes? Ou não
tendes lido na lei que, aos sábados, os sacerdotes no templo violam
o sábado, e ficam sem culpa? Pois eu vos digo que está aqui quem
é maior do que o templo. Mas, se vós soubésseis o que significa:
Misericórdia quero, e não sacrifício, não condenaríeis os inocentes.
Porque o Filho do homem até do sábado é Senhor. E, partindo
dali, chegou à sinagoga deles. E, estava ali um homem que
tinha uma das mãos mirrada; e eles, para o acusarem, o interrogaram,
dizendo: É lícito curar nos sábados? E ele lhes disse: Qual
dentre vós será o homem que tendo uma ovelha, se num sábado ela cair
numa cova, não lançará mão dela, e a levantará? Pois, quanto
mais vale um homem do que uma ovelha? É, por conseqüência, lícito
fazer bem nos sábados. Então disse àquele homem: Estende a
tua mão. E ele a estendeu, e ficou sã como a outra.

E os fariseus, tendo saído, formaram conselho contra ele, para o
matarem. Jesus, sabendo isso, retirou-se dali, e
a\-com\-pa\-nha\-ram-no grandes multidões, e ele curou a todas. E
recomendava-lhes rigorosamente que o não descobrissem, para
que se cumprisse o que fora dito pelo profeta Isaías\footnote{Is
42.1.}, que diz: Eis aqui o meu servo, que escolhi, o meuem quem a minha alma se compraz; porei sobre ele o meu
espírito, e anunciará aos gentios o juízo. Não contenderá,
nem clamará, nem alguém ouvirá pelas ruas a sua voz; não
esmagará a cana quebrada, e não apagará o morrão\footnote{Pedaço de
corda, ger. de linho, com uma das extremidades embebida em uma
solução de cal virgem e potassa para que se queimasse lentamente, e
que se mantinha acesa durante o combate, para atear fogo à pólvora
dos canhões; murrão. Mecha queimada. A parte queimada mas ainda
acesa de qualquer objeto.} que fumega, até que faça triunfar o
juízo; e no seu nome os gentios esperarão.

Trouxeram-lhe, então, um endemoninhado cego e mudo; e, de tal
modo o curou, que o cego e mudo falava e via. E toda a
multidão se admirava e dizia: Não é este o Filho de Davi? Mas
os fariseus, ouvindo isto, diziam: Este não expulsa os demônios
senão por Belzebu, príncipe dos demônios. Jesus, porém,
conhecendo os seus pensamentos, disse-lhes: Todo o reino dividido
contra si mesmo é devastado; e toda a cidade, ou casa, dividida
contra si mesma não subsistirá. E, se Satanás expulsa a
Satanás, está dividido contra si mesmo; como subsistirá, pois, o seu
reino? E, se eu expulso os demônios por Belzebu, por quem os
expulsam então vossos filhos? Portanto, eles mesmos serão os vossos
juízes. Mas, se eu expulso os demônios pelo Espírito de Deus,
logo é chegado a vós o reino de Deus. Ou, como pode alguém
entrar em casa do homem valente, e furtar os seus bens, se primeiro
não maniatar o valente, saqueando então a sua casa? Quem não
é comigo é contra mim; e quem comigo não ajunta, espalha.
Portanto, eu vos digo: Todo o pecado e blasfêmia se perdoará
aos homens; mas a blasfêmia contra o Espírito não será perdoada aos
homens. E, se qualquer disser alguma palavra contra o Filho
do homem, ser-lhe-á perdoado; mas, se alguém falar contra o Espírito
Santo, não lhe será perdoado, nem neste século nem no futuro.
Ou fazei a árvore boa, e o seu fruto bom, ou fazei a árvore
má, e o seu fruto mau; porque pelo fruto se conhece a árvore.
Raça de víboras, como podeis vós dizer boas coisas, sendo
maus? Pois do que há em abundância no coração, disso fala a boca.
O homem bom tira boas coisas do bom tesouro do seu coração, e
o homem mau do mau tesouro tira coisas más. Mas eu vos digo
que de toda a palavra ociosa\footnote{Que não dá resultados
positivos; improdutivo, improfícuo, estéril. Que não faz falta;
supérfluo, desnecessário, inútil.} que os homens disserem hão de dar
conta no dia do juízo. Porque por tuas palavras serás
justificado, e por tuas palavras serás condenado.

Então alguns dos escribas e dos fariseus tomaram a palavra,
dizendo: Mestre, quiséramos ver da tua parte algum sinal. Mas
ele lhes respondeu, e disse: Uma geração má e adúltera pede um
sinal, porém, não se lhe dará outro sinal senão o do profeta Jonas;
pois, como Jonas esteve três dias e três noites no ventre da
baleia, assim estará o Filho do homem três dias e três noites no
seio da terra. Os ninivitas ressurgirão no juízo com esta
geração, e a condenarão, porque se arrependeram com a pregação de
Jonas. E eis que está aqui quem é mais do que Jonas. A rainha
do sul\footnote{SBTB: rainha do meio-dia. Na King James, temos:
``The queen of the south''. Em grego, temos \emph{balissa tou
notou}; e \emph{notos} quer dizer tanto região do sul como vento
sul.} se levantará no dia do juízo com esta geração, e a condenará;
porque veio dos confins da terra para ouvir a sabedoria de Salomão.
E eis que está aqui quem é maior do que Salomão. E, quando o
espírito imundo tem saído do homem, anda por lugares áridos,
buscando repouso, e não o encontra. Então diz: Voltarei para
a minha casa, de onde saí. E, voltando, acha-a desocupada, varrida e
adornada. Então vai, e leva consigo outros sete espíritos
piores do que ele e, entrando, habitam ali; e são os últimos atos
desse homem piores do que os primeiros. Assim acontecerá também a
esta geração má.

E, falando ele ainda à multidão, eis que estavam fora sua mãe e
seus irmãos, pretendendo falar-lhe. E disse-lhe alguém: Eis
que estão ali fora tua mãe e teus irmãos, que querem falar-te.
Ele, porém, respondendo, disse ao que lhe falara: Quem é
minha mãe? E quem são meus irmãos? E, estendendo a sua mão
para os seus discípulos, disse: Eis aqui minha mãe e meus irmãos;
porque, qualquer que fizer a vontade de meu Pai que está nos
céus, este é meu irmão, e irmã e mãe.

\medskip

\lettrine{13}\ Tendo Jesus saído de casa, naquele dia, estava
assentado junto ao mar; e ajuntou-se muita gente ao pé dele, de
sorte que, entrando num barco, se assentou; e toda a multidão estava
em pé na praia. E falou-lhe de muitas coisas por parábolas,
dizendo: Eis que o semeador saiu a semear. E, quando semeava,
uma parte da semente caiu ao pé do caminho\footnote{KJ: And when he
sowed, some seeds fell by the way side, and the fowls came and
devoured them up. Ed. Contemp.: caiu à beira do caminho.}, e vieram
as aves, e comeram-na; e outra parte caiu em pedregais, onde não
havia terra bastante, e logo nasceu, porque não tinha terra funda;
mas, vindo o sol, queimou-se, e secou-se, porque não tinha raiz.
E outra caiu entre espinhos, e os espinhos cresceram e
sufocaram-na. E outra caiu em boa terra, e deu fruto: um a cem,
outro a sessenta e outro a trinta. Quem tem ouvidos para ouvir,
ouça. E, acercando-se dele os discípulos, disseram-lhe: Por
que lhes falas por parábolas? Ele, respondendo, disse-lhes:
Porque a vós é dado conhecer os mistérios do reino dos céus, mas a
eles não lhes é dado; porque àquele que tem, se dará, e terá
em abundância; mas àquele que não tem, até aquilo que tem lhe será
tirado. Por isso lhes falo por parábolas; porque eles, vendo,
não vêem; e, ouvindo, não ouvem nem compreendem. E neles se
cumpre a profecia de Isaías\footnote{Is 6.9,10.}, que diz: Ouvindo,
ouvireis, mas não compreendereis, e, vendo, vereis, mas não
percebereis. Porque o coração deste povo está endurecido, e
ouviram de mau grado com seus ouvidos, e fecharam seus olhos; para
que não vejam com os olhos, e ouçam com os ouvidos, e compreendam
com o coração, e se convertam, e eu os cure. Mas,
bem-aventurados os vossos olhos, porque vêem, e os vossos ouvidos,
porque ouvem. Porque em verdade vos digo que muitos profetas
e justos desejaram ver o que vós vedes, e não o viram; e ouvir o que
vós ouvis, e não o ouviram. Escutai vós, pois, a parábola do
semeador. Ouvindo alguém a palavra do reino, e não a
entendendo, vem o maligno, e arrebata o que foi semeado no seu
coração; este é o que foi semeado ao pé do caminho. O que foi
semeado em pedregais é o que ouve a palavra, e logo a recebe com
alegria; mas não tem raiz em si mesmo, antes é de pouca
duração; e, chegada a angústia e a perseguição, por causa da
palavra, logo se ofende; e o que foi semeado entre espinhos é
o que ouve a palavra, mas os cuidados deste mundo, e a sedução das
riquezas sufocam a palavra, e fica infrutífera; mas, o que
foi semeado em boa terra é o que ouve e compreende a palavra; e dá
fruto, e um produz cem, outro sessenta, e outro trinta.

Propôs-lhes outra parábola, dizendo: O reino dos céus é
semelhante ao homem que semeia a boa semente no seu campo;
mas, dormindo os homens, veio o seu inimigo, e semeou joio no
meio do trigo, e retirou-se. E, quando a erva cresceu e
frutificou, apareceu também o joio. E os servos do pai de
família, indo ter com ele, disseram-lhe: Senhor, não semeaste tu, no
teu campo, boa semente? Por que tem, então, joio? E ele lhes
disse: Um inimigo é quem fez isso. E os servos lhe disseram: Queres
pois que vamos arrancá-lo? Ele, porém, lhes disse: Não; para
que, ao colher o joio, não arranqueis também o trigo com ele.
Deixai crescer ambos juntos até à ceifa; e, por ocasião da
ceifa, direi aos ceifeiros: Colhei primeiro o joio, e atai-o em
molhos para o queimar; mas, o trigo, ajuntai-o no meu celeiro.
Outra parábola lhes propôs, dizendo: O reino dos céus é
semelhante ao grão de mostarda que o homem, pegando nele, semeou no
seu campo; o qual é, realmente, a menor de todas as sementes;
mas, crescendo, é a maior das plantas, e faz-se uma árvore, de sorte
que vêm as aves do céu, e se aninham nos seus ramos. Outra
parábola lhes disse: O reino dos céus é semelhante ao fermento, que
uma mulher toma e introduz em três medidas de farinha, até que tudo
esteja levedado. Tudo isto disse Jesus, por parábolas à
multidão, e nada lhes falava sem parábolas; para que se
cumprisse o que fora dito pelo profeta\footnote{Sl 78.2.}, que
disse: Abrirei em parábolas a minha boca; publicarei coisas ocultas
desde a fundação do mundo. Então, tendo despedido a multidão,
foi Jesus para casa. E chegaram ao pé dele os seus discípulos,
dizendo: Explica-nos a parábola do joio do campo. E ele,
respondendo, disse-lhes: O que semeia a boa semente, é o Filho do
homem; o campo é o mundo; e a boa semente são os filhos do
reino; e o joio são os filhos do maligno; o inimigo, que o
semeou, é o diabo; e a ceifa é o fim do mundo; e os ceifeiros são os
anjos. Assim como o joio é colhido e queimado no fogo, assim
será na consumação deste mundo. Mandará o Filho do homem os
seus anjos, e eles colherão do seu reino tudo o que causa escândalo,
e os que cometem iniqüidade. E lançá-los-ão na fornalha de
fogo; ali haverá pranto e ranger de dentes. Então os justos
resplandecerão como o sol, no reino de seu Pai. Quem tem ouvidos
para ouvir, ouça.

Também o reino dos céus é semelhante a um tesouro escondido num
campo, que um homem achou e escondeu; e, pelo gozo dele, vai, vende
tudo quanto tem, e compra aquele campo. Outrossim, o reino
dos céus é semelhante ao homem, negociante, que busca boas pérolas;
e, encontrando uma pérola de grande valor, foi, vendeu tudo
quanto tinha, e comprou-a. Igualmente o reino dos céus é
semelhante a uma rede lançada ao mar, e que apanha toda a qualidade
de peixes. E, estando cheia, a puxam para a praia; e,
assentando-se, apanham para os cestos os bons; os ruins, porém,
lançam fora. Assim será na consumação dos séculos: virão os
anjos, e separarão os maus de entre os justos, e lançá-los-ão
na fornalha de fogo; ali haverá pranto e ranger de dentes. E
disse-lhes Jesus: Entendestes todas estas coisas? Disseram-lhe eles:
Sim, Senhor. E ele disse-lhes: Por isso, todo o escriba
instruído acerca do reino dos céus é semelhante a um pai de família,
que tira do seu tesouro coisas novas e velhas.

E aconteceu que Jesus, concluindo estas parábolas, se retirou
dali. E, chegando à sua pátria, ensinava-os na sinagoga
deles, de sorte que se maravilhavam, e diziam: De onde veio a este a
sabedoria, e estas maravilhas? Não é este o filho do
carpinteiro? e não se chama sua mãe Maria, e seus irmãos Tiago, e
José, e Simão, e Judas? E não estão entre nós todas as suas
irmãs? De onde lhe veio, pois, tudo isto? E escandalizavam-se
nele. Jesus, porém, lhes disse: Não há profeta sem honra, a não ser
na sua pátria e na sua casa. E não fez ali muitas maravilhas,
por causa da incredulidade deles.

\medskip

\lettrine{14}\ Naquele tempo ouviu Herodes, o
tetrarca\footnote{Chefe ou governador de uma tetrarquia --- cada uma
das quatro partes (províncias ou governos) em que se dividiam alguns
estados.}, a fama de Jesus, e disse aos seus criados: Este é
João o Batista; ressuscitou dos mortos, e por isso estas maravilhas
operam nele. Porque Herodes tinha prendido João, e tinha-o
maniatado e encerrado no cárcere, por causa de Herodias, mulher de
seu irmão Filipe; porque João lhe dissera: Não te é lícito
possuí-la. E, querendo matá-lo, temia o povo; porque o tinham
como profeta. Festejando-se, porém, o dia natalício de Herodes,
dançou a filha de Herodias diante dele, e agradou a Herodes. Por
isso prometeu, com juramento, dar-lhe tudo o que pedisse; e ela,
instruída previamente por sua mãe, disse: Dá-me aqui, num prato, a
cabeça de João o Batista. E o rei afligiu-se, mas, por causa do
juramento, e dos que estavam à mesa com ele, ordenou que se lhe
desse. E mandou degolar João no cárcere. E a sua
cabeça foi trazida num prato, e dada à jovem, e ela a levou a sua
mãe. E chegaram os seus discípulos, e levaram o corpo, e o
sepultaram; e foram anunciá-lo a Jesus.

E Jesus, ouvindo isto, retirou-se dali num barco, para um lugar
deserto, apartado; e, sa\-ben\-do-o o povo, seguiu-o a pé desde as
cidades. E, Jesus, saindo, viu uma grande multidão, e
possuído de íntima compaixão para com ela, curou os seus enfermos.
E, sendo chegada a tarde, os seus discípulos aproximaram-se
dele, dizendo: O lugar é deserto, e a hora é já avançada; despede a
multidão, para que vão pelas aldeias, e comprem comida para si.
Jesus, porém, lhes disse: Não é mister\footnote{Necessidade,
urgência.} que vão; dai-lhes vós de comer. Então eles lhe
disseram: Não temos aqui senão cinco pães e dois peixes. E
ele disse: Trazei-mos aqui. E, tendo mandado que a multidão
se assentasse sobre a erva, tomou os cinco pães e os dois peixes, e,
erguendo os olhos ao céu, os abençoou, e, partindo os pães, deu-os
aos discípulos, e os discípulos à multidão. E comeram todos,
e saciaram-se; e levantaram dos pedaços, que sobejaram, doze
alcofas\footnote{Alcofa: Cesto flexível, de vime, de esparto ou de
folha de palma, achatado e com asas.} cheias. E os que
comeram foram quase cinco mil homens, além das mulheres e crianças.

E logo ordenou Jesus que os seus discípulos entrassem no barco, e
fossem adiante para o outro lado, enquanto despedia a multidão.
E, despedida a multidão, subiu ao monte para orar, à parte.
E, chegada já a tarde, estava ali só. E o barco estava já no
meio do mar, açoitado pelas ondas; porque o vento era contrário;
mas, à quarta vigília da noite, dirigiu-se Jesus para eles,
andando por cima do mar. E os discípulos, vendo-o andando
sobre o mar, assustaram-se, dizendo: É um fantasma. E gritaram com
medo. Jesus, porém, lhes falou logo, dizendo: Tende bom
ânimo, sou eu, não temais. E respondeu-lhe Pedro, e disse:
Senhor, se és tu, manda-me ir ter contigo por cima das águas.
E ele disse: Vem. E Pedro, descendo do barco, andou sobre as
águas para ir ter com Jesus. Mas, sentindo o vento forte,
teve medo; e, começando a ir para o fundo, clamou, dizendo: Senhor,
salva-me! E logo Jesus, estendendo a mão, segurou-o, e
disse-lhe: Homem de pouca fé, por que duvidaste? E, quando
subiram para o barco, acalmou o vento. Então aproximaram-se
os que estavam no barco, e \textcolor{red}{adoraram-no}, dizendo: \textcolor{red}{És
verdadeiramente o Filho de Deus}.

E, tendo passado para o outro lado, chegaram à terra de Genesaré.
E, quando os homens daquele lugar o conheceram, mandaram por
todas aquelas terras em redor e trouxeram-lhe todos os que estavam
enfermos. E rogavam-lhe que ao menos eles pudessem tocar a
orla da sua roupa; e todos os que a tocavam ficavam sãos.

\medskip

\lettrine{15}\ Então chegaram ao pé de Jesus\footnote{Creio
aqui caber uma atualização nesta expressão \emph{ao pé de Jesus}.
King James: Then came to Jesus scribes and Pharisees, which were of
Jerusalem, saying, \ldots{} Ed. Contemp.: Então vieram de Jerusalém
a Jesus alguns fariseus e escribas, e perguntaram: \dots{}} uns
escribas e fariseus de Jerusalém, dizendo: Por que transgridem
os teus discípulos a tradição dos anciãos? Pois não lavam as mãos
quando comem pão. Ele, porém, respondendo, disse-lhes: Por que
transgredis vós, também, o mandamento de Deus pela vossa tradição?
Porque Deus ordenou, dizendo: Honra a teu pai e a tua
mãe\footnote{Ex 21.17.}; e: Quem maldisser ao pai ou à mãe,
certamente morrerá\footnote{Lv 20.9.}. Mas vós dizeis: Qualquer
que disser ao pai ou à mãe: É oferta ao Senhor o que poderias
aproveitar de mim; esse não precisa honrar nem a seu pai nem a sua
mãe, e assim invalidastes, pela vossa tradição, o mandamento de
Deus. Hipócritas, bem profetizou Isaías\footnote{Is 29.13.} a
vosso respeito, dizendo: Este povo se aproxima de mim com a sua
boca e me honra com os seus lábios, mas o seu coração está longe de
mim. Mas, em vão me adoram, ensinando doutrinas que são
preceitos dos homens.

E, chamando a si a multidão, disse-lhes: Ouvi, e entendei:

O que contamina o homem não é o que entra na boca, mas o que sai
da boca, isso é o que contamina o homem. Então, acercando-se
dele os seus discípulos, disseram-lhe: Sabes que os fariseus,
ouvindo essas palavras, se escandalizaram? Ele, porém,
respondendo, disse: Toda a planta, que meu Pai celestial não
plantou, será arrancada. Deixai-os; são condutores cegos.
Ora, se um cego guiar outro cego, ambos cairão na cova. E
Pedro, tomando a palavra, disse-lhe: Explica-nos essa parábola.
Jesus, porém, disse: Até vós mesmos estais ainda sem
entender? Ainda não compreendeis que tudo o que entra pela
boca desce para o ventre, e é lançado fora? Mas, \textcolor{red}{o
que sai da boca, procede do coração, e isso contamina o homem}.
\textcolor{red}{Porque do coração procedem os maus pensamentos,
mortes, adultérios, prostituição, furtos, falsos testemunhos e
blasfêmias}. \textcolor{red}{São estas coisas que contaminam o
homem}; mas comer sem lavar as mãos, isso não contamina o homem.

E, partindo Jesus dali, foi para as partes de Tiro e de Sidom.
E eis que uma mulher cananéia, que saíra daquelas cercanias,
clamou, dizendo: Senhor, Filho de Davi, tem misericórdia de mim, que
minha filha está miseravelmente endemoninhada. Mas ele não
lhe respondeu palavra. E os seus discípulos, chegando ao pé
dele\footnote{King James: And his disciples came and besought him,
saying, \ldots{} Ed. Contemp.: De modo que os seus discípulos,
aproximando-se dele, lhe rogaram: \ldots{}}, rogaram-lhe, dizendo:
Despede-a, que vem gritando atrás de nós. E ele, respondendo,
disse: Eu não fui enviado senão às ovelhas perdidas da casa de
Israel. Então chegou ela, e adorou-o, dizendo: Senhor,
socorre-me! Ele, porém, respondendo, disse: Não é bom pegar
no pão dos filhos e deitá-lo aos cachorrinhos. E ela disse:
Sim, Senhor, mas também os cachorrinhos comem das migalhas que caem
da mesa dos seus senhores. Então respondeu Jesus, e
disse-lhe: " mulher, grande é a tua fé! Seja isso feito para contigo
como tu desejas. E desde aquela hora a sua filha ficou sã.

Partindo Jesus dali, chegou ao pé\footnote{King James: And Jesus
departed from thence, and came nigh unto the sea of Galilee; and
went up into a mountain, and sat down there. Ed. Contemp.: Jesus
partiu dali, e foi para junto do mar da Galiléia. Então subiu a um
monte, e assentou-se.} do mar da Galiléia, e, subindo a um monte,
as\-sen\-tou-se lá. E veio ter com ele grandes multidões, que
traziam coxos, cegos, mudos, aleijados, e outros muitos, e os
puseram aos pés de Jesus, e ele os sarou, de tal sorte que a
multidão se maravilhou vendo os mudos a falar, os aleijados sãos, os
coxos a andar, e os cegos a ver; e glorificava o Deus de Israel.
E Jesus, chamando os seus discípulos, disse: Tenho compaixão
da multidão, porque já está comigo há três dias, e não tem o que
comer; e não quero despedi-la em jejum, para que não desfaleça no
caminho. E os seus discípulos disseram-lhe: De onde nos
viriam, num deserto, tantos pães, para saciar tal multidão? E
Jesus disse-lhes: Quantos pães tendes? E eles disseram: Sete, e uns
poucos peixinhos\footnote{SBTB grafa \emph{de} peixinhos.
Simplesmente: ``e uns poucos peixinhos'' ou ``e alguns
peixinhos''.}. Então mandou à multidão que se assentasse no
chão, e, tomando os sete pães e os peixes, e dando graças,
partiu-os, e deu-os aos seus discípulos, e os discípulos à multidão.
E todos comeram e se saciaram; e levantaram, do que sobejou,
sete cestos cheios de pedaços. Ora, os que tinham comido eram
quatro mil homens, além de mulheres e crianças. E, tendo
despedido a multidão, entrou no barco, e dirigiu-se ao território de
Magadã.

\medskip

\lettrine{16}\ E, chegando-se os fariseus e os saduceus, para o
tentarem, pe\-di\-ram-lhe que lhes mostrasse algum sinal do céu. Mas
ele, respondendo, disse-lhes: Quando é chegada a tarde, dizeis:
Haverá bom tempo, porque o céu está rubro. E, pela manhã: Hoje
haverá tempestade, porque o céu está de um vermelho sombrio.
Hipócritas, sabeis discernir a face do céu, e não conheceis os
sinais dos tempos? Uma geração má e adúltera pede um sinal, e
nenhum sinal lhe será dado, senão o sinal do profeta Jonas. E,
deixando-os, retirou-se.

E, passando seus discípulos para o outro lado, tinham-se esquecido
de trazer pão. E Jesus disse-lhes: Adverti, e acautelai-vos do
fermento dos fariseus e saduceus. E eles arrazoavam entre si,
dizendo: É porque não trouxemos pão. E Jesus, percebendo isso,
disse: Por que arrazoais entre vós, homens de pouca fé, sobre o não
terdes trazido pão? Não compreendeis ainda, nem vos lembrais dos
cinco pães para cinco mil homens, e de quantas alcofas levantastes?
Nem dos sete pães para quatro mil, e de quantos cestos
levantastes? Como não compreendestes que não vos falei a
respeito do pão, mas que vos guardásseis do fermento dos fariseus e
saduceus? Então compreenderam que não dissera que se
guardassem do fermento do pão, mas da doutrina dos fariseus.

E, chegando Jesus às partes de Cesaréia de Filipe, interrogou os
seus discípulos, dizendo: Quem dizem os homens ser o Filho do homem?
E eles disseram: Uns, João o Batista; outros, Elias; e
outros, Jeremias, ou um dos profetas. Disse-lhes ele: E vós,
quem dizeis que eu sou? E Simão Pedro, respondendo, disse:
\textcolor{red}{Tu és o Cristo, o Filho do Deus vivo}. E Jesus,
respondendo, disse-lhe: Bem-aventurado és tu, Simão Barjonas, porque
to não revelou a carne e o sangue, mas meu Pai, que está nos céus.
Pois também eu te digo que tu és Pedro, e sobre esta pedra
edificarei a minha igreja, e as portas do inferno não prevalecerão
contra ela; e eu te darei as chaves do reino dos céus; e tudo
o que ligares na terra será ligado nos céus, e tudo o que desligares
na terra será desligado nos céus. Então mandou aos seus
discípulos que a ninguém dissessem que ele era Jesus o Cristo.

Desde então começou Jesus a mostrar aos seus discípulos que
convinha ir a Jerusalém, e padecer muitas coisas dos anciãos, e dos
principais dos sacerdotes, e dos escribas, e ser morto, e
\textcolor{red}{ressuscitar ao terceiro dia}. E Pedro, tomando-o de
parte, começou a repreendê-lo, dizendo: Senhor, tem compaixão de ti;
de modo nenhum te acontecerá isso. Ele, porém, voltando-se,
disse a Pedro: Para trás de mim, Satanás, que me serves de
escândalo; porque não compreendes as coisas que são de Deus, mas só
as que são dos homens.

Então disse Jesus aos seus discípulos: Se alguém quiser vir após
mim, renuncie-se a si mesmo, tome sobre si a sua cruz, e siga-me;
porque aquele que quiser salvar a sua vida, perdê-la-á, e
quem perder a sua vida por amor de mim, achá-la-á. Pois
\textcolor{red}{que aproveita ao homem ganhar o mundo inteiro, se perder a
sua alma? Ou que dará o homem em recompensa da sua alma?}
Porque o Filho do homem virá na glória de seu Pai, com os
seus anjos; e então dará a cada um segundo as suas obras. Em
verdade vos digo que alguns há, dos que aqui estão, que não provarão
a morte até que vejam vir o Filho do homem no seu reino.

\medskip

\lettrine{17}\ Seis dias depois, tomou Jesus consigo a Pedro, e
a Tiago, e a João, seu irmão, e os conduziu em particular a um alto
monte, e transfigurou-se diante deles; e o seu rosto
resplandeceu como o sol, e as suas vestes se tornaram brancas como a
luz. E eis que lhes apareceram Moisés e Elias, falando com ele.
E Pedro, tomando a palavra, disse a Jesus: Senhor, bom é
estarmos aqui; se queres, façamos aqui três tabernáculos, um para
ti, um para Moisés, e um para Elias. E, estando ele ainda a
falar, eis que uma nuvem luminosa os cobriu. E da nuvem saiu uma voz
que dizia: \textcolor{red}{Este é o meu amado Filho, em quem me comprazo;
escutai-o}. E os discípulos, ouvindo isto, caíram sobre os seus
rostos, e tiveram grande medo. E, aproximando-se Jesus,
tocou-lhes, e disse: Levantai-vos, e não tenhais medo. E,
erguendo eles os olhos, ninguém viram senão unicamente a Jesus.
E, descendo eles do monte, Jesus lhes ordenou, dizendo: A
ninguém conteis a visão, \textcolor{red}{até que o Filho do homem seja
ressuscitado dentre os mortos}. E os seus discípulos o
interrogaram, dizendo: Por que dizem então os escribas que é mister
que Elias venha primeiro? E Jesus, respondendo, disse-lhes:
Em verdade Elias virá primeiro, e restaurará todas as coisas;
mas digo-vos que \textcolor{red}{Elias já veio}, e não o conheceram,
mas fizeram-lhe tudo o que quiseram. Assim farão eles também padecer
o Filho do homem. Então entenderam os discípulos que lhes
falara de João o Batista.

E, quando chegaram à multidão, aproximou-se-lhe um homem,
pondo-se de joelhos diante dele, e dizendo: Senhor, tem
misericórdia de meu filho, que é lunático\footnote{KJ: lunatic. Ed.
Contemp.: epiléptico.} e sofre muito; pois muitas vezes cai no fogo,
e muitas vezes na água; e trouxe-o aos teus discípulos; e não
puderam curá-lo. E Jesus, respondendo, disse: " geração
incrédula e perversa! Até quando estarei eu convosco, e até quando
vos sofrerei? Trazei-mo aqui. E, repreendeu Jesus o demônio,
que saiu dele, e desde aquela hora o menino sarou. Então os
discípulos, aproximando-se de Jesus em particular, disseram: Por que
não pudemos nós expulsá-lo? E Jesus lhes disse: Por causa de
vossa pouca fé; porque em verdade vos digo que, se tiverdes fé como
um grão de mostarda, direis a este monte: Passa daqui para acolá, e
há de passar; e nada vos será impossível. Mas esta casta de
demônios não se expulsa senão pela \textcolor{red}{oração} e pelo
\textcolor{red}{jejum}.

Ora, achando-se eles na Galiléia, disse-lhes Jesus: \textcolor{red}{O
Filho do homem será entregue nas mãos dos homens}; \textcolor{red}{e
matá-lo-ão, e ao terceiro dia ressuscitará}. E eles se entristeceram
muito.

E, chegando eles a Cafarnaum, aproximaram-se de Pedro os que
cobravam as dracmas, e disseram: O vosso mestre não paga as dracmas?
Disse ele: Sim. E, entrando em casa, Jesus se lhe antecipou,
dizendo: Que te parece, Simão? De quem cobram os reis da terra os
tributos, ou o censo? Dos seus filhos, ou dos alheios?
Disse-lhe Pedro: Dos alheios. Disse-lhe Jesus: Logo, estão
livres os filhos. Mas, para que os não escandalizemos, vai ao
mar, lança o anzol, tira o primeiro peixe que subir, e abrindo-lhe a
boca, encontrarás um estáter; toma-o, e dá-o por mim e por ti.

\medskip

\lettrine{18}\ Naquela mesma hora chegaram os discípulos ao pé
de Jesus, dizendo: Quem é o maior no reino dos céus? E Jesus,
chamando um menino, o pôs no meio deles, e disse: Em verdade vos
digo que, se não vos converterdes e não vos fizerdes como meninos,
de modo algum entrareis no reino dos céus. Portanto, aquele que
se tornar humilde como este menino, esse é o maior no reino dos
céus. E qualquer que receber em meu nome um menino, tal como
este, a mim me recebe. Mas, qualquer que escandalizar um destes
pequeninos, que crêem em mim, melhor lhe fora que se lhe pendurasse
ao pescoço uma mó de azenha\footnote{Moinho de roda, movido a água;
atafona.}, e se submergisse na profundeza do mar.

Ai do mundo, por causa dos escândalos; porque é mister que venham
escândalos, mas ai daquele homem por quem o escândalo vem!
Portanto, se a tua mão ou o teu pé te escandalizar, corta-o, e
atira-o para longe de ti; melhor te é entrar na vida coxo, ou
aleijado, do que, tendo duas mãos ou dois pés, seres lançado no fogo
eterno. E, se o teu olho te escandalizar, arranca-o, e atira-o
para longe de ti; melhor te é entrar na vida com um só olho, do que,
tendo dois olhos, seres lançado no \textcolor{red}{fogo do inferno}.
Vede, não desprezeis algum destes pequeninos, porque eu vos
digo que os seus anjos nos céus sempre vêem a face de meu Pai que
está nos céus. Porque o Filho do homem veio salvar o que se
tinha perdido. Que vos parece? Se algum homem tiver cem
ovelhas, e uma delas se desgarrar, não irá pelos montes, deixando as
noventa e nove, em busca da que se desgarrou? E, se
porventura achá-la, em verdade vos digo que maior prazer tem por
aquela do que pelas noventa e nove que se não desgarraram.
Assim, também, não é vontade de vosso Pai, que está nos céus,
que um destes pequeninos se perca.

Ora, se teu irmão pecar contra ti, vai, e repreende-o entre ti e
ele só; se te ouvir, ganhaste a teu irmão; mas, se não te
ouvir, leva ainda contigo um ou dois, para que pela boca de duas ou
três testemunhas toda a palavra seja confirmada. E, se não as
escutar, dize-o à igreja; e, se também não escutar a igreja,
considera-o como um gentio e publicano. Em verdade vos digo
que tudo o que ligardes na terra será ligado no céu, e tudo o que
desligardes na terra será desligado no céu. Também vos digo
que, se dois de vós concordarem na terra acerca de qualquer coisa
que pedirem, isso lhes será feito por meu Pai, que está nos céus.
Porque, \textcolor{red}{onde estiverem dois ou três reunidos em meu
nome, aí estou eu no meio deles}.

Então Pedro, aproximando-se dele, disse: Senhor, até quantas
vezes pecará meu irmão contra mim, e eu lhe perdoarei? Até sete?
Jesus lhe disse: Não te digo que até sete; mas, até setenta
vezes sete. Por isso o reino dos céus pode comparar-se a um
certo rei que quis fazer contas com os seus servos; e,
começando a fazer contas, foi-lhe apresentado um que lhe devia dez
mil talentos; e, não tendo ele com que pagar, o seu senhor
mandou que ele, e sua mulher e seus filhos fossem vendidos, com tudo
quanto tinha, para que a dívida se lhe pagasse. Então aquele
servo, prostrando-se, o reverenciava, dizendo: Senhor, sê generoso
para comigo, e tudo te pagarei. Então o senhor daquele servo,
movido de íntima compaixão, soltou-o e perdoou-lhe a dívida.
Saindo, porém, aquele servo, encontrou um dos seus conservos,
que lhe devia cem denários\footnote{A SBTB usa o termo
\emph{dinheiro}.  Na Revista e Corrigida, há uma nota de
rodapé: ``denários''. Embora a palavra dinheiro tenha sua etmologia
no lat.vulg. dinarius, o denário era um tipo de moeda da época. Na
época de Cristo, o dinheiro era cunhado em três metais próprios:
ouro, prata e cobre, bronze ou latão. E havia moedas judaicas,
gregas e romanas. Por exemplo, a dracma (Lc 15.8) é uma moeda grega;
o \emph{lepton}, moeda judaica e o denário de prata era a moeda
romana básica. O termo ``denário'' (\emph{deni} = dez por vez)
deriva do fato de que a princípio era equivalente em prata a dez
asses de cobre. Cito o Novo Dicionário da Bíblia: ``O bronze (em
grego \emph{chalkos}) é a palavra usada em geral para significar
dinheiro, como em Mc~6.8 e 12.41, mas como apenas as moedas de menor
valor, como o `as' romano (gr. \emph{assarion}) e o \emph{lepton}
judaico eram cunhadas em bronze, o termo mais comum e geral usado
com o sentido de dinheiro no Novo Testamento era `prata' (em grego,
\emph{argyrion} --- Lc~9.3; At~8.20, etc.)''. 1Tm~6.10, que diz que o amor ao
dinheiro é a raiz de todos os males. A frase ``amor ao dinheiro''
corresponde, no grego, ao termo \emph{philargyria}, ou ``amor, ou
apego, afinidade, à prata''. Ainda outros termos empregados como
``dinheiro'' no Novo Testamento: \emph{chr\~ema} -- propriedade, riqueza, mas também dinheiro (At~4.37; 8.18,20; 24.26); \emph{kerma} -- ``troco pequeno'' (Jo 2.15);  \emph{nomisma} -- ``dinheiro introduzido no uso comum pela lei (\emph{nomos})'' Ex.: Mt 22.19: ``Mostrai-me a moeda do tributo. E
  eles lhe apresentaram \emph{um dinheiro}''. O ``dinheiro'', aqui,
  significa a moeda legal para o pagamento do imposto.}, e, lançando mão dele, sufocava-o, dizendo: Paga-me o que me deves. Então o seu companheiro, prostrando-se a seus pés, rogava-lhe,
dizendo: Sê generoso para comigo, e tudo te pagarei. Ele,
porém, não quis, antes foi encerrá-lo na prisão, até que pagasse a
dívida. Vendo, pois, os seus conservos o que acontecia,
contristaram-se muito, e foram declarar ao seu senhor tudo o que se
passara. Então o seu senhor, chamando-o à sua presença,
disse-lhe: Servo malvado, perdoei-te toda aquela dívida, porque me
suplicaste. Não devias tu, igualmente, ter compaixão do teu
companheiro, como eu também tive misericórdia de ti? E,
indignado, o seu senhor o entregou aos atormentadores, até que
pagasse tudo o que devia. Assim vos fará, também, meu Pai
celestial, se do coração não perdoardes, cada um a seu irmão, as
suas ofensas.

\medskip

\lettrine{19}\ E aconteceu que, concluindo Jesus estes
discursos, saiu da Galiléia, e dirigiu-se aos confins da Judéia,
além do Jordão; e seguiram-no grandes multidões, e curou-as
ali.

Então chegaram ao pé dele\footnote{King James: The Pharisees also
came unto him, tempting him, and saying unto him, \ldots{} Ed.
Contemp.: Então vieram a ele os fariseus para testá-lo, e
perguntaram: \ldots{}} os fariseus, tentando-o, e dizendo-lhe: É
lícito ao homem repudiar sua mulher por qualquer motivo? Ele,
porém, respondendo, disse-lhes: Não tendes lido que aquele que os
fez no princípio macho e fêmea os fez, e disse: Portanto,
deixará o homem pai e mãe, e se unirá a sua mulher, e serão dois
numa só carne? Assim não são mais dois, mas uma só carne.
Portanto, o que Deus ajuntou não o separe o homem. Disseram-lhe
eles: Então, por que mandou Moisés dar-lhe carta de divórcio, e
repudiá-la? Disse-lhes ele: Moisés, por causa da dureza dos
vossos corações, vos permitiu repudiar vossas mulheres; mas ao
princípio não foi assim. Eu vos digo, porém, que
\textcolor{red}{qualquer que repudiar sua mulher, não sendo por causa de
fornicação, e casar com outra, comete adultério; e o que casar com a
repudiada também comete adultério}. Disseram-lhe seus
discípulos: Se assim é a condição do homem relativamente à mulher,
não convém casar. Ele, porém, lhes disse: Nem todos podem
receber esta palavra, mas só aqueles a quem foi concedido.
Porque há eunucos que assim nasceram do ventre da mãe; e há
eunucos que foram castrados pelos homens; e há eunucos que se
castraram a si mesmos, por causa do reino dos céus. Quem pode
receber isto, receba-o.

Trouxeram-lhe, então, alguns meninos, para que sobre eles pusesse
as mãos, e orasse; mas os discípulos os repreendiam. Jesus,
porém, disse: Deixai os meninos, e não os estorveis\footnote{Impedir
(a concretização de algo); frustrar. Embaraçar, dificultar (a
realização de alguma coisa).  Pôr(-se) estorvo; importunar(-se),
incomodar(-se). Impedir ou tolher (liberdade de movimento, livre
trânsito etc.) a. Desviar, interceptar (algo). Provocar recíproco
embaraço.} de vir a mim; porque dos tais é o reino dos céus.
E, tendo-lhes imposto as mãos, partiu dali.

E eis que, aproximando-se dele um jovem, disse-lhe: Bom Mestre,
que bem farei para conseguir a vida eterna? E ele disse-lhe:
Por que me chamas bom? Não há bom senão um só, que é Deus. Se
queres, porém, entrar na vida, guarda os mandamentos.
Disse-lhe ele: Quais? E Jesus disse: Não matarás, não
cometerás adultério, não furtarás, não dirás falso testemunho;
honra teu pai e tua mãe, e amarás o teu próximo como a ti
mesmo. Disse-lhe o jovem: Tudo isso tenho guardado desde a
minha mocidade; que me falta ainda? Disse-lhe Jesus: Se
queres ser perfeito, vai, vende tudo o que tens e dá-o aos pobres, e
terás um tesouro no céu; e vem, e segue-me. E o jovem,
ouvindo esta palavra, retirou-se triste, porque possuía muitas
propriedades.

Disse então Jesus aos seus discípulos: Em verdade vos digo que é
difícil entrar um rico no reino dos céus. E, outra vez vos
digo que \textcolor{red}{é mais fácil passar um camelo pelo fundo de uma
agulha do que entrar um rico no reino de Deus}. Os seus
discípulos, ouvindo isto, admiraram-se muito, dizendo: Quem poderá
pois salvar-se? E Jesus, olhando para eles, disse-lhes:
\textcolor{red}{Aos homens é isso impossível, mas a Deus tudo é possível}.
Então Pedro, tomando a palavra, disse-lhe: Eis que nós
deixamos tudo, e te seguimos; que receberemos? E Jesus
disse-lhes: Em verdade vos digo que vós, que me seguistes, quando,
na regeneração, o Filho do homem se assentar no trono da sua glória,
também vos assentareis sobre doze tronos, para julgar as doze tribos
de Israel. E todo aquele que tiver deixado casas, ou irmãos,
ou irmãs, ou pai, ou mãe, ou mulher, ou filhos, ou terras, por amor
de meu nome, receberá cem vezes tanto, e herdará a vida eterna.
Porém, muitos primeiros serão os derradeiros, e muitos
derradeiros serão os primeiros.

\medskip

\lettrine{20}\ Porque o reino dos céus é semelhante a um homem,
pai de família, que saiu de madrugada a assalariar trabalhadores
para a sua vinha. E, ajustando com os trabalhadores a um
denário\footnote{SBTB: dinheiro.} por dia, mandou-os para a sua
vinha. E, saindo perto da hora terceira, viu outros que estavam
ociosos na praça, e disse-lhes: Ide vós também para a vinha, e
dar-vos-ei o que for justo. E eles foram. Saindo outra vez,
perto da hora sexta e nona, fez o mesmo. E, saindo perto da hora
undécima, encontrou outros que estavam ociosos, e perguntou-lhes:
Por que estais ociosos todo o dia? Disseram-lhe eles: Porque
ninguém nos assalariou. Diz-lhes ele: Ide vós também para a vinha, e
recebereis o que for justo. E, aproximando-se a noite, diz o
senhor da vinha ao seu mordomo: Chama os trabalhadores, e paga-lhes
o jornal\footnote{KJ: hire. Ed. Contemp.: salário}, começando pelos
derradeiros, até aos primeiros. E, chegando os que tinham ido
perto da hora undécima, receberam um denário cada um. Vindo,
porém, os primeiros, cuidaram que haviam de receber mais; mas do
mesmo modo receberam um denário cada um. E, recebendo-o,
murmuravam contra o pai de família, dizendo: Estes
derradeiros trabalharam só uma hora, e tu os igualaste conosco, que
suportamos a fadiga e a calma do dia. Mas ele, respondendo,
disse a um deles: Amigo, não te faço agravo; não ajustaste tu comigo
um denário? Toma o que é teu, e retira-te; eu quero dar a
este derradeiro tanto como a ti. Ou não me é lícito fazer o
que quiser do que é meu? Ou é mau o teu olho porque eu sou bom?
Assim os derradeiros serão primeiros, e os primeiros
derradeiros; porque muitos são chamados, mas poucos escolhidos.

E, subindo Jesus a Jerusalém, chamou de parte os seus doze
discípulos, e no caminho disse-lhes: Eis que vamos para
Jerusalém, e o Filho do homem será entregue aos príncipes dos
sacerdotes, e aos escribas, e condená-lo-ão à morte. E o
entregarão aos gentios para que dele escarneçam, e o açoitem e
crucifiquem, e \textcolor{red}{ao terceiro dia ressuscitará}.

Então se aproximou dele a mãe dos filhos de Zebedeu, com seus
filhos, adorando-o, e fazendo-lhe um pedido. E ele diz-lhe:
Que queres? Ela respondeu: Dize que estes meus dois filhos se
assentem, um à tua direita e outro à tua esquerda, no teu reino.
Jesus, porém, respondendo, disse: Não sabeis o que pedis.
Podeis vós beber o cálice que eu hei de beber, e ser batizados com o
batismo com que eu sou batizado? Dizem-lhe eles: Podemos. E
diz-lhes ele: Na verdade bebereis o meu cálice e sereis batizados
com o batismo com que eu sou batizado, mas o assentar-se à minha
direita ou à minha esquerda não me pertence dá-lo, mas é para
aqueles para quem meu Pai o tem preparado. E, quando os dez
ouviram isto, indignaram-se contra os dois irmãos. Então
Jesus, chamando-os para junto de si, disse: Bem sabeis que pelos
príncipes dos gentios são estes dominados, e que os grandes exercem
autoridade sobre eles. Não será assim entre vós; mas todo
aquele que quiser entre vós fazer-se grande seja vosso serviçal;
e, qualquer que entre vós quiser ser o primeiro, seja vosso
servo; bem como o Filho do homem não veio para ser servido,
mas para servir, e para \textcolor{red}{dar a sua vida em resgate de
muitos}.

E, saindo eles de Jericó, seguiu-o grande multidão. E eis
que dois cegos, assentados junto do caminho, ouvindo que Jesus
passava, clamaram, dizendo: Senhor, Filho de Davi, tem misericórdia
de nós! E a multidão os repreendia, para que se calassem;
eles, porém, cada vez clamavam mais, dizendo: Senhor, Filho de Davi,
tem misericórdia de nós! E Jesus, parando, chamou-os, e
disse: Que quereis que vos faça? Disseram-lhe eles: Senhor,
que os nossos olhos sejam abertos. Então Jesus,
\textcolor{red}{movido de íntima compaixão}, tocou-lhes nos olhos, e logo
viram; e eles o seguiram.

\medskip

\lettrine{21}\ E, quando se aproximaram de Jerusalém, e
chegaram a Betfagé, ao Monte das Oliveiras, enviou, então, Jesus
dois discípulos, dizendo-lhes: Ide à aldeia que está defronte de
vós, e logo encontrareis uma jumenta presa, e um jumentinho com ela;
desprendei-a, e trazei-mos. E, se alguém vos disser alguma
coisa, direis que o Senhor os há de mister\footnote{Necessitar.}; e
logo os enviará. Ora, tudo isto aconteceu para que se cumprisse
o que foi dito pelo profeta, que diz: Dizei à filha de Sião: Eis
que o teu Rei aí te vem, manso, e assentado sobre uma jumenta, e
sobre um jumentinho, filho de animal de carga. E, indo os
discípulos, e fazendo como Jesus lhes ordenara, trouxeram a
jumenta e o jumentinho, e sobre eles puseram as suas vestes, e
fizeram-no assentar em cima. E muitíssima gente estendia as suas
vestes pelo caminho, e outros cortavam ramos de árvores, e os
espalhavam pelo caminho. E a multidão que ia adiante, e a que
seguia, clamava, dizendo: \textcolor{red}{Hosana ao Filho de Davi; bendito o
que vem em nome do Senhor. Hosana nas alturas}! E, entrando
ele em Jerusalém, toda a cidade se alvoroçou, dizendo: Quem é este?
E a multidão dizia: Este é Jesus, o profeta de Nazaré da
Galiléia.

E entrou Jesus no templo de Deus, e expulsou todos os que vendiam
e compravam no templo, e derribou as mesas dos cambistas e as
cadeiras dos que vendiam pombas; e disse-lhes: Está
escrito\footnote{Is 56.7.}: A minha casa será chamada casa de
oração; mas vós a tendes convertido em covil de ladrões. E
foram ter com ele no templo cegos e coxos, e curou-os. Vendo,
então, os principais dos sacerdotes e os escribas as maravilhas que
fazia, e os meninos clamando no templo: Hosana ao Filho de Davi,
indignaram-se, e disseram-lhe: Ouves o que estes dizem? E
Jesus lhes disse: Sim; nunca lestes: Pela boca dos meninos e das
criancinhas de peito tiraste o perfeito louvor? E,
deixando-os, saiu da cidade para Betânia, e ali passou a noite.

E, de manhã, voltando para a cidade, teve fome; e,
avistando uma figueira perto do caminho, dirigiu-se a ela, e não
achou nela senão folhas. E disse-lhe: Nunca mais nasça fruto de ti!
E a figueira secou imediatamente. E os discípulos, vendo
isto, maravilharam-se, dizendo: Como secou imediatamente a figueira?
Jesus, porém, respondendo, disse-lhes: Em verdade vos digo
que, se tiverdes fé e não duvidardes, não só fareis o que foi feito
à figueira, mas até se a este monte disserdes: Ergue-te, e
precipita-te no mar, assim será feito; e, tudo o que pedirdes
na oração, crendo, o recebereis.

E, chegando ao templo, acercaram-se dele, estando já ensinando,
os príncipes dos sacerdotes e os anciãos do povo, dizendo: Com que
autoridade fazes isto? e quem te deu tal autoridade? E Jesus,
respondendo, disse-lhes: Eu também vos perguntarei uma coisa; se ma
disserdes, também eu vos direi com que autoridade faço isto.
O batismo de João, de onde era? Do céu, ou dos homens? E
pensavam entre si, dizendo: Se dissermos: Do céu, ele nos dirá:
Então por que não o crestes? E, se dissermos: Dos homens,
tememos o povo, porque todos consideram João como profeta. E,
respondendo a Jesus, disseram: Não sabemos. Ele disse-lhes: Nem eu
vos digo com que autoridade faço isto.

Mas, que vos parece? Um homem tinha dois filhos, e, dirigindo-se
ao primeiro, disse: Filho, vai trabalhar hoje na minha vinha.
Ele, porém, respondendo, disse: Não quero. Mas depois,
arrependendo-se, foi. E, dirigindo-se ao segundo, falou-lhe
de igual modo; e, respondendo ele, disse: Eu vou, senhor; e não foi.
Qual dos dois fez a vontade do pai? Disseram-lhe eles: O
primeiro. Disse-lhes Jesus: Em verdade vos digo que os publicanos e
as meretrizes entram adiante de vós no reino de Deus. Porque
João veio a vós no caminho da justiça, e não o crestes, mas os
publicanos e as meretrizes o creram; vós, porém, vendo isto, nem
depois vos arrependestes para o crer.

Ouvi, ainda, outra parábola: Houve um homem, pai de família, que
plantou uma vinha, e circundou-a de um valado, e construiu nela um
lagar, e edificou uma torre, e arrendou-a a uns lavradores, e
ausentou-se para longe. E, chegando o tempo dos frutos,
enviou os seus servos aos lavradores, para receber os seus frutos.
E os lavradores, apoderando-se dos servos, feriram um,
mataram outro, e apedrejaram outro. Depois enviou outros
servos, em maior número do que os primeiros; e eles fizeram-lhes o
mesmo. E, por último, enviou-lhes seu filho, dizendo: Terão
respeito a meu filho. Mas os lavradores, vendo o filho,
disseram entre si: Este é o herdeiro; vinde, matemo-lo, e
apoderemo-nos da sua herança. E, lançando mão dele, o
arrastaram para fora da vinha, e o mataram. Quando, pois,
vier o senhor da vinha, que fará àqueles lavradores?
Dizem-lhe eles: Dará afrontosa morte aos maus, e arrendará a
vinha a outros lavradores, que a seu tempo lhe dêem os frutos.
Diz-lhes Jesus: Nunca lestes nas Escrituras: A pedra, que os
edificadores rejeitaram, essa foi posta por cabeça do ângulo; pelo
Senhor foi feito isto, e é maravilhoso aos nossos olhos?
Portanto, eu vos digo que \textcolor{red}{o reino de Deus vos será
tirado, e será dado a uma nação que dê os seus frutos}. E,
quem cair sobre esta pedra, despedaçar-se-á; e aquele sobre quem ela
cair ficará reduzido a pó. E os príncipes dos sacerdotes e os
fariseus, ouvindo estas palavras, entenderam que falava deles;
e, pretendendo prendê-lo, recearam o povo, porquanto o tinham
por profeta.

\medskip

\lettrine{22}\ Então Jesus, tomando a palavra, tornou a
falar-lhes em parábolas, dizendo: O reino dos céus é semelhante
a um certo rei que celebrou as bodas de seu filho; e enviou os
seus servos a chamar os convidados para as bodas, e estes não
quiseram vir. Depois, enviou outros servos, dizendo: Dizei aos
convidados: Eis que tenho o meu jantar preparado, os meus bois e
cevados já mortos, e tudo já pronto; vinde às bodas. Eles,
porém, não fazendo caso, foram, um para o seu campo, outro para o
seu tráfico; e os outros, apoderando-se dos servos, os
ultrajaram e mataram. E o rei, tendo notícia disto,
encolerizou-se e, enviando os seus exércitos, destruiu aqueles
homicidas, e incendiou a sua cidade. Então diz aos servos: As
bodas, na verdade, estão preparadas, mas os convidados não eram
dignos. Ide, pois, às saídas dos caminhos, e convidai para as
bodas a todos os que encontrardes. E os servos, saindo pelos
caminhos, ajuntaram todos quantos encontraram, tanto maus como bons;
e a festa nupcial foi cheia de convidados. E o rei, entrando
para ver os convidados, viu ali um homem que não estava trajado com
veste de núpcias. E disse-lhe: Amigo, como entraste aqui, não
tendo veste nupcial? E ele emudeceu. Disse, então, o rei aos
servos: Amarrai-o de pés e mãos, levai-o, e lançai-o nas trevas
exteriores; ali haverá pranto e ranger de dentes. Porque
\textcolor{red}{muitos são chamados, mas poucos escolhidos}.

Então, retirando-se os fariseus, consultaram entre si como o
surpreenderiam nalguma palavra; e enviaram-lhe os seus
discípulos, com os herodianos, dizendo: Mestre, bem sabemos que és
verdadeiro, e ensinas o caminho de Deus segundo a verdade, e de
ninguém se te dá\footnote{KJ: And they sent out unto him their
disciples with the Herodians, saying, Master, we know that thou art
true, and teachest the way of God in truth, neither carest thou for
any man: for thou regardest not the person of men. Ed. Contemp.: 	``e
não dás preferência a ninguém''.}, porque não olhas a aparência dos
homens. Dize-nos, pois, que te parece? É lícito pagar o
tributo a César, ou não? Jesus, porém, conhecendo a sua
malícia, disse: Por que me experimentais, hipócritas?
Mostrai-me a moeda do tributo. E eles lhe apresentaram um
dinheiro\footnote{\emph{Nomisma}: dinheiro introduzido no uso comum
pela lei (\emph{nomos}). Neste caso, significa a moeda legal para o
pagamento do imposto.}. E ele diz-lhes: De quem é esta efígie
e esta inscrição? Dizem-lhe eles: De César. Então ele lhes
disse: Dai pois a César o que é de César, e a Deus o que é de Deus.
E eles, ouvindo isto, maravilharam-se, e, deixando-o, se
retiraram.

No mesmo dia chegaram junto dele os saduceus, que dizem não haver
ressurreição, e o interrogaram, dizendo: Mestre, Moisés
disse: Se morrer alguém, não tendo filhos, casará o seu irmão com a
mulher dele, e suscitará descendência a seu irmão. Ora, houve
entre nós sete irmãos; e o primeiro, tendo casado, morreu e, não
tendo descendência, deixou sua mulher a seu irmão. Da mesma
sorte o segundo, e o terceiro, até ao sétimo; por fim, depois
de todos, morreu também a mulher. Portanto, na ressurreição,
de qual dos sete será a mulher, visto que todos a possuíram?
Jesus, porém, respondendo, disse-lhes: \textcolor{red}{Errais, não
conhecendo as Escrituras, nem o poder de Deus}.
\textcolor{red}{Porque na ressurreição nem casam nem são dados em
casamento; mas serão como os anjos de Deus no céu}. E,
\textcolor{red}{acerca da ressurreição dos mortos}, não tendes lido o que
Deus vos declarou, dizendo: Eu sou o Deus de Abraão, o Deus
de Isaque, e o Deus de Jacó? Ora, Deus não é Deus dos mortos, mas
dos vivos. E, as turbas, ouvindo isto, ficaram maravilhadas
da sua doutrina.

E os fariseus, ouvindo que ele fizera emudecer os saduceus,
reuniram-se no mesmo lugar. E um deles, doutor da lei,
interrogou-o para o experimentar, dizendo: Mestre, qual é o
grande mandamento na lei? E Jesus disse-lhe: \textcolor{red}{Amarás o
Senhor teu Deus de todo o teu coração, e de toda a tua alma, e de
todo o teu pensamento}\footnote{KJ: Jesus said unto him, Thou shalt
love the Lord thy God with all thy heart, and with all thy soul, and
with all thy mind. Ed. Contemp.: ``de todo o teu entendimento''.}.
\textcolor{red}{Este é o primeiro e grande mandamento}.
\textcolor{red}{E o segundo, semelhante a este, é: Amarás o teu
próximo como a ti mesmo}. \textcolor{red}{Destes dois mandamentos
dependem toda a lei e os profetas}.

E, estando reunidos os fariseus, interrogou-os Jesus,
dizendo: Que pensais vós do Cristo? De quem é filho? Eles
disseram-lhe: De Davi. Disse-lhes ele: Como é então que Davi,
em espírito, lhe chama Senhor, dizendo: Disse o Senhor ao meu
Senhor: Assenta-te à minha direita, até que eu ponha os teus
inimigos por escabelo de teus pés?\footnote{Sl 110.1.} Se
Davi, pois, lhe chama Senhor, como é seu filho? E ninguém
podia responder-lhe uma palavra; nem desde aquele dia ousou mais
alguém interrogá-lo.

\medskip

\lettrine{23}\ Então falou Jesus à multidão, e aos seus
discípulos, dizendo: Na cadeira de Moisés estão assentados os
escribas e fariseus. Todas as coisas, pois, que vos disserem que
observeis, observai-as e fazei-as; mas não procedais em conformidade
com as suas obras, porque dizem e não fazem; pois atam fardos
pesados e difíceis de suportar, e os põem aos ombros dos homens;
eles, porém, nem com o dedo querem movê-los; e fazem todas as
obras a fim de serem vistos pelos homens; pois trazem largos
filactérios\footnote{Cada uma das duas caixinhas que contêm uma
faixa de pergaminho com passagens bíblicas que os judeus trazem
junto à testa e ao braço esquerdo, durante a oração matinal dos dias
úteis, com o fito de lembrarem-se das palavras de Deus.}, e alargam
as franjas das suas vestes, e amam os primeiros lugares nas
ceias e as primeiras cadeiras nas sinagogas, e as saudações nas
praças, e o serem chamados pelos homens: Rabi, Rabi. Vós, porém,
não queirais ser chamados Rabi, porque \textcolor{red}{um só é o vosso
Mestre, a saber, o Cristo, e todos vós sois irmãos}. E a ninguém
na terra chameis vosso pai, porque um só é o vosso Pai, o qual está
nos céus. Nem vos chameis mestres, porque \textcolor{red}{um só é o
vosso Mestre, que é o Cristo}. O maior dentre vós será vosso
servo. E o que a si mesmo se exaltar será humilhado; e o que
a si mesmo se humilhar será exaltado.

Mas ai de vós, escribas e fariseus, hipócritas! pois que fechais
aos homens o reino dos céus; e nem vós entrais nem deixais entrar
aos que estão entrando. Ai de vós, escribas e fariseus,
hipócritas! pois que devorais as casas das viúvas, sob pretexto de
prolongadas orações; por isso sofrereis mais rigoroso juízo.
Ai de vós, escribas e fariseus, hipócritas! pois que
percorreis o mar e a terra para fazer um prosélito; e, depois de o
terdes feito, o fazeis filho do inferno duas vezes mais do que vós.
Ai de vós, condutores cegos! pois que dizeis: Qualquer que
jurar pelo templo, isso nada é; mas o que jurar pelo ouro do templo,
esse é devedor. Insensatos e cegos! Pois qual é maior: o
ouro, ou o templo, que santifica o ouro? E aquele que jurar
pelo altar isso nada é; mas aquele que jurar pela oferta que está
sobre o altar, esse é devedor. Insensatos e cegos! Pois qual
é maior: a oferta, ou o altar, que santifica a oferta?
Portanto, o que jurar pelo altar, jura por ele e por tudo o
que sobre ele está; e, o que jurar pelo templo, jura por ele
e por aquele que nele habita; e, o que jurar pelo céu, jura
pelo trono de Deus e por aquele que está assentado nele. Ai
de vós, escribas e fariseus, hipócritas! pois que dizimais a
hortelã, o endro e o cominho, e desprezais o mais importante da lei,
o juízo, a misericórdia e a fé; deveis, porém, fazer estas coisas, e
não omitir aquelas. Condutores cegos! que coais um mosquito e
engolis um camelo. Ai de vós, escribas e fariseus,
hipócritas! pois que limpais o exterior do copo e do prato, mas o
interior está cheio de rapina e de iniqüidade. Fariseu cego!
limpa primeiro o interior do copo e do prato, para que também o
exterior fique limpo. Ai de vós, escribas e fariseus,
hipócritas! pois que sois semelhantes aos sepulcros caiados, que por
fora realmente parecem formosos, mas interiormente estão cheios de
ossos de mortos e de toda a imundícia. Assim também vós
exteriormente pareceis justos aos homens, mas interiormente estais
cheios de hipocrisia e de iniqüidade. Ai de vós, escribas e
fariseus, hipócritas! pois que edificais os sepulcros dos profetas e
adornais os monumentos dos justos, e dizeis: Se existíssemos
no tempo de nossos pais, nunca nos associaríamos com eles para
derramar o sangue dos profetas. Assim, vós mesmos testificais
que sois filhos dos que mataram os profetas. Enchei vós,
pois, a medida de vossos pais. Serpentes, raça de víboras!
\textcolor{red}{como escapareis da condenação do inferno}?

Portanto, eis que eu vos envio profetas, sábios e escribas; a uns
deles matareis e crucificareis; e a outros deles açoitareis nas
vossas sinagogas e os perseguireis de cidade em cidade; para
que sobre vós caia todo o sangue justo, que foi derramado sobre a
terra, desde o sangue de Abel, o justo, até ao sangue de Zacarias,
filho de Baraquias, que matastes entre o santuário e o altar.
Em verdade vos digo que todas estas coisas hão de vir sobre
esta geração. Jerusalém, Jerusalém, que matas os profetas, e
apedrejas os que te são enviados! Quantas vezes quis eu ajuntar os
teus filhos, como a galinha ajunta os seus pintos debaixo das asas,
e tu não quiseste! Eis que a vossa casa vai ficar-vos
deserta; porque eu vos digo que desde agora me não vereis
mais, até que digais: \textcolor{red}{Bendito o que vem em nome do Senhor}.

\medskip

\lettrine{24}\ E, quando Jesus ia saindo do templo,
aproximaram-se dele os seus discípulos para lhe mostrarem a
estrutura do templo. Jesus, porém, lhes disse: Não vedes tudo
isto? Em verdade vos digo que não ficará aqui pedra sobre pedra que
não seja derrubada. E, estando assentado no Monte das Oliveiras,
chegaram-se a ele os seus discípulos em particular, dizendo:
Dize-nos, quando serão essas coisas, e que sinal haverá da tua vinda
e do fim do mundo?

E Jesus, respondendo, disse-lhes: Acautelai-vos, que ninguém vos
engane; porque muitos virão em meu nome, dizendo: Eu sou o
Cristo; e enganarão a muitos. E ouvireis de guerras e de rumores
de guerras; olhai, não vos assusteis, porque é mister que isso tudo
aconteça, mas ainda não é o fim. Porquanto se levantará nação
contra nação, e reino contra reino, e haverá fomes, e pestes, e
terremotos, em vários lugares. Mas todas estas coisas são o
princípio de dores. Então vos hão de entregar para serdes
atormentados, e matar-vos-ão; e sereis odiados de todas as nações
por causa do meu nome. Nesse tempo muitos serão
escandalizados, e trair-se-ão uns aos outros, e uns aos outros se
odiarão. E surgirão muitos falsos profetas, e enganarão a
muitos. E, por se multiplicar a iniqüidade, o amor de muitos
esfriará. Mas aquele que perseverar até ao fim será salvo.
E este evangelho do reino será pregado em todo o mundo, em
testemunho a todas as nações, e então virá o fim. Quando,
pois, virdes que a abominação da desolação, de que falou o profeta
Daniel, está no lugar santo; quem lê, atenda\footnote{``Quem lê,
entenda''. King James: ``understand''. A Revista e Corrigida da
Imprensa Bíblica Brasileira, 1987, registra ``atenda''; mas na
veiculada pela Sociedade Bíblica do Brasil na Bíblia Online, diz:
``quem lê, quem entenda''. Outra tradução proposta seria: atente.};
então, os que estiverem na Judéia, fujam para os montes;
e quem estiver sobre o telhado não desça a tirar alguma coisa
de sua casa; e quem estiver no campo não volte atrás a buscar
as suas vestes. Mas ai das grávidas e das que amamentarem
naqueles dias! E orai para que a vossa fuga não aconteça no
inverno nem no sábado; porque haverá então grande aflição,
como nunca houve desde o princípio do mundo até agora, nem tampouco
há de haver. E, se aqueles dias não fossem abreviados,
nenhuma carne se salvaria; mas por causa dos escolhidos serão
abreviados aqueles dias. Então, se alguém vos disser: Eis que
o Cristo está aqui, ou ali, não lhe deis crédito; porque
\textcolor{red}{surgirão falsos cristos e falsos profetas, e farão tão
grandes sinais e prodígios que, se possível fora, enganariam até os
escolhidos}. Eis que eu vo-lo tenho predito. Portanto,
se vos disserem: Eis que ele está no deserto, não saiais. Eis que
ele está no interior da casa; não acrediteis. Porque, assim
como o relâmpago sai do oriente e se mostra até ao ocidente, assim
será também a \textcolor{red}{vinda do Filho do homem}. Pois onde
estiver o cadáver, aí se ajuntarão as águias. E, logo depois
da aflição daqueles dias, o sol escurecerá, e a lua não dará a sua
luz, e as estrelas cairão do céu, e as potências dos céus serão
abaladas. Então aparecerá no céu o sinal do Filho do homem; e
todas as tribos da terra se lamentarão, e verão o Filho do homem,
vindo sobre as nuvens do céu, com poder e grande glória. E
ele enviará os seus anjos com rijo clamor de trombeta, os quais
ajuntarão os seus escolhidos desde os quatro ventos, de uma à outra
extremidade dos céus.

Aprendei, pois, esta parábola da figueira: Quando já os seus
ramos se tornam tenros e brotam folhas, sabeis que está próximo o
verão. Igualmente, quando virdes todas estas coisas, sabei
que ele está próximo, às portas. Em verdade vos digo que não
passará esta geração sem que todas estas coisas aconteçam.
\textcolor{red}{O céu e a terra passarão, mas as minhas palavras não
hão de passar}. Mas daquele dia e hora ninguém sabe, nem os
anjos do céu, mas unicamente meu Pai. E, como foi nos dias de
Noé, assim será também \textcolor{red}{a vinda do Filho do homem}.
Porquanto, assim como, nos dias anteriores ao dilúvio,
comiam, bebiam, casavam e davam-se em casamento, até ao dia em que
Noé entrou na arca, e não o perceberam, até que veio o
dilúvio, e os levou a todos, assim será também \textcolor{red}{a vinda do
Filho do homem}. Então, estando dois no campo, será levado
um, e deixado o outro; estando duas moendo no moinho, será
levada uma, e deixada outra. Vigiai, pois, porque não sabeis
a que hora há de vir o vosso Senhor. Mas considerai isto: se
o pai de família soubesse a que vigília da noite havia de vir o
ladrão, vigiaria e não deixaria minar a sua casa. Por isso,
estai vós apercebidos também; porque o Filho do homem há de vir à
hora em que não penseis. Quem é, pois, o servo fiel e
prudente, que o seu senhor constituiu sobre a sua casa, para dar o
sustento a seu tempo? Bem-aventurado aquele servo que o seu
senhor, quando vier, achar servindo assim. Em verdade vos
digo que o porá sobre todos os seus bens. Mas se aquele mau
servo disser no seu coração: O meu senhor tarde virá; e
começar a espancar os seus conservos, e a comer e a beber com os
ébrios, virá o senhor daquele servo num dia em que o não
espera, e à hora em que ele não sabe, e separá-lo-á, e
destinará a sua parte com os hipócritas; ali haverá pranto e ranger
de dentes.

\medskip

\lettrine{25}\ Então o reino dos céus será semelhante a dez
virgens que, tomando as suas lâmpadas, saíram ao encontro do esposo.
E cinco delas eram prudentes, e cinco loucas. As loucas,
tomando as suas lâmpadas, não levaram azeite consigo. Mas as
prudentes levaram azeite em suas vasilhas, com as suas lâmpadas.
E, tardando o esposo, tosquenejaram\footnote{Toscanejar:
cabecear com sono, abrindo e fechando os olhos repetidamente;
dormitar, cochilar.} todas, e adormeceram. Mas à meia-noite
ouviu-se um clamor: Aí vem o esposo, saí-lhe ao encontro. Então
todas aquelas virgens se levantaram, e prepararam as suas lâmpadas.
E as loucas disseram às prudentes: Dai-nos do vosso azeite,
porque as nossas lâmpadas se apagam. Mas as prudentes
responderam, dizendo: Não seja caso que nos falte a nós e a vós, ide
antes aos que o vendem, e comprai-o para vós. E, tendo elas
ido comprá-lo, chegou o esposo, e as que estavam preparadas entraram
com ele para as bodas, e fechou-se a porta. E depois chegaram
também as outras virgens, dizendo: Senhor, Senhor, abre-nos.
E ele, respondendo, disse: Em verdade vos digo que vos não
conheço. \textcolor{red}{Vigiai, pois, porque não sabeis o dia nem a
hora em que o Filho do homem há de vir}.

Porque isto é também como um homem que, partindo para fora da
terra, chamou os seus servos, e entregou-lhes os seus bens. E
a um deu cinco talentos, e a outro dois, e a outro um, a cada um
segundo a sua capacidade, e ausentou-se logo para longe. E,
tendo ele partido, o que recebera cinco talentos negociou com eles,
e granjeou outros cinco talentos. Da mesma sorte, o que
recebera dois, granjeou também outros dois. Mas o que
recebera um, foi e cavou na terra e escondeu o dinheiro do seu
senhor. E muito tempo depois veio o senhor daqueles servos, e
fez contas com eles. Então aproximou-se o que recebera cinco
talentos, e trouxe-lhe outros cinco talentos, dizendo: Senhor,
entregaste-me cinco talentos; eis aqui outros cinco talentos que
granjeei com eles. E o seu senhor lhe disse: Bem está, servo
bom e fiel. Sobre o pouco foste fiel, sobre muito te colocarei;
entra no gozo do teu senhor. E, chegando também o que tinha
recebido dois talentos, disse: Senhor, entregaste-me dois talentos;
eis que com eles granjeei outros dois talentos. Disse-lhe o
seu Senhor: Bem está, bom e fiel servo. Sobre o pouco foste fiel,
sobre muito te colocarei; entra no gozo do teu senhor. Mas,
chegando também o que recebera um talento, disse: Senhor, eu
conhecia-te, que és um homem duro, que ceifas onde não semeaste e
ajuntas onde não espalhaste; e, atemorizado, escondi na terra
o teu talento; aqui tens o que é teu. Respondendo, porém, o
seu senhor, disse-lhe: Mau e negligente servo; sabias que ceifo onde
não semeei e ajunto onde não espalhei? Devias então ter dado
o meu dinheiro aos banqueiros e, quando eu viesse, receberia o meu
com os juros. Tirai-lhe pois o talento, e dai-o ao que tem os
dez talentos. Porque a qualquer que tiver será dado, e terá
em abundância; mas ao que não tiver até o que tem ser-lhe-á tirado.
Lançai, pois, o servo inútil nas trevas exteriores; ali
haverá pranto e ranger de dentes.

E quando o Filho do homem vier em sua glória, e todos os santos
anjos com ele, então se assentará no trono da sua glória; e
todas as nações serão reunidas diante dele, e apartará uns dos
outros, como o pastor aparta dos bodes as ovelhas; e porá as
ovelhas à sua direita, mas os bodes à esquerda. Então dirá o
Rei aos que estiverem à sua direita: Vinde, benditos de meu Pai,
possuí por herança o reino que vos está preparado desde a fundação
do mundo; porque tive fome, e destes-me de comer; tive sede,
e destes-me de beber; era estrangeiro, e hospedastes-me;
estava nu, e vestistes-me; adoeci, e visitastes-me; estive na
prisão, e fostes ver-me. Então os justos lhe responderão,
dizendo: Senhor, quando te vimos com fome, e te demos de comer? ou
com sede, e te demos de beber? E quando te vimos estrangeiro,
e te hospedamos? ou nu, e te vestimos? E quando te vimos
enfermo, ou na prisão, e fomos ver-te? E, respondendo o Rei,
lhes dirá: Em verdade vos digo que quando o fizestes a um destes
meus pequeninos irmãos, a mim o fizestes. Então dirá também
aos que estiverem à sua esquerda: Apartai-vos de mim, malditos, para
o \textcolor{red}{fogo eterno, preparado para o diabo e seus anjos};
porque tive fome, e não me destes de comer; tive sede, e não
me destes de beber; sendo estrangeiro, não me recolhestes;
estando nu, não me vestistes; e enfermo, e na prisão, não me
visitastes. Então eles também lhe responderão, dizendo:
Senhor, quando te vimos com fome, ou com sede, ou estrangeiro, ou
nu, ou enfermo, ou na prisão, e não te servimos? Então lhes
responderá, dizendo: Em verdade vos digo que, quando a um destes
pequeninos o não fizestes, não o fizestes a mim. E
\textcolor{red}{irão estes para o tormento eterno, mas os justos para a vida
eterna}.

\medskip

\lettrine{26}\ E aconteceu que, quando Jesus concluiu todos
estes discursos, disse aos seus discípulos: Bem sabeis que daqui
a dois dias é a páscoa; e \textcolor{red}{o Filho do homem será entregue
para ser crucificado}. Depois os príncipes dos sacerdotes, e os
escribas, e os anciãos do povo reuniram-se na sala do sumo
sacerdote, o qual se chamava Caifás. E consultaram-se mutuamente
para prenderem Jesus com dolo e o matarem. Mas diziam: Não
durante a festa, para que não haja alvoroço entre o povo.

E, estando Jesus em Betânia, em casa de Simão, o leproso,
aproximou-se dele uma mulher com um vaso de alabastro, com
ungüento de grande valor, e derramou-lho sobre a cabeça, quando ele
estava assentado à mesa. E os seus discípulos, vendo isto,
indignaram-se, dizendo: Por que é este desperdício? Pois este
ungüento podia vender-se por grande preço, e dar-se o dinheiro aos
pobres. Jesus, porém, conhecendo isto, disse-lhes: Por que
afligis esta mulher? Pois praticou uma boa ação para comigo.
Porquanto sempre tendes convosco os pobres, mas a mim não me
haveis de ter sempre. Ora, derramando ela este ungüento sobre
o meu corpo, fê-lo preparando-me para o meu sepultamento. Em
verdade vos digo que, onde quer que este evangelho for pregado em
todo o mundo, também será referido o que ela fez, para memória sua.

Então um dos doze, chamado Judas Iscariotes, foi ter com os
príncipes dos sacerdotes, e disse: Que me quereis dar, e eu
vo-lo entregarei? E eles lhe pesaram trinta moedas de prata,
e desde então buscava oportunidade para o entregar.

E, no primeiro dia da festa dos pães ázimos, chegaram os
discípulos junto de Jesus, dizendo: Onde queres que façamos os
preparativos para comeres a páscoa? E ele disse: Ide à
cidade, a um certo homem, e dizei-lhe: O Mestre diz: O meu tempo
está próximo; em tua casa celebrarei a páscoa com os meus
discípulos. E os discípulos fizeram como Jesus lhes ordenara,
e prepararam a páscoa. E, chegada a tarde, assentou-se à mesa
com os doze. E, comendo eles, disse: Em verdade vos digo que
um de vós me há de trair. E eles, entristecendo-se muito,
começaram cada um a dizer-lhe: Porventura sou eu, Senhor? E
ele, respondendo, disse: O que põe comigo a mão no prato, esse me há
de trair. Em verdade o Filho do homem vai, como acerca dele
está escrito, mas ai daquele homem por quem o Filho do homem é
traído! Bom seria para esse homem se não houvera nascido. E,
respondendo Judas, o que o traía, disse: Porventura sou eu, Rabi?
Ele disse: Tu o disseste.

E, quando comiam, Jesus tomou o pão, e abençoando-o, o partiu, e
o deu aos discípulos, e disse: Tomai, comei, isto é o meu corpo.
E, tomando o cálice, e dando graças, deu-lho, dizendo: Bebei
dele todos; porque isto é o meu sangue, \textcolor{red}{o sangue do
novo testamento, que é derramado por muitos, para remissão dos
pecados}. E digo-vos que, desde agora, não beberei deste
fruto da vide, até aquele dia em que o beba novo convosco no reino
de meu Pai. E, tendo cantado o hino, saíram para o Monte das
Oliveiras.

Então Jesus lhes disse: Todos vós esta noite vos escandalizareis
em mim; porque está escrito\footnote{Zc 13.7.}: Ferirei o pastor, e
as ovelhas do rebanho se dispersarão. Mas, \textcolor{red}{depois de
eu ressuscitar}, irei adiante de vós para a Galiléia. Mas
Pedro, respondendo, disse-lhe: Ainda que todos se escandalizem em
ti, eu nunca me escandalizarei. Disse-lhe Jesus: Em verdade
te digo que, nesta mesma noite, antes que o galo cante, três vezes
me negarás. Disse-lhe Pedro: Ainda que me seja mister morrer
contigo, não te negarei. E todos os discípulos disseram o mesmo.

Então chegou Jesus com eles a um lugar chamado Getsêmani, e disse
a seus discípulos: Assentai-vos aqui, enquanto vou além orar.
E, levando consigo Pedro e os dois filhos de Zebedeu, começou
a entristecer-se e a angustiar-se muito. Então lhes disse: A
minha alma está cheia de tristeza até a morte; ficai aqui, e
velai\footnote{Velar: Passar a noite, ou boa parte dela, acordado.
Estar alerta; vigiar.} comigo. E, indo um pouco mais para
diante, prostrou-se sobre o seu rosto, orando e dizendo: Meu Pai, se
é possível, passe de mim este cálice; todavia, não seja como eu
quero, mas como tu queres. E, voltando para os seus
discípulos, achou-os adormecidos; e disse a Pedro: Então nem uma
hora pudeste velar comigo? Vigiai e orai, para que não
entreis em tentação; na verdade, o espírito está pronto, mas a carne
é fraca. E, indo segunda vez, orou, dizendo: Pai meu, se este
cálice não pode passar de mim sem eu o beber, faça-se a tua vontade.
E, voltando, achou-os outra vez adormecidos; porque os seus
olhos estavam pesados. E, deixando-os de novo, foi orar pela
terceira vez, dizendo as mesmas palavras. Então chegou junto
dos seus discípulos, e disse-lhes: Dormi agora, e repousai; eis que
é chegada a hora, e o Filho do homem será entregue nas mãos dos
pecadores. Levantai-vos, partamos; eis que é chegado o que me
trai.

E, estando ele ainda a falar, eis que chegou Judas, um dos doze,
e com ele grande multidão com espadas e varapaus, enviada pelos
príncipes dos sacerdotes e pelos anciãos do povo. E o que o
traía tinha-lhes dado um sinal, dizendo: O que eu beijar é esse;
prendei-o. E logo, aproximando-se de Jesus, disse: Eu te
saúdo, Rabi; e beijou-o. Jesus, porém, lhe disse: Amigo, a
que vieste? Então, aproximando-se eles, lançaram mão de Jesus, e o
prenderam. E eis que um dos que estavam com Jesus, estendendo
a mão, puxou da espada e, ferindo o servo do sumo sacerdote,
cortou-lhe uma orelha. Então Jesus disse-lhe: Embainha a tua
espada; porque todos os que lançarem mão da espada, à espada
morrerão. Ou pensas tu que eu não poderia agora orar a meu
Pai, e que ele não me daria mais de doze legiões de anjos?
Como, pois, se cumpririam as Escrituras, que dizem que assim
convém que aconteça? Então disse Jesus à multidão: Saístes,
como para um salteador, com espadas e varapaus para me prender?
Todos os dias me assentava junto de vós, ensinando no templo, e não
me prendestes. Mas tudo isto aconteceu para que se cumpram as
escrituras dos profetas. Então, todos os discípulos, deixando-o,
fugiram.

E os que prenderam a Jesus o conduziram à casa do sumo sacerdote
Caifás, onde os escribas e os anciãos estavam reunidos. E
Pedro o seguiu de longe, até ao pátio do sumo sacerdote e, entrando,
assentou-se entre os criados, para ver o fim. Ora, os
príncipes dos sacerdotes, e os anciãos, e todo o conselho, buscavam
falso testemunho contra Jesus, para poderem dar-lhe a morte;
e não o achavam; apesar de se apresentarem muitas testemunhas
falsas, não o achavam. Mas, por fim chegaram duas testemunhas
falsas, e disseram: Este disse: Eu posso derrubar o templo de
Deus, e reedificá-lo em três dias. E, levantando-se o sumo
sacerdote, disse-lhe: Não respondes coisa alguma ao que estes depõem
contra ti? Jesus, porém, guardava silêncio. E, insistindo o
sumo sacerdote, disse-lhe: Conjuro-te pelo Deus vivo que nos digas
se tu és o Cristo, o Filho de Deus. Disse-lhe Jesus: Tu o
disseste; digo-vos, porém, que vereis em breve o Filho do homem
assentado à direita do Poder, e vindo sobre as nuvens do céu.
Então o sumo sacerdote rasgou as suas vestes, dizendo:
Blasfemou; para que precisamos ainda de testemunhas? Eis que bem
ouvistes agora a sua blasfêmia. Que vos parece? E eles,
respondendo, disseram: É réu de morte. Então cuspiram-lhe no
rosto e lhe davam punhadas\footnote{Golpe com o punho (``mão
fechada''); murro, soco.}, e outros o esbofeteavam, dizendo:
Profetiza-nos, Cristo, quem é o que te bateu?

Ora, Pedro estava assentado fora, no pátio; e, aproximando-se
dele uma criada, disse: Tu também estavas com Jesus, o galileu.
Mas ele negou diante de todos, dizendo: Não sei o que dizes.
E, saindo para o vestíbulo\footnote{Pátio ou pórtico
exterior, de acesso à entrada principal de uma construção.}, outra
criada o viu, e disse aos que ali estavam: Este também estava com
Jesus, o Nazareno. E ele negou outra vez com juramento: Não
conheço tal homem. E, daí a pouco, aproximando-se os que ali
estavam, disseram a Pedro: Verdadeiramente também tu és deles, pois
a tua fala te denuncia. Então começou ele a praguejar e a
jurar, dizendo: Não conheço esse homem. E imediatamente o galo
cantou. E lembrou-se Pedro das palavras de Jesus, que lhe
dissera: Antes que o galo cante, três vezes me negarás. E, saindo
dali, chorou amargamente.

\medskip

\lettrine{27}\ E, chegando a manhã, todos os príncipes dos
sacerdotes, e os anciãos do povo, formavam juntamente conselho
contra Jesus, para o matarem; e maniatando-o, o levaram e
entregaram ao governador\footnote{SBTB: presidente. Excetuando-se a
ACFiel, as demais versões usam ``governador''. King James, por
exemplo: \emph{governor}. E na Revista e Corrigida, há uma nota de
rodapé: ``Ou governador''. O termo grego aqui é \emph{hegemon}:
``guia, condutor; chefe, comandante; em Roma: imperador. Novo
Testamento: governador de província'' (Dicionário
Grego-Português/Português-Grego, de Isidro Pereira). E vários
versículos há ainda que empregam ``presidente'' em lugar de
``governador''.} Pôncio Pilatos. Então Judas, o que o traíra,
vendo que fora condenado, trouxe, arrependido, as trinta moedas de
prata aos príncipes dos sacerdotes e aos anciãos, dizendo:
Pequei, traindo o sangue inocente. Eles, porém, disseram: Que nos
importa? Isso é contigo. E ele, atirando para o templo as moedas
de prata, retirou-se e foi-se enforcar. E os príncipes dos
sacerdotes, tomando as moedas de prata, disseram: Não é lícito
colocá-las no cofre das ofertas, porque são preço de sangue. E,
tendo deliberado em conselho, compraram com elas o campo de um
oleiro, para sepultura dos estrangeiros. Por isso foi chamado
aquele campo, até ao dia de hoje, Campo de Sangue. Então se
realizou o que vaticinara o profeta Jeremias: Tomaram as trinta
moedas de prata, preço do que foi avaliado, que certos filhos de
Israel avaliaram, e deram-nas pelo campo do oleiro, segundo o
que o Senhor determinou.

E foi Jesus apresentado ao governador\footnote{SBTB:
presidente.}, e o governador\footnote{idem.} o interrogou, dizendo:
És tu o Rei dos Judeus? E disse-lhe Jesus: Tu o dizes. E,
sendo acusado pelos príncipes dos sacerdotes e pelos anciãos, nada
respondeu. Disse-lhe então Pilatos: Não ouves quanto
testificam contra ti? E nem uma palavra lhe respondeu, de
sorte que o governador\footnote{idem.} estava muito maravilhado.
Ora, por ocasião da festa, costumava o
governador\footnote{idem.} soltar um preso, escolhendo o povo aquele
que quisesse. E tinham então um preso bem conhecido, chamado
Barrabás. Portanto, estando eles reunidos, disse-lhes
Pilatos: Qual quereis que vos solte? Barrabás, ou Jesus, chamado
Cristo? Porque sabia que por inveja o haviam entregado.
E, estando ele assentado no tribunal, sua mulher mandou-lhe
dizer: Não entres na questão desse justo, porque num sonho muito
sofri por causa dele. Mas os príncipes dos sacerdotes e os
anciãos persuadiram à multidão que pedisse Barrabás e matasse Jesus.
E, respondendo o governador\footnote{idem.}, disse-lhes: Qual
desses dois quereis vós que eu solte? E eles disseram: Barrabás.
Disse-lhes Pilatos: Que farei então de Jesus, chamado Cristo?
Disseram-lhe todos: Seja crucificado. O
governador\footnote{idem.}, porém, disse: Mas que mal fez ele? E
eles mais clamavam, dizendo: Seja crucificado. Então Pilatos,
vendo que nada aproveitava, antes o tumulto crescia, tomando água,
lavou as mãos diante da multidão, dizendo: Estou inocente do sangue
deste justo. Considerai isso. E, respondendo todo o povo,
disse: O seu sangue caia sobre nós e sobre nossos filhos.

Então soltou-lhes Barrabás, e, tendo mandado açoitar a Jesus,
entregou-o para ser crucificado. E logo os soldados do
governador\footnote{idem.}, conduzindo Jesus à audiência, reuniram
junto dele toda a coorte\footnote{Cada uma das dez unidades de uma
legião do exército romano.}. E, despindo-o, o cobriram com
uma capa de escarlate; e, tecendo uma coroa de espinhos,
puseram-lha na cabeça, e em sua mão direita uma cana; e, ajoelhando
diante dele, o escarneciam, dizendo: Salve, Rei dos judeus.
E, cuspindo nele, tiraram-lhe a cana, e batiam-lhe com ela na
cabeça. E, depois de o haverem escarnecido, tiraram-lhe a
capa, vestiram-lhe as suas vestes e o levaram para ser crucificado.
E, quando saíam, encontraram um homem cireneu, chamado Simão,
a quem constrangeram a levar a sua cruz.

E, chegando ao lugar chamado Gólgota, que se diz: Lugar da
Caveira, deram-lhe a beber vinagre misturado com fel; mas
ele, provando-o, não quis beber. E, havendo-o crucificado,
repartiram as suas vestes, lançando sortes, para que se cumprisse o
que foi dito pelo profeta: Repartiram entre si as minhas vestes, e
sobre a minha túnica lançaram sortes\footnote{Sl 28.18.}. E,
assentados, o guardavam ali. E por cima da sua cabeça puseram
escrita a sua acusação: ESTE É JESUS, O REI DOS JUDEUS. E
foram crucificados com ele dois salteadores, um à direita, e outro à
esquerda. E os que passavam blasfemavam dele, meneando as
cabeças, e dizendo: Tu, que destróis o templo, e em três dias
o reedificas, salva-te a ti mesmo. Se és Filho de Deus, desce da
cruz. E da mesma maneira também os príncipes dos sacerdotes,
com os escribas, e anciãos, e fariseus, escarnecendo, diziam:
Salvou os outros, e a si mesmo não pode salvar-se. Se é o Rei
de Israel, desça agora da cruz, e crê-lo-emos. Confiou em
Deus; livre-o agora, se o ama; porque disse: Sou Filho de Deus.
E o mesmo lhe lançaram também em rosto os salteadores que com
ele estavam crucificados. E desde a hora sexta houve trevas
sobre toda a terra, até à hora nona. E perto da hora nona
exclamou Jesus em alta voz, dizendo: Eli, Eli, lamá sabactâni; isto
é, Deus meu, Deus meu, por que me desamparaste\footnote{Sl 22.1.}?
E alguns dos que ali estavam, ouvindo isto, diziam: Este
chama por Elias, e logo um deles, correndo, tomou uma
esponja, e embebeu-a em vinagre, e, pondo-a numa cana, dava-lhe de
beber. Os outros, porém, diziam: Deixa, vejamos se Elias vem
livrá-lo.

E Jesus, clamando outra vez com grande voz, rendeu o espírito.
E eis que o véu do templo se rasgou em dois, de alto a baixo;
e tremeu a terra, e fenderam-se as pedras; e abriram-se os
sepulcros, e muitos corpos de santos que dormiam foram
ressuscitados; e, saindo dos sepulcros, depois da
ressurreição dele, entraram na cidade santa, e apareceram a muitos.
E o centurião e os que com ele guardavam a Jesus, vendo o
terremoto, e as coisas que haviam sucedido, tiveram grande temor, e
disseram: Verdadeiramente este era o\footnote{SBTB: ``era Filho de
Deus''. Com a palavra, aqui, o irmão Hélio de Menezes: ``\ldots
Verdadeiramente este era O (artigo definido masculino e singular)
FILHO DE DEUS''. O Texto Recebido diz: `alhywv 230 (ADV yeou 2316
(N-GSM) uiov 5207 (N-NSM) hn 2258 5713 (V-IXI-3S) outov 3778
(D-NSM)'. A construção e o contexto exigem o artigo definido. A
tradução literal tem que ser `em-a-verdade, (o) Filho de Deus era
este.' Cristo não é `um filho de Deus' (a NVI está definitivamente
errada), não é `um dos filhos de Deus' (os crentes e os anjos também
o são), não é `um filho de um deus' (deus, com inicial minúscula)
(NASV 1963): Cristo é `\textcolor{red}{o} Filho de Deus', é o
total+exclusivo Filho de o total $+$ exclusivo Deus. O TC diz:
`alhywv uios yeou hn outov.' Mas, mesmo assim, também a construção e
o contexto recomendam o arquivo definido.''} Filho de Deus. E
estavam ali, olhando de longe, muitas mulheres que tinham seguido
Jesus desde a Galiléia, para o servir; entre as quais estavam
Maria Madalena, e Maria, mãe de Tiago e de José, e a mãe dos filhos
de Zebedeu.

E, vinda já a tarde, chegou um homem rico, de Arimatéia, por nome
José, que também era discípulo de Jesus. Este foi ter com
Pilatos, e pediu-lhe o corpo de Jesus. Então Pilatos mandou que o
corpo lhe fosse dado. E José, tomando o corpo, envolveu-o num
fino e limpo lençol, e o pôs no seu sepulcro novo, que havia
aberto em rocha, e, rodando uma grande pedra para a porta do
sepulcro, retirou-se. E estavam ali Maria Madalena e a outra
Maria, assentadas defronte do sepulcro. E no dia seguinte,
que é o dia depois da Preparação, reuniram-se os príncipes dos
sacerdotes e os fariseus em casa de Pilatos, dizendo: Senhor,
lembramo-nos de que aquele enganador, vivendo ainda, disse: Depois
de três dias ressuscitarei. Manda, pois, que o sepulcro seja
guardado com segurança até ao terceiro dia, não se dê o caso que os
seus discípulos vão de noite, e o furtem, e digam ao povo:
Ressuscitou dentre os mortos; e assim o último erro será pior do que
o primeiro. E disse-lhes Pilatos: Tendes a guarda; ide,
guardai-o como entenderdes. E, indo eles, seguraram o
sepulcro com a guarda, selando a pedra.

\medskip

\lettrine{28}\ E, no fim do sábado, quando já despontava o
primeiro dia da semana, Maria Madalena e a outra Maria foram ver o
sepulcro. E eis que houvera um grande terremoto, porque um anjo
do Senhor, descendo do céu, chegou, removendo a pedra da porta, e
sentou-se sobre ela. E o seu aspecto era como um relâmpago, e as
suas vestes brancas como neve. E os guardas, com medo dele,
ficaram muito assombrados, e como mortos. Mas o anjo,
respondendo, disse às mulheres: Não tenhais medo; pois eu sei que
buscais a Jesus, que foi crucificado. Ele não está aqui, porque
já ressuscitou, como havia dito. Vinde, vede o lugar onde o Senhor
jazia. Ide pois, imediatamente, e dizei aos seus discípulos que
já ressuscitou dentre os mortos. E eis que ele vai adiante de vós
para a Galiléia; ali o vereis. Eis que eu vo-lo tenho dito. E,
saindo elas pressurosamente do sepulcro, com temor e grande alegria,
correram a anunciá-lo aos seus discípulos. E, indo elas a dar as
novas aos seus discípulos, eis que Jesus lhes sai ao encontro,
dizendo: Eu vos saúdo. E elas, chegando, abraçaram os seus pés, e o
adoraram. Então Jesus disse-lhes: Não temais; ide dizer a
meus irmãos que vão à Galiléia, e lá me verão.

E, quando iam, eis que alguns da guarda, chegando à cidade,
anunciaram aos príncipes dos sacerdotes todas as coisas que haviam
acontecido. E, congregados eles com os anciãos, e tomando
conselho entre si, deram muito dinheiro aos soldados,
dizendo: Dizei: Vieram de noite os seus discípulos e,
dormindo nós, o furtaram. E, se isto chegar a ser ouvido pelo
governador\footnote{SBTB: presidente.}, nós o persuadiremos, e vos
poremos em segurança. E eles, recebendo o dinheiro, fizeram
como estavam instruídos. E foi divulgado este dito entre os judeus,
até ao dia de hoje.

E os onze discípulos partiram para a Galiléia, para o monte que
Jesus lhes tinha designado. E, quando o viram, o adoraram;
mas alguns duvidaram. E, chegando-se Jesus, falou-lhes,
dizendo:  É-me dado todo o poder no céu e na terra. Portanto
ide, fazei discípulos de todas as nações, batizando-os em nome do
Pai, e do Filho, e do Espírito Santo; ensinando-os a guardar
todas as coisas que eu vos tenho mandado; e eis que eu estou
convosco todos os dias, até a consumação dos séculos. Amém.



