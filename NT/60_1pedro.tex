\addchap{Primeira Epístola de Pedro}

\lettrine{1} Pedro, apóstolo de Jesus Cristo, aos estrangeiros
dispersos no Ponto, Galácia, Capadócia, Ásia e Bitínia; eleitos
segundo a presciência de Deus Pai, em santificação do Espírito, para
a obediência e aspersão do sangue de Jesus Cristo: Graça e paz vos
sejam multiplicadas.

Bendito seja o Deus e Pai de nosso Senhor Jesus Cristo que,
segundo a sua grande misericórdia, nos gerou de novo para uma viva
esperança, pela ressurreição de Jesus Cristo dentre os mortos,
para uma herança incorruptível, incontaminável, e que não se
pode murchar, guardada nos céus para vós, que mediante a fé
estais guardados na virtude de Deus para a salvação, já prestes para
se revelar no último tempo.

Em que vós grandemente vos alegrais, ainda que agora importa,
sendo necessário, que estejais por um pouco contristados com várias
tentações, para que a prova da vossa fé, muito mais preciosa do
que o ouro que perece e é provado pelo fogo, se ache em louvor, e
honra, e glória, na revelação de Jesus Cristo; ao qual, não o
havendo visto, amais; no qual, não o vendo agora, mas crendo, vos
alegrais com gozo inefável e glorioso; alcançando o fim da vossa
fé, a salvação das vossas almas.

Da qual salvação inquiriram e trataram diligentemente os profetas
que profetizaram da graça que vos foi dada, indagando que
tempo ou que ocasião de tempo o Espírito de Cristo, que estava
neles, indicava, anteriormente testificando os sofrimentos que a
Cristo haviam de vir, e a glória que se lhes havia de seguir.
Aos quais foi revelado que, não para si mesmos, mas para nós,
eles ministravam estas coisas que agora vos foram anunciadas por
aqueles que, pelo Espírito Santo enviado do céu, vos pregaram o
evangelho; para as quais coisas os anjos desejam bem atentar.

Portanto, cingindo os lombos do vosso entendimento, sede sóbrios,
e esperai inteiramente na graça que se vos ofereceu na revelação de
Jesus Cristo; como filhos obedientes, não vos conformando com
as concupiscências que antes havia em vossa ignorância; mas,
como é santo aquele que vos chamou, sede vós também santos em toda a
vossa maneira de viver; porquanto está escrito: Sede santos,
porque eu sou santo. E, se invocais por Pai aquele que, sem
acepção de pessoas, julga segundo a obra de cada um, andai em temor,
durante o tempo da vossa peregrinação, sabendo que não foi
com coisas corruptíveis, como prata ou ouro, que fostes resgatados
da vossa vã maneira de viver que por tradição recebestes dos vossos
pais, mas com o precioso sangue de Cristo, como de um
cordeiro imaculado e incontaminado, o qual, na verdade, em
outro tempo foi conhecido, ainda antes da fundação do mundo, mas
manifestado nestes últimos tempos por amor de vós; e por ele
credes em Deus, que o ressuscitou dentre os mortos, e lhe deu
glória, para que a vossa fé e esperança estivessem em Deus;
purificando as vossas almas pelo Espírito na obediência à
verdade, para o amor fraternal, não fingido; amai-vos ardentemente
uns aos outros com um coração puro; sendo de novo gerados,
não de semente corruptível, mas da incorruptível, pela palavra de
Deus, viva, e que permanece para sempre.

Porque toda a carne é como a erva, e toda a glória do homem como
a flor da erva. Secou-se a erva, e caiu a sua flor; mas a
palavra do Senhor permanece para sempre. E esta é a palavra que
entre vós foi evangelizada.

\medskip

\lettrine{2} Deixando, pois, toda a malícia, e todo o engano,
e fingimentos, e invejas, e todas as murmurações, desejai
afetuosamente, como meninos novamente nascidos, o leite racional,
não falsificado, para que por ele vades crescendo; se é que já
provastes que o Senhor é benigno.

E, chegando-vos para ele, pedra viva, reprovada, na verdade, pelos
homens, mas para com Deus eleita e preciosa, vós também, como
pedras vivas, sois edificados casa espiritual e sacerdócio santo,
para oferecer sacrifícios espirituais agradáveis a Deus por Jesus
Cristo. Por isso também na Escritura se contém: Eis que ponho em
Sião a pedra principal da esquina, eleita e preciosa; e quem nela
crer não será confundido. E assim para vós, os que credes, é
preciosa, mas, para os rebeldes, a pedra que os edificadores
reprovaram, essa foi a principal da esquina, e uma pedra de
tropeço e rocha de escândalo, para aqueles que tropeçam na palavra,
sendo desobedientes; para o que também foram destinados. Mas vós
sois a geração eleita, o sacerdócio real, a nação santa, o povo
adquirido, para que anuncieis as virtudes daquele que vos chamou das
trevas para a sua maravilhosa luz; vós, que em outro tempo
não éreis povo, mas agora sois povo de Deus; que não tínheis
alcançado misericórdia, mas agora alcançastes misericórdia.
Amados, peço-vos, como a peregrinos e forasteiros, que vos
abstenhais das concupiscências carnais que combatem contra a alma;
tendo o vosso viver honesto entre os gentios; para que,
naquilo em que falam mal de vós, como de malfeitores, glorifiquem a
Deus no dia da visitação, pelas boas obras que em vós observem.

Sujeitai-vos, pois, a toda a ordenação humana por amor do Senhor;
quer ao rei, como superior; quer aos governadores, como por
ele enviados para castigo dos malfeitores, e para louvor dos que
fazem o bem. Porque assim é a vontade de Deus, que, fazendo
bem, tapeis a boca à ignorância dos homens insensatos; como
livres, e não tendo a liberdade por cobertura da malícia, mas como
servos de Deus. Honrai a todos. Amai a fraternidade. Temei a
Deus. Honrai ao rei. Vós, servos, sujeitai-vos com todo o
temor aos senhores, não somente aos bons e humanos, mas também aos
maus. Porque é coisa agradável, que alguém, por causa da
consciência para com Deus, sofra agravos, padecendo injustamente.
Porque, que glória será essa, se, pecando, sois esbofeteados
e sofreis? Mas se, fazendo o bem, sois afligidos e o sofreis, isso é
agradável a Deus. Porque para isto sois chamados; pois também
Cristo padeceu por nós, deixando-nos o exemplo, para que sigais as
suas pisadas. O qual não cometeu pecado, nem na sua boca se
achou engano. O qual, quando o injuriavam, não injuriava, e
quando padecia não ameaçava, mas entregava-se àquele que julga
justamente; levando ele mesmo em seu corpo os nossos pecados
sobre o madeiro, para que, mortos para os pecados, pudéssemos viver
para a justiça; e pelas suas feridas fostes sarados. Porque
éreis como ovelhas desgarradas; mas agora tendes voltado ao Pastor e
Bispo das vossas almas.

\medskip

\lettrine{3} Semelhantemente, vós, mulheres, sede sujeitas aos
vossos próprios maridos; para que também, se alguns não obedecem à
palavra, pelo porte de suas mulheres sejam ganhos sem palavra;
considerando a vossa vida casta, em temor. O enfeite delas
não seja o exterior, no frisado dos cabelos, no uso de jóias de
ouro, na compostura dos vestidos; mas o homem encoberto no
coração; no incorruptível traje de um espírito manso e quieto, que é
precioso diante de Deus. Porque assim se adornavam também
antigamente as santas mulheres que esperavam em Deus, e estavam
sujeitas aos seus próprios maridos; como Sara obedecia a Abraão,
chamando-lhe senhor; da qual vós sois filhas, fazendo o bem, e não
temendo nenhum espanto. Igualmente vós, maridos, coabitai com
elas com entendimento, dando honra à mulher, como vaso mais fraco;
como sendo vós os seus co-herdeiros da graça da vida; para que não
sejam impedidas as vossas orações.

E, finalmente, sede todos de um mesmo sentimento, compassivos,
amando os irmãos, entranhavelmente misericordiosos e afáveis.
Não tornando mal por mal, ou injúria por injúria; antes, pelo
contrário, bendizendo; sabendo que para isto fostes chamados, para
que por herança alcanceis a bênção. Porque quem quer amar a
vida, e ver os dias bons, refreie a sua língua do mal, e os seus
lábios não falem engano. Aparte-se do mal, e faça o bem;
busque a paz, e siga-a. Porque os olhos do Senhor estão sobre
os justos, e os seus ouvidos atentos às suas orações; mas o rosto do
Senhor é contra os que fazem o mal. E qual é aquele que vos
fará mal, se fordes zelosos do bem? Mas também, se padecerdes
por amor da justiça, sois bem-aventurados. E não temais com medo
deles, nem vos turbeis; antes, santificai ao Senhor Deus em
vossos corações; e estai sempre preparados para responder com
mansidão e temor a qualquer que vos pedir a razão da esperança que
há em vós, tendo uma boa consciência, para que, naquilo em
que falam mal de vós, como de malfeitores, fiquem confundidos os que
blasfemam do vosso bom porte em Cristo. Porque melhor é que
padeçais fazendo bem (se a vontade de Deus assim o quer), do que
fazendo mal.

Porque também Cristo padeceu uma vez pelos pecados, o justo pelos
injustos, para levar-nos a Deus; mortificado, na verdade, na carne,
mas vivificado pelo Espírito; no qual também foi, e pregou
aos espíritos em prisão; os quais noutro tempo foram
rebeldes, quando a longanimidade de Deus esperava nos dias de Noé,
enquanto se preparava a arca; na qual poucas (isto é, oito) almas se
salvaram pela água; que também, como uma verdadeira figura,
agora vos salva, o batismo, não do despojamento da imundícia da
carne, mas da indagação de uma boa consciência para com Deus, pela
ressurreição de Jesus Cristo; o qual está à destra de Deus,
tendo subido ao céu, havendo-se-lhe sujeitado os anjos, e as
autoridades, e as potências.

\medskip

\lettrine{4} Ora, pois, já que Cristo padeceu por nós na
carne, armai-vos também vós com este pensamento, que aquele que
padeceu na carne já cessou do pecado; para que, no tempo que vos
resta na carne, não vivais mais segundo as concupiscências dos
homens, mas segundo a vontade de Deus. Porque é bastante que no
tempo passado da vida fizéssemos a vontade dos gentios, andando em
dissoluções, concupiscências, borrachices\footnote{Ou borracheira:
Palavras ou comportamento de bêbado. Grosseria, indelicadeza.
Despropósito, disparate. Obra malfeita.}, glutonarias, bebedices e
abomináveis idolatrias; e acham estranho não correrdes com eles
no mesmo desenfreamento de dissolução, blasfemando de vós. Os
quais hão de dar conta ao que está preparado para julgar os vivos e
os mortos. Porque por isto foi pregado o evangelho também aos
mortos, para que, na verdade, fossem julgados segundo os homens na
carne, mas vivessem segundo Deus em espírito.

E já está próximo o fim de todas as coisas; portanto sede sóbrios
e vigiai em oração. Mas, sobretudo, tende ardente amor uns para
com os outros; porque o amor cobrirá a multidão de pecados.
Sendo hospitaleiros uns para com os outros, sem murmurações,
cada um administre aos outros o dom como o recebeu, como bons
despenseiros da multiforme graça de Deus. Se alguém falar,
fale segundo as palavras de Deus; se alguém administrar, administre
segundo o poder que Deus dá; para que em tudo Deus seja glorificado
por Jesus Cristo, a quem pertence a glória e poder para todo o
sempre. Amém.

Amados, não estranheis a ardente prova que vem sobre vós para vos
tentar, como se coisa estranha vos acontecesse; mas
alegrai-vos no fato de serdes participantes das aflições de Cristo,
para que também na revelação da sua glória vos regozijeis e
alegreis. Se pelo nome de Cristo sois
vituperados\footnote{Vituperar: Tratar com vitupérios; injuriar,
insultar, afrontar. Repreender com dureza. Censurar, desaprovar.
Desprezar, desestimar, menoscabar.}, bem-aventurados sois, porque
sobre vós repousa o Espírito da glória e de Deus; quanto a eles, é
ele, sim, blasfemado, mas quanto a vós, é glorificado. Que
nenhum de vós padeça como homicida, ou ladrão, ou malfeitor, ou como
o que se entremete em negócios alheios; mas, se padece como
cristão, não se envergonhe, antes glorifique a Deus nesta parte.
Porque já é tempo que comece o julgamento pela casa de Deus;
e, se primeiro começa por nós, qual será o fim daqueles que são
desobedientes ao evangelho de Deus? E, se o justo apenas se
salva, onde aparecerá o ímpio e o pecador? Portanto também os
que padecem segundo a vontade de Deus encomendem-lhe as suas almas,
como ao fiel Criador, fazendo o bem.

\medskip

\lettrine{5} Aos presbíteros, que estão entre vós, admoesto
eu, que sou também presbítero com eles, e testemunha das aflições de
Cristo, e participante da glória que se há de revelar:
Apascentai o rebanho de Deus, que está entre vós, tendo cuidado
dele, não por força, mas voluntariamente; nem por torpe ganância,
mas de ânimo pronto; nem como tendo domínio sobre a herança de
Deus, mas servindo de exemplo ao rebanho. E, quando aparecer o
Sumo Pastor, alcançareis a incorruptível coroa da glória.

Semelhantemente vós jovens, sede sujeitos aos anciãos; e sede
todos sujeitos uns aos outros, e revesti-vos de humildade, porque
Deus resiste aos soberbos, mas dá graça aos humildes.
Humilhai-vos, pois, debaixo da potente mão de Deus, para que a
seu tempo vos exalte; lançando sobre ele toda a vossa ansiedade,
porque ele tem cuidado de vós.

Sede sóbrios; vigiai; porque o diabo, vosso adversário, anda em
derredor, bramando como leão, buscando a quem possa tragar; ao
qual resisti firmes na fé, sabendo que as mesmas aflições se cumprem
entre os vossos irmãos no mundo.

E o Deus de toda a graça, que em Cristo Jesus vos chamou à sua
eterna glória, depois de haverdes padecido um pouco, ele mesmo vos
aperfeiçoará, confirmará, fortificará e fortalecerá. A ele
seja a glória e o poderio para todo o sempre. Amém. Por
Silvano, vosso fiel irmão, como cuido, escrevi abreviadamente,
exortando e testificando que esta é a verdadeira graça de Deus, na
qual estais firmes. A vossa co-eleita em Babilônia vos saúda,
e meu filho Marcos. Saudai-vos uns aos outros com ósculo de
amor. Paz seja com todos vós que estais em Cristo Jesus. Amém.

