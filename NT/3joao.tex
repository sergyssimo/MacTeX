\thispagestyle{empty}

\chapter*{Terceira Epístola de João}

O presbítero ao amado Gaio, a quem em verdade eu amo. Amado,
desejo que te vá bem em todas as coisas, e que tenhas saúde, assim
como bem vai a tua alma.

Porque muito me alegrei quando os irmãos vieram, e testificaram da
tua verdade, como tu andas na verdade. Não tenho maior gozo do
que este, o de ouvir que os meus filhos andam na verdade. Amado,
procedes fielmente em tudo o que fazes para com os irmãos, e para
com os estranhos, que em presença da igreja testificaram do teu
amor; aos quais, se conduzires como é digno para com Deus, bem
farás; porque pelo seu Nome saíram, nada tomando dos gentios.
Portanto, aos tais devemos receber, para que sejamos
cooperadores da verdade.

Tenho escrito à igreja; mas Diótrefes, que procura ter entre eles
o primado, não nos recebe. Por isso, se eu for, trarei à
memória as obras que ele faz, proferindo contra nós palavras
maliciosas; e, não contente com isto, não recebe os irmãos, e impede
os que querem recebê-los, e os lança fora da igreja. Amado,
não sigas o mal, mas o bem. Quem faz o bem é de Deus; mas quem faz o
mal não tem visto a Deus.

Todos dão testemunho de Demétrio, até a mesma verdade; e também
nós testemunhamos; e vós bem sabeis que o nosso testemunho é
verdadeiro. Tinha muito que escrever, mas não quero
escrever-te com tinta e pena. Espero, porém, ver-te
brevemente, e falaremos face a face. Paz seja contigo. Os
amigos te saúdam. Saúda os amigos pelo seu nome.

