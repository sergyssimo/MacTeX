\addchap{Primeira Epístola de João}

\lettrine{1} O que era desde o princípio, o que ouvimos, o que
vimos com os nossos olhos, o que temos contemplado, e as nossas mãos
tocaram da Palavra da vida (porque a vida foi manifestada, e nós
a vimos, e testificamos dela, e vos anunciamos a vida eterna, que
estava com o Pai, e nos foi manifestada); o que vimos e ouvimos,
isso vos anunciamos, para que também tenhais comunhão conosco; e a
nossa comunhão é com o Pai, e com seu Filho Jesus Cristo. Estas
coisas vos escrevemos, para que o vosso gozo se cumpra.

E esta é a mensagem que dele ouvimos, e vos anunciamos: que Deus é
luz, e não há nele trevas nenhumas. Se dissermos que temos
comunhão com ele, e andarmos em trevas, mentimos, e não praticamos a
verdade. Mas, se andarmos na luz, como ele na luz está, temos
comunhão uns com os outros, e o \textbf{sangue de Jesus Cristo, seu
Filho, nos purifica de todo o pecado}.

Se dissermos que não temos pecado, enganamo-nos a nós mesmos, e
não há verdade em nós. Se confessarmos os nossos pecados, ele é
fiel e justo para nos perdoar os pecados, e nos purificar de toda a
injustiça. Se dissermos que não pecamos, fazemo-lo mentiroso,
e a sua palavra não está em nós.

\medskip

\lettrine{2} Meus filhinhos, estas coisas vos escrevo, para
que não pequeis; e, se alguém pecar, temos um \textbf{Advogado para
com o Pai, Jesus Cristo, o justo}. E ele é a propiciação pelos
nossos pecados, e não somente pelos nossos, mas também pelos de todo
o mundo.

E nisto sabemos que o conhecemos: se guardarmos os seus
mandamentos. Aquele que diz: Eu o conheço\footnote{SBTB: Eu
conheço-o}, e não guarda os seus mandamentos, é mentiroso, e nele
não está a verdade. Mas qualquer que guarda a sua palavra, o
amor de Deus está nele verdadeiramente aperfeiçoado; nisto
conhecemos que estamos nele. Aquele que diz que está nele,
também deve andar como ele andou.

Irmãos, não vos escrevo mandamento novo, mas o mandamento antigo,
que desde o princípio tivestes. Este mandamento antigo é a palavra
que desde o princípio ouvistes. Outra vez vos escrevo um
mandamento novo, que é verdadeiro nele e em vós; porque vão passando
as trevas, e já a verdadeira luz ilumina. Aquele que diz que
está na luz, e odeia a seu irmão, até agora está em trevas.
Aquele que ama a seu irmão está na luz, e nele não há
escândalo. Mas aquele que odeia a seu irmão está em trevas, e
anda em trevas, e não sabe para onde deva ir; porque as trevas lhe
cegaram os olhos.

Filhinhos, escrevo-vos, porque pelo seu nome vos são perdoados os
pecados. Pais, escrevo-vos, porque conhecestes aquele que é
desde o princípio. Jovens, escrevo-vos, porque vencestes o maligno.
Eu vos escrevi, filhos, porque conhecestes o Pai. Eu vos
escrevi, pais, porque já conhecestes aquele que é desde o princípio.
Eu vos escrevi, jovens, porque sois fortes, e a palavra de Deus está
em vós, e já vencestes o maligno. Não ameis o mundo, nem o
que no mundo há. Se alguém ama o mundo, o amor do Pai não está nele.
Porque tudo o que há no mundo, a concupiscência da carne, a
concupiscência dos olhos e a soberba da vida, não é do Pai, mas do
mundo. E o mundo passa, e a sua concupiscência; mas aquele
que faz a vontade de Deus permanece para sempre.

Filhinhos, é já a última hora; e, como ouvistes que vem o
anticristo, também agora muitos se têm feito anticristos, por onde
conhecemos que é já a última hora. Saíram de nós, mas não
eram de nós; porque, se fossem de nós, ficariam conosco; mas isto é
para que se manifestasse que não são todos de nós.

E vós tendes a unção do Santo, e sabeis tudo. Não vos
escrevi porque não soubésseis a verdade, mas porque a sabeis, e
porque nenhuma mentira vem da verdade. Quem é o mentiroso,
senão aquele que nega que Jesus é o Cristo? É o anticristo esse
mesmo que nega o Pai e o Filho. Qualquer que nega o Filho,
também não tem o Pai; mas aquele que confessa o Filho, tem também o
Pai. Portanto, o que desde o princípio ouvistes permaneça em
vós. Se em vós permanecer o que desde o princípio ouvistes, também
permanecereis no Filho e no Pai. E esta é a promessa que ele
nos fez: \textbf{a vida eterna}. Estas coisas vos escrevi
acerca dos que vos enganam. E a unção que vós recebestes
dele, fica em vós, e não tendes necessidade de que alguém vos
ensine; mas, como a sua unção vos ensina todas as coisas, e é
verdadeira, e não é mentira, como ela vos ensinou, assim nele
permanecereis.

E agora, filhinhos, permanecei nele; para que, quando ele se
manifestar, tenhamos confiança, e não sejamos confundidos por ele na
sua vinda. Se sabeis que ele é justo, sabeis que todo aquele
que pratica a justiça é nascido dele.

\medskip

\lettrine{3} Vede quão grande amor nos tem concedido o Pai,
que fôssemos chamados \textbf{filhos de Deus}. Por isso o mundo não
nos conhece; porque não o conhece a ele. Amados, agora somos
filhos de Deus, e ainda não é manifestado o que havemos de ser. Mas
sabemos que, quando ele se manifestar, seremos semelhantes a ele;
porque assim como é o veremos. E qualquer que nele tem esta
esperança purifica-se a si mesmo, como também ele é puro.

Qualquer que comete pecado, também comete iniqüidade; porque o
pecado é iniqüidade. E bem sabeis que ele se manifestou para
tirar os nossos pecados; e nele não há pecado. Qualquer que
permanece nele não peca; qualquer que peca não o viu nem o conheceu.
Filhinhos, ninguém vos engane. Quem pratica justiça é justo,
assim como ele é justo. Quem comete o pecado é do diabo; porque
o diabo peca desde o princípio. Para isto o Filho de Deus se
manifestou: para desfazer as obras do diabo. Qualquer que é
nascido de Deus não comete pecado; porque a sua semente permanece
nele; e não pode pecar, porque é nascido de Deus. Nisto são
manifestos os filhos de Deus, e os filhos do diabo. Qualquer que não
pratica a justiça, e não ama a seu irmão, não é de Deus.

Porque esta é a mensagem que ouvistes desde o princípio:
\textbf{que nos amemos uns aos outros}. Não como Caim, que
era do maligno, e matou a seu irmão. E por que causa o matou? Porque
as suas obras eram más e as de seu irmão justas. Meus irmãos,
não vos maravilheis, se o mundo vos odeia.

Nós sabemos que passamos da morte para a vida, porque amamos os
irmãos. Quem não ama a seu irmão permanece na morte. Qualquer
que odeia a seu irmão é homicida. E vós sabeis que nenhum homicida
tem a vida eterna permanecendo nele. \textbf{Conhecemos o
amor nisto}: que ele deu a sua vida por nós, e nós devemos dar a
vida pelos irmãos. Quem, pois, tiver bens do mundo, e, vendo
o seu irmão necessitado, lhe cerrar as suas entranhas, como estará
nele o amor de Deus? Meus filhinhos, não amemos de palavra,
nem de língua, mas por obra e em verdade. E nisto conhecemos
que somos da verdade, e diante dele asseguraremos nossos corações.

Sabendo que, se o nosso coração nos condena, maior é Deus do que
o nosso coração, e conhece todas as coisas. Amados, se o
nosso coração não nos condena, temos confiança para com Deus;
e qualquer coisa que lhe pedirmos, dele a receberemos, porque
guardamos os seus mandamentos, e fazemos o que é agradável à sua
vista.

E \textbf{o seu mandamento é este}: que creiamos no nome de seu
Filho Jesus Cristo, e nos amemos uns aos outros, segundo o seu
mandamento. E aquele que guarda os seus mandamentos nele
está, e ele nele. E nisto conhecemos que ele está em nós, pelo
\textbf{Espírito} que nos tem dado.

\medskip

\lettrine{4} Amados, não creiais a todo o espírito, mas provai
se os espíritos são de Deus, porque já muitos falsos profetas se têm
levantado no mundo. Nisto conhecereis o Espírito de Deus: Todo o
espírito que confessa que \textbf{Jesus Cristo veio em carne} é de
Deus; e todo o espírito que não confessa que Jesus Cristo veio
em carne não é de Deus; mas este é o espírito do anticristo, do qual
já ouvistes que há de vir, e eis que já está no mundo.

Filhinhos, sois de Deus, e já os tendes vencido; porque
\textbf{maior é o que está em vós do que o que está no mundo}.
Do mundo são, por isso falam do mundo, e o mundo os ouve.
Nós somos de Deus; aquele que conhece a Deus ouve-nos; aquele
que não é de Deus não nos ouve. Nisto conhecemos nós o espírito da
verdade e o espírito do erro.

Amados, amemo-nos uns aos outros; porque o amor é de Deus; e
qualquer que ama é nascido de Deus e conhece a Deus. Aquele que
não ama não conhece a Deus; porque \textbf{Deus é amor}. Nisto
se manifesta o amor de Deus para conosco: que Deus enviou seu Filho
unigênito ao mundo, para que por ele vivamos. Nisto está o
amor, não em que nós tenhamos amado a Deus, mas em que ele nos amou
a nós, e enviou seu Filho para propiciação pelos nossos pecados.
Amados, se Deus assim nos amou, também nós devemos amar uns
aos outros. Ninguém jamais viu a Deus; se nos amamos uns aos
outros, Deus está em nós, e em nós é perfeito o seu amor.
Nisto conhecemos que estamos nele, e ele em nós, pois que nos
deu do seu Espírito.

E vimos, e testificamos que \textbf{o Pai enviou seu Filho para
Salvador do mundo}. Qualquer que confessar que Jesus é o
Filho de Deus, Deus está nele, e ele em Deus. E nós
conhecemos, e cremos no amor que Deus nos tem. \textbf{Deus é amor};
e quem está em amor está em Deus, e Deus nele.

Nisto é perfeito o amor para conosco, para que no dia do juízo
tenhamos confiança; porque, qual ele é, somos nós também neste
mundo. No amor não há temor, antes o perfeito amor lança fora
o temor; porque o temor tem consigo a pena, e o que teme não é
perfeito em amor. Nós o amamos a ele porque ele nos amou
primeiro. Se alguém diz: Eu amo a Deus, e odeia a seu irmão,
é mentiroso. Pois quem não ama a seu irmão, ao qual viu, como pode
amar a Deus, a quem não viu? E dele temos este mandamento:
que quem ama a Deus, ame também a seu irmão.

\medskip

\lettrine{5} Todo aquele que crê que Jesus é o Cristo,
é nascido de Deus; e todo aquele que ama ao que o gerou também ama
ao que dele é nascido. Nisto conhecemos que amamos os filhos de
Deus, quando amamos a Deus e guardamos os seus mandamentos.
Porque este é o amor de Deus: que guardemos os seus mandamentos;
e os seus mandamentos não são pesados. Porque todo o que é
nascido de Deus vence o mundo; \textbf{e esta é a vitória que vence
o mundo, a nossa fé}. Quem é que vence o mundo, senão aquele que
crê que Jesus é o Filho de Deus?

Este é aquele que veio por água e sangue, isto é, Jesus Cristo;
não só por água, mas por água e por sangue. E o Espírito é o que
testifica, porque o Espírito é a verdade. Porque três são os que
testificam no céu: o Pai, a Palavra, e o Espírito Santo; e estes
três são um. E três são os que testificam na terra: o Espírito,
e a água e o sangue; e estes três concordam num. Se recebemos o
testemunho dos homens, o testemunho de Deus é maior; porque o
testemunho de Deus é este, que de seu Filho testificou.

Quem crê no Filho de Deus, em si mesmo tem o testemunho; quem a
Deus não crê, mentiroso o fez, porquanto não creu no testemunho que
Deus de seu Filho deu. E o testemunho é este: que
\textbf{Deus nos deu a vida eterna; e esta vida está em seu Filho}.
Quem tem o Filho tem a vida; quem não tem o Filho de Deus não
tem a vida. Estas coisas vos escrevi a vós, os que credes no
nome do Filho de Deus, para que saibais que tendes a vida eterna, e
para que creiais no nome do Filho de Deus.

E esta é a confiança que temos nele, que, se pedirmos alguma
coisa, segundo a sua vontade, ele nos ouve. E, se sabemos que
nos ouve em tudo o que pedimos, sabemos que alcançamos as petições
que lhe fizemos. Se alguém vir pecar seu irmão, pecado que
não é para morte, orará, e Deus dará a vida àqueles que não pecarem
para morte. Há pecado para morte, e por esse não digo que ore.
\textbf{Toda a iniqüidade é pecado}, e há pecado que não é
para morte.

Sabemos que todo aquele que é nascido de Deus não peca; mas o que
de Deus é gerado conserva-se a si mesmo, e o maligno não lhe toca.
Sabemos que somos de Deus, e que \textbf{todo o mundo está no
maligno}. E sabemos que já o Filho de Deus é vindo, e nos deu
entendimento para conhecermos o que é verdadeiro; e no que é
verdadeiro estamos, isto é, em seu Filho\marginpar{\tiny{Jesus é
Deus}} \textbf{Jesus Cristo. Este é o verdadeiro Deus e a vida
eterna}. Filhinhos, guardai-vos dos ídolos. Amém.

