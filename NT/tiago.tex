\thispagestyle{empty}
\chapter*{Epístola de Tiago}

\lettrine{1} Tiago, servo de Deus, e do Senhor Jesus Cristo,
às doze tribos que andam dispersas, saúde.

Meus irmãos, tende grande gozo quando cairdes em várias tentações;
sabendo que a prova da vossa fé opera a paciência. Tenha,
porém, a paciência a sua obra perfeita, para que sejais perfeitos e
completos, sem faltar em coisa alguma. E, se algum de vós tem
falta de sabedoria, peça-a a Deus, que a todos dá liberalmente, e o
não lança em rosto, e ser-lhe-á dada. Peça-a, porém, com fé, em
nada duvidando; porque o que duvida é semelhante à onda do mar, que
é levada pelo vento, e lançada de uma para outra parte. Não
pense tal homem que receberá do Senhor alguma coisa. O homem de
coração dobre é inconstante em todos os seus caminhos. Mas
glorie-se o irmão abatido na sua exaltação, e o rico em seu
abatimento; porque ele passará como a flor da erva. Porque
sai o sol com ardor, e a erva seca, e a sua flor cai, e a formosa
aparência do seu aspecto perece; assim se murchará também o rico em
seus caminhos. Bem-aventurado o homem que suporta a tentação;
porque, quando for provado, receberá a coroa da vida, a qual o
Senhor tem prometido aos que o amam.

Ninguém, sendo tentado, diga: De Deus sou tentado; porque Deus
não pode ser tentado pelo mal, e a ninguém tenta. Mas cada um
é tentado, quando atraído e engodado pela sua própria
concupiscência. Depois, havendo a concupiscência concebido,
dá à luz o pecado; e o pecado, sendo consumado, gera a morte.
Não erreis, meus amados irmãos. Toda a boa dádiva e
todo o dom perfeito vem do alto, descendo do Pai das luzes, em quem
não há mudança nem sombra de variação. Segundo a sua vontade,
ele nos gerou pela palavra da verdade, para que fôssemos como
primícias das suas criaturas.

Portanto, meus amados irmãos, todo o homem seja pronto para
ouvir, tardio para falar, tardio para se irar. Porque a ira
do homem não opera a justiça de Deus. Por isso, rejeitando
toda a imundícia e superfluidade\footnote{Qualidade de supérfluo.
Coisa supérflua.} de malícia, recebei com mansidão a palavra em vós
enxertada, a qual pode salvar as vossas almas. E sede
cumpridores da palavra, e não somente ouvintes, enganando-vos com
falsos discursos. Porque, se alguém é ouvinte da palavra, e
não cumpridor, é semelhante ao homem que contempla ao espelho o seu
rosto natural; porque se contempla a si mesmo, e vai-se, e
logo se esquece de como era. Aquele, porém, que atenta bem
para a lei perfeita da liberdade, e nisso persevera, não sendo
ouvinte esquecidiço, mas fazedor da obra, este tal será
bem-aventurado no seu feito. Se alguém entre vós cuida ser
religioso, e não refreia a sua língua, antes engana o seu coração, a
religião desse é vã. A religião pura e imaculada para com
Deus, o Pai, é esta: Visitar os órfãos e as viúvas nas suas
tribulações, e guardar-se da corrupção do mundo.

\medskip

\lettrine{2} Meus irmãos, não tenhais a fé de nosso Senhor
Jesus Cristo, Senhor da glória, em acepção de pessoas. Porque,
se no vosso ajuntamento entrar algum homem com anel de ouro no dedo,
com trajes preciosos, e entrar também algum pobre com sórdido traje,
e atentardes para o que traz o traje precioso, e lhe disserdes:
Assenta-te tu aqui num lugar de honra, e disserdes ao pobre: Tu,
fica aí em pé, ou assenta-te abaixo do meu estrado, porventura
não fizestes distinção entre vós mesmos, e não vos fizestes juízes
de maus pensamentos? Ouvi, meus amados irmãos: Porventura não
escolheu Deus aos pobres deste mundo para serem ricos na fé, e
herdeiros do reino que prometeu aos que o amam? Mas vós
desonrastes o pobre. Porventura não vos oprimem os ricos, e não vos
arrastam aos tribunais? Porventura não blasfemam eles o bom nome
que sobre vós foi invocado?

Todavia, se cumprirdes, conforme a Escritura, a lei real: Amarás a
teu próximo como a ti mesmo, bem fazeis. Mas, se fazeis acepção
de pessoas, cometeis pecado, e sois redargüidos\footnote{Redargüir:
Replicar argumentando; responder argüindo; replicar. Acusar;
recriminar.} pela lei como transgressores. Porque qualquer
que guardar toda a lei, e tropeçar em um só ponto, tornou-se culpado
de todos. Porque aquele que disse: Não cometerás adultério,
também disse: Não matarás. Se tu pois não cometeres adultério, mas
matares, estás feito transgressor da lei. Assim falai, e
assim procedei, como devendo ser julgados pela lei da liberdade.
Porque o juízo será sem misericórdia sobre aquele que não fez
misericórdia; e a misericórdia triunfa do juízo.

Meus irmãos, que aproveita se alguém disser que tem fé, e não
tiver as obras? Porventura a fé pode salvá-lo? E, se o irmão
ou a irmã estiverem nus, e tiverem falta de mantimento quotidiano,
e algum de vós lhes disser: Ide em paz, aquentai-vos, e
fartai-vos; e não lhes derdes as coisas necessárias para o corpo,
que proveito virá daí? Assim também a fé, se não tiver as
obras, é morta em si mesma. Mas dirá alguém: Tu tens a fé, e
eu tenho as obras; mostra-me a tua fé sem as tuas obras, e eu te
mostrarei a minha fé pelas minhas obras. Tu crês que há um só
Deus; fazes bem. Também os demônios o crêem, e estremecem.
Mas, ó homem vão, queres tu saber que a fé sem as obras é
morta? Porventura o nosso pai Abraão não foi justificado
pelas obras, quando ofereceu sobre o altar o seu filho Isaque?
Bem vês que a fé cooperou com as suas obras, e que pelas
obras a fé foi aperfeiçoada. E cumpriu-se a Escritura, que
diz: E creu Abraão em Deus, e foi-lhe isso imputado como justiça, e
foi chamado o amigo de Deus. Vedes então que o homem é
justificado pelas obras, e não somente pela fé. E de igual
modo Raabe, a meretriz, não foi também justificada pelas obras,
quando recolheu os emissários, e os despediu por outro caminho?
Porque, assim como o corpo sem o espírito está morto, assim
também a fé sem obras é morta.

\medskip

\lettrine{3} Meus irmãos, muitos de vós não sejam mestres,
sabendo que receberemos mais duro juízo. Porque todos tropeçamos
em muitas coisas. Se alguém não tropeça em palavra, o tal é
perfeito, e poderoso para também refrear todo o corpo. Ora, nós
pomos freio nas bocas dos cavalos, para que nos obedeçam; e
conseguimos dirigir todo o seu corpo. Vede também as naus que,
sendo tão grandes, e levadas de impetuosos ventos, se viram com um
bem pequeno leme para onde quer a vontade daquele que as governa.
Assim também a língua é um pequeno membro, e gloria-se de
grandes coisas. Vede quão grande bosque um pequeno fogo incendeia.
A língua também é um fogo; como mundo de iniqüidade, a língua
está posta entre os nossos membros, e contamina todo o corpo, e
inflama o curso da natureza, e é inflamada pelo inferno. Porque
toda a natureza, tanto de bestas feras como de aves, tanto de
répteis como de animais do mar, se amansa e foi domada pela natureza
humana; mas nenhum homem pode domar a língua. É um mal que não
se pode refrear; está cheia de peçonha mortal. Com ela
bendizemos a Deus e Pai, e com ela amaldiçoamos os homens, feitos à
semelhança de Deus. De uma mesma boca procede bênção e
maldição. Meus irmãos, não convém que isto se faça assim.
Porventura deita alguma fonte de um mesmo manancial água doce
e água amargosa? Meus irmãos, pode também a figueira produzir
azeitonas, ou a videira figos? Assim tampouco pode uma fonte dar
água salgada e doce.

Quem dentre vós é sábio e entendido? Mostre pelo seu bom trato as
suas obras em mansidão de sabedoria. Mas, se tendes amarga
inveja, e sentimento faccioso em vosso coração, não vos glorieis,
nem mintais contra a verdade. Essa não é a sabedoria que vem
do alto, mas é terrena, animal e diabólica. Porque onde há
inveja e espírito faccioso aí há perturbação e toda a obra perversa.
Mas a sabedoria que do alto vem é, primeiramente pura, depois
pacífica, moderada, tratável, cheia de misericórdia e de bons
frutos, sem parcialidade, e sem hipocrisia. Ora, o fruto da
justiça semeia-se na paz, para os que exercitam a paz.

\medskip

\lettrine{4} De onde vêm as guerras e pelejas entre vós?
Porventura não vêm disto, a saber, dos vossos deleites, que nos
vossos membros guerreiam? Cobiçais, e nada tendes; matais, e
sois invejosos, e nada podeis alcançar; combateis e guerreais, e
nada tendes, porque não pedis. Pedis, e não recebeis, porque
pedis mal, para o gastardes em vossos deleites. Adúlteros e
adúlteras, não sabeis vós que a amizade do mundo é inimizade contra
Deus? Portanto, qualquer que quiser ser amigo do mundo constitui-se
inimigo de Deus. Ou cuidais vós que em vão diz a Escritura: O
Espírito que em nós habita tem ciúmes? Antes, ele dá maior
graça. Portanto diz: Deus resiste aos soberbos, mas dá graça aos
humildes. Sujeitai-vos, pois, a Deus, resisti ao diabo, e ele
fugirá de vós. Chegai-vos a Deus, e ele se chegará a vós.
Alimpai as mãos, pecadores; e, vós de duplo ânimo, purificai os
corações. Senti as vossas misérias, e lamentai e chorai;
converta-se o vosso riso em pranto, e o vosso gozo em tristeza.
Humilhai-vos perante o Senhor, e ele vos exaltará.

Irmãos, não faleis mal uns dos outros. Quem fala mal de um irmão,
e julga a seu irmão, fala mal da lei, e julga a lei; e, se tu julgas
a lei, já não és observador da lei, mas juiz. Há só um
legislador que pode salvar e destruir. Tu, porém, quem és, que
julgas a outrem? Eia agora vós, que dizeis: Hoje, ou amanhã,
iremos a tal cidade, e lá passaremos um ano, e contrataremos, e
ganharemos; digo-vos que não sabeis o que acontecerá amanhã.
Porque, que é a vossa vida? É um vapor que aparece por um pouco, e
depois se desvanece. Em lugar do que devíeis dizer: Se o
Senhor quiser, e se vivermos, faremos isto ou aquilo. Mas
agora vos gloriais em vossas presunções; toda a glória tal como esta
é maligna. Aquele, pois, que sabe fazer o bem e não o faz,
comete pecado.

\medskip

\lettrine{5} Eia, pois, agora vós, ricos, chorai e pranteai,
por vossas misérias, que sobre vós hão de vir. As vossas
riquezas estão apodrecidas, e as vossas vestes estão comidas de
traça. O vosso ouro e a vossa prata se enferrujaram; e a sua
ferrugem dará testemunho contra vós, e comerá como fogo a vossa
carne. Entesourastes para os últimos dias. Eis que o
salário\footnote{SBTB: jornal. KJ: the \textbf{hire} of the
labourers.} dos trabalhadores que ceifaram as vossas terras, e que
por vós foi diminuído, clama; e os clamores dos que ceifaram
entraram nos ouvidos do Senhor dos exércitos. Deliciosamente
vivestes sobre a terra, e vos deleitastes; cevastes os vossos
corações, como num dia de matança. Condenastes e matastes o
justo; ele não vos resistiu. Sede pois, irmãos, pacientes até à
vinda do Senhor. Eis que o lavrador espera o precioso fruto da
terra, aguardando-o com paciência, até que receba a chuva
temporã\footnote{Que vem ou acontece fora ou antes do tempo próprio;
extemporâneo.} e serôdia\footnote{Que vem tarde, fora do tempo;
tardio.}. Sede vós também pacientes, fortalecei os vossos
corações; porque já a vinda do Senhor está próxima. Irmãos, não
vos queixeis uns contra os outros, para que não sejais condenados.
Eis que o juiz está à porta. Meus irmãos, tomai por exemplo
de aflição e paciência os profetas que falaram em nome do Senhor.
Eis que temos por bem-aventurados os que sofreram. Ouvistes
qual foi a paciência de Jó, e vistes o fim que o Senhor lhe deu;
porque o Senhor é muito misericordioso e piedoso.

Mas, sobretudo, meus irmãos, não jureis, nem pelo céu, nem pela
terra, nem façais qualquer outro juramento; mas que a vossa palavra
seja sim, sim, e não, não; para que não caiais em condenação.
Está alguém entre vós aflito? Ore. Está alguém contente?
Cante louvores. Está alguém entre vós doente? Chame os
presbíteros da igreja, e orem sobre ele, ungindo-o com azeite em
nome do Senhor; e a oração da fé salvará o doente, e o Senhor
o levantará; e, se houver cometido pecados, ser-lhe-ão perdoados.
Confessai as vossas culpas uns aos outros, e orai uns pelos
outros, para que sareis. A oração feita por um justo pode muito em
seus efeitos. Elias era homem sujeito às mesmas paixões que
nós e, orando, pediu que não chovesse e, por três anos e seis meses,
não choveu sobre a terra. E orou outra vez, e o céu deu
chuva, e a terra produziu o seu fruto. Irmãos, se algum
dentre vós se tem desviado da verdade, e alguém o converter,
saiba que aquele que fizer converter do erro do seu caminho
um pecador, salvará da morte uma alma, e cobrirá uma multidão de
pecados.

