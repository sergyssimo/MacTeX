\thispagestyle{empty}
\chapter*{Epístola aos Efésios}

\lettrine{1} Paulo, apóstolo de Jesus Cristo, pela vontade de
Deus, aos santos que estão em Éfeso, e fiéis em Cristo Jesus: A
vós graça, e paz da parte de Deus nosso Pai e do Senhor Jesus
Cristo!

Bendito o Deus e Pai de nosso Senhor Jesus Cristo, o qual nos
abençoou com todas as bênçãos espirituais nos lugares celestiais em
Cristo; como também nos elegeu nele antes da fundação do mundo,
para que fôssemos santos e irrepreensíveis diante dele em amor;
e nos predestinou para filhos de adoção por Jesus Cristo, para
si mesmo, segundo o beneplácito\footnote{Consentimento, licença,
aprovação.} de sua vontade, para louvor e glória da sua graça,
pela qual nos fez agradáveis a si no Amado, em quem temos a
redenção pelo seu sangue, a remissão das ofensas, segundo as
riquezas da sua graça, que ele fez abundar para conosco em toda
a sabedoria e prudência; descobrindo-nos o mistério da sua
vontade, segundo o seu beneplácito, que propusera em si mesmo,
de tornar a congregar em Cristo todas as coisas, na
dispensação da plenitude dos tempos, tanto as que estão nos céus
como as que estão na terra; nele, digo, em quem também fomos
feitos herança, havendo sido predestinados, conforme o propósito
daquele que faz todas as coisas, segundo o conselho da sua vontade;
com o fim de sermos para louvor da sua glória, nós os que
primeiro esperamos em Cristo; em quem também vós estais,
depois que ouvistes a palavra da verdade, o evangelho da vossa
salvação; e, tendo nele também crido, fostes selados com o Espírito
Santo da promessa. O qual é o penhor da nossa herança, para
redenção da possessão adquirida, para louvor da sua glória.

Por isso, ouvindo eu também a fé que entre vós há no Senhor
Jesus, e o vosso amor para com todos os santos, não cesso de
dar graças a Deus por vós, lembrando-me de vós nas minhas orações:
para que o Deus de nosso Senhor Jesus Cristo, o Pai da
glória, vos dê em seu conhecimento o espírito de sabedoria e de
revelação; tendo iluminados os olhos do vosso entendimento,
para que saibais qual seja a esperança da sua vocação, e quais as
riquezas da glória da sua herança nos santos; e qual a
sobreexcelente grandeza do seu poder sobre nós, os que cremos,
segundo a operação da força do seu poder, que manifestou em
Cristo, ressuscitando-o dentre os mortos, e pondo-o à sua direita
nos céus. Acima de todo o principado, e poder, e potestade, e
domínio, e de todo o nome que se nomeia, não só neste século, mas
também no vindouro; e sujeitou todas as coisas a seus pés, e
sobre todas as coisas o constituiu como cabeça da igreja, que
é o seu corpo, a plenitude daquele que cumpre tudo em todos.

\medskip

\lettrine{2} E vos vivificou, estando vós mortos em ofensas e
pecados, em que noutro tempo andastes segundo o curso deste
mundo, segundo o príncipe das potestades do ar, do espírito que
agora opera nos filhos da desobediência. Entre os quais todos
nós também antes andávamos nos desejos da nossa carne, fazendo a
vontade da carne e dos pensamentos; e éramos por natureza filhos da
ira, como os outros também.

Mas Deus, que é riquíssimo em misericórdia, pelo seu muito amor
com que nos amou, estando nós ainda mortos em nossas ofensas,
nos vivificou juntamente com Cristo (pela graça sois salvos), e
nos ressuscitou juntamente com ele e nos fez assentar nos lugares
celestiais, em Cristo Jesus; para mostrar nos séculos vindouros
as abundantes riquezas da sua graça pela sua benignidade para
conosco em Cristo Jesus. Porque pela graça sois salvos, por meio
da fé; e isto não vem de vós, é dom de Deus. Não vem das obras,
para que ninguém se glorie; porque somos feitura sua, criados
em Cristo Jesus para as boas obras, as quais Deus preparou para que
andássemos nelas.

Portanto, lembrai-vos de que vós noutro tempo éreis gentios na
carne, e chamados incircuncisão pelos que na carne se chamam
circuncisão feita pela mão dos homens; que naquele tempo
estáveis sem Cristo, separados da comunidade de Israel, e estranhos
às alianças da promessa, não tendo esperança, e sem Deus no mundo.
Mas agora em Cristo Jesus, vós, que antes estáveis longe, já
pelo sangue de Cristo chegastes perto.

Porque ele é a nossa paz, o qual de ambos os povos fez um; e,
derrubando a parede de separação que estava no meio, na sua
carne desfez a inimizade, isto é, a lei dos mandamentos, que
consistia em ordenanças, para criar em si mesmo dos dois um novo
homem, fazendo a paz, e pela cruz reconciliar ambos com Deus
em um corpo, matando com ela as inimizades. E, vindo, ele
evangelizou a paz, a vós que estáveis longe, e aos que estavam
perto; porque por ele ambos temos acesso ao Pai em um mesmo
Espírito. Assim que já não sois estrangeiros, nem
forasteiros, mas concidadãos dos santos, e da família de Deus;
edificados sobre o fundamento dos apóstolos e dos profetas,
de que Jesus Cristo é a principal pedra da esquina; no qual
todo o edifício, bem ajustado, cresce para templo santo no Senhor.
No qual também vós juntamente sois edificados para morada de
Deus em Espírito.

\medskip

\lettrine{3} Por esta causa eu, Paulo, sou o prisioneiro de
Jesus Cristo por vós, os gentios; se é que tendes ouvido a
dispensação da graça de Deus, que para convosco me foi dada;
como me foi este mistério manifestado pela revelação, como antes
um pouco vos escrevi; por isso, quando ledes, podeis perceber a
minha compreensão do mistério de Cristo, o qual noutros séculos
não foi manifestado aos filhos dos homens, como agora tem sido
revelado pelo Espírito aos seus santos apóstolos e profetas; a
saber, que os gentios são co-herdeiros, e de um mesmo corpo, e
participantes da promessa em Cristo pelo evangelho; do qual fui
feito ministro, pelo dom da graça de Deus, que me foi dado segundo a
operação do seu poder. A mim, o mínimo de todos os santos, me
foi dada esta graça de anunciar entre os gentios, por meio do
evangelho, as riquezas incompreensíveis de Cristo, e demonstrar
a todos qual seja a dispensação do mistério, que desde os séculos
esteve oculto em Deus, que tudo criou por meio de Jesus Cristo;
para que agora, pela igreja, a multiforme sabedoria de Deus
seja conhecida dos principados e potestades nos céus, segundo
o eterno propósito que fez em Cristo Jesus nosso Senhor, no
qual temos ousadia e acesso com confiança, pela nossa fé nele.
Portanto, vos peço que não desfaleçais nas minhas tribulações
por vós, que são a vossa glória.

Por causa disto me ponho de joelhos perante o Pai de nosso Senhor
Jesus Cristo, do qual toda a família nos céus e na terra toma
o nome, para que, segundo as riquezas da sua glória, vos
conceda que sejais corroborados com poder pelo seu Espírito no homem
interior; para que Cristo habite pela fé nos vossos corações;
a fim de, estando arraigados e fundados em amor, poderdes
perfeitamente compreender, com todos os santos, qual seja a largura,
e o comprimento, e a altura, e a profundidade, e conhecer o
amor de Cristo, que excede todo o entendimento, para que sejais
cheios de toda a plenitude de Deus. Ora, àquele que é
poderoso para fazer tudo muito mais abundantemente além daquilo que
pedimos ou pensamos, segundo o poder que em nós opera, a esse
glória na igreja, por Jesus Cristo, em todas as gerações, para todo
o sempre. Amém.

\medskip

\lettrine{4} Rogo-vos, pois, eu, o preso do Senhor, que andeis
como é digno da vocação com que fostes chamados, com toda a
humildade e mansidão, com longanimidade, suportando-vos uns aos
outros em amor, procurando guardar a unidade do Espírito pelo
vínculo da paz.

Há um só corpo e um só Espírito, como também fostes chamados em
uma só esperança da vossa vocação; um só Senhor, uma só fé, um
só batismo; um só Deus e Pai de todos, o qual é sobre todos, e
por todos e em todos vós. Mas a graça foi dada a cada um de nós
segundo a medida do dom de Cristo. Por isso diz: Subindo ao
alto, levou cativo o cativeiro, e deu dons aos homens. Ora, isto
--- ele subiu --- que é, senão que também antes tinha descido às
partes mais baixas da terra? Aquele que desceu é também o
mesmo que subiu acima de todos os céus, para cumprir todas as
coisas. E ele mesmo deu uns para apóstolos, e outros para
profetas, e outros para evangelistas, e outros para pastores e
doutores, querendo o aperfeiçoamento dos santos, para a obra
do ministério, para edificação do corpo de Cristo; até que
todos cheguemos à unidade da fé, e ao conhecimento do Filho de Deus,
a homem perfeito, à medida da estatura completa de Cristo,
para que não sejamos mais meninos inconstantes, levados em
roda por todo o vento de doutrina, pelo engano dos homens que com
astúcia enganam fraudulosamente. Antes, seguindo a verdade em
amor, cresçamos em tudo naquele que é a cabeça, Cristo, do
qual todo o corpo, bem ajustado, e ligado pelo auxílio de todas as
juntas, segundo a justa operação de cada parte, faz o aumento do
corpo, para sua edificação em amor.

E digo isto, e testifico no Senhor, para que não andeis mais como
andam também os outros gentios, na vaidade da sua mente.
Entenebrecidos no entendimento, separados da vida de Deus
pela ignorância que há neles, pela dureza do seu coração; os
quais, havendo perdido todo o sentimento, se entregaram à
dissolução, para com avidez cometerem toda a impureza. Mas
vós não aprendestes assim a Cristo, se é que o tendes ouvido,
e nele fostes ensinados, como está a verdade em Jesus; que,
quanto ao trato passado, vos despojeis do velho homem, que se
corrompe pelas concupiscências do engano; e vos renoveis no
espírito da vossa mente; e vos revistais do novo homem, que
segundo Deus é criado em verdadeira justiça e santidade. Por
isso deixai a mentira, e falai a verdade cada um com o seu próximo;
porque somos membros uns dos outros. Irai-vos, e não pequeis;
não se ponha o sol sobre a vossa ira. Não deis lugar ao
diabo. Aquele que furtava, não furte mais; antes trabalhe,
fazendo com as mãos o que é bom, para que tenha o que repartir com o
que tiver necessidade. Não saia da vossa boca nenhuma palavra
torpe, mas só a que for boa para promover a edificação, para que dê
graça aos que a ouvem. E não entristeçais o Espírito Santo de
Deus, no qual estais selados para o dia da redenção. Toda a
amargura, e ira, e cólera, e gritaria, e blasfêmia e toda a malícia
sejam tiradas dentre vós, antes sede uns para com os outros
benignos, misericordiosos, perdoando-vos uns aos outros, como também
Deus vos perdoou em Cristo.

\medskip

\lettrine{5} Sede, pois, imitadores de Deus, como filhos
amados; e andai em amor, como também Cristo vos amou, e se
entregou a si mesmo por nós, em oferta e sacrifício a Deus, em
cheiro suave.

Mas a prostituição, e toda a impureza ou avareza, nem ainda se
nomeie entre vós, como convém a santos; nem torpezas, nem
parvoíces\footnote{Qualidade ou estado de parvo (pequeno, limitado,
apoucado; tolo). Parvidade, parvoeira, parvulez.}, nem
chocarrices\footnote{Chalaça (dito zombeteiro, gracejo de mau gosto,
ou insolente; graçola,  caçoada, troça, zombaria). Gracejo atrevido;
truanice.}, que não convêm; mas antes, ações de graças. Porque
bem sabeis isto: que nenhum devasso, ou impuro, ou avarento, o qual
é idólatra, tem herança no reino de Cristo e de Deus. Ninguém
vos engane com palavras vãs; porque por estas coisas vem a ira de
Deus sobre os filhos da desobediência. Portanto, não sejais seus
companheiros. Porque noutro tempo éreis trevas, mas agora sois
luz no Senhor; andai como filhos da luz (porque o fruto do
Espírito está em toda a bondade, e justiça e verdade);
aprovando o que é agradável ao Senhor. E não
comuniqueis com as obras infrutuosas das trevas, mas antes
condenai-as. Porque o que eles fazem em oculto até dizê-lo é
torpe. Mas todas estas coisas se manifestam, sendo condenadas
pela luz, porque a luz tudo manifesta. Por isso diz:
Desperta, tu que dormes, e levanta-te dentre os mortos, e Cristo te
esclarecerá. Portanto, vede prudentemente como andais, não
como néscios, mas como sábios, remindo o tempo; porquanto os
dias são maus. Por isso não sejais insensatos, mas entendei
qual seja a vontade do Senhor. E não vos embriagueis com
vinho, em que há contenda, mas enchei-vos do Espírito;
falando entre vós em salmos, e hinos, e cânticos espirituais;
cantando e salmodiando ao Senhor no vosso coração; dando
sempre graças por tudo a nosso Deus e Pai, em nome de nosso Senhor
Jesus Cristo; sujeitando-vos uns aos outros no temor de Deus.

Vós, mulheres, sujeitai-vos a vossos maridos, como ao Senhor;
porque o marido é a cabeça da mulher, como também Cristo é a
cabeça da igreja, sendo ele próprio o salvador do corpo. De
sorte que, assim como a igreja está sujeita a Cristo, assim também
as mulheres sejam em tudo sujeitas a seus maridos.

Vós, maridos, amai vossas mulheres, como também Cristo amou a
igreja, e a si mesmo se entregou por ela, para a santificar,
purificando-a com a lavagem da água, pela palavra, para a
apresentar a si mesmo igreja gloriosa, sem mácula, nem ruga, nem
coisa semelhante, mas santa e irrepreensível. Assim devem os
maridos amar as suas próprias mulheres, como a seus próprios corpos.
Quem ama a sua mulher, ama-se a si mesmo. Porque nunca
ninguém odiou a sua própria carne; antes a alimenta e sustenta, como
também o Senhor à igreja; porque somos membros do seu corpo,
da sua carne, e dos seus ossos. Por isso deixará o homem seu
pai e sua mãe, e se unirá a sua mulher; e serão dois numa carne.
Grande é este mistério; digo-o, porém, a respeito de Cristo e
da igreja. Assim também vós, cada um em particular, ame a sua
própria mulher como a si mesmo, e a mulher reverencie o marido.

\medskip

\lettrine{6} Vós, filhos, sede obedientes a vossos pais no
Senhor, porque isto é justo. Honra a teu pai e a tua mãe, que é
o primeiro mandamento com promessa; para que te vá bem, e vivas
muito tempo sobre a terra. E vós, pais, não provoqueis à ira a
vossos filhos, mas criai-os na doutrina e admoestação do Senhor.
Vós, servos, obedecei a vossos senhores segundo a carne, com
temor e tremor, na sinceridade de vosso coração, como a Cristo;
não servindo à vista, como para agradar aos homens, mas como
servos de Cristo, fazendo de coração a vontade de Deus; servindo
de boa vontade como ao Senhor, e não como aos homens. Sabendo
que cada um receberá do Senhor todo o bem que fizer, seja servo,
seja livre. E vós, senhores, fazei o mesmo para com eles,
deixando as ameaças, sabendo também que o Senhor deles e vosso está
no céu, e que para com ele não há acepção de pessoas.

No demais, irmãos meus, fortalecei-vos no Senhor e na força do
seu poder. Revesti-vos de toda a armadura de Deus, para que
possais estar firmes contra as astutas ciladas do diabo.
Porque não temos que lutar contra a carne e o sangue, mas,
sim, contra os principados, contra as potestades, contra os
príncipes das trevas deste século, contra as hostes espirituais da
maldade, nos lugares celestiais. Portanto, tomai toda a
armadura de Deus, para que possais resistir no dia mau e, havendo
feito tudo, ficar firmes. Estai, pois, firmes, tendo cingidos
os vossos lombos com a verdade, e vestida a couraça da justiça;
e calçados os pés na preparação do evangelho da paz;
tomando sobretudo o escudo da fé, com o qual podereis apagar
todos os dardos inflamados do maligno. Tomai também o
capacete da salvação, e a espada do Espírito, que é a palavra de
Deus; orando em todo o tempo com toda a oração e súplica no
Espírito, e vigiando nisto com toda a perseverança e súplica por
todos os santos, e por mim; para que me seja dada, no abrir
da minha boca, a palavra com confiança, para fazer notório o
mistério do evangelho, pelo qual sou embaixador em cadeias;
para que possa falar dele livremente, como me convém falar.

Ora, para que vós também possais saber dos meus negócios, e o que
eu faço, Tíquico, irmão amado, e fiel ministro do Senhor, vos
informará de tudo. O qual vos enviei para o mesmo fim, para
que saibais do nosso estado, e ele console os vossos corações.
Paz seja com os irmãos, e amor com fé da parte de Deus Pai e
da do Senhor Jesus Cristo. A graça seja com todos os que amam
a nosso Senhor Jesus Cristo em sinceridade. Amém.

