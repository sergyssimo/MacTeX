\addchap{Segunda Epístola de Paulo a Timóteo}

\lettrine{1} Paulo, apóstolo de Jesus Cristo, pela vontade de
Deus, segundo a promessa da vida que está em Cristo Jesus, a
Timóteo, meu amado filho: Graça, misericórdia, e paz da parte de
Deus Pai, e da de Cristo Jesus, Senhor nosso. Dou graças a Deus,
a quem desde os meus antepassados sirvo com uma consciência pura, de
que sem cessar faço memória de ti nas minhas orações noite e dia;
desejando muito ver-te, lembrando-me das tuas lágrimas, para me
encher de gozo; trazendo à memória a fé não fingida que em ti
há, a qual habitou primeiro em tua avó Lóide, e em tua mãe Eunice, e
estou certo de que também habita em ti.

Por cujo motivo te lembro que despertes o dom de Deus que existe
em ti pela imposição das minhas mãos. Porque Deus não nos deu o
espírito de temor, mas de fortaleza, e de amor, e de moderação.
Portanto, não te envergonhes do testemunho de nosso Senhor, nem
de mim, que sou prisioneiro seu; antes participa das aflições do
evangelho segundo o poder de Deus, que nos salvou, e chamou com
uma santa vocação; não segundo as nossas obras, mas segundo o seu
próprio propósito e graça que nos foi dada em Cristo Jesus antes dos
tempos dos séculos; e que é manifesta agora pela aparição de
nosso Salvador Jesus Cristo, o qual aboliu a morte, e trouxe à luz a
vida e a incorrupção pelo evangelho; para o que fui
constituído pregador, e apóstolo, e doutor dos gentios. Por
cuja causa padeço também isto, mas não me envergonho; porque eu sei
em quem tenho crido, e estou certo de que é poderoso para guardar o
meu depósito até àquele dia. Conserva o modelo das sãs
palavras que de mim tens ouvido, na fé e no amor que há em Cristo
Jesus. Guarda o bom depósito pelo Espírito Santo que habita
em nós.

Bem sabes isto, que os que estão na Ásia todos se apartaram de
mim; entre os quais foram Figelo e Hermógenes. O Senhor
conceda misericórdia à casa de Onesíforo, porque muitas vezes me
recreou, e não se envergonhou das minhas cadeias. Antes,
vindo ele a Roma, com muito cuidado me procurou e me achou. O
Senhor lhe conceda que naquele dia ache misericórdia diante do
Senhor. E, quanto me ajudou em Éfeso, melhor o sabes tu.

\medskip

\lettrine{2} Tu, pois, meu filho, fortifica-te na graça que há
em Cristo Jesus. E o que de mim, entre muitas testemunhas,
ouviste, confia-o a homens fiéis, que sejam idôneos para também
ensinarem os outros. Sofre, pois, comigo, as aflições, como bom
soldado de Jesus Cristo. Ninguém que milita se embaraça com
negócios desta vida, a fim de agradar àquele que o alistou para a
guerra. E, se alguém também milita, não é coroado se não militar
legitimamente. O lavrador que trabalha deve ser o primeiro a
gozar dos frutos. Considera o que digo, porque o Senhor te dará
entendimento em tudo.

Lembra-te de que Jesus Cristo, que é da descendência de Davi,
ressuscitou dentre os mortos, segundo o meu evangelho; por isso
sofro trabalhos e até prisões, como um malfeitor; mas a palavra de
Deus não está presa. Portanto, tudo sofro por amor dos
escolhidos, para que também eles alcancem a salvação que está em
Cristo Jesus com glória eterna. Palavra fiel é esta: que, se
morrermos com ele, também com ele viveremos; se sofrermos,
também com ele reinaremos; se o negarmos, também ele nos negará;
se formos infiéis, ele permanece fiel; não pode negar-se a si
mesmo.

Traze estas coisas à memória, ordenando-lhes diante do Senhor que
não tenham contendas de palavras, que para nada aproveitam e são
para perversão dos ouvintes. Procura apresentar-te a Deus
aprovado, como obreiro que não tem de que se envergonhar, que maneja
bem a palavra da verdade. Mas evita os falatórios profanos,
porque produzirão maior impiedade. E a palavra desses roerá
como gangrena; entre os quais são Himeneu e Fileto; os quais
se desviaram da verdade, dizendo que a ressurreição era já feita, e
perverteram a fé de alguns.

Todavia o fundamento de Deus fica firme, tendo este selo: O
Senhor conhece os que são seus, e qualquer que profere o nome de
Cristo aparte-se da iniqüidade. Ora, numa grande casa não
somente há vasos de ouro e de prata, mas também de pau e de barro;
uns para honra, outros, porém, para desonra. De sorte que, se
alguém se purificar destas coisas, será vaso para honra, santificado
e idôneo para uso do Senhor, e preparado para toda a boa obra.

Foge também das paixões da mocidade; e segue a justiça, a fé, o
amor, e a paz com os que, com um coração puro, invocam o Senhor.
E rejeita as questões loucas, e sem instrução, sabendo que
produzem contendas. E ao servo do Senhor não convém
contender, mas sim, ser manso para com todos, apto para ensinar,
sofredor; instruindo com mansidão os que resistem, a ver se
porventura Deus lhes dará arrependimento para conhecerem a verdade,
e tornarem a despertar, desprendendo-se dos laços do diabo,
em que à vontade dele estão presos.

\medskip

\lettrine{3} Sabe, porém, isto: que nos últimos dias
sobrevirão tempos trabalhosos. Porque haverá homens amantes de
si mesmos, avarentos, presunçosos, soberbos, blasfemos,
desobedientes a pais e mães, ingratos, profanos, sem afeto
natural, irreconciliáveis, caluniadores, incontinentes, cruéis, sem
amor para com os bons, traidores, obstinados, orgulhosos, mais
amigos dos deleites do que amigos de Deus, tendo aparência de
piedade, mas negando a eficácia dela. Destes afasta-te. Porque
deste número são os que se introduzem pelas casas, e levam cativas
mulheres néscias carregadas de pecados, levadas de várias
concupiscências; que aprendem sempre, e nunca podem chegar ao
conhecimento da verdade. E, como Janes e Jambres resistiram a
Moisés, assim também estes resistem à verdade, sendo homens
corruptos de entendimento e réprobos quanto à fé. Não irão,
porém, avante; porque a todos será manifesto o seu desvario, como
também o foi o daqueles.

Tu, porém, tens seguido a minha doutrina, modo de viver,
intenção, fé, longanimidade, amor, paciência, perseguições e
aflições tais quais me aconteceram em Antioquia, em Icônio, e em
Listra; quantas perseguições sofri, e o Senhor de todas me livrou;
e também todos os que piamente querem viver em Cristo Jesus
padecerão perseguições. Mas os homens maus e enganadores irão
de mal para pior, enganando e sendo enganados. Tu, porém,
permanece naquilo que aprendeste, e de que foste inteirado, sabendo
de quem o tens aprendido, e que desde a tua meninice sabes as
sagradas Escrituras, que podem fazer-te sábio para a salvação, pela
fé que há em Cristo Jesus. Toda a Escritura é divinamente
inspirada, e proveitosa para ensinar, para
redargüir\footnote{Replicar argumentando; responder argüindo;
replicar.}, para corrigir, para instruir em justiça; para que
o homem de Deus seja perfeito, e perfeitamente instruído para toda a
boa obra.

\medskip

\lettrine{4} Conjuro-te, pois, diante de Deus, e do Senhor
Jesus Cristo, que há de julgar os vivos e os mortos, na sua vinda e
no seu reino, que pregues a palavra, instes\footnote{Instar:
pedir, solicitar, com instância; insistir.} a tempo e fora de tempo,
redarguas, repreendas, exortes, com toda a longanimidade e doutrina.
Porque virá tempo em que não suportarão a sã doutrina; mas,
tendo comichão nos ouvidos, amontoarão para si doutores conforme as
suas próprias concupiscências; e desviarão os ouvidos da
verdade, voltando às fábulas. Mas tu, sê sóbrio em tudo, sofre
as aflições, faze a obra de um evangelista, cumpre o teu ministério.
Porque eu já estou sendo oferecido por aspersão de sacrifício, e
o tempo da minha partida está próximo. Combati o bom combate,
acabei a carreira, guardei a fé. Desde agora, a coroa da justiça
me está guardada, a qual o Senhor, justo juiz, me dará naquele dia;
e não somente a mim, mas também a todos os que amarem a sua vinda.

Procura vir ter comigo depressa, porque Demas me
desamparou, amando o presente século, e foi para Tessalônica,
Crescente para Galácia, Tito para Dalmácia. Só Lucas está
comigo. Toma Marcos, e traze-o contigo, porque me é muito útil para
o ministério. Também enviei Tíquico a Éfeso. Quando
vieres, traze a capa que deixei em Trôade, em casa de Carpo, e os
livros, principalmente os pergaminhos. Alexandre, o
latoeiro\footnote{Fabricante ou vendedor de lata e/ou de latão.
Funileiro: Fabricante de funis. Aquele que trabalha em
folha-de-flandres.}, causou-me muitos males; o Senhor lhe pague
segundo as suas obras. Tu, guarda-te também dele, porque
resistiu muito às nossas palavras.

Ninguém me assistiu na minha primeira defesa, antes todos me
desampararam. Que isto lhes não seja imputado. Mas o Senhor
assistiu-me e fortaleceu-me, para que por mim fosse cumprida a
pregação, e todos os gentios a ouvissem; e fiquei livre da boca do
leão. E o Senhor me livrará de toda a má obra, e guardar-me-á
para o seu reino celestial; a quem seja glória para todo o sempre.
Amém. Saúda a Prisca e a Áqüila, e à casa de Onesíforo.
Erasto ficou em Corinto, e deixei Trófimo doente em Mileto.
Procura vir antes do inverno. Êubulo, e Prudente, e Lino, e
Cláudia, e todos os irmãos te saúdam. O Senhor Jesus Cristo
seja com o teu espírito. A graça seja convosco. Amém.

