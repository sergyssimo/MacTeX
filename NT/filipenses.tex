\thispagestyle{empty}
\chapter*{Epístola de Paulo aos Filipenses}

\lettrine{1} Paulo e Timóteo, servos de Jesus Cristo, a todos
os santos em Cristo Jesus, que estão em Filipos, com os bispos e
diáconos: Graça a vós, e paz da parte de Deus nosso Pai e da do
Senhor Jesus Cristo.

Dou graças ao meu Deus todas as vezes que me lembro de vós,
fazendo sempre com alegria oração por vós em todas as minhas
súplicas, pela vossa cooperação no evangelho desde o primeiro
dia até agora. Tendo por certo isto mesmo, que aquele que em vós
começou a boa obra a aperfeiçoará até ao dia de Jesus Cristo;
como tenho por justo sentir isto de vós todos, porque vos
retenho em meu coração, pois todos vós fostes participantes da minha
graça, tanto nas minhas prisões como na minha defesa e confirmação
do evangelho. Porque Deus me é testemunha das saudades que de
todos vós tenho, em entranhável afeição de Jesus Cristo.

E peço isto: que o vosso amor cresça mais e mais em ciência e em
todo o conhecimento, para que aproveis as coisas excelentes,
para que sejais sinceros, e sem escândalo algum até ao dia de
Cristo; cheios dos frutos de justiça, que são por Jesus
Cristo, para glória e louvor de Deus.

E quero, irmãos, que saibais que as coisas que me aconteceram
contribuíram para maior proveito do evangelho; de maneira que
as minhas prisões em Cristo foram manifestas por toda a guarda
pretoriana, e por todos os demais lugares; e muitos dos
irmãos no Senhor, tomando ânimo com as minhas prisões, ousam falar a
palavra mais confiadamente, sem temor. Verdade é que também
alguns pregam a Cristo por inveja e porfia, mas outros de boa
vontade; uns, na verdade, anunciam a Cristo por contenção,
não puramente, julgando acrescentar aflição às minhas prisões.
Mas outros, por amor, sabendo que fui posto para defesa do
evangelho. Mas que importa? Contanto que Cristo seja
anunciado de toda a maneira, ou com fingimento ou em verdade, nisto
me regozijo, e me regozijarei ainda. Porque sei que disto me
resultará salvação, pela vossa oração e pelo socorro do Espírito de
Jesus Cristo, segundo a minha intensa expectação e esperança,
de que em nada serei confundido; antes, com toda a confiança, Cristo
será, tanto agora como sempre, engrandecido no meu corpo, seja pela
vida, seja pela morte.

Porque para mim o viver é Cristo, e o morrer é ganho. Mas,
se o viver na carne me der fruto da minha obra, não sei então o que
deva escolher. Mas de ambos os lados estou em aperto, tendo
desejo de partir, e estar com Cristo, porque isto é ainda muito
melhor. Mas julgo mais necessário, por amor de vós, ficar na
carne. E, tendo esta confiança, sei que ficarei, e
permanecerei com todos vós para proveito vosso e gozo da fé,
para que a vossa glória cresça por mim em Cristo Jesus, pela
minha nova ida a vós.

Somente deveis portar-vos dignamente conforme o evangelho de
Cristo, para que, quer vá e vos veja, quer esteja ausente, ouça
acerca de vós que estais num mesmo espírito, combatendo juntamente
com o mesmo ânimo pela fé do evangelho. E em nada vos
espanteis dos que resistem, o que para eles, na verdade, é indício
de perdição, mas para vós de salvação, e isto de Deus. Porque
a vós vos foi concedido, em relação a Cristo, não somente crer nele,
como também padecer por ele, tendo o mesmo combate que já em
mim tendes visto e agora ouvis estar em mim.

\medskip

\lettrine{2} Portanto, se há algum conforto em Cristo, se
alguma consolação de amor, se alguma comunhão no Espírito, se alguns
entranháveis afetos e compaixões, completai o meu gozo, para que
sintais o mesmo, tendo o mesmo amor, o mesmo ânimo, sentindo uma
mesma coisa. Nada façais por contenda ou por vanglória, mas por
humildade; cada um considere os outros superiores a si mesmo.
Não atente cada um para o que é propriamente seu, mas cada qual
também para o que é dos outros. De sorte que haja em vós o mesmo
sentimento que houve também em Cristo Jesus, que, sendo em forma
de Deus, não teve por usurpação ser igual a Deus, mas
esvaziou-se a si mesmo, tomando a forma de servo, fazendo-se
semelhante aos homens; e, achado na forma de homem, humilhou-se
a si mesmo, sendo obediente até à morte, e morte de cruz. Por
isso, também Deus o exaltou soberanamente, e lhe deu um nome que é
sobre todo o nome; para que ao nome de Jesus se dobre todo o
joelho dos que estão nos céus, e na terra, e debaixo da terra,
e toda a língua confesse que Jesus Cristo é o Senhor, para
glória de Deus Pai.

De sorte que, meus amados, assim como sempre obedecestes, não só
na minha presença, mas muito mais agora na minha ausência, assim
também operai a vossa salvação com temor e tremor; porque
Deus é o que opera em vós tanto o querer como o efetuar, segundo a
sua boa vontade.

Fazei todas as coisas sem murmurações nem contendas; para
que sejais irrepreensíveis e sinceros, filhos de Deus inculpáveis,
no meio de uma geração corrompida e perversa, entre a qual
resplandeceis como astros no mundo; retendo a palavra da
vida, para que no dia de Cristo possa gloriar-me de não ter corrido
nem trabalhado em vão. E, ainda que seja oferecido por
libação sobre o sacrifício e serviço da vossa fé, folgo e me
regozijo com todos vós. E vós também regozijai-vos e
alegrai-vos comigo por isto mesmo. E espero no Senhor Jesus
que em breve vos mandarei Timóteo, para que também eu esteja de bom
ânimo, sabendo dos vossos negócios. Porque a ninguém tenho de
igual sentimento, que sinceramente cuide do vosso estado;
porque todos buscam o que é seu, e não o que é de Cristo
Jesus. Mas bem sabeis qual a sua experiência, e que serviu
comigo no evangelho, como filho ao pai. De sorte que espero
vo-lo enviar logo que tenha provido a meus negócios. Mas
confio no Senhor, que também eu mesmo em breve irei ter convosco.
Julguei, contudo, necessário mandar-vos Epafrodito, meu irmão
e cooperador, e companheiro nos combates, e vosso enviado para
prover às minhas necessidades. Porquanto tinha muitas
saudades de vós todos, e estava muito angustiado de que tivésseis
ouvido que ele estivera doente. E de fato esteve doente, e
quase à morte; mas Deus se apiedou dele, e não somente dele, mas
também de mim, para que eu não tivesse tristeza sobre tristeza.
Por isso vo-lo enviei mais depressa, para que, vendo-o outra
vez, vos regozijeis, e eu tenha menos tristeza. Recebei-o,
pois, no Senhor com todo o gozo, e tende-o em honra; porque
pela obra de Cristo chegou até bem próximo da morte, não fazendo
caso da vida para suprir para comigo a falta do vosso serviço.

\medskip

\lettrine{3} Resta, irmãos meus, que vos regozijeis no Senhor.
Não me aborreço de escrever-vos as mesmas coisas, e é segurança para
vós. Guardai-vos dos cães, guardai-vos dos maus obreiros,
guardai-vos da circuncisão; porque a circuncisão somos nós, que
servimos a Deus em espírito, e nos gloriamos em Jesus Cristo, e não
confiamos na carne.

Ainda que também podia confiar na carne; se algum outro cuida que
pode confiar na carne, ainda mais eu: Circuncidado ao oitavo
dia, da linhagem de Israel, da tribo de Benjamim, hebreu de hebreus;
segundo a lei, fui fariseu; segundo o zelo, perseguidor da
igreja; segundo a justiça que há na lei, irrepreensível. Mas o
que para mim era ganho reputei-o perda por Cristo. E, na
verdade, tenho também por perda todas as coisas, pela excelência do
conhecimento de Cristo Jesus, meu Senhor; pelo qual sofri a perda de
todas estas coisas, e as considero como escória, para que possa
ganhar a Cristo, e seja achado nele, não tendo a minha justiça
que vem da lei, mas a que vem pela fé em Cristo, a saber, a justiça
que vem de Deus pela fé; para conhecê-lo, e à virtude da sua
ressurreição, e à comunicação de suas aflições, sendo feito conforme
à sua morte; para ver se de alguma maneira posso chegar à
ressurreição dentre os mortos. Não que já a tenha alcançado,
ou que seja perfeito; mas prossigo para alcançar aquilo para o que
fui também preso por Cristo Jesus. Irmãos, quanto a mim, não
julgo que o haja alcançado; mas uma coisa faço, e é que,
esquecendo-me das coisas que atrás ficam, e avançando para as que
estão diante de mim, prossigo para o alvo, pelo prêmio da
soberana vocação de Deus em Cristo Jesus.

Por isso todos quantos já somos perfeitos, sintamos isto mesmo;
e, se sentis alguma coisa de outra maneira, também Deus vo-lo
revelará. Mas, naquilo a que já chegamos, andemos segundo a
mesma regra, e sintamos o mesmo.

Sede também meus imitadores, irmãos, e tende cuidado, segundo o
exemplo que tendes em nós, pelos que assim andam. Porque
muitos há, dos quais muitas vezes vos disse, e agora também digo,
chorando, que são inimigos da cruz de Cristo, cujo fim é a
perdição; cujo Deus é o ventre, e cuja glória é para confusão deles,
que só pensam nas coisas terrenas. Mas a nossa cidade está
nos céus, de onde também esperamos o Salvador, o Senhor Jesus
Cristo, que transformará o nosso corpo abatido, para ser
conforme o seu corpo glorioso, segundo o seu eficaz poder de
sujeitar também a si todas as coisas.

\medskip

\lettrine{4} Portanto, meus amados e mui queridos irmãos,
minha alegria e coroa, estai assim firmes no Senhor, amados.
Rogo a Evódia, e rogo a Síntique, que sintam o mesmo no Senhor.
E peço-te também a ti, meu verdadeiro companheiro, que ajudes
essas mulheres que trabalharam comigo no evangelho, e com Clemente,
e com os outros cooperadores, cujos nomes estão no livro da vida.
Regozijai-vos sempre no Senhor; outra vez digo, regozijai-vos.
Seja a vossa eqüidade notória a todos os homens. Perto está o
Senhor. Não estejais inquietos por coisa alguma; antes as vossas
petições sejam em tudo conhecidas diante de Deus pela oração e
súplica, com ação de graças. E a paz de Deus, que excede todo o
entendimento, guardará os vossos corações e os vossos sentimentos em
Cristo Jesus. Quanto ao mais, irmãos, tudo o que é verdadeiro,
tudo o que é honesto, tudo o que é justo, tudo o que é puro, tudo o
que é amável, tudo o que é de boa fama, se há alguma virtude, e se
há algum louvor, nisso pensai. O que também aprendestes, e
recebestes, e ouvistes, e vistes em mim, isso fazei; e o Deus de paz
será convosco.

Ora, muito me regozijei no Senhor por finalmente reviver a vossa
lembrança de mim; pois já vos tínheis lembrado, mas não tínheis tido
oportunidade. Não digo isto como por necessidade, porque já
aprendi a contentar-me com o que tenho. Sei estar abatido, e
sei também ter abundância; em toda a maneira, e em todas as coisas
estou instruído, tanto a ter fartura, como a ter fome; tanto a ter
abundância, como a padecer necessidade. Posso todas as coisas
em Cristo que me fortalece. Todavia fizestes bem em tomar
parte na minha aflição. E bem sabeis também, ó filipenses,
que, no princípio do evangelho, quando parti da Macedônia, nenhuma
igreja comunicou comigo com respeito a dar e a receber, senão vós
somente; porque também uma e outra vez me mandastes o
necessário a Tessalônica. Não que procure dádivas, mas
procuro o fruto que cresça para a vossa conta. Mas bastante
tenho recebido, e tenho abundância. Cheio estou, depois que recebi
de Epafrodito o que da vossa parte me foi enviado, como cheiro de
suavidade e sacrifício agradável e aprazível a Deus. O meu
Deus, segundo as suas riquezas, suprirá todas as vossas necessidades
em glória, por Cristo Jesus.

Ora, a nosso Deus e Pai seja dada glória para todo o sempre.
Amém. Saudai a todos os santos em Cristo Jesus. Os irmãos que
estão comigo vos saúdam. Todos os santos vos saúdam, mas
principalmente os que são da casa de César. A graça de nosso
Senhor Jesus Cristo seja com vós todos. Amém.

