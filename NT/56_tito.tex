\addchap{Epístola de Paulo a Tito}

\lettrine{1} Paulo, servo de Deus, e apóstolo de Jesus Cristo,
segundo a fé dos eleitos de Deus, e o conhecimento da verdade, que é
segundo a piedade, em esperança da vida eterna, a qual Deus, que
não pode mentir, prometeu antes dos tempos dos séculos; mas a
seu tempo manifestou a sua palavra pela pregação que me foi confiada
segundo o mandamento de Deus, nosso Salvador; a Tito, meu
verdadeiro filho, segundo a fé comum: Graça, misericórdia, e paz da
parte de Deus Pai, e da do Senhor Jesus Cristo, nosso Salvador.

Por esta causa te deixei em Creta, para que pusesses em boa ordem
as coisas que ainda restam, e de cidade em cidade estabelecesses
presbíteros, como já te mandei: aquele que for irrepreensível,
marido de uma mulher, que tenha filhos fiéis, que não possam ser
acusados de dissolução nem são desobedientes. Porque convém que
o bispo seja irrepreensível, como despenseiro da casa de Deus, não
soberbo, nem iracundo, nem dado ao vinho, nem espancador, nem
cobiçoso de torpe ganância; mas dado à hospitalidade, amigo do
bem, moderado, justo, santo, temperante; retendo firme a fiel
palavra, que é conforme a doutrina, para que seja poderoso, tanto
para admoestar com a sã doutrina, como para convencer os
contradizentes. Porque há muitos desordenados, faladores,
vãos e enganadores, principalmente os da circuncisão, aos
quais convém tapar a boca; homens que transtornam casas inteiras
ensinando o que não convém, por torpe ganância. Um deles, seu
próprio profeta, disse: Os cretenses são sempre mentirosos, bestas
ruins, ventres preguiçosos. Este testemunho é verdadeiro.
Portanto, repreende-os severamente, para que sejam sãos na fé.
Não dando ouvidos às fábulas judaicas, nem aos mandamentos de
homens que se desviam da verdade. Todas as coisas são puras
para os puros, mas nada é puro para os contaminados e infiéis; antes
o seu entendimento e consciência estão contaminados.
Confessam que conhecem a Deus, mas negam-no com as obras,
sendo abomináveis, e desobedientes, e reprovados para toda a boa
obra.

\medskip

\lettrine{2} Tu, porém, fala o que convém à sã doutrina.
Os velhos, que sejam sóbrios, graves, prudentes, sãos na fé, no
amor, e na paciência; as mulheres idosas, semelhantemente, que
sejam sérias no seu viver, como convém a santas, não caluniadoras,
não dadas a muito vinho, mestras no bem; para que ensinem as
mulheres novas a serem prudentes, a amarem seus maridos, a amarem
seus filhos, a serem moderadas, castas, boas donas de casa,
sujeitas a seus maridos, a fim de que a palavra de Deus não seja
blasfemada. Exorta semelhantemente os jovens a que sejam
moderados. Em tudo te dá por exemplo de boas obras; na doutrina
mostra incorrupção, gravidade, sinceridade, linguagem sã e
irrepreensível, para que o adversário se envergonhe, não tendo
nenhum mal que dizer de nós. Exorta os servos a que se sujeitem
a seus senhores, e em tudo agradem, não contradizendo, não
defraudando, antes mostrando toda a boa lealdade, para que em tudo
sejam ornamento da doutrina de Deus, nosso Salvador.

Porque a graça de Deus se há manifestado, trazendo salvação a
todos os homens, ensinando-nos que, renunciando à impiedade e
às concupiscências mundanas, vivamos neste presente século sóbria, e
justa, e piamente, aguardando a bem-aventurada esperança e o
aparecimento da glória do grande Deus e nosso Senhor Jesus Cristo;
o qual se deu a si mesmo por nós para nos remir de toda a
iniqüidade, e purificar para si um povo seu especial, zeloso de boas
obras.

Fala disto, e exorta e repreende com toda a autoridade. Ninguém
te despreze.

\medskip

\lettrine{3} Admoesta-os a que se sujeitem aos principados e
potestades, que lhes obedeçam, e estejam preparados para toda a boa
obra; que a ninguém infamem, nem sejam contenciosos, mas
modestos, mostrando toda a mansidão para com todos os homens.
Porque também nós éramos noutro tempo insensatos, desobedientes,
extraviados, servindo a várias concupiscências e deleites, vivendo
em malícia e inveja, odiosos, odiando-nos uns aos outros. Mas
quando apareceu a benignidade e amor de Deus, nosso Salvador, para
com os homens, não pelas obras de justiça que houvéssemos feito,
mas segundo a sua misericórdia, nos salvou pela lavagem da
regeneração e da renovação do Espírito Santo, que abundantemente
ele derramou sobre nós por Jesus Cristo nosso Salvador; para
que, sendo justificados pela sua graça, sejamos feitos herdeiros
segundo a esperança da vida eterna. Fiel é a palavra, e isto
quero que deveras afirmes, para que os que crêem em Deus procurem
aplicar-se às boas obras; estas coisas são boas e proveitosas aos
homens.

Mas não entres em questões loucas, genealogias e contendas, e nos
debates acerca da lei; porque são coisas inúteis e vãs. Ao
homem herege, depois de uma e outra admoestação, evita-o,
sabendo que esse tal está pervertido, e peca, estando já em
si mesmo condenado. Quando te enviar Ártemas, ou Tíquico,
procura vir ter comigo a Nicópolis; porque deliberei invernar ali.
Acompanha com muito cuidado Zenas, doutor da lei, e Apolo,
para que nada lhes falte. E os nossos aprendam também a
aplicar-se às boas obras, nas coisas necessárias, para que não sejam
infrutuosos. Saúdam-te todos os que estão comigo. Saúda tu os
que nos amam na fé. A graça seja com vós todos. Amém.

