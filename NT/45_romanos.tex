\thispagestyle{empty}
\chapter*{Epístola aos Romanos}

\lettrine{1} Paulo, servo de Jesus Cristo, chamado para
apóstolo, separado para o evangelho de Deus. O qual antes
prometeu pelos seus profetas nas santas escrituras, acerca de
seu Filho, que nasceu da descendência de Davi segundo a carne,
declarado Filho de Deus em poder, segundo o Espírito de
santificação, pela ressurreição dos mortos, Jesus Cristo, nosso
Senhor, pelo qual recebemos a graça e o apostolado, para a
obediência da fé entre todas as gentes pelo seu nome, entre as
quais sois também vós chamados para serdes de Jesus Cristo. A
todos os que estais em Roma, amados de Deus, chamados santos: Graça
e paz de Deus nosso Pai, e do Senhor Jesus Cristo.

Primeiramente dou graças ao meu Deus por Jesus Cristo, acerca de
vós todos, porque em todo o mundo é anunciada a vossa fé. Porque
Deus, a quem sirvo em meu espírito, no evangelho de seu Filho, me é
testemunha de como incessantemente faço menção de vós,
pedindo sempre em minhas orações que nalgum tempo, pela
vontade de Deus, se me ofereça boa ocasião de ir ter convosco.
Porque desejo ver-vos, para vos comunicar algum dom
espiritual, a fim de que sejais confortados; isto é, para que
juntamente convosco eu seja consolado pela fé mútua, assim vossa
como minha. Não quero, porém, irmãos, que ignoreis que muitas
vezes propus ir ter convosco (mas até agora tenho sido impedido)
para também ter entre vós algum fruto, como também entre os demais
gentios. Eu sou devedor, tanto a gregos como a bárbaros,
tanto a sábios como a ignorantes. E assim, quanto está em
mim, estou pronto para também vos anunciar o evangelho, a vós que
estais em Roma.

Porque não me envergonho do evangelho de Cristo, pois é o poder
de Deus para salvação de todo aquele que crê; primeiro do judeu, e
também do grego. Porque nele se descobre a justiça de Deus de
fé em fé, como está escrito: Mas o justo viverá da fé. Porque
do céu se manifesta a ira de Deus sobre toda a impiedade e injustiça
dos homens, que detêm a verdade em injustiça.

Porquanto o que de Deus se pode conhecer neles se manifesta,
porque Deus lho manifestou. Porque as suas coisas invisíveis,
desde a criação do mundo, tanto o seu eterno poder, como a sua
divindade, se entendem, e claramente se vêem pelas coisas que estão
criadas, para que eles fiquem inescusáveis; porquanto, tendo
conhecido a Deus, não o glorificaram como Deus, nem lhe deram
graças, antes em seus discursos se desvaneceram, e o seu coração
insensato se obscureceu. Dizendo-se sábios, tornaram-se
loucos. E mudaram a glória do Deus incorruptível em
semelhança da imagem de homem corruptível, e de aves, e de
quadrúpedes, e de répteis. Por isso também Deus os entregou
às concupiscências de seus corações, à imundícia, para desonrarem
seus corpos entre si; pois mudaram a verdade de Deus em
mentira, e honraram e serviram mais a criatura do que o Criador, que
é bendito eternamente. Amém. Por isso Deus os abandonou às
paixões infames. Porque até as suas mulheres mudaram o uso natural,
no contrário à natureza. E, semelhantemente, também os
homens, deixando o uso natural da mulher, se inflamaram em sua
sensualidade uns para com os outros, homens com homens, cometendo
torpeza e recebendo em si mesmos a recompensa que convinha ao seu
erro. E, como eles não se importaram de ter conhecimento de
Deus, assim Deus os entregou a um sentimento perverso, para fazerem
coisas que não convêm; estando cheios de toda a iniqüidade,
prostituição, malícia, avareza, maldade; cheios de inveja,
homicídio, contenda, engano, malignidade; sendo murmuradores,
detratores, aborrecedores de Deus, injuriadores, soberbos,
presunçosos, inventores de males, desobedientes aos pais e às mães;
néscios, infiéis nos contratos, sem afeição natural,
irreconciliáveis, sem misericórdia; os quais, conhecendo a
justiça de Deus (que são dignos de morte os que tais coisas
praticam), não somente as fazem, mas também consentem aos que as
fazem.

\medskip

\lettrine{2} Portanto, és inescusável quando julgas, ó homem,
quem quer que sejas, porque te condenas a ti mesmo naquilo em que
julgas a outro; pois tu, que julgas, fazes o mesmo. E bem
sabemos que o juízo de Deus é segundo a verdade sobre os que tais
coisas fazem. E tu, ó homem, que julgas os que fazem tais
coisas, cuidas que, fazendo-as tu, escaparás ao juízo de Deus?
Ou desprezas tu as riquezas da sua benignidade, e paciência e
longanimidade, ignorando que a benignidade de Deus te leva ao
arrependimento? Mas, segundo a tua dureza e teu coração
impenitente, entesouras ira para ti no dia da ira e da manifestação
do juízo de Deus; o qual recompensará cada um segundo as suas
obras; a saber: a vida eterna aos que, com perseverança em fazer
bem, procuram glória, honra e incorrupção; mas a indignação e a
ira aos que são contenciosos, desobedientes à verdade e obedientes à
iniqüidade; tribulação e angústia sobre toda a alma do homem que
faz o mal; primeiramente do judeu e também do grego; glória,
porém, e honra e paz a qualquer que pratica o bem; primeiramente ao
judeu e também ao grego; porque, para com Deus, não há
acepção de pessoas. Porque todos os que sem lei pecaram, sem
lei também perecerão; e todos os que sob a lei pecaram, pela lei
serão julgados. Porque os que ouvem a lei não são justos
diante de Deus, mas os que praticam a lei hão de ser justificados.
Porque, quando os gentios, que não têm lei, fazem
naturalmente as coisas que são da lei, não tendo eles lei, para si
mesmos são lei; os quais mostram a obra da lei escrita em
seus corações, testificando juntamente a sua consciência, e os seus
pensamentos, quer acusando-os, quer defendendo-os; no dia em
que Deus há de julgar os segredos dos homens, por Jesus Cristo,
segundo o meu evangelho.

Eis que tu que tens por sobrenome judeu, e repousas na lei, e te
glorias em Deus; e sabes a sua vontade e aprovas as coisas
excelentes, sendo instruído por lei; e confias que és guia
dos cegos, luz dos que estão em trevas, instrutor dos
néscios, mestre de crianças, que tens a forma da ciência e da
verdade na lei; tu, pois, que ensinas a outro, não te ensinas
a ti mesmo? Tu, que pregas que não se deve furtar, furtas?
Tu, que dizes que não se deve adulterar, adulteras? Tu, que
abominas os ídolos, cometes sacrilégio? Tu, que te glorias na
lei, desonras a Deus pela transgressão da lei? Porque, como
está escrito, o nome de Deus é blasfemado entre os gentios por causa
de vós. Porque a circuncisão é, na verdade, proveitosa, se tu
guardares a lei; mas, se tu és transgressor da lei, a tua
circuncisão se torna em incircuncisão. Se, pois, a
incircuncisão guardar os preceitos da lei, porventura a
incircuncisão não será reputada como circuncisão? E a
incircuncisão que por natureza o é, se cumpre a lei, não te julgará
porventura a ti, que pela letra e circuncisão és transgressor da
lei? Porque não é judeu o que o é exteriormente, nem é
circuncisão a que o é exteriormente na carne. Mas é judeu o
que o é no interior, e circuncisão a que é do coração, no espírito,
não na letra; cujo louvor não provém dos homens, mas de Deus.

\medskip

\lettrine{3} Qual é, pois, a vantagem do judeu? Ou qual a
utilidade da circuncisão? Muita, em toda a maneira, porque,
primeiramente, as palavras de Deus lhe foram confiadas. Pois
quê? Se alguns foram incrédulos, a sua incredulidade aniquilará a
fidelidade de Deus? De maneira nenhuma; sempre seja Deus
verdadeiro, e todo o homem mentiroso; como está escrito: Para que
sejas justificado em tuas palavras, e venças quando fores julgado.
E, se a nossa injustiça for causa da justiça de Deus, que
diremos? Porventura será Deus injusto, trazendo ira sobre nós? (Falo
como homem.) De maneira nenhuma; de outro modo, como julgará
Deus o mundo? Mas, se pela minha mentira abundou mais a verdade
de Deus para glória sua, por que sou eu ainda julgado também como
pecador? E por que não dizemos (como somos blasfemados, e como
alguns dizem que dizemos): Façamos males, para que venham bens? A
condenação desses é justa. Pois quê? Somos nós mais excelentes?
De maneira nenhuma, pois já dantes demonstramos que, tanto judeus
como gregos, todos estão debaixo do pecado; como está
escrito: Não há um justo, nem um sequer. Não há ninguém que
entenda; não há ninguém que busque a Deus. Todos se
extraviaram, e juntamente se fizeram inúteis. Não há quem faça o
bem, não há nem um só. A sua garganta é um sepulcro aberto;
com as suas línguas tratam enganosamente; peçonha de áspides está
debaixo de seus lábios; cuja boca está cheia de maldição e
amargura. Os seus pés são ligeiros para derramar sangue.
Em seus caminhos há destruição e miséria; e não
conheceram o caminho da paz. Não há temor de Deus diante de
seus olhos.

Ora, nós sabemos que tudo o que a lei diz, aos que estão debaixo
da lei o diz, para que toda a boca esteja fechada e todo o mundo
seja condenável diante de Deus. Por isso nenhuma carne será
justificada diante dele pelas obras da lei, porque pela lei vem o
conhecimento do pecado. Mas agora se manifestou sem a lei a
justiça de Deus, tendo o testemunho da lei e dos profetas;
isto é, a justiça de Deus pela fé em Jesus Cristo para todos
e sobre todos os que crêem; porque não há diferença. Porque
todos pecaram e destituídos estão da glória de Deus; sendo
justificados gratuitamente pela sua graça, pela redenção que há em
Cristo Jesus. Ao qual Deus propôs para propiciação pela fé no
seu sangue, para demonstrar a sua justiça pela remissão dos pecados
dantes cometidos, sob a paciência de Deus; para demonstração
da sua justiça neste tempo presente, para que ele seja justo e
justificador daquele que tem fé em Jesus. Onde está logo a
jactância? É excluída. Por qual lei? Das obras? Não; mas pela lei da
fé. Concluímos, pois, que o homem é justificado pela fé sem
as obras da lei. É porventura Deus somente dos judeus? E não
o é também dos gentios? Também dos gentios, certamente, visto
que Deus é um só, que justifica pela fé a circuncisão, e por meio da
fé a incircuncisão. Anulamos, pois, a lei pela fé? De maneira
nenhuma, antes estabelecemos a lei.

\medskip

\lettrine{4} Que diremos, pois, ter alcançado Abraão, nosso
pai segundo a carne? Porque, se Abraão foi justificado pelas
obras, tem de que se gloriar, mas não diante de Deus. Pois, que
diz a Escritura? Creu Abraão em Deus, e isso lhe foi imputado como
justiça. Ora, àquele que faz qualquer obra não lhe é imputado o
galardão segundo a graça, mas segundo a dívida. Mas, àquele que
não pratica, mas crê naquele que justifica o ímpio, a sua fé lhe é
imputada como justiça. Assim também Davi declara bem-aventurado
o homem a quem Deus imputa a justiça sem as obras, dizendo:
Bem-aventurados aqueles cujas maldades são perdoadas, e cujos
pecados são cobertos. Bem-aventurado o homem a quem o Senhor não
imputa o pecado.

Vem, pois, esta bem-aventurança sobre a circuncisão somente, ou
também sobre a incircuncisão? Porque dizemos que a fé foi imputada
como justiça a Abraão. Como lhe foi, pois, imputada? Estando
na circuncisão ou na incircuncisão? Não na circuncisão, mas na
incircuncisão. E recebeu o sinal da circuncisão, selo da
justiça da fé, quando estava na incircuncisão, para que fosse pai de
todos os que crêem, estando eles também na incircuncisão; a fim de
que também a justiça lhes seja imputada; e fosse pai da
circuncisão, daqueles que não somente são da circuncisão, mas que
também andam nas pisadas daquela fé que teve nosso pai Abraão, que
tivera na incircuncisão. Porque a promessa de que havia de
ser herdeiro do mundo não foi feita pela lei a Abraão, ou à sua
posteridade, mas pela justiça da fé. Porque, se os que são da
lei são herdeiros, logo a fé é vã e a promessa é aniquilada.
Porque a lei opera a ira. Porque onde não há lei também não
há transgressão. Portanto, é pela fé, para que seja segundo a
graça, a fim de que a promessa seja firme a toda a posteridade, não
somente à que é da lei, mas também à que é da fé que teve Abraão, o
qual é pai de todos nós,

está escrito: Por pai de muitas nações te constituí) perante
aquele no qual creu, a saber, Deus, o qual vivifica os mortos, e
chama as coisas que não são como se já fossem. O qual, em
esperança, creu contra a esperança, tanto que ele tornou-se pai de
muitas nações, conforme o que lhe fora dito: Assim será a tua
descendência. E não enfraquecendo na fé, não atentou para o
seu próprio corpo já amortecido, pois era já de quase cem anos, nem
tampouco para o amortecimento do ventre de Sara. E não
duvidou da promessa de Deus por incredulidade, mas foi fortificado
na fé, dando glória a Deus, e estando certíssimo de que o que
ele tinha prometido também era poderoso para o fazer. Assim
isso lhe foi também imputado como justiça.

Ora, não só por causa dele está escrito, que lhe fosse tomado em
conta, mas também por nós, a quem será tomado em conta, os
que cremos naquele que dentre os mortos ressuscitou a Jesus nosso
Senhor; o qual por nossos pecados foi entregue, e ressuscitou
para nossa justificação.

\medskip

\lettrine{5} Tendo sido, pois, justificados pela fé, temos paz
com Deus, por nosso Senhor Jesus Cristo; pelo qual também temos
entrada pela fé a esta graça, na qual estamos firmes, e nos
gloriamos na esperança da glória de Deus. E não somente isto,
mas também nos gloriamos nas tribulações; sabendo que a tribulação
produz a paciência, e a paciência a experiência, e a experiência
a esperança. E a esperança não traz confusão, porquanto o amor
de Deus está derramado em nossos corações pelo Espírito Santo que
nos foi dado.

Porque Cristo, estando nós ainda fracos, morreu a seu tempo pelos
ímpios. Porque apenas alguém morrerá por um justo; pois poderá
ser que pelo bom alguém ouse morrer. Mas Deus prova o seu amor
para conosco, em que Cristo morreu por nós, sendo nós ainda
pecadores. Logo muito mais agora, tendo sido justificados pelo
seu sangue, seremos por ele salvos da ira. Porque se nós,
sendo inimigos, fomos reconciliados com Deus pela morte de seu
Filho, muito mais, tendo sido já reconciliados, seremos salvos pela
sua vida. E não somente isto, mas também nos gloriamos em
Deus por nosso Senhor Jesus Cristo, pelo qual agora alcançamos a
reconciliação. Portanto, como por um homem entrou o pecado no
mundo, e pelo pecado a morte, assim também a morte passou a todos os
homens, por isso que todos pecaram. Porque até à lei estava o
pecado no mundo, mas o pecado não é imputado, não havendo lei.
No entanto, a morte reinou desde Adão até Moisés, até sobre
aqueles que não tinham pecado à semelhança da transgressão de Adão,
o qual é a figura daquele que havia de vir. Mas não é assim o
dom gratuito como a ofensa. Porque, se pela ofensa de um morreram
muitos, muito mais a graça de Deus, e o dom pela graça, que é de um
só homem, Jesus Cristo, abundou sobre muitos. E não foi assim
o dom como a ofensa, por um só que pecou. Porque o juízo veio de uma
só ofensa, na verdade, para condenação, mas o dom gratuito veio de
muitas ofensas para justificação. Porque, se pela ofensa de
um só, a morte reinou por esse, muito mais os que recebem a
abundância da graça, e do dom da justiça, reinarão em vida por um
só, Jesus Cristo. Pois assim como por uma só ofensa veio o
juízo sobre todos os homens para condenação, assim também por um só
ato de justiça veio a graça sobre todos os homens para justificação
de vida. Porque, como pela desobediência de um só homem,
muitos foram feitos pecadores, assim pela obediência de um muitos
serão feitos justos. Veio, porém, a lei para que a ofensa
abundasse; mas, onde o pecado abundou, superabundou a graça;
para que, assim como o pecado reinou na morte, também a graça
reinasse pela justiça para a vida eterna, por Jesus Cristo nosso
Senhor.

\medskip

\lettrine{6} Que diremos pois? Permaneceremos no pecado, para
que a graça abunde? De modo nenhum. Nós, que estamos mortos para
o pecado, como viveremos ainda nele? Ou não sabeis que todos
quantos fomos batizados em Jesus Cristo fomos batizados na sua
morte? De sorte que fomos sepultados com ele pelo batismo na
morte; para que, como Cristo foi ressuscitado dentre os mortos, pela
glória do Pai, assim andemos nós também em novidade de vida.
Porque, se fomos plantados juntamente com ele na semelhança da
sua morte, também o seremos na da sua ressurreição; sabendo
isto, que o nosso homem velho foi com ele crucificado, para que o
corpo do pecado seja desfeito, para que não sirvamos mais ao pecado.
Porque aquele que está morto está justificado do pecado.
Ora, se já morremos com Cristo, cremos que também com ele
viveremos; sabendo que, tendo sido Cristo ressuscitado dentre os
mortos, já não morre; a morte não mais tem domínio sobre ele.
Pois, quanto a ter morrido, de uma vez morreu para o pecado;
mas, quanto a viver, vive para Deus. Assim também vós
considerai-vos como mortos para o pecado, mas vivos para Deus em
Cristo Jesus nosso Senhor. Não reine, portanto, o pecado em
vosso corpo mortal, para lhe obedecerdes em suas concupiscências;
nem tampouco apresenteis os vossos membros ao pecado por
instrumentos de iniqüidade; mas apresentai-vos a Deus, como vivos
dentre mortos, e os vossos membros a Deus, como instrumentos de
justiça. Porque o pecado não terá domínio sobre vós, pois não
estais debaixo da lei, mas debaixo da graça. Pois que?
Pecaremos porque não estamos debaixo da lei, mas debaixo da graça?
De modo nenhum. Não sabeis vós que a quem vos apresentardes
por servos para lhe obedecer, sois servos daquele a quem obedeceis,
ou do pecado para a morte, ou da obediência para a justiça?
Mas graças a Deus que, tendo sido servos do pecado,
obedecestes de coração à forma de doutrina a que fostes entregues.
E, libertados do pecado, fostes feitos servos da justiça.
Falo como homem, pela fraqueza da vossa carne; pois que,
assim como apresentastes os vossos membros para servirem à
imundícia, e à maldade para maldade, assim apresentai agora os
vossos membros para servirem à justiça para santificação.
Porque, quando éreis servos do pecado, estáveis livres da
justiça. E que fruto tínheis então das coisas de que agora
vos envergonhais? Porque o fim delas é a morte. Mas agora,
libertados do pecado, e feitos servos de Deus, tendes o vosso fruto
para santificação, e por fim a vida eterna. Porque o salário
do pecado é a morte, mas o dom gratuito de Deus é a vida eterna, por
Cristo Jesus nosso Senhor.

\medskip

\lettrine{7} Não sabeis vós, irmãos (pois que falo aos que
sabem a lei), que a lei tem domínio sobre o homem por todo o tempo
que vive? Porque a mulher que está sujeita ao marido, enquanto
ele viver, está-lhe ligada pela lei; mas, morto o marido, está livre
da lei do marido. De sorte que, vivendo o marido, será chamada
adúltera se for de outro marido; mas, morto o marido, livre está da
lei, e assim não será adúltera, se for de outro marido. Assim,
meus irmãos, também vós estais mortos para a lei pelo corpo de
Cristo, para que sejais de outro, daquele que ressuscitou dentre os
mortos, a fim de que demos fruto para Deus. Porque, quando
estávamos na carne, as paixões dos pecados, que são pela lei,
operavam em nossos membros para darem fruto para a morte. Mas
agora temos sido libertados da lei, tendo morrido para aquilo em que
estávamos retidos; para que sirvamos em novidade de espírito, e não
na velhice da letra.

Que diremos pois? É a lei pecado? De modo nenhum. Mas eu não
conheci o pecado senão pela lei; porque eu não conheceria a
concupiscência, se a lei não dissesse: Não cobiçarás. Mas o
pecado, tomando ocasião pelo mandamento, operou em mim toda a
concupiscência; porquanto sem a lei estava morto o pecado. E eu,
nalgum tempo, vivia sem lei, mas, vindo o mandamento, reviveu o
pecado, e eu morri. E o mandamento que era para vida, achei
eu que me era para morte. Porque o pecado, tomando ocasião
pelo mandamento, me enganou, e por ele me matou. E assim a
lei é santa, e o mandamento santo, justo e bom. Logo
tornou-se-me o bom em morte? De modo nenhum; mas o pecado, para que
se mostrasse pecado, operou em mim a morte pelo bem; a fim de que
pelo mandamento o pecado se fizesse excessivamente maligno.

Porque bem sabemos que a lei é espiritual; mas eu sou carnal,
vendido sob o pecado. Porque o que faço não o aprovo; pois o
que quero isso não faço, mas o que aborreço isso faço. E, se
faço o que não quero, consinto com a lei, que é boa. De
maneira que agora já não sou eu que faço isto, mas o pecado que
habita em mim. Porque eu sei que em mim, isto é, na minha
carne, não habita bem algum; e com efeito o querer está em mim, mas
não consigo realizar o bem. Porque não faço o bem que quero,
mas o mal que não quero esse faço. Ora, se eu faço o que não
quero, já o não faço eu, mas o pecado que habita em mim. Acho
então esta lei em mim, que, quando quero fazer o bem, o mal está
comigo. Porque, segundo o homem interior, tenho prazer na lei
de Deus; mas vejo nos meus membros outra lei, que batalha
contra a lei do meu entendimento, e me prende debaixo da lei do
pecado que está nos meus membros. Miserável homem que eu sou!
quem me livrará do corpo desta morte? Dou graças a Deus por
Jesus Cristo nosso Senhor. Assim que eu mesmo com o entendimento
sirvo à lei de Deus, mas com a carne à lei do pecado.

\medskip

\lettrine{8} Portanto, agora nenhuma condenação há para os que
estão em Cristo Jesus, que não andam segundo a carne, mas segundo o
Espírito. Porque a lei do Espírito de vida, em Cristo Jesus, me
livrou da lei do pecado e da morte. Porquanto o que era
impossível à lei, visto como estava enferma pela carne, Deus,
enviando o seu Filho em semelhança da carne do pecado, pelo pecado
condenou o pecado na carne; para que a justiça da lei se
cumprisse em nós, que não andamos segundo a carne, mas segundo o
Espírito. Porque os que são segundo a carne inclinam-se para as
coisas da carne; mas os que são segundo o Espírito para as coisas do
Espírito. Porque a inclinação da carne é morte; mas a inclinação
do Espírito é vida e paz. Porquanto a inclinação da carne é
inimizade contra Deus, pois não é sujeita à lei de Deus, nem, em
verdade, o pode ser. Portanto, os que estão na carne não podem
agradar a Deus. Vós, porém, não estais na carne, mas no
Espírito, se é que o Espírito de Deus habita em vós. Mas, se alguém
não tem o Espírito de Cristo, esse tal não é dele.

E, se Cristo está em vós, o corpo, na verdade, está morto por
causa do pecado, mas o espírito vive por causa da justiça. E,
se o Espírito daquele que dentre os mortos ressuscitou a Jesus
habita em vós, aquele que dentre os mortos ressuscitou a Cristo
também vivificará os vossos corpos mortais, pelo seu Espírito que em
vós habita. De maneira que, irmãos, somos devedores, não à
carne para viver segundo a carne. Porque, se viverdes segundo
a carne, morrereis; mas, se pelo Espírito mortificardes as obras do
corpo, vivereis. Porque todos os que são guiados pelo
Espírito de Deus, esses são filhos de Deus. Porque não
recebestes o espírito de escravidão, para outra vez estardes em
temor, mas recebestes o Espírito de adoção de filhos, pelo qual
clamamos: Aba, Pai. O mesmo Espírito testifica com o nosso
espírito que somos filhos de Deus.

E, se nós somos filhos, somos logo herdeiros também, herdeiros de
Deus, e co-herdeiros de Cristo: se é certo que com ele padecemos,
para que também com ele sejamos glorificados. Porque para mim
tenho por certo que as aflições deste tempo presente não são para
comparar com a glória que em nós há de ser revelada. Porque a
ardente expectação da criatura espera a manifestação dos filhos de
Deus. Porque a criação ficou sujeita à vaidade, não por sua
vontade, mas por causa do que a sujeitou, na esperança de que
também a mesma criatura será libertada da servidão da corrupção,
para a liberdade da glória dos filhos de Deus. Porque sabemos
que toda a criação geme e está juntamente com dores de parto até
agora. E não só ela, mas nós mesmos, que temos as primícias
do Espírito, também gememos em nós mesmos, esperando a adoção, a
saber, a redenção do nosso corpo. Porque em esperança fomos
salvos. Ora a esperança que se vê não é esperança; porque o que
alguém vê como o esperará? Mas, se esperamos o que não vemos,
com paciência o esperamos.

E da mesma maneira também o Espírito ajuda as nossas fraquezas;
porque não sabemos o que havemos de pedir como convém, mas o mesmo
Espírito intercede por nós com gemidos inexprimíveis. E
aquele que examina os corações sabe qual é a intenção do Espírito; e
é ele que segundo Deus intercede pelos santos. E sabemos que
todas as coisas contribuem juntamente para o bem daqueles que amam a
Deus, daqueles que são chamados segundo o seu propósito.

Porque os que dantes conheceu também os predestinou para serem
conformes à imagem de seu Filho, a fim de que ele seja o primogênito
entre muitos irmãos. E aos que predestinou a estes também
chamou; e aos que chamou a estes também justificou; e aos que
justificou a estes também glorificou.

Que diremos, pois, a estas coisas? Se Deus é por nós, quem será
contra nós? Aquele que nem mesmo a seu próprio Filho poupou,
antes o entregou por todos nós, como nos não dará também com ele
todas as coisas? Quem intentará acusação contra os escolhidos
de Deus? É Deus quem os justifica. Quem é que condena? Pois é
Cristo quem morreu, ou antes quem ressuscitou dentre os mortos, o
qual está à direita de Deus, e também intercede por nós. Quem
nos separará do amor de Cristo? A tribulação, ou a angústia, ou a
perseguição, ou a fome, ou a nudez, ou o perigo, ou a espada?
Como está escrito: Por amor de ti somos entregues à morte
todo o dia; somos reputados como ovelhas para o matadouro.
Mas em todas estas coisas somos mais do que vencedores, por
aquele que nos amou. Porque estou certo de que, nem a morte,
nem a vida, nem os anjos, nem os principados, nem as potestades, nem
o presente, nem o porvir, nem a altura, nem a profundidade,
nem alguma outra criatura nos poderá separar do amor de Deus, que
está em Cristo Jesus nosso Senhor.

\medskip

\lettrine{9} Em Cristo digo a verdade, não minto (dando-me
testemunho a minha consciência no Espírito Santo): que tenho
grande tristeza e contínua dor no meu coração. Porque eu mesmo
poderia desejar ser anátema de Cristo, por amor de meus irmãos, que
são meus parentes segundo a carne; que são israelitas, dos quais
é a adoção de filhos, e a glória, e as alianças, e a lei, e o culto,
e as promessas; dos quais são os pais, e dos quais é Cristo
segundo a carne, o qual é sobre todos, Deus bendito eternamente.
Amém.

Não que a palavra de Deus haja faltado, porque nem todos os que
são de Israel são israelitas; nem por serem descendência de
Abraão são todos filhos; mas: Em Isaque será chamada a tua
descendência. Isto é, não são os filhos da carne que são filhos
de Deus, mas os filhos da promessa são contados como descendência.
Porque a palavra da promessa é esta: Por este tempo virei, e
Sara terá um filho. E não somente esta, mas também Rebeca,
quando concebeu de um, de Isaque, nosso pai; porque, não
tendo eles ainda nascido, nem tendo feito bem ou mal (para que o
propósito de Deus, segundo a eleição, ficasse firme, não por causa
das obras, mas por aquele que chama), foi-lhe dito a ela: O
maior servirá o menor. Como está escrito: Amei a Jacó, e
odiei a Esaú.

Que diremos pois? que há injustiça da parte de Deus? De maneira
nenhuma. Pois diz a Moisés: Compadecer-me-ei de quem me
compadecer, e terei misericórdia de quem eu tiver misericórdia.
Assim, pois, isto não depende do que quer, nem do que corre,
mas de Deus, que se compadece. Porque diz a Escritura a
Faraó: Para isto mesmo te levantei; para em ti mostrar o meu poder,
e para que o meu nome seja anunciado em toda a terra. Logo,
pois, compadece-se de quem quer, e endurece a quem quer.
Dir-me-ás então: Por que se queixa ele ainda? Porquanto, quem
tem resistido à sua vontade? Mas, ó homem, quem és tu, que a
Deus replicas? Porventura a coisa formada dirá ao que a formou: Por
que me fizeste assim? Ou não tem o oleiro poder sobre o
barro, para da mesma massa fazer um vaso para honra e outro para
desonra? E que direis se Deus, querendo mostrar a sua ira, e
dar a conhecer o seu poder, suportou com muita paciência os vasos da
ira, preparados para a perdição; para que também desse a
conhecer as riquezas da sua glória nos vasos de misericórdia, que
para glória já dantes preparou, os quais somos nós, a quem
também chamou, não só dentre os judeus, mas também dentre os
gentios?

Como também diz em Oséias: Chamarei meu povo ao que não era meu
povo; e amada à que não era amada. E sucederá que no lugar em
que lhes foi dito: Vós não sois meu povo; aí serão chamados filhos
do Deus vivo. Também Isaías clama acerca de Israel: Ainda que
o número dos filhos de Israel seja como a areia do mar, o
remanescente é que será salvo. Porque ele completará a obra e
abreviá-la-á em justiça; porque o Senhor fará breve a obra sobre a
terra. E como antes disse Isaías: Se o Senhor dos Exércitos
nos não deixara descendência, teríamos nos tornado como Sodoma, e
teríamos sido feitos como Gomorra.

Que diremos pois? Que os gentios, que não buscavam a justiça,
alcançaram a justiça? Sim, mas a justiça que é pela fé. Mas
Israel, que buscava a lei da justiça, não chegou à lei da justiça.
Por quê? Porque não foi pela fé, mas como que pelas obras da
lei; tropeçaram na pedra de tropeço; como está escrito: Eis
que eu ponho em Sião uma pedra de tropeço, e uma rocha de escândalo;
e todo aquele que crer nela não será confundido.

\medskip

\lettrine{10} Irmãos, o bom desejo do meu coração e a oração a
Deus por Israel é para sua salvação. Porque lhes dou testemunho
de que têm zelo de Deus, mas não com entendimento. Porquanto,
não conhecendo a justiça de Deus, e procurando estabelecer a sua
própria justiça, não se sujeitaram à justiça de Deus. Porque o
fim da lei é Cristo para justiça de todo aquele que crê. Ora
Moisés descreve a justiça que é pela lei, dizendo: O homem que fizer
estas coisas viverá por elas. Mas a justiça que é pela fé diz
assim: Não digas em teu coração: Quem subirá ao céu? (isto é, a
trazer do alto a Cristo.) Ou: Quem descerá ao abismo? (isto é, a
tornar a trazer dentre os mortos a Cristo.) Mas que diz? A
palavra está junto de ti, na tua boca e no teu coração; esta é a
palavra da fé, que pregamos, a saber: Se com a tua boca
confessares ao Senhor Jesus, e em teu coração creres que Deus o
ressuscitou dentre os mortos, serás salvo. Visto que com o
coração se crê para a justiça, e com a boca se faz confissão para a
salvação. Porque a Escritura diz: Todo aquele que nele crer
não será confundido.

Porquanto não há diferença entre judeu e grego; porque um mesmo é
o Senhor de todos, rico para com todos os que o invocam.
Porque todo aquele que invocar o nome do SENHOR será salvo.
Como, pois, invocarão aquele em quem não creram? e como
crerão naquele de quem não ouviram? e como ouvirão, se não há quem
pregue? E como pregarão, se não forem enviados? como está
escrito: Quão formosos os pés dos que anunciam o evangelho de paz;
dos que trazem alegres novas de boas coisas. Mas nem todos
têm obedecido ao evangelho; pois Isaías diz: SENHOR, quem creu na
nossa pregação? De sorte que a fé é pelo ouvir, e o ouvir
pela palavra de Deus. Mas digo: Porventura não ouviram? Sim,
por certo, pois por toda a terra saiu a voz deles, e as suas
palavras até aos confins do mundo. Mas digo: Porventura
Israel não o soube? Primeiramente diz Moisés: Eu vos porei em ciúmes
com aqueles que não são povo, com gente insensata vos provocarei à
ira. E Isaías ousadamente diz: Fui achado pelos que não me
buscavam, fui manifestado aos que por mim não perguntavam.
Mas para Israel diz: Todo o dia estendi as minhas mãos a um
povo rebelde e contradizente.

\medskip

\lettrine{11} Digo, pois: Porventura rejeitou Deus o seu povo?
De modo nenhum; porque também eu sou israelita, da descendência de
Abraão, da tribo de Benjamim. Deus não rejeitou o seu povo, que
antes conheceu. Ou não sabeis o que a Escritura diz de Elias, como
fala a Deus contra Israel, dizendo: Senhor, mataram os teus
profetas, e derribaram os teus altares; e só eu fiquei, e buscam a
minha alma? Mas que lhe diz a resposta divina? Reservei para mim
sete mil homens, que não dobraram os joelhos a Baal. Assim,
pois, também agora neste tempo ficou um remanescente, segundo a
eleição da graça. Mas se é por graça, já não é pelas obras; de
outra maneira, a graça já não é graça. Se, porém, é pelas obras, já
não é mais graça; de outra maneira a obra já não é obra. Pois
quê? O que Israel buscava não o alcançou; mas os eleitos o
alcançaram, e os outros foram endurecidos. Como está escrito:
Deus lhes deu espírito de profundo sono, olhos para não verem, e
ouvidos para não ouvirem, até ao dia de hoje. E Davi diz:
Torne-se-lhes a sua mesa em laço, e em armadilha, e em tropeço, por
sua retribuição; escureçam-se-lhes os olhos para não verem, e
encurvem-se-lhes continuamente as costas. Digo, pois:
Porventura tropeçaram, para que caíssem? De modo nenhum, mas pela
sua queda veio a salvação aos gentios, para os incitar à
emulação\footnote{Sentimento que incita a igualar ou superar outrem.
Competição, rivalidade, concorrência. Estímulo, incentivo.}.
E se a sua queda é a riqueza do mundo, e a sua diminuição a
riqueza dos gentios, quanto mais a sua plenitude! Porque
convosco falo, gentios, que, enquanto for apóstolo dos gentios,
exalto o meu ministério; para ver se de alguma maneira posso
incitar à emulação os da minha carne e salvar alguns deles.
Porque, se a sua rejeição é a reconciliação do mundo, qual
será a sua admissão, senão a vida dentre os mortos? E, se as
primícias são santas, também a massa o é; se a raiz é santa, também
os ramos o são. E se alguns dos ramos foram quebrados, e tu,
sendo zambujeiro\footnote{Espécie de oliveira brava, de madeira
rija; azambujeiro.}, foste enxertado em lugar deles, e feito
participante da raiz e da seiva da oliveira, não te glories
contra os ramos; e, se contra eles te gloriares, não és tu que
sustentas a raiz, mas a raiz a ti. Dirás, pois: Os ramos
foram quebrados, para que eu fosse enxertado. Está bem; pela
sua incredulidade foram quebrados, e tu estás em pé pela fé. Então
não te ensoberbeças, mas teme. Porque, se Deus não poupou os
ramos naturais, teme que não te poupe a ti também. Considera,
pois, a bondade e a severidade de Deus: para com os que caíram,
severidade; mas para contigo, benignidade, se permaneceres na sua
benignidade; de outra maneira também tu serás cortado. E
também eles, se não permanecerem na incredulidade, serão enxertados;
porque poderoso é Deus para os tornar a enxertar. Porque, se
tu foste cortado do natural zambujeiro e, contra a natureza,
enxertado na boa oliveira, quanto mais esses, que são naturais,
serão enxertados na sua própria oliveira! Porque não quero,
irmãos, que ignoreis este segredo (para que não presumais de vós
mesmos): que o endurecimento veio em parte sobre Israel, até que a
plenitude dos gentios haja entrado. E assim todo o Israel
será salvo, como está escrito: De Sião virá o Libertador, e desviará
de Jacó as impiedades. E esta será a minha aliança com eles,
quando eu tirar os seus pecados. Assim que, quanto ao
evangelho, são inimigos por causa de vós; mas, quanto à eleição,
amados por causa dos pais. Porque os dons e a vocação de Deus
são sem arrependimento. Porque assim como vós também
antigamente fostes desobedientes a Deus, mas agora alcançastes
misericórdia pela desobediência deles, assim também estes
agora foram desobedientes, para também alcançarem misericórdia pela
misericórdia a vós demonstrada. Porque Deus encerrou a todos
debaixo da desobediência, para com todos usar de misericórdia.

Ó profundidade das riquezas, tanto da sabedoria, como da ciência
de Deus! Quão insondáveis são os seus juízos, e quão inescrutáveis
os seus caminhos! Porque, quem compreendeu a mente do Senhor?
ou quem foi seu conselheiro? Ou quem lhe deu primeiro a ele,
para que lhe seja recompensado? Porque dele e por ele, e para
ele, são todas as coisas; glória, pois, a ele eternamente. Amém.

\medskip

\lettrine{12} Rogo-vos, pois, irmãos, pela compaixão de Deus,
que apresenteis os vossos corpos em sacrifício vivo, santo e
agradável a Deus, que é o vosso culto racional. E não sede
conformados com este mundo, mas sede transformados pela renovação do
vosso entendimento, para que experimenteis qual seja a boa,
agradável, e perfeita vontade de Deus. Porque pela graça que me
é dada, digo a cada um dentre vós que não pense de si mesmo além do
que convém; antes, pense com moderação, conforme a medida da fé que
Deus repartiu a cada um. Porque assim como em um corpo temos
muitos membros, e nem todos os membros têm a mesma operação,
assim nós, que somos muitos, somos um só corpo em Cristo, mas
individualmente somos membros uns dos outros. De modo que, tendo
diferentes dons, segundo a graça que nos é dada, se é profecia, seja
ela segundo a medida da fé; se é ministério, seja em ministrar;
se é ensinar, haja dedicação ao ensino; ou o que exorta, use
esse dom em exortar; o que reparte, faça-o com liberalidade; o que
preside, com cuidado; o que exercita misericórdia, com alegria.
O amor seja não fingido. Aborrecei o mal e apegai-vos ao bem.
Amai-vos cordialmente uns aos outros com amor fraternal,
preferindo-vos em honra uns aos outros. Não sejais vagarosos
no cuidado; sede fervorosos no espírito, servindo ao Senhor;
alegrai-vos na esperança, sede pacientes na tribulação,
perseverai na oração; comunicai com os santos nas suas
necessidades, segui a hospitalidade; abençoai aos que vos
perseguem, abençoai, e não amaldiçoeis. Alegrai-vos com os
que se alegram; e chorai com os que choram; sede unânimes
entre vós; não ambicioneis coisas altas, mas acomodai-vos às
humildes; não sejais sábios em vós mesmos; a ninguém torneis
mal por mal; procurai as coisas honestas, perante todos os homens.
Se for possível, quanto estiver em vós, tende paz com todos
os homens. Não vos vingueis a vós mesmos, amados, mas dai
lugar à ira, porque está escrito: Minha é a vingança; eu
recompensarei, diz o Senhor. Portanto, se o teu inimigo tiver
fome, dá-lhe de comer; se tiver sede, dá-lhe de beber; porque,
fazendo isto, amontoarás brasas de fogo sobre a sua cabeça.
Não te deixes vencer do mal, mas vence o mal com o bem.

\medskip

\lettrine{13} Toda a alma esteja sujeita às potestades
superiores; porque não há potestade que não venha de Deus; e as
potestades que há foram ordenadas por Deus. Por isso quem
resiste à potestade resiste à ordenação de Deus; e os que resistem
trarão sobre si mesmos a condenação. Porque os magistrados não
são terror para as boas obras, mas para as más. Queres tu, pois, não
temer a potestade? Faze o bem, e terás louvor dela. Porque ela é
ministro de Deus para teu bem. Mas, se fizeres o mal, teme, pois não
traz debalde a espada; porque é ministro de Deus, e vingador para
castigar o que faz o mal. Portanto é necessário que lhe estejais
sujeitos, não somente pelo castigo, mas também pela consciência.
Por esta razão também pagais tributos, porque são ministros de
Deus, atendendo sempre a isto mesmo.

Portanto, dai a cada um o que deveis: a quem tributo, tributo; a
quem imposto, imposto; a quem temor, temor; a quem honra, honra.
A ninguém devais coisa alguma, a não ser o amor com que vos
ameis uns aos outros; porque quem ama aos outros cumpriu a lei.
Com efeito: Não adulterarás, não matarás, não furtarás, não
darás falso testemunho, não cobiçarás; e se há algum outro
mandamento, tudo nesta palavra se resume: Amarás ao teu próximo como
a ti mesmo. O amor não faz mal ao próximo. De sorte que o
cumprimento da lei é o amor.

E isto digo, conhecendo o tempo, que já é hora de despertarmos do
sono; porque a nossa salvação está agora mais perto de nós do que
quando aceitamos a fé. A noite é passada, e o dia é chegado.
Rejeitemos, pois, as obras das trevas, e vistamo-nos das armas da
luz. Andemos honestamente, como de dia; não em glutonarias,
nem em bebedeiras, nem em desonestidades, nem em
dissoluções\footnote{Dissolução: perversão de costumes; devassidão;
libertinagem.}, nem em contendas e inveja. Mas revesti-vos do
Senhor Jesus Cristo, e não tenhais cuidado da carne em suas
concupiscências.

\medskip

\lettrine{14} Ora, quanto ao que está enfermo na fé,
recebei-o, não em contendas sobre dúvidas. Porque um crê que de
tudo se pode comer, e outro, que é fraco, come legumes. O que
come não despreze o que não come; e o que não come, não julgue o que
come; porque Deus o recebeu por seu. Quem és tu, que julgas o
servo alheio? Para seu próprio SENHOR ele está em pé ou cai. Mas
estará firme, porque poderoso é Deus para o firmar. Um faz
diferença entre dia e dia, mas outro julga iguais todos os dias.
Cada um esteja inteiramente seguro em sua própria mente. Aquele
que faz caso do dia, para o Senhor o faz e o que não faz caso do dia
para o Senhor o não faz. O que come, para o Senhor come, porque dá
graças a Deus; e o que não come, para o SENHOR não come, e dá graças
a Deus. Porque nenhum de nós vive para si, e nenhum morre para
si. Porque, se vivemos, para o Senhor vivemos; se morremos, para
o Senhor morremos. De sorte que, ou vivamos ou morramos, somos do
Senhor. Porque foi para isto que morreu Cristo, e ressurgiu, e
tornou a viver, para ser Senhor, tanto dos mortos, como dos vivos.
Mas tu, por que julgas teu irmão? Ou tu, também, por que
desprezas teu irmão? Pois todos havemos de comparecer ante o
tribunal de Cristo. Porque está escrito: Como eu vivo, diz o
Senhor, que todo o joelho se dobrará a mim, e toda a língua
confessará a Deus. De maneira que cada um de nós dará conta
de si mesmo a Deus. Assim que não nos julguemos mais uns aos
outros; antes seja o vosso propósito não pôr tropeço ou escândalo ao
irmão. Eu sei, e estou certo no Senhor Jesus, que nenhuma
coisa é de si mesma imunda, a não ser para aquele que a tem por
imunda; para esse é imunda. Mas, se por causa da comida se
contrista teu irmão, já não andas conforme o amor. Não destruas por
causa da tua comida aquele por quem Cristo morreu. Não seja,
pois, blasfemado o vosso bem; porque o reino de Deus não é
comida nem bebida, mas justiça, e paz, e alegria no Espírito Santo.
Porque quem nisto serve a Cristo agradável é a Deus e aceito
aos homens. Sigamos, pois, as coisas que servem para a paz e
para a edificação de uns para com os outros. Não destruas por
causa da comida a obra de Deus. É verdade que tudo é limpo, mas mal
vai para o homem que come com escândalo. Bom é não comer
carne, nem beber vinho, nem fazer outras coisas em que teu irmão
tropece, ou se escandalize, ou se enfraqueça. Tens tu fé?
Tem-na em ti mesmo diante de Deus. Bem-aventurado aquele que não se
condena a si mesmo naquilo que aprova. Mas aquele que tem
dúvidas, se come está condenado, porque não come por fé; e tudo o
que não é de fé é pecado.

\medskip

\lettrine{15} Mas nós, que somos fortes, devemos suportar as
fraquezas dos fracos, e não agradar a nós mesmos. Portanto cada
um de nós agrade ao seu próximo no que é bom para edificação.
Porque também Cristo não agradou a si mesmo, mas, como está
escrito: Sobre mim caíram as injúrias dos que te injuriavam.
Porque tudo o que dantes foi escrito, para nosso ensino foi
escrito, para que pela paciência e consolação das Escrituras
tenhamos esperança.

Ora, o Deus de paciência e consolação vos conceda o mesmo
sentimento uns para com os outros, segundo Cristo Jesus. Para
que concordes, a uma boca, glorifiqueis ao Deus e Pai de nosso
Senhor Jesus Cristo.

Portanto recebei-vos uns aos outros, como também Cristo nos
recebeu para glória de Deus. Digo, pois, que Jesus Cristo foi
ministro da circuncisão, por causa da verdade de Deus, para que
confirmasse as promessas feitas aos pais; e para que os gentios
glorifiquem a Deus pela sua misericórdia, como está escrito:
Portanto eu te louvarei entre os gentios, e cantarei ao teu nome.
E outra vez diz: Alegrai-vos, gentios, com o seu povo.
E outra vez: Louvai ao Senhor, todos os gentios, e celebrai-o
todos os povos. Outra vez diz Isaías: Uma raiz em Jessé
haverá, e naquele que se levantar para reger os gentios, os gentios
esperarão.

Ora o Deus de esperança vos encha de todo o gozo e paz em crença,
para que abundeis em esperança pela virtude do Espírito Santo.

Eu próprio, meus irmãos, certo estou, a respeito de vós, que vós
mesmos estais cheios de bondade, cheios de todo o conhecimento,
podendo admoestar-vos uns aos outros. Mas, irmãos, em parte
vos escrevi mais ousadamente, como para vos trazer outra vez isto à
memória, pela graça que por Deus me foi dada; que seja
ministro de Jesus Cristo para os gentios, ministrando o evangelho de
Deus, para que seja agradável a oferta dos gentios, santificada pelo
Espírito Santo.

De sorte que tenho glória em Jesus Cristo nas coisas que
pertencem a Deus. Porque não ousarei dizer coisa alguma, que
Cristo por mim não tenha feito, para fazer obedientes os gentios,
por palavra e por obras; pelo poder dos sinais e prodígios,
na virtude do Espírito de Deus; de maneira que desde Jerusalém, e
arredores, até ao Ilírico, tenho pregado o evangelho de Jesus
Cristo. E desta maneira me esforcei por anunciar o evangelho,
não onde Cristo foi nomeado, para não edificar sobre fundamento
alheio; antes, como está escrito: Aqueles a quem não foi
anunciado, o verão, e os que não ouviram o entenderão.

Por isso também muitas vezes tenho sido impedido de ir ter
convosco. Mas agora, que não tenho mais demora nestes sítios,
e tendo já há muitos anos grande desejo de ir ter convosco,
quando partir para Espanha irei ter convosco; pois espero que
de passagem vos verei, e que para lá seja encaminhado por vós,
depois de ter gozado um pouco da vossa companhia. Mas agora
vou a Jerusalém para ministrar aos santos. Porque pareceu bem
à Macedônia e à Acaia fazerem uma coleta para os pobres dentre os
santos que estão em Jerusalém. Isto lhes pareceu bem, como
devedores que são para com eles. Porque, se os gentios foram
participantes dos seus bens espirituais, devem também ministrar-lhes
os temporais. Assim que, concluído isto, e havendo-lhes
consignado este fruto, de lá, passando por vós, irei à Espanha.
E bem sei que, indo ter convosco, chegarei com a plenitude da
bênção do evangelho de Cristo.

E rogo-vos, irmãos, por nosso Senhor Jesus Cristo e pelo amor do
Espírito, que combatais comigo nas vossas orações por mim a Deus;
para que seja livre dos rebeldes que estão na Judéia, e que
esta minha administração, que em Jerusalém faço, seja bem aceita
pelos santos; a fim de que, pela vontade de Deus, chegue a
vós com alegria, e possa recrear-me convosco. E o Deus de paz
seja com todos vós. Amém.

\medskip

\lettrine{16} Recomendo-vos, pois, Febe, nossa irmã, a qual
serve na igreja que está em Cencréia, para que a recebais no
Senhor, como convém aos santos, e a ajudeis em qualquer coisa que de
vós necessitar; porque tem hospedado a muitos, como também a mim
mesmo. Saudai a Priscila e a Áqüila, meus cooperadores em Cristo
Jesus, os quais pela minha vida expuseram as suas cabeças; o que
não só eu lhes agradeço, mas também todas as igrejas dos gentios.
Saudai também a igreja que está em sua casa. Saudai a Epêneto,
meu amado, que é as primícias da Acaia em Cristo. Saudai a
Maria, que trabalhou muito por nós. Saudai a Andrônico e a
Júnias, meus parentes e meus companheiros na prisão, os quais se
distinguiram entre os apóstolos e que foram antes de mim em Cristo.
Saudai a Amplias, meu amado no Senhor. Saudai a Urbano,
nosso cooperador em Cristo, e a Estáquis, meu amado. Saudai a
Apeles, aprovado em Cristo. Saudai aos da família de Aristóbulo.
Saudai a Herodião, meu parente. Saudai aos da família de
Narciso, os que estão no SENHOR. Saudai a Trifena e a
Trifosa, as quais trabalham no Senhor. Saudai à amada Pérside, a
qual muito trabalhou no SENHOR. Saudai a Rufo, eleito no
Senhor, e a sua mãe e minha. Saudai a Asíncrito, a Flegonte,
a Hermes, a Pátrobas, a Hermas, e aos irmãos que estão com eles.
Saudai a Filólogo e a Júlia, a Nereu e a sua irmã, e a
Olimpas, e a todos os santos que com eles estão. Saudai-vos
uns aos outros com santo ósculo. As igrejas de Cristo vos saúdam.

E rogo-vos, irmãos, que noteis os que promovem dissensões e
escândalos contra a doutrina que aprendestes; desviai-vos deles.
Porque os tais não servem a nosso Senhor Jesus Cristo, mas ao
seu ventre; e com suaves palavras e lisonjas enganam os corações dos
simples. Quanto à vossa obediência, é ela conhecida de todos.
Comprazo-me, pois, em vós; e quero que sejais sábios no bem, mas
simples no mal. E o Deus de paz esmagará em breve Satanás
debaixo dos vossos pés. A graça de nosso Senhor Jesus Cristo seja
convosco. Amém.

Saúdam-vos Timóteo, meu cooperador, e Lúcio, Jasom e Sosípatro,
meus parentes. Eu, Tércio, que esta carta escrevi, vos saúdo
no Senhor. Saúda-vos Gaio, meu hospedeiro, e de toda a
igreja. Saúda-vos Erasto, procurador da cidade, e também o irmão
Quarto. A graça de nosso Senhor Jesus Cristo seja com todos
vós. Amém.

Ora, àquele que é poderoso para vos confirmar segundo o meu
evangelho e a pregação de Jesus Cristo, conforme a revelação do
mistério que desde tempos eternos esteve oculto, mas que se
manifestou agora, e se notificou pelas Escrituras dos profetas,
segundo o mandamento do Deus eterno, a todas as nações para
obediência da fé; ao único Deus, sábio, seja dada glória por
Jesus Cristo para todo o sempre. Amém.

