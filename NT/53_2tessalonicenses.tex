\thispagestyle{empty}
\chapter*{Segunda Epístola de Paulo aos Tessalonicenses}

\lettrine{1} Paulo, e Silvano, e Timóteo, à igreja dos
tessalonicenses, em Deus nosso Pai, e no Senhor Jesus Cristo:
Graça e paz a vós da parte de Deus nosso Pai, e da do Senhor
Jesus Cristo.

Sempre devemos, irmãos, dar graças a Deus por vós, como é justo,
porque a vossa fé cresce muitíssimo e o amor de cada um de vós
aumenta de uns para com os outros, de maneira que nós mesmos nos
gloriamos de vós nas igrejas de Deus por causa da vossa paciência e
fé, e em todas as vossas perseguições e aflições que suportais;
prova clara do justo juízo de Deus, para que sejais havidos por
dignos do reino de Deus, pelo qual também padeceis; se de fato é
justo diante de Deus que dê em paga tribulação aos que vos
atribulam, e a vós, que sois atribulados, descanso conosco,
quando se manifestar o Senhor Jesus desde o céu com os anjos do seu
poder, como labareda de fogo, tomando vingança dos que não
conhecem a Deus e dos que não obedecem ao evangelho de nosso Senhor
Jesus Cristo; os quais, por castigo, padecerão eterna perdição,
ante a face do Senhor e a glória do seu poder, quando vier
para ser glorificado nos seus santos, e para se fazer admirável
naquele dia em todos os que crêem (porquanto o nosso testemunho foi
crido entre vós).

Por isso também rogamos sempre por vós, para que o nosso Deus vos
faça dignos da sua vocação, e cumpra todo o desejo da sua bondade, e
a obra da fé com poder; para que o nome de nosso Senhor Jesus
Cristo seja em vós glorificado, e vós nele, segundo a graça de nosso
Deus e do Senhor Jesus Cristo.

\medskip

\lettrine{2} Ora, irmãos, rogamo-vos, pela vinda de nosso
Senhor Jesus Cristo, e pela nossa reunião com ele, que não vos
movais facilmente do vosso entendimento, nem vos perturbeis, quer
por espírito, quer por palavra, quer por epístola, como de nós, como
se o dia de Cristo estivesse já perto.

Ninguém de maneira alguma vos engane; porque não será assim sem
que antes venha a apostasia, e se manifeste o homem do pecado, o
filho da perdição, o qual se opõe, e se levanta contra tudo o
que se chama Deus, ou se adora; de sorte que se assentará, como
Deus, no templo de Deus, querendo parecer Deus. Não vos lembrais
de que estas coisas vos dizia quando ainda estava convosco? E
agora vós sabeis o que o detém, para que a seu próprio tempo seja
manifestado. Porque já o mistério da injustiça opera; somente há
um que agora resiste até que do meio seja tirado; e então será
revelado o iníquo, a quem o Senhor desfará pelo assopro da sua boca,
e aniquilará pelo esplendor da sua vinda; a esse cuja vinda é
segundo a eficácia de Satanás, com todo o poder, e sinais e
prodígios de mentira, e com todo o engano da injustiça para
os que perecem, porque não receberam o amor da verdade para se
salvarem. E por isso Deus lhes enviará a operação do erro,
para que creiam a mentira; para que sejam julgados todos os
que não creram a verdade, antes tiveram prazer na iniqüidade.

Mas devemos sempre dar graças a Deus por vós, irmãos amados do
Senhor, por vos ter Deus elegido desde o princípio para a salvação,
em santificação do Espírito, e fé da verdade; para o que pelo
nosso evangelho vos chamou, para alcançardes a glória de nosso
Senhor Jesus Cristo. Então, irmãos, estai firmes e retende as
tradições que vos foram ensinadas, seja por palavra, seja por
epístola nossa.

E o próprio nosso Senhor Jesus Cristo e nosso Deus e Pai, que nos
amou, e em graça nos deu uma eterna consolação e boa esperança,
console os vossos corações, e vos confirme em toda a boa
palavra e obra.

\medskip

\lettrine{3} No demais, irmãos, rogai por nós, para que a
palavra do Senhor tenha livre curso e seja glorificada, como também
o é entre vós; e para que sejamos livres de homens dissolutos e
maus; porque a fé não é de todos. Mas fiel é o Senhor, que vos
confirmará, e guardará do maligno. E confiamos quanto a vós no
Senhor, que não só fazeis como fareis o que vos mandamos. Ora o
Senhor encaminhe os vossos corações no amor de Deus, e na paciência
de Cristo.

Mandamo-vos, porém, irmãos, em nome de nosso Senhor Jesus Cristo,
que vos aparteis de todo o irmão que anda desordenadamente, e não
segundo a tradição que de nós recebeu. Porque vós mesmos sabeis
como convém imitar-nos, pois que não nos houvemos desordenadamente
entre vós, nem de graça comemos o pão de homem algum, mas com
trabalho e fadiga, trabalhando noite e dia, para não sermos pesados
a nenhum de vós. Não porque não tivéssemos autoridade, mas para
vos dar em nós mesmos exemplo, para nos imitardes. Porque,
quando ainda estávamos convosco, vos mandamos isto, que, se alguém
não quiser trabalhar, não coma também. Porquanto ouvimos que
alguns entre vós andam desordenadamente, não trabalhando, antes
fazendo coisas vãs. A esses tais, porém, mandamos, e
exortamos por nosso Senhor Jesus Cristo, que, trabalhando com
sossego, comam o seu próprio pão. E vós, irmãos, não vos
canseis de fazer o bem. Mas, se alguém não obedecer à nossa
palavra por esta carta, notai o tal, e não vos mistureis com ele,
para que se envergonhe. Todavia não o tenhais como inimigo,
mas admoestai-o como irmão.

Ora, o mesmo Senhor da paz vos dê sempre paz de toda a maneira. O
Senhor seja com todos vós. Saudação da minha própria mão, de
mim, Paulo, que é o sinal em todas as epístolas; assim escrevo.
A graça de nosso Senhor Jesus Cristo seja com todos vós.
Amém.

