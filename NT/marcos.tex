\chapter*{O Evangelho de Marcos}

\lettrine{1} Princípio do Evangelho de Jesus Cristo, Filho de
Deus; como está escrito nos profetas: Eis que eu envio o meu
anjo ante a tua face, o qual preparará o teu caminho diante de ti.
Voz do que clama no deserto: Preparai o caminho do Senhor,
endireitai as suas veredas. Apareceu João batizando no deserto,
e pregando o batismo de arrependimento, para remissão dos pecados.
E toda a província da Judéia e os de Jerusalém iam ter com ele;
e todos eram batizados por ele no rio Jordão, confessando os seus
pecados. E João andava vestido de pêlos de camelo, e com um
cinto de couro em redor de seus lombos, e comia gafanhotos e mel
silvestre. E pregava, dizendo: Após mim vem aquele que é mais
forte do que eu, do qual não sou digno de, abaixando-me, desatar a
correia das suas alparcas. Eu, em verdade, tenho-vos batizado
com água; ele, porém, vos batizará com o Espírito Santo.

E aconteceu naqueles dias que Jesus, tendo ido de Nazaré da
Galiléia, foi batizado por João, no Jordão. E, logo que saiu
da água, viu os céus abertos, e o Espírito, que como pomba descia
sobre ele. E ouviu-se uma voz dos céus, que dizia: \textcolor{red}{Tu
és o meu Filho amado em quem me comprazo}. E logo o Espírito
o impeliu para o deserto. E ali esteve no deserto quarenta
dias, tentado por Satanás. E vivia entre as feras, e os anjos o
serviam.

E, depois que João foi entregue à prisão, veio Jesus para a
Galiléia, pregando o evangelho do reino de Deus, e dizendo:
\textcolor{red}{O tempo está cumprido, e o reino de Deus está próximo.
Arrependei-vos, e crede no evangelho}. E, andando junto do
mar da Galiléia, viu Simão, e André, seu irmão, que lançavam a rede
ao mar, pois eram pescadores. E Jesus lhes disse: Vinde após
mim, e eu farei que sejais pescadores de homens. E, deixando
logo as suas redes, o seguiram. E, passando dali um pouco
mais adiante, viu Tiago, filho de Zebedeu, e João, seu irmão, que
estavam no barco consertando as redes, e logo os chamou. E
eles, deixando o seu pai Zebedeu no barco com os
jornaleiros\footnote{Diz-se de ou trabalhador a quem se paga jornal
(remuneração salarial feita por dia de trabalho). Ed. Contemp.:
empregados. KJ: And straightway he called them: and they left their
father Zebedee in the ship with the hired servants, and went after
him.}, foram após ele. Entraram em Cafarnaum e, logo no
sábado, indo ele à sinagoga, ali ensinava. E maravilharam-se
da sua doutrina, porque os ensinava como tendo autoridade, e não
como os escribas.

E estava na sinagoga deles um homem com um espírito imundo, o
qual exclamou, dizendo: Ah! que temos contigo, Jesus
Nazareno? Vieste destruir-nos? Bem sei quem és: o Santo de Deus.
E repreendeu-o Jesus, dizendo: Cala-te, e sai dele.
Então o espírito imundo, convulsionando-o, e clamando com
grande voz, saiu dele. E todos se admiraram, a ponto de
perguntarem entre si, dizendo: Que é isto? Que nova doutrina é esta?
Pois com autoridade ordena aos espíritos imundos, e eles lhe
obedecem! E logo correu a sua fama por toda a província da
Galiléia.

E logo, saindo da sinagoga, foram à casa de Simão e de André com
Tiago e João. E a sogra de Simão estava deitada com febre; e
logo lhe falaram dela. Então, chegando-se a ela, tomou-a pela
mão, e levantou-a; e imediatamente a febre a deixou, e servia-os.
E, tendo chegado a tarde, quando já se estava pondo o sol,
trouxeram-lhe todos os que se achavam enfermos, e os endemoninhados.
E toda a cidade se ajuntou à porta. E curou muitos que
se achavam enfermos de diversas enfermidades, e expulsou muitos
demônios, porém não deixava falar os demônios, porque o conheciam.
E, levantando-se de manhã, muito cedo, fazendo ainda escuro,
saiu, e foi para um lugar deserto, e ali orava. E seguiram-no
Simão e os que com ele estavam. E, achando-o, lhe disseram:
Todos te buscam. E ele lhes disse: Vamos às aldeias vizinhas,
para que eu ali também pregue; porque para isso vim. E
pregava nas sinagogas deles, por toda a Galiléia, e expulsava os
demônios.

E aproximou-se dele um leproso que, rogando-lhe, e pondo-se de
joelhos diante dele, lhe dizia: Se queres, bem podes limpar-me.
E Jesus, movido de grande compaixão, estendeu a mão, e
tocou-o, e disse-lhe: Quero, sê limpo. E, tendo ele dito
isto, logo a lepra desapareceu, e ficou limpo. E,
advertindo-o severamente, logo o despediu. E disse-lhe: Olha,
não digas nada a ninguém; porém vai, mostra-te ao sacerdote, e
oferece pela tua purificação o que Moisés determinou, para lhes
servir de testemunho. Mas, tendo ele saído, começou a
apregoar muitas coisas, e a divulgar o que acontecera; de sorte que
Jesus já não podia entrar publicamente na cidade, mas conservava-se
fora em lugares desertos; e de todas as partes iam ter com ele.

\medskip

\lettrine{2} E alguns dias depois entrou outra vez em
Cafarnaum, e soube-se que estava em casa. E logo se ajuntaram
tantos, que nem ainda nos lugares junto à porta cabiam; e
anunciava-lhes a palavra. E vieram ter com ele conduzindo um
paralítico, trazido por quatro. E, não podendo aproximar-se
dele, por causa da multidão, descobriram o telhado onde estava, e,
fazendo um buraco, baixaram o leito em que jazia o paralítico. E
Jesus, vendo a fé deles, disse ao paralítico: Filho, perdoados estão
os teus pecados. E estavam ali assentados alguns dos escribas,
que arrazoavam em seus corações, dizendo: Por que diz este assim
blasfêmias? \textcolor{red}{Quem pode perdoar pecados, senão Deus}? E
Jesus, conhecendo logo em seu espírito que assim arrazoavam entre
si, lhes disse: Por que arrazoais sobre estas coisas em vossos
corações? Qual é mais fácil? dizer ao paralítico: Estão
perdoados os teus pecados; ou dizer-lhe: Levanta-te, e toma o teu
leito, e anda? Ora, para que saibais que \textcolor{red}{o Filho do
homem tem na terra poder para perdoar pecados} (disse ao
paralítico), a ti te digo: Levanta-te, toma o teu leito, e
vai para tua casa. E levantou-se e, tomando logo o leito,
saiu em presença de todos, de sorte que todos se admiraram e
glorificaram a Deus, dizendo: Nunca tal vimos.

E tornou a sair para o mar, e toda a multidão ia ter com ele, e
ele os ensinava. E, passando, viu Levi, filho de Alfeu,
sentado na recebedoria, e disse-lhe: Segue-me. E, levantando-se, o
seguiu. E aconteceu que, estando sentado à mesa em casa
deste, também estavam sentados à mesa com Jesus e seus discípulos
muitos publicanos e pecadores; porque eram muitos, e o tinham
seguido. E os escribas e fariseus, vendo-o comer com os
publicanos e pecadores, disseram aos seus discípulos: Por que come e
bebe ele com os publicanos e pecadores? E Jesus, tendo ouvido
isto, disse-lhes: Os sãos não necessitam de médico, mas, sim, os que
estão doentes; eu não vim chamar os justos, mas, sim, os pecadores
ao arrependimento.

Ora, os discípulos de João e os fariseus jejuavam; e foram e
disseram-lhe: Por que jejuam os discípulos de João e os dos
fariseus, e não jejuam os teus discípulos? E Jesus
disse-lhes: Podem porventura os filhos das bodas jejuar enquanto
está com eles o esposo? Enquanto têm consigo o esposo, não podem
jejuar; mas dias virão em que lhes será tirado o esposo, e
então jejuarão naqueles dias. Ninguém deita remendo de pano
novo em roupa velha; doutra sorte o mesmo remendo novo rompe o
velho, e a rotura fica maior. E ninguém deita vinho novo em
odres velhos; doutra sorte, o vinho novo rompe os odres e entorna-se
o vinho, e os odres estragam-se; o vinho novo deve ser deitado em
odres novos. E aconteceu que, passando ele num sábado pelas
searas, os seus discípulos, caminhando, começaram a colher espigas.
E os fariseus lhe disseram: Vês? Por que fazem no sábado o
que não é lícito? Mas ele disse-lhes: Nunca lestes o que fez
Davi, quando estava em necessidade e teve fome, ele e os que com ele
estavam? Como entrou na casa de Deus, no tempo de Abiatar,
sumo sacerdote, e comeu os pães da proposição, dos quais não era
lícito comer senão aos sacerdotes, dando também aos que com ele
estavam? E disse-lhes: O sábado foi feito por causa do homem,
e não o homem por causa do sábado. Assim o Filho do homem até
do sábado é Senhor.

\medskip

\lettrine{3} E outra vez entrou na sinagoga, e estava ali um
homem que tinha uma das mãos mirrada. E estavam observando-o se
curaria no sábado, para o acusarem. E disse ao homem que tinha a
mão mirrada: Levanta-te e vem para o meio. E perguntou-lhes: É
lícito no sábado fazer bem, ou fazer mal? salvar a vida, ou matar? E
eles calaram-se. E, olhando para eles em redor com indignação,
condoendo-se da dureza do seu coração, disse ao homem: Estende a tua
mão. E ele a estendeu, e foi-lhe restituída a sua mão, sã como a
outra. E, tendo saído os fariseus, tomaram logo conselho com os
herodianos contra ele, procurando ver como o matariam. E
retirou-se Jesus com os seus discípulos para o mar, e seguia-o uma
grande multidão da Galiléia e da Judéia, e de Jerusalém, e da
Iduméia, e de além do Jordão, e de perto de Tiro e de Sidom; uma
grande multidão que, ouvindo quão grandes coisas fazia, vinha ter
com ele. E ele disse aos seus discípulos que lhe tivessem sempre
pronto um barquinho junto dele, por causa da multidão, para que o
não oprimisse, porque tinha curado a muitos, de tal maneira
que todos quantos tinham algum mal se arrojavam sobre ele, para lhe
tocarem. E os espíritos imundos vendo-o, prostravam-se diante
dele, e clamavam, dizendo: \textcolor{red}{Tu és o Filho de Deus}. E
ele os ameaçava muito, para que não o manifestassem.

E subiu ao monte, e chamou para si os que ele quis; e vieram a
ele. E nomeou doze para que estivessem com ele e os mandasse
a pregar, e para que tivessem o poder de curar as
enfermidades e expulsar os demônios: A Simão, a quem pôs o
nome de Pedro, e a Tiago, filho de Zebedeu, e a João, irmão
de Tiago, aos quais pôs o nome de Boanerges, que significa: Filhos
do trovão; e a André, e a Filipe, e a Bartolomeu, e a Mateus,
e a Tomé, e a Tiago, filho de Alfeu, e a Tadeu, e a Simão o Zelote,
e a Judas Iscariotes, o que o entregou. E foram para
uma casa. E afluiu outra vez a multidão, de tal maneira que nem
sequer podiam comer pão. E, quando os seus ouviram isto,
saíram para o prender; porque diziam: Está fora de si.

E os escribas, que tinham descido de Jerusalém, diziam: Tem
Belzebu, e pelo príncipe dos demônios expulsa os demônios. E,
chamando-os a si, disse-lhes por parábolas: Como pode Satanás
expulsar Satanás? E, se um reino se dividir contra si mesmo,
tal reino não pode subsistir; e, se uma casa se dividir
contra si mesma, tal casa não pode subsistir. E, se Satanás
se levantar contra si mesmo, e for dividido, não pode subsistir;
antes tem fim. Ninguém pode roubar os bens do valente,
entrando-lhe em sua casa, se primeiro não maniatar o valente; e
então roubará a sua casa. Na verdade vos digo que todos os
pecados serão perdoados aos filhos dos homens, e toda a sorte de
blasfêmias, com que blasfemarem; \textcolor{red}{qualquer}, porém,
\textcolor{red}{que blasfemar contra o Espírito Santo, nunca obterá perdão,
mas será réu do eterno juízo}

 (Porque diziam: Tem espírito imundo).

Chegaram, então, seus irmãos e sua mãe; e, estando fora,
mandaram-no chamar. E a multidão estava assentada ao redor
dele, e disseram-lhe: Eis que tua mãe e teus irmãos te procuram, e
estão lá fora. E ele lhes respondeu, dizendo: Quem é minha
mãe e meus irmãos? E, olhando em redor para os que estavam
assentados junto dele, disse: Eis aqui minha mãe e meus irmãos.
Porquanto, qualquer que fizer a vontade de Deus, esse é meu
irmão, e minha irmã, e minha mãe.

\medskip

\lettrine{4} E outra vez começou a ensinar junto do mar, e
ajuntou-se a ele grande multidão, de sorte que ele entrou e
assentou-se num barco, sobre o mar; e toda a multidão estava em
terra junto do mar. E ensinava-lhes muitas coisas por parábolas,
e lhes dizia na sua doutrina: Ouvi: Eis que saiu o semeador a
semear. E aconteceu que semeando ele, uma parte da semente caiu
junto do caminho, e vieram as aves do céu, e a comeram; e outra
caiu sobre pedregais, onde não havia muita terra, e nasceu logo,
porque não tinha terra profunda; mas, saindo o sol, queimou-se;
e, porque não tinha raiz, secou-se. E outra caiu entre espinhos
e, crescendo os espinhos, a sufocaram e não deu fruto. E outra
caiu em boa terra e deu fruto, que vingou e cresceu; e um produziu
trinta, outro sessenta, e outro cem. E disse-lhes: Quem tem
ouvidos para ouvir, ouça. E, quando se achou só, os que
estavam junto dele com os doze interrogaram-no acerca da parábola.
E ele disse-lhes: A vós vos é dado saber os mistérios do
reino de Deus, mas aos que estão de fora todas estas coisas se dizem
por parábolas, para que, vendo, vejam, e não percebam; e,
ouvindo, ouçam, e não entendam; para que não se convertam, e lhes
sejam perdoados os pecados. E disse-lhes: Não percebeis esta
parábola? Como, pois, entendereis todas as parábolas? O que
semeia, semeia a palavra; e, os que estão junto do caminho
são aqueles em quem a palavra é semeada; mas, tendo-a eles ouvido,
vem logo Satanás e tira a palavra que foi semeada nos seus corações.
E da mesma forma os que recebem a semente sobre pedregais; os
quais, ouvindo a palavra, logo com prazer a recebem; mas não
têm raiz em si mesmos, antes são temporãos; depois, sobrevindo
tribulação ou perseguição, por causa da palavra, logo se
escandalizam. E outros são os que recebem a semente entre
espinhos, os quais ouvem a palavra; mas os cuidados deste
mundo, e os enganos das riquezas e as ambições de outras coisas,
entrando, sufocam a palavra, e fica infrutífera. E os que
recebem a semente em boa terra são os que ouvem a palavra e a
recebem, e dão fruto, um a trinta, outro a sessenta, outro a cem,
por um.

E disse-lhes: Vem porventura a candeia para se meter debaixo do
alqueire, ou debaixo da cama? Não vem antes para se colocar no
velador? Porque nada há encoberto que não haja de ser
manifesto; e nada se faz para ficar oculto, mas para ser descoberto.
Se alguém tem ouvidos para ouvir, ouça. E disse-lhes:
Atendei ao que ides ouvir. Com a medida com que medirdes vos medirão
a vós, e ser-vos-á ainda acrescentada a vós que ouvis. Porque
ao que tem, ser-lhe-á dado; e, ao que não tem, até o que tem lhe
será tirado. E dizia: O reino de Deus é assim como se um
homem lançasse semente à terra. E dormisse, e se levantasse
de noite ou de dia, e a semente brotasse e crescesse, não sabendo
ele como. Porque a terra por si mesma frutifica, primeiro a
erva, depois a espiga, por último o grão cheio na espiga. E,
quando já o fruto se mostra, mete-se-lhe logo a foice, porque está
chegada a ceifa. E dizia: A que assemelharemos o reino de
Deus? Ou com que parábola o representaremos? É como um grão
de mostarda, que, quando se semeia na terra, é a menor de todas as
sementes que há na terra; mas, tendo sido semeado, cresce; e
faz-se a maior de todas as hortaliças, e cria grandes ramos, de tal
maneira que as aves do céu podem aninhar-se debaixo da sua sombra.
E com muitas parábolas tais lhes dirigia a palavra, segundo o
que podiam compreender. E sem parábolas nunca lhes falava;
porém, tudo declarava em particular aos seus discípulos.

E, naquele dia, sendo já tarde, disse-lhes: Passemos para o outro
lado. E eles, deixando a multidão, o levaram consigo, assim
como estava, no barco; e havia também com ele outros barquinhos.
E levantou-se grande temporal de vento, e subiam as ondas por
cima do barco, de maneira que já se enchia. E ele estava na
popa, dormindo sobre uma almofada, e despertaram-no, dizendo-lhe:
Mestre, não se te dá\footnote{``Mestre, não te importa que
pereçamos?'' King James: ``Master, carest thou not that we
perish?''.} que pereçamos? E ele, despertando, repreendeu o
vento, e disse ao mar: Cala-te, aquieta-te. E o vento se aquietou, e
houve grande bonança. E disse-lhes: Por que sois tão tímidos?
Ainda não tendes fé? E sentiram um grande temor, e diziam uns
aos outros: Mas quem é este, que até o vento e o mar lhe obedecem?

\medskip

\lettrine{5} E chegaram ao outro lado do mar, à província dos
gadarenos. E, saindo ele do barco, lhe saiu logo ao seu
encontro, dos sepulcros, um homem com espírito imundo; o qual
tinha a sua morada nos sepulcros, e nem ainda com cadeias o podia
alguém prender; porque, tendo sido muitas vezes preso com
grilhões e cadeias, as cadeias foram por ele feitas em pedaços, e os
grilhões em migalhas, e ninguém o podia amansar. E andava
sempre, de dia e de noite, clamando pelos montes, e pelos sepulcros,
e ferindo-se com pedras. E, quando viu Jesus ao longe, correu e
adorou-o. E, clamando com grande voz, disse: Que tenho eu
contigo, Jesus, Filho do Deus Altíssimo? Conjuro-te por Deus que não
me atormentes. (Porque lhe dizia: Sai deste homem, espírito
imundo.) E perguntou-lhe: Qual é o teu nome? E lhe respondeu,
dizendo: Legião é o meu nome, porque somos muitos. E
rogava-lhe muito que os não enviasse para fora daquela província.
E andava ali pastando no monte uma grande manada de porcos.
E todos aqueles demônios lhe rogaram, dizendo: Manda-nos para
aqueles porcos, para que entremos neles. E Jesus logo lho
permitiu. E, saindo aqueles espíritos imundos, entraram nos porcos;
e a manada se precipitou por um despenhadeiro no mar (eram quase
dois mil), e afogaram-se no mar. E os que apascentavam os
porcos fugiram, e o anunciaram na cidade e nos campos; e saíram
muitos a ver o que era aquilo que tinha acontecido. E foram
ter com Jesus, e viram o endemoninhado, o que tivera a legião,
assentado, vestido e em perfeito juízo, e temeram. E os que
aquilo tinham visto contaram-lhes o que acontecera ao endemoninhado,
e acerca dos porcos. E começaram a rogar-lhe que saísse dos
seus termos. E, entrando ele no barco, rogava-lhe o que fora
endemoninhado que o deixasse estar com ele. Jesus, porém, não
lho permitiu, mas disse-lhe: Vai para tua casa, para os teus, e
anuncia-lhes \textcolor{red}{quão grandes coisas o Senhor te fez, e como
teve misericórdia de ti}. E ele foi, e começou a anunciar em
Decápolis \textcolor{red}{quão grandes coisas Jesus lhe fizera}; e todos se
maravilharam.

E, passando Jesus outra vez num barco para o outro lado,
ajuntou-se a ele uma grande multidão; e ele estava junto do mar.
E eis que chegou um dos principais da sinagoga, por nome
Jairo, e, vendo-o, prostrou-se aos seus pés, e rogava-lhe
muito, dizendo: Minha filha está moribunda; rogo-te que venhas e lhe
imponhas as mãos, para que sare, e viva. E foi com ele, e
seguia-o uma grande multidão, que o apertava. E certa mulher
que, havia doze anos, tinha um fluxo de sangue, e que havia
padecido muito com muitos médicos, e despendido tudo quanto tinha,
nada lhe aproveitando isso, antes indo a pior; ouvindo falar
de Jesus, veio por detrás, entre a multidão, e tocou na sua veste.
Porque dizia: Se tão-somente tocar nas suas vestes, sararei.
E logo se lhe secou a fonte do seu sangue; e sentiu no seu
corpo estar já curada daquele mal. E logo Jesus, conhecendo
que a virtude de si mesmo saíra, voltou-se para a multidão, e disse:
Quem tocou nas minhas vestes? E disseram-lhe os seus
discípulos: Vês que a multidão te aperta, e dizes: Quem me tocou?
E ele olhava em redor, para ver a que isto fizera.
Então a mulher, que sabia o que lhe tinha acontecido, temendo
e tremendo, aproximou-se, e prostrou-se diante dele, e disse-lhe
toda a verdade. E ele lhe disse: Filha, a tua fé te salvou;
vai em paz, e sê curada deste teu mal.

Estando ele ainda falando, chegaram alguns do principal da
sinagoga, a quem disseram: A tua filha está morta; para que enfadas
mais o Mestre? E Jesus, tendo ouvido estas palavras, disse ao
principal da sinagoga: Não temas, crê somente. E não permitiu
que alguém o seguisse, a não ser Pedro, Tiago, e João, irmão de
Tiago. E, tendo chegado à casa do principal da sinagoga, viu
o alvoroço, e os que choravam muito e pranteavam. E,
entrando, disse-lhes: Por que vos alvoroçais e chorais? A menina não
está morta, mas dorme. E riam-se dele; porém ele, tendo-os
feito sair, tomou consigo o pai e a mãe da menina, e os que com ele
estavam, e entrou onde a menina estava deitada. E, tomando a
mão da menina, disse-lhe: Talita cumi; que, traduzido, é: Menina, a
ti te digo, levanta-te. E logo a menina se levantou, e
andava, pois já tinha doze anos; e assombraram-se com grande
espanto. E mandou-lhes expressamente que ninguém o soubesse;
e disse que lhe dessem de comer.

\medskip

\lettrine{6} E, partindo dali, chegou à sua pátria, e os seus
discípulos o seguiram. E, chegando o sábado, começou a ensinar
na sinagoga; e muitos, ouvindo-o, se admiravam, dizendo: De onde lhe
vêm estas coisas? E que sabedoria é esta que lhe foi dada? E como se
fazem tais maravilhas por suas mãos? Não é este o carpinteiro,
filho de Maria, e irmão de Tiago, e de José, e de Judas e de Simão?
E não estão aqui conosco suas irmãs? E escandalizavam-se nele. E
Jesus lhes dizia: Não há profeta sem honra senão na sua pátria,
entre os seus parentes, e na sua casa. E não podia fazer ali
obras maravilhosas; somente curou alguns poucos enfermos,
impondo-lhes as mãos. E estava admirado da incredulidade deles.
E percorreu as aldeias vizinhas, ensinando.

Chamou a si os doze, e começou a enviá-los a dois e dois, e
deu-lhes poder sobre os espíritos imundos; e ordenou-lhes que
nada tomassem para o caminho, senão somente um bordão; nem alforje,
nem pão, nem dinheiro no cinto; mas que calçassem alparcas, e
que não vestissem duas túnicas. E dizia-lhes: Na casa em que
entrardes, ficai nela até partirdes dali. E tantos quantos
vos não receberem, nem vos ouvirem, saindo dali, sacudi o pó que
estiver debaixo dos vossos pés, em testemunho contra eles. Em
verdade vos digo que haverá mais tolerância no dia de juízo para
Sodoma e Gomorra, do que para os daquela cidade. E, saindo
eles, pregavam que se arrependessem. E expulsavam muitos
demônios, e ungiam muitos enfermos com óleo, e os curavam.

E ouviu isto o rei Herodes (porque o nome de Jesus se tornara
notório), e disse: João, o que batizava, ressuscitou dentre os
mortos, e por isso estas maravilhas operam nele. Outros
diziam: É Elias. E diziam outros: É um profeta, ou como um dos
profetas. Herodes, porém, ouvindo isto, disse: Este é João,
que mandei degolar; ressuscitou dentre os mortos. Porquanto o
mesmo Herodes mandara prender a João, e encerrá-lo maniatado no
cárcere, por causa de Herodias, mulher de Filipe, seu irmão,
porquanto tinha casado com ela. Pois João dizia a Herodes:
Não te é lícito possuir a mulher de teu irmão. E Herodias o
espiava, e queria matá-lo, mas não podia. Porque Herodes
temia a João, sabendo que era homem justo e santo; e guardava-o com
segurança, e fazia muitas coisas, atendendo-o, e de boa mente o
ouvia. E, chegando uma ocasião favorável em que Herodes, no
dia dos seus anos, dava uma ceia aos grandes, e tribunos, e
príncipes da Galiléia, entrou a filha da mesma Herodias, e
dançou, e agradou a Herodes e aos que estavam com ele à mesa. Disse
então o rei à menina: Pede-me o que quiseres, e eu to darei.
E jurou-lhe, dizendo: Tudo o que me pedires te darei, até
metade do meu reino. E, saindo ela, perguntou a sua mãe: Que
pedirei? E ela disse: A cabeça de João o Batista. E, entrando
logo, apressadamente, pediu ao rei, dizendo: Quero que imediatamente
me dês num prato a cabeça de João o Batista. E o rei
entristeceu-se muito; todavia, por causa do juramento e dos que
estavam com ele à mesa, não lha quis negar. E, enviando logo
o rei o executor, mandou que lhe trouxessem ali a cabeça de João. E
ele foi, e degolou-o na prisão; e trouxe a cabeça num prato,
e deu-a à menina, e a menina a deu a sua mãe. E os seus
discípulos, tendo ouvido isto, foram, tomaram o seu corpo, e o
puseram num sepulcro.

E os apóstolos ajuntaram-se a Jesus, e contaram-lhe tudo, tanto o
que tinham feito como o que tinham ensinado. E ele
disse-lhes: Vinde vós, aqui à parte, a um lugar deserto, e repousai
um pouco. Porque havia muitos que iam e vinham, e não tinham tempo
para comer. E foram sós num barco para um lugar deserto.
E a multidão viu-os partir, e muitos o conheceram; e correram
para lá, a pé, de todas as cidades, e ali chegaram primeiro do que
eles, e aproximavam-se dele. E Jesus, saindo, viu uma grande
multidão, e teve compaixão deles, porque eram como ovelhas que não
têm pastor; e começou a ensinar-lhes muitas coisas. E, como o
dia fosse já muito adiantado, os seus discípulos se aproximaram
dele, e lhe disseram: O lugar é deserto, e o dia está já muito
adiantado. Despede-os, para que vão aos lugares e aldeias
circunvizinhas, e comprem pão para si; porque não têm que comer.
Ele, porém, respondendo, lhes disse: Dai-lhes vós de comer. E
eles disseram-lhe: Iremos nós, e compraremos duzentos
denários\footnote{SBTB: dinheiros.} de pão para lhes darmos de
comer? E ele disse-lhes: Quantos pães tendes? Ide ver. E,
sabendo-o eles, disseram: Cinco pães e dois peixes. E
ordenou-lhes que fizessem assentar a todos, em
ranchos\footnote{Houaiss: grupo de pessoas reunidas para determinado
fim, esp. em marcha ou jornada. Ex.: r. de peregrinos. Ed. Contemp.
e RA: em grupos. KJ: And he commanded them to make all sit down by
companies upon the green grass.}, sobre a erva verde. E
assentaram-se repartidos de cem em cem, e de cinqüenta em cinqüenta.
E, tomando ele os cinco pães e os dois peixes, levantou os
olhos ao céu, abençoou e partiu os pães, e deu-os aos seus
discípulos para que os pusessem diante deles. E repartiu os dois
peixes por todos. E todos comeram, e ficaram fartos; e
levantaram doze alcofas cheias de pedaços de pão e de peixe.
E os que comeram os pães eram quase cinco mil homens.

E logo obrigou os seus discípulos a subir para o barco, e passar
adiante, para o outro lado, a Betsaida, enquanto ele despedia a
multidão. E, tendo-os despedido, foi ao monte a orar.
E, sobrevindo a tarde, estava o barco no meio do mar e ele,
sozinho, em terra. E vendo que se fatigavam a remar, porque o
vento lhes era contrário, perto da quarta vigília da noite
aproximou-se deles, andando sobre o mar, e queria passar-lhes
adiante. Mas, quando eles o viram andar sobre o mar, cuidaram
que era um fantasma, e deram grandes gritos. Porque todos o
viam, e perturbaram-se; mas logo falou com eles, e disse-lhes: Tende
bom ânimo; sou eu, não temais. E subiu para o barco, para
estar com eles, e o vento se aquietou; e entre si ficaram muito
assombrados e maravilhados; pois não tinham compreendido o
milagre dos pães; antes o seu coração estava endurecido. E,
quando já estavam no outro lado, dirigiram-se à terra de Genesaré, e
ali atracaram. E, saindo eles do barco, logo o conheceram;
e, correndo toda a terra em redor, começaram a trazer em
leitos, aonde quer que sabiam que ele estava, os que se achavam
enfermos. E, onde quer que entrava, ou em cidade, ou aldeias,
ou no campo, apresentavam os enfermos nas praças, e rogavam-lhe que
os deixasse tocar ao menos na orla da sua roupa; e todos os que lhe
tocavam saravam.

\medskip

\lettrine{7} E ajuntaram-se a ele os fariseus, e alguns dos
escribas que tinham vindo de Jerusalém. E, vendo que alguns dos
seus discípulos comiam pão com as mãos impuras, isto é, por lavar,
os repreendiam. Porque os fariseus, e todos os judeus,
conservando a tradição dos antigos, não comem sem lavar as mãos
muitas vezes; e, quando voltam do mercado, se não se lavarem,
não comem. E muitas outras coisas há que receberam para observar,
como lavar os copos, e os jarros, e os vasos de metal e as camas.
Depois perguntaram-lhe os fariseus e os escribas: Por que não
andam os teus discípulos conforme a tradição dos antigos, mas comem
o pão com as mãos por lavar? E ele, respondendo, disse-lhes: Bem
profetizou Isaías acerca de vós, hipócritas, como está escrito: Este
povo honra-me com os lábios, mas o seu coração está longe de mim;
em vão, porém, me honram, ensinando doutrinas que são
mandamentos de homens. Porque, deixando o mandamento de Deus,
retendes a tradição dos homens; como o lavar dos jarros e dos copos;
e fazeis muitas outras coisas semelhantes a estas. E dizia-lhes:
Bem invalidais o mandamento de Deus para guardardes a vossa
tradição. Porque Moisés disse: Honra a teu pai e a tua mãe; e
quem maldisser, ou o pai ou a mãe, certamente morrerá. Vós,
porém, dizeis: Se um homem disser ao pai ou à mãe: Aquilo que
poderias aproveitar de mim é Corbã, isto é, oferta ao Senhor;
nada mais lhe deixais fazer por seu pai ou por sua mãe,
invalidando assim a palavra de Deus pela vossa tradição, que
vós ordenastes. E muitas coisas fazeis semelhantes a estas.
E, chamando outra vez a multidão, disse-lhes: Ouvi-me vós,
todos, e compreendei. Nada há, fora do homem, que, entrando
nele, o possa contaminar; mas o que sai dele isso é que contamina o
homem. Se alguém tem ouvidos para ouvir, ouça. Depois,
quando deixou a multidão, e entrou em casa, os seus discípulos o
interrogavam acerca desta parábola. E ele disse-lhes: Assim
também vós estais sem entendimento? Não compreendeis que tudo o que
de fora entra no homem não o pode contaminar, porque não
entra no seu coração, mas no ventre, e é lançado fora, ficando puras
todas as comidas? E dizia: O que sai do homem isso contamina
o homem. Porque do interior do coração dos homens saem os
maus pensamentos, os adultérios, as prostituições, os homicídios,
os furtos, a avareza, as maldades, o engano, a
dissolução\footnote{Perversão de costumes; devassidão;
libertinagem.}, a inveja, a blasfêmia, a soberba, a loucura.
Todos estes males procedem de dentro e contaminam o homem.

E, levantando-se dali, foi para os termos de Tiro e de Sidom. E,
entrando numa casa, não queria que alguém o soubesse, mas não pôde
esconder-se; porque uma mulher, cuja filha tinha um espírito
imundo, ouvindo falar dele, foi e lançou-se aos seus pés. E
esta mulher era grega, siro-fenícia de nação, e rogava-lhe que
expulsasse de sua filha o demônio. Mas Jesus disse-lhe: Deixa
primeiro saciar os filhos; porque não convém tomar o pão dos filhos
e lançá-lo aos cachorrinhos. Ela, porém, respondeu, e
disse-lhe: Sim, Senhor; mas também os cachorrinhos comem, debaixo da
mesa, as migalhas dos filhos. Então ele disse-lhe: Por essa
palavra, vai; o \textcolor{red}{demônio} já saiu de tua filha. E,
indo ela para sua casa, achou a filha deitada sobre a cama, e que o
demônio já tinha saído.

E ele, tornando a sair dos termos de Tiro e de Sidom, foi até ao
mar da Galiléia, pelos confins de Decápolis. E trouxeram-lhe
um surdo, que falava dificilmente; e rogaram-lhe que pusesse a mão
sobre ele. E, tirando-o à parte, de entre a multidão, pôs-lhe
os dedos nos ouvidos; e, cuspindo, tocou-lhe na língua. E,
levantando os olhos ao céu, suspirou, e disse: Efatá; isto é,
Abre-te. E logo se abriram os seus ouvidos, e a prisão da
língua se desfez, e falava perfeitamente. E ordenou-lhes que
a ninguém o dissessem; mas, quanto mais lhos proibia, tanto mais o
divulgavam. E, admirando-se sobremaneira, diziam: Tudo faz
bem; faz ouvir os surdos e falar os mudos.

\medskip

\lettrine{8} Naqueles dias, havendo uma grande multidão, e não
tendo quê comer, Jesus chamou a si os seus discípulos, e disse-lhes:
Tenho compaixão da multidão, porque há já três dias que estão
comigo, e não têm quê comer. E, se os deixar ir em jejum, para
suas casas, desfalecerão no caminho, porque alguns deles vieram de
longe. E os seus discípulos responderam-lhe: De onde poderá
alguém satisfazê-los de pão aqui no deserto? E perguntou-lhes:
Quantos pães tendes? E disseram-lhe: Sete. E ordenou à multidão
que se assentasse no chão. E, tomando os sete pães, e tendo dado
graças, partiu-os, e deu-os aos seus discípulos, para que os
pusessem diante deles, e puseram-nos diante da multidão. Tinham
também alguns peixinhos; e, tendo dado graças, ordenou que também
lhos pusessem diante. E comeram, e saciaram-se; e dos pedaços
que sobejaram levantaram sete cestos. E os que comeram eram
quase quatro mil; e despediu-os.

E, entrando logo no barco, com os seus discípulos, foi para as
partes de Dalmanuta. E saíram os fariseus, e começaram a
disputar com ele, pedindo-lhe, para o tentarem, um sinal do céu.
E, suspirando profundamente em seu espírito, disse: Por que
pede esta geração um sinal? Em verdade vos digo que a esta geração
não se dará sinal algum. E, deixando-os, tornou a entrar no
barco, e foi para o outro lado. E eles se esqueceram de levar
pão e, no barco, não tinham consigo senão um pão. E
ordenou-lhes, dizendo: Olhai, guardai-vos do fermento dos fariseus e
do fermento de Herodes. E arrazoavam entre si, dizendo: É
porque não temos pão. E Jesus, conhecendo isto, disse-lhes:
Para que arrazoais, que não tendes pão? Não considerastes, nem
compreendestes ainda? Tendes ainda o vosso coração endurecido?
Tendo olhos, não vedes? e tendo ouvidos, não ouvis? e não vos
lembrais, quando parti os cinco pães entre os cinco mil,
quantas alcofas cheias de pedaços levantastes? Disseram-lhe: Doze.
E, quando parti os sete entre os quatro mil, quantos cestos
cheios de pedaços levantastes? E disseram-lhe: Sete. E ele
lhes disse: Como não entendeis ainda?

E chegou a Betsaida; e trouxeram-lhe um cego, e rogaram-lhe que o
tocasse. E, tomando o cego pela mão, levou-o para fora da
aldeia; e, cuspindo-lhe nos olhos, e impondo-lhe as mãos,
perguntou-lhe se via alguma coisa. E, levantando ele os
olhos, disse: Vejo os homens; pois os vejo como árvores que andam.
Depois disto, tornou a pôr-lhe as mãos sobre os olhos, e fez
olhar para cima: e ele ficou restaurado, e viu cada homem
claramente. E mandou-o para sua casa, dizendo: Nem entres na
aldeia, nem o digas a ninguém na aldeia.

E saiu Jesus, e os seus discípulos, para as aldeias de Cesaréia
de Filipe; e no caminho perguntou aos seus discípulos, dizendo: Quem
dizem os homens que eu sou? E eles responderam: João o
Batista; e outros: Elias; mas outros: Um dos profetas. E ele
lhes disse: Mas vós, quem dizeis que eu sou? E, respondendo Pedro,
lhe disse: Tu és o Cristo. E admoestou-os, para que a ninguém
dissessem aquilo dele. E começou a ensinar-lhes que importava
que o Filho do homem padecesse muito, e que fosse rejeitado pelos
anciãos e príncipes dos sacerdotes, e pelos escribas, e que fosse
morto, mas que depois de três dias ressuscitaria. E dizia
abertamente estas palavras. E Pedro o tomou à parte, e começou a
repreendê-lo. Mas ele, virando-se, e olhando para os seus
discípulos, repreendeu a Pedro, dizendo: Retira-te de diante de mim,
Satanás; porque não compreendes as coisas que são de Deus, mas as
que são dos homens. E chamando a si a multidão, com os seus
discípulos, disse-lhes: Se alguém quiser vir após mim, negue-se a si
mesmo, e tome a sua cruz, e siga-me. Porque qualquer que
quiser salvar a sua vida, perdê-la-á, mas, qualquer que perder a sua
vida por amor de mim e do evangelho, esse a salvará. Pois,
que aproveitaria ao homem ganhar todo o mundo e perder a sua alma?
Ou, que daria o homem pelo resgate da sua alma?
Porquanto, qualquer que, entre esta geração adúltera e
pecadora, se envergonhar de mim e das minhas palavras, também o
Filho do homem se envergonhará dele, quando vier na glória de seu
Pai, com os santos anjos.

\medskip

\lettrine{9} Dizia-lhes também: Em verdade vos digo que, dos
que aqui estão, alguns há que não provarão a morte sem que vejam
chegado o reino de Deus com poder. E seis dias depois Jesus
tomou consigo a Pedro, a Tiago, e a João, e os levou sós, em
particular, a um alto monte; e transfigurou-se diante deles; e
as suas vestes tornaram-se resplandecentes, extremamente brancas
como a neve, tais como nenhum lavadeiro sobre a terra
as\footnote{SBTB: ``\emph{os} poderia branquear''.} poderia
branquear. E apareceu-lhes Elias, com Moisés, e falavam com
Jesus. E Pedro, tomando a palavra, disse a Jesus: Mestre, é bom
que estejamos aqui; e façamos três cabanas, uma para ti, outra para
Moisés, e outra para Elias. Pois não sabia o que dizia, porque
estavam assombrados. E desceu uma nuvem que os cobriu com a sua
sombra, e saiu da nuvem uma voz que dizia: \textcolor{red}{Este é o meu
filho amado; a ele ouvi}. E, tendo olhado em redor, ninguém mais
viram, senão só Jesus com eles. E, descendo eles do monte,
ordenou-lhes que a ninguém contassem o que tinham visto, até que o
Filho do homem ressuscitasse dentre os mortos. E eles
retiveram o caso entre si, perguntando uns aos outros que seria
aquilo, ressuscitar dentre os mortos. E interrogaram-no,
dizendo: Por que dizem os escribas que é necessário que Elias venha
primeiro? E, respondendo ele, disse-lhes: Em verdade Elias
virá primeiro, e todas as coisas restaurará; e, como está escrito do
Filho do homem, que ele deva padecer muito e ser aviltado.
Digo-vos, porém, que Elias já veio, e fizeram-lhe tudo o que
quiseram, como dele está escrito.

E, quando se aproximou dos discípulos, viu ao redor deles grande
multidão, e alguns escribas que disputavam com eles. E logo
toda a multidão, vendo-o, ficou espantada e, correndo para ele, o
saudaram. E perguntou aos escribas: Que é que discutis com
eles? E um da multidão, respondendo, disse: Mestre, trouxe-te
o meu filho, que tem um espírito mudo; e este, onde quer que
o apanha, despedaça-o, e ele espuma, e range os dentes, e vai
definhando; e eu disse aos teus discípulos que o expulsassem, e não
puderam. E ele, respondendo-lhes, disse: Ó geração incrédula!
Até quando estarei convosco? Até quando vos sofrerei ainda?
Trazei-mo. E trouxeram-lho; e quando ele o viu, logo o
espírito o agitou com violência, e, caindo o endemoninhado por
terra, revolvia-se, escumando. E perguntou ao pai dele:
Quanto tempo há que lhe sucede isto? E ele disse-lhe: Desde a
infância. E muitas vezes o tem lançado no fogo, e na água,
para o destruir; mas, se tu podes fazer alguma coisa, tem compaixão
de nós, e ajuda-nos. E Jesus disse-lhe: Se tu podes crer,
tudo é possível ao que crê. E logo o pai do menino, clamando,
com lágrimas, disse: Eu creio, Senhor! Ajuda a minha incredulidade.
E Jesus, vendo que a multidão concorria, repreendeu o
espírito imundo, dizendo-lhe: Espírito mudo e surdo, eu te ordeno:
Sai dele, e não entres mais nele. E ele, clamando, e
agitando-o com violência, saiu; e ficou o menino como morto, de tal
maneira que muitos diziam que estava morto. Mas Jesus,
tomando-o pela mão, o ergueu, e ele se levantou. E, quando
entrou em casa, os seus discípulos lhe perguntaram à parte: Por que
o não pudemos nós expulsar? E disse-lhes: Esta casta não pode
sair com coisa alguma, a não ser com oração e jejum.

E, tendo partido dali, caminharam pela Galiléia, e não queria que
alguém o soubesse; porque ensinava os seus discípulos, e lhes
dizia: O Filho do homem será entregue nas mãos dos homens, e
matá-lo-ão; e, morto ele, ressuscitará ao terceiro dia. Mas
eles não entendiam esta palavra, e receavam interrogá-lo. E
chegou a Cafarnaum e, entrando em casa, perguntou-lhes: Que estáveis
vós discutindo pelo caminho? Mas eles calaram-se; porque pelo
caminho tinham disputado entre si qual era o maior. E ele,
assentando-se, chamou os doze, e disse-lhes: Se alguém quiser ser o
primeiro, será o derradeiro de todos e o servo de todos. E,
lançando mão de um menino, pô-lo no meio deles e, tomando-o nos seus
braços, disse-lhes: Qualquer que receber um destes meninos em
meu nome, a mim me recebe; e qualquer que a mim me receber, recebe,
não a mim, mas ao que me enviou. E João lhe respondeu,
dizendo: Mestre, vimos um que em teu nome expulsava demônios, o qual
não nos segue; e nós lho proibimos, porque não nos segue.
Jesus, porém, disse: Não lho proibais; porque ninguém há que
faça milagre em meu nome e possa logo falar mal de mim.
Porque quem não é contra nós, é por nós.

Porquanto, qualquer que vos der a beber um copo de água em meu
nome, porque sois discípulos de Cristo, em verdade vos digo que não
perderá o seu galardão. E qualquer que escandalizar um destes
pequeninos que crêem em mim, melhor lhe fora que lhe pusessem ao
pescoço uma mó de atafona\footnote{Moinho manual ou movido por
cavalgaduras. Azenha.}, e que fosse lançado no mar. E, se a
tua mão te escandalizar, corta-a; melhor é para ti entrares na vida
aleijado do que, tendo duas mãos, ires para o \textcolor{red}{inferno}, para
o fogo que nunca se apaga, onde o seu bicho não morre, e o
fogo nunca se apaga. E, se o teu pé te escandalizar, corta-o;
melhor é para ti entrares coxo na vida do que, tendo dois pés, seres
lançado no inferno, no fogo que nunca se apaga, onde o seu
bicho não morre, e o fogo nunca se apaga. E, se o teu olho te
escandalizar, lança-o fora; melhor é para ti entrares no reino de
Deus com um só olho do que, tendo dois olhos, seres lançado no fogo
do inferno, onde o seu bicho não morre, e o fogo nunca se
apaga. Porque cada um será salgado com fogo, e cada
sacrifício será salgado com sal. Bom é o sal; mas, se o sal
se tornar insípido, com que o temperareis? Tende sal em vós mesmos,
e paz uns com os outros.

\medskip

\lettrine{10} E, levantando-se dali, foi para os termos da
Judéia, além do Jordão, e a multidão se reuniu em torno dele; e
tornou a ensiná-los, como tinha por costume. E, aproximando-se
dele os fariseus, perguntaram-lhe, tentando-o: É lícito ao homem
repudiar sua mulher? Mas ele, respondendo, disse-lhes: Que vos
mandou Moisés? E eles disseram: Moisés permitiu escrever carta
de divórcio e repudiar. E Jesus, respondendo, disse-lhes: Pela
dureza dos vossos corações vos deixou ele escrito esse mandamento;
porém, desde o princípio da criação, Deus os fez macho e fêmea.
Por isso \textcolor{red}{deixará o homem a seu pai e a sua mãe, e
unir-se-á a sua mulher}, \textcolor{red}{e serão os dois uma só carne; e
assim já não serão dois, mas uma só carne}. \textcolor{red}{Portanto, o
que Deus ajuntou não o separe o homem}. E em casa tornaram os
discípulos a interrogá-lo acerca disto mesmo. E ele lhes
disse: \textcolor{red}{Qualquer que deixar a sua mulher e casar com outra,
adultera contra ela}. \textcolor{red}{E, se a mulher deixar a seu
marido, e casar com outro, adultera}.

E traziam-lhe meninos para que lhes tocasse, mas os discípulos
repreendiam aos que lhos traziam. Jesus, porém, vendo isto,
indignou-se, e disse-lhes: Deixai vir os meninos a mim, e não os
impeçais; porque dos tais é o reino de Deus. Em verdade vos
digo que qualquer que não receber o reino de Deus como menino, de
maneira nenhuma entrará nele. E, tomando-os nos seus braços,
e impondo-lhes as mãos, os abençoou.

E, pondo-se a caminho, correu para ele um homem, o qual se
ajoelhou diante dele, e lhe perguntou: Bom Mestre, que farei para
herdar a vida eterna? E Jesus lhe disse: Por que me chamas
bom? Ninguém há bom senão um, que é Deus. Tu sabes os
mandamentos: Não adulterarás; não matarás; não furtarás; não dirás
falso testemunho; não defraudarás alguém; honra a teu pai e a tua
mãe. Ele, porém, respondendo, lhe disse: Mestre, tudo isso
guardei desde a minha mocidade. E Jesus, olhando para ele, o
amou e lhe disse: Falta-te uma coisa: vai, vende tudo quanto tens, e
dá-o aos pobres, e terás um tesouro no céu; e vem, toma a cruz, e
segue-me. Mas ele, pesaroso desta palavra, retirou-se triste;
porque possuía muitas propriedades. Então Jesus, olhando em
redor, disse aos seus discípulos: Quão dificilmente entrarão no
reino de Deus os que têm riquezas! E os discípulos se
admiraram destas suas palavras; mas Jesus, tornando a falar,
disse-lhes: Filhos, quão difícil é, para os que confiam nas
riquezas, entrar no reino de Deus! É mais fácil passar um
camelo pelo fundo de uma agulha, do que entrar um rico no reino de
Deus. E eles se admiravam ainda mais, dizendo entre si: Quem
poderá, pois, salvar-se? Jesus, porém, olhando para eles,
disse: Para os homens é impossível, mas não para Deus, porque para
Deus todas as coisas são possíveis. E Pedro começou a
dizer-lhe: Eis que nós tudo deixamos, e te seguimos. E Jesus,
respondendo, disse: Em verdade vos digo que ninguém há, que tenha
deixado casa, ou irmãos, ou irmãs, ou pai, ou mãe, ou mulher, ou
filhos, ou campos, por amor de mim e do evangelho, que não
receba cem vezes tanto, já neste tempo, em casas, e irmãos, e irmãs,
e mães, e filhos, e campos, com perseguições; e no século futuro a
vida eterna. Porém muitos primeiros serão derradeiros, e
muitos derradeiros serão primeiros.

E iam no caminho, subindo para Jerusalém; e Jesus ia adiante
deles. E eles maravilhavam-se, e seguiam-no atemorizados. E,
tornando a tomar consigo os doze, começou a dizer-lhes as coisas que
lhe deviam sobrevir, dizendo: Eis que nós subimos a
Jerusalém, e o Filho do homem será entregue aos príncipes dos
sacerdotes, e aos escribas, e o condenarão à morte, e o entregarão
aos gentios. E o escarnecerão, e açoitarão, e cuspirão nele,
e o matarão; e, \textcolor{red}{ao terceiro dia, ressuscitará}. E
aproximaram-se dele Tiago e João, filhos de Zebedeu, dizendo:
Mestre, queremos que nos faças o que te pedirmos. E ele lhes
disse: Que quereis que vos faça? E eles lhe disseram:
Concede-nos que na tua glória nos assentemos, um à tua direita, e
outro à tua esquerda. Mas Jesus lhes disse: Não sabeis o que
pedis; podeis vós beber o cálice que eu bebo, e ser batizados com o
batismo com que eu sou batizado? E eles lhe disseram:
Podemos. Jesus, porém, disse-lhes: Em verdade, vós bebereis o cálice
que eu beber, e sereis batizados com o batismo com que eu sou
batizado; mas, o assentar-se à minha direita, ou à minha
esquerda, não me pertence a mim concedê-lo, mas isso é para aqueles
a quem está reservado. E os dez, tendo ouvido isto, começaram
a indignar-se contra Tiago e João. Mas Jesus, chamando-os a
si, disse-lhes: Sabeis que os que julgam ser príncipes dos gentios,
deles se assenhoreiam, e os seus grandes usam de autoridade sobre
elas; mas entre vós não será assim; antes, qualquer que entre
vós quiser ser grande, será vosso serviçal; e qualquer que
dentre vós quiser ser o primeiro, será servo de todos.
\textcolor{red}{Porque o Filho do homem também não veio para ser
servido, mas para servir e dar a sua vida em resgate de muitos}.

Depois, foram para Jericó. E, saindo ele de Jericó com seus
discípulos e uma grande multidão, Bartimeu, o cego, filho de Timeu,
estava assentado junto do caminho, mendigando. E, ouvindo que
era Jesus de Nazaré, começou a clamar, e a dizer: Jesus, filho de
Davi, tem misericórdia de mim. E muitos o repreendiam, para
que se calasse; mas ele clamava cada vez mais: Filho de Davi! Tem
misericórdia de mim. E Jesus, parando, disse que o chamassem;
e chamaram o cego, dizendo-lhe: Tem bom ânimo; levanta-te, que ele
te chama. E ele, lançando de si a sua capa, levantou-se, e
foi ter com Jesus. E Jesus, falando, disse-lhe: Que queres
que te faça? E o cego lhe disse: Mestre, que eu tenha vista.
E Jesus lhe disse: Vai, a tua fé te salvou. E logo viu, e
seguiu a Jesus pelo caminho.

\medskip

\lettrine{11} E, logo que se aproximaram de Jerusalém, de
Betfagé e de Betânia, junto do Monte das Oliveiras, enviou dois dos
seus discípulos, e disse-lhes: Ide à aldeia que está defronte de
vós; e, logo que ali entrardes, encontrareis preso um jumentinho,
sobre o qual ainda não montou homem algum; soltai-o, e trazei-mo.
E, se alguém vos disser: Por que fazeis isso? dizei-lhe que o
Senhor precisa dele, e logo o deixará trazer para aqui. E foram,
e encontraram o jumentinho preso fora da porta, entre dois caminhos,
e o soltaram. E alguns dos que ali estavam lhes disseram: Que
fazeis, soltando o jumentinho? Eles, porém, disseram-lhes como
Jesus lhes tinha mandado; e deixaram-nos ir. E levaram o
jumentinho a Jesus, e lançaram sobre ele as suas vestes, e
assentou-se sobre ele. E muitos estendiam as suas vestes pelo
caminho, e outros cortavam ramos das árvores, e os espalhavam pelo
caminho. E aqueles que iam adiante, e os que seguiam, clamavam,
dizendo: \textcolor{red}{Hosana, bendito o que vem em nome do Senhor};
\textcolor{red}{bendito o reino do nosso pai Davi, que vem em nome do
Senhor. Hosana nas alturas}. E Jesus entrou em Jerusalém, no
templo, e, tendo visto tudo em redor, como fosse já tarde, saiu para
Betânia com os doze.

E, no dia seguinte, quando saíram de Betânia, teve fome.
E, vendo de longe uma figueira que tinha folhas, foi ver se
nela acharia alguma coisa; e, chegando a ela, não achou senão
folhas, porque não era tempo de figos. E Jesus, falando,
disse à figueira: Nunca mais coma alguém fruto de ti. E os seus
discípulos ouviram isto. E vieram a Jerusalém; e Jesus,
entrando no templo, começou a expulsar os que vendiam e compravam no
templo; e derrubou as mesas dos cambiadores e as cadeiras dos que
vendiam pombas. E não consentia que alguém levasse algum vaso
pelo templo. E os ensinava, dizendo: Não está escrito: A
minha casa será chamada, por todas as nações, casa de oração? Mas
vós a tendes feito covil de ladrões. E os escribas e
príncipes dos sacerdotes, tendo ouvido isto, buscavam ocasião para o
matar; pois eles o temiam, porque toda a multidão estava admirada
acerca da sua doutrina. E, sendo já tarde,
saiu\footnote{SBTB: ``saiu para fora''. Pleonasmo vicioso. King
James: ``he went out of the city''.} da cidade. E eles,
passando pela manhã, viram que a figueira se tinha secado desde as
raízes. E Pedro, lembrando-se, disse-lhe: Mestre, eis que a
figueira, que tu amaldiçoaste, se secou. E Jesus,
respondendo, disse-lhes: Tende fé em Deus; porque em verdade
vos digo que qualquer que disser a este monte: Ergue-te e lança-te
no mar, e não duvidar em seu coração, mas crer que se fará aquilo
que diz, tudo o que disser lhe será feito. Por isso vos digo
que todas as coisas que pedirdes, orando, crede receber, e
tê-las-eis. E, quando estiverdes orando, perdoai, se tendes
alguma coisa contra alguém, para que vosso Pai, que está nos céus,
vos perdoe as vossas ofensas. Mas, se vós não perdoardes,
também vosso Pai, que está nos céus, vos não perdoará as vossas
ofensas.

E tornaram a Jerusalém, e, andando ele pelo templo, os principais
dos sacerdotes, e os escribas, e os anciãos, se aproximaram dele.
E lhe disseram: Com que autoridade fazes tu estas coisas? ou
quem te deu tal autoridade para fazer estas coisas? Mas
Jesus, respondendo, disse-lhes: Também eu vos perguntarei uma coisa,
e respondei-me; e então vos direi com que autoridade faço estas
coisas: O batismo de João era do céu ou dos homens?
Respondei-me. E eles arrazoavam entre si, dizendo: Se
dissermos: Do céu, ele nos dirá: Então por que o não crestes?
Se, porém, dissermos: Dos homens, tememos o povo. Porque
todos sustentavam que João verdadeiramente era profeta. E,
respondendo, disseram a Jesus: Não sabemos. E Jesus lhes replicou:
Também eu vos não direi com que autoridade faço estas coisas.

\medskip

\lettrine{12} E começou a falar-lhes por parábolas: Um homem
plantou uma vinha, e cercou-a de um valado, e fundou nela um lagar,
e edificou uma torre, e arrendou-a a uns lavradores, e partiu para
fora da terra. E, chegado o tempo, mandou um servo aos
lavradores para que recebesse, dos lavradores, do fruto da vinha.
Mas estes, apoderando-se dele, o feriram e o mandaram embora
vazio. E tornou a enviar-lhes outro servo; e eles,
apedrejando-o, o feriram na cabeça, e o mandaram embora, tendo-o
afrontado. E tornou a enviar-lhes outro, e a este mataram; e a
outros muitos, dos quais a uns feriram e a outros mataram. Tendo
ele, pois, ainda um seu filho amado, enviou-o também a estes por
derradeiro, dizendo: Ao menos terão respeito ao meu filho. Mas
aqueles lavradores disseram entre si: Este é o herdeiro; vamos,
matemo-lo, e a herança será nossa. E, pegando dele, o mataram, e
o lançaram fora da vinha. Que fará, pois, o Senhor da vinha?
Virá, e destruirá os lavradores, e dará a vinha a outros.
Ainda não lestes esta Escritura: A pedra, que os edificadores
rejeitaram, esta foi posta por cabeça de esquina; isto foi
feito pelo Senhor e é coisa maravilhosa aos nossos olhos? E
buscavam prendê-lo, mas temiam a multidão; porque entendiam que
contra eles dizia esta parábola; e, deixando-o, foram-se.

E enviaram-lhe alguns dos fariseus e dos herodianos, para que o
apanhassem nalguma palavra. E, chegando eles, disseram-lhe:
Mestre, sabemos que és homem de verdade, e de ninguém se te
dá\footnote{Bíblia de Jerusalém: ``e não dás preferência a
ninguém''. King James: ``And when they were come, they say unto him,
Master, we know that thou art true, and carest for no man: for thou
regardest not the person of men, but teachest the way of God in
truth: Is it lawful to give tribute to Caesar, or not?''}, porque
não olhas à aparência dos homens, antes com verdade ensinas o
caminho de Deus; é lícito dar o tributo a César, ou não? Daremos, ou
não daremos? Então ele, conhecendo a sua hipocrisia,
disse-lhes: Por que me tentais? Trazei-me uma moeda, para que a
veja. E eles lha trouxeram. E disse-lhes: De quem é esta
imagem e inscrição? E eles lhe disseram: De César. E Jesus,
respondendo, disse-lhes: Dai pois a César o que é de César, e a Deus
o que é de Deus. E maravilharam-se dele.

Então os saduceus, que dizem que não há ressurreição,
aproximaram-se dele, e perguntaram-lhe, dizendo: Mestre,
Moisés nos escreveu que, se morresse o irmão de alguém, e deixasse a
mulher e não deixasse filhos, seu irmão tomasse a mulher dele, e
suscitasse descendência a seu irmão. Ora, havia sete irmãos,
e o primeiro tomou a mulher, e morreu sem deixar descendência;
e o segundo também a tomou e morreu, e nem este deixou
descendência; e o terceiro da mesma maneira. E tomaram-na os
sete, sem, contudo, terem deixado descendência. Finalmente, depois
de todos, morreu também a mulher. Na ressurreição, pois,
quando ressuscitarem, de qual destes será a mulher? Porque os sete a
tiveram por mulher. E Jesus, respondendo, disse-lhes:
Porventura não errais vós em razão de não saberdes as Escrituras nem
o poder de Deus? Porquanto, quando ressuscitarem dentre os
mortos, nem casarão, nem se darão em casamento, mas serão como os
anjos que estão nos céus. E, acerca dos mortos que houverem
de ressuscitar, não tendes lido no livro de Moisés como Deus lhe
falou na sarça, dizendo: Eu sou o Deus de Abraão, e o Deus de
Isaque, e o Deus de Jacó? Ora, Deus não é de mortos, mas sim,
é Deus de vivos. Por isso vós errais muito.

Aproximou-se dele um dos escribas que os tinha ouvido disputar, e
sabendo que lhes tinha respondido bem, perguntou-lhe: Qual é o
primeiro de todos os mandamentos? E Jesus respondeu-lhe: O
primeiro de todos os mandamentos é: Ouve, Israel, o Senhor nosso
Deus é o único Senhor. Amarás, pois, ao Senhor teu Deus de
todo o teu coração, e de toda a tua alma, e de todo o teu
entendimento, e de todas as tuas forças; este é o primeiro
mandamento. E o segundo, semelhante a este, é: Amarás o teu
próximo como a ti mesmo. Não há outro mandamento maior do que estes.
E o escriba lhe disse: Muito bem, Mestre, e com verdade
disseste que há um só Deus, e que não há outro além dele; e
que amá-lo de todo o coração, e de todo o entendimento, e de toda a
alma, e de todas as forças, e amar o próximo como a si mesmo, é mais
do que todos os holocaustos e sacrifícios. E Jesus, vendo que
havia respondido sabiamente, disse-lhe: Não estás longe do reino de
Deus. E já ninguém ousava perguntar-lhe mais nada.

E, falando Jesus, dizia, ensinando no templo: Como dizem os
escribas que o Cristo é filho de Davi? O próprio Davi disse
pelo Espírito Santo: O Senhor disse ao meu Senhor: Assenta-te à
minha direita até que eu ponha os teus inimigos por escabelo dos
teus pés. Pois, se Davi mesmo lhe chama Senhor, como é logo
seu filho? E a grande multidão o ouvia de boa vontade. E,
ensinando-os, dizia-lhes: Guardai-vos dos escribas, que gostam de
andar com vestes compridas, e das saudações nas praças, e das
primeiras cadeiras nas sinagogas, e dos primeiros assentos nas
ceias; que devoram as casas das viúvas, e isso com pretexto
de largas orações. Estes receberão mais grave condenação.

E, estando Jesus assentado defronte da arca do tesouro, observava
a maneira como a multidão lançava o dinheiro na arca do tesouro; e
muitos ricos deitavam muito. Vindo, porém, uma pobre viúva,
deitou dois leptos, que valiam um quadrante\footnote{SBTB: deitou
duas pequenas moedas, que valiam meio centavo. A palavra grega é
``quadrante'': uma moeda romana equivalente a $1/4$ da moeda de
cobre \emph{as} --- era a menor das moedas romanas. As versões RA,
BJ e RC registram ``quadrante'', sendo que nesta última há uma nota:
``um centavo em moeda brasileira''. Além disso, a tradução literal
de ``duas pequenas moedas'' é ``dois \emph{lepta}'' ou leptos.}.
E, chamando os seus discípulos, disse-lhes: Em verdade vos
digo que esta pobre viúva deitou mais do que todos os que deitaram
na arca do tesouro; porque todos ali deitaram do que lhes
sobejava, mas esta, da sua pobreza, deitou tudo o que tinha, todo o
seu sustento.

\medskip

\lettrine{13} E, saindo ele do templo, disse-lhe um dos seus
discípulos: Mestre, olha que pedras, e que edifícios! E,
respondendo Jesus, disse-lhe: Vês estes grandes edifícios? Não
ficará pedra sobre pedra que não seja derrubada. E,
assentando-se ele no Monte das Oliveiras, defronte do templo, Pedro,
e Tiago, e João e André lhe perguntaram em particular: Dize-nos,
quando serão essas coisas, e que sinal haverá quando todas elas
estiverem para se cumprir.

E Jesus, respondendo-lhes, começou a dizer: Olhai que ninguém vos
engane; porque muitos virão em meu nome, dizendo: Eu sou o
Cristo; e enganarão a muitos. E, quando ouvirdes de guerras e de
rumores de guerras, não vos perturbeis; porque assim deve acontecer;
mas ainda não será o fim. Porque se levantará nação contra
nação, e reino contra reino, e haverá terremotos em diversos
lugares, e haverá fomes e tribulações. Estas coisas são os
princípios das dores. Mas olhai por vós mesmos, porque vos
entregarão aos concílios e às sinagogas; e sereis açoitados, e
sereis apresentados perante presidentes e reis, por amor de mim,
para lhes servir de testemunho. Mas importa que o evangelho
seja primeiramente pregado entre todas as nações. Quando,
pois, vos conduzirem e vos entregarem, não estejais solícitos de
antemão pelo que haveis de dizer, nem premediteis; mas, o que vos
for dado naquela hora, isso falai, porque não sois vós os que
falais, mas o Espírito Santo. E o irmão entregará à morte o
irmão, e o pai ao filho; e levantar-se-ão os filhos contra os pais,
e os farão morrer. E sereis odiados por todos por amor do meu
nome; mas quem perseverar até ao fim, esse será salvo.

Ora, quando vós virdes a abominação do assolamento, que foi
predito por Daniel o profeta, estar onde não deve estar (quem lê,
entenda), então os que estiverem na Judéia fujam para os montes.
E o que estiver sobre o telhado não desça para casa, nem
entre a tomar coisa alguma de sua casa; e o que estiver no
campo não volte atrás, para tomar as suas vestes. Mas ai das
grávidas, e das que criarem naqueles dias! Orai, pois, para
que a vossa fuga não suceda no inverno. Porque naqueles dias
haverá uma aflição tal, qual nunca houve desde o princípio da
criação, que Deus criou, até agora, nem jamais haverá. E, se
o Senhor não abreviasse aqueles dias, nenhuma carne se salvaria;
mas, por causa dos eleitos que escolheu, abreviou aqueles dias.
E então, se alguém vos disser: Eis aqui o Cristo; ou: Ei-lo
ali; não acrediteis. Porque se levantarão falsos cristos, e
falsos profetas, e farão sinais e prodígios, para enganarem, se for
possível, até os escolhidos. Mas vós vede; eis que de antemão
vos tenho dito tudo.

Ora, naqueles dias, depois daquela aflição, o sol se escurecerá,
e a lua não dará a sua luz. E as estrelas cairão do céu, e as
forças que estão nos céus serão abaladas. E então verão vir o
Filho do homem nas nuvens, com grande poder e glória. E ele
enviará os seus anjos, e ajuntará os seus escolhidos, desde os
quatro ventos, da extremidade da terra até a extremidade do céu.

Aprendei, pois, a parábola da figueira: Quando já o seu ramo se
torna tenro, e brota folhas, bem sabeis que já está próximo o verão.
Assim também vós, quando virdes sucederem estas coisas, sabei
que já está perto, às portas. Na verdade vos digo que não
passará esta geração, sem que todas estas coisas aconteçam.
Passará o céu e a terra, mas as minhas palavras não passarão.
Mas daquele dia e hora ninguém sabe, nem os anjos que estão
no céu, nem o Filho, senão o Pai. Olhai, vigiai e orai;
porque não sabeis quando chegará o tempo. É como se um homem,
partindo para fora da terra, deixasse a sua casa, e desse autoridade
aos seus servos, e a cada um a sua obra, e mandasse ao porteiro que
vigiasse. Vigiai, pois, porque não sabeis quando virá o
senhor da casa; se à tarde, se à meia-noite, se ao cantar do galo,
se pela manhã, para que, vindo de improviso, não vos ache
dormindo. E as coisas que vos digo, digo-as a todos: Vigiai.

\medskip

\lettrine{14} E dali a dois dias era a páscoa, e a festa dos
pães ázimos; e os principais dos sacerdotes e os escribas buscavam
como o prenderiam com dolo, e o matariam. Mas eles diziam: Não
na festa, para que porventura não se faça alvoroço entre o povo.
E, estando ele em Betânia, assentado à mesa, em casa de Simão, o
leproso, veio uma mulher, que trazia um vaso de alabastro, com
ungüento de nardo puro, de muito preço, e quebrando o vaso, lho
derramou sobre a cabeça. E alguns houve que em si mesmos se
indignaram, e disseram: Para que se fez este desperdício de
ungüento? Porque podia vender-se por mais de trezentos
denários\footnote{SBTB: dinheiros.}, e dá-lo aos pobres. E bramavam
contra ela. Jesus, porém, disse: Deixai-a, por que a molestais?
Ela fez-me boa obra. Porque sempre tendes os pobres convosco, e
podeis fazer-lhes bem, quando quiserdes; mas a mim nem sempre me
tendes. Esta fez o que podia; antecipou-se a ungir o meu corpo
para a sepultura. Em verdade vos digo que, em todas as partes do
mundo onde este evangelho for pregado, também o que ela fez será
contado para sua memória. E Judas Iscariotes, um dos doze,
foi ter com os principais dos sacerdotes para lho entregar. E
eles, ouvindo-o, folgaram, e prometeram dar-lhe dinheiro; e buscava
como o entregaria em ocasião oportuna.

E, no primeiro dia dos pães ázimos, quando sacrificavam a páscoa,
disseram-lhe os discípulos: Aonde queres que vamos fazer os
preparativos para comer a páscoa? E enviou dois dos seus
discípulos, e disse-lhes: Ide à cidade, e um homem, que leva um
cântaro de água, vos encontrará; segui-o. E, onde quer que
entrar, dizei ao senhor da casa: O Mestre diz: Onde está o aposento
em que hei de comer a páscoa com os meus discípulos? E ele
vos mostrará um grande cenáculo mobiliado\footnote{SBTB: mobilado.
Regionalismo: Portugal.} e preparado; preparai-a ali. E,
saindo os seus discípulos, foram à cidade, e acharam como lhes tinha
dito, e prepararam a páscoa. E, chegada a tarde, foi com os
doze. E, quando estavam assentados a comer, disse Jesus: Em
verdade vos digo que um de vós, que comigo come, há de trair-me.
E eles começaram a entristecer-se e a dizer-lhe um após
outro: Sou eu? E outro disse: Sou eu? Mas ele, respondendo,
disse-lhes: É um dos doze, que põe comigo a mão no prato. Na
verdade o Filho do homem vai, como dele está escrito, mas ai daquele
homem por quem o Filho do homem é traído! Bom seria para o tal homem
não haver nascido. E, comendo eles, tomou Jesus pão e,
abençoando-o, o partiu e deu-lho, e disse: Tomai, comei, isto é o
meu corpo. E, tomando o cálice, e dando graças, deu-lho; e
todos beberam dele. E disse-lhes: \textcolor{red}{Isto é o meu
sangue, o sangue do novo testamento, que por muitos é derramado}.
Em verdade vos digo que não beberei mais do fruto da vide,
até àquele dia em que o beber, novo, no reino de Deus. E,
tendo cantado o hino, saíram para o Monte das Oliveiras. E
disse-lhes Jesus: Todos vós esta noite vos escandalizareis em mim;
porque está escrito: Ferirei o pastor, e as ovelhas se dispersarão.
Mas, depois que eu houver ressuscitado, irei adiante de vós
para a Galiléia. E disse-lhe Pedro: Ainda que todos se
escandalizem, nunca, porém, eu. E disse-lhe Jesus: Em verdade
te digo que hoje, nesta noite, antes que o galo cante duas vezes,
três vezes me negarás. Mas ele disse com mais veemência:
Ainda que me seja necessário morrer contigo, de modo nenhum te
negarei. E da mesma maneira diziam todos também.

E foram a um lugar chamado Getsêmani, e disse aos seus
discípulos: Assentai-vos aqui, enquanto eu oro. E tomou
consigo a Pedro, e a Tiago, e a João, e começou a ter pavor, e a
angustiar-se. E disse-lhes: A minha alma está profundamente
triste até a morte; ficai aqui, e vigiai. E, tendo ido um
pouco mais adiante, prostrou-se em terra; e orou para que, se fosse
possível, passasse dele aquela hora. E disse: Aba, Pai, todas
as coisas te são possíveis; afasta de mim este cálice; não seja,
porém, o que eu quero, mas o que tu queres. E, chegando,
achou-os dormindo; e disse a Pedro: Simão, dormes? Não podes vigiar
uma hora? Vigiai e orai, para que não entreis em tentação; o
espírito, na verdade, está pronto, mas a carne é fraca. E foi
outra vez e orou, dizendo as mesmas palavras. E, voltando,
achou-os outra vez dormindo, porque os seus olhos estavam pesados, e
não sabiam o que responder-lhe. E voltou terceira vez, e
disse-lhes: Dormi agora, e descansai. Basta; é chegada a hora. Eis
que o Filho do homem vai ser entregue nas mãos dos pecadores.
Levantai-vos, vamos; eis que está perto o que me trai.

E logo, falando ele ainda, veio Judas, que era um dos doze, da
parte dos principais dos sacerdotes, e dos escribas e dos anciãos, e
com ele uma grande multidão com espadas e varapaus. Ora, o
que o traía tinha-lhes dado um sinal, dizendo: Aquele que eu beijar,
esse é; prendei-o, e levai-o com segurança. E, logo que
chegou, aproximou-se dele, e disse-lhe: Rabi, Rabi. E beijou-o.
E lançaram-lhe as mãos, e o prenderam. E um dos que
ali estavam presentes, puxando da espada, feriu o servo do sumo
sacerdote, e cortou-lhe uma orelha. E, respondendo Jesus,
disse-lhes: Saístes com espadas e varapaus a prender-me, como a um
salteador? Todos os dias estava convosco ensinando no templo,
e não me prendestes; mas isto é para que as Escrituras se cumpram.
Então, deixando-o, todos fugiram. E um certo jovem o
seguia, envolto em um lençol sobre o corpo nu. E lançaram-lhe a mão.
Mas ele, largando o lençol, fugiu nu.

E levaram Jesus ao sumo sacerdote, e ajuntaram-se todos os
principais dos sacerdotes, e os anciãos e os escribas. E
Pedro o seguiu de longe até dentro do pátio do sumo sacerdote, e
estava assentado com os servidores, aquentando-se ao lume. E
os principais dos sacerdotes e todo o concílio buscavam algum
testemunho contra Jesus, para o matar, e não o achavam.
Porque muitos testificavam falsamente contra ele, mas os
testemunhos não eram coerentes. E, levantando-se alguns,
testificaram falsamente contra ele, dizendo: Nós ouvimos-lhe
dizer: Eu derrubarei este templo, construído por mãos de homens, e
em três dias edificarei outro, não feito por mãos de homens.
E nem assim o seu testemunho era coerente. E,
levantando-se o sumo sacerdote no Sinédrio, perguntou a Jesus,
dizendo: Nada respondes? Que testificam estes contra ti? Mas
ele calou-se, e nada respondeu. O sumo sacerdote lhe tornou a
perguntar, e disse-lhe: És tu o Cristo, Filho do Deus Bendito?
E Jesus disse-lhe: Eu o sou, e vereis o Filho do homem
assentado à direita do poder de Deus, e vindo sobre as nuvens do
céu. E o sumo sacerdote, rasgando as suas vestes, disse: Para
que necessitamos de mais testemunhas? Vós ouvistes a
blasfêmia; que vos parece? E todos o consideraram culpado de morte.
E alguns começaram a cuspir nele, e a cobrir-lhe o rosto, e a
dar-lhe punhadas, e a dizer-lhe: Profetiza. E os servidores
davam-lhe bofetadas.

E, estando Pedro embaixo, no átrio, chegou uma das criadas do
sumo sacerdote; e, vendo a Pedro, que se estava aquentando,
olhou para ele, e disse: Tu também estavas com Jesus Nazareno.
Mas ele negou-o, dizendo: Não o conheço, nem sei o que dizes.
E saiu\footnote{SBTB: saiu fora --- pleonasmo.} ao alpendre, e o
galo cantou. E a criada, vendo-o outra vez, começou a dizer
aos que ali estavam: Este é um dos tais. Mas ele o negou
outra vez. E pouco depois os que ali estavam disseram outra vez a
Pedro: Verdadeiramente tu és um deles, porque és também galileu, e
tua fala é semelhante. E ele começou a praguejar, e a jurar:
Não conheço esse homem de quem falais. E o galo cantou
segunda vez. E Pedro lembrou-se da palavra que Jesus lhe tinha dito:
Antes que o galo cante duas vezes, três vezes me negarás. E,
retirando-se dali, chorou.

\medskip

\lettrine{15} E, logo ao amanhecer, os principais dos
sacerdotes, com os anciãos, e os escribas, e todo o Sinédrio,
tiveram conselho; e, ligando Jesus, o levaram e entregaram a
Pilatos. E Pilatos lhe perguntou: Tu és o Rei dos Judeus? E ele,
respondendo, disse-lhe: Tu o dizes. E os principais dos
sacerdotes o acusavam de muitas coisas; porém ele nada respondia.
E Pilatos o interrogou outra vez, dizendo: Nada respondes? Vê
quantas coisas testificam contra ti. Mas Jesus nada mais
respondeu, de maneira que Pilatos se maravilhava. Ora, no dia da
festa costumava soltar-lhes um preso qualquer que eles pedissem.
E havia um chamado Barrabás, que, preso com outros amotinadores,
tinha num motim cometido uma morte. E a multidão, dando gritos,
começou a pedir que fizesse como sempre lhes tinha feito. E
Pilatos lhes respondeu, dizendo: Quereis que vos solte o Rei dos
Judeus? Porque ele bem sabia que por inveja os principais dos
sacerdotes o tinham entregado. Mas os principais dos
sacerdotes incitaram a multidão para que fosse solto antes Barrabás.
E Pilatos, respondendo, lhes disse outra vez: Que quereis,
pois, que faça daquele a quem chamais Rei dos Judeus? E eles
tornaram a clamar: Crucifica-o. Mas Pilatos lhes disse: Mas
que mal fez? E eles cada vez clamavam mais: Crucifica-o.

Então Pilatos, querendo satisfazer a multidão, soltou-lhe
Barrabás e, açoitado Jesus, o entregou para ser crucificado.
E os soldados o levaram dentro à sala, que é a da audiência,
e convocaram toda a coorte. E vestiram-no de
púrpura\footnote{Houaiss: cor vibrante vermelho-escura, tendente
para o roxo. Substância corante, vermelho-escura, extraída de
moluscos. Derivação (por metonímia): tecido vermelho, tingido com
essa substância --- era muito valorizado na Antiguidade e na Idade
Média, dava status e era símbolo do poder real e eclesiástico.
Derivação (por metonímia): vestimenta dos reis; o trono, a dignidade
real. Diacronismo (antigo): entre os antigos romanos, a dignidade
dos cônsules.}, e tecendo uma coroa de espinhos, lha puseram na
cabeça. E começaram a saudá-lo, dizendo: Salve, Rei dos
Judeus! E feriram-no na cabeça com uma cana, e cuspiram nele
e, postos de joelhos, o adoraram. E, havendo-o escarnecido,
despiram-lhe a púrpura, e o vestiram com as suas próprias vestes; e
o levaram para fora a fim de o crucificarem. E constrangeram
um certo Simão, cireneu, pai de Alexandre e de Rufo, que por ali
passava, vindo do campo, a que levasse a cruz.

E levaram-no ao lugar do Gólgota, que se traduz por lugar da
Caveira. E deram-lhe a beber vinho com mirra, mas ele não o
tomou. E, havendo-o crucificado, repartiram as suas vestes,
lançando sobre elas sortes, para saber o que cada um levaria.
E era a hora terceira, e o crucificaram. E por cima
dele estava escrita a sua acusação: O REI DOS JUDEUS. E
crucificaram com ele dois salteadores, um à sua direita, e outro à
esquerda. E cumprindo-se a escritura que diz: E com os
malfeitores foi contado. E os que passavam blasfemavam dele,
meneando as suas cabeças, e dizendo: Ah! tu que derrubas o templo, e
em três dias o edificas, salva-te a ti mesmo, e desce da
cruz. E da mesma maneira também os principais dos sacerdotes,
com os escribas, diziam uns para os outros, zombando: Salvou os
outros, e não pode salvar-se a si mesmo. O Cristo, o Rei de
Israel, desça agora da cruz, para que o vejamos e acreditemos.
Também os que com ele foram crucificados o injuriavam.

E, chegada a hora sexta, houve trevas sobre toda a terra até a
hora nona. E, à hora nona, Jesus exclamou com grande voz,
dizendo: Eloí, Eloí, lamá sabactâni? Que, traduzido, é: Deus meu,
Deus meu, por que me desamparaste? E alguns dos que ali
estavam, ouvindo isto, diziam: Eis que chama por Elias. E um
deles correu a embeber uma esponja em vinagre e, pondo-a numa cana,
deu-lho a beber, dizendo: Deixai, vejamos se virá Elias tirá-lo.
E Jesus, dando um grande brado, expirou. E o véu do
templo se rasgou em dois, de alto a baixo. E o centurião, que
estava defronte dele, vendo que assim clamando expirara, disse:
Verdadeiramente este homem era o Filho de Deus. E também ali
estavam algumas mulheres, olhando de longe, entre as quais também
Maria Madalena, e Maria, mãe de Tiago, o menor, e de José, e Salomé;
as quais também o seguiam, e o serviam, quando estava na
Galiléia; e muitas outras, que tinham subido com ele a Jerusalém.

E, chegada a tarde, porquanto era o dia da preparação, isto é, a
véspera do sábado, chegou José de Arimatéia, senador honrado,
que também esperava o reino de Deus, e ousadamente foi a Pilatos, e
pediu o corpo de Jesus. E Pilatos se maravilhou de que já
estivesse morto. E, chamando o centurião, perguntou-lhe se já havia
muito que tinha morrido. E, tendo-se certificado pelo
centurião, deu o corpo a José; o qual comprara um lençol
fino, e, tirando-o da cruz, o envolveu nele, e o depositou num
sepulcro lavrado numa rocha; e revolveu uma pedra para a porta do
sepulcro. E Maria Madalena e Maria, mãe de José, observavam
onde o punham.

\medskip

\lettrine{16} E, passado o sábado, Maria Madalena, e Maria,
mãe de Tiago, e Salomé, compraram aromas para irem ungi-lo. E,
no primeiro dia da semana, foram ao sepulcro, de manhã cedo, ao
nascer do sol. E diziam umas às outras: Quem nos revolverá a
pedra da porta do sepulcro? E, olhando, viram que já a pedra
estava revolvida; e era ela muito grande. E, entrando no
sepulcro, viram um jovem assentado à direita, vestido de uma roupa
comprida, branca; e ficaram espantadas. Ele, porém, disse-lhes:
Não vos assusteis; buscais a Jesus Nazareno, que foi crucificado; já
ressuscitou, não está aqui; eis aqui o lugar onde o puseram. Mas
ide, dizei a seus discípulos, e a Pedro, que ele vai adiante de vós
para a Galiléia; ali o vereis, como ele vos disse. E, saindo
elas apressadamente, fugiram do sepulcro, porque estavam possuídas
de temor e assombro; e nada diziam a ninguém porque temiam.

E Jesus, tendo ressuscitado na manhã do primeiro dia da semana,
apareceu primeiramente a Maria Madalena, da qual tinha expulsado
sete demônios. E, partindo ela, anunciou-o àqueles que tinham
estado com ele, os quais estavam tristes, e chorando. E,
ouvindo eles que vivia, e que tinha sido visto por ela, não o
creram. E depois manifestou-se de outra forma a dois deles,
que iam de caminho para o campo. E, indo estes, anunciaram-no
aos outros, mas nem ainda estes creram.

Finalmente apareceu aos onze, estando eles assentados à mesa, e
lan\-çou-lhes em rosto a sua incredulidade e dureza de coração, por
não haverem crido nos que o tinham visto já ressuscitado. E
disse-lhes: \textcolor{red}{Ide por todo o mundo, pregai o evangelho a toda
criatura}. \textcolor{red}{Quem crer e for batizado será salvo; mas quem não
crer será condenado}. \textcolor{red}{E estes sinais seguirão aos que crerem:
Em meu nome expulsarão os demônios; falarão novas línguas};
\textcolor{red}{pegarão nas serpentes; e, se beberem alguma coisa mortífera,
não lhes fará dano algum; e porão as mãos sobre os enfermos, e os
curarão}.

Ora, o Senhor, depois de lhes ter falado, \textcolor{red}{foi recebido no céu, e
assentou-se à direita de Deus}. E eles, tendo partido,
pregaram por todas as partes, cooperando com eles o Senhor, e
confirmando a palavra com os sinais que se seguiram. Amém.
