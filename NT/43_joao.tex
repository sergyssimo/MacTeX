\addchap{O Evangelho de João}

\lettrine{1} No princípio era o Verbo, e o Verbo estava com
Deus, e o Verbo era Deus. Ele estava no princípio com Deus.
Todas as coisas foram feitas por ele, e sem ele nada do que foi
feito se fez. Nele estava a vida, e a vida era a luz dos homens.

E a luz resplandece nas trevas, e as trevas não a compreenderam.
Houve um homem enviado de Deus, cujo nome era João. Este
veio para testemunho, para que testificasse da luz, para que todos
cressem por ele. Não era ele a luz, mas para que testificasse da
luz. Ali estava a luz verdadeira, que ilumina a todo o homem que
vem ao mundo. Estava no mundo, e o mundo foi feito por ele, e
o mundo não o conheceu. Veio para o que era seu, e os seus
não o receberam. Mas, a todos quantos o receberam, deu-lhes o
poder de serem feitos filhos de Deus, aos que crêem no seu nome;
os quais não nasceram do sangue, nem da vontade da carne, nem
da vontade do homem, mas de Deus. E o Verbo se fez carne, e
habitou entre nós, e vimos a sua glória, como a glória do unigênito
do Pai, cheio de graça e de verdade.

João testificou dele, e clamou, dizendo: Este era aquele de quem
eu dizia: O que vem após mim é antes de mim, porque foi primeiro do
que eu. E todos nós recebemos também da sua plenitude, e
graça por graça. Porque a lei foi dada por Moisés; a graça e
a verdade vieram por Jesus Cristo. Deus nunca foi visto por
alguém. O Filho unigênito, que está no seio do Pai, esse o revelou.

E este é o testemunho de João, quando os judeus mandaram de
Jerusalém sacerdotes e levitas para que lhe perguntassem: Quem és
tu? E confessou, e não negou; confessou: Eu não sou o Cristo.
E perguntaram-lhe: Então quê? És tu Elias? E disse: Não sou.
És tu profeta? E respondeu: Não. Disseram-lhe pois: Quem és?
para que demos resposta àqueles que nos enviaram; que dizes de ti
mesmo? Disse: Eu sou a voz do que clama no deserto:
Endireitai o caminho do Senhor, como disse o profeta Isaías.
E os que tinham sido enviados eram dos fariseus. E
perguntaram-lhe, e disseram-lhe: Por que batizas, pois, se tu não és
o Cristo, nem Elias, nem o profeta? João respondeu-lhes,
dizendo: Eu batizo com água; mas no meio de vós está um a quem vós
não conheceis. Este é aquele que vem após mim, que é antes de
mim, do qual eu não sou digno de desatar a correia da alparca.
Estas coisas aconteceram em Betânia, do outro lado do Jordão,
onde João estava batizando.

No dia seguinte João viu a Jesus, que vinha para ele, e disse:
Eis o Cordeiro de Deus, que tira o pecado do mundo. Este é
aquele do qual eu disse: Após mim vem um homem que é antes de mim,
porque foi primeiro do que eu. E eu não o conhecia; mas, para
que ele fosse manifestado a Israel, vim eu, por isso, batizando com
água. E João testificou, dizendo: Eu vi o Espírito descer do
céu como pomba, e repousar sobre ele. E eu não o conhecia,
mas o que me mandou a batizar com água, esse me disse: Sobre aquele
que vires descer o Espírito, e sobre ele repousar, esse é o que
batiza com o Espírito Santo. E eu vi, e tenho testificado que
este é o Filho de Deus. No dia seguinte João estava outra vez
ali, e dois dos seus discípulos; e, vendo passar a Jesus,
disse: Eis aqui o Cordeiro de Deus.

E os dois discípulos ouviram-no dizer isto, e seguiram a Jesus.
E Jesus, voltando-se e vendo que eles o seguiam, disse-lhes:
Que buscais? E eles disseram: Rabi (que, traduzido, quer dizer
Mestre), onde moras? Ele lhes disse: Vinde, e vede. Foram, e
viram onde morava, e ficaram com ele aquele dia; e era já quase a
hora décima. Era André, irmão de Simão Pedro, um dos dois que
ouviram aquilo de João, e o haviam seguido. Este achou
primeiro a seu irmão Simão, e disse-lhe: Achamos o Messias (que,
traduzido, é o Cristo). E levou-o a Jesus. E, olhando Jesus
para ele, disse: Tu és Simão, filho de Jonas; tu serás chamado Cefas
(que quer dizer Pedro).

No dia seguinte quis Jesus ir à Galiléia, e achou a Filipe, e
disse-lhe: Segue-me. E Filipe era de Betsaida, cidade de
André e de Pedro. Filipe achou Natanael, e disse-lhe: Havemos
achado aquele de quem Moisés escreveu na lei, e os profetas: Jesus
de Nazaré, filho de José. Disse-lhe Natanael: Pode vir alguma
coisa boa de Nazaré? Disse-lhe Filipe: Vem, e vê. Jesus viu
Natanael vir ter com ele, e disse dele: Eis aqui um verdadeiro
israelita, em quem não há dolo. Disse-lhe Natanael: De onde
me conheces tu? Jesus respondeu, e disse-lhe: Antes que Filipe te
chamasse, te vi eu, estando tu debaixo da figueira. Natanael
respondeu, e disse-lhe: Rabi, tu és o Filho de Deus; tu és o Rei de
Israel. Jesus respondeu, e disse-lhe: Porque te disse: Vi-te
debaixo da figueira, crês? Coisas maiores do que estas verás.
E disse-lhe: Na verdade, na verdade vos digo que daqui em
diante vereis o céu aberto, e os anjos de Deus subindo e descendo
sobre o Filho do homem.

\medskip

\lettrine{2} E, ao terceiro dia, fizeram-se umas bodas em Caná
da Galiléia; e estava ali a mãe de Jesus. E foi também convidado
Jesus e os seus discípulos para as bodas. E, faltando vinho, a
mãe de Jesus lhe disse: Não têm vinho. Disse-lhe Jesus: Mulher,
que tenho eu contigo? Ainda não é chegada a minha hora. Sua mãe
disse aos serventes: Fazei tudo quanto ele vos disser. E estavam
ali postas seis talhas de pedra, para as purificações dos judeus, e
em cada uma cabiam dois ou três almudes\footnote{Antiga unidade de
medida de capacidade para líquidos, equivalente a 12 canadas (antiga
unidade de medida de capacidade para líquidos, equivalente a quatro
quartilhos, ou seja, 2,662 litros), ou seja, 31,94 litros.}.
Disse-lhes Jesus: Enchei de água essas talhas. E encheram-nas
até em cima. E disse-lhes: Tirai agora, e levai ao mestre-sala.
E levaram. E, logo que o mestre-sala\footnote{Empregado da casa
real que nas recepções do paço e noutros atos solenes dirigia o
cerimonial; mestre-de-cerimônias.} provou a água feita vinho (não
sabendo de onde viera, se bem que o sabiam os serventes que tinham
tirado a água), chamou o mestre-sala ao esposo, e disse-lhe:
Todo o homem põe primeiro o vinho bom e, quando já têm bebido bem,
então o inferior; mas tu guardaste até agora o bom vinho.
Jesus principiou assim os seus sinais em Caná da Galiléia, e
manifestou a sua glória; e os seus discípulos creram nele.

Depois disto desceu a Cafarnaum, ele, e sua mãe, e seus irmãos, e
seus discípulos; e ficaram ali não muitos dias. E estava
próxima a páscoa dos judeus, e Jesus subiu a Jerusalém. E
achou no templo os que vendiam bois, e ovelhas, e pombos, e os
cambiadores assentados. E tendo feito um azorrague de
cordéis, lançou todos fora do templo, também os bois e ovelhas; e
espalhou o dinheiro dos cambiadores, e derribou as mesas; e
disse aos que vendiam pombos: Tirai daqui estes, e não façais da
casa de meu Pai casa de venda. E os seus discípulos
lembraram-se do que está escrito: O zelo da tua casa me devorará.
Responderam, pois, os judeus, e disseram-lhe: Que sinal nos
mostras para fazeres isto? Jesus respondeu, e disse-lhes:
Derribai este templo, e em três dias o levantarei. Disseram,
pois, os judeus: Em quarenta e seis anos foi edificado este templo,
e tu o levantarás em três dias? Mas ele falava do templo do
seu corpo. Quando, pois, ressuscitou dentre os mortos, os
seus discípulos lembraram-se de que lhes dissera isto; e creram na
Escritura, e na palavra que Jesus tinha dito.

E, estando ele em Jerusalém pela páscoa, durante a festa, muitos,
vendo os sinais que fazia, creram no seu nome. Mas o mesmo
Jesus não confiava neles, porque a todos conhecia; e não
necessitava de que alguém testificasse do homem, porque ele bem
sabia o que havia no homem.

\medskip

\lettrine{3} E havia entre os fariseus um homem, chamado
Nicodemos, príncipe dos judeus. Este foi ter de noite com Jesus,
e disse-lhe: Rabi, bem sabemos que és Mestre, vindo de Deus; porque
ninguém pode fazer estes sinais que tu fazes, se Deus não for com
ele. Jesus respondeu, e disse-lhe: Na verdade, na verdade te
digo que aquele que não nascer de novo, não pode ver o reino de
Deus. Disse-lhe Nicodemos: Como pode um homem nascer, sendo
velho? Pode, porventura, tornar a entrar no ventre de sua mãe, e
nascer? Jesus respondeu: Na verdade, na verdade te digo que
aquele que não nascer da água e do Espírito, não pode entrar no
reino de Deus. O que é nascido da carne é carne, e o que é
nascido do Espírito é espírito. Não te maravilhes de te ter
dito: Necessário vos é nascer de novo. O vento assopra onde
quer, e ouves a sua voz, mas não sabes de onde vem, nem para onde
vai; assim é todo aquele que é nascido do Espírito. Nicodemos
respondeu, e disse-lhe: Como pode ser isso? Jesus respondeu,
e disse-lhe: Tu és mestre de Israel, e não sabes isto? Na
verdade, na verdade te digo que nós dizemos o que sabemos, e
testificamos o que vimos; e não aceitais o nosso testemunho.
Se vos falei de coisas terrestres, e não crestes, como
crereis, se vos falar das celestiais? Ora, ninguém subiu ao
céu, senão o que desceu do céu, o Filho do homem, que está no céu.
E, como Moisés levantou a serpente no deserto, assim importa
que o Filho do homem seja levantado; para que todo aquele que
nele crê não pereça, mas tenha a vida eterna. Porque Deus
amou o mundo de tal maneira que deu o seu Filho unigênito, para que
todo aquele que nele crê não pereça, mas tenha a vida eterna.
Porque Deus enviou o seu Filho ao mundo, não para que
condenasse o mundo, mas para que o mundo fosse salvo por ele.
Quem crê nele não é condenado; mas quem não crê já está
condenado, porquanto não crê no nome do unigênito Filho de Deus.
E a condenação é esta: Que a luz veio ao mundo, e os homens
amaram mais as trevas do que a luz, porque as suas obras eram más.
Porque todo aquele que faz o mal odeia a luz, e não vem para
a luz, para que as suas obras não sejam reprovadas. Mas quem
pratica a verdade vem para a luz, a fim de que as suas obras sejam
manifestas, porque são feitas em Deus.

Depois disto foi Jesus com os seus discípulos para a terra da
Judéia; e estava ali com eles, e batizava. Ora, João batizava
também em Enom, junto a Salim, porque havia ali muitas águas; e
vinham ali, e eram batizados. Porque ainda João não tinha
sido lançado na prisão. Houve então uma questão entre os
discípulos de João e os judeus acerca da purificação. E foram
ter com João, e disseram-lhe: Rabi, aquele que estava contigo além
do Jordão, do qual tu deste testemunho, ei-lo batizando, e todos vão
ter com ele. João respondeu, e disse: O homem não pode
receber coisa alguma, se não lhe for dada do céu. Vós mesmos
me sois testemunhas de que disse: Eu não sou o Cristo, mas sou
enviado adiante dele. Aquele que tem a esposa é o esposo; mas
o amigo do esposo, que lhe assiste e o ouve, alegra-se muito com a
voz do esposo. Assim, pois, já este meu gozo está cumprido. É
necessário que ele cresça e que eu diminua. Aquele que vem de
cima é sobre todos; aquele que vem da terra é da terra e fala da
terra. Aquele que vem do céu é sobre todos. E aquilo que ele
viu e ouviu isso testifica; e ninguém aceita o seu testemunho.
Aquele que aceitou o seu testemunho, esse confirmou que Deus
é verdadeiro. Porque aquele que Deus enviou fala as palavras
de Deus; pois não lhe dá Deus o Espírito por medida. O Pai
ama o Filho, e todas as coisas entregou nas suas mãos. Aquele
que crê no Filho tem a vida eterna; mas aquele que não crê no Filho
não verá a vida, mas a ira de Deus sobre ele permanece.

\medskip

\lettrine{4} E quando o Senhor entendeu que os fariseus tinham
ouvido que Jesus fazia e batizava mais discípulos do que João
 (ainda que Jesus mesmo não batizava, mas os seus discípulos),
deixou a Judéia, e foi outra vez para a Galiléia.

E era-lhe necessário passar por Samaria. Foi, pois, a uma
cidade de Samaria, chamada Sicar, junto da herdade que Jacó tinha
dado a seu filho José. E estava ali a fonte de Jacó. Jesus,
pois, cansado do caminho, assentou-se assim junto da fonte. Era isto
quase à hora sexta. Veio uma mulher de Samaria tirar água.
Disse-lhe Jesus: Dá-me de beber. Porque os seus discípulos
tinham ido à cidade comprar comida. Disse-lhe, pois, a mulher
samaritana: Como, sendo tu judeu, me pedes de beber a mim, que sou
mulher samaritana? (porque os judeus não se comunicam com os
samaritanos). Jesus respondeu, e disse-lhe: Se tu conheceras
o dom de Deus, e quem é o que te diz: Dá-me de beber, tu lhe
pedirias, e ele te daria água viva. Disse-lhe a mulher:
Senhor, tu não tens com que a tirar, e o poço é fundo; onde, pois,
tens a água viva? És tu maior do que o nosso pai Jacó, que
nos deu o poço, bebendo ele próprio dele, e os seus filhos, e o seu
gado? Jesus respondeu, e disse-lhe: Qualquer que beber desta
água tornará a ter sede; mas aquele que beber da água que eu
lhe der nunca terá sede, porque a água que eu lhe der se fará nele
uma fonte de água que salte para a vida eterna. Disse-lhe a
mulher: Senhor, dá-me dessa água, para que não mais tenha sede, e
não venha aqui tirá-la. Disse-lhe Jesus: Vai, chama o teu
marido, e vem cá. A mulher respondeu, e disse: Não tenho
marido. Disse-lhe Jesus: Disseste bem: Não tenho marido;
porque tiveste cinco maridos, e o que agora tens não é teu
marido; isto disseste com verdade. Disse-lhe a mulher:
Senhor, vejo que és profeta. Nossos pais adoraram neste
monte, e vós dizeis que é em Jerusalém o lugar onde se deve adorar.
Disse-lhe Jesus: Mulher, crê-me que a hora vem, em que nem
neste monte nem em Jerusalém adorareis o Pai. Vós adorais o
que não sabeis; nós adoramos o que sabemos porque a salvação vem dos
judeus. Mas a hora vem, e agora é, em que os verdadeiros
adoradores adorarão o Pai em espírito e em verdade; porque o Pai
procura a tais que assim o adorem. Deus é Espírito, e importa
que os que o adoram o adorem em espírito e em verdade. A
mulher disse-lhe: Eu sei que o Messias (que se chama o Cristo) vem;
quando ele vier, nos anunciará tudo. Jesus disse-lhe: Eu o
sou, eu que falo contigo.

E nisto vieram os seus discípulos, e maravilharam-se de que
estivesse falando com uma mulher; todavia nenhum lhe disse: Que
perguntas? ou: Por que falas com ela? Deixou, pois, a mulher
o seu cântaro, e foi à cidade, e disse àqueles homens: Vinde,
vede um homem que me disse tudo quanto tenho feito. Porventura não é
este o Cristo? Saíram, pois, da cidade, e foram ter com ele.
E entretanto os seus discípulos lhe rogaram, dizendo: Rabi,
come. Ele, porém, lhes disse: Uma comida tenho para comer,
que vós não conheceis. Então os discípulos diziam uns aos
outros: Trouxe-lhe, porventura, alguém algo de comer? Jesus
disse-lhes: A minha comida é fazer a vontade daquele que me enviou,
e realizar a sua obra. Não dizeis vós que ainda há quatro
meses até que venha a ceifa? Eis que eu vos digo: Levantai os vossos
olhos, e vede as terras, que já estão brancas para a ceifa. E
o que ceifa recebe galardão, e ajunta fruto para a vida eterna; para
que, assim o que semeia como o que ceifa, ambos se regozijem.
Porque nisto é verdadeiro o ditado, que um é o que semeia, e
outro o que ceifa. Eu vos enviei a ceifar onde vós não
trabalhastes; outros trabalharam, e vós entrastes no seu trabalho.
E muitos dos samaritanos daquela cidade creram nele, pela
palavra da mulher, que testificou: Disse-me tudo quanto tenho feito.
Indo, pois, ter com ele os samaritanos, rogaram-lhe que
ficasse com eles; e ficou ali dois dias. E muitos mais creram
nele, por causa da sua palavra. E diziam à mulher: Já não é
pelo teu dito que nós cremos; porque nós mesmos o temos ouvido, e
sabemos que este é verdadeiramente o Cristo, o Salvador do mundo.

E dois dias depois partiu dali, e foi para a Galiléia.
Porque Jesus mesmo testificou que um profeta não tem honra na
sua própria pátria. Chegando, pois, à Galiléia, os galileus o
receberam, vistas todas as coisas que fizera em Jerusalém, no dia da
festa; porque também eles tinham ido à festa. Segunda vez foi
Jesus a Caná da Galiléia, onde da água fizera vinho. E havia ali um
nobre, cujo filho estava enfermo em Cafarnaum. Ouvindo este
que Jesus vinha da Judéia para a Galiléia, foi ter com ele, e
rogou-lhe que descesse, e curasse o seu filho, porque já estava à
morte. Então Jesus lhe disse: Se não virdes sinais e
milagres, não crereis. Disse-lhe o nobre: Senhor, desce,
antes que meu filho morra. Disse-lhe Jesus: Vai, o teu filho
vive. E o homem creu na palavra que Jesus lhe disse, e partiu.
E descendo ele logo, saíram-lhe ao encontro os seus servos, e
lhe anunciaram, dizendo: O teu filho vive. Perguntou-lhes,
pois, a que hora se achara melhor. E disseram-lhe: Ontem às sete
horas a febre o deixou. Entendeu, pois, o pai que era aquela
hora a mesma em que Jesus lhe disse: O teu filho vive; e creu ele, e
toda a sua casa. Jesus fez este segundo milagre, quando ia da
Judéia para a Galiléia.

\medskip

\lettrine{5} Depois disto havia uma festa entre os judeus, e
Jesus subiu a Jerusalém. Ora, em Jerusalém há, próximo à porta
das ovelhas, um tanque, chamado em hebreu Betesda, o qual tem cinco
alpendres. Nestes jazia grande multidão de enfermos, cegos,
mancos e ressicados\footnote{Ressicar: tornar resseco, ressecado;
ressecar(-se), ressequir(-se).}, esperando o movimento da água.
Porquanto um anjo descia em certo tempo ao tanque, e agitava a
água; e o primeiro que ali descia, depois do movimento da água,
sarava de qualquer enfermidade que tivesse. E estava ali um
homem que, havia trinta e oito anos, se achava enfermo. E Jesus,
vendo este deitado, e sabendo que estava neste estado havia muito
tempo, disse-lhe: Queres ficar são? O enfermo respondeu-lhe:
Senhor, não tenho homem algum que, quando a água é agitada, me ponha
no tanque; mas, enquanto eu vou, desce outro antes de mim. Jesus
disse-lhe: Levanta-te, toma o teu leito, e anda. Logo aquele
homem ficou são; e tomou o seu leito, e andava. E aquele dia era
sábado. Então os judeus disseram àquele que tinha sido
curado: É sábado, não te é lícito levar o leito. Ele
respondeu-lhes: Aquele que me curou, ele próprio disse: Toma o teu
leito, e anda. Perguntaram-lhe, pois: Quem é o homem que te
disse: Toma o teu leito, e anda? E o que fora curado não
sabia quem era; porque Jesus se havia retirado, em razão de naquele
lugar haver grande multidão. Depois Jesus encontrou-o no
templo, e disse-lhe: Eis que já estás são; não peques mais, para que
não te suceda alguma coisa pior. E aquele homem foi, e
anunciou aos judeus que Jesus era o que o curara. E por esta
causa os judeus perseguiram a Jesus, e procuravam matá-lo, porque
fazia estas coisas no sábado.

E Jesus lhes respondeu: Meu Pai trabalha até agora, e eu trabalho
também. Por isso, pois, os judeus ainda mais procuravam
matá-lo, porque não só quebrantava o sábado, mas também dizia que
Deus era seu próprio Pai, fazendo-se igual a Deus. Mas Jesus
respondeu, e disse-lhes: Na verdade, na verdade vos digo que o Filho
por si mesmo não pode fazer coisa alguma, se o não vir fazer o Pai;
porque tudo quanto ele faz, o Filho o faz igualmente. Porque
o Pai ama o Filho, e mostra-lhe tudo o que faz; e ele lhe mostrará
maiores obras do que estas, para que vos maravilheis. Pois,
assim como o Pai ressuscita os mortos, e os vivifica, assim também o
Filho vivifica aqueles que quer. E também o Pai a ninguém
julga, mas deu ao Filho todo o juízo; para que todos honrem o
Filho, como honram o Pai. Quem não honra o Filho, não honra o Pai
que o enviou. Na verdade, na verdade vos digo que quem ouve a
minha palavra, e crê naquele que me enviou, tem a vida eterna, e não
entrará em condenação, mas passou da morte para a vida. Em
verdade, em verdade vos digo que vem a hora, e agora é, em que os
mortos ouvirão a voz do Filho de Deus, e os que a ouvirem viverão.
Porque, como o Pai tem a vida em si mesmo, assim deu também
ao Filho ter a vida em si mesmo; e deu-lhe o poder de exercer
o juízo, porque é o Filho do homem. Não vos maravilheis
disto; porque vem a hora em que todos os que estão nos sepulcros
ouvirão a sua voz. E os que fizeram o bem sairão para a
ressurreição da vida; e os que fizeram o mal para a ressurreição da
condenação. Eu não posso de mim mesmo fazer coisa alguma.
Como ouço, assim julgo; e o meu juízo é justo, porque não busco a
minha vontade, mas a vontade do Pai que me enviou.

Se eu testifico de mim mesmo, o meu testemunho não é verdadeiro.
Há outro que testifica de mim, e sei que o testemunho que ele
dá de mim é verdadeiro. Vós mandastes mensageiros a João, e
ele deu testemunho da verdade. Eu, porém, não recebo
testemunho de homem; mas digo isto, para que vos salveis. Ele
era a candeia que ardia e alumiava, e vós quisestes alegrar-vos por
um pouco de tempo com a sua luz. Mas eu tenho maior
testemunho do que o de João; porque as obras que o Pai me deu para
realizar, as mesmas obras que eu faço, testificam de mim, que o Pai
me enviou. E o Pai, que me enviou, ele mesmo testificou de
mim. Vós nunca ouvistes a sua voz, nem vistes o seu parecer.
E a sua palavra não permanece em vós, porque naquele que ele
enviou não credes vós. Examinais as Escrituras, porque vós
cuidais ter nelas a vida eterna, e são elas que de mim testificam;
e não quereis vir a mim para terdes vida. Eu não
recebo glória dos homens; mas bem vos conheço, que não tendes
em vós o amor de Deus. Eu vim em nome de meu Pai, e não me
aceitais; se outro vier em seu próprio nome, a esse aceitareis.
Como podeis vós crer, recebendo honra uns dos outros, e não
buscando a honra que vem só de Deus? Não cuideis que eu vos
hei de acusar para com o Pai. Há um que vos acusa, Moisés, em quem
vós esperais. Porque, se vós crêsseis em Moisés, creríeis em
mim; porque de mim escreveu ele. Mas, se não credes nos seus
escritos, como crereis nas minhas palavras?

\medskip

\lettrine{6} Depois disto partiu Jesus para o outro lado do
mar da Galiléia, que é o de Tiberíades. E grande multidão o
seguia, porque via os sinais que operava sobre os enfermos. E
Jesus subiu ao monte, e assentou-se ali com os seus discípulos.
E a páscoa, a festa dos judeus, estava próxima. Então Jesus,
levantando os olhos, e vendo que uma grande multidão vinha ter com
ele, disse a Filipe: Onde compraremos pão, para estes comerem?
Mas dizia isto para o experimentar; porque ele bem sabia o que
havia de fazer. Filipe respondeu-lhe: Duzentos
denários\footnote{SBTB: dinheiros} de pão não lhes bastarão, para
que cada um deles tome um pouco. E um dos seus discípulos,
André, irmão de Simão Pedro, disse-lhe: Está aqui um rapaz que
tem cinco pães de cevada e dois peixinhos; mas que é isto para
tantos? E disse Jesus: Mandai assentar os homens. E havia
muita relva naquele lugar. Assentaram-se, pois, os homens em número
de quase cinco mil. E Jesus tomou os pães e, havendo dado
graças, repartiu-os pelos discípulos, e os discípulos pelos que
estavam assentados; e igualmente também dos peixes, quanto eles
queriam. E, quando estavam saciados, disse aos seus
discípulos: Recolhei os pedaços que sobejaram, para que nada se
perca. Recolheram-nos, pois, e encheram doze alcofas de
pedaços dos cinco pães de cevada, que sobejaram aos que haviam
comido. Vendo, pois, aqueles homens o milagre que Jesus tinha
feito, diziam: Este é verdadeiramente o profeta que devia vir ao
mundo.

Sabendo, pois, Jesus que haviam de vir arrebatá-lo, para o
fazerem rei, tornou a retirar-se, ele só, para o monte. E,
quando veio a tarde, os seus discípulos desceram para o mar.
E, entrando no barco, atravessaram o mar em direção a
Cafarnaum; e era já escuro, e ainda Jesus não tinha chegado ao pé
deles. E o mar se levantou, porque um grande vento assoprava.
E, tendo navegado uns vinte e cinco ou trinta estádios, viram
a Jesus, andando sobre o mar e aproximando-se do barco; e temeram.
Mas ele lhes disse: Sou eu, não temais. Então eles de
boa mente o receberam no barco; e logo o barco chegou à terra para
onde iam.

No dia seguinte, a multidão que estava do outro lado do mar,
vendo que não havia ali mais do que um barquinho, a não ser aquele
no qual os discípulos haviam entrado, e que Jesus não entrara com os
seus discípulos naquele barquinho, mas que os seus discípulos tinham
ido sozinhos

outros barquinhos tinham chegado de Tiberíades, perto do
lugar onde comeram o pão, havendo o Senhor dado graças).
Vendo, pois, a multidão que Jesus não estava ali nem os seus
discípulos, entraram eles também nos barcos, e foram a Cafarnaum, em
busca de Jesus. E, achando-o no outro lado do mar,
disseram-lhe: Rabi, quando chegaste aqui? Jesus
respondeu-lhes, e disse: Na verdade, na verdade vos digo que me
buscais, não pelos sinais que vistes, mas porque comestes do pão e
vos saciastes. Trabalhai, não pela comida que perece, mas
pela comida que permanece para a vida eterna, a qual o Filho do
homem vos dará; porque a este o Pai, Deus, o selou.

Disseram-lhe, pois: Que faremos para executarmos as obras de
Deus? Jesus respondeu, e disse-lhes: A obra de Deus é esta:
Que creiais naquele que ele enviou. Disseram-lhe, pois: Que
sinal, pois, fazes tu, para que o vejamos, e creiamos em ti? Que
operas tu? Nossos pais comeram o maná no deserto, como está
escrito: Deu-lhes a comer o pão do céu. Disse-lhes, pois,
Jesus: Na verdade, na verdade vos digo: Moisés não vos deu o pão do
céu; mas meu Pai vos dá o verdadeiro pão do céu. Porque o pão
de Deus é aquele que desce do céu e dá vida ao mundo.
Disseram-lhe, pois: Senhor, dá-nos sempre desse pão. E
Jesus lhes disse: Eu sou o pão da vida; aquele que vem a mim não
terá fome, e quem crê em mim nunca terá sede. Mas já vos
disse que também vós me vistes, e contudo não credes. Todo o
que o Pai me dá virá a mim; e o que vem a mim de maneira nenhuma o
lançarei fora. Porque eu desci do céu, não para fazer a minha
vontade, mas a vontade daquele que me enviou. E a vontade do
Pai que me enviou é esta: Que nenhum de todos aqueles que me deu se
perca, mas que o ressuscite no último dia. Porquanto a
vontade daquele que me enviou é esta: Que todo aquele que vê o
Filho, e crê nele, tenha a vida eterna; e eu o ressuscitarei no
último dia. Murmuravam, pois, dele os judeus, porque dissera:
Eu sou o pão que desceu do céu. E diziam: Não é este Jesus, o
filho de José, cujo pai e mãe nós conhecemos? Como, pois, diz ele:
Desci do céu? Respondeu, pois, Jesus, e disse-lhes: Não
murmureis entre vós. Ninguém pode vir a mim, se o Pai que me
enviou o não trouxer; e eu o ressuscitarei no último dia.
Está escrito nos profetas: E serão todos ensinados por Deus.
Portanto, todo aquele que do Pai ouviu e aprendeu vem a mim.
Não que alguém visse ao Pai, a não ser aquele que é de Deus;
este tem visto ao Pai. Na verdade, na verdade vos digo que
aquele que crê em mim tem a vida eterna. Eu sou o pão da
vida. Vossos pais comeram o maná no deserto, e morreram.
Este é o pão que desce do céu, para que o que dele comer não
morra. Eu sou o pão vivo que desceu do céu; se alguém comer
deste pão, viverá para sempre; e o pão que eu der é a minha carne,
que eu darei pela vida do mundo. Disputavam, pois, os judeus
entre si, dizendo: Como nos pode dar este a sua carne a comer?
Jesus, pois, lhes disse: Na verdade, na verdade vos digo que,
se não comerdes a carne do Filho do homem, e não beberdes o seu
sangue, não tereis vida em vós mesmos. Quem come a minha
carne e bebe o meu sangue tem a vida eterna, e eu o ressuscitarei no
último dia. Porque a minha carne verdadeiramente é comida, e
o meu sangue verdadeiramente é bebida. Quem come a minha
carne e bebe o meu sangue permanece em mim e eu nele. Assim
como o Pai, que vive, me enviou, e eu vivo pelo Pai, assim, quem de
mim se alimenta, também viverá por mim. Este é o pão que
desceu do céu; não é o caso de vossos pais, que comeram o maná e
morreram; quem comer este pão viverá para sempre. Ele disse
estas coisas na sinagoga, ensinando em Cafarnaum.

Muitos, pois, dos seus discípulos, ouvindo isto, disseram: Duro é
este discurso; quem o pode ouvir? Sabendo, pois, Jesus em si
mesmo que os seus discípulos murmuravam disto, disse-lhes: Isto
escandaliza-vos? Que seria, pois, se vísseis subir o Filho do
homem para onde primeiro estava? O espírito é o que vivifica,
a carne para nada aproveita; as palavras que eu vos disse são
espírito e vida. Mas há alguns de vós que não crêem. Porque
bem sabia Jesus, desde o princípio, quem eram os que não criam, e
quem era o que o havia de entregar. E dizia: Por isso eu vos
disse que ninguém pode vir a mim, se por meu Pai não lhe for
concedido. Desde então muitos dos seus discípulos tornaram
para trás, e já não andavam com ele. Então disse Jesus aos
doze: Quereis vós também retirar-vos? Respondeu-lhe, pois,
Simão Pedro: Senhor, para quem iremos nós? Tu tens as palavras da
vida eterna. E nós temos crido e conhecido que tu és o
Cristo, o Filho do Deus vivente. Respondeu-lhe Jesus: Não vos
escolhi a vós os doze? e um de vós é um diabo. E isto dizia
ele de Judas Iscariotes, filho de Simão; porque este o havia de
entregar, sendo um dos doze.

\medskip

\lettrine{7} E depois disto Jesus andava pela Galiléia, e já
não queria andar pela Judéia, pois os judeus procuravam matá-lo.
E estava próxima a festa dos judeus, a dos tabernáculos.
Disseram-lhe, pois, seus irmãos: Sai daqui, e vai para a Judéia,
para que também os teus discípulos vejam as obras que fazes.
Porque não há ninguém que procure ser conhecido que faça coisa
alguma em oculto. Se fazes estas coisas, manifesta-te ao mundo.
Porque nem mesmo seus irmãos criam nele. Disse-lhes, pois,
Jesus: Ainda não é chegado o meu tempo, mas o vosso tempo sempre
está pronto. O mundo não vos pode odiar, mas ele me odeia a mim,
porquanto dele testifico que as suas obras são más. Subi vós a
esta festa; eu não subo ainda a esta festa, porque ainda o meu tempo
não está cumprido. E, havendo-lhes dito isto, ficou na Galiléia.
Mas, quando seus irmãos já tinham subido à festa, então subiu
ele também, não manifestamente, mas como em oculto. Ora, os
judeus procuravam-no na festa, e diziam: Onde está ele? E
havia grande murmuração entre a multidão a respeito dele. Diziam
alguns: Ele é bom. E outros diziam: Não, antes engana o povo.
Todavia ninguém falava dele abertamente, por medo dos judeus.

Mas, no meio da festa subiu Jesus ao templo, e ensinava. E
os judeus maravilhavam-se, dizendo: Como sabe este letras, não as
tendo aprendido? Jesus lhes respondeu, e disse: A minha
doutrina não é minha, mas daquele que me enviou. Se alguém
quiser fazer a vontade dele, pela mesma doutrina conhecerá se ela é
de Deus, ou se eu falo de mim mesmo. Quem fala de si mesmo
busca a sua própria glória; mas o que busca a glória daquele que o
enviou, esse é verdadeiro, e não há nele injustiça. Não vos
deu Moisés a lei? e nenhum de vós observa a lei. Por que procurais
matar-me? A multidão respondeu, e disse: Tens demônio; quem
procura matar-te? Respondeu Jesus, e disse-lhes: Fiz uma só
obra, e todos vos maravilhais. Pelo motivo de que Moisés vos
deu a circuncisão (não que fosse de Moisés, mas dos pais), no sábado
circuncidais um homem. Se o homem recebe a circuncisão no
sábado, para que a lei de Moisés não seja quebrantada, indignais-vos
contra mim, porque no sábado curei de todo um homem? Não
julgueis segundo a aparência, mas julgai segundo a reta justiça.
Então alguns dos de Jerusalém diziam: Não é este o que
procuram matar? E ei-lo aí está falando abertamente, e nada
lhe dizem. Porventura sabem verdadeiramente os príncipes que de fato
este é o Cristo? Todavia bem sabemos de onde este é; mas,
quando vier o Cristo, ninguém saberá de onde ele é. Clamava,
pois, Jesus no templo, ensinando, e dizendo: Vós conheceis-me, e
sabeis de onde sou; e eu não vim de mim mesmo, mas aquele que me
enviou é verdadeiro, o qual vós não conheceis. Mas eu
conheço-o, porque dele sou e ele me enviou. Procuravam, pois,
prendê-lo, mas ninguém lançou mão dele, porque ainda não era chegada
a sua hora. E muitos da multidão creram nele, e diziam:
Quando o Cristo vier, fará ainda mais sinais do que os que este tem
feito? Os fariseus ouviram que a multidão murmurava dele
estas coisas; e os fariseus e os principais dos sacerdotes mandaram
servidores para o prenderem. Disse-lhes, pois, Jesus: Ainda
um pouco de tempo estou convosco, e depois vou para aquele que me
enviou. Vós me buscareis, e não me achareis; e onde eu estou,
vós não podeis vir. Disseram, pois, os judeus uns para os
outros: Para onde irá este, que o não acharemos? Irá porventura para
os dispersos entre os gregos, e ensinará os gregos? Que
palavra é esta que disse: Buscar-me-eis, e não me achareis; e: Aonde
eu estou vós não podeis ir?

E no último dia, o grande dia da festa, Jesus pôs-se em pé, e
clamou, dizendo: Se alguém tem sede, venha a mim, e beba.
Quem crê em mim, como diz a Escritura, rios de água viva
correrão do seu ventre. E isto disse ele do Espírito que
haviam de receber os que nele cressem; porque o Espírito Santo ainda
não fora dado, por ainda Jesus não ter sido glorificado.
Então muitos da multidão, ouvindo esta palavra, diziam:
Verdadeiramente este é o Profeta. Outros diziam: Este é o
Cristo; mas diziam outros: Vem, pois, o Cristo da Galiléia?
Não diz a Escritura que o Cristo vem da descendência de Davi,
e de Belém, da aldeia de onde era Davi? Assim entre o povo
havia dissensão por causa dele. E alguns deles queriam
prendê-lo, mas ninguém lançou mão dele.

E os servidores foram ter com os principais dos sacerdotes e
fariseus; e eles lhes perguntaram: Por que não o trouxestes?
Responderam os servidores: Nunca homem algum falou assim como
este homem. Responderam-lhes, pois, os fariseus: Também vós
fostes enganados? Creu nele porventura algum dos principais
ou dos fariseus? Mas esta multidão, que não sabe a lei, é
maldita. Nicodemos, que era um deles (o que de noite fora ter
com Jesus), disse-lhes: Porventura condena a nossa lei um
homem sem primeiro o ouvir e ter conhecimento do que faz?
Responderam eles, e disseram-lhe: És tu também da Galiléia?
Examina, e verás que da Galiléia nenhum profeta surgiu. E
cada um foi para sua casa.

\medskip

\lettrine{8} Jesus, porém, foi para o Monte das Oliveiras.
E pela manhã cedo tornou para o templo, e todo o povo vinha ter
com ele, e, assentando-se, os ensinava. E os escribas e fariseus
trouxeram-lhe uma mulher apanhada em adultério; e, pondo-a no
meio, disseram-lhe: Mestre, esta mulher foi apanhada, no próprio
ato, adulterando. E na lei nos mandou Moisés que as tais sejam
apedrejadas. Tu, pois, que dizes? Isto diziam eles, tentando-o,
para que tivessem de que o acusar. Mas Jesus, inclinando-se,
escrevia com o dedo na terra. E, como insistissem,
perguntando-lhe, endireitou-se, e disse-lhes: Aquele que de entre
vós está sem pecado seja o primeiro que atire pedra contra ela.
E, tornando a inclinar-se, escrevia na terra. Quando ouviram
isto, redargüidos\footnote{Redargüir: replicar argumentando;
responder argüindo; replicar. Acusar; recriminar.} da consciência,
saíram um a um, a começar pelos mais velhos até aos últimos; ficou
só Jesus e a mulher que estava no meio. E, endireitando-se
Jesus, e não vendo ninguém mais do que a mulher, disse-lhe: Mulher,
onde estão aqueles teus acusadores? Ninguém te condenou? E
ela disse: Ninguém, Senhor. E disse-lhe Jesus: Nem eu também te
condeno; vai-te, e não peques mais.

Falou-lhes, pois, Jesus outra vez, dizendo: Eu sou a luz do
mundo; quem me segue não andará em trevas, mas terá a luz da vida.
Disseram-lhe, pois, os fariseus: Tu testificas de ti mesmo; o
teu testemunho não é verdadeiro. Respondeu Jesus, e
disse-lhes: Ainda que eu testifico de mim mesmo, o meu testemunho é
verdadeiro, porque sei de onde vim, e para onde vou; mas vós não
sabeis de onde venho, nem para onde vou. Vós julgais segundo
a carne; eu a ninguém julgo. E, se na verdade julgo, o meu
juízo é verdadeiro, porque não sou eu só, mas eu e o Pai que me
enviou. E na vossa lei está também escrito que o testemunho
de dois homens é verdadeiro. Eu sou o que testifico de mim
mesmo, e de mim testifica também o Pai que me enviou.
Disseram-lhe, pois: Onde está teu Pai? Jesus respondeu: Não
me conheceis a mim, nem a meu Pai; se vós me conhecêsseis a mim,
também conheceríeis a meu Pai. Estas palavras disse Jesus no
lugar do tesouro, ensinando no templo, e ninguém o prendeu, porque
ainda não era chegada a sua hora.

Disse-lhes, pois, Jesus outra vez: Eu retiro-me, e buscar-me-eis,
e morrereis no vosso pecado. Para onde eu vou, não podeis vós vir.
Diziam, pois, os judeus: Porventura quererá matar-se a si
mesmo, pois diz: Para onde eu vou não podeis vir? E
dizia-lhes: Vós sois de baixo, eu sou de cima; vós sois deste mundo,
eu não sou deste mundo. Por isso vos disse que morrereis em
vossos pecados, porque se não crerdes que eu sou, morrereis em
vossos pecados. Disseram-lhe, pois: Quem és tu? Jesus lhes
disse: Isso mesmo que já desde o princípio vos disse. Muito
tenho que dizer e julgar de vós, mas aquele que me enviou é
verdadeiro; e o que dele tenho ouvido, isso falo ao mundo.
Mas não entenderam que ele lhes falava do Pai.
Disse-lhes, pois, Jesus: Quando levantardes o Filho do homem,
então conhecereis quem eu sou, e que nada faço por mim mesmo; mas
falo como meu Pai me ensinou. E aquele que me enviou está
comigo. O Pai não me tem deixado só, porque eu faço sempre o que lhe
agrada. Dizendo ele estas coisas, muitos creram nele.

Jesus dizia, pois, aos judeus que criam nele: Se vós
permanecerdes na minha palavra, verdadeiramente sereis meus
discípulos; e conhecereis a verdade, e a verdade vos
libertará. Responderam-lhe: Somos descendência de Abraão, e
nunca servimos a ninguém; como dizes tu: Sereis livres?
Respondeu-lhes Jesus: Em verdade, em verdade vos digo que
todo aquele que comete pecado é servo do pecado. Ora o servo
não fica para sempre em casa; o Filho fica para sempre. Se,
pois, o Filho vos libertar, verdadeiramente sereis livres.
Bem sei que sois descendência de Abraão; contudo, procurais
matar-me, porque a minha palavra não entra em vós.

Eu falo do que vi junto de meu Pai, e vós fazeis o que também
vistes junto de vosso pai. Responderam, e disseram-lhe: Nosso
pai é Abraão. Jesus disse-lhes: Se fôsseis filhos de Abraão, faríeis
as obras de Abraão. Mas agora procurais matar-me, a mim,
homem que vos tem dito a verdade que de Deus tem ouvido; Abraão não
fez isto. Vós fazeis as obras de vosso pai. Disseram-lhe,
pois: Nós não somos nascidos de prostituição; temos um Pai, que é
Deus. Disse-lhes, pois, Jesus: Se Deus fosse o vosso Pai,
certamente me amaríeis, pois que eu saí, e vim de Deus; não vim de
mim mesmo, mas ele me enviou. Por que não entendeis a minha
linguagem? Por não poderdes ouvir a minha palavra. Vós tendes
por pai ao diabo, e quereis satisfazer os desejos de vosso pai. Ele
foi homicida desde o princípio, e não se firmou na verdade, porque
não há verdade nele. Quando ele profere mentira, fala do que lhe é
próprio, porque é mentiroso, e pai da mentira. Mas, porque
vos digo a verdade, não me credes.

Quem dentre vós me convence de pecado? E se vos digo a verdade,
por que não credes? Quem é de Deus escuta as palavras de
Deus; por isso vós não as escutais, porque não sois de Deus.
Responderam, pois, os judeus, e disseram-lhe: Não dizemos nós
bem que és samaritano, e que tens demônio? Jesus respondeu:
Eu não tenho demônio, antes honro a meu Pai, e vós me desonrais.
Eu não busco a minha glória; há quem a busque, e julgue.

Em verdade, em verdade vos digo que, se alguém guardar a minha
palavra, nunca verá a morte. Disseram-lhe, pois, os judeus:
Agora conhecemos que tens demônio. Morreu Abraão e os profetas; e tu
dizes: Se alguém guardar a minha palavra, nunca provará a morte.
És tu maior do que o nosso pai Abraão, que morreu? E também
os profetas morreram. Quem te fazes tu ser? Jesus respondeu:
Se eu me glorifico a mim mesmo, a minha glória não é nada; quem me
glorifica é meu Pai, o qual dizeis que é vosso Deus. E vós
não o conheceis, mas eu conheço-o. E, se disser que o não conheço,
serei mentiroso como vós; mas conheço-o e guardo a sua palavra.
Abraão, vosso pai, exultou por ver o meu dia, e viu-o, e
alegrou-se. Disseram-lhe, pois, os judeus: Ainda não tens
cinqüenta anos, e viste Abraão? Disse-lhes Jesus: Em verdade,
em verdade vos digo que antes que Abraão existisse, eu sou.
Então pegaram em pedras para lhe atirarem; mas Jesus
ocultou-se, e saiu do templo, passando pelo meio deles, e assim se
retirou.

\medskip

\lettrine{9} E, passando Jesus, viu um homem cego de nascença.
E os seus discípulos lhe perguntaram, dizendo: Rabi, quem pecou,
este ou seus pais, para que nascesse cego? Jesus respondeu: Nem
ele pecou nem seus pais; mas foi assim para que se manifestem nele
as obras de Deus. Convém que eu faça as obras daquele que me
enviou, enquanto é dia; a noite vem, quando ninguém pode trabalhar.
Enquanto estou no mundo, sou a luz do mundo. Tendo dito
isto, cuspiu na terra, e com a saliva fez lodo, e untou com o lodo
os olhos do cego. E disse-lhe: Vai, lava-te no tanque de Siloé
(que significa o Enviado). Foi, pois, e lavou-se, e voltou vendo.

Então os vizinhos, e aqueles que dantes tinham visto que era cego,
diziam: Não é este aquele que estava assentado e mendigava? Uns
diziam: É este. E outros: Parece-se com ele. Ele dizia: Sou eu.
Diziam-lhe, pois: Como se te abriram os olhos? Ele
respondeu, e disse: O homem, chamado Jesus, fez lodo, e untou-me os
olhos, e disse-me: Vai ao tanque de Siloé, e lava-te. Então fui, e
lavei-me, e vi. Disseram-lhe, pois: Onde está ele? Respondeu:
Não sei.

Levaram, pois, aos fariseus o que dantes era cego. E era
sábado quando Jesus fez o lodo e lhe abriu os olhos.
Tornaram, pois, também os fariseus a perguntar-lhe como vira,
e ele lhes disse: Pôs-me lodo sobre os olhos, lavei-me, e vejo.
Então alguns dos fariseus diziam: Este homem não é de Deus,
pois não guarda o sábado. Diziam outros: Como pode um homem pecador
fazer tais sinais? E havia dissensão entre eles. Tornaram,
pois, a dizer ao cego: Tu, que dizes daquele que te abriu os olhos?
E ele respondeu: Que é profeta. Os judeus, porém, não creram
que ele tivesse sido cego, e que agora visse, enquanto não chamaram
os pais do que agora via. E perguntaram-lhes, dizendo: É este
o vosso filho, que vós dizeis ter nascido cego? Como, pois, vê
agora? Seus pais lhes responderam, e disseram: Sabemos que
este é o nosso filho, e que nasceu cego; mas como agora vê,
não sabemos; ou quem lhe tenha aberto os olhos, não sabemos. Tem
idade, perguntai-lho a ele mesmo; e ele falará por si mesmo.
Seus pais disseram isto, porque temiam os judeus. Porquanto
já os judeus tinham resolvido que, se alguém confessasse ser ele o
Cristo, fosse expulso da sinagoga. Por isso é que seus pais
disseram: Tem idade, perguntai-lho a ele mesmo. Chamaram,
pois, pela segunda vez o homem que tinha sido cego, e disseram-lhe:
Dá glória a Deus; nós sabemos que esse homem é pecador.
Respondeu ele pois, e disse: Se é pecador, não sei; uma coisa
sei, é que, havendo eu sido cego, agora vejo. E tornaram a
dizer-lhe: Que te fez ele? Como te abriu os olhos?
Respondeu-lhes: Já vo-lo disse, e não ouvistes; para que o
quereis tornar a ouvir? Quereis vós porventura fazer-vos também seus
discípulos? Então o injuriaram, e disseram: Discípulo dele
sejas tu; nós, porém, somos discípulos de Moisés. Nós bem
sabemos que Deus falou a Moisés, mas este não sabemos de onde é.
O homem respondeu, e disse-lhes: Nisto, pois, está a
maravilha, que vós não saibais de onde ele é, e contudo me abrisse
os olhos. Ora, nós sabemos que Deus não ouve a pecadores;
mas, se alguém é temente a Deus, e faz a sua vontade, a esse ouve.
Desde o princípio do mundo nunca se ouviu que alguém abrisse
os olhos a um cego de nascença. Se este não fosse de Deus,
nada poderia fazer. Responderam eles, e disseram-lhe: Tu és
nascido todo em pecados, e nos ensinas a nós? E expulsaram-no.

Jesus ouviu que o tinham expulsado e, encontrando-o, disse-lhe:
Crês tu no Filho de Deus? Ele respondeu, e disse: Quem é ele,
Senhor, para que nele creia? E Jesus lhe disse: Tu já o tens
visto, e é aquele que fala contigo. Ele disse: Creio, Senhor.
E o adorou.

E disse-lhe Jesus: Eu vim a este mundo para juízo, a fim de que
os que não vêem vejam, e os que vêem sejam cegos. E aqueles
dos fariseus, que estavam com ele, ouvindo isto, disseram-lhe:
Também nós somos cegos? Disse-lhes Jesus: Se fôsseis cegos,
não teríeis pecado; mas como agora dizeis: Vemos; por isso o vosso
pecado permanece.

\medskip

\lettrine{10} Na verdade, na verdade vos digo que aquele que
não entra pela porta no curral das ovelhas, mas sobe por outra
parte, é ladrão e salteador. Aquele, porém, que entra pela porta
é o pastor das ovelhas. A este o porteiro abre, e as ovelhas
ouvem a sua voz, e chama pelo nome às suas ovelhas, e as traz para
fora. E, quando tira para fora as suas ovelhas, vai adiante
delas, e as ovelhas o seguem, porque conhecem a sua voz. Mas de
modo nenhum seguirão o estranho, antes fugirão dele, porque não
conhecem a voz dos estranhos. Jesus disse-lhes esta parábola;
mas eles não entenderam o que era que lhes dizia. Tornou, pois,
Jesus a dizer-lhes: Em verdade, em verdade vos digo que eu sou a
porta das ovelhas. Todos quantos vieram antes de mim são ladrões
e salteadores; mas as ovelhas não os ouviram. Eu sou a porta; se
alguém entrar por mim, salvar-se-á, e entrará, e sairá, e achará
pastagens. O ladrão não vem senão a roubar, a matar, e a
destruir; eu vim para que tenham vida, e a tenham com abundância.
Eu sou o bom Pastor; o bom Pastor dá a sua vida pelas
ovelhas. Mas o mercenário, e o que não é pastor, de quem não
são as ovelhas, vê vir o lobo, e deixa as ovelhas, e foge; e o lobo
as arrebata e dispersa as ovelhas. Ora, o mercenário foge,
porque é mercenário, e não tem cuidado das ovelhas. Eu sou o
bom Pastor, e conheço as minhas ovelhas, e das minhas sou conhecido.
Assim como o Pai me conhece a mim, também eu conheço o Pai, e
dou a minha vida pelas ovelhas. Ainda tenho outras ovelhas
que não são deste aprisco; também me convém agregar estas, e elas
ouvirão a minha voz, e haverá um rebanho e um Pastor. Por
isto o Pai me ama, porque dou a minha vida para tornar a tomá-la.
Ninguém ma tira de mim, mas eu de mim mesmo a dou; tenho
poder para a dar, e poder para tornar a tomá-la. Este mandamento
recebi de meu Pai.

Tornou, pois, a haver divisão entre os judeus por causa destas
palavras. E muitos deles diziam: Tem demônio, e está fora de
si; por que o ouvis? Diziam outros: Estas palavras não são de
endemoninhado. Pode, porventura, um demônio abrir os olhos aos
cegos?

E em Jerusalém havia a festa da dedicação, e era inverno.
E Jesus andava passeando no templo, no alpendre de Salomão.
Rodearam-no, pois, os judeus, e disseram-lhe: Até quando
terás a nossa alma suspensa? Se tu és o Cristo, dize-no-lo
abertamente. Respondeu-lhes Jesus: Já vo-lo tenho dito, e não
o credes. As obras que eu faço, em nome de meu Pai, essas testificam
de mim. Mas vós não credes porque não sois das minhas
ovelhas, como já vo-lo tenho dito. As minhas ovelhas ouvem a
minha voz, e eu conheço-as, e elas me seguem; e dou-lhes a
vida eterna, e nunca hão de perecer, e ninguém as arrebatará da
minha mão. Meu Pai, que mas deu, é maior do que todos; e
ninguém pode arrebatá-las da mão de meu Pai. Eu e o Pai somos
um. Os judeus pegaram então outra vez em pedras para o
apedrejar. Respondeu-lhes Jesus: Tenho-vos mostrado muitas
obras boas procedentes de meu Pai; por qual destas obras me
apedrejais? Os judeus responderam, dizendo-lhe: Não te
apedrejamos por alguma obra boa, mas pela blasfêmia; porque, sendo
tu homem, te fazes Deus a ti mesmo. Respondeu-lhes Jesus: Não
está escrito na vossa lei: Eu disse: Sois deuses? Pois, se a
lei chamou deuses àqueles a quem a palavra de Deus foi dirigida (e a
Escritura não pode ser anulada), àquele a quem o Pai
santificou, e enviou ao mundo, vós dizeis: Blasfemas, porque disse:
Sou Filho de Deus? Se não faço as obras de meu Pai, não me
acrediteis. Mas, se as faço, e não credes em mim, crede nas
obras; para que conheçais e acrediteis que o Pai está em mim e eu
nele.

Procuravam, pois, prendê-lo outra vez, mas ele escapou-se de suas
mãos, e retirou-se outra vez para além do Jordão, para o
lugar onde João tinha primeiramente batizado; e ali ficou. E
muitos iam ter com ele, e diziam: Na verdade João não fez sinal
algum, mas tudo quanto João disse deste era verdade. E muitos
ali creram nele.

\medskip

\lettrine{11} Estava, porém, enfermo um certo Lázaro, de
Betânia, aldeia de Maria e de sua irmã Marta. E Maria era aquela
que tinha ungido o Senhor com ungüento, e lhe tinha enxugado os pés
com os seus cabelos, cujo irmão Lázaro estava enfermo.
Mandaram-lhe, pois, suas irmãs dizer: Senhor, eis que está
enfermo aquele que tu amas. E Jesus, ouvindo isto, disse: Esta
enfermidade não é para morte, mas para glória de Deus, para que o
Filho de Deus seja glorificado por ela. Ora, Jesus amava a
Marta, e a sua irmã, e a Lázaro. Ouvindo, pois, que estava
enfermo, ficou ainda dois dias no lugar onde estava. Depois
disto, disse aos seus discípulos: Vamos outra vez para a Judéia.
Disseram-lhe os discípulos: Rabi, ainda agora os judeus
procuravam apedrejar-te, e tornas para lá? Jesus respondeu: Não
há doze horas no dia? Se alguém andar de dia, não tropeça, porque vê
a luz deste mundo; mas, se andar de noite, tropeça, porque
nele não há luz. Assim falou; e depois disse-lhes: Lázaro, o
nosso amigo, dorme, mas vou despertá-lo do sono. Disseram,
pois, os seus discípulos: Senhor, se dorme, estará salvo. Mas
Jesus dizia isto da sua morte; eles, porém, cuidavam que falava do
repouso do sono. Então Jesus disse-lhes claramente: Lázaro
está morto; e folgo, por amor de vós, de que eu lá não
estivesse, para que acrediteis; mas vamos ter com ele. Disse,
pois, Tomé, chamado Dídimo, aos condiscípulos: Vamos nós também,
para morrermos com ele.

Chegando, pois, Jesus, achou que já havia quatro dias que estava
na sepultura.

Betânia distava de Jerusalém quase quinze estádios.) E
muitos dos judeus tinham ido consolar a Marta e a Maria, acerca de
seu irmão. Ouvindo, pois, Marta que Jesus vinha, saiu-lhe ao
encontro; Maria, porém, ficou assentada em casa. Disse, pois,
Marta a Jesus: Senhor, se tu estivesses aqui, meu irmão não teria
morrido. Mas também agora sei que tudo quanto pedires a Deus,
Deus to concederá. Disse-lhe Jesus: Teu irmão há de
ressuscitar. Disse-lhe Marta: Eu sei que há de ressuscitar na
ressurreição do último dia. Disse-lhe Jesus: Eu sou a
ressurreição e a vida; quem crê em mim, ainda que esteja morto,
viverá; e todo aquele que vive, e crê em mim, nunca morrerá.
Crês tu isto? Disse-lhe ela: Sim, Senhor, creio que tu és o
Cristo, o Filho de Deus, que havia de vir ao mundo. E, dito
isto, partiu, e chamou em segredo a Maria, sua irmã, dizendo: O
Mestre está aqui, e te chama.\footnote{SBTB: está cá, e chama-te.
KJ: The Master is come, and calleth for thee. RA: O Mestre chegou e
te chama. RC: O Mestre está aqui e chama-te.} Ela, ouvindo
isto, levantou-se logo, e foi ter com ele. (Ainda Jesus não
tinha chegado à aldeia, mas estava no lugar onde Marta o
encontrara.) Vendo, pois, os judeus, que estavam com ela em
casa e a consolavam, que Maria apressadamente se levantara e saíra,
seguiram-na, dizendo: Vai ao sepulcro para chorar ali. Tendo,
pois, Maria chegado aonde Jesus estava, e vendo-o, lançou-se aos
seus pés, dizendo-lhe: Senhor, se tu estivesses aqui, meu irmão não
teria morrido.

Jesus pois, quando a viu chorar, e também chorando os judeus que
com ela vinham, moveu-se muito em espírito, e perturbou-se. E
disse: Onde o pusestes? Disseram-lhe: Senhor, vem, e vê.
Jesus chorou. Disseram, pois, os judeus: Vede como o
amava. E alguns deles disseram: Não podia ele, que abriu os
olhos ao cego, fazer também com que este não morresse? Jesus,
pois, movendo-se outra vez muito em si mesmo, veio ao sepulcro; e
era uma caverna, e tinha uma pedra posta sobre ela. Disse
Jesus: Tirai a pedra. Marta, irmã do defunto, disse-lhe: Senhor, já
cheira mal, porque é já de quatro dias. Disse-lhe Jesus: Não
te hei dito que, se creres, verás a glória de Deus? Tiraram,
pois, a pedra de onde o defunto jazia. E Jesus, levantando os olhos
para cima, disse: Pai, graças te dou, por me haveres ouvido.
Eu bem sei que sempre me ouves, mas eu disse isto por causa
da multidão que está em redor, para que creiam que tu me enviaste.
E, tendo dito isto, clamou com grande voz: Lázaro, vem para
fora\footnote{SBTB: sai para fora. KJ: Lazarus, come forth.}.
E o defunto saiu, tendo as mãos e os pés ligados com faixas,
e o seu rosto envolto num lenço. Disse-lhes Jesus: Desligai-o, e
deixai-o ir.

Muitos, pois, dentre os judeus que tinham vindo a Maria, e que
tinham visto o que Jesus fizera, creram nele. Mas alguns
deles foram ter com os fariseus, e disseram-lhes o que Jesus tinha
feito. Depois os principais dos sacerdotes e os fariseus
formaram conselho, e diziam: Que faremos? porquanto este homem faz
muitos sinais. Se o deixamos assim, todos crerão nele, e
virão os romanos, e tirar-nos-ão o nosso lugar e a nação. E
Caifás, um deles que era sumo sacerdote naquele ano, lhes disse: Vós
nada sabeis, nem considerais que nos convém que um homem
morra pelo povo, e que não pereça toda a nação. Ora ele não
disse isto de si mesmo, mas, sendo o sumo sacerdote naquele ano,
profetizou que Jesus devia morrer pela nação. E não somente
pela nação, mas também para reunir em um corpo os filhos de Deus que
andavam dispersos. Desde aquele dia, pois, consultavam-se
para o matarem. Jesus, pois, já não andava manifestamente
entre os judeus, mas retirou-se dali para a terra junto do deserto,
para uma cidade chamada Efraim; e ali ficou com os seus discípulos.
E estava próxima a páscoa dos judeus, e muitos daquela região
subiram a Jerusalém antes da páscoa para se purificarem.
Buscavam, pois, a Jesus, e diziam uns aos outros, estando no
templo: Que vos parece? Não virá à festa? Ora, os principais
dos sacerdotes e os fariseus tinham dado ordem para que, se alguém
soubesse onde ele estava, o denunciasse, para o prenderem.

\medskip

\lettrine{12} Foi, pois, Jesus seis dias antes da páscoa a
Betânia, onde estava Lázaro, o que falecera, e a quem ressuscitara
dentre os mortos. Fizeram-lhe, pois, ali uma ceia, e Marta
servia, e Lázaro era um dos que estavam à mesa com ele. Então
Maria, tomando um arrátel\footnote{Antiga unidade de medida de peso,
equivalente a 459g ou 16 onças; libra.} de ungüento de nardo puro,
de muito preço, ungiu os pés de Jesus, e enxugou-lhe os pés com os
seus cabelos; e encheu-se a casa do cheiro do ungüento. Então,
um dos seus discípulos, Judas Iscariotes, filho de Simão, o que
havia de traí-lo, disse: Por que não se vendeu este ungüento por
trezentos denários\footnote{SBTB: dinheiros.} e não se deu aos
pobres? Ora, ele disse isto, não pelo cuidado que tivesse dos
pobres, mas porque era ladrão e tinha a bolsa, e tirava o que ali se
lançava. Disse, pois, Jesus: Deixai-a; para o dia da minha
sepultura guardou isto; porque os pobres sempre os tendes
convosco, mas a mim nem sempre me tendes. E muita gente dos
judeus soube que ele estava ali; e foram, não só por causa de Jesus,
mas também para ver a Lázaro, a quem ressuscitara dentre os mortos.
E os principais dos sacerdotes tomaram deliberação para matar
também a Lázaro; porque muitos dos judeus, por causa dele,
iam e criam em Jesus.

No dia seguinte, ouvindo uma grande multidão, que viera à festa,
que Jesus vinha a Jerusalém, tomaram ramos de palmeiras, e
saíram-lhe ao encontro, e clamavam: Hosana! Bendito o Rei de Israel
que vem em nome do Senhor. E achou Jesus um jumentinho, e
assentou-se sobre ele, como está escrito: Não temas, ó filha
de Sião; eis que o teu Rei vem assentado sobre o filho de uma
jumenta. Os seus discípulos, porém, não entenderam isto no
princípio; mas, quando Jesus foi glorificado, então se lembraram de
que isto estava escrito dele, e que isto lhe fizeram. A
multidão, pois, que estava com ele quando Lázaro foi chamado da
sepultura, testificava que ele o ressuscitara dentre os mortos.
Por isso a multidão lhe saiu ao encontro, porque tinham
ouvido que ele fizera este sinal. Disseram, pois, os fariseus
entre si: Vedes que nada aproveitais? Eis que toda a gente vai após
ele.

Ora, havia alguns gregos, entre os que tinham subido a adorar no
dia da festa. Estes, pois, dirigiram-se a Filipe, que era de
Betsaida da Galiléia, e rogaram-lhe, dizendo: Senhor, queríamos ver
a Jesus. Filipe foi dizê-lo a André, e então André e Filipe o
disseram a Jesus. E Jesus lhes respondeu, dizendo: É chegada
a hora em que o Filho do homem há de ser glorificado. Na
verdade, na verdade vos digo que, se o grão de trigo, caindo na
terra, não morrer, fica ele só; mas se morrer, dá muito fruto.
Quem ama a sua vida perdê-la-á, e quem neste mundo odeia a
sua vida, guardá-la-á para a vida eterna. Se alguém me serve,
siga-me, e onde eu estiver, ali estará também o meu servo. E, se
alguém me servir, meu Pai o honrará.

Agora a minha alma está perturbada; e que direi eu? Pai, salva-me
desta hora; mas para isto vim a esta hora. Pai, glorifica o
teu nome. Então veio uma voz do céu que dizia: Já o tenho
glorificado, e outra vez o glorificarei. Ora, a multidão que
ali estava, e que a ouvira, dizia que havia sido um trovão. Outros
diziam: Um anjo lhe falou. Respondeu Jesus, e disse: Não veio
esta voz por amor de mim, mas por amor de vós. Agora é o
juízo deste mundo; agora será expulso o príncipe deste mundo.
E eu, quando for levantado da terra, todos atrairei a mim.
E dizia isto, significando de que morte havia de morrer.
Respondeu-lhe a multidão: Nós temos ouvido da lei, que o
Cristo permanece para sempre; e como dizes tu que convém que o Filho
do homem seja levantado? Quem é esse Filho do homem?
Disse-lhes, pois, Jesus: A luz ainda está convosco por um
pouco de tempo. Andai enquanto tendes luz, para que as trevas não
vos apanhem; pois quem anda nas trevas não sabe para onde vai.
Enquanto tendes luz, crede na luz, para que sejais filhos da
luz. Estas coisas disse Jesus e, retirando-se, escondeu-se deles.

E, ainda que tinha feito tantos sinais diante deles, não criam
nele; para que se cumprisse a palavra do profeta Isaías, que
diz: Senhor, quem creu na nossa pregação? E a quem foi revelado o
braço do Senhor? Por isso não podiam crer, então Isaías disse
outra vez: Cegou-lhes os olhos, e endureceu-lhes o coração, a
fim de que não vejam com os olhos, e compreendam no coração, e se
convertam, e eu os cure. Isaías disse isto quando viu a sua
glória e falou dele.

Apesar de tudo, até muitos dos principais creram nele; mas não o
confessavam por causa dos fariseus, para não serem expulsos da
sinagoga. Porque amavam mais a glória dos homens do que a
glória de Deus.

E Jesus clamou, e disse: Quem crê em mim, crê, não em mim, mas
naquele que me enviou. E quem me vê a mim, vê aquele que me
enviou. Eu sou a luz que vim ao mundo, para que todo aquele
que crê em mim não permaneça nas trevas. E se alguém ouvir as
minhas palavras, e não crer, eu não o julgo; porque eu vim, não para
julgar o mundo, mas para salvar o mundo. Quem me rejeitar a
mim, e não receber as minhas palavras, já tem quem o julgue; a
palavra que tenho pregado, essa o há de julgar no último dia.
Porque eu não tenho falado de mim mesmo; mas o Pai, que me
enviou, ele me deu mandamento sobre o que hei de dizer e sobre o que
hei de falar. E sei que o seu mandamento é a vida eterna.
Portanto, o que eu falo, falo-o como o Pai mo tem dito.

\medskip

\lettrine{13} Ora, antes da festa da páscoa, sabendo Jesus que
já era chegada a sua hora de passar deste mundo para o Pai, como
havia amado os seus, que estavam no mundo, amou-os até o fim. E,
acabada a ceia, tendo o diabo posto no coração de Judas Iscariotes,
filho de Simão, que o traísse, Jesus, sabendo que o Pai tinha
depositado nas suas mãos todas as coisas, e que havia saído de Deus
e ia para Deus, levantou-se da ceia, tirou as vestes, e, tomando
uma toalha, cingiu-se. Depois deitou água numa bacia, e começou
a lavar os pés aos discípulos, e a enxugar-lhos com a toalha com que
estava cingido. Aproximou-se, pois, de Simão Pedro, que lhe
disse: Senhor, tu lavas-me os pés a mim? Respondeu Jesus, e
disse-lhe: O que eu faço não o sabes tu agora, mas tu o saberás
depois. Disse-lhe Pedro: Nunca me lavarás os pés. Respondeu-lhe
Jesus: Se eu te não lavar, não tens parte comigo. Disse-lhe
Simão Pedro: Senhor, não só os meus pés, mas também as mãos e a
cabeça. Disse-lhe Jesus: Aquele que está lavado não necessita
de lavar senão os pés, pois no mais todo está limpo. Ora vós estais
limpos, mas não todos. Porque bem sabia ele quem o havia de
trair; por isso disse: Nem todos estais limpos. Depois que
lhes lavou os pés, e tomou as suas vestes, e se assentou outra vez à
mesa, disse-lhes: Entendeis o que vos tenho feito? Vós me
chamais Mestre e Senhor, e dizeis bem, porque eu o sou. Ora,
se eu, Senhor e Mestre, vos lavei os pés, vós deveis também lavar os
pés uns aos outros. Porque eu vos dei o exemplo, para que,
como eu vos fiz, façais vós também. Na verdade, na verdade
vos digo que não é o servo maior do que o seu senhor, nem o enviado
maior do que aquele que o enviou. Se sabeis estas coisas,
bem-aventurados sois se as fizerdes.

Não falo de todos vós; eu bem sei os que tenho escolhido; mas
para que se cumpra a Escritura: O que come o pão comigo, levantou
contra mim o seu calcanhar. Desde agora vo-lo digo, antes que
aconteça, para que, quando acontecer, acrediteis que eu sou.
Na verdade, na verdade vos digo: Se alguém receber o que eu
enviar, me recebe a mim, e quem me recebe a mim, recebe aquele que
me enviou. Tendo Jesus dito isto, turbou-se em espírito, e
afirmou, dizendo: Na verdade, na verdade vos digo que um de vós me
há de trair. Então os discípulos olhavam uns para os outros,
duvidando de quem ele falava. Ora, um de seus discípulos,
aquele a quem Jesus amava, estava reclinado no seio de Jesus.
Então Simão Pedro fez sinal a este, para que perguntasse quem
era aquele de quem ele falava. E, inclinando-se ele sobre o
peito de Jesus, disse-lhe: Senhor, quem é? Jesus respondeu: É
aquele a quem eu der o bocado molhado. E, molhando o bocado, o deu a
Judas Iscariotes, filho de Simão. E, após o bocado, entrou
nele Satanás. Disse, pois, Jesus: O que fazes, faze-o depressa.
E nenhum dos que estavam assentados à mesa compreendeu a que
propósito lhe dissera isto. Porque, como Judas tinha a bolsa,
pensavam alguns que Jesus lhe tinha dito: Compra o que nos é
necessário para a festa; ou que desse alguma coisa aos pobres.
E, tendo Judas tomado o bocado, saiu logo. E era já noite.

Tendo ele, pois, saído, disse Jesus: Agora é glorificado o Filho
do homem, e Deus é glorificado nele. Se Deus é glorificado
nele, também Deus o glorificará em si mesmo, e logo o há de
glorificar. Filhinhos, ainda por um pouco estou convosco. Vós
me buscareis, mas, como tenho dito aos judeus: Para onde eu vou não
podeis vós ir; eu vo-lo digo também agora. Um novo mandamento
vos dou: Que vos ameis uns aos outros; como eu vos amei a vós, que
também vós uns aos outros vos ameis. Nisto todos conhecerão
que sois meus discípulos, se vos amardes uns aos outros.

Disse-lhe Simão Pedro: Senhor, para onde vais? Jesus lhe
respondeu: Para onde eu vou não podes agora seguir-me, mas depois me
seguirás. Disse-lhe Pedro: Por que não posso seguir-te agora?
Por ti darei a minha vida. Respondeu-lhe Jesus: Tu darás a
tua vida por mim? Na verdade, na verdade te digo que não cantará o
galo enquanto não me tiveres negado três vezes.

\medskip

\lettrine{14} Não se turbe o vosso coração; credes em Deus,
crede também em mim. Na casa de meu Pai há muitas moradas; se
não fosse assim, eu vo-lo teria dito. Vou preparar-vos lugar. E
quando eu for, e vos preparar lugar, virei outra vez, e vos levarei
para mim mesmo, para que onde eu estiver estejais vós também.

Mesmo vós sabeis para onde vou, e conheceis o caminho.
Disse-lhe Tomé: Senhor, nós não sabemos para onde vais; e como
podemos saber o caminho? Disse-lhe Jesus: Eu sou o caminho, e a
verdade e a vida; ninguém vem ao Pai, senão por mim. Se vós me
conhecêsseis a mim, também conheceríeis a meu Pai; e já desde agora
o conheceis, e o tendes visto. Disse-lhe Filipe: Senhor,
mostra-nos o Pai, o que nos basta. Disse-lhe Jesus: Estou há
tanto tempo convosco, e não me tendes conhecido, Filipe? Quem me vê
a mim vê o Pai; e como dizes tu: Mostra-nos o Pai? Não crês
tu que eu estou no Pai, e que o Pai está em mim? As palavras que eu
vos digo não as digo de mim mesmo, mas o Pai, que está em mim, é
quem faz as obras. Crede-me que estou no Pai, e o Pai em mim;
crede-me, ao menos, por causa das mesmas obras.

Na verdade, na verdade vos digo que aquele que crê em mim também
fará as obras que eu faço, e as fará maiores do que estas, porque eu
vou para meu Pai. E tudo quanto pedirdes em meu nome eu o
farei, para que o Pai seja glorificado no Filho. Se pedirdes
alguma coisa em meu nome, eu o farei.

Se me amais, guardai os meus mandamentos. E eu rogarei ao
Pai, e ele vos dará outro Consolador, para que fique convosco para
sempre; o Espírito de verdade, que o mundo não pode receber,
porque não o vê nem o conhece; mas vós o conheceis, porque habita
convosco, e estará em vós.

Não vos deixarei órfãos; voltarei para vós. Ainda um
pouco, e o mundo não me verá mais, mas vós me vereis; porque eu
vivo, e vós vivereis. Naquele dia conhecereis que estou em
meu Pai, e vós em mim, e eu em vós. Aquele que tem os meus
mandamentos e os guarda esse é o que me ama; e aquele que me ama
será amado de meu Pai, e eu o amarei, e me manifestarei a ele.
Disse-lhe Judas (não o Iscariotes): Senhor, de onde vem que
te hás de manifestar a nós, e não ao mundo? Jesus respondeu,
e disse-lhe: Se alguém me ama, guardará a minha palavra, e meu Pai o
amará, e viremos para ele, e faremos nele morada. Quem não me
ama não guarda as minhas palavras; ora, a palavra que ouvistes não é
minha, mas do Pai que me enviou.

Tenho-vos dito isto, estando convosco. Mas aquele
Consolador, o Espírito Santo, que o Pai enviará em meu nome, esse
vos ensinará todas as coisas, e vos fará lembrar de tudo quanto vos
tenho dito. Deixo-vos a paz, a minha paz vos dou; não vo-la
dou como o mundo a dá. Não se turbe o vosso coração, nem se
atemorize.

Ouvistes que eu vos disse: Vou, e venho para vós. Se me amásseis,
certamente exultaríeis porque eu disse: Vou para o Pai; porque meu
Pai é maior do que eu. Eu vo-lo disse agora antes que
aconteça, para que, quando acontecer, vós acrediteis. Já não
falarei muito convosco, porque se aproxima o príncipe deste mundo, e
nada tem em mim; mas é para que o mundo saiba que eu amo o
Pai, e que faço como o Pai me mandou. Levantai-vos, vamo-nos daqui.

\medskip

\lettrine{15} Eu sou a videira verdadeira, e meu Pai é o
lavrador. Toda a vara em mim, que não dá fruto, a tira; e limpa
toda aquela que dá fruto, para que dê mais fruto. Vós já estais
limpos, pela palavra que vos tenho falado. Estai em mim, e eu em
vós; como a vara de si mesma não pode dar fruto, se não estiver na
videira, assim também vós, se não estiverdes em mim. Eu sou a
videira, vós as varas; quem está em mim, e eu nele, esse dá muito
fruto; porque sem mim nada podeis fazer. Se alguém não estiver
em mim, será lançado fora, como a vara, e secará; e os colhem e
lançam no fogo, e ardem. Se vós estiverdes em mim, e as minhas
palavras estiverem em vós, pedireis tudo o que quiserdes, e vos será
feito. Nisto é glorificado meu Pai, que deis muito fruto; e
assim sereis meus discípulos.

Como o Pai me amou, também eu vos amei a vós; permanecei no meu
amor. Se guardardes os meus mandamentos, permanecereis no meu
amor; do mesmo modo que eu tenho guardado os mandamentos de meu Pai,
e permaneço no seu amor. Tenho-vos dito isto, para que o meu
gozo permaneça em vós, e o vosso gozo seja completo. O meu
mandamento é este: Que vos ameis uns aos outros, assim como eu vos
amei. Ninguém tem maior amor do que este, de dar alguém a sua
vida pelos seus amigos. Vós sereis meus amigos, se fizerdes o
que eu vos mando. Já vos não chamarei servos, porque o servo
não sabe o que faz o seu senhor; mas tenho-vos chamado amigos,
porque tudo quanto ouvi de meu Pai vos tenho feito conhecer.
Não me escolhestes vós a mim, mas eu vos escolhi a vós, e vos
nomeei, para que vades e deis fruto, e o vosso fruto permaneça; a
fim de que tudo quanto em meu nome pedirdes ao Pai ele vo-lo
conceda. Isto vos mando: Que vos ameis uns aos outros.

Se o mundo vos odeia, sabei que, primeiro do que a vós, me odiou
a mim. Se vós fôsseis do mundo, o mundo amaria o que era seu,
mas porque não sois do mundo, antes eu vos escolhi do mundo, por
isso é que o mundo vos odeia. Lembrai-vos da palavra que vos
disse: Não é o servo maior do que o seu Senhor. Se a mim me
perseguiram, também vos perseguirão a vós; se guardaram a minha
palavra, também guardarão a vossa. Mas tudo isto vos farão
por causa do meu nome, porque não conhecem aquele que me enviou.
Se eu não viera, nem lhes houvera falado, não teriam pecado,
mas agora não têm desculpa do seu pecado. Aquele que me
odeia, odeia também a meu Pai. Se eu entre eles não fizesse
tais obras, quais nenhum outro tem feito, não teriam pecado; mas
agora, viram-nas e me odiaram a mim e a meu Pai. Mas é para
que se cumpra a palavra que está escrita na sua lei: Odiaram-me sem
causa.

Mas, quando vier o Consolador, que eu da parte do Pai vos hei de
enviar, aquele Espírito de verdade, que procede do Pai, ele
testificará de mim. E vós também testificareis, pois
estivestes comigo desde o princípio.

\medskip

\lettrine{16} Tenho-vos dito estas coisas para que vos não
escandalizeis. Expulsar-vos-ão das sinagogas; vem mesmo a hora
em que qualquer que vos matar cuidará fazer um serviço a Deus. E
isto vos farão, porque não conheceram ao Pai nem a mim. Mas
tenho-vos dito isto, a fim de que, quando chegar aquela hora, vos
lembreis de que já vo-lo tinha dito. E eu não vos disse isto desde o
princípio, porque estava convosco. E agora vou para aquele que
me enviou; e nenhum de vós me pergunta: Para onde vais? Antes,
porque isto vos tenho dito, o vosso coração se encheu de tristeza.

Todavia digo-vos a verdade, que vos convém que eu vá; porque, se
eu não for, o Consolador não virá a vós; mas, quando eu for, vo-lo
enviarei. E, quando ele vier, convencerá o mundo do pecado, e da
justiça e do juízo. Do pecado, porque não crêem em mim;
da justiça, porque vou para meu Pai, e não me vereis mais;
e do juízo, porque já o príncipe deste mundo está julgado.
Ainda tenho muito que vos dizer, mas vós não o podeis
suportar agora. Mas, quando vier aquele, o Espírito de
verdade, ele vos guiará em toda a verdade; porque não falará de si
mesmo, mas dirá tudo o que tiver ouvido, e vos anunciará o que há de
vir. Ele me glorificará, porque há de receber do que é meu, e
vo-lo há de anunciar. Tudo quanto o Pai tem é meu; por isso
vos disse que há de receber do que é meu e vo-lo há de anunciar.

Um pouco, e não me vereis; e outra vez um pouco, e ver-me-eis;
porquanto vou para o Pai. Então alguns dos seus discípulos
disseram uns aos outros: Que é isto que nos diz? Um pouco, e não me
vereis; e outra vez um pouco, e ver-me-eis; e: Porquanto vou para o
Pai? Diziam, pois: Que quer dizer isto: Um pouco? Não sabemos
o que diz. Conheceu, pois, Jesus que o queriam interrogar, e
disse-lhes: Indagais entre vós acerca disto que disse: Um pouco, e
não me vereis, e outra vez um pouco, e ver-me-eis? Na
verdade, na verdade vos digo que vós chorareis e vos lamentareis, e
o mundo se alegrará, e vós estareis tristes, mas a vossa tristeza se
converterá em alegria. A mulher, quando está para dar à luz,
sente tristeza, porque é chegada a sua hora; mas, depois de ter dado
à luz a criança, já não se lembra da aflição, pelo prazer de haver
nascido um homem no mundo. Assim também vós agora, na
verdade, tendes tristeza; mas outra vez vos verei, e o vosso coração
se alegrará, e a vossa alegria ninguém vo-la tirará.

E naquele dia nada me perguntareis. Na verdade, na verdade vos
digo que tudo quanto pedirdes a meu Pai, em meu nome, ele vo-lo há
de dar. Até agora nada pedistes em meu nome; pedi, e
recebereis, para que o vosso gozo se cumpra. Disse-vos isto
por parábolas; chega, porém, a hora em que não vos falarei mais por
parábolas, mas abertamente vos falarei acerca do Pai. Naquele
dia pedireis em meu nome, e não vos digo que eu rogarei por vós ao
Pai; pois o mesmo Pai vos ama, visto como vós me amastes, e
crestes que saí de Deus.

Saí do Pai, e vim ao mundo; outra vez deixo o mundo, e vou para o
Pai. Disseram-lhe os seus discípulos: Eis que agora falas
abertamente, e não dizes parábola alguma. Agora conhecemos
que sabes tudo, e não precisas de que alguém te interrogue. Por isso
cremos que saíste de Deus. Respondeu-lhes Jesus: Credes
agora? Eis que chega a hora, e já se aproxima, em que vós
sereis dispersos cada um para sua parte, e me deixareis só; mas não
estou só, porque o Pai está comigo. Tenho-vos dito isto, para
que em mim tenhais paz; no mundo tereis aflições, mas tende bom
ânimo, eu venci o mundo.

\medskip

\lettrine{17} Jesus falou assim e, levantando seus olhos ao
céu, disse: Pai, é chegada a hora; glorifica a teu Filho, para que
também o teu Filho te glorifique a ti; assim como lhe deste
poder sobre toda a carne, para que dê a vida eterna a todos quantos
lhe deste. E a vida eterna é esta: que te conheçam, a ti só, por
único Deus verdadeiro, e a Jesus Cristo, a quem enviaste. Eu
glorifiquei-te na terra, tendo consumado a obra que me deste a
fazer. E agora glorifica-me tu, ó Pai, junto de ti mesmo, com
aquela glória que tinha contigo antes que o mundo existisse.

Manifestei o teu nome aos homens que do mundo me deste; eram teus,
e tu mos deste, e guardaram a tua palavra. Agora já têm
conhecido que tudo quanto me deste provém de ti; porque lhes dei
as palavras que tu me deste; e eles as receberam, e têm
verdadeiramente conhecido que saí de ti, e creram que me enviaste.
Eu rogo por eles; não rogo pelo mundo, mas por aqueles que me
deste, porque são teus. E todas as minhas coisas são tuas, e
as tuas coisas são minhas; e nisso sou glorificado.

E eu já não estou mais no mundo, mas eles estão no mundo, e eu
vou para ti. Pai santo, guarda em teu nome aqueles que me deste,
para que sejam um, assim como nós. Estando eu com eles no
mundo, guardava-os em teu nome. Tenho guardado aqueles que tu me
deste, e nenhum deles se perdeu, senão o filho da perdição, para que
a Escritura se cumprisse. Mas agora vou para ti, e digo isto
no mundo, para que tenham a minha alegria completa em si mesmos.
Dei-lhes a tua palavra, e o mundo os odiou, porque não são do
mundo, assim como eu não sou do mundo. Não peço que os tires
do mundo, mas que os livres do mal. Não são do mundo, como eu
do mundo não sou.

Santifica-os na tua verdade; a tua palavra é a verdade.
Assim como tu me enviaste ao mundo, também eu os enviei ao
mundo. E por eles me santifico a mim mesmo, para que também
eles sejam santificados na verdade.

E não rogo somente por estes, mas também por aqueles que pela sua
palavra hão de crer em mim; para que todos sejam um, como tu,
ó Pai, o és em mim, e eu em ti; que também eles sejam um em nós,
para que o mundo creia que tu me enviaste. E eu dei-lhes a
glória que a mim me deste, para que sejam um, como nós somos um.
Eu neles, e tu em mim, para que eles sejam perfeitos em
unidade, e para que o mundo conheça que tu me enviaste a mim, e que
os tens amado a eles como me tens amado a mim.

Pai, aqueles que me deste quero que, onde eu estiver, também eles
estejam comigo, para que vejam a minha glória que me deste; porque
tu me amaste antes da fundação do mundo. Pai justo, o mundo
não te conheceu; mas eu te conheci, e estes conheceram que tu me
enviaste a mim. E eu lhes fiz conhecer o teu nome, e lho
farei conhecer mais, para que o amor com que me tens amado esteja
neles, e eu neles esteja.

\medskip

\lettrine{18} Tendo Jesus dito isto, saiu com os seus
discípulos para além do ribeiro de Cedrom, onde havia um horto, no
qual ele entrou e seus discípulos. E Judas, que o traía, também
conhecia aquele lugar, porque Jesus muitas vezes se ajuntava ali com
os seus discípulos. Tendo, pois, Judas recebido a coorte e
oficiais dos principais sacerdotes e fariseus, veio para ali com
lanternas, e archotes e armas. Sabendo, pois, Jesus todas as
coisas que sobre ele haviam de vir, adiantou-se, e disse-lhes: A
quem buscais? Responderam-lhe: A Jesus Nazareno. Disse-lhes
Jesus: Sou eu. E Judas, que o traía, estava com eles. Quando,
pois, lhes disse: Sou eu, recuaram, e caíram por terra.
Tornou-lhes, pois, a perguntar: A quem buscais? E eles disseram:
A Jesus Nazareno. Jesus respondeu: Já vos disse que sou eu; se,
pois, me buscais a mim, deixai ir estes; para que se cumprisse a
palavra que tinha dito: Dos que me deste nenhum deles perdi.
Então Simão Pedro, que tinha espada, desembainhou-a, e feriu
o servo do sumo sacerdote, cortando-lhe a orelha direita. E o nome
do servo era Malco. Mas Jesus disse a Pedro: Põe a tua espada
na bainha; não beberei eu o cálice que o Pai me deu? Então a
coorte, e o tribuno, e os servos dos judeus prenderam a Jesus e o
maniataram.

E conduziram-no primeiramente a Anás, por ser sogro de Caifás,
que era o sumo sacerdote daquele ano. Ora, Caifás era quem
tinha aconselhado aos judeus que convinha que um homem morresse pelo
povo. E Simão Pedro e outro discípulo seguiam a Jesus. E este
discípulo era conhecido do sumo sacerdote, e entrou com Jesus na
sala do sumo sacerdote. E Pedro estava da parte de fora, à
porta. Saiu então o outro discípulo que era conhecido do sumo
sacerdote, e falou à porteira, levando Pedro para dentro.
Então a porteira disse a Pedro: Não és tu também dos
discípulos deste homem? Disse ele: Não sou. Ora, estavam ali
os servos e os servidores, que tinham feito brasas, e se aquentavam,
porque fazia frio; e com eles estava Pedro, aquentando-se também.
E o sumo sacerdote interrogou Jesus acerca dos seus
discípulos e da sua doutrina. Jesus lhe respondeu: Eu falei
abertamente ao mundo; eu sempre ensinei na sinagoga e no templo,
onde os judeus sempre se ajuntam, e nada disse em oculto.
Para que me perguntas a mim? Pergunta aos que ouviram o que é
que lhes ensinei; eis que eles sabem o que eu lhes tenho dito.
E, tendo dito isto, um dos servidores que ali estavam, deu
uma bofetada em Jesus, dizendo: Assim respondes ao sumo sacerdote?
Respondeu-lhe Jesus: Se falei mal, dá testemunho do mal; e,
se bem, por que me feres? E Anás mandou-o, maniatado, ao sumo
sacerdote Caifás. E Simão Pedro estava ali, e aquentava-se.
Disseram-lhe, pois: Não és também tu um dos seus discípulos? Ele
negou, e disse: Não sou. E um dos servos do sumo sacerdote,
parente daquele a quem Pedro cortara a orelha, disse: Não te vi eu
no horto com ele? E Pedro negou outra vez, e logo o galo
cantou.

Depois levaram Jesus da casa de Caifás para a audiência. E era
pela manhã cedo. E não entraram na audiência, para não se
contaminarem, mas poderem comer a páscoa. Então Pilatos
saiu\footnote{SBTB: saiu fora.} e disse-lhes: Que acusação trazeis
contra este homem? Responderam, e disseram-lhe: Se este não
fosse malfeitor, não to entregaríamos. Disse-lhes, pois,
Pilatos: Levai-o vós, e julgai-o segundo a vossa lei. Disseram-lhe
então os judeus: A nós não nos é lícito matar pessoa alguma.

que se cumprisse a palavra que Jesus tinha dito,
significando de que morte havia de morrer). Tornou, pois, a
entrar Pilatos na audiência, e chamou a Jesus, e disse-lhe: Tu és o
Rei dos Judeus? Respondeu-lhe Jesus: Tu dizes isso de ti
mesmo, ou disseram-to outros de mim? Pilatos respondeu:
Porventura sou eu judeu? A tua nação e os principais dos sacerdotes
entregaram-te a mim. Que fizeste? Respondeu Jesus: O meu
reino não é deste mundo; se o meu reino fosse deste mundo,
pelejariam os meus servos, para que eu não fosse entregue aos
judeus; mas agora o meu reino não é daqui. Disse-lhe, pois,
Pilatos: Logo tu és rei? Jesus respondeu: Tu dizes que eu sou rei.
Eu para isso nasci, e para isso vim ao mundo, a fim de dar
testemunho da verdade. Todo aquele que é da verdade ouve a minha
voz. Disse-lhe Pilatos: Que é a verdade? E, dizendo isto,
tornou a ir ter com os judeus, e disse-lhes: Não acho nele crime
algum. Mas vós tendes por costume que eu vos solte alguém
pela páscoa. Quereis, pois, que vos solte o Rei dos Judeus?
Então todos tornaram a clamar, dizendo: Este não, mas
Barrabás. E Barrabás era um salteador.

\medskip

\lettrine{19} Pilatos, pois, tomou então a Jesus, e o açoitou.
E os soldados, tecendo uma coroa de espinhos, lha puseram sobre
a cabeça, e lhe vestiram roupa de púrpura. E diziam: Salve, Rei
dos Judeus. E davam-lhe bofetadas. Então Pilatos saiu outra
vez\footnote{SBTB: saiu outra vez fora.}, e disse-lhes: Eis aqui
vo-lo trago fora, para que saibais que não acho nele crime algum.
Saiu, pois, Jesus\footnote{SBTB: Saiu, pois, Jesus fora.},
levando a coroa de espinhos e roupa de púrpura. E disse-lhes
Pilatos: Eis aqui o homem. Vendo-o, pois, os principais dos
sacerdotes e os servos, clamaram, dizendo: Crucifica-o, crucifica-o.
Disse-lhes Pilatos: Tomai-o vós, e crucificai-o; porque eu nenhum
crime acho nele. Responderam-lhe os judeus: Nós temos uma lei e,
segundo a nossa lei, deve morrer, porque se fez Filho de Deus. E
Pilatos, quando ouviu esta palavra, mais atemorizado ficou. E
entrou outra vez na audiência, e disse a Jesus: De onde és tu? Mas
Jesus não lhe deu resposta. Disse-lhe, pois, Pilatos: Não me
falas a mim? Não sabes tu que tenho poder para te crucificar e tenho
poder para te soltar? Respondeu Jesus: Nenhum poder terias
contra mim, se de cima não te fosse dado; mas aquele que me entregou
a ti maior pecado tem. Desde então Pilatos procurava
soltá-lo; mas os judeus clamavam, dizendo: Se soltas este, não és
amigo de César; qualquer que se faz rei é contra César.
Ouvindo, pois, Pilatos este dito, levou Jesus para fora, e
assentou-se no tribunal, no lugar chamado Litóstrotos, e em hebraico
Gabatá. E era a preparação da páscoa, e quase à hora sexta; e
disse aos judeus: Eis aqui o vosso Rei. Mas eles bradaram:
Tira, tira, crucifica-o. Disse-lhes Pilatos: Hei de crucificar o
vosso Rei? Responderam os principais dos sacerdotes: Não temos rei,
senão César.

Então, conseqüentemente entregou-lho, para que fosse crucificado.
E tomaram a Jesus, e o levaram. E, levando ele às costas a
sua cruz, saiu para o lugar chamado Caveira, que em hebraico se
chama Gólgota, onde o crucificaram, e com ele outros dois, um
de cada lado, e Jesus no meio.

E Pilatos escreveu também um título, e pô-lo em cima da cruz; e
nele estava escrito: JESUS NAZARENO, O REI DOS JUDEUS. E
muitos dos judeus leram este título; porque o lugar onde Jesus
estava crucificado era próximo da cidade; e estava escrito em
hebraico, grego e latim. Diziam, pois, os principais
sacerdotes dos judeus a Pilatos: Não escrevas, O Rei dos Judeus, mas
que ele disse: Sou o Rei dos Judeus. Respondeu Pilatos: O que
escrevi, escrevi. Tendo, pois, os soldados crucificado a
Jesus, tomaram as suas vestes, e fizeram quatro partes, para cada
soldado uma parte; e também a túnica. A túnica, porém, tecida toda
de alto a baixo, não tinha costura. Disseram, pois, uns aos
outros: Não a rasguemos, mas lancemos sortes sobre ela, para ver de
quem será. Para que se cumprisse a Escritura que diz: Repartiram
entre si as minhas vestes, e sobre a minha vestidura lançaram
sortes. Os soldados, pois, fizeram estas coisas. E junto à
cruz de Jesus estava sua mãe, e a irmã de sua mãe, Maria mulher de
Clopas, e Maria Madalena. Ora Jesus, vendo ali sua mãe, e que
o discípulo a quem ele amava estava presente, disse a sua mãe:
Mulher, eis aí o teu filho. Depois disse ao discípulo: Eis aí
tua mãe. E desde aquela hora o discípulo a recebeu em sua casa.
Depois, sabendo Jesus que já todas as coisas estavam
terminadas, para que a Escritura se cumprisse, disse: Tenho sede.
Estava, pois, ali um vaso cheio de vinagre. E encheram de
vinagre uma esponja, e, pondo-a num hissope, lha chegaram à boca.
E, quando Jesus tomou o vinagre, disse: Está consumado. E,
inclinando a cabeça, entregou o espírito.

Os judeus, pois, para que no sábado não ficassem os corpos na
cruz, visto como era a preparação (pois era grande o dia de sábado),
rogaram a Pilatos que se lhes quebrassem as pernas, e fossem
tirados. Foram, pois, os soldados, e, na verdade, quebraram
as pernas ao primeiro, e ao outro que como ele fora crucificado;
mas, vindo a Jesus, e vendo-o já morto, não lhe quebraram as
pernas. Contudo um dos soldados lhe furou o lado com uma
lança, e logo saiu sangue e água. E aquele que o viu
testificou, e o seu testemunho é verdadeiro; e sabe que é verdade o
que diz, para que também vós o creiais. Porque isto aconteceu
para que se cumprisse a Escritura, que diz: Nenhum dos seus ossos
será quebrado. E outra vez diz a Escritura: Verão aquele que
traspassaram.

Depois disto, José de Arimatéia (o que era discípulo de Jesus,
mas oculto, por medo dos judeus) rogou a Pilatos que lhe permitisse
tirar o corpo de Jesus. E Pilatos lho permitiu. Então foi e tirou o
corpo de Jesus. E foi também Nicodemos (aquele que
anteriormente se dirigira de noite a Jesus), levando quase cem
arráteis\footnote{Unidade de medida de peso correspondente a 459 g
ou 16 onças; libra.} de um composto de mirra e aloés.
Tomaram, pois, o corpo de Jesus e o envolveram em lençóis com
as especiarias, como os judeus costumam fazer, na preparação para o
sepulcro. E havia um horto naquele lugar onde fora
crucificado, e no horto um sepulcro novo, em que ainda ninguém havia
sido posto. Ali, pois (por causa da preparação dos judeus, e
por estar perto aquele sepulcro), puseram a Jesus.

\medskip

\lettrine{20} E no primeiro dia da semana, Maria Madalena foi
ao sepulcro de madrugada, sendo ainda escuro, e viu a pedra tirada
do sepulcro. Correu, pois, e foi a Simão Pedro, e ao outro
discípulo, a quem Jesus amava, e disse-lhes: Levaram o Senhor do
sepulcro, e não sabemos onde o puseram. Então Pedro saiu com o
outro discípulo, e foram ao sepulcro. E os dois corriam juntos,
mas o outro discípulo correu mais apressadamente do que Pedro, e
chegou primeiro ao sepulcro. E, abaixando-se, viu no chão os
lençóis; todavia não entrou. Chegou, pois, Simão Pedro, que o
seguia, e entrou no sepulcro, e viu no chão os lençóis, e que o
lenço, que tinha estado sobre a sua cabeça, não estava com os
lençóis, mas enrolado num lugar à parte. Então entrou também o
outro discípulo, que chegara primeiro ao sepulcro, e viu, e creu.
Porque ainda não sabiam a Escritura, que era necessário que
ressuscitasse dentre os mortos. Tornaram, pois, os discípulos
para casa.

E Maria estava chorando fora, junto ao sepulcro. Estando ela,
pois, chorando, abaixou-se para o sepulcro. E viu dois anjos
vestidos de branco, assentados onde jazera o corpo de Jesus, um à
cabeceira e outro aos pés. E disseram-lhe eles: Mulher, por
que choras? Ela lhes disse: Porque levaram o meu Senhor, e não sei
onde o puseram. E, tendo dito isto, voltou-se para trás, e
viu Jesus em pé, mas não sabia que era Jesus. Disse-lhe
Jesus: Mulher, por que choras? Quem buscas? Ela, cuidando que era o
hortelão\footnote{Aquele que trata de horta; horteleiro.},
disse-lhe: Senhor, se tu o levaste, dize-me onde o puseste, e eu o
levarei. Disse-lhe Jesus: Maria! Ela, voltando-se, disse-lhe:
Raboni (que quer dizer, Mestre). Disse-lhe Jesus: Não me
detenhas, porque ainda não subi para meu Pai, mas vai para meus
irmãos, e dize-lhes que eu subo para meu Pai e vosso Pai, meu Deus e
vosso Deus. Maria Madalena foi e anunciou aos discípulos que
vira o Senhor, e que ele lhe dissera isto.

Chegada, pois, a tarde daquele dia, o primeiro da semana, e
cerradas as portas onde os discípulos, com medo dos judeus, se
tinham ajuntado, chegou Jesus, e pôs-se no meio, e disse-lhes: Paz
seja convosco. E, dizendo isto, mostrou-lhes as suas mãos e o
lado. De sorte que os discípulos se alegraram, vendo o Senhor.
Disse-lhes, pois, Jesus outra vez: Paz seja convosco; assim
como o Pai me enviou, também eu vos envio a vós. E, havendo
dito isto, assoprou sobre eles e disse-lhes: Recebei o Espírito
Santo. Àqueles a quem perdoardes os pecados lhes são
perdoados; e àqueles a quem os retiverdes lhes são retidos.
Ora, Tomé, um dos doze, chamado Dídimo, não estava com eles
quando veio Jesus. Disseram-lhe, pois, os outros discípulos:
Vimos o Senhor. Mas ele disse-lhes: Se eu não vir o sinal dos cravos
em suas mãos, e não puser o dedo no lugar dos cravos, e não puser a
minha mão no seu lado, de maneira nenhuma o crerei.

E oito dias depois estavam outra vez os seus discípulos dentro, e
com eles Tomé. Chegou Jesus, estando as portas fechadas, e
apresentou-se no meio, e disse: Paz seja convosco. Depois
disse a Tomé: Põe aqui o teu dedo, e vê as minhas mãos; e chega a
tua mão, e põe-na no meu lado; e não sejas incrédulo, mas crente.
E Tomé respondeu, e disse-lhe: Senhor meu, e Deus meu!
Disse-lhe Jesus: Porque me viste, Tomé, creste;
bem-aventurados os que não viram e creram. Jesus, pois,
operou também em presença de seus discípulos muitos outros sinais,
que não estão escritos neste livro. Estes, porém, foram
escritos para que creiais que Jesus é o Cristo, o Filho de Deus, e
para que, crendo, tenhais vida em seu nome.

\medskip

\lettrine{21} Depois disto manifestou-se Jesus outra vez aos
discípulos junto do mar de Tiberíades; e manifestou-se assim:
Estavam juntos Simão Pedro, e Tomé, chamado Dídimo, e Natanael,
que era de Caná da Galiléia, os filhos de Zebedeu, e outros dois dos
seus discípulos. Disse-lhes Simão Pedro: Vou pescar. Dizem-lhe
eles: Também nós vamos contigo. Foram, e subiram logo para o barco,
e naquela noite nada apanharam. E, sendo já manhã, Jesus se
apresentou na praia, mas os discípulos não conheceram que era Jesus.
Disse-lhes, pois, Jesus: Filhos, tendes alguma coisa de comer?
Responderam-lhe: Não. E ele lhes disse: Lançai a rede para o
lado direito do barco, e achareis. Lançaram-na, pois, e já não a
podiam tirar, pela multidão dos peixes. Então aquele discípulo,
a quem Jesus amava, disse a Pedro: É o Senhor. E, quando Simão Pedro
ouviu que era o Senhor, cingiu-se com a túnica (porque estava nu) e
lançou-se ao mar. E os outros discípulos foram com o barco
(porque não estavam distantes da terra senão quase duzentos
côvados), levando a rede cheia de peixes. Logo que desceram para
terra, viram ali brasas, e um peixe posto em cima, e pão.
Disse-lhes Jesus: Trazei dos peixes que agora apanhastes.
Simão Pedro subiu e puxou a rede para terra, cheia de cento e
cinqüenta e três grandes peixes e, sendo tantos, não se rompeu a
rede. Disse-lhes Jesus: Vinde, comei. E nenhum dos discípulos
ousava perguntar-lhe: Quem és tu? sabendo que era o Senhor.
Chegou, pois, Jesus, e tomou o pão, e deu-lhes e,
semelhantemente o peixe. E já era a terceira vez que Jesus se
manifestava aos seus discípulos, depois de ter ressuscitado dentre
os mortos.

E, depois de terem jantado, disse Jesus a Simão Pedro: Simão,
filho de Jonas, amas-me mais do que estes? E ele respondeu: Sim,
Senhor, tu sabes que te amo. Disse-lhe: Apascenta os meus cordeiros.
Tornou a dizer-lhe segunda vez: Simão, filho de Jonas,
amas-me? Disse-lhe: Sim, Senhor, tu sabes que te amo. Disse-lhe:
Apascenta as minhas ovelhas. Disse-lhe terceira vez: Simão,
filho de Jonas, amas-me? Simão entristeceu-se por lhe ter dito
terceira vez: Amas-me? E disse-lhe: Senhor, tu sabes tudo; tu sabes
que eu te amo. Jesus disse-lhe: Apascenta as minhas ovelhas.
Na verdade, na verdade te digo que, quando eras mais moço, te
cingias a ti mesmo, e andavas por onde querias; mas, quando já fores
velho, estenderás as tuas mãos, e outro te cingirá, e te levará para
onde tu não queiras. E disse isto, significando com que morte
havia ele de glorificar a Deus. E, dito isto, disse-lhe: Segue-me.

E Pedro, voltando-se, viu que o seguia aquele discípulo a quem
Jesus amava, e que na ceia se recostara também sobre o seu peito, e
que dissera: Senhor, quem é que te há de trair? Vendo Pedro a
este, disse a Jesus: Senhor, e deste que será? Disse-lhe
Jesus: Se eu quero que ele fique até que eu venha, que te importa a
ti? Segue-me tu. Divulgou-se, pois, entre os irmãos este
dito, que aquele discípulo não havia de morrer. Jesus, porém, não
lhe disse que não morreria, mas: Se eu quero que ele fique até que
eu venha, que te importa a ti? Este é o discípulo que
testifica destas coisas e as escreveu; e sabemos que o seu
testemunho é verdadeiro. Há, porém, ainda muitas outras
coisas que Jesus fez; e se cada uma das quais fosse escrita, cuido
que nem ainda o mundo todo poderia conter os livros que se
escrevessem. Amém.
