\addchap{Primeira Epístola de Paulo aos Tessalonicenses}

\lettrine{1} Paulo, e Silvano, e Timóteo, à igreja dos
tessalonicenses em Deus, o Pai, e no Senhor Jesus Cristo: Graça e
paz tenhais de Deus nosso Pai e do Senhor Jesus Cristo.

Sempre damos graças a Deus por vós todos, fazendo menção de vós em
nossas orações, lembrando-nos sem cessar da obra da vossa fé, do
trabalho do amor, e da paciência da esperança em nosso Senhor Jesus
Cristo, diante de nosso Deus e Pai, sabendo, amados irmãos, que
a vossa eleição é de Deus; porque o nosso evangelho não foi a
vós somente em palavras, mas também em poder, e no Espírito Santo, e
em muita certeza, como bem sabeis quais fomos entre vós, por amor de
vós.

E vós fostes feitos nossos imitadores, e do Senhor, recebendo a
palavra em muita tribulação, com gozo do Espírito Santo. De
maneira que fostes exemplo para todos os fiéis na Macedônia e Acaia.
Porque por vós soou a palavra do Senhor, não somente na
Macedônia e Acaia, mas também em todos os lugares a vossa fé para
com Deus se espalhou, de tal maneira que já dela não temos
necessidade de falar coisa alguma; porque eles mesmos anunciam
de nós qual a entrada que tivemos para convosco, e como dos ídolos
vos convertestes a Deus, para servir o Deus vivo e verdadeiro,
e esperar dos céus a seu Filho, a quem ressuscitou dentre os
mortos, a saber, Jesus, que nos livra da ira futura.

\medskip

\lettrine{2} Porque vós mesmos, irmãos, bem sabeis que a nossa
entrada para convosco não foi vã; mas, mesmo depois de termos
antes padecido, e sido agravados em Filipos, como sabeis,
tornamo-nos ousados em nosso Deus, para vos falar o evangelho de
Deus com grande combate. Porque a nossa exortação não foi com
engano, nem com imundícia, nem com fraudulência; mas, como fomos
aprovados de Deus para que o evangelho nos fosse confiado, assim
falamos, não como para agradar aos homens, mas a Deus, que prova os
nossos corações. Porque, como bem sabeis, nunca usamos de
palavras lisonjeiras, nem houve um pretexto de avareza; Deus é
testemunha; e não buscamos glória dos homens, nem de vós, nem de
outros, ainda que podíamos, como apóstolos de Cristo, ser-vos
pesados; antes fomos brandos entre vós, como a ama que cria seus
filhos.

Assim nós, sendo-vos tão afeiçoados, de boa vontade quiséramos
comunicar-vos, não somente o evangelho de Deus, mas ainda as nossas
próprias almas; porquanto nos éreis muito queridos. Porque bem
vos lembrais, irmãos, do nosso trabalho e fadiga; pois, trabalhando
noite e dia, para não sermos pesados a nenhum de vós, vos pregamos o
evangelho de Deus. Vós e Deus sois testemunhas de quão santa,
e justa, e irrepreensivelmente nos houvemos para convosco, os que
crestes. Assim como bem sabeis de que modo vos exortávamos e
consolávamos, a cada um de vós, como o pai a seus filhos;
para que vos conduzísseis dignamente para com Deus, que vos
chama para o seu reino e glória.

Por isso também damos, sem cessar, graças a Deus, pois, havendo
recebido de nós a palavra da pregação de Deus, a recebestes, não
como palavra de homens, mas (segundo é, na verdade), como palavra de
Deus, a qual também opera em vós, os que crestes. Porque vós,
irmãos, haveis sido feitos imitadores das igrejas de Deus que na
Judéia estão em Jesus Cristo; porquanto também padecestes de vossos
próprios concidadãos o mesmo que os judeus lhes fizeram a eles,
os quais também mataram o Senhor Jesus e os seus próprios
profetas, e nos têm perseguido; e não agradam a Deus, e são
contrários a todos os homens, e nos impedem de pregar aos
gentios as palavras da salvação, a fim de encherem sempre a medida
de seus pecados; mas a ira de Deus caiu sobre eles até ao fim.

Nós, porém, irmãos, sendo privados de vós por um momento de
tempo, de vista, mas não do coração, tanto mais procuramos com
grande desejo ver o vosso rosto; por isso bem quisemos uma e
outra vez ir ter convosco, pelo menos eu, Paulo, mas Satanás no-lo
impediu. Porque, qual é a nossa esperança, ou gozo, ou coroa
de glória? Porventura não o sois vós também diante de nosso Senhor
Jesus Cristo em sua vinda? Na verdade vós sois a nossa glória
e gozo.

\medskip

\lettrine{3} Por isso, não podendo esperar mais, achamos por
bem ficar sozinhos em Atenas; e enviamos Timóteo, nosso irmão, e
ministro de Deus, e nosso cooperador no evangelho de Cristo, para
vos confortar e vos exortar acerca da vossa fé; para que ninguém
se comova por estas tribulações; porque vós mesmos sabeis que para
isto fomos ordenados, pois, estando ainda convosco, vos
predizíamos que havíamos de ser afligidos, como sucedeu, e vós o
sabeis. Portanto, não podendo eu também esperar mais, mandei-o
saber da vossa fé, temendo que o tentador vos tentasse, e o nosso
trabalho viesse a ser inútil.

Vindo, porém, agora Timóteo de vós para nós, e trazendo-nos boas
novas da vossa fé e amor, e de como sempre tendes boa lembrança de
nós, desejando muito ver-nos, como nós também a vós; por esta
razão, irmãos, ficamos consolados acerca de vós, em toda a nossa
aflição e necessidade, pela vossa fé, porque agora vivemos, se
estais firmes no Senhor. Porque, que ação de graças poderemos
dar a Deus por vós, por todo o gozo com que nos regozijamos por
vossa causa diante do nosso Deus, orando abundantemente dia e
noite, para que possamos ver o vosso rosto, e supramos o que falta à
vossa fé?

Ora, o mesmo nosso Deus e Pai, e nosso Senhor Jesus Cristo,
encaminhe a nossa viagem para vós. E o Senhor vos aumente, e
faça crescer em amor uns para com os outros, e para com todos, como
também o fazemos para convosco; para confirmar os vossos
corações, para que sejais irrepreensíveis em santidade diante de
nosso Deus e Pai, na vinda de nosso Senhor Jesus Cristo com todos os
seus santos.

\medskip

\lettrine{4} Finalmente, irmãos, vos rogamos e exortamos no
Senhor Jesus, que assim como recebestes de nós, de que maneira
convém andar e agradar a Deus, assim andai, para que possais
progredir cada vez mais. Porque vós bem sabeis que mandamentos
vos temos dado pelo Senhor Jesus. Porque esta é a vontade de
Deus, a vossa santificação; que vos abstenhais da prostituição;
que cada um de vós saiba possuir o seu vaso em santificação e
honra; não na paixão da concupiscência, como os gentios, que não
conhecem a Deus. Ninguém oprima ou engane a seu irmão em negócio
algum, porque o Senhor é vingador de todas estas coisas, como também
antes vo-lo dissemos e testificamos. Porque não nos chamou Deus
para a imundícia, mas para a santificação. Portanto, quem
despreza isto não despreza ao homem, mas sim a Deus, que nos deu
também o seu Espírito Santo.

Quanto, porém, ao amor fraternal, não necessitais de que vos
escreva, visto que vós mesmos estais instruídos por Deus que vos
ameis uns aos outros; porque também já assim o fazeis para
com todos os irmãos que estão por toda a Macedônia. Exortamo-vos,
porém, a que ainda nisto aumenteis cada vez mais. E procureis
viver quietos, e tratar dos vossos próprios negócios, e trabalhar
com vossas próprias mãos, como já vo-lo temos mandado; para
que andeis honestamente para com os que estão de fora, e não
necessiteis de coisa alguma.

Não quero, porém, irmãos, que sejais ignorantes acerca dos que já
dormem, para que não vos entristeçais, como os demais, que não têm
esperança. Porque, se cremos que Jesus morreu e ressuscitou,
assim também aos que em Jesus dormem, Deus os tornará a trazer com
ele. Dizemo-vos, pois, isto, pela palavra do Senhor: que nós,
os que ficarmos vivos para a vinda do Senhor, não precederemos os
que dormem. Porque o mesmo Senhor descerá do céu com alarido,
e com voz de arcanjo, e com a trombeta de Deus; e os que morreram em
Cristo ressuscitarão primeiro. Depois nós, os que ficarmos
vivos, seremos arrebatados juntamente com eles nas nuvens, a
encontrar o Senhor nos ares, e assim estaremos sempre com o Senhor.
Portanto, consolai-vos uns aos outros com estas palavras.

\medskip

\lettrine{5} Mas, irmãos, acerca dos tempos e das estações,
não necessitais de que se vos escreva; porque vós mesmos sabeis
muito bem que o dia do Senhor virá como o ladrão de noite; pois
que, quando disserem: Há paz e segurança, então lhes sobrevirá
repentina destruição, como as dores de parto àquela que está
grávida, e de modo nenhum escaparão. Mas vós, irmãos, já não
estais em trevas, para que aquele dia vos surpreenda como um ladrão;
porque todos vós sois filhos da luz e filhos do dia; nós não
somos da noite nem das trevas.

Não durmamos, pois, como os demais, mas vigiemos, e sejamos
sóbrios; porque os que dormem, dormem de noite, e os que se
embebedam, embebedam-se de noite. Mas nós, que somos do dia,
sejamos sóbrios, vestindo-nos da couraça da fé e do amor, e tendo
por capacete a esperança da salvação; porque Deus não nos
destinou para a ira, mas para a aquisição da salvação, por nosso
Senhor Jesus Cristo, que morreu por nós, para que, quer
vigiemos, quer durmamos, vivamos juntamente com ele.

Por isso exortai-vos uns aos outros, e edificai-vos uns aos
outros, como também o fazeis. E rogamo-vos, irmãos, que
reconheçais os que trabalham entre vós e que presidem sobre vós no
Senhor, e vos admoestam; e que os tenhais em grande estima e
amor, por causa da sua obra. Tende paz entre vós. Rogamo-vos,
também, irmãos, que admoesteis os desordeiros, consoleis os de pouco
ânimo, sustenteis os fracos, e sejais pacientes para com todos.
Vede que ninguém dê a outrem mal por mal, mas segui sempre o
bem, tanto uns para com os outros, como para com todos.

Regozijai-vos sempre. Orai sem cessar. Em tudo dai
graças, porque esta é a vontade de Deus em Cristo Jesus para
convosco. Não extingais o Espírito. Não desprezeis as
profecias. Examinai tudo. Retende o bem. Abstende-vos
de toda a aparência do mal.

E o mesmo Deus de paz vos santifique em tudo; e todo o vosso
espírito, e alma, e corpo, sejam plenamente conservados
irrepreensíveis para a vinda de nosso Senhor Jesus Cristo.
Fiel é o que vos chama, o qual também o fará. Irmãos,
orai por nós. Saudai a todos os irmãos com ósculo santo.
Pelo Senhor vos conjuro que esta epístola seja lida a todos
os santos irmãos. A graça de nosso Senhor Jesus Cristo seja
convosco. Amém.

