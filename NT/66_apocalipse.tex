\thispagestyle{empty}
\chapter*{Apocalipse}

\lettrine{1} Revelação de Jesus Cristo, a qual Deus lhe deu,
para mostrar aos seus servos as coisas que brevemente devem
acontecer; e pelo seu anjo as enviou, e as notificou a João seu
servo; o qual testificou da palavra de Deus, e do testemunho de
Jesus Cristo, e de tudo o que tem visto.

Bem-aventurado aquele que lê, e os que ouvem as palavras desta
profecia, e guardam as coisas que nela estão escritas; porque o
tempo está próximo. João, às sete igrejas que estão na Ásia:
Graça e paz seja convosco da parte daquele que é, e que era, e que
há de vir, e da dos sete espíritos que estão diante do seu trono;
e da parte de Jesus Cristo, que é a fiel testemunha, o
primogênito dentre os mortos e o príncipe dos reis da terra. Àquele
que nos amou, e em seu sangue nos lavou dos nossos pecados, e
nos fez reis e sacerdotes para Deus e seu Pai; a ele glória e poder
para todo o sempre. Amém. Eis que vem com as nuvens, e todo o
olho o verá, até os mesmos que o traspassaram; e todas as tribos da
terra se lamentarão sobre ele. Sim. Amém. Eu sou o Alfa e o
Ômega, o princípio e o fim, diz o Senhor, que é, e que era, e que há
de vir, o Todo-Poderoso.

Eu, João, que também sou vosso irmão, e companheiro na aflição, e
no reino, e paciência de Jesus Cristo, estava na ilha chamada
Patmos, por causa da palavra de Deus, e pelo testemunho de Jesus
Cristo. Eu fui arrebatado no Espírito\footnote{RA e RC: em
espírito; KJ: I was in the Spirit on the Lord´s day \ldots{}.} no
dia do Senhor, e ouvi detrás de mim uma grande voz, como de
trombeta, que dizia: Eu sou o Alfa e o Ômega, o primeiro e o
derradeiro; e o que vês, escreve-o num livro, e envia-o às sete
igrejas que estão na Ásia: a Éfeso, e a Esmirna, e a Pérgamo, e a
Tiatira, e a Sardes, e a Filadélfia, e a Laodicéia. E
virei-me para ver quem falava comigo. E, virando-me, vi sete
castiçais de ouro; e no meio dos sete castiçais um semelhante
ao Filho do homem, vestido até aos pés de uma roupa comprida, e
cingido pelos peitos com um cinto de ouro. E a sua cabeça e
cabelos eram brancos como lã branca, como a neve, e os seus olhos
como chama de fogo; e os seus pés, semelhantes a latão
reluzente, como se tivessem sido refinados numa fornalha, e a sua
voz como a voz de muitas águas. E ele tinha na sua destra
sete estrelas; e da sua boca saía uma aguda espada de dois fios; e o
seu rosto era como o sol, quando na sua força resplandece. E
eu, quando vi, caí a seus pés como morto; e ele pôs sobre mim a sua
destra, dizendo-me: Não temas; eu sou o primeiro e o último;
e o que vivo e fui morto, mas eis aqui estou vivo para todo o
sempre. Amém. E tenho as chaves da morte e do inferno.
Escreve as coisas que tens visto, e as que são, e as que
depois destas hão de acontecer; o mistério das sete estrelas,
que viste na minha destra, e dos sete castiçais de ouro. As sete
estrelas são os anjos das sete igrejas, e os sete castiçais, que
viste, são as sete igrejas.

\medskip

\lettrine{2} Escreve ao anjo da igreja que está em
\textbf{Éfeso}: Isto diz aquele que tem na sua destra as sete
estrelas, que anda no meio dos sete castiçais de ouro: Conheço
as tuas obras, e o teu trabalho, e a tua paciência, e que não podes
sofrer os maus;\footnote{\emph{Suportar os maus}. KJ: ``\ldots and
how thou canst not \emph{bear} them which are evil \ldots''.} e
puseste à prova os que dizem ser apóstolos, e o não são, e tu os
achaste mentirosos. E sofreste, e tens paciência; e trabalhaste
pelo meu nome, e não te cansaste. Tenho, porém, contra ti que
deixaste o teu primeiro amor. Lembra-te, pois, de onde caíste, e
arrepende-te, e pratica as primeiras obras; quando não, brevemente a
ti virei, e tirarei do seu lugar o teu castiçal, se não te
arrependeres. Tens, porém, isto: que odeias as obras dos
nicolaítas, as quais eu também odeio. Quem tem ouvidos, ouça o
que o Espírito diz às igrejas: Ao que vencer, dar-lhe-ei a comer da
árvore da vida, que está no meio do paraíso de Deus.

E ao anjo da igreja que está em \textbf{Esmirna}, escreve: Isto
diz o primeiro e o último, que foi morto, e reviveu: Conheço as
tuas obras, e tribulação, e pobreza (mas tu  és rico), e a blasfêmia
dos que se dizem judeus, e não o são, mas são a sinagoga de Satanás.
Nada temas das coisas que hás de padecer. Eis que o diabo
lançará alguns de vós na prisão, para que sejais tentados; e tereis
uma tribulação de dez dias. Sê fiel até à morte, e dar-te-ei a coroa
da vida. Quem tem ouvidos, ouça o que o Espírito diz às
igrejas: O  que vencer não receberá o dano da segunda morte.

E ao anjo da igreja que está em \textbf{Pérgamo} escreve: Isto
diz aquele que tem a espada aguda de dois fios: Conheço as
tuas obras, e onde habitas, que é onde está o trono de Satanás; e
reténs o meu nome, e não negaste a minha fé, ainda nos dias de
Antipas, minha fiel testemunha, o qual foi morto entre vós, onde
Satanás habita. Mas algumas poucas coisas tenho contra ti,
porque tens lá os que seguem a doutrina de Balaão, o qual ensinava
Balaque a lançar tropeços diante dos filhos de Israel, para que
comessem dos sacrifícios da idolatria, e se prostituíssem.
Assim  tens também os que seguem a doutrina dos nicolaítas, o
que eu odeio. Arrepende-te, pois, quando não em breve virei a
ti, e contra eles batalharei com a espada da minha boca. Quem
tem ouvidos, ouça o que o Espírito diz às igrejas: Ao que vencer
darei a comer do maná escondido, e dar-lhe-ei uma pedra branca, e na
pedra um novo nome escrito, o qual ninguém conhece senão aquele que
o recebe.

E ao anjo da igreja de \textbf{Tiatira} escreve: Isto diz o Filho
de Deus, que tem seus olhos como chama de fogo, e os pés semelhantes
ao latão reluzente: Eu conheço as tuas obras, e o teu amor, e
o teu serviço, e a tua fé, e a tua paciência, e que as tuas últimas
obras são mais do que as primeiras. Mas tenho contra ti que
toleras Jezabel, mulher que se diz profetisa, ensinar e enganar os
meus servos, para que se prostituam e comam dos sacrifícios da
idolatria. E dei-lhe tempo para que se  arrependesse da sua
prostituição; e não se arrependeu. Eis que a porei numa cama,
e sobre os que adulteram com ela virá grande tribulação, se não se
arrependerem das suas obras. E ferirei de morte a seus
filhos, e todas as igrejas saberão que eu sou aquele que sonda os
rins e os corações. E darei a cada um de vós segundo as vossas
obras. Mas eu vos digo a vós, e aos restantes que estão em
Tiatira, a todos quantos não têm esta doutrina, e não conheceram,
como dizem, as profundezas de Satanás, que outra carga vos não
porei. Mas o que tendes, retende-o até que eu venha. E
ao que vencer, e guardar até ao fim  as minhas obras, eu lhe darei
poder sobre as nações, e com vara de ferro as regerá; e serão
quebradas como vasos de oleiro; como também recebi de meu Pai.
E dar-lhe-ei a estrela da manhã. Quem tem ouvidos,
ouça o que o Espírito diz às igrejas.

\medskip

\lettrine{3} E ao anjo da igreja que está em \textbf{Sardes}
escreve: Isto diz o que tem os sete espíritos de Deus, e as sete
estrelas: Conheço as tuas obras, que tens nome de que vives, e estás
morto. Sê vigilante, e confirma os restantes, que estavam para
morrer; porque não achei as tuas obras perfeitas diante de Deus.
Lembra-te, pois, do que tens recebido e ouvido, e guarda-o, e
arrepende-te. E, se não vigiares, virei sobre ti como um ladrão, e
não saberás a que hora sobre ti virei. Mas também tens em Sardes
algumas pessoas que não contaminaram suas vestes, e comigo andarão
de branco; porquanto são dignas disso. O que vencer será vestido
de vestes brancas, e de maneira nenhuma riscarei o seu nome do livro
da vida; e confessarei o seu nome diante de meu Pai e diante dos
seus anjos. Quem tem ouvidos, ouça o que o Espírito diz às
igrejas.

E ao anjo da igreja que está em \textbf{Filadélfia} escreve: Isto
diz o que é santo, o que é verdadeiro, o que tem a chave de Davi; o
que abre, e ninguém fecha; e fecha, e ninguém abre: Conheço as
tuas obras; eis que diante de ti pus uma porta aberta, e ninguém a
pode fechar; tendo pouca força, guardaste a minha palavra, e não
negaste o meu nome. Eis que eu farei aos da sinagoga de Satanás,
aos que se dizem judeus, e não são, mas mentem: eis que eu farei que
venham, e adorem prostrados a teus pés, e saibam que eu te amo.
Como guardaste a palavra da minha paciência, também eu te
guardarei da hora da tentação que há de vir sobre todo o mundo, para
tentar os que habitam na terra. Eis que venho sem demora;
guarda o que tens, para que ninguém tome a tua coroa. A quem
vencer, eu o farei coluna no templo do meu Deus, e dele nunca sairá;
e escreverei sobre ele o nome do meu Deus, e o nome da cidade do meu
Deus, a nova Jerusalém, que desce do céu, do meu Deus, e também o
meu novo nome. Quem tem ouvidos, ouça o que o Espírito diz às
igrejas.

E ao anjo da igreja que está em \textbf{Laodicéia} escreve: Isto
diz o Amém, a testemunha fiel e verdadeira, o princípio da criação
de Deus: Conheço as tuas obras, que nem és frio nem quente;
quem dera foras frio ou quente! Assim, porque és morno, e não
és frio nem quente, vomitar-te-ei da minha boca. Como dizes:
Rico sou, e estou enriquecido, e de nada tenho falta; e não sabes
que és um desgraçado, e miserável, e pobre, e cego, e nu;
aconselho-te que de mim compres ouro provado no fogo, para
que te enriqueças; e roupas brancas, para que te vistas, e não
apareça a vergonha da tua nudez; e que unjas os teus olhos com
colírio, para que vejas. Eu repreendo e castigo a todos
quantos amo; sê pois zeloso, e arrepende-te. Eis que estou à
porta, e bato; se alguém ouvir a minha voz, e abrir a porta,
entrarei em sua casa, e com ele cearei, e ele comigo. Ao que
vencer lhe concederei que se assente comigo no meu trono; assim como
eu venci, e me assentei com meu Pai no seu trono. Quem tem
ouvidos, ouça o que o Espírito diz às igrejas.

\medskip

\lettrine{4} Depois destas coisas, olhei, e eis que estava uma
porta aberta no céu; e a primeira voz que, como de trombeta, ouvira
falar comigo, disse: Sobe aqui, e mostrar-te-ei as coisas que depois
destas devem acontecer. E logo fui arrebatado no
Espírito,\footnote{RA, RC e KJ: \textbf{e}spírito} e eis que um
trono estava posto no céu, e um assentado sobre o trono. E o que
estava assentado era, na aparência, semelhante à pedra
jaspe\footnote{Variedade semicristalina de quartzo opaco, de cores
diversas, sendo a cor mais comum a vermelha.} e
sardônica\footnote{Variedade de calcedônia, escuro-alaranjada ou
vermelho-pardacenta.}; e o arco celeste estava ao redor do trono, e
parecia semelhante à esmeralda. E ao redor do trono havia vinte
e quatro tronos; e vi assentados sobre os tronos vinte e quatro
anciãos vestidos de vestes brancas; e tinham sobre suas cabeças
coroas de ouro. E do trono saíam relâmpagos, e trovões, e vozes;
e diante do trono ardiam sete lâmpadas de fogo, as quais são os sete
espíritos de Deus. E havia diante do trono como que um mar de
vidro, semelhante ao cristal. E no meio do trono, e ao redor do
trono, quatro animais cheios de olhos, por diante e por detrás.
E o primeiro animal era semelhante a um leão, e o segundo animal
semelhante a um bezerro, e tinha o terceiro animal o rosto como de
homem, e o quarto animal era semelhante a uma águia voando.

E os quatro animais tinham, cada um de per si, seis asas, e ao
redor, e por dentro, estavam cheios de olhos; e não descansam nem de
dia nem de noite, dizendo: Santo, Santo, Santo, é o Senhor Deus, o
Todo-Poderoso, que era, e que é, e que há de vir. E, quando os
animais davam glória, e honra, e ações de graças ao que estava
assentado sobre o trono, ao que vive para todo o sempre, os
vinte e quatro anciãos prostravam-se diante do que estava assentado
sobre o trono, e adoravam o que vive para todo o sempre; e lançavam
as suas coroas diante do trono, dizendo: Digno és, Senhor, de
receber glória, e honra, e poder; porque tu criaste todas as coisas,
e por tua vontade são e foram criadas.

\medskip

\lettrine{5} E vi na destra do que estava assentado sobre o
trono um livro escrito por dentro e por fora, selado com sete selos.
E vi um anjo forte, bradando com grande voz: Quem é digno de
abrir o livro e de desatar os seus selos? E ninguém no céu, nem
na terra, nem debaixo da terra, podia abrir o livro, nem olhar para
ele. E eu chorava muito, porque ninguém fora achado digno de
abrir o livro, nem de o ler, nem de olhar para ele. E disse-me
um dos anciãos: Não chores; eis aqui o Leão da tribo de Judá, a raiz
de Davi, que venceu, para abrir o livro e desatar os seus sete
selos.

E olhei, e eis que estava no meio do trono e dos quatro animais
viventes e entre os anciãos um Cordeiro, como havendo sido morto, e
tinha sete chifres e sete olhos, que são os sete espíritos de Deus
enviados a toda a terra. E veio, e tomou o livro da destra do
que estava assentado no trono. E, havendo tomado o livro, os
quatro animais e os vinte e quatro anciãos prostraram-se diante do
Cordeiro, tendo todos eles harpas e salvas de ouro cheias de
incenso, que são as orações dos santos. E cantavam um novo
cântico, dizendo: Digno és de tomar o livro, e de abrir os seus
selos; porque foste morto, e com o teu sangue compraste para Deus
homens de toda a tribo, e língua, e povo, e nação; e para o
nosso Deus os fizeste reis e sacerdotes; e eles reinarão sobre a
terra. E olhei, e ouvi a voz de muitos anjos ao redor do
trono, e dos animais, e dos anciãos; e era o número deles milhões de
milhões, e milhares de milhares, que com grande voz diziam:
Digno é o Cordeiro, que foi morto, de receber o poder, e riquezas, e
sabedoria, e força, e honra, e glória, e ações de graças. E
ouvi toda a criatura que está no céu, e na terra, e debaixo da
terra, e que está no mar, e a todas as coisas que neles há, dizer:
Ao que está assentado sobre o trono, e ao Cordeiro, sejam dadas
ações de graças, e honra, e glória, e poder para todo o sempre.
E os quatro animais diziam: Amém. E os vinte e quatro anciãos
prostraram-se, e adoraram ao que vive para todo o sempre.

\medskip

\lettrine{6} E, havendo o Cordeiro aberto \textbf{um dos
selos}, olhei, e ouvi um dos quatro animais, que dizia como em voz
de trovão: Vem, e vê. E olhei, e eis um cavalo branco; e o que
estava assentado sobre ele tinha um arco; e foi-lhe dada uma coroa,
e saiu vitorioso, e para vencer.

E, havendo aberto o \textbf{segundo selo}, ouvi o segundo animal,
dizendo: Vem, e vê. E saiu outro cavalo, vermelho; e ao que
estava assentado sobre ele foi dado que tirasse a paz da terra, e
que se matassem uns aos outros; e foi-lhe dada uma grande espada.
E, havendo aberto o \textbf{terceiro selo}, ouvi dizer
o\footnote{SBTB: ao terceiro.} terceiro animal: Vem, e vê. E olhei,
e eis um cavalo preto e o que sobre ele estava assentado tinha uma
balança na mão. E ouvi uma voz no meio dos quatro animais, que
dizia: Uma medida de trigo por um denário,\footnote{SBTB: dinheiro.}
e três medidas de cevada por um denário; e não danifiques o azeite e
o vinho. E, havendo aberto o \textbf{quarto selo}, ouvi a voz do
quarto animal, que dizia: Vem, e vê. E olhei, e eis um cavalo
amarelo, e o que estava assentado sobre ele tinha por nome Morte; e
o inferno o seguia; e foi-lhes dado poder para matar a quarta parte
da terra, com espada, e com fome, e com peste, e com as feras da
terra.

E, havendo aberto o \textbf{quinto selo}, vi debaixo do altar as
almas dos que foram mortos por amor da palavra de Deus e por amor do
testemunho que deram. E clamavam com grande voz, dizendo: Até
quando, ó verdadeiro e santo Dominador\footnote{RA: Até quando, ó
Soberano, santo e verdadeiro \ldots{}. KJ: How long, O Lord, holy
and true, ldots{}.}, não julgas e vingas o nosso sangue dos que
habitam sobre a terra? E foram dadas a cada um compridas
vestes brancas e foi-lhes dito que repousassem ainda um pouco de
tempo, até que também se completasse o número de seus conservos e
seus irmãos, que haviam de ser mortos como eles foram. E,
havendo aberto o \textbf{sexto selo}, olhei, e eis que houve um
grande tremor de terra; e o sol tornou-se negro como saco de
cilício\footnote{Antiga veste ou faixa de crina ou de pano grosseiro
e áspero usado sobre a pele por penitência. Cinto ou cordão eriçado
de cerdas ou correntes de ferro, cheio de pontas, com que os
penitentes cingem o corpo diretamente sobre a pele. Sentido
figurado: sacrifício ou mortificação a que alguém se sujeita
voluntariamente.}, e a lua tornou-se como sangue; e as
estrelas do céu caíram sobre a terra, como quando a figueira lança
de si os seus figos verdes, abalada por um vento forte. E o
céu retirou-se como um livro que se enrola; e todos os montes e
ilhas foram removidos dos seus lugares. E os reis da terra, e
os grandes, e os ricos, e os tribunos, e os poderosos, e todo o
servo, e todo o livre, se esconderam nas cavernas e nas rochas das
montanhas; e diziam aos montes e aos rochedos: Caí sobre nós,
e escondei-nos do rosto daquele que está assentado sobre o trono, e
da ira do Cordeiro; porque é vindo o grande dia da sua ira; e
quem poderá subsistir?

\medskip

\lettrine{7} E depois destas coisas vi quatro anjos que
estavam sobre os quatro cantos da terra, retendo os quatro ventos da
terra, para que nenhum vento soprasse sobre a terra, nem sobre o
mar, nem contra árvore alguma. E vi outro anjo subir do lado do
sol nascente, e que tinha o selo do Deus vivo; e clamou com grande
voz aos quatro anjos, a quem fora dado o poder de danificar a terra
e o mar, dizendo: Não danifiqueis a terra, nem o mar, nem as
árvores, até que hajamos assinalado nas suas testas os servos do
nosso Deus. E ouvi o número dos assinalados, e eram
\textbf{cento e quarenta e quatro mil} assinalados, de todas as
tribos dos filhos de Israel. Da tribo de Judá, havia doze mil
assinalados; da tribo de Rúben, doze mil assinalados; da tribo de
Gade, doze mil assinalados; da tribo de Aser, doze mil
assinalados; da tribo de Naftali, doze mil assinalados; da tribo de
Manassés, doze mil assinalados; da tribo de Simeão, doze mil
assinalados; da tribo de Levi, doze mil assinalados; da tribo de
Issacar, doze mil assinalados; da tribo de Zebulom, doze mil
assinalados; da tribo de José, doze mil assinalados; da tribo de
Benjamim, doze mil assinalados. Depois destas coisas olhei, e
eis aqui uma multidão, a qual ninguém podia contar, de todas as
nações, e tribos, e povos, e línguas, que estavam diante do trono, e
perante o Cordeiro, trajando vestes brancas e com palmas nas suas
mãos; e clamavam com grande voz, dizendo: Salvação ao nosso
Deus, que está assentado no trono, e ao Cordeiro. E todos os
anjos estavam ao redor do trono, e dos anciãos, e dos quatro
animais; e prostraram-se diante do trono sobre seus rostos, e
adoraram a Deus, dizendo: Amém. Louvor, e glória, e
sabedoria, e ação de graças, e honra, e poder, e força ao nosso
Deus, para todo o sempre. Amém.

E um dos anciãos me falou, dizendo: Estes que estão vestidos de
vestes brancas, quem são, e de onde vieram? E eu disse-lhe:
Senhor, tu sabes. E ele disse-me: Estes são os que vieram da grande
tribulação, e lavaram as suas vestes e as branquearam no sangue do
Cordeiro. Por isso estão diante do trono de Deus, e o servem
de dia e de noite no seu templo; e aquele que está assentado sobre o
trono os cobrirá com a sua sombra. Nunca mais terão fome,
nunca mais terão sede; nem sol nem calma\footnote{Grande calor
atmosférico, em geral sem vento. RA: ardor.} alguma cairá sobre
eles. Porque o Cordeiro que está no meio do trono os
apascentará, e lhes servirá de guia para as fontes das águas da
vida; e Deus limpará de seus olhos toda a lágrima.

\medskip

\lettrine{8} E, havendo aberto o \textbf{sétimo selo}, fez-se
silêncio no céu quase por meia hora. E vi os sete anjos, que
estavam diante de Deus, e foram-lhes dadas sete trombetas. E
veio outro anjo, e pôs-se junto ao altar, tendo um incensário de
ouro; e foi-lhe dado muito incenso, para o pôr com as orações de
todos os santos sobre o altar de ouro, que está diante do trono.
E a fumaça do incenso subiu com as orações dos santos desde a
mão do anjo até diante de Deus. E o anjo tomou o incensário, e o
encheu do fogo do altar, e o lançou sobre a terra; e houve depois
vozes, e trovões, e relâmpagos e terremotos. E os sete anjos,
que tinham as sete trombetas, prepararam-se para tocá-las.

E o primeiro anjo tocou a sua trombeta, e houve saraiva e fogo
misturado com sangue, e foram lançados na terra, que foi queimada na
sua terça parte; queimou-se a terça parte das árvores, e toda a erva
verde foi queimada. E o segundo anjo tocou a trombeta; e foi
lançada no mar uma coisa como um grande monte ardendo em fogo, e
tornou-se em sangue a terça parte do mar. E morreu a terça parte
das criaturas que tinham vida no mar; e perdeu-se a terça parte das
naus. E o terceiro anjo tocou a sua trombeta, e caiu do céu
uma grande estrela ardendo como uma tocha, e caiu sobre a terça
parte dos rios, e sobre as fontes das águas. E o nome da
estrela era Absinto, e a terça parte das águas tornou-se em
absinto\footnote{Pequena erva aromática européia, da família das
compostas (Artemisia absinthium), dotada de propriedades amargas, e
cujas sumidades floridas são empregadas no licor de absinto, muito
conhecido por suas propriedades tóxicas; absinto-comum,
absinto-maior, absinto-grande, losna, vermute. Bebida alcoólica
muito amarga, preparada com as folhas dessa planta. Fig. Pesar,
mágoa, amargura.}, e muitos homens morreram das águas, porque se
tornaram amargas. E o quarto anjo tocou a sua trombeta, e foi
ferida a terça parte do sol, e a terça parte da lua, e a terça parte
das estrelas; para que a terça parte deles se escurecesse, e a terça
parte do dia não brilhasse, e semelhantemente a noite. E
olhei, e ouvi um anjo voar pelo meio do céu, dizendo com grande voz:
Ai! ai! ai! dos que habitam sobre a terra! por causa das outras
vozes das trombetas dos três anjos que hão de ainda tocar.

\medskip

\lettrine{9} E o quinto anjo tocou a sua trombeta, e vi uma
estrela que do céu caiu na terra; e foi-lhe dada a chave do poço do
abismo. E abriu o poço do abismo, e subiu fumaça do poço, como a
fumaça de uma grande fornalha, e com a fumaça do poço escureceu-se o
sol e o ar. E da fumaça vieram gafanhotos sobre a terra; e
foi-lhes dado poder, como o poder que têm os escorpiões da terra.
E foi-lhes dito que não fizessem dano à erva da terra, nem a
verdura alguma, nem a árvore alguma, mas somente aos homens que não
têm nas suas testas o sinal de Deus. E foi-lhes permitido, não
que os matassem, mas que por cinco meses os atormentassem; e o seu
tormento era semelhante ao tormento do escorpião, quando fere o
homem. E naqueles dias os homens buscarão a morte, e não a
acharão; e desejarão morrer, e a morte fugirá deles. E o parecer
dos gafanhotos era semelhante ao de cavalos aparelhados para a
guerra; e sobre as suas cabeças havia umas como coroas semelhantes
ao ouro; e os seus rostos eram como rostos de homens. E tinham
cabelos como cabelos de mulheres, e os seus dentes eram como de
leões. E tinham couraças como couraças de ferro; e o ruído das
suas asas era como o ruído de carros, quando muitos cavalos correm
ao combate. E tinham caudas semelhantes às dos escorpiões, e
aguilhões nas suas caudas; e o seu poder era para danificar os
homens por cinco meses. E tinham sobre si rei, o anjo do
abismo; em hebreu era o seu nome \textbf{Abadom}, e em grego
\textbf{Apoliom}. Passado é já um ai; eis que depois disso
vêm ainda dois ais.

E tocou o sexto anjo a sua trombeta, e ouvi uma voz que vinha das
quatro pontas do altar de ouro, que estava diante de Deus, a
qual dizia ao sexto anjo, que tinha a trombeta: Solta os quatro
anjos, que estão presos junto ao grande rio Eufrates. E foram
soltos os quatro anjos, que estavam preparados para a hora, e dia, e
mês, e ano, a fim de matarem a terça parte dos homens. E o
número dos exércitos dos cavaleiros era de duzentos milhões; e ouvi
o número deles. E assim vi os cavalos nesta visão; e os que
sobre eles cavalgavam tinham couraças de fogo, e de jacinto, e de
enxofre; e as cabeças dos cavalos eram como cabeças de leões; e de
suas bocas saía fogo e fumaça e enxofre. Por estes três foi
morta a terça parte dos homens, isto é pelo fogo, pela fumaça, e
pelo enxofre, que saíam das suas bocas. Porque o poder dos
cavalos está na sua boca e nas suas caudas. Porquanto as suas caudas
são semelhantes a serpentes, e têm cabeças, e com elas danificam.
E os outros homens, que não foram mortos por estas pragas,
não se arrependeram das obras de suas mãos, para não adorarem os
demônios, e os ídolos de ouro, e de prata, e de bronze, e de pedra,
e de madeira, que nem podem ver, nem ouvir, nem andar. E não
se arrependeram dos seus homicídios, nem das suas feitiçarias, nem
da sua prostituição, nem dos seus furtos.

\medskip

\lettrine{10} E vi outro anjo forte, que descia do céu,
vestido de uma nuvem; e por cima da sua cabeça estava o arco
celeste, e o seu rosto era como o sol, e os seus pés como colunas de
fogo; e tinha na sua mão um livrinho aberto. E pôs o seu pé
direito sobre o mar, e o esquerdo sobre a terra; e clamou com
grande voz, como quando ruge um leão; e, havendo clamado, os sete
trovões emitiram as suas vozes. E, quando os sete trovões
acabaram de emitir as suas vozes, eu ia escrever; mas ouvi uma voz
do céu, que me dizia: Sela o que os sete trovões emitiram, e não o
escrevas. E o anjo que vi estar sobre o mar e sobre a terra
levantou a sua mão ao céu, e jurou por aquele que vive para todo
o sempre, o qual criou o céu e o que nele há, e a terra e o que nela
há, e o mar e o que nele há, que não haveria mais demora; mas
nos dias da voz do sétimo anjo, quando tocar a sua trombeta, se
cumprirá o segredo de Deus, como anunciou aos profetas, seus servos.

E a voz que eu do céu tinha ouvido tornou a falar comigo, e disse:
Vai, e toma o livrinho aberto da mão do anjo que está em pé sobre o
mar e sobre a terra. E fui ao anjo, dizendo-lhe: Dá-me o
livrinho. E ele disse-me: Toma-o, e come-o, e ele fará amargo o teu
ventre, mas na tua boca será doce como mel. E tomei o
livrinho da mão do anjo, e comi-o; e na minha boca era doce como
mel; e, havendo-o comido, o meu ventre ficou amargo. E ele
disse-me: Importa que profetizes outra vez a muitos povos, e nações,
e línguas e reis.

\medskip

\lettrine{11} E foi-me dada uma cana semelhante a uma vara; e
chegou o anjo, e disse: Levanta-te, e mede o templo de Deus, e o
altar, e os que nele adoram. E deixa o átrio que está fora do
templo, e não o meças; porque foi dado às nações, e pisarão a cidade
santa por \textbf{quarenta e dois meses}.

E darei poder às minhas \textbf{duas testemunhas}, e profetizarão
por \textbf{mil duzentos e sessenta dias}, vestidas de saco.
Estas são as duas oliveiras e os dois castiçais que estão diante
do Deus da terra. E, se alguém lhes quiser fazer mal, fogo sairá
da sua boca, e devorará os seus inimigos; e, se alguém lhes quiser
fazer mal, importa que assim seja morto. Estes têm poder para
fechar o céu, para que não chova, nos dias da sua profecia; e têm
poder sobre as águas para convertê-las em sangue, e para ferir a
terra com toda a sorte de pragas, todas quantas vezes quiserem.
E, quando acabarem o seu testemunho, a besta que sobe do abismo
lhes fará guerra, e os vencerá, e os matará. E jazerão os seus
corpos mortos na praça da grande cidade que espiritualmente se chama
Sodoma e Egito, onde o seu Senhor também foi crucificado. E
homens de vários povos, e tribos, e línguas, e nações verão seus
corpos mortos por três dias e meio, e não permitirão que os seus
corpos mortos sejam postos em sepulcros. E os que habitam na
terra se regozijarão sobre eles, e se alegrarão, e mandarão
presentes uns aos outros; porquanto estes dois profetas tinham
atormentado os que habitam sobre a terra. E depois daqueles
três dias e meio o espírito de vida, vindo de Deus, entrou neles; e
puseram-se sobre seus pés, e caiu grande temor sobre os que os
viram. E ouviram uma grande voz do céu, que lhes dizia: Subi
para aqui. E subiram ao céu em uma nuvem; e os seus inimigos os
viram. E naquela mesma hora houve um grande terremoto, e caiu
a décima parte da cidade, e no terremoto foram mortos sete mil
homens; e os demais ficaram muito atemorizados, e deram glória ao
Deus do céu.

É passado o segundo ai; eis que o terceiro ai cedo virá. E
o \textbf{sétimo anjo tocou a sua trombeta}, e houve no céu grandes
vozes, que diziam: Os reinos do mundo vieram a ser de nosso Senhor e
do seu Cristo, e ele reinará para todo o sempre. E os vinte e
quatro anciãos, que estão assentados em seus tronos diante de Deus,
prostraram-se sobre seus rostos e adoraram a Deus, dizendo:
Graças te damos, Senhor Deus Todo-Poderoso, que és, e que eras, e
que hás de vir, que tomaste o teu grande poder, e reinaste. E
iraram-se as nações, e veio a tua ira, e o tempo dos mortos, para
que sejam julgados, e o tempo de dares o galardão aos profetas, teus
servos, e aos santos, e aos que temem o teu nome, a pequenos e a
grandes, e o tempo de destruíres os que destroem a terra. E
abriu-se no céu o templo de Deus, e a arca da sua aliança foi vista
no seu templo; e houve relâmpagos, e vozes, e trovões, e terremotos
e grande saraiva.

\medskip

\lettrine{12} E viu-se um grande sinal no céu: uma
\textbf{mulher} vestida do sol, tendo a lua debaixo dos seus pés, e
uma coroa de doze estrelas sobre a sua cabeça. E estava grávida,
e com dores de parto, e gritava com ânsias de dar à luz. E
viu-se outro sinal no céu; e eis que era um grande \textbf{dragão
vermelho}, que tinha sete cabeças e dez chifres, e sobre as suas
cabeças sete diademas\footnote{Faixa ornamental com que os soberanos
cingem a cabeça: diadema real. Coroa, grinalda. Jóia ou ornato
circular que cinge os cabelos e/ou adorna a fronte.}. E a sua
cauda levou após si a terça parte das estrelas do céu, e lançou-as
sobre a terra; e o dragão parou diante da mulher que havia de dar à
luz, para que, dando ela à luz, lhe tragasse o filho. E deu à
luz um filho homem que há de reger todas as nações com vara de
ferro; e o seu filho foi arrebatado para Deus e para o seu trono.
E a mulher fugiu para o deserto, onde já tinha lugar preparado
por Deus, para que ali fosse alimentada durante \textbf{mil duzentos
e sessenta dias}. E houve batalha no céu; \textbf{Miguel} e os
seus anjos batalhavam contra o dragão, e batalhavam o dragão e os
seus anjos; mas não prevaleceram, nem mais o seu lugar se achou
nos céus. E foi precipitado o {grande dragão}, a \textbf{antiga
serpente}, chamada o \textbf{Diabo}, e \textbf{Satanás}, que engana
todo o mundo; ele foi precipitado na terra, e os seus anjos foram
lançados com ele. E ouvi uma grande voz no céu, que dizia:
Agora é chegada a salvação, e a força, e o reino do nosso Deus, e o
poder do seu Cristo; porque já o acusador de nossos irmãos é
derrubado, o qual diante do nosso Deus os acusava de dia e de noite.
E eles o venceram pelo sangue do Cordeiro e pela palavra do
seu testemunho; e não amaram as suas vidas até à morte.

Por isso alegrai-vos, ó céus, e vós que neles habitais. Ai dos
que habitam na terra e no mar; porque o diabo desceu a vós, e tem
grande ira, sabendo que já tem pouco tempo. E, quando o
dragão viu que fora lançado na terra, perseguiu a mulher que dera à
luz o filho homem. E foram dadas à mulher duas asas de grande
águia, para que voasse para o deserto, ao seu lugar, onde é
sustentada por \textbf{um tempo, e tempos, e metade de um tempo},
fora da vista da serpente. E a serpente lançou da sua boca,
atrás da mulher, água como um rio, para que pela corrente a fizesse
arrebatar. E a terra ajudou a mulher; e a terra abriu a sua
boca, e tragou o rio que o dragão lançara da sua boca. E o
dragão irou-se contra a mulher, e foi fazer guerra ao remanescente
da sua semente, os que guardam os mandamentos de Deus, e têm o
testemunho de Jesus Cristo. E o dragão parou sobre a areia do
mar.

\medskip

\lettrine{13} E eu pus-me sobre a areia do mar, e vi subir do
mar uma \textbf{besta} que tinha sete cabeças e dez chifres, e sobre
os seus chifres dez diademas, e sobre as suas cabeças um nome de
blasfêmia. E a besta que vi era semelhante ao leopardo, e os
seus pés como os de urso, e a sua boca como a de leão; e o dragão
deu-lhe o seu poder, e o seu trono, e grande poderio. E vi uma
das suas cabeças como ferida de morte, e a sua chaga mortal foi
curada; e toda a terra se maravilhou após a besta. E adoraram o
dragão que deu à besta o seu poder; e adoraram a besta, dizendo:
Quem é semelhante à besta? Quem poderá batalhar contra ela? E
foi-lhe dada uma boca, para proferir grandes coisas e blasfêmias; e
deu-se-lhe poder para agir por \textbf{quarenta e dois meses}. E
abriu a sua boca em blasfêmias contra Deus, para blasfemar do seu
nome, e do seu tabernáculo, e dos que habitam no céu. E foi-lhe
permitido fazer guerra aos santos, e vencê-los; e deu-se-lhe poder
sobre toda a tribo, e língua, e nação. E adoraram-na todos os
que habitam sobre a terra, esses cujos nomes não estão escritos no
livro da vida do Cordeiro que foi morto desde a fundação do mundo.
Se alguém tem ouvidos, ouça. Se alguém leva em cativeiro,
em cativeiro irá; se alguém matar à espada, necessário é que à
espada seja morto. Aqui está a paciência e a fé dos santos.

E vi subir da terra \textbf{outra besta}, e tinha dois chifres
semelhantes aos de um cordeiro; e falava como o dragão. E
exerce todo o poder da primeira besta na sua presença, e faz que a
terra e os que nela habitam adorem a primeira besta, cuja chaga
mortal fora curada. E faz grandes sinais, de maneira que até
fogo faz descer do céu à terra, à vista dos homens. E engana
os que habitam na terra com sinais que lhe foi permitido que fizesse
em presença da besta, dizendo aos que habitam na terra que fizessem
uma imagem à besta que recebera a ferida da espada e vivia. E
foi-lhe concedido que desse espírito à imagem da besta, para que
também a imagem da besta falasse, e fizesse que fossem mortos todos
os que não adorassem a imagem da besta. E faz que a todos,
pequenos e grandes, ricos e pobres, livres e servos, lhes seja posto
um sinal na sua mão direita, ou nas suas testas, para que
ninguém possa comprar ou vender, senão aquele que tiver o sinal, ou
o nome da besta, ou o número do seu nome. Aqui há sabedoria.
Aquele que tem entendimento, calcule o número da besta; porque é o
número de um homem, e o seu número é \textbf{seiscentos e sessenta e
seis}.

\medskip

\lettrine{14} E olhei, e eis que estava o Cordeiro sobre o
monte Sião, e com ele \textbf{cento e quarenta e quatro mil}, que em
suas testas tinham escrito o nome de seu Pai. E ouvi uma voz do
céu, como a voz de muitas águas, e como a voz de um grande trovão; e
ouvi uma voz de harpistas, que tocavam com as suas harpas. E
cantavam um como cântico novo diante do trono, e diante dos quatro
animais e dos anciãos; e ninguém podia aprender aquele cântico,
senão os cento e quarenta e quatro mil que foram comprados da terra.
Estes são os que não estão contaminados com mulheres; porque são
virgens. Estes são os que seguem o Cordeiro para onde quer que vá.
Estes são os que dentre os homens foram comprados como primícias
para Deus e para o Cordeiro. E na sua boca não se achou engano;
porque são irrepreensíveis diante do trono de Deus.

E vi outro anjo voar pelo meio do céu, e tinha o evangelho eterno,
para o proclamar aos que habitam sobre a terra, e a toda a nação, e
tribo, e língua, e povo, dizendo com grande voz: Temei a Deus, e
dai-lhe glória; porque é vinda a hora do seu juízo. E adorai aquele
que fez o céu, e a terra, e o mar, e as fontes das águas. E
outro anjo seguiu, dizendo: Caiu, caiu \textbf{Babilônia}, aquela
grande cidade, que a todas as nações deu a beber do vinho da ira da
sua prostituição. E seguiu-os o terceiro anjo, dizendo com
grande voz: Se alguém adorar a besta, e a sua imagem, e receber o
sinal na sua testa, ou na sua mão, também este beberá do
vinho da ira de Deus, que se deitou, não misturado, no cálice da sua
ira; e será atormentado com fogo e enxofre diante dos santos anjos e
diante do Cordeiro. E a fumaça do seu tormento sobe para todo
o sempre; e não têm repouso nem de dia nem de noite os que adoram a
besta e a sua imagem, e aquele que receber o sinal do seu nome.
Aqui está a paciência dos santos; aqui estão os que guardam
os mandamentos de Deus e a fé em Jesus.

E ouvi uma voz do céu, que me dizia: Escreve: Bem-aventurados os
mortos que desde agora morrem no Senhor. Sim, diz o Espírito, para
que descansem dos seus trabalhos, e as suas obras os seguem.
E olhei, e eis uma nuvem branca, e assentado sobre a nuvem um
semelhante ao Filho do homem, que tinha sobre a sua cabeça uma coroa
de ouro, e na sua mão uma foice aguda. E outro anjo saiu do
templo, clamando com grande voz ao que estava assentado sobre a
nuvem: Lança a tua foice, e sega; a hora de segar te é vinda, porque
já a seara da terra está madura. E aquele que estava
assentado sobre a nuvem meteu a sua foice à terra, e a terra foi
segada. E saiu do templo, que está no céu, outro anjo, o qual
também tinha uma foice aguda. E saiu do altar outro anjo, que
tinha poder sobre o fogo, e clamou com grande voz ao que tinha a
foice aguda, dizendo: Lança a tua foice aguda, e vindima os cachos
da vinha da terra, porque já as suas uvas estão maduras. E o
anjo lançou a sua foice à terra e vindimou as uvas da vinha da
terra, e atirou-as no grande lagar da ira de Deus. E o lagar
foi pisado fora da cidade, e saiu sangue do lagar até aos freios dos
cavalos, pelo espaço de mil e seiscentos estádios\footnote{Um
estádio = 184,9m.}.

\medskip

\lettrine{15} E vi outro grande e admirável sinal no céu: sete
anjos, que tinham as \textbf{sete últimas pragas}; porque nelas é
consumada a ira de Deus. E vi um como mar de vidro misturado com
fogo; e também os que saíram vitoriosos da besta, e da sua imagem, e
do seu sinal, e do número do seu nome, que estavam junto ao mar de
vidro, e tinham as harpas de Deus. E cantavam o cântico de
Moisés, servo de Deus, e o cântico do Cordeiro, dizendo: Grandes e
maravilhosas são as tuas obras, Senhor Deus Todo-Poderoso! Justos e
verdadeiros são os teus caminhos, ó Rei dos santos. Quem te não
temerá, ó Senhor, e não magnificará o teu nome? Porque só tu és
santo; por isso todas as nações virão, e se prostrarão diante de ti,
porque os teus juízos são manifestos.

E depois disto olhei, e eis que o templo do tabernáculo do
testemunho se abriu no céu. E os sete anjos que tinham as sete
pragas saíram do templo, vestidos de linho puro e resplandecente, e
cingidos com cintos de ouro pelos peitos. E um dos quatro
animais deu aos sete anjos sete taças de ouro, cheias da ira de
Deus, que vive para todo o sempre. E o templo encheu-se com a
fumaça da glória de Deus e do seu poder; e ninguém podia entrar no
templo, até que se consumassem as sete pragas dos sete anjos.

\medskip

\lettrine{16} E ouvi, vinda do templo, uma grande voz, que
dizia aos sete anjos: Ide, e derramai sobre a terra as sete taças da
ira de Deus. E foi o primeiro, e derramou a sua taça sobre a
terra, e fez-se uma chaga má e maligna nos homens que tinham o sinal
da besta e que adoravam a sua imagem. E o segundo anjo derramou
a sua taça no mar, que se tornou em sangue como de um morto, e
morreu no mar toda a alma vivente. E o terceiro anjo derramou a
sua taça nos rios e nas fontes das águas, e se tornaram em sangue.
E ouvi o anjo das águas, que dizia: Justo és tu, ó Senhor, que
és, e que eras, e santo és, porque julgaste estas coisas. Visto
como derramaram o sangue dos santos e dos profetas, também tu lhes
deste o sangue a beber; porque disto são merecedores. E ouvi
outro do altar, que dizia: Na verdade, ó Senhor Deus Todo-Poderoso,
verdadeiros e justos são os teus juízos.

E o quarto anjo derramou a sua taça sobre o sol, e foi-lhe
permitido que abrasasse os homens com fogo. E os homens foram
abrasados com grandes calores, e blasfemaram o nome de Deus, que tem
poder sobre estas pragas; e não se arrependeram para lhe darem
glória. E o quinto anjo derramou a sua taça sobre o trono da
besta, e o seu reino se fez tenebroso; e eles mordiam as suas
línguas de dor. E por causa das suas dores, e por causa das
suas chagas, blasfemaram do Deus do céu; e não se arrependeram das
suas obras.

E o sexto anjo derramou a sua taça sobre o grande rio Eufrates; e
a sua água secou-se, para que se preparasse o caminho dos reis do
oriente. E da boca do dragão, e da boca da besta, e da boca
do falso profeta vi sair três espíritos imundos, semelhantes a rãs.
Porque são espíritos de demônios, que fazem prodígios; os
quais vão ao encontro dos reis da terra e de todo o mundo, para os
congregar para a batalha, naquele grande dia do Deus Todo-Poderoso.
Eis que venho como ladrão. Bem-aventurado aquele que vigia, e
guarda as suas roupas, para que não ande nu, e não se vejam as suas
vergonhas. E os congregaram no lugar que em hebreu se chama
\textbf{Armagedom}.

E o sétimo anjo derramou a sua taça no ar, e saiu grande voz do
templo do céu, do trono, dizendo: Está feito. E houve vozes,
e trovões, e relâmpagos, e um grande terremoto, como nunca tinha
havido desde que há homens sobre a terra; tal foi este tão grande
terremoto. E a grande cidade fendeu-se em três partes, e as
cidades das nações caíram; e da grande Babilônia se lembrou Deus,
para lhe dar o cálice do vinho da indignação da sua ira. E
toda a ilha fugiu; e os montes não se acharam. E sobre os
homens caiu do céu uma grande saraiva, pedras do peso de um
talento\footnote{Um talento = 12.600 gramas de prata.}; e os homens
blasfemaram de Deus por causa da praga da saraiva; porque a sua
praga era mui grande.

\medskip

\lettrine{17} E veio um dos sete anjos que tinham as sete
taças, e falou comigo, dizendo-me: Vem, mostrar-te-ei a condenação
da grande prostituta que está assentada sobre muitas águas; com
a qual se prostituíram os reis da terra; e os que habitam na terra
se embebedaram com o vinho da sua prostituição. E levou-me em
espírito a um deserto, e vi uma \textbf{mulher assentada sobre uma
besta} de cor de escarlata\footnote{Escarlate: De cor vermelha muito
viva, e rutilante. Certo tecido de seda ou lã, dessa cor. Certa
tinta vermelha, usada em pintura.}, que estava cheia de nomes de
blasfêmia, e tinha sete cabeças e dez chifres. E a mulher estava
vestida de púrpura e de escarlata, e adornada com ouro, e pedras
preciosas e pérolas; e tinha na sua mão um cálice de ouro cheio das
abominações e da imundícia da sua prostituição; e na sua testa
estava escrito o nome: Mistério, a grande Babilônia, a mãe das
prostituições e abominações da terra. E vi que a mulher estava
embriagada do sangue dos santos, e do sangue das testemunhas de
\textbf{Jesus}. E, vendo-a eu, maravilhei-me com grande admiração.

E o anjo me disse: Por que te admiras? Eu te direi o mistério da
mulher, e da besta que a traz, a qual tem sete cabeças e dez
chifres. A besta que viste foi e já não é, e há de subir do
abismo, e irá à perdição; e os que habitam na terra (cujos nomes não
estão escritos no livro da vida, desde a fundação do mundo) se
admirarão, vendo a besta que era e já não é, mas que virá. Aqui
o sentido, que tem sabedoria. As sete cabeças são sete montes, sobre
os quais a mulher está assentada. E são também sete reis;
cinco já caíram, e um existe; outro ainda não é vindo; e, quando
vier, convém que dure um pouco de tempo. E a besta que era e
já não é, é ela também o oitavo, e é dos sete, e vai à perdição.
E os dez chifres que viste são dez reis, que ainda não
receberam o reino, mas receberão poder como reis por uma hora,
juntamente com a besta. Estes têm um mesmo intento, e
entregarão o seu poder e autoridade à besta.

Estes combaterão contra o Cordeiro, e o Cordeiro os vencerá,
porque é o Senhor dos senhores e o Rei dos reis; vencerão os que
estão com ele, chamados, e eleitos, e fiéis. E disse-me: As
águas que viste, onde se assenta a prostituta, são povos, e
multidões, e nações, e línguas. E os dez chifres que viste na
besta são os que odiarão a prostituta, e a colocarão desolada e nua,
e comerão a sua carne, e a queimarão no fogo. Porque Deus tem
posto em seus corações, que cumpram o seu intento, e tenham uma
mesma idéia, e que dêem à besta o seu reino, até que se cumpram as
palavras de Deus. E a mulher que viste é a grande cidade que
reina sobre os reis da terra.

\medskip

\lettrine{18} E depois destas coisas vi descer do céu outro
anjo, que tinha grande poder, e a terra foi iluminada com a sua
glória. E clamou fortemente com grande voz, dizendo: Caiu, caiu
a grande Babilônia, e se tornou morada de demônios, e covil de todo
espírito imundo, e esconderijo de toda ave imunda e odiável.
Porque todas as nações beberam do vinho da ira da sua
prostituição, e os reis da terra se prostituíram com ela; e os
mercadores da terra se enriqueceram com a abundância de suas
delícias. E ouvi outra voz do céu, que dizia: Sai dela, povo
meu, para que não sejas participante dos seus pecados, e para que
não incorras nas suas pragas. Porque já os seus pecados se
acumularam até ao céu, e Deus se lembrou das iniqüidades dela.
Tornai-lhe a dar como ela vos tem dado, e retribuí-lhe em dobro
conforme as suas obras; no cálice em que vos deu de beber, dai-lhe a
ela em dobro. Quanto ela se glorificou, e em delícias esteve,
foi-lhe outro tanto de tormento e pranto; porque diz em seu coração:
Estou assentada como rainha, e não sou viúva, e não verei o pranto.
Portanto, num dia virão as suas pragas, a morte, e o pranto, e a
fome; e será queimada no fogo; porque é forte o Senhor Deus que a
julga.

E os reis da terra, que se prostituíram com ela, e viveram em
delícias, a chorarão, e sobre ela prantearão, quando virem a fumaça
do seu incêndio; estando de longe pelo temor do seu tormento,
dizendo: Ai! ai daquela grande Babilônia, aquela forte cidade! pois
numa hora veio o seu juízo. E sobre ela choram e lamentam os
mercadores da terra; porque ninguém mais compra as suas mercadorias:
Mercadorias de ouro, e de prata, e de pedras preciosas, e de
pérolas, e de linho fino, e de púrpura, e de seda, e de escarlata; e
toda a madeira odorífera, e todo o vaso de marfim, e todo o vaso de
madeira preciosíssima, de bronze e de ferro, e de mármore; e
canela, e perfume, e mirra, e incenso, e vinho, e azeite, e flor de
farinha, e trigo, e gado, e ovelhas; e cavalos, e carros, e corpos e
almas de homens. E o fruto do desejo da tua alma foi-se de
ti; e todas as coisas gostosas e excelentes se foram de ti, e não
mais as acharás. Os mercadores destas coisas, que com elas se
enriqueceram, estarão de longe, pelo temor do seu tormento, chorando
e lamentando, e dizendo: Ai, ai daquela grande cidade! que
estava vestida de linho fino, de púrpura, de escarlata; e adornada
com ouro e pedras preciosas e pérolas! porque numa hora foram
assoladas tantas riquezas. E todo o piloto, e todo o que
navega em naus, e todo o marinheiro, e todos os que negociam no mar
se puseram de longe; e, vendo a fumaça do seu incêndio,
clamaram, dizendo: Que cidade é semelhante a esta grande cidade?
E lançaram pó sobre as suas cabeças, e clamaram, chorando, e
lamentando, e dizendo: Ai, ai daquela grande cidade! na qual todos
os que tinham naus no mar se enriqueceram em razão da sua opulência;
porque numa hora foi assolada. Alegra-te sobre ela, ó céu, e
vós, santos apóstolos e profetas; porque já Deus julgou a vossa
causa quanto a ela. E um forte anjo levantou uma pedra como
uma grande mó, e lançou-a no mar, dizendo: Com igual ímpeto será
lançada Babilônia, aquela grande cidade, e não será jamais achada.
E em ti não se ouvirá mais a voz de harpistas, e de músicos,
e de flautistas, e de trombeteiros, e nenhum artífice de arte alguma
se achará mais em ti; e ruído de mó em ti não se ouvirá mais;
e luz de candeia não mais luzirá em ti, e voz de esposo e de
esposa não mais em ti se ouvirá; porque os teus mercadores eram os
grandes da terra; porque todas as nações foram enganadas pelas tuas
feitiçarias. E nela se achou o sangue dos profetas, e dos
santos, e de todos os que foram mortos na terra.

\medskip

\lettrine{19} E, depois destas coisas, ouvi no céu como que
uma grande voz de uma grande multidão, que dizia: Aleluia! Salvação,
e glória, e honra, e poder pertencem ao Senhor nosso Deus;
porque verdadeiros e justos são os seus juízos, pois julgou a
grande prostituta, que havia corrompido a terra com a sua
prostituição, e das mãos dela vingou o sangue dos seus servos. E
outra vez disseram: Aleluia! E a fumaça dela sobe para todo o
sempre. E os vinte e quatro anciãos, e os quatro animais,
prostraram-se e adoraram a Deus, que estava assentado no trono,
dizendo: Amém. Aleluia!

E saiu uma voz do trono, que dizia: Louvai o nosso Deus, vós,
todos os seus servos, e vós que o temeis, assim pequenos como
grandes. E ouvi como que a voz de uma grande multidão, e como
que a voz de muitas águas, e como que a voz de grandes trovões, que
dizia: Aleluia! pois já o Senhor Deus Todo-Poderoso reina.
Regozijemo-nos, e alegremo-nos, e demos-lhe glória; porque
vindas são as bodas do Cordeiro, e já a sua esposa se aprontou.
E foi-lhe dado que se vestisse de linho fino, puro e
resplandecente; porque o linho fino são as justiças dos santos.
E disse-me: Escreve: Bem-aventurados aqueles que são chamados à
ceia das bodas do Cordeiro. E disse-me: Estas são as verdadeiras
palavras de Deus. E eu lancei-me a seus pés para o adorar;
mas ele disse-me: Olha não faças tal; sou teu conservo, e de teus
irmãos, que têm o testemunho de \textbf{Jesus}. Adora a Deus; porque
\textbf{o testemunho de Jesus é o espírito de profecia}.

E vi o céu aberto, e eis um cavalo branco; e o que estava
assentado sobre ele chama-se Fiel e Verdadeiro; e julga e peleja com
justiça. E os seus olhos eram como chama de fogo; e sobre a
sua cabeça havia muitos diademas; e tinha um nome escrito, que
ninguém sabia senão ele mesmo. E estava vestido de uma veste
salpicada de sangue; e o nome pelo qual se chama é a \textbf{Palavra
de Deus}. E seguiam-no os exércitos no céu em cavalos
brancos, e vestidos de linho fino, branco e puro. E da sua
boca saía uma aguda espada, para ferir com ela as nações; e ele as
regerá com vara de ferro; e ele mesmo é o que pisa o lagar do vinho
do furor e da ira do Deus Todo-Poderoso. E no manto e na sua
coxa tem escrito este nome: \textbf{Rei dos reis, e Senhor dos
senhores}. E vi um anjo que estava no sol, e clamou com
grande voz, dizendo a todas as aves que voavam pelo meio do céu:
Vinde, e ajuntai-vos à ceia do grande Deus; para que comais a
carne dos reis, e a carne dos tribunos, e a carne dos fortes, e a
carne dos cavalos e dos que sobre eles se assentam; e a carne de
todos os homens, livres e servos, pequenos e grandes. E vi a
besta, e os reis da terra, e os seus exércitos reunidos, para
fazerem guerra àquele que estava assentado sobre o cavalo, e ao seu
exército. E a besta foi presa, e com ela o falso profeta, que
diante dela fizera os sinais, com que enganou os que receberam o
sinal da besta, e adoraram a sua imagem. Estes dois foram lançados
vivos no lago de fogo que arde com enxofre. E os demais foram
mortos com a espada que saía da boca do que estava assentado sobre o
cavalo, e todas as aves se fartaram das suas carnes.

\medskip

\lettrine{20} E vi descer do céu um anjo, que tinha a chave do
abismo, e uma grande cadeia na sua mão. Ele prendeu o dragão, a
antiga serpente, que é o Diabo e Satanás, e amarrou-o por mil anos.
E lançou-o no abismo, e ali o encerrou, e pôs selo sobre ele,
para que não mais engane as nações, até que os mil anos se acabem. E
depois importa que seja solto por um pouco de tempo. E vi
tronos; e assentaram-se sobre eles, e foi-lhes dado o poder de
julgar; e vi as almas daqueles que foram degolados pelo testemunho
de Jesus, e pela palavra de Deus, e que não adoraram a besta, nem a
sua imagem, e não receberam o sinal em suas testas nem em suas mãos;
e viveram, e reinaram com Cristo durante mil anos. Mas os outros
mortos não reviveram, até que os mil anos se acabaram. \textbf{Esta
é a primeira ressurreição}. Bem-aventurado e santo aquele que
tem parte na primeira ressurreição; sobre estes não tem poder a
segunda morte; mas serão sacerdotes de Deus e de Cristo, e reinarão
com ele mil anos. E, acabando-se os mil anos, Satanás será solto
da sua prisão, e sairá a enganar as nações que estão sobre os
quatro cantos da terra, Gogue e Magogue, cujo número é como a areia
do mar, para as ajuntar em batalha. E subiram sobre a largura da
terra, e cercaram o arraial dos santos e a cidade amada; e de Deus
desceu fogo, do céu, e os devorou. E o diabo, que os
enganava, foi lançado no lago de fogo e enxofre, onde está a besta e
o falso profeta; e de dia e de noite serão atormentados para todo o
sempre.

E vi um grande trono branco, e o que estava assentado sobre ele,
de cuja presença fugiu a terra e o céu; e não se achou lugar para
eles. E vi os mortos, grandes e pequenos, que estavam diante
de Deus, e abriram-se os livros; e abriu-se outro livro, que é o da
vida. E os mortos foram julgados pelas coisas que estavam escritas
nos livros, segundo as suas obras. E deu o mar os mortos que
nele havia; e a morte e o inferno deram os mortos que neles havia; e
foram julgados cada um segundo as suas obras. E a morte e o
inferno foram lançados no lago de fogo. \textbf{Esta é a segunda
morte}. E aquele que não foi achado escrito no livro da vida
foi lançado no lago de fogo.

\medskip

\lettrine{21} E vi um novo céu, e uma nova terra. Porque já o
primeiro céu e a primeira terra passaram, e o mar já não existe.
E eu, João, vi a santa cidade, a nova Jerusalém, que de Deus
descia do céu, adereçada como uma esposa ataviada\footnote{Ataviar:
ornar, adornar; enfeitar, adereçar, aformosear.} para o seu marido.
E ouvi uma grande voz do céu, que dizia: Eis aqui o tabernáculo
de Deus com os homens, pois com eles habitará, e eles serão o seu
povo, e o mesmo Deus estará com eles, e será o seu Deus. E Deus
limpará de seus olhos toda a lágrima; e não haverá mais morte, nem
pranto, nem clamor, nem dor; porque já as primeiras coisas são
passadas. E o que estava assentado sobre o trono disse: Eis que
faço novas todas as coisas. E disse-me: Escreve; porque estas
palavras são verdadeiras e fiéis. E disse-me mais: Está
cumprido. Eu sou o Alfa e o Ômega, o princípio e o fim. A quem quer
que tiver sede, de graça lhe darei da fonte da água da vida.
Quem vencer, herdará todas as coisas; e eu serei seu Deus, e ele
será meu filho. Mas, quanto aos tímidos, e aos incrédulos, e aos
abomináveis, e aos homicidas, e aos fornicadores, e aos feiticeiros,
e aos idólatras e a todos os mentirosos, a sua parte será no lago
que arde com fogo e enxofre; o que é a \textbf{segunda morte}.

E veio a mim um dos sete anjos que tinham as sete taças cheias das
últimas sete pragas, e falou comigo, dizendo: Vem, mostrar-te-ei a
esposa, a mulher do Cordeiro. E levou-me em espírito a um
grande e alto monte, e mostrou-me a grande cidade, a santa
Jerusalém, que de Deus descia do céu. E tinha a glória de
Deus; e a sua luz era semelhante a uma pedra preciosíssima, como a
pedra de jaspe, como o cristal resplandecente. E tinha um
grande e alto muro com doze portas, e nas portas doze anjos, e nomes
escritos sobre elas, que são os nomes das doze tribos dos filhos de
Israel. Do lado do levante\footnote{Este. Vento muito forte,
que sopra a oeste, no Mediterrâneo e na costa meridional da França e
da Espanha. Os países do Mediterrâneo oriental.} tinha três portas,
do lado do norte, três portas, do lado do sul, três portas, do lado
do poente, três portas. E o muro da cidade tinha doze
fundamentos, e neles os nomes dos doze apóstolos do Cordeiro.
E aquele que falava comigo tinha uma cana de ouro, para medir
a cidade, e as suas portas, e o seu muro. E a cidade estava
situada em quadrado; e o seu comprimento era tanto como a sua
largura. E mediu a cidade com a cana até doze mil
estádios\footnote{Estádio (400 côvados) = 180 metros. Logo, 12 mil
estádios corresponde a 2.160 Km. Em tempo: 1 côvado = 45 cm.}; e o
seu comprimento, largura e altura eram iguais. E mediu o seu
muro, de cento e quarenta e quatro côvados, conforme a medida de
homem, que é a de um anjo. E a construção do seu muro era de
jaspe\footnote{ Pedra preciosa, variedade transparente de coríndon,
cuja cor varia do azul-celeste ao azul-escuro.}, e a cidade de ouro
puro, semelhante a vidro puro. E os fundamentos do muro da
cidade estavam adornados de toda a pedra preciosa. O primeiro
fundamento era jaspe; o segundo, safira; o terceiro,
calcedônia\footnote{Variedade de sílica microcristalina,
transparente ou translúcida.}; o quarto, esmeralda; o quinto,
sardônica\footnote{Variedade de calcedônia, escuro-alaranjada ou
vermelho-pardacenta; sardônia.}; o sexto, sárdio\footnote{Variedade
vermelha da calcedônia.}; o sétimo, crisólito\footnote{Pedra
preciosa da cor do ouro; crisólita.}; o oitavo,
berilo\footnote{Mineral hexagonal, silicato de alumínio e glucínio,
pedra semipreciosa.}; o nono, topázio\footnote{Mineral ortorrômbico,
fluossilicato fluorífero de alumínio, pedra preciosa.}; o décimo,
crisópraso\footnote{Variedade de calcedônia verde-clara.}; o
undécimo, jacinto\footnote{O jacinto (hyakinthos) era uma pedra
azul, água-marinha (a variedade azul do berilo), ou safira. Esse
nome era usado para indicar a cor azul (no grego clássico, como
substantivo, indicava a flor que também se chamava 'jacinto'), como
em Ap 9.17 (hyakinthinos), que algumas versões traduzem por
'jacinto' enquanto que certa revisão traduz como 'safira'. Quanto a
Ex 28.19 e 39.12, ainda que na versão RA-SBB diga 'jacinto',
provavelmente não se tratava da mesma pedra (em heb., leshem), mas
de alguma pedra provavelmente amarela, ainda que não identificada.
Algumas versões preferem 'âmbar'.}; o duodécimo,
ametista\footnote{Pedra semipreciosa, variedade roxa do quartzo.}.
E as doze portas eram doze pérolas; cada uma das portas era
uma pérola; e a praça da cidade de ouro puro, como vidro
transparente. E nela não vi templo, porque o seu templo é o
Senhor Deus Todo-Poderoso, e o Cordeiro. E a cidade não
necessita de sol nem de lua, para que nela resplandeçam, porque a
glória de Deus a tem iluminado, e o Cordeiro é a sua lâmpada.
E as nações dos salvos andarão à sua luz; e os reis da terra
trarão para ela a sua glória e honra. E as suas portas não se
fecharão de dia, porque ali não haverá noite. E a ela trarão
a glória e honra das nações. E não entrará nela coisa alguma
que contamine, e cometa abominação e mentira; mas só os que estão
inscritos no livro da vida do Cordeiro.

\medskip

\lettrine{22} E mostrou-me o rio puro da água da vida, claro
como cristal, que procedia do trono de Deus e do Cordeiro. No
meio da sua praça, e de um e de outro lado do rio, estava a
\textbf{árvore da vida}, que produz doze frutos, dando seu fruto de
mês em mês; e as folhas da árvore são para a saúde das nações. E
ali nunca mais haverá maldição contra alguém; e nela estará o trono
de Deus e do Cordeiro, e os seus servos o servirão. E verão o
seu rosto, e nas suas testas estará o seu nome. E ali não haverá
mais noite, e não necessitarão de lâmpada nem de luz do sol, porque
o Senhor Deus os ilumina; e reinarão para todo o sempre.

E disse-me: Estas palavras são fiéis e verdadeiras; e o Senhor, o
Deus dos santos profetas, enviou o seu anjo, para mostrar aos seus
servos as coisas que em breve hão de acontecer. Eis que presto
venho: Bem-aventurado aquele que guarda as palavras da profecia
deste livro. E eu, João, sou aquele que vi e ouvi estas coisas.
E, havendo-as ouvido e visto, prostrei-me aos pés do anjo que mas
mostrava para o adorar. E disse-me: Olha, não faças tal; porque
eu sou conservo teu e de teus irmãos, os profetas, e dos que guardam
as palavras deste livro. Adora a Deus. E disse-me: Não seles
as palavras da profecia deste livro; porque próximo está o tempo.
Quem é injusto, faça injustiça ainda; e quem está sujo,
suje-se ainda; e quem é justo, faça justiça ainda; e quem é santo,
seja santificado ainda. E eis que cedo venho, e o meu
galardão está comigo, para dar a cada um segundo a sua obra.
Eu sou o Alfa e o Ômega, o princípio e o fim, o primeiro e o
derradeiro. Bem-aventurados aqueles que guardam os seus
mandamentos, para que tenham direito à árvore da vida, e possam
entrar na cidade pelas portas. Ficarão de fora os cães e os
feiticeiros, e os que se prostituem, e os homicidas, e os idólatras,
e qualquer que ama e comete a mentira. \textbf{Eu, Jesus},
enviei o meu anjo, para vos testificar estas coisas nas igrejas. Eu
sou a raiz e a geração de Davi, a resplandecente estrela da manhã.
E o Espírito e a esposa dizem: Vem. E quem ouve, diga: Vem. E
quem tem sede, venha; e quem quiser, tome de graça da água da vida.
Porque eu testifico a todo aquele que ouvir as palavras da
profecia deste livro que, se alguém lhes acrescentar alguma coisa,
Deus fará vir sobre ele as pragas que estão escritas neste livro;
 e, se alguém tirar quaisquer palavras do livro desta
profecia, Deus tirará a sua parte do livro da vida, e da cidade
santa, e das coisas que estão escritas neste livro.

Aquele que testifica estas coisas diz: Certamente cedo venho.
Amém. Ora vem, Senhor Jesus. A graça de nosso Senhor Jesus
Cristo seja com todos vós. Amém.

