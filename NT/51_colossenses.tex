\thispagestyle{empty}
\chapter*{Epístola de Paulo aos Colossenses}

\lettrine{1} Paulo, apóstolo de Jesus Cristo, pela vontade de
Deus, e o irmão Timóteo, aos santos e irmãos fiéis em Cristo,
que estão em Colossos: Graça a vós, e paz da parte de Deus nosso Pai
e do Senhor Jesus Cristo.

Graças damos a Deus, Pai de nosso Senhor Jesus Cristo, orando
sempre por vós, porquanto ouvimos da vossa fé em Cristo Jesus, e
do amor que tendes para com todos os santos; por causa da
esperança que vos está reservada nos céus, da qual já antes ouvistes
pela palavra da verdade do evangelho, que já chegou a vós, como
também está em todo o mundo; e já vai frutificando, como também
entre vós, desde o dia em que ouvistes e conhecestes a graça de Deus
em verdade; como aprendestes de Epafras, nosso amado conservo,
que para vós é um fiel ministro de Cristo, o qual nos declarou
também o vosso amor no Espírito.

Por esta razão, nós também, desde o dia em que o ouvimos, não
cessamos de orar por vós, e de pedir que sejais cheios do
conhecimento da sua vontade, em toda a sabedoria e inteligência
espiritual; para que possais andar dignamente diante do
Senhor, agradando-lhe em tudo, frutificando em toda a boa obra, e
crescendo no conhecimento de Deus; corroborados em toda a
fortaleza, segundo a força da sua glória, em toda a paciência, e
longanimidade com gozo.

Dando graças ao Pai que nos fez idôneos para participar da
herança dos santos na luz; o qual nos tirou da potestade das
trevas, e nos transportou para o reino do Filho do seu amor;
em quem temos a redenção pelo seu sangue, a saber, a remissão
dos pecados; o qual é imagem do Deus invisível, o primogênito
de toda a criação; porque nele foram criadas todas as coisas
que há nos céus e na terra, visíveis e invisíveis, sejam tronos,
sejam dominações, sejam principados, sejam potestades. Tudo foi
criado por ele e para ele. E ele é antes de todas as coisas,
e todas as coisas subsistem por ele. E ele é a cabeça do
corpo, da igreja; é o princípio e o primogênito dentre os mortos,
para que em tudo tenha a preeminência. Porque foi do agrado
do Pai que toda a plenitude nele habitasse, e que, havendo
por ele feito a paz pelo sangue da sua cruz, por meio dele
reconciliasse consigo mesmo todas as coisas, tanto as que estão na
terra, como as que estão nos céus. A vós também, que noutro
tempo éreis estranhos, e inimigos no entendimento pelas vossas obras
más, agora contudo vos reconciliou no corpo da sua carne,
pela morte, para perante ele vos apresentar santos, e
irrepreensíveis, e inculpáveis, se, na verdade, permanecerdes
fundados e firmes na fé, e não vos moverdes da esperança do
evangelho que tendes ouvido, o qual foi pregado a toda criatura que
há debaixo do céu, e do qual eu, Paulo, estou feito ministro.

Regozijo-me agora no que padeço por vós, e na minha carne cumpro
o resto das aflições de Cristo, pelo seu corpo, que é a igreja;
da qual eu estou feito ministro segundo a dispensação de
Deus, que me foi concedida para convosco, para cumprir a palavra de
Deus; o mistério que esteve oculto desde todos os séculos, e
em todas as gerações, e que agora foi manifesto aos seus santos;
aos quais Deus quis fazer conhecer quais são as riquezas da
glória deste mistério entre os gentios, que é Cristo em vós,
esperança da glória; a quem anunciamos, admoestando a todo o
homem, e ensinando a todo o homem em toda a sabedoria; para que
apresentemos todo o homem perfeito em Jesus Cristo; e para
isto também trabalho, combatendo segundo a sua eficácia, que opera
em mim poderosamente.

\medskip

\lettrine{2} Porque quero que saibais quão grande combate
tenho por vós, e pelos que estão em Laodicéia, e por quantos não
viram o meu rosto em carne; para que os seus corações sejam
consolados, e estejam unidos em amor, e enriquecidos da plenitude da
inteligência, para conhecimento do mistério de Deus e Pai, e de
Cristo, em quem estão escondidos todos os tesouros da sabedoria
e da ciência.

E digo isto, para que ninguém vos engane com palavras persuasivas.
Porque, ainda que esteja ausente quanto ao corpo, contudo, em
espírito estou convosco, regozijando-me e vendo a vossa ordem e a
firmeza da vossa fé em Cristo. Como, pois, recebestes o Senhor
Jesus Cristo, assim também andai nele, arraigados e
sobreedificados nele, e confirmados na fé, assim como fostes
ensinados, nela abundando em ação de graças. Tende cuidado, para
que ninguém vos faça presa sua, por meio de filosofias e vãs
sutilezas, segundo a tradição dos homens, segundo os rudimentos do
mundo, e não segundo Cristo; porque nele habita corporalmente
toda a plenitude da divindade; e estais perfeitos nele, que é
a cabeça de todo o principado e potestade; no qual também
estais circuncidados com a circuncisão não feita por mão no despojo
do corpo dos pecados da carne, a circuncisão de Cristo;
sepultados com ele no batismo, nele também ressuscitastes
pela fé no poder de Deus, que o ressuscitou dentre os mortos.

E, quando vós estáveis mortos nos pecados, e na incircuncisão da
vossa carne, vos vivificou juntamente com ele, perdoando-vos todas
as ofensas, havendo riscado a cédula que era contra nós nas
suas ordenanças, a qual de alguma maneira nos era contrária, e a
tirou do meio de nós, cravando-a na cruz. E, despojando os
principados e potestades, os expôs publicamente e deles triunfou em
si mesmo.

Portanto, ninguém vos julgue pelo comer, ou pelo beber, ou por
causa dos dias de festa, ou da lua nova, ou dos sábados, que
são sombras das coisas futuras, mas o corpo é de Cristo.
Ninguém vos domine a seu bel-prazer com pretexto de humildade
e culto dos anjos, envolvendo-se em coisas que não viu; estando
debalde inchado na sua carnal compreensão, e não ligado à
cabeça, da qual todo o corpo, provido e organizado pelas juntas e
ligaduras, vai crescendo em aumento de Deus. Se, pois, estais
mortos com Cristo quanto aos rudimentos do mundo, por que vos
carregam ainda de ordenanças, como se vivêsseis no mundo, tais como:
Não toques, não proves, não manuseies? As quais coisas
todas perecem pelo uso, segundo os preceitos e doutrinas dos homens;
as quais têm, na verdade, alguma aparência de sabedoria, em
devoção voluntária, humildade, e em disciplina do corpo, mas não são
de valor algum senão para a satisfação da carne.

\medskip

\lettrine{3} Portanto, se já ressuscitastes com Cristo, buscai
as coisas que são de cima, onde Cristo está assentado à destra de
Deus. Pensai nas coisas que são de cima, e não nas que são da
terra; porque já estais mortos, e a vossa vida está escondida
com Cristo em Deus. Quando Cristo, que é a nossa vida, se
manifestar, então também vós vos manifestareis com ele em glória.

Mortificai, pois, os vossos membros, que estão sobre a terra: a
prostituição, a impureza, a afeição desordenada, a vil
concupiscência, e a avareza, que é idolatria; pelas quais coisas
vem a ira de Deus sobre os filhos da desobediência; nas quais,
também, em outro tempo andastes, quando vivíeis nelas.

Mas agora, despojai-vos também de tudo: da ira, da cólera, da
malícia, da maledicência, das palavras torpes da vossa boca. Não
mintais uns aos outros, pois que já vos despistes do velho homem com
os seus feitos, e vos vestistes do novo, que se renova para o
conhecimento, segundo a imagem daquele que o criou; onde não
há grego, nem judeu, circuncisão, nem incircuncisão, bárbaro, cita,
servo ou livre; mas Cristo é tudo em todos.

Revesti-vos, pois, como eleitos de Deus, santos e amados, de
entranhas de misericórdia, de benignidade, humildade, mansidão,
longanimidade; suportando-vos uns aos outros, e perdoando-vos
uns aos outros, se alguém tiver queixa contra outro; assim como
Cristo vos perdoou, assim fazei vós também. E, sobre tudo
isto, revesti-vos de amor, que é o vínculo da perfeição. E a
paz de Deus, para a qual também fostes chamados em um corpo, domine
em vossos corações; e sede agradecidos. A palavra de Cristo
habite em vós abundantemente, em toda a sabedoria, ensinando-vos e
admoestando-vos uns aos outros, com salmos, hinos e cânticos
espirituais, cantando ao Senhor com graça em vosso coração.
E, quanto fizerdes por palavras ou por obras, fazei tudo em
nome do Senhor Jesus, dando por ele graças a Deus Pai.

Vós, mulheres, estai sujeitas a vossos próprios maridos, como
convém no Senhor. Vós, maridos, amai a vossas mulheres, e não
vos irriteis contra elas. Vós, filhos, obedecei em tudo a
vossos pais, porque isto é agradável ao Senhor. Vós, pais,
não irriteis a vossos filhos, para que não percam o ânimo.
Vós, servos, obedecei em tudo a vossos senhores segundo a
carne, não servindo só na aparência, como para agradar aos homens,
mas em simplicidade de coração, temendo a Deus. E tudo quanto
fizerdes, fazei-o de todo o coração, como ao Senhor, e não aos
homens, sabendo que recebereis do Senhor o galardão da
herança, porque a Cristo, o Senhor, servis. Mas quem fizer
agravo receberá o agravo que fizer; pois não há acepção de pessoas.

\medskip

\lettrine{4} Vós, senhores, fazei o que for de justiça e
eqüidade a vossos servos, sabendo que também tendes um Senhor nos
céus.

Perseverai em oração, velando nela com ação de graças; orando
também juntamente por nós, para que Deus nos abra a porta da
palavra, a fim de falarmos do mistério de Cristo, pelo qual estou
também preso; para que o manifeste, como me convém falar.

Andai com sabedoria para com os que estão de fora, remindo o
tempo. A vossa palavra seja sempre agradável, temperada com sal,
para que saibais como vos convém responder a cada um.

Tíquico, irmão amado e fiel ministro, e conservo no Senhor, vos
fará saber o meu estado; o qual vos enviei para o mesmo fim,
para que saiba do vosso estado e console os vossos corações;
juntamente com Onésimo, amado e fiel irmão, que é dos vossos;
eles vos farão saber tudo o que por aqui se passa. Aristarco,
que está preso comigo, vos saúda, e Marcos, o sobrinho de Barnabé,
acerca do qual já recebestes mandamentos; se ele for ter convosco,
recebei-o; e Jesus, chamado Justo; os quais são da
circuncisão; são estes unicamente os meus cooperadores no reino de
Deus; e para mim têm sido consolação. Saúda-vos Epafras, que
é dos vossos, servo de Cristo, combatendo sempre por vós em orações,
para que vos conserveis firmes, perfeitos e consumados em toda a
vontade de Deus. Pois eu lhe dou testemunho de que tem grande
zelo por vós, e pelos que estão em Laodicéia, e pelos que estão em
Hierápolis. Saúda-vos Lucas, o médico amado, e Demas.
Saudai aos irmãos que estão em Laodicéia e a Ninfa e à igreja
que está em sua casa. E, quando esta epístola tiver sido lida
entre vós, fazei que também o seja na igreja dos laodicenses, e a
que veio de Laodicéia lede-a vós também. E dizei a Arquipo:
Atenta para o ministério que recebeste no Senhor, para que o
cumpras. Saudação de minha mão, de Paulo. Lembrai-vos das
minhas prisões. A graça seja convosco. Amém.

