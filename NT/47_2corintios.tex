\addchap{Segunda Epístola de Paulo aos Coríntios}

\lettrine{1}\ Paulo, apóstolo de Jesus Cristo, pela vontade de
Deus, e o irmão Timóteo, à igreja de Deus, que está em Corinto, com
todos os santos que estão em toda a Acaia. Graça a vós e paz da
parte de Deus nosso Pai, e da do Senhor Jesus Cristo.

Bendito seja o Deus e Pai de nosso Senhor Jesus Cristo, o Pai das
misericórdias e o Deus de toda a consolação; que nos consola em
toda a nossa tribulação, para que também possamos consolar os que
estiverem em alguma tribulação, com a consolação com que nós mesmos
somos consolados por Deus. Porque, como as aflições de Cristo
são abundantes em nós, assim também é abundante a nossa consolação
por meio de Cristo. Mas, se somos atribulados, é para vossa
consolação e salvação; ou, se somos consolados, para vossa
consolação é, a qual se opera suportando com paciência as mesmas
aflições que nós também padecemos.

E a nossa esperança acerca de vós é firme, sabendo que, como sois
participantes das aflições, assim o sereis também da consolação.
Porque não queremos, irmãos, que ignoreis a tribulação que nos
sobreveio na Ásia, pois que fomos sobremaneira agravados mais do que
podíamos suportar, de modo tal que até da vida desesperamos. Mas
já em nós mesmos tínhamos a sentença de morte, para que não
confiássemos em nós, mas em Deus, que ressuscita os mortos; o
qual nos livrou de tão grande morte, e livra; em quem esperamos que
também nos livrará ainda, ajudando-nos também vós com orações
por nós, para que pela mercê, que por muitas pessoas nos foi feita,
por muitas também sejam dadas graças a nosso respeito.

Porque a nossa glória é esta: o testemunho da nossa consciência,
de que com simplicidade e sinceridade de Deus, não com sabedoria
carnal, mas na graça de Deus, temos vivido no mundo, e de modo
particular convosco. Porque nenhumas outras coisas vos
escrevemos, senão as que já sabeis ou também reconheceis; e espero
que também até ao fim as reconhecereis. Como também já em
parte reconhecestes em nós, que somos a vossa glória, como também
vós sereis a nossa no dia do Senhor Jesus.

E com esta confiança quis primeiro ir ter convosco, para que
tivésseis uma segunda graça; e por vós passar à Macedônia, e
da Macedônia ir outra vez ter convosco, e ser guiado por vós à
Judéia. E, deliberando isto, usei porventura de leviandade?
Ou o que delibero, o delibero segundo a carne, para que haja em mim
sim, sim, e não, não? Antes, como Deus é fiel, a nossa
palavra para convosco não foi sim e não. Porque o Filho de
Deus, Jesus Cristo, que entre vós foi pregado por nós, isto é, por
mim, Silvano e Timóteo, não foi sim e não; mas nele houve sim.
Porque todas quantas promessas há de Deus, são nele sim, e
por ele o Amém, para glória de Deus por nós. Mas o que nos
confirma convosco em Cristo, e o que nos ungiu, é Deus, o
qual também nos selou e deu o penhor do Espírito em nossos corações.
Invoco, porém, a Deus por testemunha sobre a minha alma, que
para vos poupar não tenho até agora ido a Corinto; não que
tenhamos domínio sobre a vossa fé, mas porque somos cooperadores de
vosso gozo; porque pela fé estais em pé.


\medskip

\lettrine{2}\ Mas deliberei isto comigo mesmo: não ir mais ter
convosco em tristeza. Porque, se eu vos entristeço, quem é que
me alegrará, senão aquele que por mim foi contristado? E
escrevi-vos isto mesmo, para que, quando lá for, não tenha tristeza
da parte dos que deveriam alegrar-me; confiando em vós todos, que a
minha alegria é a de todos vós. Porque em muita tribulação e
angústia do coração vos escrevi, com muitas lágrimas, não para que
vos entristecêsseis, mas para que conhecêsseis o amor que
abundantemente vos tenho.

Porque, se alguém me contristou, não me contristou a mim senão em
parte, para vos não sobrecarregar a vós todos. Basta-lhe ao tal
esta repreensão feita por muitos. De maneira que pelo contrário
deveis antes perdoar-lhe e consolá-lo, para que o tal não seja de
modo algum devorado de demasiada tristeza. Por isso vos rogo que
confirmeis para com ele o vosso amor. E para isso vos escrevi
também, para por esta prova saber se sois obedientes em tudo.
E a quem perdoardes alguma coisa, também eu; porque, o que eu
também perdoei, se é que tenho perdoado, por amor de vós o fiz na
presença de Cristo; para que não sejamos vencidos por Satanás;
porque não ignoramos os seus ardis.

Ora, quando cheguei a Trôade para pregar o evangelho de Cristo, e
abrindo-se-me uma porta no Senhor, não tive descanso no meu
espírito, porque não achei ali meu irmão Tito; mas, despedindo-me
deles, parti para a Macedônia. E graças a Deus, que sempre
nos faz triunfar em Cristo, e por meio de nós manifesta em todo o
lugar a fragrância do seu conhecimento. Porque para Deus
somos o bom perfume de Cristo, nos que se salvam e nos que se
perdem. Para estes certamente cheiro de morte para morte; mas
para aqueles cheiro de vida para vida. E para estas coisas quem é
idôneo? Porque nós não somos, como muitos, falsificadores da
palavra de Deus, antes falamos de Cristo com sinceridade, como de
Deus na presença de Deus.

\medskip

\lettrine{3}\ Porventura começamos outra vez a louvar-nos a nós
mesmos? Ou necessitamos, como alguns, de cartas de recomendação para
vós, ou de recomendação de vós? Vós sois a nossa carta, escrita
em nossos corações, conhecida e lida por todos os homens. Porque
já é manifesto que vós sois a carta de Cristo, ministrada por nós, e
escrita, não com tinta, mas com o Espírito do Deus vivo, não em
tábuas de pedra, mas nas tábuas de carne do coração. E é por
Cristo que temos tal confiança em Deus; não que sejamos capazes,
por nós, de pensar alguma coisa, como de nós mesmos; mas a nossa
capacidade vem de Deus, o qual nos fez também capazes de ser
ministros de um novo testamento, não da letra, mas do espírito;
porque a letra mata e o espírito vivifica.

E, se o ministério da morte, gravado com letras em pedras, veio em
glória, de maneira que os filhos de Israel não podiam fitar os olhos
na face de Moisés, por causa da glória do seu rosto, a qual era
transitória, como não será de maior glória o ministério do
Espírito? Porque, se o ministério da condenação foi glorioso,
muito mais excederá em glória o ministério da justiça. Porque
também o que foi glorificado nesta parte não foi glorificado, por
causa desta excelente glória. Porque, se o que era
transitório foi para glória, muito mais é em glória o que permanece.

Tendo, pois, tal esperança, usamos de muita ousadia no falar.
E não somos como Moisés, que punha um véu sobre a sua face,
para que os filhos de Israel não olhassem firmemente para o fim
daquilo que era transitório. Mas os seus sentidos foram
endurecidos; porque até hoje o mesmo véu está por levantar na lição
do velho testamento, o qual foi por Cristo abolido; e até
hoje, quando é lido Moisés, o véu está posto sobre o coração deles.
Mas, quando se converterem ao Senhor, então o véu se tirará.
Ora, o Senhor é Espírito\footnote{KJ: Now the Lord is that
Spirit: and where the Spirit of the Lord is, there is liberty. RA:
 Ora, o Senhor é o Espírito; e, onde está o Espírito do Senhor, aí há
liberdade.}; e onde está o Espírito do Senhor, aí há liberdade.
Mas todos nós, com rosto descoberto, refletindo como um
espelho a glória do Senhor, somos transformados de glória em glória
na mesma imagem, como pelo Espírito do Senhor.

\medskip

\lettrine{4}\ Por isso, tendo este ministério, segundo a
misericórdia que nos foi feita, não desfalecemos; antes,
rejeitamos as coisas que por vergonha se ocultam, não andando com
astúcia nem falsificando a palavra de Deus; e assim nos recomendamos
à consciência de todo o homem, na presença de Deus, pela
manifestação da verdade. Mas, se ainda o nosso evangelho está
encoberto, para os que se perdem está encoberto. Nos quais o
deus deste século cegou os entendimentos dos incrédulos, para que
lhes não resplandeça a luz do evangelho da glória de Cristo, que é a
imagem de Deus. Porque não nos pregamos a nós mesmos, mas a
Cristo Jesus, o Senhor; e nós mesmos somos vossos servos por amor de
Jesus. Porque Deus, que disse que das trevas resplandecesse a
luz, é quem resplandeceu em nossos corações, para iluminação do
conhecimento da glória de Deus, na face de Jesus Cristo. Temos,
porém, este tesouro em vasos de barro, para que a excelência do
poder seja de Deus, e não de nós.

Em tudo somos atribulados, mas não angustiados; perplexos, mas não
desanimados. Perseguidos, mas não desamparados; abatidos, mas
não destruídos; trazendo sempre por toda a parte a
mortificação do Senhor Jesus no nosso corpo, para que a vida de
Jesus se manifeste também nos nossos corpos; e assim nós, que
vivemos, estamos sempre entregues à morte por amor de Jesus, para
que a vida de Jesus se manifeste também na nossa carne mortal.
De maneira que em nós opera a morte, mas em vós a vida.
E temos portanto o mesmo espírito de fé, como está escrito:
Cri, por isso falei; nós cremos também, por isso também falamos.
Sabendo que o que ressuscitou o Senhor Jesus nos ressuscitará
também por Jesus, e nos apresentará convosco. Porque tudo
isto é por amor de vós, para que a graça, multiplicada por meio de
muitos, faça abundar a ação de graças para glória de Deus.
Por isso não desfalecemos; mas, ainda que o nosso homem
exterior se corrompa, o interior, contudo, se renova de dia em dia.
Porque a nossa leve e momentânea tribulação produz para nós
um peso eterno de glória mui excelente; não atentando nós nas
coisas que se vêem, mas nas que se não vêem; porque as que se vêem
são temporais, e as que se não vêem são eternas.

\medskip

\lettrine{5}\ Porque sabemos que, se a nossa casa terrestre
deste tabernáculo se desfizer, temos de Deus um edifício, uma casa
não feita por mãos, eterna, nos céus. E por isso também gememos,
desejando ser revestidos da nossa habitação, que é do céu; se,
todavia, estando vestidos, não formos achados nus. Porque também
nós, os que estamos neste tabernáculo, gememos carregados; não
porque queremos ser despidos, mas revestidos, para que o mortal seja
absorvido pela vida. Ora, quem para isto mesmo nos preparou foi
Deus, o qual nos deu também o penhor do Espírito. Por isso
estamos sempre de bom ânimo, sabendo que, enquanto estamos no corpo,
vivemos ausentes do Senhor (porque andamos por fé, e não por
vista). Mas temos confiança e desejamos antes deixar este corpo,
para habitar com o Senhor. Pois que muito desejamos também
ser-lhe agradáveis, quer presentes, quer ausentes. Porque
todos devemos comparecer ante o tribunal de Cristo, para que cada um
receba segundo o que tiver feito por meio do corpo, ou bem, ou mal.
Assim que, sabendo o temor que se deve ao Senhor, persuadimos
os homens à fé, mas somos manifestos a Deus; e espero que nas vossas
consciências sejamos também manifestos.

Porque não nos recomendamos outra vez a vós; mas damo-vos ocasião
de vos gloriardes de nós, para que tenhais que responder aos que se
gloriam na aparência e não no coração. Porque, se
enlouquecemos, é para Deus; e, se conservamos o juízo, é para vós.
Porque o amor de Cristo nos constrange, julgando nós assim:
que, se um morreu por todos, logo todos morreram. E ele
morreu por todos, para que os que vivem não vivam mais para si, mas
para aquele que por eles morreu e ressuscitou.

Assim que daqui por diante a ninguém conhecemos segundo a carne,
e, ainda que também tenhamos conhecido Cristo segundo a carne,
contudo agora já não o conhecemos deste modo. Assim que, se
alguém está em Cristo, nova criatura é; as coisas velhas já
passaram; eis que tudo se fez novo. E tudo isto provém de
Deus, que nos reconciliou consigo mesmo por Jesus Cristo, e nos deu
o ministério da reconciliação; isto é, Deus estava em Cristo
reconciliando consigo o mundo, não lhes imputando os seus pecados; e
pôs em nós a palavra da reconciliação. De sorte que somos
embaixadores da parte de Cristo, como se Deus por nós rogasse.
Rogamo-vos, pois, da parte de Cristo, que vos reconcilieis com Deus.
Àquele que não conheceu pecado, o fez pecado por nós; para
que nele fôssemos feitos justiça de Deus.

\medskip

\lettrine{6}\ E nós, cooperando também com ele, vos exortamos a
que não recebais a graça de Deus em vão (porque diz: Ouvi-te em
tempo aceitável e socorri-te no dia da salvação; eis aqui agora o
tempo aceitável, eis aqui agora o dia da salvação). Não dando
nós escândalo em coisa alguma, para que o nosso ministério não seja
censurado; antes, como ministros de Deus, tornando-nos
recomendáveis em tudo; na muita paciência, nas aflições, nas
necessidades, nas angústias, nos açoites, nas prisões, nos
tumultos, nos trabalhos, nas vigílias, nos jejuns, na pureza, na
ciência, na longanimidade, na benignidade, no Espírito Santo, no
amor não fingido, na palavra da verdade, no poder de Deus, pelas
armas da justiça, à direita e à esquerda, por honra e por
desonra, por infâmia e por boa fama; como enganadores, e sendo
verdadeiros; como desconhecidos, mas sendo bem conhecidos; como
morrendo, e eis que vivemos; como castigados, e não mortos;
como contristados, mas sempre alegres; como pobres, mas
enriquecendo a muitos; como nada tendo, e possuindo tudo.

Ó coríntios, a nossa boca está aberta para vós, o nosso coração
está dilatado. Não estais estreitados em nós; mas estais
estreitados nos vossos próprios afetos. Ora, em recompensa
disto, (falo como a filhos) dilatai-vos também vós. Não vos
prendais a um jugo desigual com os infiéis; porque, que sociedade
tem a justiça com a injustiça? E que comunhão tem a luz com as
trevas? E que concórdia há entre Cristo e Belial? Ou que
parte tem o fiel com o infiel? E que consenso tem o templo de
Deus com os ídolos? Porque vós sois o templo do Deus vivente, como
Deus disse: Neles habitarei, e entre eles andarei; e eu serei o seu
Deus e eles serão o meu povo. Por isso saí do meio deles, e
apartai-vos, diz o Senhor; e não toqueis nada imundo, e eu vos
receberei; e eu serei para vós Pai, e vós sereis para mim
filhos e filhas, diz o Senhor Todo-Poderoso.

\medskip

\lettrine{7}\ Ora, amados, pois que temos tais promessas,
purifiquemo-nos de toda a imundícia da carne e do espírito,
aperfeiçoando a santificação no temor de Deus. Recebei-nos em
vossos corações; a ninguém agravamos, a ninguém corrompemos, de
ninguém buscamos o nosso proveito. Não digo isto para vossa
condenação; pois já antes tinha dito que estais em nossos corações
para juntamente morrer e viver. Grande é a ousadia da minha fala
para convosco, e grande a minha jactância\footnote{Atitude de alguém
que se manifesta com arrogância e tem alta opinião de si mesmo;
vaidade, orgulho, arrogância. Pretensão de bravura ou altos méritos
e conquistas; atitude de quem conta bravatas; fanfarrice.} a
respeito de vós; estou cheio de consolação; transbordo de gozo em
todas as nossas tribulações.

Porque, mesmo quando chegamos à Macedônia, a nossa carne não teve
repouso algum; antes em tudo fomos atribulados: por fora combates,
temores por dentro. Mas Deus, que consola os abatidos, nos
consolou com a vinda de Tito. E não somente com a sua vinda, mas
também pela consolação com que foi consolado por vós, constando-nos
as vossas saudades, o vosso choro, o vosso zelo por mim, de maneira
que muito me regozijei. Porquanto, ainda que vos contristei com
a minha carta, não me arrependo, embora já me tivesse arrependido
por ver que aquela carta vos contristou, ainda que por pouco tempo.
Agora folgo, não porque fostes contristados, mas porque fostes
contristados para arrependimento; pois fostes contristados segundo
Deus; de maneira que por nós não padecestes dano em coisa alguma.
Porque a tristeza segundo Deus opera arrependimento para a
salvação, da qual ninguém se arrepende; mas a tristeza do mundo
opera a morte. Porque, quanto cuidado não produziu isto mesmo
em vós que, segundo Deus, fostes contristados! que apologia, que
indignação, que temor, que saudades, que zelo, que vingança! Em tudo
mostrastes estar puros neste negócio.

Portanto, ainda que vos escrevi, não foi por causa do que fez o
agravo, nem por causa do que sofreu o agravo, mas para que o vosso
grande cuidado por nós fosse manifesto diante de Deus. Por
isso fomos consolados pela vossa consolação, e muito mais nos
alegramos pela alegria de Tito, porque o seu espírito foi recreado
por vós todos. Porque, se nalguma coisa me gloriei de vós
para com ele, não fiquei envergonhado; mas, como vos dissemos tudo
com verdade, também a nossa glória para com Tito se achou
verdadeira. E o seu entranhável afeto para convosco é mais
abundante, lembrando-se da obediência de vós todos, e de como o
recebestes com temor e tremor. Regozijo-me de em tudo poder
confiar em vós.

\medskip

\lettrine{8}\ Também, irmãos, vos fazemos conhecer a graça de
Deus dada às igrejas da Macedônia; como em muita prova de
tribulação houve abundância do seu gozo, e como a sua profunda
pobreza abundou em riquezas da sua generosidade. Porque, segundo
o seu poder (o que eu mesmo testifico) e ainda acima do seu poder,
deram voluntariamente. Pedindo-nos com muitos rogos que
aceitássemos a graça e a comunicação deste serviço, que se fazia
para com os santos. E não somente fizeram como nós esperávamos,
mas a si mesmos se deram primeiramente ao Senhor, e depois a nós,
pela vontade de Deus. De maneira que exortamos a Tito que, assim
como antes tinha começado, assim também acabasse esta graça entre
vós.

Portanto, assim como em tudo abundais em fé, e em palavra, e em
ciência, e em toda a diligência, e em vosso amor para conosco, assim
também abundeis nesta graça. Não digo isto como quem manda, mas
para provar, pela diligência dos outros, a sinceridade de vosso
amor. Porque já sabeis a graça de nosso Senhor Jesus Cristo que,
sendo rico, por amor de vós se fez pobre; para que pela sua pobreza
enriquecêsseis. E nisto dou o meu parecer; pois isto convém a
vós que, desde o ano passado, começastes; e não foi só praticar, mas
também querer. Agora, porém, completai também o já começado,
para que, assim como houve a prontidão de vontade, haja também o
cumprimento, segundo o que tendes. Porque, se há prontidão de
vontade, será aceita segundo o que qualquer tem, e não segundo o que
não tem. Mas, não digo isto para que os outros tenham alívio,
e vós opressão, mas para igualdade; neste tempo presente, a
vossa abundância supra a falta dos outros, para que também a sua
abundância supra a vossa falta, e haja igualdade; como está
escrito: O que muito colheu não teve de mais; e o que pouco, não
teve de menos.

Mas, graças a Deus, que pôs a mesma solicitude por vós no coração
de Tito; pois aceitou a exortação, e muito diligente partiu
voluntariamente para vós. E com ele enviamos aquele irmão
cujo louvor no evangelho está espalhado em todas as igrejas.
E não só isto, mas foi também escolhido pelas igrejas para
companheiro da nossa viagem, nesta graça que por nós é ministrada
para glória do mesmo Senhor, e prontidão do vosso ânimo;
evitando isto, que alguém nos vitupere por esta abundância,
que por nós é ministrada; pois zelamos do que é honesto, não
só diante do Senhor, mas também diante dos homens. Com eles
enviamos também outro nosso irmão, o qual muitas vezes, e em muitas
coisas, já experimentamos ser diligente, e agora muito mais
diligente ainda pela muita confiança que em vós tem. Quanto a
Tito, é meu companheiro, e cooperador para convosco; quanto a nossos
irmãos, são embaixadores das igrejas e glória de Cristo.
Portanto, mostrai para com eles, e perante a face das
igrejas, a prova do vosso amor, e da nossa glória acerca de vós.

\medskip

\lettrine{9}\ Quanto à administração que se faz a favor dos
santos, não necessito escrever-vos; porque bem sei a prontidão
do vosso ânimo, da qual me glorio de vós para com os macedônios; que
a Acaia está pronta desde o ano passado; e o vosso zelo tem
estimulado muitos. Mas enviei estes irmãos, para que a nossa
glória, acerca de vós, não seja vã nesta parte; para que (como já
disse) possais estar prontos, a fim de, se acaso os macedônios
vierem comigo, e vos acharem desapercebidos, não nos envergonharmos
nós (para não dizermos vós) deste firme fundamento de glória.
Portanto, tive por coisa necessária exortar estes irmãos, para
que primeiro fossem ter convosco, e preparassem de antemão a vossa
bênção, já antes anunciada, para que esteja pronta como bênção, e
não como avareza.

E digo isto: Que o que semeia pouco, pouco também ceifará; e o que
semeia em abundância, em abundância ceifará. Cada um contribua
segundo propôs no seu coração; não com tristeza, ou por necessidade;
porque Deus ama ao que dá com alegria. E Deus é poderoso para
fazer abundar em vós toda a graça, a fim de que tendo sempre, em
tudo, toda a suficiência, abundeis em toda a boa obra; conforme
está escrito: Espalhou, deu aos pobres; a sua justiça permanece para
sempre. Ora, aquele que dá a semente ao que semeia, também
vos dê pão para comer, e multiplique a vossa sementeira, e aumente
os frutos da vossa justiça; para que em tudo enriqueçais para
toda a beneficência, a qual faz que por nós se dêem graças a Deus.
Porque a administração deste serviço, não só supre as
necessidades dos santos, mas também é abundante em muitas graças,
que se dão a Deus. Visto como, na prova desta administração,
glorificam a Deus pela submissão, que confessais quanto ao evangelho
de Cristo, e pela liberalidade de vossos dons para com eles, e para
com todos; e pela sua oração por vós, tendo de vós saudades,
por causa da excelente graça de Deus que em vós há. Graças a
Deus, pois, pelo seu dom inefável\footnote{Que não se pode exprimir
por palavras; indizível.}.

\medskip

\lettrine{10}\ Além disto, eu, Paulo, vos rogo, pela mansidão e
benignidade de Cristo, eu que, na verdade, quando presente entre
vós, sou humilde, mas ausente, ousado para convosco; rogo-vos,
pois, que, quando estiver presente, não me veja obrigado a usar com
confiança da ousadia que espero ter com alguns, que nos julgam, como
se andássemos segundo a carne. Porque, andando na carne, não
militamos segundo a carne. Porque as armas da nossa milícia não
são carnais, mas sim poderosas em Deus para destruição das
fortalezas; destruindo os conselhos, e toda a altivez que se
levanta contra o conhecimento de Deus, e levando cativo todo o
entendimento à obediência de Cristo; e estando prontos para
vingar toda a desobediência, quando for cumprida a vossa obediência.

Olhais para as coisas segundo a aparência? Se alguém confia de si
mesmo que é de Cristo, pense outra vez isto consigo, que, assim como
ele é de Cristo, também nós de Cristo somos. Porque, ainda que
eu me glorie mais alguma coisa do nosso poder, o qual o Senhor nos
deu para edificação, e não para vossa destruição, não me
envergonharei. Para que não pareça como se quisera intimidar-vos
por cartas. Porque as suas cartas, dizem, são graves e
fortes, mas a presença do corpo é fraca, e a palavra desprezível.
Pense o tal isto, que, quais somos na palavra por cartas,
estando ausentes, tais seremos também por obra, estando presentes.

Porque não ousamos classificar-nos, ou comparar-nos com alguns,
que se louvam a si mesmos; mas estes que se medem a si mesmos, e se
comparam consigo mesmos, estão sem entendimento. Porém, não
nos gloriaremos fora da medida, mas conforme a reta medida que Deus
nos deu, para chegarmos até vós; porque não nos estendemos
além do que convém, como se não houvéssemos de chegar até vós, pois
já chegamos também até vós no evangelho de Cristo, não nos
gloriando fora da medida nos trabalhos alheios; antes tendo
esperança de que, crescendo a vossa fé, seremos abundantemente
engrandecidos entre vós, conforme a nossa regra, para
anunciar o evangelho nos lugares que estão além de vós e não em
campo de outrem, para não nos gloriarmos no que estava já preparado.
Aquele, porém, que se gloria, glorie-se no Senhor.
Porque não é aprovado quem a si mesmo se louva, mas, sim,
aquele a quem o Senhor louva.

\medskip

\lettrine{11}\ Quisera eu me suportásseis um pouco na minha
loucura! Suportai-me, porém, ainda. Porque estou zeloso de vós
com zelo de Deus; porque vos tenho preparado para vos apresentar
como uma virgem pura a um marido, a saber, a Cristo. Mas temo
que, assim como a serpente enganou Eva com a sua astúcia, assim
também sejam de alguma sorte corrompidos os vossos sentidos, e se
apartem da simplicidade que há em Cristo. Porque, se alguém for
pregar-vos outro Jesus que nós não temos pregado, ou se recebeis
outro espírito que não recebestes, ou outro evangelho que não
abraçastes, com razão o sofrereis.

Porque penso que em nada fui inferior aos mais excelentes
apóstolos. E, se sou rude na palavra, não o sou contudo na
ciência; mas já em todas as coisas nos temos feito conhecer
totalmente entre vós. Pequei, porventura, humilhando-me a mim
mesmo, para que vós fôsseis exaltados, porque de graça vos anunciei
o evangelho de Deus? Outras igrejas despojei eu para vos servir,
recebendo delas salário; e quando estava presente convosco, e tinha
necessidade, a ninguém fui pesado. Porque os irmãos que vieram
da Macedônia supriram a minha necessidade; e em tudo me guardei de
vos ser pesado, e ainda me guardarei. Como a verdade de
Cristo está em mim, esta glória não me será impedida nas regiões da
Acaia. Por quê? Porque não vos amo? Deus o sabe. Mas o
que eu faço o farei, para cortar ocasião aos que buscam ocasião, a
fim de que, naquilo em que se gloriam, sejam achados assim como nós.
Porque tais falsos apóstolos são obreiros fraudulentos,
transfigurando-se em apóstolos de Cristo. E não é maravilha,
porque o próprio Satanás se transfigura em anjo de luz. Não é
muito, pois, que os seus ministros se transfigurem em ministros da
justiça; o fim dos quais será conforme as suas obras.

Outra vez digo: Ninguém me julgue insensato, ou então recebei-me
como insensato, para que também me glorie um pouco. O que
digo, não o digo segundo o Senhor, mas como por loucura, nesta
confiança de gloriar-me. Pois que muitos se gloriam segundo a
carne, eu também me gloriarei. Porque, sendo vós sensatos, de
boa mente tolerais os insensatos. Pois sois sofredores, se
alguém vos põe em servidão, se alguém vos devora, se alguém vos
apanha, se alguém se exalta, se alguém vos fere no rosto.
Envergonhado o digo, como se nós fôssemos fracos, mas no que
qualquer tem ousadia (com insensatez falo) também eu tenho ousadia.

São hebreus? também eu. São israelitas? também eu. São
descendência de Abraão? também eu. São ministros de Cristo?
(falo como fora de mim) eu ainda mais: em trabalhos, muito mais; em
açoites, mais do que eles; em prisões, muito mais; em perigo de
morte, muitas vezes. Recebi dos judeus cinco
quarentenas\footnote{Porção ou número de quarenta coisas.} de
açoites menos um. Três vezes fui açoitado com varas, uma vez
fui apedrejado, três vezes sofri naufrágio, uma noite e um dia
passei no abismo; em viagens muitas vezes, em perigos de
rios, em perigos de salteadores, em perigos dos da minha nação, em
perigos dos gentios, em perigos na cidade, em perigos no deserto, em
perigos no mar, em perigos entre os falsos irmãos; em
trabalhos e fadiga, em vigílias muitas vezes, em fome e sede, em
jejum muitas vezes, em frio e nudez. Além das coisas
exteriores, me oprime cada dia o cuidado de todas as igrejas.
Quem enfraquece, que eu também não enfraqueça? Quem se
escandaliza, que eu me não abrase? Se convém gloriar-me,
gloriar-me-ei no que diz respeito à minha fraqueza. O Deus e
Pai de nosso Senhor Jesus Cristo, que é eternamente bendito, sabe
que não minto. Em Damasco, o que governava sob o rei Aretas
pôs guardas às portas da cidade dos damascenos, para me prenderem.
E fui descido num cesto por uma janela da muralha; e assim
escapei das suas mãos.

\medskip

\lettrine{12}\ Em verdade que não convém gloriar-me; mas
passarei às visões e revelações do Senhor. Conheço um homem em
Cristo que há catorze anos (se no corpo, não sei, se fora do corpo,
não sei; Deus o sabe) foi arrebatado ao terceiro céu. E sei que
o tal homem (se no corpo, se fora do corpo, não sei; Deus o sabe)
foi arrebatado ao paraíso; e ouviu palavras inefáveis, que ao
homem não é lícito falar. De alguém assim me gloriarei eu, mas
de mim mesmo não me gloriarei, senão nas minhas fraquezas.
Porque, se quiser gloriar-me, não serei néscio, porque direi a
verdade; mas deixo isto, para que ninguém cuide de mim mais do que
em mim vê ou de mim ouve. E, para que não me exaltasse pela
excelência das revelações, foi-me dado um espinho na carne, a saber,
um mensageiro de Satanás para me esbofetear, a fim de não me
exaltar. Acerca do qual três vezes orei ao Senhor para que se
desviasse de mim. E disse-me: A minha graça te basta, porque o
meu poder se aperfeiçoa na fraqueza. De boa vontade, pois, me
gloriarei nas minhas fraquezas, para que em mim habite o poder de
Cristo. Por isso sinto prazer nas fraquezas, nas injúrias,
nas necessidades, nas perseguições, nas angústias por amor de
Cristo. Porque quando estou fraco então sou forte.

Fui néscio em gloriar-me; vós me constrangestes. Eu devia ter
sido louvado por vós, visto que em nada fui inferior aos mais
excelentes apóstolos, ainda que nada sou. Os sinais do meu
apostolado foram manifestados entre vós com toda a paciência, por
sinais, prodígios e maravilhas. Pois, em que tendes vós sido
inferiores às outras igrejas, a não ser que eu mesmo vos não fui
pesado? Perdoai-me este agravo. Eis aqui estou pronto para
pela terceira vez ir ter convosco, e não vos serei pesado, pois que
não busco o que é vosso, mas sim a vós: porque não devem os filhos
entesourar para os pais, mas os pais para os filhos. Eu de
muito boa vontade gastarei, e me deixarei gastar pelas vossas almas,
ainda que, amando-vos cada vez mais, seja menos amado. Mas
seja assim; eu não vos fui pesado mas, sendo astuto, vos tomei com
dolo. Porventura aproveitei-me de vós por algum daqueles que
vos enviei? Roguei a Tito, e enviei com ele um irmão.
Porventura Tito se aproveitou de vós? Não andamos porventura no
mesmo espírito, sobre as mesmas pisadas? Cuidais que ainda
nos desculpamos convosco? Falamos em Cristo perante Deus, e tudo
isto, ó amados, para vossa edificação. Porque receio que,
quando chegar, não vos ache como eu quereria, e eu seja achado de
vós como não quereríeis; que de alguma maneira haja pendências,
invejas, iras, porfias, detrações, mexericos, orgulhos, tumultos;
que, quando for outra vez, o meu Deus me humilhe para
convosco, e chore por muitos daqueles que dantes pecaram, e não se
arrependeram da imundícia, e prostituição, e desonestidade que
cometeram.

\medskip

\lettrine{13}\ É esta a terceira vez que vou ter convosco. Por
boca de duas ou três testemunhas será confirmada toda a palavra.
Já anteriormente o disse, e segunda vez o digo como quando
estava presente; mas agora, estando ausente, o escrevo aos que antes
pecaram e a todos os mais, que, se outra vez for, não lhes
perdoarei; visto que buscais uma prova de Cristo que fala em
mim, o qual não é fraco para convosco, antes é poderoso entre vós.
Porque, ainda que foi crucificado por fraqueza, vive, contudo,
pelo poder de Deus. Porque nós também somos fracos nele, mas
viveremos com ele pelo poder de Deus em vós. Examinai-vos a vós
mesmos, se permaneceis na fé; provai-vos a vós mesmos. Ou não sabeis
quanto a vós mesmos, que Jesus Cristo está em vós? Se não é que já
estais reprovados. Mas espero que entendereis que nós não somos
reprovados.

Ora, eu rogo a Deus que não façais mal algum, não para que sejamos
achados aprovados, mas para que vós façais o bem, embora nós sejamos
como reprovados. Porque nada podemos contra a verdade, senão
pela verdade. Porque nos regozijamos de estar fracos, quando vós
estais fortes; e o que desejamos é a vossa perfeição.
Portanto, escrevo estas coisas estando ausente, para que,
estando presente, não use de rigor, segundo o poder que o Senhor me
deu para edificação, e não para destruição.

Quanto ao mais, irmãos, regozijai-vos, sede perfeitos, sede
consolados, sede de um mesmo parecer, vivei em paz; e o Deus de amor
e de paz será convosco. Saudai-vos uns aos outros com ósculo
santo. Todos os santos vos saúdam. A graça do Senhor
Jesus Cristo, e o amor de Deus, e a comunhão do Espírito Santo seja
com todos vós. Amém.

