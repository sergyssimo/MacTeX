\thispagestyle{empty}
\chapter*{Segunda Epístola de Pedro}

\lettrine{1} Simão Pedro, servo e apóstolo de Jesus Cristo,
aos que conosco alcançaram fé igualmente preciosa pela justiça do
nosso Deus e Salvador Jesus Cristo: Graça e paz vos sejam
multiplicadas, pelo conhecimento de Deus, e de Jesus nosso Senhor;
visto como o seu divino poder nos deu tudo o que diz respeito à
vida e piedade, pelo conhecimento daquele que nos chamou pela sua
glória e virtude; pelas quais ele nos tem dado grandíssimas e
preciosas promessas, para que por elas fiqueis participantes da
natureza divina, havendo escapado da corrupção, que pela
concupiscência há no mundo.

E vós também, pondo nisto mesmo toda a diligência, acrescentai à
vossa fé a virtude, e à virtude a ciência, e à ciência a
temperança\footnote{Virtude que modera os apetites, as paixões.
Moderação no comer ou no beber, sobriedade. Parcimônia.}, e à
temperança a paciência, e à paciência a piedade, e à piedade o
amor fraternal, e ao amor fraternal a caridade. Porque, se em
vós houver e abundarem estas coisas, não vos deixarão ociosos nem
estéreis no conhecimento de nosso Senhor Jesus Cristo. Pois
aquele em quem não há estas coisas é cego, nada vendo ao longe,
havendo-se esquecido da purificação dos seus antigos pecados.
Portanto, irmãos, procurai fazer cada vez mais firme a vossa
vocação e eleição; porque, fazendo isto, nunca jamais tropeçareis.
Porque assim vos será amplamente concedida a entrada no reino
eterno de nosso Senhor e Salvador Jesus Cristo.

Por isso não deixarei de exortar-vos sempre acerca destas coisas,
ainda que bem as saibais, e estejais confirmados na presente
verdade. E tenho por justo, enquanto estiver neste
tabernáculo, despertar-vos com admoestações, sabendo que
brevemente hei de deixar este meu tabernáculo, como também nosso
Senhor Jesus Cristo já mo tem revelado. Mas também eu
procurarei em toda a ocasião que depois da minha morte tenhais
lembrança destas coisas.

Porque não vos fizemos saber a virtude e a vinda de nosso Senhor
Jesus Cristo, seguindo fábulas artificialmente compostas; mas nós
mesmos vimos a sua majestade. Porquanto ele recebeu de Deus
Pai honra e glória, quando da magnífica glória lhe foi dirigida a
seguinte voz: Este é o meu Filho amado, em quem me tenho comprazido.
E ouvimos esta voz dirigida do céu, estando nós com ele no
monte santo.

E temos, mui firme, a palavra dos profetas, à qual bem fazeis em
estar atentos, como a uma luz que alumia em lugar escuro, até que o
dia amanheça, e a estrela da alva apareça em vossos corações.
Sabendo primeiramente isto: que nenhuma profecia da Escritura
é de particular interpretação. Porque a profecia nunca foi
produzida por vontade de homem algum, mas os homens santos de Deus
falaram inspirados pelo Espírito Santo.

\medskip

\lettrine{2} E também houve entre o povo falsos profetas, como
entre vós haverá também falsos doutores, que introduzirão
encobertamente heresias de perdição, e negarão o Senhor que os
resgatou, trazendo sobre si mesmos repentina perdição. E muitos
seguirão as suas dissoluções\footnote{Dissolução: perversão de
costumes; devassidão; libertinagem.}, pelos quais será blasfemado o
caminho da verdade.

E por avareza farão de vós negócio com palavras fingidas; sobre os
quais já de largo tempo não será tardia a sentença, e a sua perdição
não dormita. Porque, se Deus não perdoou aos anjos que pecaram,
mas, havendo-os lançado no inferno, os entregou às cadeias da
escuridão, ficando reservados para o juízo; e não perdoou ao
mundo antigo, mas guardou a Noé, pregoeiro da justiça, com mais sete
pessoas, ao trazer o dilúvio sobre o mundo dos ímpios; e
condenou à destruição as cidades de Sodoma e Gomorra, reduzindo-as a
cinza, e pondo-as para exemplo aos que vivessem impiamente

E livrou o justo Ló, enfadado da vida dissoluta dos homens
abomináveis (porque este justo, habitando entre eles, afligia
todos os dias a sua alma justa, vendo e ouvindo sobre as suas obras
injustas); assim, sabe o Senhor livrar da tentação os piedosos,
e reservar os injustos para o dia do juízo, para serem castigados.

Mas principalmente aqueles que segundo a carne andam em
concupiscências de imundícia, e desprezam as autoridades; atrevidos,
obstinados, não receando blasfemar das dignidades; enquanto
os anjos, sendo maiores em força e poder, não pronunciam contra eles
juízo blasfemo diante do Senhor. Mas estes, como animais
irracionais, que seguem a natureza, feitos para serem presos e
mortos, blasfemando do que não entendem, perecerão na sua corrupção,
recebendo o galardão da injustiça; pois que tais homens têm
prazer nos deleites quotidianos; nódoas são eles e máculas,
deleitando-se em seus enganos, quando se banqueteiam convosco;
tendo os olhos cheios de adultério, e não cessando de pecar,
engodando as almas inconstantes, tendo o coração exercitado na
avareza, filhos de maldição; os quais, deixando o caminho
direito, erraram seguindo o caminho de Balaão, filho de Beor, que
amou o prêmio da injustiça; mas teve a repreensão da sua
transgressão; o mudo jumento, falando com voz humana, impediu a
loucura do profeta. Estes são fontes sem água, nuvens levadas
pela força do vento, para os quais a escuridão das trevas
eternamente se reserva. Porque, falando coisas mui arrogantes
de vaidades, engodam com as concupiscências da carne, e com
dissoluções, aqueles que se estavam afastando dos que andam em erro,
prometendo-lhes liberdade, sendo eles mesmos servos da
corrupção. Porque de quem alguém é vencido, do tal faz-se também
servo. Porquanto se, depois de terem escapado das corrupções
do mundo, pelo conhecimento do Senhor e Salvador Jesus Cristo, forem
outra vez envolvidos nelas e vencidos, tornou-se-lhes o último
estado pior do que o primeiro. Porque melhor lhes fora não
conhecerem o caminho da justiça, do que, conhecendo-o, desviarem-se
do santo mandamento que lhes fora dado; deste modo
sobreveio-lhes o que por um verdadeiro provérbio se diz: O cão
voltou ao seu próprio vômito, e a porca lavada ao
espojadouro\footnote{Lugar onde se espojam animais; espojeiro.
Espojar: Reduzir a pó; pulverizar. Fazer cair na terra. Estender-se
e rebolar-se no chão; espolinhar-se, rojar-se.} de lama.

\medskip

\lettrine{3} Amados, escrevo-vos agora esta segunda carta, em
ambas as quais desperto com exortação o vosso ânimo sincero;
para que vos lembreis das palavras que primeiramente foram ditas
pelos santos profetas, e do nosso mandamento, como apóstolos do
Senhor e Salvador.

Sabendo primeiro isto, que nos últimos dias virão escarnecedores,
andando segundo as suas próprias concupiscências, e dizendo:
Onde está a promessa da sua vinda? porque desde que os pais
dormiram, todas as coisas permanecem como desde o princípio da
criação. Eles voluntariamente ignoram isto, que pela palavra de
Deus já desde a antiguidade existiram os céus, e a terra, que foi
tirada da água e no meio da água subsiste. Pelas quais coisas
pereceu o mundo de então, coberto com as águas do dilúvio, mas
os céus e a terra que agora existem pela mesma palavra se reservam
como tesouro, e se guardam para o fogo, até o dia do juízo, e da
perdição dos homens ímpios.

Mas, amados, não ignoreis uma coisa, que um dia para o Senhor é
como mil anos, e mil anos como um dia.

O Senhor não retarda a sua promessa, ainda que alguns a têm por
tardia; mas é longânimo para conosco, não querendo que alguns se
percam, senão que todos venham a arrepender-se. Mas o dia do
Senhor virá como o ladrão de noite; no qual os céus passarão com
grande estrondo, e os elementos, ardendo, se desfarão, e a terra, e
as obras que nela há, se queimarão.

Havendo, pois, de perecer todas estas coisas, que pessoas vos
convém ser em santo trato, e piedade, aguardando, e
apressando-vos para a vinda do dia de Deus, em que os céus, em fogo
se desfarão, e os elementos, ardendo, se fundirão? Mas nós,
segundo a sua promessa, aguardamos novos céus e nova terra, em que
habita a justiça. Por isso, amados, aguardando estas coisas,
procurai que dele sejais achados imaculados e irrepreensíveis em
paz. E tende por salvação a longanimidade de nosso Senhor;
como também o nosso amado irmão Paulo vos escreveu, segundo a
sabedoria que lhe foi dada; falando disto, como em todas as
suas epístolas, entre as quais há pontos difíceis de entender, que
os indoutos e inconstantes torcem, e igualmente as outras
Escrituras, para sua própria perdição. Vós, portanto, amados,
sabendo isto de antemão, guardai-vos de que, pelo engano dos homens
abomináveis, sejais juntamente arrebatados, e descaiais da vossa
firmeza; antes crescei na graça e conhecimento de nosso
Senhor e Salvador, Jesus Cristo. A ele seja dada a glória, assim
agora, como no dia da eternidade. Amém.

