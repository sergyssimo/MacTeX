\thispagestyle{empty}
\chapter*{Epístola aos Gálatas}

\lettrine{1} Paulo, apóstolo (não da parte dos homens, nem por
homem algum, mas por Jesus Cristo, e por Deus Pai, que o ressuscitou
dentre os mortos), e todos os irmãos que estão comigo, às
igrejas da Galácia: Graça e paz da parte de Deus Pai e do nosso
Senhor Jesus Cristo, o qual se deu a si mesmo por nossos
pecados, para nos livrar do presente século mau, segundo a vontade
de Deus nosso Pai, ao qual seja dada glória para todo o sempre.
Amém.

Maravilho-me de que tão depressa passásseis daquele que vos chamou
à graça de Cristo para outro evangelho; o qual não é outro, mas
há alguns que vos inquietam e querem transtornar o evangelho de
Cristo. Mas, ainda que nós mesmos ou um anjo do céu vos anuncie
outro evangelho além do que já vos tenho anunciado, seja anátema.
Assim, como já vo-lo dissemos, agora de novo também vo-lo digo.
Se alguém vos anunciar outro evangelho além do que já recebestes,
seja anátema.

Porque, persuado eu agora a homens ou a Deus? ou procuro agradar
a homens? Se estivesse ainda agradando aos homens, não seria servo
de Cristo. Mas faço-vos saber, irmãos, que o evangelho que
por mim foi anunciado não é segundo os homens. Porque não o
recebi, nem aprendi de homem algum, mas pela revelação de Jesus
Cristo. Porque já ouvistes qual foi antigamente a minha
conduta no judaísmo, como sobremaneira perseguia a igreja de Deus e
a assolava. E na minha nação excedia em judaísmo a muitos da
minha idade, sendo extremamente zeloso das tradições de meus pais.
Mas, quando aprouve a Deus, que desde o ventre de minha mãe
me separou, e me chamou pela sua graça, revelar seu Filho em
mim, para que o pregasse entre os gentios, não consultei a carne nem
o sangue, nem tornei a Jerusalém, a ter com os que já antes
de mim eram apóstolos, mas parti para a Arábia, e voltei outra vez a
Damasco. Depois, passados três anos, fui a Jerusalém para ver
a Pedro, e fiquei com ele quinze dias. E não vi a nenhum
outro dos apóstolos, senão a Tiago, irmão do Senhor. Ora,
acerca do que vos escrevo, eis que diante de Deus testifico que não
minto. Depois fui para as partes da Síria e da Cilícia.
E não era conhecido de vista das igrejas da Judéia, que
estavam em Cristo; mas somente tinham ouvido dizer: Aquele
que já nos perseguiu anuncia agora a fé que antes destruía. E
glorificavam a Deus a respeito de mim.

\medskip

\lettrine{2} Depois, passados catorze anos, subi outra vez a
Jerusalém com Barnabé, levando também comigo Tito. E subi por
uma revelação, e lhes expus o evangelho, que prego entre os gentios,
e particularmente aos que estavam em estima; para que de maneira
alguma não corresse ou não tivesse corrido em vão. Mas nem ainda
Tito, que estava comigo, sendo grego, foi constrangido a
circuncidar-se; e isto por causa dos falsos irmãos que se
intrometeram, e secretamente entraram a espiar a nossa liberdade,
que temos em Cristo Jesus, para nos porem em servidão; aos quais
nem ainda por uma hora cedemos com sujeição, para que a verdade do
evangelho permanecesse entre vós. E, quanto àqueles que pareciam
ser alguma coisa (quais tenham sido noutro tempo, não se me dá; Deus
não aceita a aparência do homem), esses, digo, que pareciam ser
alguma coisa, nada me comunicaram; antes, pelo contrário, quando
viram que o evangelho da incircuncisão me estava confiado, como a
Pedro o da circuncisão (porque aquele que operou eficazmente em
Pedro para o apostolado da circuncisão, esse operou também em mim
com eficácia para com os gentios), e conhecendo Tiago, Cefas e
João, que eram considerados como as colunas, a graça que me havia
sido dada, deram-nos as destras, em comunhão comigo e com Barnabé,
para que nós fôssemos aos gentios, e eles à circuncisão;
recomendando-nos somente que nos lembrássemos dos pobres, o
que também procurei fazer com diligência.

E, chegando Pedro à Antioquia, lhe resisti na cara, porque era
repreensível. Porque, antes que alguns tivessem chegado da
parte de Tiago, comia com os gentios; mas, depois que chegaram, se
foi retirando, e se apartou deles, temendo os que eram da
circuncisão. E os outros judeus também dissimulavam com ele,
de maneira que até Barnabé se deixou levar pela sua dissimulação.
Mas, quando vi que não andavam bem e direitamente conforme a
verdade do evangelho, disse a Pedro na presença de todos: Se tu,
sendo judeu, vives como os gentios, e não como judeu, por que
obrigas os gentios a viverem como judeus? Nós somos judeus
por natureza, e não pecadores dentre os gentios. Sabendo que
o homem não é justificado pelas obras da lei, mas pela fé em Jesus
Cristo, temos também crido em Jesus Cristo, para sermos justificados
pela fé em Cristo, e não pelas obras da lei; porquanto pelas obras
da lei nenhuma carne será justificada. Pois, se nós, que
procuramos ser justificados em Cristo, nós mesmos também somos
achados pecadores, é porventura Cristo ministro do pecado? De
maneira nenhuma. Porque, se torno a edificar aquilo que
destruí, constituo-me a mim mesmo transgressor. Porque eu,
pela lei, estou morto para a lei, para viver para Deus. Já
estou crucificado com Cristo; e vivo, não mais eu, mas Cristo vive
em mim; e a vida que agora vivo na carne, vivo-a na fé do Filho de
Deus, o qual me amou, e se entregou a si mesmo por mim. Não
aniquilo a graça de Deus; porque, se a justiça provém da lei,
segue-se que Cristo morreu debalde.

\medskip

\lettrine{3}\ Ó insensatos gálatas! Quem vos fascinou para não
obedecerdes à verdade, a vós, perante os olhos de quem Jesus Cristo
foi evidenciado, crucificado, entre vós? Só quisera saber isto
de vós: recebestes o Espírito pelas obras da lei ou pela pregação da
fé? Sois vós tão insensatos que, tendo começado pelo Espírito,
acabeis agora pela carne? Será em vão que tenhais padecido
tanto? Se é que isso também foi em vão. Aquele, pois, que vos dá
o Espírito, e que opera maravilhas entre vós, fá-lo pelas obras da
lei, ou pela pregação da fé?

Assim como Abraão creu em Deus, e isso lhe foi imputado como
justiça. Sabei, pois, que os que são da fé são filhos de Abraão.
Ora, tendo a Escritura previsto que Deus havia de justificar
pela fé os gentios, anunciou primeiro o evangelho a Abraão, dizendo:
Todas as nações serão benditas em ti. De sorte que os que são da
fé são benditos com o crente Abraão. Todos aqueles, pois, que
são das obras da lei estão debaixo da maldição; porque está escrito:
Maldito todo aquele que não permanecer em todas as coisas que estão
escritas no livro da lei, para fazê-las. E é evidente que
pela lei ninguém será justificado diante de Deus, porque o justo
viverá da fé. Ora, a lei não é da fé; mas o homem, que fizer
estas coisas, por elas viverá. Cristo nos resgatou da
maldição da lei, fazendo-se maldição por nós; porque está escrito:
Maldito todo aquele que for pendurado no madeiro; para que a
bênção de Abraão chegasse aos gentios por Jesus Cristo, e para que
pela fé nós recebamos a promessa do Espírito. Irmãos, como
homem falo; se a aliança de um homem for confirmada, ninguém a anula
nem a acrescenta. Ora, as promessas foram feitas a Abraão e à
sua descendência. Não diz: E às descendências, como falando de
muitas, mas como de uma só: E à tua descendência, que é Cristo.
Mas digo isto: Que tendo sido a aliança anteriormente
confirmada por Deus em Cristo, a lei, que veio quatrocentos e trinta
anos depois, não a invalida, de forma a abolir a promessa.
Porque, se a herança provém da lei, já não provém da
promessa; mas Deus pela promessa a deu gratuitamente a Abraão.

Logo, para que é a lei? Foi ordenada por causa das transgressões,
até que viesse a posteridade a quem a promessa tinha sido feita; e
foi posta pelos anjos na mão de um medianeiro. Ora, o
medianeiro não o é de um só, mas Deus é um. Logo, a lei é
contra as promessas de Deus? De nenhuma sorte; porque, se fosse dada
uma lei que pudesse vivificar, a justiça, na verdade, teria sido
pela lei. Mas a Escritura encerrou tudo debaixo do pecado,
para que a promessa pela fé em Jesus Cristo fosse dada aos crentes.
Mas, antes que a fé viesse, estávamos guardados debaixo da
lei, e encerrados para aquela fé que se havia de manifestar.
De maneira que a lei nos serviu de aio, para nos conduzir a
Cristo, para que pela fé fôssemos justificados. Mas, depois
que veio a fé, já não estamos debaixo de aio. Porque todos
sois filhos de Deus pela fé em Cristo Jesus. Porque todos
quantos fostes batizados em Cristo já vos revestistes de Cristo.
Nisto não há judeu nem grego; não há servo nem livre; não há
macho nem fêmea; porque todos vós sois um em Cristo Jesus. E,
se sois de Cristo, então sois descendência de Abraão, e herdeiros
conforme a promessa.

\medskip

\lettrine{4} Digo, pois, que todo o tempo que o herdeiro é
menino em nada difere do servo, ainda que seja senhor de tudo;
mas está debaixo de tutores e curadores até ao tempo determinado
pelo pai. Assim também nós, quando éramos meninos, estávamos
reduzidos à servidão debaixo dos primeiros rudimentos do mundo.
Mas, vindo a plenitude dos tempos, Deus enviou seu Filho,
nascido de mulher, nascido sob a lei, para remir os que estavam
debaixo da lei, a fim de recebermos a adoção de filhos. E,
porque sois filhos, Deus enviou aos vossos corações o Espírito de
seu Filho, que clama: Aba, Pai. Assim que já não és mais servo,
mas filho; e, se és filho, és também herdeiro de Deus por Cristo.

Mas, quando não conhecíeis a Deus, servíeis aos que por natureza
não são deuses. Mas agora, conhecendo a Deus, ou, antes, sendo
conhecidos por Deus, como tornais outra vez a esses rudimentos
fracos e pobres, aos quais de novo quereis servir? Guardais
dias, e meses, e tempos, e anos. Receio de vós, que não haja
trabalhado em vão para convosco.

Irmãos, rogo-vos que sejais como eu, porque também eu sou como
vós; nenhum mal me fizestes. E vós sabeis que primeiro vos
anunciei o evangelho estando em fraqueza da carne; e não
rejeitastes, nem desprezastes isso que era uma tentação na minha
carne, antes me recebestes como um anjo de Deus, como Jesus Cristo
mesmo. Qual é, logo, a vossa bem-aventurança? Porque vos dou
testemunho de que, se possível fora, arrancaríeis os vossos olhos, e
mos daríeis. Fiz-me acaso vosso inimigo, dizendo a verdade?

Eles têm zelo por vós, não como convém; mas querem excluir-vos,
para que vós tenhais zelo por eles. É bom ser zeloso, mas
sempre do bem, e não somente quando estou presente convosco.

Meus filhinhos, por quem de novo sinto as dores de parto, até que
Cristo seja formado em vós; eu bem quisera agora estar
presente convosco, e mudar a minha voz; porque estou perplexo a
vosso respeito.

Dizei-me, os que quereis estar debaixo da lei, não ouvis vós a
lei? Porque está escrito que Abraão teve dois filhos, um da
escrava, e outro da livre. Todavia, o que era da escrava
nasceu segundo a carne, mas, o que era da livre, por promessa.
O que se  entende por alegoria; porque estas são as duas
alianças; uma, do monte Sinai, gerando filhos para a servidão, que é
Agar. Ora, esta Agar é Sinai, um monte da Arábia, que
corresponde à Jerusalém que agora existe, pois é escrava com seus
filhos. Mas a Jerusalém que é de cima é livre; a qual é mãe
de todos nós. Porque está escrito: Alegra-te, estéril, que
não dás à luz; esforça-te e clama, tu que não estás de parto; porque
os filhos da solitária são mais do que os da que tem marido.
Mas nós, irmãos, somos filhos da promessa como Isaque.
Mas, como então aquele que era gerado segundo a carne
perseguia o que o era segundo o Espírito, assim é também agora.
Mas que diz a Escritura? Lança fora a escrava e seu filho,
porque de modo algum o filho da escrava herdará com o filho da
livre. De maneira que, irmãos, somos filhos, não da escrava,
mas da livre.

\medskip

\lettrine{5} Estai, pois, firmes na liberdade com que Cristo
nos libertou, e não torneis a colocar-vos debaixo do jugo da
servidão. Eis que eu, Paulo, vos digo que, se vos deixardes
circuncidar, Cristo de nada vos aproveitará. E de novo protesto
a todo o homem, que se deixa circuncidar, que está obrigado a
guardar toda a lei. Separados estais de Cristo, vós os que vos
justificais pela lei; da graça tendes caído. Porque nós pelo
Espírito da fé aguardamos a esperança da justiça. Porque em
Jesus Cristo nem a circuncisão nem a incircuncisão tem valor algum;
mas sim a fé que opera pelo amor. Corríeis bem; quem vos
impediu, para que não obedeçais à verdade? Esta persuasão não
vem daquele que vos chamou. Um pouco de fermento leveda toda a
massa. Confio de vós, no Senhor, que nenhuma outra coisa
sentireis; mas aquele que vos inquieta, seja ele quem for, sofrerá a
condenação. Eu, porém, irmãos, se prego ainda a circuncisão,
por que sou, pois, perseguido? Logo o escândalo da cruz está
aniquilado. Eu quereria que fossem cortados aqueles que vos
andam inquietando.

Porque vós, irmãos, fostes chamados à liberdade. Não useis então
da liberdade para dar ocasião à carne, mas servi-vos uns aos outros
pelo amor. Porque toda a lei se cumpre numa só palavra,
nesta: Amarás ao teu próximo como a ti mesmo. Se vós, porém,
vos mordeis e devorais uns aos outros, vede não vos consumais também
uns aos outros. Digo, porém: Andai em Espírito, e não
cumprireis a concupiscência da carne. Porque a carne cobiça
contra o Espírito, e o Espírito contra a carne; e estes opõem-se um
ao outro, para que não façais o que quereis. Mas, se sois
guiados pelo Espírito, não estais debaixo da lei. Porque as
obras da carne são manifestas, as quais são: adultério,
prostituição, impureza, lascívia, idolatria, feitiçaria,
inimizades, porfias\footnote{Contenda obstinada de palavras;
discussão, disputa, polêmica.}, emulações\footnote{Emulação: ato ou
efeito de emular: sentimento que leva o indivíduo a tentar
igualar-se a ou superar outrem. Competição, disputa, concorrência
(ger. em sentido moralmente sadio, sem sentimentos baixos ou
violência). Atitude que, determinada por rivalidade, competição,
ciúme, inveja etc., leva alguém a recorrer à justiça em busca de um
direito que sabe inexistente.}, iras, pelejas,
dissensões\footnote{Falta de concordância a respeito de (algo);
divergência, discrepância. Estado de litígio; desavença, conflito,
disputa. Característica daquilo que discrepa; oposição.}, heresias,
invejas, homicídios, bebedices, glutonarias, e coisas
semelhantes a estas, acerca das quais vos declaro, como já antes vos
disse, que os que cometem tais coisas não herdarão o reino de Deus.
Mas o fruto do Espírito é: amor, gozo, paz, longanimidade,
benignidade, bondade, fé, mansidão, temperança. Contra estas
coisas não há lei. E os que são de Cristo crucificaram a
carne com as suas paixões e concupiscências. Se vivemos em
Espírito, andemos também em Espírito. Não sejamos cobiçosos
de vanglórias, irritando-nos uns aos outros, invejando-nos uns aos
outros.

\medskip

\lettrine{6} Irmãos, se algum homem chegar a ser surpreendido
nalguma ofensa, vós, que sois espirituais, encaminhai o tal com
espírito de mansidão; olhando por ti mesmo, para que não sejas
também tentado. Levai as cargas uns dos outros, e assim
cumprireis a lei de Cristo. Porque, se alguém cuida ser alguma
coisa, não sendo nada, engana-se a si mesmo. Mas prove cada um a
sua própria obra, e terá glória só em si mesmo, e não noutro.
Porque cada qual levará a sua própria carga. E o que é
instruído na palavra reparta de todos os seus bens com aquele que o
instrui. Não erreis: Deus não se deixa escarnecer; porque tudo o
que o homem semear, isso também ceifará. Porque o que semeia na
sua carne, da carne ceifará a corrupção; mas o que semeia no
Espírito, do Espírito ceifará a vida eterna. E não nos cansemos
de fazer bem, porque a seu tempo ceifaremos, se não houvermos
desfalecido. Então, enquanto temos tempo, façamos bem a
todos, mas principalmente aos domésticos da fé.

Vede com que grandes letras vos escrevi por minha mão.
Todos os que querem mostrar boa aparência na carne, esses vos
obrigam a circuncidar-vos, somente para não serem perseguidos por
causa da cruz de Cristo. Porque nem ainda esses mesmos que se
circuncidam guardam a lei, mas querem que vos circuncideis, para se
gloriarem na vossa carne. Mas longe esteja de mim gloriar-me,
a não ser na cruz de nosso Senhor Jesus Cristo, pela qual o mundo
está crucificado para mim e eu para o mundo. Porque em Cristo
Jesus nem a circuncisão, nem a incircuncisão tem virtude alguma, mas
sim o ser uma nova criatura. E a todos quantos andarem
conforme esta regra, paz e misericórdia sobre eles e sobre o Israel
de Deus. Desde agora ninguém me inquiete; porque trago no meu
corpo as marcas do Senhor Jesus. A graça de nosso Senhor
Jesus Cristo seja, irmãos, com o vosso espírito! Amém.

