\thispagestyle{empty}
\chapter*{Epístola aos Hebreus}

\lettrine{1} Havendo Deus antigamente falado muitas vezes, e
de muitas maneiras, aos pais, pelos profetas, a nós falou-nos nestes
últimos dias pelo Filho, a quem constituiu herdeiro de tudo, por
quem fez também o mundo. O qual, sendo o resplendor da sua
glória, e a expressa imagem da sua pessoa, e sustentando todas as
coisas pela palavra do seu poder, havendo feito por si mesmo a
purificação dos nossos pecados, assentou-se à destra da majestade
nas alturas.

Feito tanto mais excelente do que os anjos, quanto herdou mais
excelente nome do que eles. Porque, a qual dos anjos disse
jamais: Tu és meu Filho, hoje te gerei? E outra vez: Eu lhe serei
por Pai, e ele me será por Filho? E outra vez, quando introduz
no mundo o primogênito, diz: E todos os anjos de Deus o adorem.
E, quanto aos anjos, diz: Faz dos seus anjos espíritos, e de
seus ministros labareda de fogo. Mas, do Filho, diz: Ó Deus, o
teu trono subsiste pelos séculos dos séculos; cetro de eqüidade é o
cetro do teu reino. Amaste a justiça e odiaste a iniqüidade; por
isso Deus, o teu Deus, te ungiu com óleo de alegria mais do que a
teus companheiros. E: Tu, Senhor, no princípio fundaste a
terra, e os céus são obra de tuas mãos. Eles perecerão, mas
tu permanecerás; e todos eles, como roupa, envelhecerão, e
como um manto os enrolarás, e serão mudados. Mas tu és o mesmo, e os
teus anos não acabarão. E a qual dos anjos disse jamais:
Assenta-te à minha destra, até que ponha a teus inimigos por
escabelo de teus pés? Não são porventura todos eles espíritos
ministradores, enviados para servir a favor daqueles que hão de
herdar a salvação?

\medskip

\lettrine{2} Portanto, convém-nos atentar com mais diligência
para as coisas que já temos ouvido, para que em tempo algum nos
desviemos delas. Porque, se a palavra falada pelos anjos
permaneceu firme, e toda a transgressão e desobediência recebeu a
justa retribuição, como escaparemos nós, se não atentarmos para
uma tão grande salvação, a qual, começando a ser anunciada pelo
Senhor, foi-nos depois confirmada pelos que a ouviram;
testificando também Deus com eles, por sinais, e milagres, e
várias maravilhas e dons do Espírito Santo, distribuídos por sua
vontade?

Porque não foi aos anjos que sujeitou o mundo futuro, de que
falamos. Mas em certo lugar testificou alguém, dizendo: Que é o
homem, para que dele te lembres? Ou o filho do homem, para que o
visites? Tu o fizeste um pouco menor do que os anjos, de glória
e de honra o coroaste, e o constituíste sobre as obras de tuas mãos.
Todas as coisas lhe sujeitaste debaixo dos pés. Ora, visto que
lhe sujeitou todas as coisas, nada deixou que lhe não esteja
sujeito. Mas agora ainda não vemos que todas as coisas lhe estejam
sujeitas. Vemos, porém, coroado de glória e de honra aquele
Jesus que fora feito um pouco menor do que os anjos, por causa da
paixão da morte, para que, pela graça de Deus, provasse a morte por
todos.

Porque convinha que aquele, para quem são todas as coisas, e
mediante quem tudo existe, trazendo muitos filhos à glória,
consagrasse pelas aflições o príncipe da salvação deles.
Porque, assim o que santifica, como os que são santificados,
são todos de um; por cuja causa não se envergonha de lhes chamar
irmãos, dizendo: Anunciarei o teu nome a meus irmãos,
cantar-te-ei louvores no meio da congregação. E outra vez:
Porei nele a minha confiança. E outra vez: Eis-me aqui a mim, e aos
filhos que Deus me deu.

E, visto como os filhos participam da carne e do sangue, também
ele participou das mesmas coisas, para que pela morte aniquilasse o
que tinha o império da morte, isto é, o diabo; e livrasse
todos os que, com medo da morte, estavam por toda a vida sujeitos à
servidão. Porque, na verdade, ele não tomou os anjos, mas
tomou a descendência de Abraão. Por isso convinha que em tudo
fosse semelhante aos irmãos, para ser misericordioso e fiel sumo
sacerdote naquilo que é de Deus, para expiar os pecados do povo.
Porque naquilo que ele mesmo, sendo tentado, padeceu, pode
socorrer aos que são tentados.

\medskip

\lettrine{3} Por isso, irmãos santos, participantes da vocação
celestial, considerai a Jesus Cristo, apóstolo e sumo sacerdote da
nossa confissão, sendo fiel ao que o constituiu, como também o
foi Moisés em toda a sua casa. Porque ele é tido por digno de
tanto maior glória do que Moisés, quanto maior honra do que a casa
tem aquele que a edificou. Porque toda a casa é edificada por
alguém, mas o que edificou todas as coisas é Deus. E, na
verdade, Moisés foi fiel em toda a sua casa, como servo, para
testemunho das coisas que se haviam de anunciar; mas Cristo,
como Filho, sobre a sua própria casa; a qual casa somos nós, se tão
somente conservarmos firme a confiança e a glória da esperança até
ao fim.

Portanto, como diz o Espírito Santo: Se ouvirdes hoje a sua voz,
não endureçais os vossos corações, como na provocação, no dia da
tentação no deserto, onde vossos pais me tentaram, me provaram,
e viram por quarenta anos as minhas obras. Por isso me
indignei contra esta geração, e disse: Estes sempre erram em seu
coração, e não conheceram os meus caminhos. Assim jurei na
minha ira que não entrarão no meu repouso. Vede, irmãos, que
nunca haja em qualquer de vós um coração mau e infiel, para se
apartar do Deus vivo. Antes, exortai-vos uns aos outros todos
os dias, durante o tempo que se chama Hoje, para que nenhum de vós
se endureça pelo engano do pecado; porque nos tornamos
participantes de Cristo, se retivermos firmemente o princípio da
nossa confiança até ao fim. Enquanto se diz: Hoje, se
ouvirdes a sua voz, não endureçais os vossos corações, como na
provocação. Porque, havendo-a alguns ouvido, o provocaram;
mas não todos os que saíram do Egito por meio de Moisés. Mas
com quem se indignou por quarenta anos? Não foi porventura com os
que pecaram, cujos corpos caíram no deserto? E a quem jurou
que não entrariam no seu repouso, senão aos que foram desobedientes?
E vemos que não puderam entrar por causa da sua
incredulidade.

\medskip

\lettrine{4} Temamos, pois, que, porventura, deixada a
promessa de entrar no seu repouso, pareça que algum de vós fica para
trás. Porque também a nós foram pregadas as boas novas, como a
eles, mas a palavra da pregação nada lhes aproveitou, porquanto não
estava misturada com a fé naqueles que a ouviram. Porque nós, os
que temos crido, entramos no repouso, tal como disse: Assim jurei na
minha ira que não entrarão no meu repouso; embora as suas obras
estivessem acabadas desde a fundação do mundo. Porque em certo
lugar disse assim do dia sétimo: E repousou Deus de todas as suas
obras no sétimo dia. E outra vez neste lugar: Não entrarão no
meu repouso. Visto, pois, que resta que alguns entrem nele, e
que aqueles a quem primeiro foram pregadas as boas novas não
entraram por causa da desobediência, determina outra vez um
certo dia, Hoje, dizendo por Davi, muito tempo depois, como está
dito: Hoje, se ouvirdes a sua voz, não endureçais os vossos
corações. Porque, se Josué lhes houvesse dado repouso, não
falaria depois disso de outro dia. Portanto, resta ainda um
repouso para o povo de Deus. Porque aquele que entrou no seu
repouso, ele próprio repousou de suas obras, como Deus das suas.

Procuremos, pois, entrar naquele repouso, para que ninguém caia
no mesmo exemplo de desobediência. Porque a palavra de Deus é
viva e eficaz, e mais penetrante do que espada alguma de dois gumes,
e penetra até à divisão da alma e do espírito, e das juntas e
medulas, e é apta para discernir os pensamentos e intenções do
coração. E não há criatura alguma encoberta diante dele;
antes todas as coisas estão nuas e patentes aos olhos daquele com
quem temos de tratar. Visto que temos um grande sumo
sacerdote, Jesus, Filho de Deus, que penetrou nos céus, retenhamos
firmemente a nossa confissão. Porque não temos um sumo
sacerdote que não possa compadecer-se das nossas fraquezas; porém,
um que, como nós, em tudo foi tentado, mas sem pecado.
Cheguemos, pois, com confiança ao trono da graça, para que
possamos alcançar misericórdia e achar graça, a fim de sermos
ajudados em tempo oportuno.

\medskip

\lettrine{5} Porque todo o sumo sacerdote, tomado dentre os
homens, é constituído a favor dos homens nas coisas concernentes a
Deus, para que ofereça dons e sacrifícios pelos pecados; e possa
compadecer-se ternamente dos ignorantes e errados; pois também ele
mesmo está rodeado de fraqueza. E por esta causa deve ele, tanto
pelo povo, como também por si mesmo, fazer oferta pelos pecados.
E ninguém toma para si esta honra, senão o que é chamado por
Deus, como Arão. Assim também Cristo não se glorificou a si
mesmo, para se fazer sumo sacerdote, mas aquele que lhe disse: Tu és
meu Filho, hoje te gerei. Como também diz, noutro lugar: Tu és
sacerdote eternamente, segundo a ordem de Melquisedeque. O qual,
nos dias da sua carne, oferecendo, com grande clamor e lágrimas,
orações e súplicas ao que o podia livrar da morte, foi ouvido quanto
ao que temia. Ainda que era Filho, aprendeu a obediência, por
aquilo que padeceu. E, sendo ele consumado, veio a ser a causa
da eterna salvação para todos os que lhe obedecem; chamado
por Deus sumo sacerdote, segundo a ordem de Melquisedeque. Do
qual muito temos que dizer, de difícil interpretação; porquanto vos
fizestes negligentes para ouvir.

Porque, devendo já ser mestres pelo tempo, ainda necessitais de
que se vos torne a ensinar quais sejam os primeiros rudimentos das
palavras de Deus; e vos haveis feito tais que necessitais de leite,
e não de sólido mantimento. Porque qualquer que ainda se
alimenta de leite não está experimentado na palavra da justiça,
porque é menino. Mas o mantimento sólido é para os perfeitos,
os quais, em razão do costume, têm os sentidos exercitados para
discernir tanto o bem como o mal.

\medskip

\lettrine{6} Por isso, deixando os rudimentos da doutrina de
Cristo, prossigamos até à perfeição, não lançando de novo o
fundamento do arrependimento de obras mortas e de fé em Deus, e
da doutrina dos batismos, e da imposição das mãos, e da ressurreição
dos mortos, e do juízo eterno. E isto faremos, se Deus o
permitir. Porque é impossível que os que já uma vez foram
iluminados, e provaram o dom celestial, e se fizeram participantes
do Espírito Santo, e provaram a boa palavra de Deus, e as
virtudes do século futuro, e recaíram, sejam outra vez renovados
para arrependimento; pois assim, quanto a eles, de novo crucificam o
Filho de Deus, e o expõem ao vitupério. Porque a terra que
embebe a chuva, que muitas vezes cai sobre ela, e produz erva
proveitosa para aqueles por quem é lavrada, recebe a bênção de Deus;
mas a que produz espinhos e abrolhos, é reprovada, e perto está
da maldição; o seu fim é ser queimada.

Mas de vós, ó amados, esperamos coisas melhores, e coisas que
acompanham a salvação, ainda que assim falamos. Porque Deus
não é injusto para se esquecer da vossa obra, e do trabalho do amor
que para com o seu nome mostrastes, enquanto servistes aos santos; e
ainda servis. Mas desejamos que cada um de vós mostre o mesmo
cuidado até ao fim, para completa certeza da esperança; para
que vos não façais negligentes, mas sejais imitadores dos que pela
fé e paciência herdam as promessas. Porque, quando Deus fez a
promessa a Abraão, como não tinha outro maior por quem jurasse,
jurou por si mesmo, dizendo: Certamente, abençoando te
abençoarei, e multiplicando te multiplicarei. E assim,
esperando com paciência, alcançou a promessa. Porque os
homens certamente juram por alguém superior a eles, e o juramento
para confirmação é, para eles, o fim de toda a contenda. Por
isso, querendo Deus mostrar mais abundantemente a imutabilidade do
seu conselho aos herdeiros da promessa, se interpôs com juramento;
para que por duas coisas imutáveis, nas quais é impossível
que Deus minta, tenhamos a firme consolação, nós, os que pomos o
nosso refúgio em reter a esperança proposta; a qual temos
como âncora da alma, segura e firme, e que penetra até ao interior
do véu, onde Jesus, nosso precursor, entrou por nós, feito
eternamente sumo sacerdote, segundo a ordem de Melquisedeque.

\medskip

\lettrine{7} Porque este Melquisedeque, que era rei de Salém,
sacerdote do Deus Altíssimo, e que saiu ao encontro de Abraão quando
ele regressava da matança dos reis, e o abençoou; a quem também
Abraão deu o dízimo de tudo, e primeiramente é, por interpretação,
rei de justiça, e depois também rei de Salém, que é rei de paz;
sem pai, sem mãe, sem genealogia, não tendo princípio de dias
nem fim de vida, mas sendo feito semelhante ao Filho de Deus,
permanece sacerdote para sempre. Considerai, pois, quão grande
era este, a quem até o patriarca Abraão deu os dízimos dos despojos.
E os que dentre os filhos de Levi recebem o sacerdócio têm
ordem, segundo a lei, de tomar o dízimo do povo, isto é, de seus
irmãos, ainda que tenham saído dos lombos de Abraão. Mas aquele,
cuja genealogia não é contada entre eles, tomou dízimos de Abraão, e
abençoou o que tinha as promessas. Ora, sem contradição alguma,
o menor é abençoado pelo maior. E aqui certamente tomam dízimos
homens que morrem; ali, porém, aquele de quem se testifica que vive.
E, por assim dizer, por meio de Abraão, até Levi, que recebe
dízimos, pagou dízimos. Porque ainda ele estava nos lombos de
seu pai quando Melquisedeque lhe saiu ao encontro.

De sorte que, se a perfeição fosse pelo sacerdócio levítico
(porque sob ele o povo recebeu a lei), que necessidade havia logo de
que outro sacerdote se levantasse, segundo a ordem de Melquisedeque,
e não fosse chamado segundo a ordem de Arão? Porque,
mudando-se o sacerdócio, necessariamente se faz também mudança da
lei. Porque aquele de quem estas coisas se dizem pertence a
outra tribo, da qual ninguém serviu ao altar, visto ser
manifesto que nosso Senhor procedeu de Judá, e concernente a essa
tribo nunca Moisés falou de sacerdócio. E muito mais
manifesto é ainda, se à semelhança de Melquisedeque se levantar
outro sacerdote, que não foi feito segundo a lei do
mandamento carnal, mas segundo a virtude da vida incorruptível.
Porque dele assim se testifica: Tu és sacerdote eternamente,
segundo a ordem de Melquisedeque. Porque o precedente
mandamento é ab-rogado\footnote{Ab-rogar: Pôr em desuso; anular,
suprimir, revogar, derrogar. Jur. Fazer cessar a existência ou a
obrigatoriedade de (uma lei) em sua totalidade.} por causa da sua
fraqueza e inutilidade

a lei nenhuma coisa aperfeiçoou) e desta sorte é introduzida
uma melhor esperança, pela qual chegamos a Deus. E visto como
não é sem prestar juramento (porque certamente aqueles, sem
juramento, foram feitos sacerdotes, mas este com juramento
por aquele que lhe disse: Jurou o Senhor, e não se arrependerá; tu
és sacerdote eternamente, Segundo a ordem de Melquisedeque),
de tanto melhor aliança Jesus foi feito fiador. E, na
verdade, aqueles foram feitos sacerdotes em grande número, porque
pela morte foram impedidos de permanecer, mas este, porque
permanece eternamente, tem um sacerdócio perpétuo. Portanto,
pode também salvar perfeitamente os que por ele se chegam a Deus,
vivendo sempre para interceder por eles. Porque nos convinha
tal sumo sacerdote, santo, inocente, imaculado, separado dos
pecadores, e feito mais sublime do que os céus; que não
necessitasse, como os sumos sacerdotes, de oferecer cada dia
sacrifícios, primeiramente por seus próprios pecados, e depois pelos
do povo; porque isto fez ele, uma vez, oferecendo-se a si mesmo.
Porque a lei constitui sumos sacerdotes a homens fracos, mas
a palavra do juramento, que veio depois da lei, constitui ao Filho,
perfeito para sempre.

\medskip

\lettrine{8} Ora, a suma do que temos dito é que temos um sumo
sacerdote tal, que está assentado nos céus à destra do trono da
majestade, ministro do santuário, e do verdadeiro tabernáculo, o
qual o Senhor fundou, e não o homem. Porque todo o sumo
sacerdote é constituído para oferecer dons e sacrifícios; por isso
era necessário que este também tivesse alguma coisa que oferecer.
Ora, se ele estivesse na terra, nem tampouco sacerdote seria,
havendo ainda sacerdotes que oferecem dons segundo a lei, os
quais servem de exemplo e sombra das coisas celestiais, como Moisés
divinamente foi avisado, estando já para acabar o tabernáculo;
porque foi dito: Olha, faze tudo conforme o modelo que no monte se
te mostrou.

Mas agora alcançou ele ministério tanto mais excelente, quanto é
mediador de uma melhor aliança que está confirmada em melhores
promessas. Porque, se aquela primeira fora irrepreensível, nunca
se teria buscado lugar para a segunda. Porque, repreendendo-os,
lhes diz: Eis que virão dias, diz o Senhor, em que com a casa de
Israel e com a casa de Judá estabelecerei uma nova aliança, não
segundo a aliança que fiz com seus pais no dia em que os tomei pela
mão, para os tirar da terra do Egito; como não permaneceram naquela
minha aliança, eu para eles não atentei, diz o Senhor. Porque
esta é a aliança que depois daqueles dias farei com a casa de
Israel, diz o Senhor; porei as minhas leis no seu entendimento, e em
seu coração as escreverei; e eu lhes serei por Deus, e eles me serão
por povo; e não ensinará cada um a seu próximo, nem cada um
ao seu irmão, dizendo: Conhece o Senhor; porque todos me conhecerão,
desde o menor deles até ao maior. Porque serei misericordioso
para com suas iniqüidades, e de seus pecados e de suas prevaricações
não me lembrarei mais. Dizendo Nova aliança, envelheceu a
primeira. Ora, o que foi tornado velho, e se envelhece, perto está
de acabar.

\medskip

\lettrine{9} Ora, também a primeira tinha ordenanças de culto
divino, e um santuário terrestre. Porque um tabernáculo estava
preparado, o primeiro, em que havia o candeeiro, e a mesa, e os pães
da proposição; ao que se chama o santuário. Mas depois do
segundo véu estava o tabernáculo que se chama o santo dos santos,
que tinha o incensário de ouro, e a arca da aliança, coberta de
ouro toda em redor; em que estava um vaso de ouro, que continha o
maná, e a vara de Arão, que tinha florescido, e as tábuas da
aliança; e sobre a arca os querubins da glória, que faziam
sombra no propiciatório; das quais coisas não falaremos agora
particularmente. Ora, estando estas coisas assim preparadas, a
todo o tempo entravam os sacerdotes no primeiro tabernáculo,
cumprindo os serviços; mas, no segundo, só o sumo sacerdote, uma
vez no ano, não sem sangue, que oferecia por si mesmo e pelas culpas
do povo; dando nisto a entender o Espírito Santo que ainda o
caminho do santuário não estava descoberto enquanto se conservava em
pé o primeiro tabernáculo, que é uma alegoria para o tempo
presente, em que se oferecem dons e sacrifícios que, quanto à
consciência, não podem aperfeiçoar aquele que faz o serviço;
consistindo somente em comidas, e bebidas, e várias abluções
e justificações da carne, impostas até ao tempo da correção.

Mas, vindo Cristo, o sumo sacerdote dos bens futuros, por um
maior e mais perfeito tabernáculo, não feito por mãos, isto é, não
desta criação, nem por sangue de bodes e bezerros, mas por
seu próprio sangue, entrou uma vez no santuário, havendo efetuado
uma eterna redenção. Porque, se o sangue dos touros e bodes,
e a cinza de uma novilha esparzida sobre os imundos, os santifica,
quanto à purificação da carne, quanto mais o sangue de
Cristo, que pelo Espírito eterno se ofereceu a si mesmo imaculado a
Deus, purificará as vossas consciências das obras mortas, para
servirdes ao Deus vivo?

E por isso é Mediador de um novo testamento, para que, intervindo
a morte para remissão das transgressões que havia debaixo do
primeiro testamento, os chamados recebam a promessa da herança
eterna. Porque onde há testamento, é necessário que
intervenha a morte do testador. Porque um testamento tem
força onde houve morte; ou terá ele algum valor enquanto o testador
vive? Por isso também o primeiro não foi consagrado sem
sangue; porque, havendo Moisés anunciado a todo o povo todos
os mandamentos segundo a lei, tomou o sangue dos bezerros e dos
bodes, com água, lã purpúrea e hissope, e aspergiu tanto o mesmo
livro como todo o povo, dizendo: Este é o sangue do
testamento que Deus vos tem mandado. E semelhantemente
aspergiu com sangue o tabernáculo e todos os vasos do ministério.
E quase todas as coisas, segundo a lei, se purificam com
sangue; e sem derramamento de sangue não há remissão.

De sorte que era bem necessário que as figuras das coisas que
estão no céu assim se purificassem; mas as próprias coisas
celestiais com sacrifícios melhores do que estes. Porque
Cristo não entrou num santuário feito por mãos, figura do
verdadeiro, porém no mesmo céu, para agora comparecer por nós
perante a face de Deus; nem também para a si mesmo se
oferecer muitas vezes, como o sumo sacerdote cada ano entra no
santuário com sangue alheio; de outra maneira, necessário lhe
fora padecer muitas vezes desde a fundação do mundo. Mas agora na
consumação dos séculos uma vez se manifestou, para aniquilar o
pecado pelo sacrifício de si mesmo. E, como aos homens está
ordenado morrerem uma vez, vindo depois disso o juízo, assim
também Cristo, oferecendo-se uma vez para tirar os pecados de
muitos, aparecerá segunda vez, sem pecado, aos que o esperam para
salvação.

\medskip

\lettrine{10} Porque tendo a lei a sombra dos bens futuros, e
não a imagem exata das coisas, nunca, pelos mesmos sacrifícios que
continuamente se oferecem cada ano, pode aperfeiçoar os que a eles
se chegam. Doutra maneira, teriam deixado de se oferecer,
porque, purificados uma vez os ministrantes, nunca mais teriam
consciência de pecado. Nesses sacrifícios, porém, cada ano se
faz comemoração dos pecados, porque é impossível que o sangue
dos touros e dos bodes tire os pecados. Por isso, entrando no
mundo, diz: Sacrifício e oferta não quiseste, mas corpo me
preparaste; holocaustos e oblações pelo pecado não te agradaram.

Então disse: Eis aqui venho (no princípio do livro está escrito de
mim), para fazer, ó Deus, a tua vontade. Como acima diz:
Sacrifício e oferta, e holocaustos e oblações pelo pecado não
quiseste, nem te agradaram (os quais se oferecem segundo a lei).
Então disse: Eis aqui venho, para fazer, ó Deus, a tua vontade.
Tira o primeiro, para estabelecer o segundo. Na qual vontade
temos sido santificados pela oblação do corpo de Jesus Cristo, feita
uma vez. E assim todo o sacerdote aparece cada dia,
ministrando e oferecendo muitas vezes os mesmos sacrifícios, que
nunca podem tirar os pecados; mas este, havendo oferecido
para sempre um único sacrifício pelos pecados, está assentado à
destra de Deus, daqui em diante esperando até que os seus
inimigos sejam postos por escabelo de seus pés. Porque com
uma só oblação aperfeiçoou para sempre os que são santificados.
E também o Espírito Santo no-lo testifica, porque depois de
haver dito: Esta é a aliança que farei com eles depois
daqueles dias, diz o Senhor: Porei as minhas leis em seus corações,
e as escreverei em seus entendimentos; acrescenta: E jamais
me lembrarei de seus pecados e de suas iniqüidades. Ora, onde
há remissão destes, não há mais oblação pelo pecado.

Tendo, pois, irmãos, ousadia para entrar no santuário, pelo
sangue de Jesus, pelo novo e vivo caminho que ele nos
consagrou, pelo véu, isto é, pela sua carne, e tendo um
grande sacerdote sobre a casa de Deus, cheguemo-nos com
verdadeiro coração, em inteira certeza de fé, tendo os corações
purificados da má consciência, e o corpo lavado com água limpa,
retenhamos firmes a confissão da nossa esperança; porque fiel
é o que prometeu. E consideremo-nos uns aos outros, para nos
estimularmos ao amor e às boas obras, não deixando a nossa
congregação, como é costume de alguns, antes admoestando-nos uns aos
outros; e tanto mais, quanto vedes que se vai aproximando aquele
dia. Porque, se pecarmos voluntariamente, depois de termos
recebido o conhecimento da verdade, já não resta mais sacrifício
pelos pecados, mas uma certa expectação horrível de juízo, e
ardor de fogo, que há de devorar os adversários. Quebrantando
alguém a lei de Moisés, morre sem misericórdia, só pela palavra de
duas ou três testemunhas. De quanto maior castigo cuidais vós
será julgado merecedor aquele que pisar o Filho de Deus, e tiver por
profano o sangue da aliança com que foi santificado, e fizer agravo
ao Espírito da graça? Porque bem conhecemos aquele que disse:
Minha é a vingança, eu darei a recompensa, diz o Senhor. E outra
vez: O Senhor julgará o seu povo. Horrenda coisa é cair nas
mãos do Deus vivo. Lembrai-vos, porém, dos dias passados, em
que, depois de serdes iluminados, suportastes grande combate de
aflições. Em parte fostes feitos espetáculo com vitupérios e
tribulações, e em parte fostes participantes com os que assim foram
tratados. Porque também vos compadecestes das minhas prisões,
e com alegria permitistes o roubo dos vossos bens, sabendo que em
vós mesmos tendes nos céus uma possessão melhor e permanente.
Não rejeiteis, pois, a vossa confiança, que tem grande e
avultado galardão. Porque necessitais de paciência, para que,
depois de haverdes feito a vontade de Deus, possais alcançar a
promessa. Porque ainda um pouquinho de tempo, e o que há de
vir virá, e não tardará. Mas o justo viverá da fé; e, se ele
recuar, a minha alma não tem prazer nele. Nós, porém, não
somos daqueles que se retiram para a perdição, mas daqueles que
crêem para a conservação da alma.

\medskip

\lettrine{11} Ora, a fé é o firme fundamento das coisas que se
esperam, e a prova das coisas que se não vêem. Porque por ela os
antigos alcançaram testemunho. Pela fé entendemos que os mundos
pela palavra de Deus foram criados; de maneira que aquilo que se vê
não foi feito do que é aparente.

Pela fé Abel ofereceu a Deus maior sacrifício do que Caim, pelo
qual alcançou testemunho de que era justo, dando Deus testemunho dos
seus dons, e por ela, depois de morto, ainda fala. Pela fé
Enoque foi trasladado para não ver a morte, e não foi achado, porque
Deus o trasladara; visto como antes da sua trasladação alcançou
testemunho de que agradara a Deus. Ora, sem fé é impossível
agradar-lhe; porque é necessário que aquele que se aproxima de Deus
creia que ele existe, e que é galardoador dos que o buscam. Pela
fé Noé, divinamente avisado das coisas que ainda não se viam, temeu
e, para salvação da sua família, preparou a arca, pela qual condenou
o mundo, e foi feito herdeiro da justiça que é segundo a fé.
Pela fé Abraão, sendo chamado, obedeceu, indo para um lugar que
havia de receber por herança; e saiu, sem saber para onde ia.
Pela fé habitou na terra da promessa, como em terra alheia,
morando em cabanas com Isaque e Jacó, herdeiros com ele da mesma
promessa. Porque esperava a cidade que tem fundamentos, da
qual o artífice e construtor é Deus. Pela fé também a mesma
Sara recebeu a virtude de conceber, e deu à luz já fora da idade;
porquanto teve por fiel aquele que lho tinha prometido. Por
isso também de um, e esse já amortecido, descenderam tantos, em
multidão, como as estrelas do céu, e como a areia inumerável que
está na praia do mar. Todos estes morreram na fé, sem terem
recebido as promessas; mas vendo-as de longe, e crendo-as e
abraçando-as, confessaram que eram estrangeiros e peregrinos na
terra. Porque, os que isto dizem, claramente mostram que
buscam uma pátria. E se, na verdade, se lembrassem daquela de
onde haviam saído, teriam oportunidade de tornar. Mas agora
desejam uma melhor, isto é, a celestial. Por isso também Deus não se
envergonha deles, de se chamar seu Deus, porque já lhes preparou uma
cidade. Pela fé ofereceu Abraão a Isaque, quando foi provado;
sim, aquele que recebera as promessas ofereceu o seu unigênito.
Sendo-lhe dito: Em Isaque será chamada a tua descendência,
considerou que Deus era poderoso para até dentre os mortos o
ressuscitar; e daí também em figura ele o recobrou.
Pela fé Isaque abençoou Jacó e Esaú, no tocante às coisas
futuras. Pela fé Jacó, próximo da morte, abençoou cada um dos
filhos de José, e adorou encostado à ponta do seu bordão.
Pela fé José, próximo da morte, fez menção da saída dos
filhos de Israel, e deu ordem acerca de seus ossos. Pela fé
Moisés, já nascido, foi escondido três meses por seus pais, porque
viram que era um menino formoso; e não temeram o mandamento do rei.
Pela fé Moisés, sendo já grande, recusou ser chamado filho da
filha de Faraó, escolhendo antes ser maltratado com o povo de
Deus, do que por um pouco de tempo ter o gozo do pecado;
tendo por maiores riquezas o vitupério de Cristo do que os
tesouros do Egito; porque tinha em vista a recompensa. Pela
fé deixou o Egito, não temendo a ira do rei; porque ficou firme,
como vendo o invisível. Pela fé celebrou a páscoa e a
aspersão do sangue, para que o destruidor dos primogênitos lhes não
tocasse. Pela fé passaram o Mar Vermelho, como por terra
seca; o que intentando os egípcios, se afogaram. Pela fé
caíram os muros de Jericó, sendo rodeados durante sete dias.
Pela fé Raabe, a meretriz, não pereceu com os incrédulos,
acolhendo em paz os espias.

E que mais direi? Faltar-me-ia o tempo contando de Gideão, e de
Baraque, e de Sansão, e de Jefté, e de Davi, e de Samuel e dos
profetas, os quais pela fé venceram reinos, praticaram a
justiça, alcançaram promessas, fecharam as bocas dos leões,
apagaram a força do fogo, escaparam do fio da espada, da
fraqueza tiraram forças, na batalha se esforçaram, puseram em fuga
os exércitos dos estranhos. As mulheres receberam pela
ressurreição os seus mortos; uns foram torturados, não aceitando o
seu livramento, para alcançarem uma melhor ressurreição; e
outros experimentaram escárnios e açoites, e até cadeias e prisões.
Foram apedrejados, serrados, tentados, mortos ao fio da
espada; andaram vestidos de peles de ovelhas e de cabras,
desamparados, aflitos e maltratados (dos quais o mundo não
era digno), errantes pelos desertos, e montes, e pelas covas e
cavernas da terra. E todos estes, tendo tido testemunho pela
fé, não alcançaram a promessa, provendo Deus alguma coisa
melhor a nosso respeito, para que eles sem nós não fossem
aperfeiçoados.

\medskip

\lettrine{12} Portanto nós também, pois que estamos rodeados
de uma tão grande nuvem de testemunhas, deixemos todo o embaraço, e
o pecado que tão de perto nos rodeia, e corramos com paciência a
carreira que nos está proposta, olhando para Jesus, autor e
consumador da fé, o qual, pelo gozo que lhe estava proposto,
suportou a cruz, desprezando a afronta, e assentou-se à destra do
trono de Deus. Considerai, pois, aquele que suportou tais
contradições dos pecadores contra si mesmo, para que não
enfraqueçais, desfalecendo em vossos ânimos.

Ainda não resististes até ao sangue, combatendo contra o pecado.
E já vos esquecestes da exortação que argumenta convosco como
filhos: Filho meu, não desprezes a correção do Senhor, e não
desmaies quando por ele fores repreendido; porque o Senhor
corrige o que ama, e açoita a qualquer que recebe por filho. Se
suportais a correção, Deus vos trata como filhos; porque, que filho
há a quem o pai não corrija? Mas, se estais sem disciplina, da
qual todos são feitos participantes, sois então bastardos, e não
filhos. Além do que, tivemos nossos pais segundo a carne, para
nos corrigirem, e nós os reverenciamos; não nos sujeitaremos muito
mais ao Pai dos espíritos, para vivermos? Porque aqueles, na
verdade, por um pouco de tempo, nos corrigiam como bem lhes parecia;
mas este, para nosso proveito, para sermos participantes da sua
santidade. E, na verdade, toda a correção, ao presente, não
parece ser de gozo, senão de tristeza, mas depois produz um fruto
pacífico de justiça nos exercitados por ela. Portanto, tornai
a levantar as mãos cansadas, e os joelhos desconjuntados, e
fazei veredas direitas para os vossos pés, para que o que manqueja
não se desvie inteiramente, antes seja sarado. Segui a paz
com todos, e a santificação, sem a qual ninguém verá o Senhor;
tendo cuidado de que ninguém se prive da graça de Deus, e de
que nenhuma raiz de amargura, brotando, vos perturbe, e por ela
muitos se contaminem. E ninguém seja devasso, ou profano,
como Esaú, que por uma refeição vendeu o seu direito de
primogenitura. Porque bem sabeis que, querendo ele ainda
depois herdar a bênção, foi rejeitado, porque não achou lugar de
arrependimento, ainda que com lágrimas o buscou.

Porque não chegastes ao monte palpável, aceso em fogo, e à
escuridão, e às trevas, e à tempestade, e ao sonido da
trombeta, e à voz das palavras, a qual os que a ouviram pediram que
se lhes não falasse mais; porque não podiam suportar o que se
lhes mandava: Se até um animal tocar o monte será apedrejado ou
passado com um dardo. E tão terrível era a visão, que Moisés
disse: Estou todo assombrado, e tremendo. Mas chegastes ao
monte Sião, e à cidade do Deus vivo, à Jerusalém celestial, e aos
muitos milhares de anjos; à universal assembléia e igreja dos
primogênitos, que estão inscritos nos céus, e a Deus, o juiz de
todos, e aos espíritos dos justos aperfeiçoados; e a Jesus, o
Mediador de uma nova aliança, e ao sangue da aspersão, que fala
melhor do que o de Abel. Vede que não rejeiteis ao que fala;
porque, se não escaparam aqueles que rejeitaram o que na terra os
advertia, muito menos nós, se nos desviarmos daquele que é dos céus;
a voz do qual moveu então a terra, mas agora anunciou,
dizendo: Ainda uma vez comoverei, não só a terra, senão também o
céu. E esta palavra: Ainda uma vez, mostra a mudança das
coisas móveis, como coisas feitas, para que as imóveis permaneçam.
Por isso, tendo recebido um reino que não pode ser abalado,
retenhamos a graça, pela qual sirvamos a Deus agradavelmente, com
reverência e piedade; porque o nosso Deus é um fogo
consumidor.

\medskip

\lettrine{13} Permaneça o amor fraternal. Não vos
esqueçais da hospitalidade, porque por ela alguns, não o sabendo,
hospedaram anjos. Lembrai-vos dos presos, como se estivésseis
presos com eles, e dos maltratados, como sendo-o vós mesmos também
no corpo. Venerado seja entre todos o matrimônio e o leito sem
mácula; porém, aos que se dão à prostituição, e aos adúlteros, Deus
os julgará. Sejam vossos costumes sem avareza, contentando-vos
com o que tendes; porque ele disse: Não te deixarei, nem te
desampararei. E assim com confiança ousemos dizer: O Senhor é o
meu ajudador, e não temerei o que me possa fazer o homem.
Lembrai-vos dos vossos pastores, que vos falaram a palavra de
Deus, a fé dos quais imitai, atentando para a sua maneira de viver.
Jesus Cristo é o mesmo, ontem, e hoje, e eternamente. Não
vos deixeis levar em redor por doutrinas várias e estranhas, porque
bom é que o coração se fortifique com graça, e não com alimentos que
de nada aproveitaram aos que a eles se entregaram. Temos um
altar, de que não têm direito de comer os que servem ao tabernáculo.
Porque os corpos dos animais, cujo sangue é, pelo pecado,
trazido pelo sumo sacerdote para o santuário, são queimados fora do
arraial. E por isso também Jesus, para santificar o povo pelo
seu próprio sangue, padeceu fora da porta. Saiamos, pois, a
ele fora do arraial, levando o seu vitupério. Porque não
temos aqui cidade permanente, mas buscamos a futura.
Portanto, ofereçamos sempre por ele a Deus sacrifício de
louvor, isto é, o fruto dos lábios que confessam o seu nome.
E não vos esqueçais da beneficência e comunicação, porque com
tais sacrifícios Deus se agrada. Obedecei a vossos pastores,
e sujeitai-vos a eles; porque velam por vossas almas, como aqueles
que hão de dar conta delas; para que o façam com alegria e não
gemendo, porque isso não vos seria útil.

Orai por nós, porque confiamos que temos boa consciência, como
aqueles que em tudo querem portar-se honestamente. E rogo-vos
com instância que assim o façais, para que eu mais depressa vos seja
restituído. Ora, o Deus de paz, que pelo sangue da aliança
eterna tornou a trazer dos mortos a nosso Senhor Jesus Cristo,
grande pastor das ovelhas, vos aperfeiçoe em toda a boa obra,
para fazerdes a sua vontade, operando em vós o que perante ele é
agradável por Cristo Jesus, ao qual seja glória para todo o sempre.
Amém. Rogo-vos, porém, irmãos, que suporteis a palavra desta
exortação; porque abreviadamente vos escrevi. Sabei que já
está solto o irmão Timóteo, com o qual, se ele vier depressa, vos
verei. Saudai a todos os vossos chefes e a todos os santos.
Os da Itália vos saúdam. A graça seja com todos vós. Amém.

